\documentclass[options]{article}
\begin{document}
	Ex Enarratiónibus sancti Augustíni epíscopi in Psalmos
	\begin{flushright}
			 (Ps 32, sermo 1, 7-8: CCL 38, 253-254)
	\end{flushright}
\emph{Confitémini Dómino in cíthara, in psaltério decem chordárum psállite ei. Cantáte ei cánticum novum.}
 Exúite vetustátem, nostis cánticum novum. Novus homo, novum testaméntum, novum cánticum. Non pértinet novum cánticum ad hómines véteres. Non illud discunt nisi hómines novi, renováti per grátiam ex vetustáte, et pertinéntes iam ad testaméntum novum, quod est regnum cælórum. Ei suspírat omnis amor noster, et cantat cánticum novum. Cantet cánticum novum, non lingua, sed vita.\\
 
	\emph{Cantáte ei cánticum novum, bene cantáte ei.} Quærit unusquísque quómodo cantet Deo. Canta illi, sed noli male. Non vult offéndi aures suas. Bene cantáte, fratres. Si alícui bono auditóri músico, quando tibi dícitur: canta ut pláceas ei, sine áliqua instructióne músicæ artis cantáre trépidas, ne displíceas artífici, quia quod in te imperítus non agnóscit, ártifex reprehéndit: quis ófferat Deo bene cantáre, sic iudicánti de cantóre, sic examinánti ómnia, sic audiénti? Quando potes afférre tam élegans artifícium cantándi, ut tam perféctis áuribus in nullo displíceas?
	\\
	\\
	Resp 2 Resp—BeatacæciliadixitTiburtino  (new)
	\\
	\\
		
	Ecce véluti modum cantándi dat tibi: noli qu\'{æ}rere verba, quasi explicáre possis unde Deus delectátur. In iubilatióne cane. Hoc est enim bene cánere Deo, in iubilatióne cantáre. Quid est in iubilatióne cánere? Intellégere, verbis explicáre non posse quod cánitur corde. Etenim illi qui cantant, sive in messe, sive in vínea, sive in áliquo ópere fervénti, cum c\'{œ}perint in verbis canticórum exsultáre lætítia, véluti impléti tanta lætítia, ut eam verbis explicáre non possint, avértunt se a sýllabis verbórum, et eunt in sonum iubilatiónis.
	Iúbilum sonus quidam est signíficans cor parturíre quod dícere non potest. Et quem decet ista iubilátio nisi ineffábilem Deum? Ineffábilis enim est, quem fari non potes. Et si eum fari non potes, et tacére non debes, quid restat nisi ut iúbiles? Ut gáudeat cor sine verbis et imménsa latitúdo gaudiórum metas non hábeat syllabárum. 
	\emph{Bene cantáte ei in iubilatióne.}
\end{document}