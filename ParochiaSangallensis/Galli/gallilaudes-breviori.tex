% LuaLaTeX

\documentclass[a4paper, twoside, 12pt]{article}
\usepackage[latin]{babel}
%\usepackage[landscape, left=3cm, right=1.5cm, top=2cm, bottom=1cm]{geometry} % okraje stranky
\usepackage[portrait, a4paper, mag=1300, truedimen, left=0.8cm, right=0.8cm, top=0.8cm, bottom=0.8cm]{geometry} % okraje stranky

\usepackage{fontspec}
\setmainfont[FeatureFile={junicode.fea}, Ligatures={Common, TeX}, RawFeature=+fixi]{Junicode}
%\setmainfont{Junicode}

% shortcut for Junicode without ligatures (for the Czech texts)
\newfontfamily\nlfont[FeatureFile={junicode.fea}, Ligatures={Common, TeX}, RawFeature=+fixi]{Junicode}

\usepackage{multicol}
\usepackage{color}
\usepackage{lettrine}
\usepackage{fancyhdr}

% usual packages loading:
\usepackage{luatextra}
\usepackage{graphicx} % support the \includegraphics command and options
\usepackage{gregoriotex} % for gregorio score inclusion
\usepackage{gregoriosyms}
\usepackage{wrapfig} % figures wrapped by the text
\usepackage{parcolumns}
\usepackage[contents={},opacity=1,scale=1,color=black]{background}
\usepackage{tikzpagenodes}
\usepackage{calc}
\usepackage{longtable}

\setlength{\headheight}{12pt}

\input{conventuscommune.tex} % Often used macros
%%%% Preklady jednotlivych zpevu (nektere se opakuji, a je dobre mit je
% vsechny na jedne hromade)

% HOURS ---

\newcommand{\trAntI}{\translatioCantus{Muž boží měl kožený toulec, pečlivě
zavázaný, jenž mu visel na šíji a~často se ho dotýkal.}}

\newcommand{\trAntII}{\translatioCantus{Klíč od~něho tak dobře střežil, že
dokud žil v~těle, nikdo z~jeho žáků nezvěděl, co je uvnitř.}}

\newcommand{\trAntIII}{\translatioCantus{Ale když se odebral z~tohoto
života, schránku otevřeli a~objevili v~ní žíněné roucho a~měděný řetěz
potřísněný krví.}}

\newcommand{\trAntIV}{\translatioCantus{A když prohlédli mistrovo tělo,
nalezli jeho tělo na čtyřech místech hluboce zbrázděno ranami od řetězu.}}

\newcommand{\trAntV}{\translatioCantus{Krev vytékající z~těch ran, místy
prostoupila i~žíněným rouchem.}}

\newcommand{\trCapituli}{\translatioCantus{
Miláčkovi Boha a~lidí,
Mojžíšovi požehnané paměti,~\gredagger{}
dopřál slávu rovnou slávě svatých~\grestar{}
učinil ho mocným na postrach nepřátelům
a~jeho slovy zastavil divy.}}

\newcommand{\trLectioBrevis}{\translatioCantus{
Pamatujte na své představené,
kteří vám hlásali Boží slovo.
Uvažte, jak oni skončili život, a~napodobujte jejich víru.
Ježíš Kristus je stejný včera i~dnes i~navěky.
Nenechte se svést věelijakými cizími naukami.}}

\newcommand{\trRespLaud}{\translatioCantus{Spravedlivého vodil Hospodin~\grestar{}
po přímých stezkách. \Vbardot{} A~ukázal mu Boží království.}}

\newcommand{\trRespLaudB}{\translatioCantus{Na tvých hradbách, Jeruzaléme,
ustanovil jsem strážné;~\grestar{}
budou bdít nad mým lidem. \Vbardot{} Ani ve dne, ani v~noci nesmějí nikdy
mlčet.}}

\newcommand{\trVersus}{\translatioCantus{\Vbardot{} Ústa spravedlivého šeptají moudrost, aleluja.
\Rbardot{} A~jeho jazyk ohlašuje právo, aleluja.}}

\newcommand{\trAntBenedictus}{\translatioCantus{Když na bujné oře vložili
nosítka a~sňali jim uzdu, vydali se přímo k~cele božího muže.}}

\newcommand{\trPreces}{\translatioCantus{
\noindent S vděčností chvalme Krista, dobrého Pastýře, \gredagger{} který dal život za své ovce, \grestar{} a~pokorně ho prosme: \Rbardot{} Pane, buď pastýřem svého lidu.

\noindent Kriste, ty dáváš církvi pastýře, a~jejich službou se ujímáš svého lidu, \grestar{} dej, ať v~lásce těch, kteří nás vedou, poznáváme, jak nás miluješ. \Rbardot{} Pane, buď pastýřem svého lidu.

\noindent Ty stále konáš skrze své zástupce službu pastýře a~učitele, \grestar{} nepřestávej nás nikdy vést prostřednictvím svých služebníků. \Rbardot{} Pane, buď pastýřem svého lidu.

\noindent Ty prokazuješ svému lidu skrze jeho pastýře službu lékaře duše i~těla, \grestar{} ochraňuj náš život a~veď nás ke svatosti. \Rbardot{} Pane, buď pastýřem svého lidu.

\noindent Ty posíláš své svaté, aby slovem i~příkladem vedli tvůj lid k~tobě, \grestar{} na jejich přímluvu nás posiluj, abychom vytrvali na cestě, která vede k~věčnému životu. \Rbardot{} Pane, buď pastýřem svého lidu.}}

\newcommand{\trOrationis}{\translatioCantus{Bože, jenž nám dopřáváš radovat
se z~výroční slavnosti svatého tvého vyznavače Havla, uděl dobrotivě,
abychom když slavíme jeho narození, též se řídili podobou jeho skutků.
Skrze…}}
 % Czech translations of the proper texts

%%%% Vicekrat opakovane kousky

\newcommand{\anteOrationem}{
  \rubrica{Ante Orationem, cantatur a Superiore:}

  \pars{Supplicatio Litaniæ.}

  \cuminitiali{}{temporalia/supplicatiolitaniae.gtex}

  \vspace{5mm}

  \pars{Oratio Dominica.}

  \cuminitiali{}{temporalia/oratiodominica.gtex}

  \vspace{5mm}

  \rubrica{Deinde dicitur ab Hebdomadario:}

  \cuminitiali{}{temporalia/dominusvobiscum-solemnis.gtex}

  \rubrica{In choro monialium loco Dominus vobiscum dicitur:}

  \sineinitiali{temporalia/domineexaudi.gtex}
}

\setlength{\columnsep}{15pt} % prostor mezi sloupci

%%%%%%%%%%%%%%%%%%%%%%%%%%%%%%%%%%%%%%%%%%%%%%%%%%%%%%%%%%%%%%%%%%%%%%%%%%%%%%%%%%%%%%%%%%%%%%%%%%%%%%%%%%%%%
\begin{document}

% Here we set the space around the initial.
% Please report to http://home.gna.org/gregorio/gregoriotex/details for more details and options
\grechangedim{afterinitialshift}{2.2mm}{scalable}
\grechangedim{beforeinitialshift}{2.2mm}{scalable}
\grechangedim{interwordspacetext}{0.20 cm plus 0.15 cm minus 0.05 cm}{scalable}%
\grechangedim{annotationraise}{-0.2cm}{scalable}

% Here we set the initial font. Change 38 if you want a bigger initial.
% Emit the initials in red.
\grechangestyle{initial}{\color{red}\fontsize{38}{38}\selectfont}

\renewcommand{\headrulewidth}{0pt} % no horiz. rule at the header
\pagestyle{empty}

\grechangedim{spaceabovelines}{0.2cm}{scalable}%

\begin{titulusOfficii}
\dies{Die 16. Octobris.}

\vspace{3mm}

\nomenFesti{S. Galli.}

\vspace{3mm}

\textbf{Ad Laudes}
\end{titulusOfficii}

\pars{}

\scriptura{}

\vfill
\pagebreak

\cantusSineNeumas

%\vspace{-8mm}

% Psalmi festivi:
% 62, Dan3, 149

{
\grechangedim{interwordspacetext}{0.16 cm plus 0.15 cm minus 0.05 cm}{scalable}%
\cuminitiali{}{temporalia/deusinadiutorium-alter.gtex}
\grechangedim{interwordspacetext}{0.20 cm plus 0.15 cm minus 0.05 cm}{scalable}%
}

\vspace{1cm}

\cantusSineNeumas

% Hymnus. %%%
\pars{Hymnus.} \scriptura{Walahfrid Strabus (\olddag{} 849)}

\cuminitiali{I}{temporalia/hym-VitaSanctorum.gtex}

\vfill
\pagebreak

{
\vspace{-3mm}
\setlength{\columnsep}{7pt} % prostor mezi sloupci
\begin{translatioMulticol}{3}
Svatých život je cestou a~záchranou\\
Kriste, jenž dáváš mír a~bezúhonnost\\
Původci svému ti hlasem i~myslí\\
zpíváme hymnus.\\
\\
Sžíráni láskou k~tomu, v~němž moci jest\\
plnost zjevena; se vším, co v~moci své\\
mají jen zbožní a~po čem vší silou\\
a~srdcem touží.\\
\\
Pro jeho zbožnost jsi svatého Havla\\
učinil vzorem nebeské jasnosti;\\
abychom jeho učením unikli\\
temnotám mysli. \\
\\
Jako pták zpěvný první se probudil\\
a~skutky svými v~životě dosvědčil\\
to, co mu moudrost jeho učitele\\
ctnostného vlila.\columnbreak

V~slově byl mocný, v~činech úctyhodný,\\
vždy znovu bažil po věčném bohatství;\\
tak zjevně došel odměnou znamení\\
nebeské ctnosti.\\
\\
Prosíme světa, Původce a~spáso\\
na jeho prosby pohlédni teď svaté,\\
rač popřát lidu dobrotivým srdcem,\\
čeho si žádá.\\
\\
Pokojné časy a~víře stálost,\\
churavým zdraví, padlým slitování;\\
všem potom lidem dar nejblaženější\\
v~životě úděl.\\
\\
Milosti Pane, ty jenž vše předvídáš:\\
Ochrana světce ať nikdy neschází\\
těm, jimž jsi dopřál tohoto ochránce\\
za svůj vzor míti.\columnbreak

A~jeho záštitou jistě se stane,\\
Nejvyšší vládce, že tvojí nebude\\
chvály na věčnost žádoucí zbaveno\\
nikdy to místo.\\
\\
Učiň to Synu, milostivý Otče,\\
i~Duchu v~obou, jenž přítomen býváš,\\
tak jako nyní stejně i~na věčné\\
světa okruhy.\\
Amen.
\end{translatioMulticol}

\setlength{\columnsep}{30pt} % prostor mezi sloupci
}

\vfill
\pagebreak

\pars{Psalmus 1.} \scriptura{Vita S. Galli XXXII, 2; \textbf{H326}}

\vspace{-5mm}

\antiphona{I g}{temporalia/ant3.gtex}

\trAntIII

\vspace{-3mm}

\scriptura{Ps. 62.}

\initiumpsalmi{temporalia/ps62-initium-i-g-auto.gtex}

\psalmusEtTranslatioT{temporalia/ps62-comb.tex}{6.5cm}

%\antiphona{}{temporalia/ant3.gtex} % repeat the antiphon - new page

\vfill
\pagebreak

\pars{Psalmus 2.} \scriptura{Vita S. Galli XXXII, 2; \textbf{H326}}

\vspace{-5mm}

\antiphona{III a}{temporalia/ant4.gtex}

\trAntIV

\scriptura{Canticum trium puerorum, Dan. 3, 57-88 et 56}

\initiumpsalmi{temporalia/dan3-initium-iii-a-auto.gtex}

\psalmusEtTranslatioT{temporalia/dan3-comb.tex}{6.5cm}

\rubrica{Hic non dicitur Gloria Patri, neque Amen.}
\vspace{1cm}

\antiphona{}{temporalia/ant4.gtex} % repeat the antiphon - new page

\vfill
\pagebreak

\pars{Psalmus 3.} \scriptura{Vita S. Galli XXXII, 2; \textbf{H326}}

\vspace{-5mm}

\antiphona{I g}{temporalia/ant5.gtex}

\trAntV

\scriptura{Ps. 149}

\vspace{-3mm}

\initiumpsalmi{temporalia/ps149-initium-i-g-auto.gtex}

\psalmusEtTranslatioT{temporalia/ps149-comb2.tex}{6.5cm}

\vspace{-10.2mm}

\psalmusEtTranslatioT{gloriapatri.tex}{6.5cm}

\vspace{-8mm}

\vfill
%\pagebreak

\cantusSineNeumas

\pars{Lectio Brevis.} \scriptura{Heb. 13, 7-9a}

\cuminitiali{}{temporalia/lectiobrevis-Mementote.gtex}

\trLectioBrevis

\vfill
\pagebreak

\pars{Responsorium breve.} \scriptura{Is. 62, 6}

\antiphona{VI}{temporalia/resp-superteierusalem.gtex}

\trRespLaudB

\vfill
%\pagebreak

\cantusCumNeumis

\pars{Canticum Zachariæ.} \scriptura{Cf. Vita S. Galli XXX, 6; ibid. XXXIII, 1; ibid. XVII, 5; \textbf{H326}}

\vspace{-5mm}

\antiphona{VIII G}{temporalia/ant-ben-laud.gtex}

\trAntBenedictus

\scriptura{Lc. 1, 68-79}

\initiumpsalmi{temporalia/benedictus-initium-viiisoll-G-auto.gtex}

\psalmusEtTranslatioT{temporalia/benedictus-comb.tex}{6.5cm}

\antiphona{}{temporalia/ant-ben-laud.gtex} % repeat the antiphon - new page

\vfill
\pagebreak

\pars{Preces.}

\sineinitiali{}{temporalia/tonusprecumnovum.gtex}

\vspace{-1mm}

\noindent Christo, bono pastóri, \gredagger{} qui pro suis óvibus ánimam pósuit, \grestar{} laudes grati exsolvámus et supplicémus, dicéntes:

\Rbardot{} Pasce pópulum tuum, Dómine.

\noindent Christe, qui in sanctis pastóribus misericórdiam et dilectiónem tuam dignátus es osténdere, \grestar{} numquam désinas per eos nobíscum misericórditer ágere.

\Rbardot{} Pasce pópulum tuum, Dómine.

\noindent Qui múnere pastóris animárum fungi per tuos vicários pergis, \grestar{} ne destíteris nos ipse per rectóres nostros dirígere.

\Rbardot{} Pasce pópulum tuum, Dómine.

\noindent Qui in sanctis tuis, populórum dúcibus, córporum animarúmque médicus exstitísti, \grestar{} numquam cesses ministérium in nos vitæ et sanctitátis perágere.

\Rbardot{} Pasce pópulum tuum, Dómine.

\noindent Qui, prudéntia et caritáte sanctórum, tuum gregem erudísti, \grestar{} nos in sanctitáte iúgiter per pastóres nostros ædífica.

\Rbardot{} Pasce pópulum tuum, Dómine.

\vspace{2mm}

\trPreces

\vfill

\vspace{-2mm}

\pars{Oratio Dominica.}

\cuminitiali{}{temporalia/oratiodominicaalt.gtex}

\vspace{-4mm}

\vfill
\pagebreak

\cantusSineNeumas

% Oratio. %%%
\pars{Oratio.}

\cuminitiali{}{temporalia/oratio.gtex}
\trOrationis

\vspace{2mm}

\antiphona{C}{temporalia/dominusnosbenedicat.gtex}

\vspace{4mm}

\cuminitiali{}{temporalia/benedicamus-duplex-laudes.gtex}

\vfill

\begin{center}
% http://e-codices.unifr.ch/en/csg/0602/44/0/Sequence-612
\includegraphics[width=4cm]{../../Conventus/Galli/gallus.jpg}
\end{center}

\vfill

\begin{center}
In memoriam P. Jiří Reinsberg

(\oldGreStar{} die 30 Martii A.D. MCMXVIII - \oldGreDagger{} die 6 Ianuarii A.D. MMIV)
\end{center}

\end{document}
