\newcommand{\titulus}{\nomenFesti{Die II infra Octavam Ss. Petri et Pauli.}
\celebratio{Semiduplex.}}
%\newcommand{\aequus}{Petrus et Paulus}
\newcommand{\lectioi}{\pars{Lectio I.} \scriptura{Mt. 10, 16-22}

\noindent Léctio sancti Evangélii secúndum Matthǽum.

\noindent In illo témpore: Dixit Iesus discípulis suis: Ecce ego mitto vos sicut oves in médio lupórum. Et réliqua.

\scriptura{Homilia 34 in Matth., longe post initium}

\noindent Homilía sancti Ioánnis Chrysóstomi.

\noindent Sed dícere vidétur: Nolíte turbári, si, cum vos inter lupos mitto, tamquam oves et colúmbas esse iúbeo. Nam, etsi possum contrárium quoque præstáre, et non permíttere ut grave áliquid patiámini, nec lupis tamquam oves subiécti sitis, sed effícere ut leónibus terribilióres evadátis; tamen sic éxpedit fíeri: hoc et vos quoque illustrióres fáciet, et meam declarábit virtútem. Sic enim póstea dixit Paulo: Súfficit tibi grátia mea, nam virtus mea in infirmitáte perfícitur. Ipse ígitur vos ut tales essétis feci.}
\newcommand{\lectioii}{\pars{Lectio II.}

\noindent Sed inspiciámus quam prudéntiam éxigit: serpéntis certe. Nam, quemádmodum serpens totum seípsum tradit, nec mínimum curat si ipsum quoque corpus incídi necésse sit, dúmmodo caput suum íntegrum servet; eódem tu quoque modo, præter fidem cétera pérdere non cures, sive pecúnias, sive corpus, sive étiam vitam ipsam profúndere necésse sit. Fides enim caput est et radix; qua serváta, etiámsi ómnia perdas, ómnia tamen rursus maióre cum glória recuperábis. Idcírco nec símplices solum iussit esse, nec prudéntes solum; sed ambo hæc in unum míscuit, ut ea in virtútem convertántur.}
\newcommand{\lectioiii}{\pars{Lectio III.}

\noindent Quod si rebus ipsis id ita fíeri vidére desíderas, lege Actuum Apostolórum librum; perspícies profécto, cum sæpe Iudæórum pópulus in Apóstolos insurréxerit ac dentes exacúerit, illos, colúmbæ simplicitátem imitándo et cum decénti modéstia respondéndo, iram ipsórum superásse, furórem exstinxísse, ímpetum retardásse. Nam, cum illi dícerent: Nonne præcipiéndo præcépimus vobis, ne docerétis in nómine isto? quamvis innúmera possent édere mirácula, nihil tamen ásperum neque dixérunt neque fecérunt; sed summa cum mansuetúdine respondéntes dicébant: Si iustum est vos audíre magis quam Deum, iudicáte. Perspexísti simplicitátem colúmbæ, vide nunc serpéntis prudéntiam: Non enim póssumus, ínquiunt, nos quæ vídimus et audívimus, non loqui.}
\newcommand{\lectioiv}{\pars{Lectio IV.} \scriptura{Sermo 1 in natáli App. Petri et Pauli}

\noindent Sermo sancti Leónis Papæ.

\noindent Omnium quidem sanctárum solemnitátum, dilectíssimi, totus mundus est párticeps, et uníus fídei píetas éxigit, ut quidquid pro salúte universórum gestum recólitur, commúnibus ubíque gáudiis celebrétur. Verúmtamen hodiérna festívitas, præter illam reveréntiam quam toto terrárum orbe proméruit, speciáli et própria nostræ Urbis exsultatióne veneránda est; ut, ubi præcipuórum Apostolórum glorificátus est éxitus, ibi in die martýrii eórum sit lætítiæ principátus. Isti enim sunt viri, per quos tibi Evangélium Christi, Roma, resplénduit; et, quæ eras magístra erróris, facta es discípula veritátis.}
\newcommand{\lectiov}{\pars{Lectio V.}

\noindent Isti sunt patres tui veríque pastóres, qui te regnis cæléstibus inseréndam multo mélius multóque felícius condidérunt, quam illi quorum stúdio prima mœ́nium tuórum fundaménta locáta sunt; ex quibus is qui tibi nomen dedit, fratérna te cæde fœdávit. Isti sunt, qui te ad hanc glóriam provexérunt, ut gens sancta, pópulus eléctus, cívitas sacerdotális et régia, per sacram beáti Petri Sedem caput orbis effécta, látius præsidéres religióne divína quam dominatióne terréna. Quamvis enim, multis aucta victóriis, ius impérii tui terra maríque protúleris; minus tamen est quod tibi béllicus labor súbdidit, quam quod pax christiána subiécit.}
\newcommand{\lectiovi}{\pars{Lectio VI.}

\noindent Disposíto namque divínitus óperi máxime congruébat, ut multa regna uno confœderaréntur império, et cito pérvios habéret pópulos prædicátio generális, quos uníus tenéret régimen civitátis. Hæc autem cívitas ignórans suæ provectiónis auctórem, cum pene ómnibus dominarétur géntibus, ómnium géntium serviébat erróribus; et magnam sibi videbátur assumpsísse religiónem, quia nullam respúerat falsitátem. Unde, quantum erat per diábolum tenácius illigáta, tantum per Christum est mirabílius absolúta.}
\newcommand{\lectiovii}{\pars{Lectio VII.} \scriptura{Mt. 16, 13-19}

\noindent Léctio sancti Evangélii secúndum Matthǽum.

\noindent In illo témpore: Venit Iesus in partes Cæsaréæ Philíppi, et interrogábat discípulos suos, dicens: Quem dicunt hómines esse Fílium hóminis? Et réliqua.

\scriptura{Lib. 3 Comment. in Matth. cap. 16}

\noindent Homilía sancti Hierónymi Presbýteri.

\noindent Pulchre intérrogat: Quem dicunt hómines esse Fílium hóminis? quia qui de fílio hóminis loquúntur, hómines sunt; qui vero divinitátem eius intéllegunt, non hómines, sed dii appellántur. At illi dixérunt: Alii Ioánnem Baptístam, álii autem Elíam. Miror quosdam intérpretes causas errórum inquírere singulórum, et disputatiónem longíssimam téxere, quare Dóminum nostrum Iesum Christum álii Ioánnem putáverint, álii Elíam, álii Ieremíam aut unum ex prophétis; cum sic potúerint erráre in Elía et Ieremía, quo modo Heródes errávit in Ioánne, dicens: Quem ego decollávi Ioánnem, ipse surréxit a mórtuis, et virtútes operántur in eo.}
\newcommand{\lectioviii}{\pars{Lectio VIII.}

\noindent Vos autem quem me esse dícitis? Prudens lector, atténde quod, ex consequéntibus textúque sermónis, Apóstoli nequáquam hómines, sed dii appellántur. Cum enim dixísset: Quem dicunt hómines esse Fílium hóminis? subiécit: Vos autem quem me esse dícitis? Illis, quia hómines sunt, humána opinántibus, vos qui estis dii, quem me esse existimátis? Petrus ex persóna ómnium Apostolórum profitétur: Tu es Christus Fílius Dei vivi. Deum vivum appéllat, ad distinctiónem eórum deórum, qui putántur dii, sed mórtui sunt.}
\newcommand{\lectioix}{\pars{Lectio IX.}

\noindent Respóndens autem Iesus, dixit ei: Beátus es, Simon Bar-Iona. Testimónio de se Apóstoli reddit vicem. Petrus díxerat: Tu es Christus Fílius Dei vivi; mercédem recépit vera conféssio: Beátus es, Simon Bar-Iona. Quare? Quia non revelávit tibi caro et sanguis, sed revelávit Pater. Quod caro et sanguis reveláre non pótuit, Spíritus Sancti grátia revelátum est. Ergo ex confessióne sortítur vocábulum, quod revelatiónem ex Spíritu Sancto hábeat, cuius et fílius appellándus sit. Síquidem Bar-Iona in nostra lingua sonat Fílius colúmbæ.}
% LuaLaTeX

\documentclass[a4paper, twoside, 12pt]{article}
\usepackage[latin]{babel}
%\usepackage[landscape, left=3cm, right=1.5cm, top=2cm, bottom=1cm]{geometry} % okraje stranky
%\usepackage[landscape, a4paper, mag=1166, truedimen, left=2cm, right=1.5cm, top=1.6cm, bottom=0.95cm]{geometry} % okraje stranky
\usepackage[landscape, a4paper, mag=1400, truedimen, left=0.5cm, right=0.5cm, top=0.5cm, bottom=0.5cm]{geometry} % okraje stranky

\usepackage{fontspec}
\setmainfont[FeatureFile={junicode.fea}, Ligatures={Common, TeX}, RawFeature=+fixi]{Junicode}
%\setmainfont{Junicode}

% shortcut for Junicode without ligatures (for the Czech texts)
\newfontfamily\nlfont[FeatureFile={junicode.fea}, Ligatures={Common, TeX}, RawFeature=+fixi]{Junicode}

\usepackage{multicol}
\usepackage{color}
\usepackage{lettrine}
\usepackage{fancyhdr}

% usual packages loading:
\usepackage{luatextra}
\usepackage{graphicx} % support the \includegraphics command and options
\usepackage{gregoriotex} % for gregorio score inclusion
\usepackage{gregoriosyms}
\usepackage{wrapfig} % figures wrapped by the text
\usepackage{parcolumns}
\usepackage[contents={},opacity=1,scale=1,color=black]{background}
\usepackage{tikzpagenodes}
\usepackage{calc}
\usepackage{longtable}
\usetikzlibrary{calc}

\setlength{\headheight}{14.5pt}

\input{conventuscommune.tex} % Often used macros
%%%% Preklady jednotlivych zpevu (nektere se opakuji, a je dobre mit je
% vsechny na jedne hromade)

% HOURS ---

\newcommand{\trAntI}{\translatioCantus{Muž boží měl kožený toulec, pečlivě
zavázaný, jenž mu visel na šíji a~často se ho dotýkal.}}

\newcommand{\trAntII}{\translatioCantus{Klíč od~něho tak dobře střežil, že
dokud žil v~těle, nikdo z~jeho žáků nezvěděl, co je uvnitř.}}

\newcommand{\trAntIII}{\translatioCantus{Ale když se odebral z~tohoto
života, schránku otevřeli a~objevili v~ní žíněné roucho a~měděný řetěz
potřísněný krví.}}

\newcommand{\trAntIV}{\translatioCantus{A když prohlédli mistrovo tělo,
nalezli jeho tělo na čtyřech místech hluboce zbrázděno ranami od řetězu.}}

\newcommand{\trAntV}{\translatioCantus{Krev vytékající z~těch ran, místy
prostoupila i~žíněným rouchem.}}

\newcommand{\trCapituli}{\translatioCantus{
Miláčkovi Boha a~lidí,
Mojžíšovi požehnané paměti,~\gredagger{}
dopřál slávu rovnou slávě svatých~\grestar{}
učinil ho mocným na postrach nepřátelům
a~jeho slovy zastavil divy.}}

\newcommand{\trLectioBrevis}{\translatioCantus{
Pamatujte na své představené,
kteří vám hlásali Boží slovo.
Uvažte, jak oni skončili život, a~napodobujte jejich víru.
Ježíš Kristus je stejný včera i~dnes i~navěky.
Nenechte se svést věelijakými cizími naukami.}}

\newcommand{\trRespLaud}{\translatioCantus{Spravedlivého vodil Hospodin~\grestar{}
po přímých stezkách. \Vbardot{} A~ukázal mu Boží království.}}

\newcommand{\trRespLaudB}{\translatioCantus{Na tvých hradbách, Jeruzaléme,
ustanovil jsem strážné;~\grestar{}
budou bdít nad mým lidem. \Vbardot{} Ani ve dne, ani v~noci nesmějí nikdy
mlčet.}}

\newcommand{\trVersus}{\translatioCantus{\Vbardot{} Ústa spravedlivého šeptají moudrost, aleluja.
\Rbardot{} A~jeho jazyk ohlašuje právo, aleluja.}}

\newcommand{\trAntBenedictus}{\translatioCantus{Když na bujné oře vložili
nosítka a~sňali jim uzdu, vydali se přímo k~cele božího muže.}}

\newcommand{\trPreces}{\translatioCantus{
\noindent S vděčností chvalme Krista, dobrého Pastýře, \gredagger{} který dal život za své ovce, \grestar{} a~pokorně ho prosme: \Rbardot{} Pane, buď pastýřem svého lidu.

\noindent Kriste, ty dáváš církvi pastýře, a~jejich službou se ujímáš svého lidu, \grestar{} dej, ať v~lásce těch, kteří nás vedou, poznáváme, jak nás miluješ. \Rbardot{} Pane, buď pastýřem svého lidu.

\noindent Ty stále konáš skrze své zástupce službu pastýře a~učitele, \grestar{} nepřestávej nás nikdy vést prostřednictvím svých služebníků. \Rbardot{} Pane, buď pastýřem svého lidu.

\noindent Ty prokazuješ svému lidu skrze jeho pastýře službu lékaře duše i~těla, \grestar{} ochraňuj náš život a~veď nás ke svatosti. \Rbardot{} Pane, buď pastýřem svého lidu.

\noindent Ty posíláš své svaté, aby slovem i~příkladem vedli tvůj lid k~tobě, \grestar{} na jejich přímluvu nás posiluj, abychom vytrvali na cestě, která vede k~věčnému životu. \Rbardot{} Pane, buď pastýřem svého lidu.}}

\newcommand{\trOrationis}{\translatioCantus{Bože, jenž nám dopřáváš radovat
se z~výroční slavnosti svatého tvého vyznavače Havla, uděl dobrotivě,
abychom když slavíme jeho narození, též se řídili podobou jeho skutků.
Skrze…}}
 % Czech translations of the proper texts

\newcommand{\annusEditionis}{2020}

%%%% Vicekrat opakovane kousky

\newcommand{\anteOrationem}{
  \rubrica{Ante Orationem, cantatur a Superiore:}

  \pars{Supplicatio Litaniæ.}

  \cuminitiali{}{temporalia/supplicatiolitaniae.gtex}

  \pars{Oratio Dominica.}

  \cuminitiali{}{temporalia/oratiodominica.gtex}

  \rubrica{Deinde dicitur ab Hebdomadario:}

  \cuminitiali{}{temporalia/dominusvobiscum-solemnis.gtex}

  \rubrica{In choro monialium loco Dominus vobiscum dicitur:}

  \sineinitiali{temporalia/domineexaudi.gtex}
}

\setlength{\columnsep}{30pt} % prostor mezi sloupci

%%%%%%%%%%%%%%%%%%%%%%%%%%%%%%%%%%%%%%%%%%%%%%%%%%%%%%%%%%%%%%%%%%%%%%%%%%%%%%%%%%%%%%%%%%%%%%%%%%%%%%%%%%%%%
\begin{document}

% Here we set the space around the initial.
% Please report to http://home.gna.org/gregorio/gregoriotex/details for more details and options
\grechangedim{afterinitialshift}{2.2mm}{scalable}
\grechangedim{beforeinitialshift}{2.2mm}{scalable}
\grechangedim{interwordspacetext}{0.22 cm plus 0.15 cm minus 0.05 cm}{scalable}%
\grechangedim{annotationraise}{-0.2cm}{scalable}

% Here we set the initial font. Change 38 if you want a bigger initial.
% Emit the initials in red.
\grechangestyle{initial}{\color{red}\fontsize{38}{38}\selectfont}

\pagestyle{empty}

%%%% Titulni stranka
\begin{titulusOfficii}
\titulus
\end{titulusOfficii}

% graphic
%\vspace{1.5cm}
%\begin{center}
%\includegraphics[width=8cm]{emmaus.jpg}
%\end{center}

\vfill

\begin{center}
%Ad usum et secundum consuetudines chori \guillemotright{}Conventus Choralis\guillemotleft.

%Editio Sancti Wolfgangi \annusEditionis
\end{center}

\pagebreak

\renewcommand{\headrulewidth}{0pt} % no horiz. rule at the header
\fancyhf{}
\pagestyle{fancy}

\cantusSineNeumas

\ifx\festum\undefined
\else
\pars{Oratio ante divinum Officium.}

\lettrine{{\color{red}A}}{peri,} Dómine, os meum ad benedicéndum nomen sanctum tuum:
munda quoque cor meum ab ómnibus vanis, pervérsis, et aliénis
cogitatiónibus:
intelléctum illúmina, afféctum inflámma,
ut digne, atténte ac devóte hoc Offícium recitáre váleam,
et exaudíri mérear ante conspéctum Divínæ Maiestátis tuæ.
Per Christum, Dóminum nostrum.
\Rbardot{} Amen.

Dómine, in unióne illíus divínæ intentiónis,
qua ipse in terris laudes Deo persolvísti,
has tibi Horas \rubricatum{(vel \textnormal{hanc tibi Horam})} persólvo.

%\trOratioAnteOfficium

\vfill

\pars{Oratio post divinum Officium.}

\rubrica{
  Orationem sequentem devote post Officium recitantibus
  Leo Papa X. defectus, et culpas in eo persolvendo ex humana
  fragilitate contractas, indulsit, et dicitur flexis genibus.
}

\lettrine{{\color{red}S}}{acrosánctæ} et indivíduæ Trinitáti,
crucifíxi Dómini nostri Iesu Christi humanitáti,
beatíssimæ et gloriosíssimæ sempérque Vírginis Maríæ
fecúndæ integritáti, 
et ómnium Sanctórum universitáti
sit sempitérna laus, honor, virtus et glória
ab omni creatúra,
nobísque remíssio ómnium peccatórum,
per infiníta sǽcula sæculórum.
\Rbardot{} Amen.

\noindent \Vbardot{} Beáta víscera Maríæ Virginis, quæ portavérunt
ætérni Patris Fílium.\\
\Rbardot{} Et beáta úbera, quæ lactavérunt Christum Dominum.

\rubrica{Et dicitur secreto \textnormal{Pater noster.} et \textnormal{Ave María.}}

%\trOratioPostOfficium

\vfill

\hora{Ad I. Vesperas.} %%%%%%%%%%%%%%%%%%%%%%%%%%%%%%%%%%%%%%%%%%%%%%%%%%%%%
%\sideThumbs{I. Vesperæ}

\vspace{0.5cm}
\grechangedim{interwordspacetext}{0.18 cm plus 0.15 cm minus 0.05 cm}{scalable}%
\ifx\festum\undefined
\cuminitiali{}{temporalia/deusinadiutorium-alter.gtex}
\else
\cuminitiali{}{temporalia/deusinadiutorium-solemnis.gtex}
\fi
\grechangedim{interwordspacetext}{0.22 cm plus 0.15 cm minus 0.05 cm}{scalable}%

\vfill
\pagebreak

\pars{Psalmus 1.} \scriptura{Io. 21, 17; \textbf{H283}}

\vspace{-0.4cm}

\antiphona{\textit{IV A}}{temporalia/ant-petreamasme.gtex}

\scriptura{Psalmus 112.}

\initiumpsalmi{temporalia/ps112-initium-iv-A-auto.gtex}

%\psalmusEtTranslatioT{temporalia/ps112ivA-comb.tex}{10cm}
\input{temporalia/ps112ivA.tex} \Abardot{}

\vspace{-1cm}

\vfill
\pagebreak

\pars{Psalmus 2.} \scriptura{2 Cor. 12, 9; \textbf{H288}}

\vspace{-0.4cm}

\antiphona{VIII G}{temporalia/ant-libentergloriabor.gtex}

\scriptura{Psalmus 116.}

\initiumpsalmi{temporalia/ps116-initium-viii-g-auto.gtex}

%\psalmusEtTranslatioT{temporalia/ps116-comb.tex}{10cm}
\input{temporalia/ps116.tex} \Abardot{}

\vfill
\pagebreak

\pars{Psalmus 3.} \scriptura{Ac. 12, 8; \textbf{H281}}

\vspace{-0.4cm}

\antiphona{VIII a}{temporalia/ant-dixitangelusadpetrum.gtex}

\scriptura{Psalmus 145.}

\initiumpsalmi{temporalia/ps145-initium-viii-A-auto.gtex}

%\psalmusEtTranslatioT{temporalia/ps145-comb.tex}{10cm}
\input{temporalia/ps145.tex} \Abardot{}

\vfill
\pagebreak

\pars{Psalmus 4.} \scriptura{2 Cor. 11, 32.33; \textbf{H289}}

\vspace{-6mm}

\antiphona{VIII G}{temporalia/ant-damascipraepositus.gtex}

\vspace{-4mm}

\scriptura{Psalmus 146.}

\vspace{-1.5mm}

\initiumpsalmi{temporalia/ps146-initium-viii-g-auto.gtex}

\vspace{-1.5mm}

%\psalmusEtTranslatioT{temporalia/ps146-comb.tex}{10cm}
\input{temporalia/ps146.tex} \Abardot{}

\vfill
\pagebreak

\pars{Psalmus 5.} \scriptura{Ac. 12, 11; \textbf{H281}}

\vspace{-0.4cm}

\antiphona{VII c\textsuperscript{2}}{temporalia/ant-misitdominus.gtex}

\scriptura{Psalmus 147.}

\initiumpsalmi{temporalia/ps147-initium-vii-c2-auto.gtex}

%\psalmusEtTranslatioT{temporalia/ps147-comb.tex}{10cm}
\input{temporalia/ps147.tex} \Abardot{}

\vfill
\pagebreak

\pars{Capitulum.} \scriptura{Ac. 12, 1-3}

\grechangedim{interwordspacetext}{0.12 cm plus 0.15 cm minus 0.05 cm}{scalable}%
\cuminitiali{}{temporalia/capitulum-MisitHerodes.gtex}
\grechangedim{interwordspacetext}{0.22 cm plus 0.15 cm minus 0.05 cm}{scalable}

% preklad Jeruz. bible
%\trCapituliI

\vfill

\ifx\festum\undefined
\pars{Responsorium breve.} \scriptura{Ps. 18, 5}

\cuminitiali{VI}{temporalia/resp-inomnemterram.gtex}
\else
\pars{Responsorium.} \scriptura{\Rbardot{} Mt. 16, 19 \Vbardot{} ibid.; \textbf{H282}}

\cuminitiali{VIII}{temporalia/resp-tuespastorovium-CROCHU.gtex}
\fi

%\trResp

\vfill
\pagebreak

\pars{Hymnus}

\cuminitiali{I}{temporalia/hym-AureaLuce.gtex}
\vspace{-3mm}
%\input{hym-AureaLuce-bohtext.tex}

\vfill
%\pagebreak

\pars{Versus.} \scriptura{Ps. 18, 5}

% Versus. %%%
\sineinitiali{temporalia/versus-inomnem.gtex}

%\noindent \trVersus

\vfill
\pagebreak

\pars{Canticum B. Mariæ V.} \scriptura{Ac. 9, 10.11.15}

\vspace{-3mm}

{
\grechangedim{interwordspacetext}{0.18 cm plus 0.15 cm minus 0.05 cm}{scalable}%
\antiphona{VII a}{temporalia/ant-vadeanania.gtex}
\grechangedim{interwordspacetext}{0.22 cm plus 0.15 cm minus 0.05 cm}{scalable}%
}

%\trAntIMagnificat

%\vspace{-2mm}

\scriptura{Lc. 1, 46-55}

%\vspace{-2mm}

\cantusSineNeumas
\initiumpsalmi{temporalia/magnificat-initium-viisoll-a.gtex}

%\psalmusEtTranslatioT{temporalia/magnificat-I-comb.tex}{10.2cm}
\input{temporalia/magnificat-I.tex} \Abardot{}

%\vspace{-1cm}

\vfill
\pagebreak

%\sideThumbs{{\scriptsize{}Fine horarum}}

\anteOrationem

\pagebreak

% Oratio. %%%
\cuminitiali{}{temporalia/oratio.gtex}

\vspace{-1mm}
%\trOrationisI

\vfill

\rubrica{Hebdomadarius dicit iterum Dominus vobiscum, vel cantor dicit:}

\vspace{2mm}

\sineinitiali{temporalia/domineexaudi.gtex}

\rubrica{Postea cantatur a cantore:}

\vspace{2mm}

\ifx\festum\undefined
\cuminitiali{II}{temporalia/benedicamus-semiduplex-vesp.gtex}
\else
\cuminitiali{II}{temporalia/benedicamus-solemnism-1vesp.gtex}
\fi

\vspace{1mm}

\vfill
\pagebreak
\fi

\hora{Ad Matutinum.} %%%%%%%%%%%%%%%%%%%%%%%%%%%%%%%%%%%%%%%%%%%%%%%%%%%%%
%\sideThumbs{Matutinum}

\vspace{2mm}

\cuminitiali{}{temporalia/dominelabiamea.gtex}

\vspace{2mm}

\pars{Invitatorium.} \scriptura{Cantor; \textbf{H443}}

\vspace{-6mm}

\antiphona{III}{temporalia/inv-regemapostolorum.gtex}

\vfill
\pagebreak

\pars{Hymnus.}

\cuminitiali{IV}{temporalia/hym-FelixPerOmnes.gtex}
\vspace{-3mm}
%\input{hym-FelixPerOmnes-bohtext.tex}

\vfill
\pagebreak

\subhora{In I. Nocturno}

\pars{Psalmus 1.} \scriptura{Ps. 18, 5; \textbf{H360}}

\vspace{-4mm}

\antiphona{II D}{temporalia/ant-inomnemterram-FKP.gtex}

%\vspace{-5mm}

\scriptura{Ps. 18}

%\vspace{-2mm}

\initiumpsalmi{temporalia/ps18-initium-ii-D-auto.gtex}

%\psalmusEtTranslatioT{temporalia/ps18-comb.tex}{10cm}
\input{temporalia/ps18.tex}

\vfill

\antiphona{}{temporalia/ant-inomnemterram-FKP.gtex}

\vfill
\pagebreak

\pars{Psalmus 2.} \scriptura{Ps. 33, 18; \textbf{H360}}

\vspace{-4mm}

\antiphona{VII c\textsuperscript{2}}{temporalia/ant-clamaveruntiusti-FKP.gtex}

%\vspace{-5mm}

\scriptura{Ps. 33}

\initiumpsalmi{temporalia/ps33-initium-vii-c2-auto.gtex}

%\psalmusEtTranslatioT{temporalia/ps33-comb.tex}{10cm}
\input{temporalia/ps33.tex}

\vfill

\antiphona{}{temporalia/ant-clamaveruntiusti-FKP.gtex}

\vfill
\pagebreak

\pars{Psalmus 3.} \scriptura{Ps. 44, 17.18; \textbf{H360}}

\vspace{-4mm}

\antiphona{VIII a}{temporalia/ant-constitueseosprincipes-FKP.gtex}

%\vspace{-2mm}

\scriptura{Ps. 44}

%\vspace{-2mm}

\initiumpsalmi{temporalia/ps44-initium-viii-A-auto.gtex}

%\psalmusEtTranslatioT{temporalia/ps44-comb.tex}{10cm}
\input{temporalia/ps44.tex}

\vfill

\antiphona{}{temporalia/ant-constitueseosprincipes-FKP.gtex}

\vfill
\pagebreak

\pars{Versus.} \scriptura{Ps. 18, 5}

% Versus. %%%
\sineinitiali{temporalia/versus-inomnem-communis.gtex}

\vspace{5mm}

\sineinitiali{temporalia/oratiodominica-mat.gtex}

\vspace{5mm}

\pars{Absolutio.}

\cuminitiali{}{temporalia/absolutio-exaudi.gtex}

\vfill
\pagebreak

\cuminitiali{}{temporalia/benedictio-solemn-benedictione.gtex}

\vspace{7mm}

\lectioi

\noindent \Vbardot{} Tu autem, Dómine, miserére nobis.
\noindent \Rbardot{} Deo grátias.

\vfill
\pagebreak

\pars{Responsorium 1.} \scriptura{\Rbardot{} Mt. 16, 19 \Vbardot{} ibid; \textbf{H280}}

\vspace{-5mm}

\responsorium{VII}{temporalia/resp-simonpetre-CROCHU.gtex}{}

\vfill
\pagebreak

\cuminitiali{}{temporalia/benedictio-solemn-unigenitus.gtex}

\vspace{7mm}

\lectioii

\noindent \Vbardot{} Tu autem, Dómine, miserére nobis.
\noindent \Rbardot{} Deo grátias.

\vfill
\pagebreak

\pars{Responsorium 2.} \scriptura{\Rbardot{} Io. 21, 15.17 \Vbardot{} Mt. 26, 35; \textbf{H280}}

\vspace{-5mm}

\responsorium{VI}{temporalia/resp-sidiligisme-CROCHU.gtex}{}

\vfill
\pagebreak

\cuminitiali{}{temporalia/benedictio-solemn-spiritus.gtex}

\vspace{7mm}

\lectioiii

\noindent \Vbardot{} Tu autem, Dómine, miserére nobis.
\noindent \Rbardot{} Deo grátias.

\vfill
\pagebreak

\pars{Responsorium 3.} \scriptura{\Rbardot{} Mt. 16, 18.19 \Vbardot{} ibid; \textbf{H280}}

\vspace{-5mm}

\responsorium{VII}{temporalia/resp-tuespetrus-CROCHU.gtex}{}

\vfill
\pagebreak

\subhora{In II. Nocturno}

\pars{Psalmus 4.} \scriptura{Ps. 46, 10; \textbf{H361}}

\vspace{-4mm}

\antiphona{VIII G}{temporalia/ant-principespopulorum-FKP.gtex}

\scriptura{Ps. 46}

%\vspace{-2mm}

\initiumpsalmi{temporalia/ps46-initium-viii-G-auto.gtex}

%\psalmusEtTranslatioT{temporalia/ps46-comb.tex}{10cm}
\input{temporalia/ps46.tex} \Abardot{}

\vfill
\pagebreak

\pars{Psalmus 5.} \scriptura{Ps. 60, 6; \textbf{H361}}

\vspace{-4mm}

\antiphona{VIII G}{temporalia/ant-dedistihereditatem-FKP.gtex}

%\vspace{-4mm}

\scriptura{Ps. 60}

%\vspace{-2mm}

\initiumpsalmi{temporalia/ps60-initium-viii-G-auto.gtex}

%\vspace{-1.5mm}

%\psalmusEtTranslatioT{temporalia/ps60-comb.tex}{10cm}
\input{temporalia/ps60.tex} \Abardot{}

\vspace{-1cm}

\vfill
\pagebreak

\pars{Psalmus 6.} \scriptura{Ps. 63, 10; \textbf{H361}}

\vspace{-4mm}

\antiphona{VIII G}{temporalia/ant-anuntiaverunt-FKP.gtex}

%\vspace{-5mm}

\scriptura{Ps. 63}

\initiumpsalmi{temporalia/ps63-initium-viii-G-auto.gtex}

%\psalmusEtTranslatioT{temporalia/ps63-comb.tex}{10cm}
\input{temporalia/ps63.tex} \Abardot{}

%\vfill

%\antiphona{}{temporalia/ant-anuntiaverunt-FKP.gtex}

\vfill
\pagebreak

\pars{Versus.} \scriptura{Ps. 44, 17-18}

% Versus. %%%
\sineinitiali{temporalia/versus-constitues.gtex}

\vspace{5mm}

\sineinitiali{temporalia/oratiodominica-mat.gtex}

\vspace{5mm}

\pars{Absolutio.}

\cuminitiali{}{temporalia/absolutio-ipsius.gtex}

\vfill
\pagebreak

\cuminitiali{}{temporalia/benedictio-solemn-deus.gtex}

\vspace{7mm}

\lectioiv

\noindent \Vbardot{} Tu autem, Dómine, miserére nobis.
\noindent \Rbardot{} Deo grátias.

\vfill
\pagebreak

\pars{Responsorium 4.} \scriptura{\Rbardot{} Mt. 14, 28.31 \Vbardot{} ibid. 14, 30; \textbf{H281}}

\vspace{-5mm}

\responsorium{VIII}{temporalia/resp-dominesitues-CROCHU.gtex}{}

\vfill
\pagebreak

\cuminitiali{}{temporalia/benedictio-solemn-christus.gtex}

\vspace{7mm}

\lectiov

\noindent \Vbardot{} Tu autem, Dómine, miserére nobis.
\noindent \Rbardot{} Deo grátias.

\vfill
\pagebreak

\pars{Responsorium 5.} \scriptura{\Rbardot{} Ac. 12, 7.8 \Vbardot{} ibid. 12, 7; \textbf{H281}}

\vspace{-5mm}

\responsorium{VII}{temporalia/resp-surgepetre-CROCHU.gtex}{}

\vfill
\pagebreak

\cuminitiali{}{temporalia/benedictio-solemn-ignem.gtex}

\vspace{7mm}

\lectiovi

\noindent \Vbardot{} Tu autem, Dómine, miserére nobis.
\noindent \Rbardot{} Deo grátias.

\vfill
\pagebreak

\pars{Responsorium 6.} \scriptura{\Rbardot{} Mt. 16, 19 \Vbardot{} ibid.; \textbf{H282}}

\vspace{-5mm}

\responsorium{VIII}{temporalia/resp-tuespastorovium-CROCHU.gtex}{}

\vfill
\pagebreak

\subhora{In III. Nocturno}

\pars{Psalmus 7.} \scriptura{Ps. 74, 11; \textbf{H361}}

\vspace{-4mm}

\antiphona{VI F}{temporalia/ant-exaltabunturcornuaiusti-FKP.gtex}

%\vspace{-4mm}

\scriptura{Ps. 74}

%\vspace{-2mm}

\initiumpsalmi{temporalia/ps74-initium-vi-F-auto.gtex}

%\psalmusEtTranslatioT{temporalia/ps74-comb.tex}{10cm}
\input{temporalia/ps74.tex} \Abardot{}

\vfill
\pagebreak

\pars{Psalmus 8.} \scriptura{Ps. 96, 11; \textbf{H361}}

\vspace{-4mm}

\antiphona{VI F}{temporalia/ant-luxortaestiustis-FKP.gtex}

%\vspace{-3mm}

\scriptura{Ps. 96}

%\vspace{-2mm}

\initiumpsalmi{temporalia/ps96-initium-vi-F-auto.gtex}

%\vspace{-1mm}

%\psalmusEtTranslatioT{temporalia/ps96-comb.tex}{10cm}
\input{temporalia/ps96.tex} \Abardot{}

\vfill
\pagebreak

\pars{Psalmus 9.} \scriptura{Ps. 98, 7; \textbf{H361}}

\vspace{-4mm}

\antiphona{\textit{IV A}}{temporalia/ant-custodiebant-FKP.gtex}

%\vspace{-3mm}

\scriptura{Ps. 98}

%\vspace{-3mm}

\initiumpsalmi{temporalia/ps98-initium-iv-A-auto.gtex}

%\vspace{-1.5mm}

%\psalmusEtTranslatioT{temporalia/ps98-comb.tex}{10cm}
\input{temporalia/ps98.tex} \Abardot{}

\vfill
\pagebreak

\pars{Versus.} \scriptura{Ps. 138, 17}

% Versus. %%%
\sineinitiali{temporalia/versus-nimis.gtex}

\vspace{5mm}

\sineinitiali{temporalia/oratiodominica-mat.gtex}

\vspace{5mm}

\pars{Absolutio.}

\cuminitiali{}{temporalia/absolutio-avinculis.gtex}

\vfill
\pagebreak

\cuminitiali{}{temporalia/benedictio-solemn-evangelica.gtex}

\vspace{7mm}

\lectiovii

\noindent \Vbardot{} Tu autem, Dómine, miserére nobis.
\noindent \Rbardot{} Deo grátias.

\vfill
\pagebreak

\pars{Responsorium 7.} \scriptura{\Rbardot{} Lc. 22, 32 \Vbardot{} Mt. 16, 17; \textbf{H282}}

\vspace{-5mm}

\responsorium{III}{temporalia/resp-egoproterogavi-CROCHU.gtex}{}

\vfill
\pagebreak

\cuminitiali{}{temporalia/benedictio-solemn-quorum.gtex}

\vspace{7mm}

\lectioviii

\noindent \Vbardot{} Tu autem, Dómine, miserére nobis.
\noindent \Rbardot{} Deo grátias.

\vfill
\pagebreak

\pars{Responsorium 8.} \scriptura{\Rbardot{} Mt. 16, 13.16.18 \Vbardot{} ibid. 16, 17; \textbf{H282}}

\vspace{-5mm}

\responsorium{I}{temporalia/resp-quemdicunthomines-CROCHU.gtex}{}

\vfill
\pagebreak

\cuminitiali{}{temporalia/benedictio-solemn-adsocietatem.gtex}

\vspace{7mm}

\lectioix

\noindent \Vbardot{} Tu autem, Dómine, miserére nobis.
\noindent \Rbardot{} Deo grátias.

\vfill
\pagebreak

% Te Deum

%\pars{Hymnus Ambrosianus}

\vspace{-5mm}

\ifx\solemnis\undefined
\ifx\aequus\undefined
{
\pars{Hymnus Ambrosianus} \scriptura{Alio modo, iuxta morem Romanum}

\vspace{-2mm}

\grechangedim{interwordspacetext}{0.26 cm plus 0.15 cm minus 0.05 cm}{scalable}%
\cuminitiali{III}{temporalia/tedeum-romanum-gn.gtex}
\grechangedim{interwordspacetext}{0.22 cm plus 0.15 cm minus 0.05 cm}{scalable}%
}
\else
{
\pars{Hymnus Ambrosianus} \scriptura{Tonus Simplex}

\vspace{-2mm}

\grechangedim{interwordspacetext}{0.30 cm plus 0.15 cm minus 0.05 cm}{scalable}%
\cuminitiali{III}{temporalia/tedeum-simplex-gn.gtex}
\grechangedim{interwordspacetext}{0.22 cm plus 0.15 cm minus 0.05 cm}{scalable}%
}
\fi
\else
{
\pars{Hymnus Ambrosianus} \scriptura{Tonus Solemnis}

\vspace{-2mm}

\grechangedim{interwordspacetext}{0.26 cm plus 0.15 cm minus 0.05 cm}{scalable}%
\cuminitiali{III}{temporalia/tedeum-solemnis-gn.gtex}
\grechangedim{interwordspacetext}{0.22 cm plus 0.15 cm minus 0.05 cm}{scalable}%
}
\fi

\vfill
\pagebreak

\rubrica{Reliqua omittuntur, nisi Laudes separandæ sint.}

\sineinitiali{temporalia/domineexaudi.gtex}

\vfill

\pars{Oratio.}

\cuminitiali{}{temporalia/oratio.gtex}

\vfill

\noindent \Vbardot{} Dómine, exáudi oratiónem meam.
\Rbardot{} Et clamor meus ad te véniat.

\vfill

% Nocturnale Romanum 2002, p. LXXVI Benedicamus Domino seems to match
% the one from Solemn Laudes.
\cuminitiali{V}{temporalia/benedicamus-solemnis-laud.gtex}

\vfill

\noindent \Vbardot{} Fidélium ánimæ per misericórdiam Dei requiéscant in pace.
\Rbardot{} Amen.

\vfill
\pagebreak

\hora{Ad Laudes.} %%%%%%%%%%%%%%%%%%%%%%%%%%%%%%%%%%%%%%%%%%%%%%%%%%%%%
%\sideThumbs{Laudes}

\cantusSineNeumas

\vspace{0.5cm}
\grechangedim{interwordspacetext}{0.18 cm plus 0.15 cm minus 0.05 cm}{scalable}%
\ifx\festumveldominica\undefined
\cuminitiali{}{temporalia/deusinadiutorium-communis.gtex}
\else
\cuminitiali{}{temporalia/deusinadiutorium-alter.gtex}
\fi
\grechangedim{interwordspacetext}{0.22 cm plus 0.15 cm minus 0.05 cm}{scalable}%

\vfill
%\pagebreak

\pars{Psalmus 1.} \scriptura{1 Cor. 3, 6; \textbf{H288}}

\vspace{-6mm}

\antiphona{VIII G}{temporalia/ant-egoplantaviapollorigavit.gtex}

\vspace{-2mm}

\scriptura{Psalmus 92.}

\vspace{-1mm}

\initiumpsalmi{temporalia/ps92-initium-viii-g-auto.gtex}

%\psalmusEtTranslatioT{temporalia/ps92-comb.tex}{10cm}
\input{temporalia/ps92.tex} \Abardot{}

\vfill
\pagebreak

\pars{Psalmus 2.} \scriptura{Mt. 16, 18; \textbf{H283}}

\vspace{-0.4cm}

\antiphona{VII c}{temporalia/ant-tuespetrus.gtex}

\scriptura{Psalmus 99.}

\initiumpsalmi{temporalia/ps99-initium-vii-c-auto.gtex}

%\psalmusEtTranslatioT{temporalia/ps99-comb.tex}{10cm}
\input{temporalia/ps99.tex} \Abardot{}

\vfill
\pagebreak

\pars{Psalmus 3.} \scriptura{\textbf{H288}}

\vspace{-0.4cm}

\antiphona{VIII G}{temporalia/ant-sanctepauleapostole.gtex}

\scriptura{Psalmus 62.}

\initiumpsalmi{temporalia/ps62-initium-viii-G-auto.gtex}

%\psalmusEtTranslatioT{temporalia/ps62-comb.tex}{10cm}
\input{temporalia/ps62.tex} \Abardot{}

\vfill
\pagebreak

\pars{Psalmus 4.} \scriptura{Lc. 22, 32; \textbf{H283}}

\vspace{-0.4cm}

\antiphona{VIII G}{temporalia/ant-egoproterogavi.gtex}

\scriptura{Canticum trium puerorum, Dan. 3, 57-88 et 56}

\initiumpsalmi{temporalia/dan3-initium-viii-G-auto.gtex}

%\psalmusEtTranslatioT{temporalia/dan3-comb.tex}{10cm}
\input{temporalia/dan3.tex}

\rubrica{Hic non dicitur Gloria Patri, neque Amen.}

\vfill

\vspace{-6mm}

\antiphona{}{temporalia/ant-egoproterogavi.gtex} % repeat the antiphon - new page

\vfill
\pagebreak

\pars{Psalmus 5.} \scriptura{1 Cor. 15, 10; \textbf{H288}}

\vspace{-0.4cm}

\antiphona{\textit{IV A*}}{temporalia/ant-gratiadeiinme.gtex}

\scriptura{Psalmus 148.}

\initiumpsalmi{temporalia/ps148-initium-iv-A_-auto.gtex}

%\psalmusEtTranslatioT{temporalia/ps148-comb.tex}{10cm}
\input{temporalia/ps148.tex}

\rubrica{Hic non dicitur Gloria Patri.}

\vfill
\pagebreak

%
\scriptura{Psalmus 149.}

\initiumpsalmi{temporalia/ps149-initium-iv-A_-auto.gtex}

%\psalmusEtTranslatioT{temporalia/ps149-comb.tex}{10cm}
\input{temporalia/ps149.tex}

\rubrica{Hic non dicitur Gloria Patri.}

\vfill
\pagebreak

%
\scriptura{Psalmus 150.}

\initiumpsalmi{temporalia/ps150-initium-iv-A_-auto.gtex}

%\psalmusEtTranslatioT{temporalia/ps150-comb.tex}{10cm}
\input{temporalia/ps150.tex}

\vfill

\vspace{-6mm}

\antiphona{}{temporalia/ant-gratiadeiinme.gtex} % repeat the antiphon - new page

\vfill
\pagebreak

\pars{Capitulum.} \scriptura{Ac. 12, 1-3}

\grechangedim{interwordspacetext}{0.12 cm plus 0.15 cm minus 0.05 cm}{scalable}%
\cuminitiali{}{temporalia/capitulum-MisitHerodes.gtex}
\grechangedim{interwordspacetext}{0.22 cm plus 0.15 cm minus 0.05 cm}{scalable}

% preklad Jeruz. bible
%\trCapituliI

\vfill

\pars{Responsorium breve.} \scriptura{Ps. 18, 5}

\cuminitiali{VI}{temporalia/resp-inomnemterram.gtex}

%\trResp

\vfill
\pagebreak

\pars{Hymnus}

\ifx\festum\undefined
\grechangedim{interwordspacetext}{0.16 cm plus 0.15 cm minus 0.05 cm}{scalable}%
\cuminitiali{IV}{temporalia/hym-ApostolorumPassio.gtex}
\grechangedim{interwordspacetext}{0.22 cm plus 0.15 cm minus 0.05 cm}{scalable}%
\vspace{-3mm}
%\input{hym-ApostolorumPassio-bohtext.tex}
\vfill
\pagebreak
\else
\cuminitiali{IV}{temporalia/hym-JamBonePastor.gtex}
\vspace{-3mm}
%\input{hym-JamBonePastor-bohtext.tex}
\vfill
%\pagebreak
\fi

\pars{Versus.} \scriptura{Ps. 63, 10}

% Versus. %%%
\ifx\festum\undefined
\sineinitiali{temporalia/versus-annuntiaverunt-communis.gtex}
\else
\sineinitiali{temporalia/versus-annuntiaverunt.gtex}
\fi

%\noindent \trVersus

\vfill
\pagebreak

\pars{Canticum Zachariæ.} \scriptura{\textbf{H284}}

\vspace{-4mm}

{
\grechangedim{interwordspacetext}{0.18 cm plus 0.15 cm minus 0.05 cm}{scalable}%
\antiphona{VI F}{temporalia/ant-gloriosiprincipes.gtex}
\grechangedim{interwordspacetext}{0.22 cm plus 0.15 cm minus 0.05 cm}{scalable}%
}

%\trAntIMagnificat

%\vspace{-2mm}

\scriptura{Lc. 1, 68-79}

\vspace{-1mm}

\cantusSineNeumas
\ifx\solemnis\undefined
\initiumpsalmi{temporalia/benedictus-initium-vi-F-auto.gtex}

%\vspace{-1.5mm}

%\psalmusEtTranslatioT{temporalia/benedictus-II-comb.tex}{10.2cm}
\input{temporalia/benedictus-II.tex} \Abardot{}
\else
\initiumpsalmi{temporalia/benedictus-initium-visoll-F-auto.gtex}

%\vspace{-1.5mm}

%\psalmusEtTranslatioT{temporalia/benedictus-I-comb.tex}{10.2cm}
\input{temporalia/benedictus-I.tex} \Abardot{}
\fi

\vspace{-1cm}

\vfill
\pagebreak

%\sideThumbs{{\scriptsize{}Fine horarum}}

\anteOrationem

\pagebreak

% Oratio. %%%
\cuminitiali{}{temporalia/oratio.gtex}

\vspace{-1mm}
%\trOrationisI

\vfill

\rubrica{Hebdomadarius dicit iterum Dominus vobiscum, vel cantor dicit:}

\vspace{2mm}

\sineinitiali{temporalia/domineexaudi.gtex}

\rubrica{Postea cantatur a cantore:}

\vspace{2mm}

\ifx\festum\undefined
\ifx\octava\undefined
\cuminitiali{I}{temporalia/benedicamus-semiduplex-laud.gtex}
\else
\cuminitiali{VIII}{temporalia/benedicamus-duplexmajus-laudes.gtex}
\fi
\else
\cuminitiali{II}{temporalia/benedicamus-solemnism-laud.gtex}
\fi

\vspace{1mm}

\vfill
\pagebreak

\ifx\sabbatoveloctava\undefined
\ifx\festumveldominica\undefined
\hora{Ad Vesperas.} %%%%%%%%%%%%%%%%%%%%%%%%%%%%%%%%%%%%%%%%%%%%%%%%%%%%%
%\sideThumbs{Vesperæ}
\else
\hora{Ad II. Vesperas.} %%%%%%%%%%%%%%%%%%%%%%%%%%%%%%%%%%%%%%%%%%%%%%%%%%%%%
%\sideThumbs{II. Vesperæ}
\fi

\cantusSineNeumas

%\vspace{-2mm}
\grechangedim{interwordspacetext}{0.18 cm plus 0.15 cm minus 0.05 cm}{scalable}%
\ifx\festumveldominica\undefined
\cuminitiali{}{temporalia/deusinadiutorium-communis.gtex}
\else
\ifx\festum\undefined
\cuminitiali{}{temporalia/deusinadiutorium-alter.gtex}
\else
\cuminitiali{}{temporalia/deusinadiutorium-solemnis.gtex}
\fi
\fi
\grechangedim{interwordspacetext}{0.22 cm plus 0.15 cm minus 0.05 cm}{scalable}%

\vfill
%\pagebreak

%\vspace{-2mm}

\pars{Psalmus 1.} \scriptura{Ac. 3, 1; \textbf{H279}}

\vspace{-0.4cm}

\antiphona{VIII G}{temporalia/ant-petrusetjoannes.gtex}

\scriptura{Psalmus 109.}

\initiumpsalmi{temporalia/ps109-initium-viii-g-auto.gtex}

%\psalmusEtTranslatioT{temporalia/ps109-comb.tex}{10cm}
\input{temporalia/ps109.tex} \Abardot{}

\vfill
\pagebreak

\pars{Psalmus 2.} \scriptura{Ac. 3, 6; \textbf{H280}}

\vspace{-0.4cm}

\antiphona{VII d}{temporalia/ant-argentumetaurum.gtex}

\scriptura{Psalmus 112.}

\initiumpsalmi{temporalia/ps112-initium-vii-d-auto.gtex}

%\psalmusEtTranslatioT{temporalia/ps112-comb.tex}{10cm}
\input{temporalia/ps112.tex} \Abardot{}

\vfill
\pagebreak

\pars{Psalmus 3.} \scriptura{Phil. 1, 1, 21; Gal. 6, 14; \textbf{H284}}

\vspace{-0.4cm}

\antiphona{I g}{temporalia/ant-mihivivere.gtex}

\scriptura{Psalmus 115.}

\initiumpsalmi{temporalia/ps115-initium-i-g-auto.gtex}

%\psalmusEtTranslatioT{temporalia/ps115-comb.tex}{10cm}
\input{temporalia/ps115.tex} \Abardot{}

\vfill
\pagebreak

\pars{Psalmus 4.} \scriptura{2 Cor. 11, 25; \textbf{H286}}

\vspace{-0.4cm}

\antiphona{VIII G}{temporalia/ant-tervirgiscaesus.gtex}

\scriptura{Psalmus 138.}

\initiumpsalmi{temporalia/ps138-initium-viii-g-auto.gtex}

%\psalmusEtTranslatioT{temporalia/ps138-comb.tex}{10cm}
\input{temporalia/ps138.tex}

\vfill

\antiphona{}{temporalia/ant-tervirgiscaesus.gtex}

\vfill
\pagebreak

\pars{Capitulum.} \scriptura{Ac. 12, 1-3}

\grechangedim{interwordspacetext}{0.12 cm plus 0.15 cm minus 0.05 cm}{scalable}%
\cuminitiali{}{temporalia/capitulum-MisitHerodes.gtex}
\grechangedim{interwordspacetext}{0.22 cm plus 0.15 cm minus 0.05 cm}{scalable}

% preklad Jeruz. bible
%\trCapituliI

\vfill

\pars{Responsorium breve.} \scriptura{Ps. 18, 5}

\cuminitiali{VI}{temporalia/resp-inomnemterram.gtex}

%\trResp

\vfill
\pagebreak

\pars{Hymnus}

\cuminitiali{IV}{temporalia/hym-ORomaFelix.gtex}
\vspace{-3mm}
%\input{hym-ORomaFelix-bohtext.tex}

\vfill
%\pagebreak

\pars{Versus.} \scriptura{Ps. 63, 10}

% Versus. %%%
\ifx\festum\undefined
\sineinitiali{temporalia/versus-annuntiaverunt-communis.gtex}
\else
\sineinitiali{temporalia/versus-annuntiaverunt.gtex}
\fi

%\noindent \trVersus

\vfill
\pagebreak

\pars{Canticum B. Mariæ V.}

\vspace{-3mm}

{
\grechangedim{interwordspacetext}{0.18 cm plus 0.15 cm minus 0.05 cm}{scalable}%
\antiphona{I D*}{temporalia/ant-hodiesimonpetrus.gtex}
\grechangedim{interwordspacetext}{0.22 cm plus 0.15 cm minus 0.05 cm}{scalable}%
}

%\trAntIMagnificat

%\vspace{-3mm}

\scriptura{Lc. 1, 46-55}

%\vspace{-2.5mm}

\cantusSineNeumas
\ifx\solemnis\undefined
\initiumpsalmi{temporalia/magnificat-initium-i-D_.gtex}

%\vspace{-1.5mm}

%\psalmusEtTranslatioT{temporalia/magnificat-III-comb.tex}{10.2cm}
\input{temporalia/magnificat-III.tex}
\else
\initiumpsalmi{temporalia/magnificat-initium-isoll-D_.gtex}

%\vspace{-1.5mm}

%\psalmusEtTranslatioT{temporalia/magnificat-II-comb.tex}{10.2cm}
\input{temporalia/magnificat-II.tex}
\fi

\vfill

{
\grechangedim{interwordspacetext}{0.18 cm plus 0.15 cm minus 0.05 cm}{scalable}%
\antiphona{}{temporalia/ant-hodiesimonpetrus.gtex}
\grechangedim{interwordspacetext}{0.22 cm plus 0.15 cm minus 0.05 cm}{scalable}%
}

\vspace{-1cm}

\vfill
\pagebreak

%\sideThumbs{{\scriptsize{}Fine horarum}}

\anteOrationem

\pagebreak

% Oratio. %%%
\cuminitiali{}{temporalia/oratio.gtex}

\vspace{-1mm}
%\trOrationisI

\vfill

\rubrica{Hebdomadarius dicit iterum Dominus vobiscum, vel cantor dicit:}

\vspace{2mm}

\sineinitiali{temporalia/domineexaudi.gtex}

\rubrica{Postea cantatur a cantore:}

\vspace{2mm}

\ifx\festum\undefined
\cuminitiali{II}{temporalia/benedicamus-semiduplex-vesp.gtex}
\else
\cuminitiali{II}{temporalia/benedicamus-solemnism-2vesp.gtex}
\fi

\vspace{1mm}
\fi

\end{document}

