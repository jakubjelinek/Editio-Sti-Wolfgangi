% LuaLaTeX

\documentclass[a4paper, twoside, 12pt]{article}
\usepackage[latin]{babel}
%\usepackage[landscape, left=3cm, right=1.5cm, top=2cm, bottom=1cm]{geometry} % okraje stranky
%\usepackage[landscape, a4paper, mag=1166, truedimen, left=2cm, right=1.5cm, top=1.6cm, bottom=0.95cm]{geometry} % okraje stranky
\usepackage[landscape, a4paper, mag=1400, truedimen, left=0.5cm, right=0.5cm, top=0.5cm, bottom=0.5cm]{geometry} % okraje stranky

\usepackage{fontspec}
\setmainfont[FeatureFile={junicode.fea}, Ligatures={Common, TeX}, RawFeature=+fixi]{Junicode}
%\setmainfont{Junicode}

% shortcut for Junicode without ligatures (for the Czech texts)
\newfontfamily\nlfont[FeatureFile={junicode.fea}, Ligatures={Common, TeX}, RawFeature=+fixi]{Junicode}

\usepackage{multicol}
\usepackage{color}
\usepackage{lettrine}
\usepackage{fancyhdr}

% usual packages loading:
\usepackage{luatextra}
\usepackage{graphicx} % support the \includegraphics command and options
\usepackage{gregoriotex} % for gregorio score inclusion
\usepackage{gregoriosyms}
\usepackage{wrapfig} % figures wrapped by the text
\usepackage{parcolumns}
\usepackage[contents={},opacity=1,scale=1,color=black]{background}
\usepackage{tikzpagenodes}
\usepackage{calc}
\usepackage{longtable}
\usetikzlibrary{calc}

\setlength{\headheight}{14.5pt}

\input{conventuscommune.tex} % Often used macros
%%%% Preklady jednotlivych zpevu (nektere se opakuji, a je dobre mit je
% vsechny na jedne hromade)

% HOURS ---

\newcommand{\trAntI}{\translatioCantus{Muž boží měl kožený toulec, pečlivě
zavázaný, jenž mu visel na šíji a~často se ho dotýkal.}}

\newcommand{\trAntII}{\translatioCantus{Klíč od~něho tak dobře střežil, že
dokud žil v~těle, nikdo z~jeho žáků nezvěděl, co je uvnitř.}}

\newcommand{\trAntIII}{\translatioCantus{Ale když se odebral z~tohoto
života, schránku otevřeli a~objevili v~ní žíněné roucho a~měděný řetěz
potřísněný krví.}}

\newcommand{\trAntIV}{\translatioCantus{A když prohlédli mistrovo tělo,
nalezli jeho tělo na čtyřech místech hluboce zbrázděno ranami od řetězu.}}

\newcommand{\trAntV}{\translatioCantus{Krev vytékající z~těch ran, místy
prostoupila i~žíněným rouchem.}}

\newcommand{\trCapituli}{\translatioCantus{
Miláčkovi Boha a~lidí,
Mojžíšovi požehnané paměti,~\gredagger{}
dopřál slávu rovnou slávě svatých~\grestar{}
učinil ho mocným na postrach nepřátelům
a~jeho slovy zastavil divy.}}

\newcommand{\trLectioBrevis}{\translatioCantus{
Pamatujte na své představené,
kteří vám hlásali Boží slovo.
Uvažte, jak oni skončili život, a~napodobujte jejich víru.
Ježíš Kristus je stejný včera i~dnes i~navěky.
Nenechte se svést věelijakými cizími naukami.}}

\newcommand{\trRespLaud}{\translatioCantus{Spravedlivého vodil Hospodin~\grestar{}
po přímých stezkách. \Vbardot{} A~ukázal mu Boží království.}}

\newcommand{\trRespLaudB}{\translatioCantus{Na tvých hradbách, Jeruzaléme,
ustanovil jsem strážné;~\grestar{}
budou bdít nad mým lidem. \Vbardot{} Ani ve dne, ani v~noci nesmějí nikdy
mlčet.}}

\newcommand{\trVersus}{\translatioCantus{\Vbardot{} Ústa spravedlivého šeptají moudrost, aleluja.
\Rbardot{} A~jeho jazyk ohlašuje právo, aleluja.}}

\newcommand{\trAntBenedictus}{\translatioCantus{Když na bujné oře vložili
nosítka a~sňali jim uzdu, vydali se přímo k~cele božího muže.}}

\newcommand{\trPreces}{\translatioCantus{
\noindent S vděčností chvalme Krista, dobrého Pastýře, \gredagger{} který dal život za své ovce, \grestar{} a~pokorně ho prosme: \Rbardot{} Pane, buď pastýřem svého lidu.

\noindent Kriste, ty dáváš církvi pastýře, a~jejich službou se ujímáš svého lidu, \grestar{} dej, ať v~lásce těch, kteří nás vedou, poznáváme, jak nás miluješ. \Rbardot{} Pane, buď pastýřem svého lidu.

\noindent Ty stále konáš skrze své zástupce službu pastýře a~učitele, \grestar{} nepřestávej nás nikdy vést prostřednictvím svých služebníků. \Rbardot{} Pane, buď pastýřem svého lidu.

\noindent Ty prokazuješ svému lidu skrze jeho pastýře službu lékaře duše i~těla, \grestar{} ochraňuj náš život a~veď nás ke svatosti. \Rbardot{} Pane, buď pastýřem svého lidu.

\noindent Ty posíláš své svaté, aby slovem i~příkladem vedli tvůj lid k~tobě, \grestar{} na jejich přímluvu nás posiluj, abychom vytrvali na cestě, která vede k~věčnému životu. \Rbardot{} Pane, buď pastýřem svého lidu.}}

\newcommand{\trOrationis}{\translatioCantus{Bože, jenž nám dopřáváš radovat
se z~výroční slavnosti svatého tvého vyznavače Havla, uděl dobrotivě,
abychom když slavíme jeho narození, též se řídili podobou jeho skutků.
Skrze…}}
 % Czech translations of the proper texts

\newcommand{\annusEditionis}{2020}

\setlength{\columnsep}{30pt} % prostor mezi sloupci

%%%%%%%%%%%%%%%%%%%%%%%%%%%%%%%%%%%%%%%%%%%%%%%%%%%%%%%%%%%%%%%%%%%%%%%%%%%%%%%%%%%%%%%%%%%%%%%%%%%%%%%%%%%%%
\begin{document}

% Here we set the space around the initial.
% Please report to http://home.gna.org/gregorio/gregoriotex/details for more details and options
\grechangedim{afterinitialshift}{2.2mm}{scalable}
\grechangedim{beforeinitialshift}{2.2mm}{scalable}
\grechangedim{interwordspacetext}{0.22 cm plus 0.15 cm minus 0.05 cm}{scalable}%
\grechangedim{annotationraise}{-0.2cm}{scalable}

% Here we set the initial font. Change 38 if you want a bigger initial.
% Emit the initials in red.
\grechangestyle{initial}{\color{red}\fontsize{38}{38}\selectfont}

\pagestyle{empty}

%%%% Titulni stranka
\begin{titulusOfficii}
\nomenFesti{Ad Completorium.}
\celebratio{In tempore post Pentecosten.}
\end{titulusOfficii}

\vfill

\begin{center}
%Ad usum et secundum consuetudines chori \guillemotright{}Conventus Choralis\guillemotleft.

%Editio Sancti Wolfgangi \annusEditionis
\end{center}

\pagebreak

\renewcommand{\headrulewidth}{0pt} % no horiz. rule at the header
\fancyhf{}
\pagestyle{fancy}

\cantusSineNeumas

\rubrica{Lector petit benedictionem, dicens:}

\cuminitiali{}{temporalia/jubedomnebenedicere.gtex}

%\trJubeDomne

\vfill

\pars{Benedictio.}

\cuminitiali{}{temporalia/benedictio-noctemquietam.gtex}

%\trComplBenedictio

\vfill

\pars{Lectio brevis.} \scriptura{1Ptr. 5, 8-9}

\cuminitiali{}{temporalia/lectiobrevis-fratressobrii.gtex}

%\trComplLectioBr

\vfill

\noindent \Vbardot{} Adiutórium nostrum in nómine Dómini.

\noindent \Rbardot{} Qui fecit cælum, et terram.

\vfill
\pagebreak

\pars{Confessio.}

\noindent Confíteor Deo omnipoténti, beátæ Maríæ semper Vírgini, beáto
Michaéli Archángelo, beáto Ioánni Baptístæ, sanctis Apóstolis Petro
et Paulo, ómnibus Sanctis, et vobis fratres: quia peccávi nimis cogitatióne,
verbo et ópere: mea culpa, mea culpa, mea máxima culpa.
Ideo precor beátam Maríam semper Vírginem, beátum Michaélem
Archángelum, beátum Ioánnem Baptístam, sanctos Apóstolos Petrum
et Paulum, omnes Sanctos, et vos fratres, oráre pro me ad Dóminum
Deum nostrum.

\vfill

\noindent \Vbardot{} Misereátur nostri omnípotens Deus, et dimíssis peccátis nostris, perdúcat
nos ad vitam ætérnam. \Rbardot{} Amen.

\vfill

\noindent \Vbardot{} Indulgéntiam, absolutiónem et remissiónem peccatórum nostrórum tríbuat nobis
omnípotens et miséricors Dóminus. \Rbardot{} Amen.

\vfill

\rubrica{Et facta absolutione dicitur:}

\sineinitiali{temporalia/convertenosdeus.gtex}

\vfill

\cuminitiali{}{temporalia/deusinadiutorium-communis.gtex}

\vfill
\pagebreak

\pars{Psalmus 1.}

\antiphona{VIII G}{temporalia/ant-miserere.gtex}

\scriptura{Ps. 4}

\initiumpsalmi{temporalia/ps4-initium-viii-G-auto.gtex}

\input{temporalia/ps4-viii-G.tex}

\vfill
\pagebreak

\pars{Psalmus 2.} \scriptura{Ps. 90}

\initiumpsalmi{temporalia/ps90-initium-viii-G-auto.gtex}

\input{temporalia/ps90-viii-G.tex}

\pagebreak

\pars{Psalmus 3.} \scriptura{Ps. 133}

\initiumpsalmi{temporalia/ps133-initium-viii-G-auto.gtex}

\input{temporalia/ps133-viii-G.tex}

\vfill

\antiphona{}{temporalia/ant-miserere.gtex}

\vfill
\pagebreak

\pars{Hymnus.}

\rubrica{In Feriis et Officiis Simplicibus per Annum.}

\vspace{-4mm}

\antiphona{VIII}{temporalia/hym-TeLucis-feria.gtex}

\vfill

\rubrica{In Festis Minoribus per Annum.}

\vspace{-4mm}

\antiphona{VIII}{temporalia/hym-TeLucis-minor.gtex}

\vfill
\pagebreak

\rubrica{In Sabbatis et Dominicis per Annum.}

\vspace{-4mm}

\antiphona{VIII}{temporalia/hym-TeLucis-dominica.gtex}

\vfill

\rubrica{In Festis Maioribus.}

\vspace{-4mm}

\antiphona{II}{temporalia/hym-TeLucis-major.gtex}

\vfill
\pagebreak

\rubrica{In Solemnitatibus.}

\vspace{-4mm}

\antiphona{VIII}{temporalia/hym-TeLucis-solemnis.gtex}

\vfill

\rubrica{In Festis B. Mariæ Virginis.}

\vspace{-4mm}

\antiphona{II}{temporalia/hym-TeLucis-bmv.gtex}

\vfill
\pagebreak

\pars{Capitulum.} \scriptura{Ier. 14, 9}

\cuminitiali{}{temporalia/capitulum-tuautem.gtex}

% preklad Jeruz. bible
%\trComplCapituli

\vfill

\pars{Responsorium breve.} \scriptura{Ps. 30, 6}

\cuminitiali{VI}{temporalia/resp-inmanus.gtex}

%\trRespCompl
\vfill

\pars{Versus.} \scriptura{Ps. 16, 8}

\sineinitiali{temporalia/versus-custodi.gtex}

%\noindent \trComplVersus

\vfill
\pagebreak

\cantusCumNeumis

\pars{Canticum Simeonis.}

\vspace{-3mm}

\antiphona{III a}{temporalia/ant-salvanos-antiquo.gtex}

%\trAntSalvaNos

%\vspace{-1mm}

\scriptura{Lc. 2, 29-32}

\vspace{-2mm}

\initiumpsalmi{temporalia/nuncdimittis-initium-iii-a-auto.gtex}

\input{temporalia/nuncdimittis-iii-a.tex} \Abardot{}

\vfill

\rubrica{Ante Orationem, cantatur a Superiore:}

\vspace{3mm}

\pars{Supplicatio Litaniæ.}

\cuminitiali{}{temporalia/supplicatiolitaniae.gtex}

\vspace{7mm}

\pars{Oratio Dominica.}

\noindent Pater noster.

\vfill
\pagebreak

\sineinitiali{temporalia/domineexaudi-simplex.gtex}

\vspace{7mm}

\pars{Oratio.}

\cantusSineNeumas

\cuminitiali{}{temporalia/oratio-visita.gtex}

%\trComplOrationis

\vfill

%\sineinitiali{temporalia/domineexaudi-communis.gtex}

\noindent \Vbardot{} Dómine, exáudi oratiónem meam. \Rbardot{} Et clamor meus ad te véniat.

\vfill

%\vfill

\sineinitiali{temporalia/benedicamus-minor.gtex}

\vfill

\pars{Benedictio.}

\noindent Benedícat et custódiat nos omnípotens et miséricors Dóminus,~\gredagger{}
Pater, et Fílius, et Spíritus Sanctus. \Rbardot{} Amen.

\vfill
\pagebreak

\pars{Antiphona finalis B. M. V.}

\vspace{-4mm}

\antiphona{I}{temporalia/an_salve_regina.gtex}

\rubrica{vel:}

\vspace{-4mm}

\antiphona{V}{temporalia/ant-salveregina-simplex.gtex}

\vfill
\pagebreak

\rubrica{vel:}

\vspace{-4mm}

\antiphona{I}{temporalia/ant-salveregina-cisterciensis.gtex}

\rubrica{vel:}

\antiphona{VII}{temporalia/ant-subtuum.gtex}

\vspace{7mm}

\sineinitiali{temporalia/versus-gaude.gtex}

\end{document}
