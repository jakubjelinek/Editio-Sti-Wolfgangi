% LuaLaTeX

\documentclass[a4paper, twoside, 12pt]{article}
\usepackage[latin]{babel} 
%\usepackage[landscape, left=3cm, right=1.5cm, top=2cm, bottom=1cm]{geometry} % okraje stranky
%\usepackage[landscape, a4paper, mag=1166, truedimen, left=2cm, right=1.5cm, top=1.6cm, bottom=0.95cm]{geometry} % okraje stranky
\usepackage[landscape, a4paper, mag=1400, truedimen, left=2cm, right=1.5cm, top=1.6cm, bottom=0.95cm]{geometry} % okraje stranky

\usepackage{fontspec}
\setmainfont[FeatureFile={junicode.fea}, Ligatures={Common, TeX}, RawFeature=+fixi]{Junicode}
%\setmainfont{Junicode}

% shortcut for Junicode without ligatures (for the Czech texts)
\newfontfamily\nlfont[FeatureFile={junicode.fea}, Ligatures={Common, TeX}, RawFeature=+fixi]{Junicode}

% Hebrew font:
% http://scripts.sil.org/cms/scripts/page.php?site_id=nrsi&id=SILHebrUnic2
\newfontfamily\hebfont[Scale=1]{Ezra SIL}

\usepackage{multicol}
\usepackage{color}
\usepackage{lettrine}
\usepackage{fancyhdr}

% usual packages loading:
\usepackage{luatextra}
\usepackage{graphicx} % support the \includegraphics command and options
\usepackage{gregoriotex} % for gregorio score inclusion
\usepackage{gregoriosyms}
\usepackage{wrapfig} % figures wrapped by the text
\usepackage{parcolumns}
\usepackage[contents={},opacity=1,scale=1,color=black]{background}
\usepackage{tikzpagenodes}
\usepackage{calc}
\usepackage{longtable}
\usetikzlibrary{calc}

\setlength{\headheight}{14.5pt}

\input{conventuscommune.tex} % Often used macros
%%%% Preklady jednotlivych zpevu (nektere se opakuji, a je dobre mit je
% vsechny na jedne hromade)

% HOURS ---

\newcommand{\trAntI}{\translatioCantus{Muž boží měl kožený toulec, pečlivě
zavázaný, jenž mu visel na šíji a~často se ho dotýkal.}}

\newcommand{\trAntII}{\translatioCantus{Klíč od~něho tak dobře střežil, že
dokud žil v~těle, nikdo z~jeho žáků nezvěděl, co je uvnitř.}}

\newcommand{\trAntIII}{\translatioCantus{Ale když se odebral z~tohoto
života, schránku otevřeli a~objevili v~ní žíněné roucho a~měděný řetěz
potřísněný krví.}}

\newcommand{\trAntIV}{\translatioCantus{A když prohlédli mistrovo tělo,
nalezli jeho tělo na čtyřech místech hluboce zbrázděno ranami od řetězu.}}

\newcommand{\trAntV}{\translatioCantus{Krev vytékající z~těch ran, místy
prostoupila i~žíněným rouchem.}}

\newcommand{\trCapituli}{\translatioCantus{
Miláčkovi Boha a~lidí,
Mojžíšovi požehnané paměti,~\gredagger{}
dopřál slávu rovnou slávě svatých~\grestar{}
učinil ho mocným na postrach nepřátelům
a~jeho slovy zastavil divy.}}

\newcommand{\trLectioBrevis}{\translatioCantus{
Pamatujte na své představené,
kteří vám hlásali Boží slovo.
Uvažte, jak oni skončili život, a~napodobujte jejich víru.
Ježíš Kristus je stejný včera i~dnes i~navěky.
Nenechte se svést věelijakými cizími naukami.}}

\newcommand{\trRespLaud}{\translatioCantus{Spravedlivého vodil Hospodin~\grestar{}
po přímých stezkách. \Vbardot{} A~ukázal mu Boží království.}}

\newcommand{\trRespLaudB}{\translatioCantus{Na tvých hradbách, Jeruzaléme,
ustanovil jsem strážné;~\grestar{}
budou bdít nad mým lidem. \Vbardot{} Ani ve dne, ani v~noci nesmějí nikdy
mlčet.}}

\newcommand{\trVersus}{\translatioCantus{\Vbardot{} Ústa spravedlivého šeptají moudrost, aleluja.
\Rbardot{} A~jeho jazyk ohlašuje právo, aleluja.}}

\newcommand{\trAntBenedictus}{\translatioCantus{Když na bujné oře vložili
nosítka a~sňali jim uzdu, vydali se přímo k~cele božího muže.}}

\newcommand{\trPreces}{\translatioCantus{
\noindent S vděčností chvalme Krista, dobrého Pastýře, \gredagger{} který dal život za své ovce, \grestar{} a~pokorně ho prosme: \Rbardot{} Pane, buď pastýřem svého lidu.

\noindent Kriste, ty dáváš církvi pastýře, a~jejich službou se ujímáš svého lidu, \grestar{} dej, ať v~lásce těch, kteří nás vedou, poznáváme, jak nás miluješ. \Rbardot{} Pane, buď pastýřem svého lidu.

\noindent Ty stále konáš skrze své zástupce službu pastýře a~učitele, \grestar{} nepřestávej nás nikdy vést prostřednictvím svých služebníků. \Rbardot{} Pane, buď pastýřem svého lidu.

\noindent Ty prokazuješ svému lidu skrze jeho pastýře službu lékaře duše i~těla, \grestar{} ochraňuj náš život a~veď nás ke svatosti. \Rbardot{} Pane, buď pastýřem svého lidu.

\noindent Ty posíláš své svaté, aby slovem i~příkladem vedli tvůj lid k~tobě, \grestar{} na jejich přímluvu nás posiluj, abychom vytrvali na cestě, která vede k~věčnému životu. \Rbardot{} Pane, buď pastýřem svého lidu.}}

\newcommand{\trOrationis}{\translatioCantus{Bože, jenž nám dopřáváš radovat
se z~výroční slavnosti svatého tvého vyznavače Havla, uděl dobrotivě,
abychom když slavíme jeho narození, též se řídili podobou jeho skutků.
Skrze…}}
 % Czech translations of the proper texts

\newcommand{\annusEditionis}{2020}

\def\hebinitial#1{%
\leavevmode{\newbox\hebbox\setbox\hebbox\hbox{\hebfont{#1}\hskip 1mm}\kern -\wd\hebbox\hbox{\hebfont{#1}\hskip 1mm}}%
}

%%%% Vicekrat opakovane kousky

\setlength{\columnsep}{30pt} % prostor mezi sloupci

%%%%%%%%%%%%%%%%%%%%%%%%%%%%%%%%%%%%%%%%%%%%%%%%%%%%%%%%%%%%%%%%%%%%%%%%%%%%%%%%%%%%%%%%%%%%%%%%%%%%%%%%%%%%%
\begin{document}

% Here we set the space around the initial.
% Please report to http://home.gna.org/gregorio/gregoriotex/details for more details and options
\grechangedim{afterinitialshift}{2.2mm}{scalable}
\grechangedim{beforeinitialshift}{2.2mm}{scalable}

\grechangedim{interwordspacetext}{0.32 cm plus 0.15 cm minus 0.05 cm}{scalable}%
\grechangedim{annotationraise}{-0.2cm}{scalable}

% Here we set the initial font. Change 38 if you want a bigger initial.
% Emit the initials in red.
\grechangestyle{initial}{\color{red}\fontsize{38}{38}\selectfont}

\pagestyle{empty}

%%%% Titulni stranka
\begin{titulusOfficii}
\nomenFesti{Sabbato in Albis.}
\celebratio{Semiduplex.}
\end{titulusOfficii}

\pagebreak

% graphic
\renewcommand{\headrulewidth}{0pt} % no horiz. rule at the header
\fancyhf{}
\pagestyle{fancy}

\cantusSineNeumas

\hora{Ad Matutinum.}

\vspace{2mm}

\cuminitiali{}{temporalia/dominelabiamea.gtex}

\vspace{2mm}

\pars{Invitatorium.} \scriptura{Lc. 24, 34; Psalmus 94; \textbf{H232}}

\vspace{-6mm}

\antiphona{VI}{temporalia/inv-surrexitdominusvere.gtex}

\rubrica{Hymnus non dicatur.}

\vfill
\pagebreak

\pars{Psalmus 1.} \scriptura{\Abardot{} Ex. 3, 14; \textbf{H226}}

\vspace{-5mm}

\antiphona{I f}{temporalia/an-egosumquisum.gtex}

\vspace{-5mm}

\scriptura{Ps. 1}

\vspace{-2mm}

\initiumpsalmi{temporalia/ps1-initium-i-F-auto.gtex}

%\psalmusEtTranslatioT{temporalia/ps1-comb.tex}{10cm}

\input{temporalia/ps1.tex}

\vfill
\pagebreak

\pars{Psalmus 2.} \scriptura{\Abardot{} Ps. 2, 8; \textbf{H226}}

\vspace{-4mm}

\antiphona{I f}{temporalia/an-postulavi.gtex}

\vspace{-0.3cm}
\vfill
\pagebreak

\scriptura{Ps. 2}

\initiumpsalmi{temporalia/ps2-initium-i-f-auto.gtex}

%\psalmusEtTranslatioT{temporalia/ps2-comb.tex}{10cm}

\input{temporalia/ps2.tex}

\vfill
\pagebreak

\antiphona{}{temporalia/an-postulavi.gtex}

\vfill
\pagebreak

\pars{Psalmus 3.} \scriptura{\Abardot{} Ps. 3, 6; \textbf{H226}}

\vspace{-4mm}

\antiphona{VIII C}{temporalia/an-egodormivi.gtex}

\vspace{-0.3cm}

\scriptura{Ps. 3}

\initiumpsalmi{temporalia/ps3-initium-viii-c-auto.gtex}

%\psalmusEtTranslatioT{temporalia/ps3-comb.tex}{10cm}

\input{temporalia/ps3.tex}

\vfill
\pagebreak

\noindent \Vbardot{} Gavísi sunt discípuli, allelúia.
\noindent \Rbardot{} Viso Dómino, allelúia.

\noindent Pater noster.

\pars{Absolutio.}

\cuminitiali{}{temporalia/absolutio-avinculis.gtex}

\cuminitiali{}{temporalia/benedictio-solemn-evangelica.gtex}

\vfill
\pagebreak

\pars{Lectio I.} \scriptura{Io. 20, 1-9}

\noindent Léctio sancti Evangélii secúndum Ioánnem.


\noindent In illo témpore: Una sabbáti María Magdaléne venit mane, cum adhuc ténebræ essent, ad monuméntum. Et réliqua.

\vspace{7mm}

\noindent Homilía sancti Gregórii Papæ.

\scriptura{Homilia 22 in Evangelia}

\noindent Léctio sancti Evangélii, quam modo, fratres, audístis, valde in superfície histórica est apérta: sed eius nobis sunt mystéria sub brevitáte requirénda. María Magdaléne, cum adhuc ténebræ essent, venit ad monuméntum. Iuxta históriam notátur hora: iuxta intelléctum vero mýsticum, requiréntis signátur intellegéntia. María étenim auctórem ómnium, quem in carne víderat mórtuum, quærébat in monuménto; et quia hunc mínime invénit, furátum crédidit. Adhuc ergo erant ténebræ, cum venit ad monuméntum. Cucúrrit cítius, discípulis nuntiávit: sed illi præ céteris cucurrérunt, qui præ céteris amavérunt, vidélicet Petrus et Ioánnes.

\noindent \Vbardot{} Tu autem, Dómine, miserére nobis.
\noindent \Rbardot{} Deo grátias.

\vfill
\pagebreak

\pars{Responsorium 1.} \scriptura{\Rbardot{} Aug. Serm. 362; \Vbardot{} ibid. Enarr. in Psalmos: Ps: 102, sermo 2; \textbf{H203}}

\vspace{-5mm}

\responsorium{II}{temporalia/resp-christusresurgens.gtex}{}

\vfill
\pagebreak

\cuminitiali{}{temporalia/benedictio-solemn-divinum.gtex}

\vspace{7mm}

\pars{Lectio II.}

\noindent Currébant autem duo simul: sed Ioánnes præcucúrrit cítius Petro. Venit prior ad monuméntum, et íngredi non præsúmpsit. Venit ergo postérior Petrus, et intrávit. Quid, fratres, quid cursus signíficat? Numquid hæc tam subtílis Evangelístæ descríptio a mystériis vacáre credénda est? Mínime. Neque enim se Ioánnes et præísse, et non intrásse díceret, si in ipsa sui trepidatióne mystérium defuísse credidísset. Quid ergo per Ioánnem, nisi synagóga: quid per Petrum, nisi Ecclésia designátur?

\noindent \Vbardot{} Tu autem, Dómine, miserére nobis.
\noindent \Rbardot{} Deo grátias.

\vfill
\pagebreak

\pars{Responsorium 2.} \scriptura{\Rbardot{} Apoc. 7, 17; \Vbardot{} ibid. 7, 9; \textbf{H236}}

\vspace{-5mm}

\responsorium{VII}{temporalia/resp-istisuntagni.gtex}{}

\vfill
\pagebreak

\cuminitiali{}{temporalia/benedictio-solemn-adsocietatem.gtex}

\vspace{7mm}

\pars{Lectio III.}

\noindent Nec mirum esse videátur, quod per iuniórem synagóga, per seniórem vero Ecclésia signári perhibétur: quia etsi ad Dei cultum prior est synagóga, quam Ecclésia géntium, ad usum tamen sǽculi prior est multitúdo géntium, quam synagóga, Paulo attestánte, qui ait: Quia non prius quod spiritále est, sed quod animále. Per seniórem ergo Petrum significátur Ecclésia géntium: per iuniórem vero Ioánnem synagóga Iudæórum. Currunt ambo simul: quia ab ortus sui témpore usque ad occásum, pari et commúni via, etsi non pari et commúni sensu, gentílitas cum synagóga cucúrrit. Venit synagóga prior ad monuméntum, sed mínime intrávit: quia legis quidem mandáta percépit, prophetías de incarnatióne ac passióne Domínica audívit, sed crédere in mórtuum nóluit.

\noindent \Vbardot{} Tu autem, Dómine, miserére nobis.
\noindent \Rbardot{} Deo grátias.

\vfill
\pagebreak

% Te Deum

%\pars{Hymnus Ambrosianus}

\vspace{-5mm}

{
\grechangedim{interwordspacetext}{0.22 cm plus 0.15 cm minus 0.05 cm}{scalable}%
\cuminitiali{III}{temporalia/tedeum-solemnis.gtex}
\grechangedim{interwordspacetext}{0.32 cm plus 0.15 cm minus 0.05 cm}{scalable}%
}

\vfill
\pagebreak

\rubrica{Reliqua omittuntur, nisi Laudes separandæ sint.}

\pars{Oratio}

\noindent \Vbardot{} Dómine, exáudi oratiónem meam.

\noindent \Rbardot{} Et clamor meus ad te véniat.

Orémus:

\noindent Concéde, quǽsumus, omnípotens Deus: \gredagger{} ut, qui festa paschália venerándo égimus, \grestar{} per hæc contíngere ad gáudia ætérna mereámur. Per Dóminum.

\noindent \Rbardot{} Amen.

\vspace{7mm}

\pars{Conclusio}

\noindent \Vbardot{} Dómine, exáudi oratiónem meam.

\noindent \Rbardot{} Et clamor meus ad te véniat.

\noindent \Vbardot{} Benedicámus Dómino, allelúia, allelúia.

\noindent \Rbardot{} Deo grátias, allelúia, allelúia.

\noindent \Vbardot{} Fidélium ánimæ per misericórdiam Dei requiéscant in pace.

\noindent \Rbardot{} Amen.

\end{document}
