% LuaLaTeX

\documentclass[a4paper, twoside, 12pt]{article}
\usepackage[latin]{babel}
%\usepackage[landscape, left=3cm, right=1.5cm, top=2cm, bottom=1cm]{geometry} % okraje stranky
%\usepackage[landscape, a4paper, mag=1166, truedimen, left=2cm, right=1.5cm, top=1.6cm, bottom=0.95cm]{geometry} % okraje stranky
\usepackage[landscape, a4paper, mag=1400, truedimen, left=0.5cm, right=0.5cm, top=0.5cm, bottom=0.5cm]{geometry} % okraje stranky

\usepackage{fontspec}
\setmainfont[FeatureFile={junicode.fea}, Ligatures={Common, TeX}, RawFeature=+fixi]{Junicode}
%\setmainfont{Junicode}

% shortcut for Junicode without ligatures (for the Czech texts)
\newfontfamily\nlfont[FeatureFile={junicode.fea}, Ligatures={Common, TeX}, RawFeature=+fixi]{Junicode}

% Hebrew font:
% http://scripts.sil.org/cms/scripts/page.php?site_id=nrsi&id=SILHebrUnic2
\newfontfamily\hebfont[Scale=1]{Ezra SIL}

\usepackage{multicol}
\usepackage{color}
\usepackage{lettrine}
\usepackage{fancyhdr}

% usual packages loading:
\usepackage{luatextra}
\usepackage{graphicx} % support the \includegraphics command and options
\usepackage{gregoriotex} % for gregorio score inclusion
\usepackage{gregoriosyms}
\usepackage{wrapfig} % figures wrapped by the text
\usepackage{parcolumns}
\usepackage[contents={},opacity=1,scale=1,color=black]{background}
\usepackage{tikzpagenodes}
\usepackage{calc}
\usepackage{longtable}
\usetikzlibrary{calc}

\setlength{\headheight}{14.5pt}

\input{conventuscommune.tex} % Often used macros
%%%% Preklady jednotlivych zpevu (nektere se opakuji, a je dobre mit je
% vsechny na jedne hromade)

% HOURS ---

\newcommand{\trAntI}{\translatioCantus{Muž boží měl kožený toulec, pečlivě
zavázaný, jenž mu visel na šíji a~často se ho dotýkal.}}

\newcommand{\trAntII}{\translatioCantus{Klíč od~něho tak dobře střežil, že
dokud žil v~těle, nikdo z~jeho žáků nezvěděl, co je uvnitř.}}

\newcommand{\trAntIII}{\translatioCantus{Ale když se odebral z~tohoto
života, schránku otevřeli a~objevili v~ní žíněné roucho a~měděný řetěz
potřísněný krví.}}

\newcommand{\trAntIV}{\translatioCantus{A když prohlédli mistrovo tělo,
nalezli jeho tělo na čtyřech místech hluboce zbrázděno ranami od řetězu.}}

\newcommand{\trAntV}{\translatioCantus{Krev vytékající z~těch ran, místy
prostoupila i~žíněným rouchem.}}

\newcommand{\trCapituli}{\translatioCantus{
Miláčkovi Boha a~lidí,
Mojžíšovi požehnané paměti,~\gredagger{}
dopřál slávu rovnou slávě svatých~\grestar{}
učinil ho mocným na postrach nepřátelům
a~jeho slovy zastavil divy.}}

\newcommand{\trLectioBrevis}{\translatioCantus{
Pamatujte na své představené,
kteří vám hlásali Boží slovo.
Uvažte, jak oni skončili život, a~napodobujte jejich víru.
Ježíš Kristus je stejný včera i~dnes i~navěky.
Nenechte se svést věelijakými cizími naukami.}}

\newcommand{\trRespLaud}{\translatioCantus{Spravedlivého vodil Hospodin~\grestar{}
po přímých stezkách. \Vbardot{} A~ukázal mu Boží království.}}

\newcommand{\trRespLaudB}{\translatioCantus{Na tvých hradbách, Jeruzaléme,
ustanovil jsem strážné;~\grestar{}
budou bdít nad mým lidem. \Vbardot{} Ani ve dne, ani v~noci nesmějí nikdy
mlčet.}}

\newcommand{\trVersus}{\translatioCantus{\Vbardot{} Ústa spravedlivého šeptají moudrost, aleluja.
\Rbardot{} A~jeho jazyk ohlašuje právo, aleluja.}}

\newcommand{\trAntBenedictus}{\translatioCantus{Když na bujné oře vložili
nosítka a~sňali jim uzdu, vydali se přímo k~cele božího muže.}}

\newcommand{\trPreces}{\translatioCantus{
\noindent S vděčností chvalme Krista, dobrého Pastýře, \gredagger{} který dal život za své ovce, \grestar{} a~pokorně ho prosme: \Rbardot{} Pane, buď pastýřem svého lidu.

\noindent Kriste, ty dáváš církvi pastýře, a~jejich službou se ujímáš svého lidu, \grestar{} dej, ať v~lásce těch, kteří nás vedou, poznáváme, jak nás miluješ. \Rbardot{} Pane, buď pastýřem svého lidu.

\noindent Ty stále konáš skrze své zástupce službu pastýře a~učitele, \grestar{} nepřestávej nás nikdy vést prostřednictvím svých služebníků. \Rbardot{} Pane, buď pastýřem svého lidu.

\noindent Ty prokazuješ svému lidu skrze jeho pastýře službu lékaře duše i~těla, \grestar{} ochraňuj náš život a~veď nás ke svatosti. \Rbardot{} Pane, buď pastýřem svého lidu.

\noindent Ty posíláš své svaté, aby slovem i~příkladem vedli tvůj lid k~tobě, \grestar{} na jejich přímluvu nás posiluj, abychom vytrvali na cestě, která vede k~věčnému životu. \Rbardot{} Pane, buď pastýřem svého lidu.}}

\newcommand{\trOrationis}{\translatioCantus{Bože, jenž nám dopřáváš radovat
se z~výroční slavnosti svatého tvého vyznavače Havla, uděl dobrotivě,
abychom když slavíme jeho narození, též se řídili podobou jeho skutků.
Skrze…}}
 % Czech translations of the proper texts

\newcommand{\annusEditionis}{2016}

\def\hebinitial#1{%
\leavevmode{\newbox\hebbox\setbox\hebbox\hbox{\hebfont{#1}\hskip 1mm}\kern -\wd\hebbox\hbox{\hebfont{#1}\hskip 1mm}}%
}

%%%% Vicekrat opakovane kousky

\newcommand{\anteOrationem}{
  \rubrica{Ante Orationem, cantatur a Superiore:}

  \pars{Supplicatio Litaniæ.}

  \cuminitiali{}{temporalia/supplicatiolitaniae.gtex}

  \pars{Oratio Dominica.}

  \cuminitiali{}{temporalia/oratiodominica.gtex}

  \rubrica{Deinde dicitur ab Hebdomadario:}

  \cuminitiali{}{temporalia/dominusvobiscum-solemnis.gtex}

  \rubrica{In choro monialium loco Dominus vobiscum dicitur:}

  \sineinitiali{temporalia/domineexaudi.gtex}
}

\newcommand{\tuAutem}{
  \noindent \Vbardot{} Tu autem, Dómine, miserére nobis.
  \noindent \Rbardot{} Deo grátias.
}

\setlength{\columnsep}{30pt} % prostor mezi sloupci

%%%%%%%%%%%%%%%%%%%%%%%%%%%%%%%%%%%%%%%%%%%%%%%%%%%%%%%%%%%%%%%%%%%%%%%%%%%%%%%%%%%%%%%%%%%%%%%%%%%%%%%%%%%%%
\begin{document}

% Here we set the space around the initial.
% Please report to http://home.gna.org/gregorio/gregoriotex/details for more details and options
\grechangedim{afterinitialshift}{2.2mm}{scalable}
\grechangedim{beforeinitialshift}{2.2mm}{scalable}

\grechangedim{interwordspacetext}{0.22 cm plus 0.15 cm minus 0.05 cm}{scalable}%
\grechangedim{annotationraise}{-2mm}{scalable}

% Here we set the initial font. Change 38 if you want a bigger initial.
% Emit the initials in red.
\grechangestyle{initial}{\color{red}\fontsize{38}{38}\selectfont}

\pagestyle{empty}

%%%% Titulni stranka
\begin{titulusOfficii}
\dies{Die 16. Octobris.}
\nomenFesti{In Nativitate S. Galli, Confessoris.}
\celebratio{Duplex 2. classis.}
\end{titulusOfficii}

% graphic
\vspace{1.5cm}
\begin{center}
% http://e-codices.unifr.ch/en/csg/0602/42/0/Sequence-612
\includegraphics[width=8cm]{gallus2.jpg}
\end{center}

\vfill

\iffalse
\begin{center}
Ad usum et secundum consuetudines chori \guillemotright Conventus Choralis\guillemotleft.

Editio Sancti Wolfgangi \annusEditionis
\end{center}
\fi

\pagebreak

\renewcommand{\headrulewidth}{0pt} % no horiz. rule at the header
\fancyhf{}
\pagestyle{fancy}

\pars{Oratio ante divinum Officium.}

\lettrine{{\color{red}A}}{peri,} Dómine, os meum ad benedicéndum nomen sanctum tuum:
munda quoque cor meum ab ómnibus vanis, pervérsis, et aliénis
cogitatiónibus:
intelléctum illúmina, afféctum inflámma,
ut digne, atténte ac devóte hoc Offícium recitáre váleam,
et exaudíri mérear ante conspéctum Divínæ Maiestátis tuæ.
Per Christum, Dóminum nostrum.
\Rbardot{} Amen.

Dómine, in unióne illíus divínæ intentiónis,
qua ipse in terris laudes Deo persolvísti,
has tibi Horas \rubricatum{(vel \textnormal{hanc tibi Horam})} persólvo.

%\trOratioAnteOfficium

\vfill

\pars{Oratio post divinum Officium.}

\rubrica{
  Orationem sequentem devote post Officium recitantibus
  Leo Papa X. defectus, et culpas in eo persolvendo ex humana
  fragilitate contractas, indulsit, et dicitur flexis genibus.
}

\lettrine{{\color{red}S}}{acrosánctæ} et indivíduæ Trinitáti,
crucifíxi Dómini nostri Iesu Christi humanitáti,
beatíssimæ et gloriosíssimæ sempérque Vírginis Maríæ
fecúndæ integritáti,
et ómnium Sanctórum universitáti
sit sempitérna laus, honor, virtus et glória
ab omni creatúra,
nobísque remíssio ómnium peccatórum,
per infiníta sǽcula sæculórum.
\Rbardot{} Amen.

\noindent \Vbardot{} Beáta víscera Maríæ Vírginis, quæ portavérunt
ætérni Patris Fílium.\\
\Rbardot{} Et beáta úbera, quæ lactavérunt Christum Dóminum.

\rubrica{Et dicitur secreto \textnormal{Pater noster.} et \textnormal{Ave María.}}

%\trOratioPostOfficium

\vfill

\pars{ } \scriptura{ }

\cantusSineNeumas

\hora{Ad I. Vesperas.} %%%%%%%%%%%%%%%%%%%%%%%%%%%%%%%%%%%%%%%%%%%%%%%%%%%%%
%\sideThumbs{I. Vesperæ}

{
\grechangedim{interwordspacetext}{0.18 cm plus 0.15 cm minus 0.05 cm}{scalable}%
\cuminitiali{}{temporalia/deusinadiutorium-communis.gtex}
\grechangedim{interwordspacetext}{0.22 cm plus 0.15 cm minus 0.05 cm}{scalable}%
}

\vfill
\pagebreak

\cantusSineNeumas

\pars{Psalmus 1.} \scriptura{Vita S. Galli XXXII, 1.2; \textbf{H325}}

\vspace{-5mm}

\antiphona{VI F}{temporalia/ant1.gtex}

%\trAntI

\scriptura{Ps. 144, 10-21}

\initiumpsalmi{temporalia/ps144ii-initium-vi-F-auto.gtex}

%\psalmusEtTranslatioT{temporalia/ps144ii-comb.tex}{10cm}
\input{temporalia/ps144ii.tex}
\antiphona{}{temporalia/ant1.gtex} % repeat the antiphon - new page

\vfill
\pagebreak

\pars{Psalmus 2.} \scriptura{Vita S. Galli XXXII, 1; \textbf{H325}}

\vspace{-5mm}

\antiphona{VII a}{temporalia/ant2.gtex}

%\trAntII

\vspace{-2mm}

\scriptura{Ps. 145}

\initiumpsalmi{temporalia/ps145-initium-vii-a-auto.gtex}

\vspace{-2mm}

%\psalmusEtTranslatioT{temporalia/ps145-comb.tex}{9cm}
\input{temporalia/ps145.tex} \Abardot{}

%\antiphona{}{temporalia/ant2.gtex} % repeat the antiphon - new page

\vfill
\pagebreak

\pars{Psalmus 3.} \scriptura{Vita S. Galli XXXII, 2; \textbf{H326}}

\vspace{-5mm}

\antiphona{I g}{temporalia/ant3.gtex}

%\trAntIII

\scriptura{Ps. 146}

\initiumpsalmi{temporalia/ps146-initium-i-g-auto.gtex}
%\psalmusEtTranslatioT{temporalia/ps146-comb.tex}{10cm}
\input{temporalia/ps146.tex}
\antiphona{}{temporalia/ant3.gtex} % repeat the antiphon - new page

\vfill
\pagebreak

\pars{Psalmus 4.} \scriptura{Vita S. Galli XXXII, 2; \textbf{H326}}

\vspace{-5mm}

\antiphona{III a}{temporalia/ant4.gtex}

%\trAntIV

\scriptura{Ps. 147}

\initiumpsalmi{temporalia/ps147-initium-iii-a-auto.gtex}
%\psalmusEtTranslatioT{temporalia/ps147-comb.tex}{10cm}
\input{temporalia/ps147.tex} \Abardot{}
%\antiphona{}{temporalia/ant4.gtex} % repeat the antiphon - new page

\vfill
\pagebreak

\raggedcolumns

% Capitulum. %%%
\cantusSineNeumas

\pars{Capitulum.} \scriptura{Sir. 45, 1-2}

\cuminitiali{}{temporalia/capitulum-DilectusDeo.gtex}

% preklad Jeruz. bible
%\trCapituli

\vfill
\pars{Responsorium breve.} \scriptura{Ps. 36, 30}

\antiphona{VI}{temporalia/respv.gtex}

%\trRespVesp

\vfill
\pagebreak

% Hymnus. %%%
\pars{Hymnus.} \scriptura{Walahfrid Strabus (\olddag{} 849)}

{
\grechangedim{interwordspacetext}{0.20 cm plus 0.15 cm minus 0.05 cm}{scalable}%
\cuminitiali{I}{temporalia/hym-VitaSanctorum.gtex}
\grechangedim{interwordspacetext}{0.22 cm plus 0.15 cm minus 0.05 cm}{scalable}%
}

\vfill

\pars{Versus.} \scriptura{Ps. 36, 30}

% Versus. %%%
\sineinitiali{temporalia/versus-os.gtex}

%\noindent \trVersus

\vfill
\pagebreak

\cantusSineNeumas

\pars{Canticum B. Mariæ V.} \scriptura{Vita S. Galli X, 1; \textbf{H320}}

\vspace{-5mm}

\antiphona{VIII G}{temporalia/ant-magn-vesp1.gtex}

%\trAntMagnificatI

\scriptura{Lc. 1, 46-55}

\cantusSineNeumas
\initiumpsalmi{temporalia/magnificat-initium-viiisoll-G.gtex}

%\vspace{-4mm}

%\psalmusEtTranslatioT{temporalia/magnificat-comb.tex}{10.3cm}
\input{temporalia/magnificat.tex}

\antiphona{}{temporalia/ant-magn-vesp1.gtex} % repeat the antiphon - new page

\vfill
\pagebreak

\anteOrationem

\pagebreak

%% Oratio. %%%
\pars{Oratio.}

\cuminitiali{}{temporalia/oratio.gtex}
%\trOrationis

\vfill

\rubrica{Hebdomadarius dicit iterum Dominus vobiscum, vel cantor dicit:}

\vspace{2mm}

\sineinitiali{temporalia/domineexaudi.gtex}

\rubrica{Postea cantatur a cantore:}

\vspace{2mm}

\cuminitiali{II}{temporalia/benedicamus-duplex-vesperae.gtex}

\vfill
\pagebreak

\hora{Ad Matutinum.} %%%%%%%%%%%%%%%%%%%%%%%%%%%%%%%%%%%%%%%%%%%%%%%%%%%%%%%%%%
%\sideThumbs{Matutinum}

\vspace{2mm}

\cuminitiali{}{temporalia/dominelabiamea.gtex}

\vspace{2mm}

\pars{Invitatorium.} \scriptura{\textbf{H320}; Ps. 94 (Textus antiquus latinus); \textbf{H443}}

\vspace{-6mm}

\antiphona{III}{temporalia/inv-regemconfessorum.gtex}

%\trMatInvitatorium

\vfill
\pagebreak

\pars{Hymnus.}

\vspace{-5mm}

% Praglia II
\antiphona{I}{temporalia/hym-IsteConfessor-alt.gtex}

\vfill
\pagebreak

%\subhora{In I. Nocturno}

\pars{Psalmus 1.} \scriptura{Vita S. Galli I, 1; \textbf{H321}}

\vspace{-5mm}

\antiphona{I D\textsuperscript{*}}{temporalia/matant1.gtex}

%\trMatAntI

\scriptura{Psalmus 1.}

\initiumpsalmi{temporalia/ps1-initium-i-D_-auto.gtex}

%\vspace{-8mm}

%\psalmusEtTranslatioT{temporalia/ps1-comb.tex}{10cm}
\input{temporalia/ps1.tex} \Abardot{}
%\antiphona{}{temporalia/matant1.gtex} % repeat the antiphon - new page

\vfill
\pagebreak

\pars{Psalmus 2.} \scriptura{Vita S. Galli I, 1; \textbf{H321}}

\vspace{-4mm}

\antiphona{II D}{temporalia/matant2.gtex}

%\trMatAntII

%\vspace{-2mm}

\scriptura{Psalmus 2.}

\initiumpsalmi{temporalia/ps2-initium-ii-D-auto.gtex}

%\vspace{-5mm}

%\psalmusEtTranslatioT{temporalia/ps2-comb.tex}{10cm}
\input{temporalia/ps2.tex}

\vfill

\antiphona{}{temporalia/matant2.gtex} % repeat the antiphon - new page

\vfill
\pagebreak

\pars{Psalmus 3.} \scriptura{Vita S. Galli IX, 1; \textbf{H321}}

\vspace{-5mm}

\antiphona{VII a}{temporalia/matant3.gtex}

%\trMatAntIII

\scriptura{Psalmus 3.}

\initiumpsalmi{temporalia/ps3-initium-vii-a-auto.gtex}

%\psalmusEtTranslatioT{temporalia/ps3-comb.tex}{10cm}
\input{temporalia/ps3.tex} \Abardot{}
%\antiphona{}{temporalia/matant3.gtex} % repeat the antiphon - new page

\vfill
\pagebreak

\pars{Psalmus 4.} \scriptura{Vita S. Galli IX, 1; \textbf{H321}}

\vspace{-5mm}

\antiphona{VIII G\textsuperscript{2}}{temporalia/matant4.gtex}

%\trMatAntIV

\scriptura{Psalmus 4.}

\initiumpsalmi{temporalia/ps4-initium-viii-G2-auto.gtex}

%\vspace{-5mm}

%\psalmusEtTranslatioT{temporalia/ps4-comb.tex}{10cm}
\input{temporalia/ps4.tex} \Abardot{}
%\antiphona{}{temporalia/matant4.gtex} % repeat the antiphon - new page

\vfill
\pagebreak

\pars{Psalmus 5.} \scriptura{Vita S. Galli IX, 4; \textbf{H321}}

\vspace{-5mm}

\antiphona{IV E}{temporalia/matant5.gtex}

%\trMatAntV

\scriptura{Psalmus 5.}

\initiumpsalmi{temporalia/ps5-initium-iv-E-auto.gtex}

%\psalmusEtTranslatioT{temporalia/ps5-comb.tex}{10cm}
\input{temporalia/ps5.tex}
\antiphona{}{temporalia/matant5.gtex} % repeat the antiphon - new page

\vfill
\pagebreak

\pars{Psalmus 6.} \scriptura{Vita S. Galli IX, 4; \textbf{H321}}

\vspace{-5mm}

{
\grechangedim{interwordspacetext}{0.18 cm plus 0.15 cm minus 0.05 cm}{scalable}%
\antiphona{VIII G}{temporalia/matant6.gtex}
\grechangedim{interwordspacetext}{0.22 cm plus 0.15 cm minus 0.05 cm}{scalable}%
}

%\trMatAntVI

\scriptura{Psalmus 8.}

\initiumpsalmi{temporalia/ps8-initium-viii-G-auto.gtex}

%\psalmusEtTranslatioT{temporalia/ps8-comb.tex}{10cm}
\input{temporalia/ps8.tex} \Abardot{}
%\antiphona{}{temporalia/matant6.gtex} % repeat the antiphon - new page

\vfill
\pagebreak

\pars{Versus.} \scriptura{Sap. 10, 10}

\sineinitiali{temporalia/versus-justum.gtex}

%\noindent \trMatVersusI

\vfill

\sineinitiali{temporalia/oratiodominica-mat.gtex}

\vfill

\pars{Absolutio.}

\cuminitiali{}{temporalia/absolutio-exaudi.gtex}

%\trMatAbsolutioI

\vfill
\pagebreak

\cuminitiali{}{temporalia/benedictio-solemn-benedictione.gtex}

%\trMatBenedictioI

%\vfill

%\cuminitiali{}{temporalia/tonus-lectionis-solemnis.gtex}

\vfill

\iffalse
\pars{Lectio I.} \scriptura{Vita S. Galli I, 1}

\noindent Incípit Vitæ Sancti Galli.

\noindent Cum præclára sanctíssimi viri Columbáni, conversátio per omnem Hibérniam célebris haberétur, paréntes beáti Galli, secúndum sǽculum nóbiles, secúndum Deum religiósi, fílium suum primæ ætátis flore niténtem, cum oblatióne Dómino offeréntes, illíus magistério commendavérunt, ut in reguláris vitæ profíceret disciplína.
\else
% https://www.e-codices.unifr.ch/en/csg/0387/463/0/Sequence-516
\pars{Lectio I.} \scriptura{Vita S. Galli XXIX, 1-2}

\noindent Cum iam bonórum ómnium Auctor et Propagátor, athlétam suum de mundi agóne sublátum, præmiórum láureis vellet perénnibus adornáre, Willimárus présbyter véniens ad cellam viri sancti, rogávit eum ut secum egrederétur ad castrum; et ut obtinéret quod vóluit, huiúsmodi voce flébili quærimóniam submíssus explícuit: Cur, ínquiens, o Pater, me, qui tuórum mónitis doctórum innítor, quasi despéctum dereliquísti, et doctrínæ tuæ salutáribus institútis auditórem fraudásti benévolum? Cui hanc obiectiónem ascríbere possim, nisi peccatórum meórum fetóribus? Nisi enim vita mea tuo displicéret iudício, amábili me ædificatiónis tuæ non priváres solátio. Nunc ergo noli nos pro peccátis nostris abícere, sed Dómini provocátus mandátis, viam veritátis desiderántibus áperi, et sólitæ nobis benignitátis munus impénde.
\fi

\tuAutem

\vfill
\pagebreak

\pars{Responsorium 1.} \scriptura{Vita S. Galli I, 1; \textbf{H321}}

\vspace{-5mm}

\responsorium{II}{temporalia/matresp1.gtex}{}

\vfill
\pagebreak

\cuminitiali{}{temporalia/benedictio-solemn-unigenitus.gtex}

%\trMatBenedictioII

\vfill

\iffalse
\pars{Lectio II.} \scriptura{Vita S. Galli I, 1}

\noindent Cumque bonæ índolis vir charo nutrirétur afféctu, magno virtútum crevit augménto. Supérna quoque grátia se præveniénte, tanto stúdio divínas epotávit Scriptúras, ut de thesáuro suo nova proférre posset et vétera.
\else
\pars{Lectio II.} \scriptura{Vita S. Galli XXIX, 3-4}

\noindent Motus ígitur hoc supplicántis allóquio pietátis amátor, descéndit cum illo, et venérunt ad castrum. Vocáta autem multitúdine in die solémni, vir sanctus prædicatiónis dulcédine avidórum corda refécit, et tanta quæ díxerat sapiéntiæ luce vestívit, ut summa ómnium gratulatióne audítus, et plena cunctórum veneratióne sit honorátus. Bíduo ítaque íbidem ducto, tértia die febre corréptus, tantum in brevi eius violéntia depréssus est, ut nec ad cellam redíre nec cibi sustentáculum potuísset percípere. Cumque hac infirmitáte per dies quatuórdecim laborásset, die sexto décimo mensis octóbris, id est XVII (ante diem séptimum décimum) Kaléndas novémbris, explétis nonagínta quinque annis suæ ætátis, in senectúte bona, huius vitæ liberátus ergástulo, ánimam méritis plenam felícibus réddidit bonis inhæsúram perénnibus.
\fi

\tuAutem

\vfill
\pagebreak

\pars{Responsorium 2.} \scriptura{Vita S. Galli IV, 3; \textbf{H322}}

\vspace{-5mm}

\responsorium{VII}{temporalia/matresp2.gtex}{}

\vfill
\pagebreak

\cuminitiali{}{temporalia/benedictio-solemn-spiritus.gtex}

%\trMatBenedictioIII

\vfill

\iffalse
\pars{Lectio III.} \scriptura{Vita S. Galli I, 2}

\noindent Grammáticæ étiam régulas, metrorúmque subtilitátes capáci consequerétur ingénio. Obscúra autem Scripturárum tam sapiénter scire voléntibus reserávit, ut cuncti qui eius prudéntiam et sermónes audíerant, admiratióne eum et laude digníssimum iudicárent.
\else
\pars{Lectio III.} \scriptura{Vita S. Galli XXX, 1-2}

\noindent Cum ígitur audísset Ioánnes, Constantiénsis Ecclésiæ præsul, beátum Gallum apud Arbonam infirmári, ascéndens navículam, pótuum et cibórum ea secum génera tulit quæ in infirmitáte laboránti nóverat congrúere, ut vidélicet sua visitatióne, fidíssimum refovéret amícum. Cumque pórtui appropinquáret, audívit in domo presbýteri planctum magnum circa funus viri Dei celebrári, et interrogávit quæ tanti esset causa plorátus. Audiens autem Gallum venerábilem, firmíssimum suæ familiaritátis custódem, de huius sǽculi emigrásse perículis, misit se in aquam. Neque enim póterat propter nímium dolórem sustinére donec navícula litus attíngeret.
\fi

\tuAutem

\vfill
\pagebreak

\pars{Responsorium 3.} \scriptura{Vita S. Galli VI, 3; \Vbar{} ibid. VI, 4; \textbf{H322}}

\vspace{-5mm}

{
\grechangedim{interwordspacetext}{0.18 cm plus 0.15 cm minus 0.05 cm}{scalable}%
\responsorium{VIII}{temporalia/matresp3-cumdox.gtex}{}
\grechangedim{interwordspacetext}{0.22 cm plus 0.15 cm minus 0.05 cm}{scalable}%
}

\vfill
\pagebreak

\iffalse
\cuminitiali{}{temporalia/benedictio-solemn-inunitate.gtex}

%\trMatBenedictioIV

\vfill

\pars{Lectio IV.} \scriptura{Vita S. Galli I, 2}

\noindent Qua sapiéntiæ maturitáte factum est ut universórum commúni consílio, et iussióne Columbáni abbátis, per síngulos sacræ promotiónis gradus ascéndens, invítus sacerdótii suscíperet dignitátem.

\tuAutem

\vfill
\pagebreak

\pars{Responsorium 4.} \scriptura{Vita S. Galli XIV, 2; \Vbar{} ibid. X, 5; \textbf{H322}}

\vspace{-5mm}

\responsorium{VIII}{temporalia/matresp4.gtex}{}

%\vfill
\pagebreak

\subhora{In II. Nocturno}

\pars{Psalmus 7.} \scriptura{Vita S. Galli XIV, 3; \textbf{H322}}

\vspace{-5mm}

\antiphona{VI F}{temporalia/matant7.gtex}

%\trMatAntVII

\scriptura{Psalmus 10.}

\initiumpsalmi{temporalia/ps10-initium-vi-F-auto.gtex}

%\psalmusEtTranslatioT{temporalia/ps10-comb.tex}{10cm}
\input{temporalia/ps10.tex} \Abardot{}
%\antiphona{}{temporalia/matant7.gtex} % repeat the antiphon - new page

\vfill
\pagebreak

\pars{Psalmus 8.} \scriptura{Vita S. Galli XVIII, 4; \textbf{H322}}

\vspace{-5mm}

\antiphona{III b}{temporalia/matant8.gtex}

%\trMatAntVIII

\scriptura{Psalmus 14.}

\initiumpsalmi{temporalia/ps14-initium-iii-b-auto.gtex}

%\psalmusEtTranslatioT{temporalia/ps14-comb.tex}{10cm}
\input{temporalia/ps14.tex} \Abardot{}
%\antiphona{}{temporalia/matant8.gtex} % repeat the antiphon - new page

\vfill
\pagebreak

\pars{Psalmus 9.} \scriptura{Vita S. Galli XXIV, 2; \textbf{H322}}

\vspace{-5mm}

\antiphona{VIII G\textsuperscript{2}}{temporalia/matant9.gtex}

%\trMatAntIX

\scriptura{Psalmus 20.}

\initiumpsalmi{temporalia/ps20-initium-viii-G2-auto.gtex}

%\psalmusEtTranslatioT{temporalia/ps20-comb.tex}{10cm}
\input{temporalia/ps20.tex}
\antiphona{}{temporalia/matant9.gtex} % repeat the antiphon - new page

\vfill
\pagebreak

\pars{Psalmus 10.} \scriptura{Vita S. Galli XXIV, 3; \textbf{H323}}

\vspace{-5mm}

\antiphona{VIII G}{temporalia/matant10.gtex}

%\trMatAntX

\scriptura{Psalmus 23.}

\initiumpsalmi{temporalia/ps23-initium-viii-G-auto.gtex}

%\psalmusEtTranslatioT{temporalia/ps23-comb.tex}{10cm}
\input{temporalia/ps23.tex} \Abardot{}
%\antiphona{}{temporalia/matant10.gtex} % repeat the antiphon - new page

\vfill
\pagebreak

\pars{Psalmus 11.} \scriptura{Vita S. Galli XXV, 5; \textbf{H323}}

\vspace{-5mm}

\antiphona{I g}{temporalia/matant11.gtex}

%\trMatAntXI

\scriptura{Psalmus 64.}

\initiumpsalmi{temporalia/ps64-initium-i-g-auto.gtex}

%\psalmusEtTranslatioT{temporalia/ps64-comb.tex}{10cm}
\input{temporalia/ps64.tex}
\antiphona{}{temporalia/matant11.gtex} % repeat the antiphon - new page

\vfill
\pagebreak

\pars{Psalmus 12.} \scriptura{\textbf{H323}}

\vspace{-5mm}

\antiphona{III a}{temporalia/matant12.gtex}

%\trMatAntXII

\scriptura{Psalmus 91.}

\initiumpsalmi{temporalia/ps91-initium-iii-a-auto.gtex}

%\psalmusEtTranslatioT{temporalia/ps91-comb.tex}{10cm}
\input{temporalia/ps91.tex}
\antiphona{}{temporalia/matant12.gtex} % repeat the antiphon - new page

\vfill
\pagebreak

\pars{Versus.}

\sineinitiali{temporalia/versus-amavit.gtex}

%\noindent \trMatVersusII

\vfill

\sineinitiali{temporalia/oratiodominica-mat.gtex}

\vfill

\pars{Absolutio.}

\cuminitiali{}{temporalia/absolutio-ipsius.gtex}

%\trMatAbsolutioII

\vfill
\pagebreak

\cuminitiali{}{temporalia/benedictio-solemn-deus.gtex}

%\trMatBenedictioV

\vfill

\pars{Lectio V.} \scriptura{Vita S. Galli I, 3; ibid. II, 1}

\noindent Ergo dum sacris instáret offíciis, die noctúque précibus Dóminum placávit et lácrymis, et supérni inspectóris óculis placére desíderans, pro virtútum et vitæ méritis amabátur ab ómnibus, plácuit univérsis. Dum hæc ageréntur, quotídie beátus Columbánus, evangélicam cúpiens ássequi perfectiónem, ut vidélicet ómnibus quæ habébat relíctis, crucem suam tólleret, et nudus Dóminum sequerétur.

\tuAutem

\vfill
%\pagebreak

\pars{Responsorium 5.} \scriptura{Vita S. Galli XI, 4; \textbf{H323}}

\vspace{-5mm}

\responsorium{IV}{temporalia/matresp5.gtex}{}

%\vfill
\pagebreak

\cuminitiali{}{temporalia/benedictio-solemn-christus.gtex}

%\trMatBenedictioVI

\vfill

\pars{Lectio VI.} \scriptura{Cf. Vita S. Galli II, 2; ibid. IV, 2}

\noindent Ascendéntes ígitur navim, venérunt Británniam, et inde ad Gállias transfretárunt, ubi habitácula construéntes, et hómines Idololátras íbidem commanéntes docébant adoráre Patrem et Fílium et Spíritum Sanctum.

\tuAutem

\vfill
%\pagebreak

\pars{Responsorium 6.} \scriptura{Vita S. Galli XI, 5; \textbf{H323}}

\vspace{-5mm}

\responsorium{II}{temporalia/matresp6.gtex}{}

%\vfill
\pagebreak

\cuminitiali{}{temporalia/benedictio-solemn-ignem.gtex}

%\trMatBenedictioVII

\vfill

\pars{Lectio VII.} \scriptura{Vita S. Galli IV, 3}

\noindent Beátus vero Gallus, zelo pietátis armátus, fana in quibus dæmóniis sacrificábant, igni succéndit, et quæcúmque invénit obláta, demérsit in lacum. Qua causa permóti ira, et ínito consílio, Gallum perímere voluérunt: Columbánum vero flagéllis cæsum et contuméliis afféctum de suis fínibus proturbáre cœpérunt.

\tuAutem

\vfill
%\pagebreak

\pars{Responsorium 7.} \scriptura{Vita S. Galli XII, 3; \Vbar{} ibid. XI, 5; \textbf{H323}}

\vspace{-5mm}

\responsorium{VIII}{temporalia/matresp7.gtex}{}

%\vfill
\pagebreak

\cuminitiali{}{temporalia/benedictio-solemn-acunctis.gtex}

%\trMatBenedictioVIII

\vfill

\pars{Lectio VIII.} \scriptura{Cf. Vita S. Galli VIII, 4; ibid. IX, 1}

\noindent His ígitur provocáti iniúriis, Itáliam pétere decrevérunt. Igitur cum proficiscéndi tempus instáret, beátum Gallum repentína febris invásit. Unde abbátis sui pédibus advolútus, indicávit se infirmitáte veheménti laboráre, et ídeo iter propósitum non posse perfícere.

\tuAutem

\vfill
%\pagebreak

\pars{Responsorium 8.} \scriptura{Vita S. Galli XXVI, 1.2; \Vbar{} ibid. XXVI, 2; \textbf{H324}}

\vspace{-5mm}

\responsorium{II}{temporalia/matresp8.gtex}{}

%\vfill
\pagebreak

\subhora{In III. Nocturno}

\pars{Ad Cantica.} \scriptura{Vita S. Galli XXIX, 4; \textbf{H324}}

\vspace{-5mm}

\antiphona{I g}{temporalia/matant13.gtex}

%\trMatAntIX

\scriptura{Canticum Beatitudo Sapientis; Sir. 14, 22; ibid. 15, 3.4.6}

\initiumpsalmi{temporalia/beatitudosapientis-initium-i-g-auto.gtex}

%\psalmusEtTranslatioT{temporalia/beatitudosapientis-comb.tex}{10cm}
\input{temporalia/beatitudosapientis.tex}
\vfill

\scriptura{Canticum Ieremiæ; Ier. 17, 7-8}

%\psalmusEtTranslatioT{temporalia/jeremiae-comb.tex}{10cm}
\input{temporalia/jeremiae.tex}
\vfill

\scriptura{Canticum Ecclesiasticæ; Sir. 31, 8-11}

%\psalmusEtTranslatioT{temporalia/ecclesiasticus31-comb.tex}{10cm}
\input{temporalia/ecclesiasticus31.tex}
\antiphona{}{temporalia/matant13.gtex} % repeat the antiphon - new page

\vfill
\pagebreak

\pars{Versus.} \scriptura{Ps. 20, 6}

\sineinitiali{temporalia/versus-magna.gtex}

%\noindent \trMatVersusIII

\vfill

\sineinitiali{temporalia/oratiodominica-mat.gtex}

\vfill

\pars{Absolutio.}

\cuminitiali{}{temporalia/absolutio-avinculis.gtex}

%\trMatAbsolutioIII

\vfill
\pagebreak

\cuminitiali{}{temporalia/benedictio-solemn-evangelica.gtex}

%\trMatBenedictioIX

\vfill

% Léctio sancti Evangélii secúndum Lucam.
\pars{Lectio IX.} \scriptura{Lc. 12, 35-40}

\noindent Léctio sancti Evangélii secúndum Lucam.

\noindent In illo témpore: Dixit Iesus discípulis suis: Sint lumbi vestri præcíncti, et lucérnæ ardéntes in mánibus vestris, et vos símiles homínibus exspectántibus dóminum suum, quando revertátur a núptiis. Et réliqua.

% Homilía sancti Gregórii Papæ.
\scriptura{Homilia 13. in Evang.}

\noindent Homilía sancti Gregórii Papæ.

\noindent Sancti Evangélii, fratres charíssimi, apérta vobis est léctio recitáta. Sed ne alíquibus ipsa eius planíties alta fortásse videátur, eam sub brevitáte trans\-cúr\-ri\-mus, quátenus eius exposítio ita nesciéntibus fiat cógnita, ut tamen sciéntibus non sit onerósa. Quia viris luxúria in lumbis sit, féminis in umbilíco, testátur Dóminus, qui de diábolo ad beátum Iob lóquitur, dicens: Virtus eius in lumbis eius, et fortitúdo illíus in umbilíco ventris eius. A principáli ígitur sexu lumbórum nómine luxúria designátur, cum Dóminus dicit: Sint lumbi vestri præcíncti. Lumbos enim præcíngimus cum carnis luxúriam per continéntiam coarctámus. Sed quia minus est mala non ágere, nisi étiam quisque stúdeat et bonis opéribus insudáre, prótinus ádditur: Et lucérnæ ardéntes in mánibus vestris. Lucérnas quippe ardéntes in mánibus tenémus cum per bona ópera próximis nostris lucis exémpla monstrámus. De quibus profécto opéribus Dóminus dicit: Lúceat lux vestra coram homínibus, ut vídeant ópera vestra bona, et gloríficent Patrem vestrum qui in cœlis est. Duo autem sunt quæ iubéntur, et lumbos restríngere, et lucérnas ténere, ut et mundítia sit castitátis in córpore, et lumen veritátis in operatióne. Redemptóri étenim nostro unum sine áltero placére nequáquam potest, si aut is qui bona agit adhuc luxúriæ inquinaménta non déserit, aut is qui castitáte præéminet necdum se per bona ópera exércet. Nec cástitas ergo magna est sine bono ópere, nec opus bonum est áliquod sine castitáte. Sed et si utrúmque ágitur, restat ut quisquis ille est spe ad supérnam pátriam tendat, et nequáquam se a vítiis pro mundi huius ho\-nes\-tá\-te contíneat. Qui etsi quædam bona aliquándo pro honestáte ínchoat, in eius tamen intentióne non debet permanére, nec per bona ópera præséntis mundi glóriam quærére, sed totam spem in Redemptóris sui advéntum constítuat. Unde et prótinus súbditur: Et vos símiles homínibus exspectántibus dóminum suum, quando revertátur a núptiis. Ad núptias quippe Dóminus ábiit, quia resúrgens a mórtuis, ascéndens in cœlum, supérnam sibi angelórum multitúdinem novus homo copulávit. Qui tunc revértitur, cum nobis iam per iudícium manifestátur.

\tuAutem

\vfill
\pagebreak

\pars{Responsorium 9.} \scriptura{Vita S. Galli XXIX, 4; \Vbar{} ibid. XXIX, 4; \Vbar{} ibid. XXIX, 1; \textbf{H324}}

\vspace{-5mm}

\responsorium{I}{temporalia/matresp9.gtex}{}

\vfill
\pagebreak

\cuminitiali{}{temporalia/benedictio-solemn-ille.gtex}

%\trMatBenedictioX

\vfill

\pars{Lectio X.} \scriptura{Homilia 13. in Evang.}

\noindent Bene autem de servis exspectántibus súbditur: Ut cum vénerit et pul\-sá\-ve\-rit, conféstim apériant ei. Venit quippe Dóminus cum ad iudícium pró\-pe\-rat, pulsat vero, cum iam per ægritúdinis moléstias esse mortem vicínam desígnat. Cui conféstim aperímus, si hunc cum amóre suscípimus. Aperíre enim iúdici pulsánti non vult, qui exíre de córpore trépidat, et vidére eum quem contempsísse se méminit iúdicem formídat. Qui autem de sua spe et operatióne secúrus est, pulsánti conféstim áperit, quia lætus iúdicem sústinet; et cum tempus propínquæ mortis agnóverit, de glória retributiónis hilaréscit. Unde et prótinus súbditur: Beáti sunt servi illi, quos cum vénerit dóminus, invénerit vigilántes. Vígilat qui ad aspéctum veri lúminis mentis óculos apértos tenet, vígilat qui servat operándo quod credit, vígilat qui a se torpóris et negligéntiæ ténebras repéllit. Hinc étenim Paulus dicit: Evigiláte, iusti, et nolíte peccáre. Hinc rursus ait; Hora est iam nos de somno súrgere. Sed véniens Dóminus quid servis vigilántibus exhíbeat audiámus: Amen dico vobis quod præcínget se, et fáciet eos discúmbere, et tránsiens ministrábit illis. Præcínget se, id est ad retributiónem præparábit; et fáciet illos discúmbere, id est in ætérna quiéte refovéri. Discúmbere quippe nostrum in regno quiéscere est. Unde rursum Dóminus dicit: Vénient et recúmbent cum Abraham, Isaac et Iacob. Tránsiens autem Dóminus minístrat, quia lucis suæ illustratióne nos sátiat. Transíre vero dictum est, cum de iudício ad regnum redit. Vel certe Dóminus nobis post iudícium transit, quia ab humanitátis forma in divinitátis suæ contemplatiónem nos élevat. Et transíre eius est in claritátis suæ speculatiónem nos dúcere, cum eum quem in humanitáte in iudício cérnimus, étiam in divinitáte post iudícium vidémus. Ad iudícium quippe véniens, in forma servi ómnibus appáret, quia scriptum est: Vidébunt in quem transfixérunt. Sed cum repróbi in supplícium córruunt, iusti ad claritátis eius glóriam pertrahúntur, sicut scriptum est: Tollátur ímpius, ne vídeat glóriam Dei.

\tuAutem

\vfill
%\pagebreak

\pars{Responsorium 10.} \scriptura{Vita S. Galli XXX, 6; \textbf{H324}}

\vspace{-5mm}

\responsorium{VII}{temporalia/matresp10.gtex}{}

%\vfill
\pagebreak

\cuminitiali{}{temporalia/benedictio-solemn-cujus-ipse.gtex}

%\trMatBenedictioXI

\vfill

\pars{Lectio XI.} \scriptura{Homilia 13. in Evang.}

\noindent Sed quid si servi in prima vigília negligéntes exístunt? Prima quippe vigília primæ ætátis custódia est. Sed neque sic desperándum est, et a bono ópere cessándum. Nam longanimitátis suæ patiéntiam insínuans Dóminus, subdit: Et si vénerit in secúnda vigília, et si in tértia vigília vénerit, et ita in\-vé\-ne\-rit, beáti sunt servi illi. Prima quippe vigília primǽvum tempus est, id est puerítia. Secúnda, adolescéntia vel iuvéntus, quæ auctoritáte sacri elóquii unum sunt, dicénte Salomóne: Lætáre iúvenis in adolescéntia tua. Tértia autem, senéctus accípitur. Qui ergo vigiláre in prima vigília nóluit custódiat vel secúndam, ut qui convérti a pravitátibus suis in puerítia negléxit ad vias vitæ saltem in témpore iuventútis evígilet. Et qui evigiláre in secúnda vigília nóluit tértiæ vigíliæ remédia non amíttat, ut qui in iuventúte ad vias vitæ non evígilat saltem in senectúte resipíscat. Pensáte, fratres charíssimi, quia conclúsit Dei píetas durítiam nostram. Non est iam quid homo excusatiónis invéniat. Deus despícitur, et exspéctat; contémni se videt, et révocat; iniúriam de contémptu suo súscipit, et tamen quandóque reverténtibus étiam prǽmia promíttit. Sed nemo hanc eius longanimitátem négligat, quia tanto districtiórem iustítiam in iudício éxiget, quanto longiórem patiéntiam ante iudícium prærogávit. Hinc étenim Paulus dicit: Ignóras quóniam benígnitas Dei ad pœniténtiam te addúcit? Tu autem secúndum durítiam tuam et cor impœ́nitens thesáurizas tibi iram in die iræ et revelatiónis iusti iudícii Dei. Hinc Psalmísta ait: Deus iudex iustus, fortis, et longánimis. Dictúrus quippe longánimem, præmísit iustum, ut quem vides peccáta delinquéntium diu patiénter ferre, scias hunc étiam quandóque distrícte iudicáre. Hinc per quemdam sapiéntem dícitur: Altíssimus enim est pátiens rédditor. Pátiens enim rédditor dícitur, quia peccáta hóminum et pátitur et reddit. Nam quos diu, ut convertántur, tólerat, non convérsos dúrius damnat. Ad excutiéndam vero mentis nostræ desídiam, étiam exterióra damna per similitúdinem ad médium deducúntur, ut per hæc ánimus ad sui custódiam suscitétur. Nam dícitur: Hoc autem scitóte, quia si sciret paterfamílias qua hora fur veníret, vigiláret útique, et non síneret pérfodi domum suam. Ex qua præmíssa similitúdine étiam exhortátio subinfértur, cum dícitur: Et vos estóte paráti, quia qua hora non putátis Fílius hóminis véniet. Nesciénte enim pat\-re\-fa\-mí\-li\-as fur domum pérfodit, quia dum a sui custódia spíritus dormit, improvísa mors véniens carnis nostræ habitáculum irrúmpit, et eum quem dóminum domus invénerit dormiéntem necat, quia cum ventúra damna spíritus mínime prǽvidet, hunc mors ad supplícium nesciéntem rapit. Furi autem resísteret, si vigiláret, quia advéntum iúdicis, qui occúlte ánimam rapit, prǽcavens, ei pœniténdo occúrreret, ne impœ́nitens períret.

\tuAutem

\vfill
%\pagebreak

\pars{Responsorium 11.} \scriptura{Vita S. Galli XXXI, 1.2; \textbf{H324}}

\vspace{-5mm}

\responsorium{IV}{temporalia/matresp11.gtex}{}

%\vfill
\pagebreak

\cuminitiali{}{temporalia/benedictio-solemn-adsocietatem.gtex}

%\trMatBenedictioXII

\vfill

\pars{Lectio XII.} \scriptura{Homilia 13. in Evang.}

\noindent Horam vero últimam Dóminus noster idcírco vóluit nobis esse incógnitam, ut semper possit esse suspécta, ut dum illam prævidére non póssumus, ad illam sine intermissióne præparémur. Proínde, fratres mei, in conditióne mortalitátis vestræ mentis óculos fígite, veniénti vos iúdici per fletus quotídie et laménta præparáte. Et cum certa mors máneat ómnibus, nolíte de temporális vitæ providéntia incérta cogitáre. Terrenárum rerum vos cura non ággravet. Quantislíbet enim auri et argénti mólibus circumdétur, quibuslíbet pretiósis véstibus induátur caro, quid est áliud quam caro? Nolíte ergo atténdere quid habétis, sed quid estis. Vultis audíre quid estis? Prophéta índicat, dicens: Vere fenum est pópulus. Si enim fenum pópulus non est, ubi sunt illi qui ea quæ hódie cólimus nobíscum transácto anno beáti Galli natalítia ce\-leb\-ravérunt? O quanta et quália de præséntis vitæ provisióne cogitábant, sed, subripiénte mortis artículo, repénte in his quæ prævidére nolébant invénti sunt, et cuncta simul temporália quæ congregáta quasi stabíliter ténere videbántur amisérunt. Si ergo transácta multitúdo géneris humáni per nativitátem víruit in carne, per mortem áruit in púlvere, vidélicet fenum fuit. Quia ígitur moméntis suis horæ fúgiunt, ágite, fratres charíssimi, ut in boni óperis mercéde teneántur. Audíte quid sápiens Sálomon dicat: Quodcúmque potest manus tua fácere, instánter operáre, quia nec opus, nec sciéntia, nec rátio, nec sapiéntia erunt apud ínferos, quo tu próperas. Quia ergo et ventúræ mortis tempus ignorámus, et post mortem operári non póssumus, súperest ut ante mortem témpora indúlta rapiámus. Sic enim sic mors ipsa cum vénerit vincétur, si priúsquam véniat semper timeátur.

\tuAutem

\vfill
\pagebreak

\pars{Responsorium 12.} \scriptura{\textbf{H325}}

\vspace{-5mm}

\responsorium{I}{temporalia/matresp12.gtex}{}

\vfill
\pagebreak
\fi

% Te Deum

{
\pars{Hymnus Ambrosianus} \scriptura{Tonus Solemnis}

\vspace{-2mm}

\grechangedim{interwordspacetext}{0.26 cm plus 0.15 cm minus 0.05 cm}{scalable}%
\cuminitiali{III}{temporalia/tedeum-solemnis-gn.gtex}
\grechangedim{interwordspacetext}{0.22 cm plus 0.15 cm minus 0.05 cm}{scalable}%
}

\vfill
\pagebreak

\iffalse
% Evangelium

\cuminitiali{}{temporalia/tonus-evangelii-b.gtex}

\vfill

\scriptura{Lc. 12, 35-40}

\noindent Léctio sancti Evangélii secúndum Lucam.

\noindent In illo témpore: Dixit Jesus discípulis suis: Sint lumbi vestri præcíncti, et lucérnæ ardéntes in mánibus vestris, et vos símiles homínibus exspectántibus dóminum suum, quando revertátur a núptiis: ut, cum vénerit, et pulsáverit, conféstim apériant ei. Beáti servi illi, quos cum vénerit dóminus, invénerit vigilántes: amen dico vobis, quod præcínget se, et fáciet illos discúmbere, et tránsiens ministrábit illis. Et si vénerit in secúnda vigília, et si in tértia vigília vénerit, et ita invénerit, beáti sunt, beáti sunt servi illi. Hoc autem scitóte, quóniam si sciret paterfamílias, quia hora fur vénerit, vigiláret útique, et non síneret pérfodi domum suam. Et vos estúte paráti, quis qua hora non putátis. Fílius hóminis véniet.

\vfill
\cuminitiali{I}{temporalia/tedecetlaus.gtex}

%\trTeDecetLaus
\fi

\vfill
\pagebreak

\sineinitiali{temporalia/domineexaudi.gtex}

\vfill

\pars{Oratio.}

\cuminitiali{}{temporalia/oratio.gtex}
%\trOrationis

\vfill

\noindent \Vbardot{} Dómine, exáudi oratiónem meam.
\Rbardot{} Et clamor meus ad te véniat.

\vfill

% Nocturnale Romanum 2002, p. LXXVI Benedicamus Domino seems to match
% the one from Solemn Laudes.
\cuminitiali{V}{temporalia/benedicamus-solemnis-laud.gtex}

\vfill

\noindent \Vbardot{} Fidélium ánimæ per misericórdiam Dei requiéscant in pace.
\Rbardot{} Amen.

%\trFideliumAnimae

\vfill
\pagebreak

\hora{Ad Laudes.} %%%%%%%%%%%%%%%%%%%%%%%%%%%%%%%%%%%%%%%%%%%%%%%%%%%%%%%%%%
%\sideThumbs{Laudes}

% Psalmi festivi (AM33, pg. 721):
% 66 // 92, 99, 62, Dan3, 148+149+150

%\vspace{1cm}
\cuminitiali{}{temporalia/deusinadiutorium-communis.gtex}
\vspace{1mm}

\cantusSineNeumas

\pars{Psalmus 1.} \scriptura{Vita S. Galli XXXII, 1.2; \textbf{H325}}

\vspace{-5mm}

\antiphona{VI F}{temporalia/ant1.gtex}

%\trAntI

\scriptura{Ps. 50}

\initiumpsalmi{temporalia/ps50-initium-vi-F-auto.gtex}

%\psalmusEtTranslatioT{temporalia/ps50-comb.tex}{10cm}
\input{temporalia/ps50.tex}
\antiphona{}{temporalia/ant1.gtex} % repeat the antiphon - new page

\vfill
\pagebreak

\pars{Psalmus 2.} \scriptura{Vita S. Galli XXXII, 1; \textbf{H325}}

\vspace{-5mm}

\antiphona{VII a}{temporalia/ant2.gtex}

%\trAntII

\scriptura{Ps. 117}

\initiumpsalmi{temporalia/ps117-initium-vii-a-auto.gtex}

%\psalmusEtTranslatioT{temporalia/ps117-comb.tex}{10cm}
\input{temporalia/ps117.tex}

\vfill
\pagebreak

\antiphona{}{temporalia/ant2.gtex} % repeat the antiphon - new page

\vfill
\pagebreak

\pars{Psalmus 3.} \scriptura{Vita S. Galli XXXII, 2; \textbf{H326}}

\vspace{-5mm}

\antiphona{I g}{temporalia/ant3.gtex}

%\trAntIII

\vspace{-3mm}

\scriptura{Ps. 62.}

\initiumpsalmi{temporalia/ps62-initium-i-g-auto.gtex}

%\vspace{-7mm}

%\psalmusEtTranslatioT{temporalia/ps62-comb.tex}{10cm}
\input{temporalia/ps62.tex} \Abardot{}
%\antiphona{}{temporalia/ant3.gtex} % repeat the antiphon - new page

\vfill
\pagebreak

\pars{Psalmus 4.} \scriptura{Vita S. Galli XXXII, 2; \textbf{H326}}

\vspace{-5mm}

\antiphona{III a}{temporalia/ant4.gtex}

%\trAntIV

\scriptura{Canticum trium puerorum, Dan. 3, 57-88 et 56}

\initiumpsalmi{temporalia/dan3-initium-iii-a-auto.gtex}

%\psalmusEtTranslatioT{temporalia/dan3-comb.tex}{10cm}
\input{temporalia/dan3.tex}

\rubrica{Hic non dicitur Gloria Patri, neque Amen.}
\vspace{1cm}

\antiphona{}{temporalia/ant4.gtex} % repeat the antiphon - new page

\vfill
\pagebreak

\pars{Psalmus 5.} \scriptura{Vita S. Galli XXXII, 2; \textbf{H326}}

\vspace{-5mm}

\antiphona{I g}{temporalia/ant5.gtex}

%\trAntV

\scriptura{Ps. 148}

\initiumpsalmi{temporalia/ps148-initium-i-g-auto.gtex}

\newlength{\psVItransW}
\setlength{\psVItransW}{10.5cm}

%\psalmusEtTranslatioT{temporalia/ps148-comb.tex}{10cm}
\input{temporalia/ps148.tex}
%\vspace{-5mm}

\rubrica{Hic non dicitur Gloria Patri.}

\vfill
\pagebreak

%
\scriptura{Ps. 149}

\initiumpsalmi{temporalia/ps149-initium-i-g-auto.gtex}

%\vspace{-3mm}

%\psalmusEtTranslatioT{temporalia/ps149-comb.tex}{10cm}
\input{temporalia/ps149alt.tex}
%\vspace{-8mm}

\rubrica{Hic non dicitur Gloria Patri.}

\vfill
\pagebreak

%
\scriptura{Ps. 150}

\initiumpsalmi{temporalia/ps150-initium-i-g-auto.gtex}

%\psalmusEtTranslatioT{temporalia/ps150-comb.tex}{10cm}
\input{temporalia/ps150alt.tex}
\antiphona{}{temporalia/ant5.gtex} % repeat the antiphon - new page

\vfill
\pagebreak

\cantusSineNeumas

\pars{Capitulum.} \scriptura{Sir. 45, 1-2}

\cuminitiali{}{temporalia/capitulum-DilectusDeo.gtex}

% preklad Jeruz. bible
%\trCapituli

\vfill

\pars{Responsorium breve.} \scriptura{Sap. 10, 10}

\antiphona{VI}{temporalia/respl.gtex}

%\trRespLaud

\vfill
\pagebreak

% Hymnus. %%%
\pars{Hymnus.} \scriptura{Walahfrid Strabus (\olddag{} 849)}

{
\grechangedim{interwordspacetext}{0.20 cm plus 0.15 cm minus 0.05 cm}{scalable}%
\cuminitiali{I}{temporalia/hym-VitaSanctorum.gtex}
\grechangedim{interwordspacetext}{0.22 cm plus 0.15 cm minus 0.05 cm}{scalable}%
}

\vfill
%\pagebreak

\pars{Versus.} \scriptura{Ps. 36, 30}

% Versus. %%%
\sineinitiali{temporalia/versus-os.gtex}

%\noindent \trVersus

\vfill
\pagebreak

\cantusSineNeumas

\pars{Canticum Zachariæ.} \scriptura{Cf. Vita S. Galli XXX, 6; ibid. XXXIII, 1; ibid. XVII, 5; \textbf{H326}}

\vspace{-5mm}

\antiphona{VIII G}{temporalia/ant-ben-laud.gtex}

%\trAntBenedictus

\scriptura{Lc. 1, 68-79}

\initiumpsalmi{temporalia/benedictus-initium-viiisoll-G-auto.gtex}

%\psalmusEtTranslatioT{temporalia/benedictus-comb.tex}{10cm}
\input{temporalia/benedictus.tex}
\antiphona{}{temporalia/ant-ben-laud.gtex} % repeat the antiphon - new page

\vfill
\pagebreak

\cantusSineNeumas

\anteOrationem

\pagebreak

% Oratio. %%%
\pars{Oratio.}

\cuminitiali{}{temporalia/oratio.gtex}
%\trOrationis

\vfill

\rubrica{Hebdomadarius dicit iterum Dominus vobiscum, vel cantor dicit:}

\vspace{2mm}

\sineinitiali{temporalia/domineexaudi.gtex}

\rubrica{Postea cantatur a cantore:}

\vspace{2mm}

\cuminitiali{}{temporalia/benedicamus-duplex-laudes.gtex}
\vfill
\pagebreak

%\setcounter{Thumb}{0}
\hora{Ad Vesperas.} %%%%%%%%%%%%%%%%%%%%%%%%%%%%%%%%%%%%%%%%%%%%%%%%%%%%%
%\sideThumbs{II. Vesperæ}

{
\grechangedim{interwordspacetext}{0.18 cm plus 0.15 cm minus 0.05 cm}{scalable}%
\cuminitiali{}{temporalia/deusinadiutorium-communis.gtex}
\grechangedim{interwordspacetext}{0.22 cm plus 0.15 cm minus 0.05 cm}{scalable}%
}

\vfill
%\pagebreak

\cantusSineNeumas

\pars{Psalmus 1.} \scriptura{Vita S. Galli XXXII, 1.2; \textbf{H325}}

\vspace{-5mm}

\antiphona{VI F}{temporalia/ant1.gtex}

%\trAntI

\scriptura{Ps. 109}

\initiumpsalmi{temporalia/ps109-initium-vi-F-auto.gtex}

%\psalmusEtTranslatioT{temporalia/ps109-comb.tex}{10cm}
\input{temporalia/ps109.tex}

\vfill

\antiphona{}{temporalia/ant1.gtex} % repeat the antiphon - new page

\vfill
\pagebreak

\pars{Psalmus 2.} \scriptura{Vita S. Galli XXXII, 1; \textbf{H325}}

\vspace{-5mm}

\antiphona{VII a}{temporalia/ant2.gtex}

%\trAntII

\scriptura{Ps. 110}

\initiumpsalmi{temporalia/ps110-initium-vii-a-auto.gtex}
%\psalmusEtTranslatioT{temporalia/ps110-comb.tex}{10cm}
\input{temporalia/ps110.tex} \Abardot{}
%\antiphona{}{temporalia/ant2.gtex} % repeat the antiphon - new page

\vfill
\pagebreak

\pars{Psalmus 3.} \scriptura{Vita S. Galli XXXII, 2; \textbf{H326}}

\vspace{-5mm}

\antiphona{I g}{temporalia/ant3.gtex}

%\trAntIII

\scriptura{Ps. 111}

\initiumpsalmi{temporalia/ps111-initium-i-g-auto.gtex}
%\psalmusEtTranslatioT{temporalia/ps111-comb.tex}{10cm}
\input{temporalia/ps111.tex} \Abardot{}
%\antiphona{}{temporalia/ant3.gtex} % repeat the antiphon - new page

\vfill
\pagebreak

\pars{Psalmus 4.} \scriptura{Vita S. Galli XXXII, 2; \textbf{H326}}

\vspace{-5mm}

\antiphona{III a}{temporalia/ant4.gtex}

%\trAntIV

\scriptura{Ps. 112}

\initiumpsalmi{temporalia/ps112-initium-iii-a-auto.gtex}
%\psalmusEtTranslatioT{temporalia/ps112-comb.tex}{10cm}
\input{temporalia/ps112.tex} \Abardot{}
%\antiphona{}{temporalia/ant4.gtex} % repeat the antiphon - new page

\vfill
\pagebreak

\raggedcolumns

% Capitulum. %%%
\cantusSineNeumas

\pars{Capitulum.} \scriptura{Sir. 45, 1-2}

\cuminitiali{}{temporalia/capitulum-DilectusDeo.gtex}

% preklad Jeruz. bible
%\trCapituli

\vfill
\pars{Responsorium breve.} \scriptura{Ps. 36, 30}

\antiphona{VI}{temporalia/respv.gtex}

%\trRespVesp

\vfill
\pagebreak

% Hymnus. %%%
\pars{Hymnus.} \scriptura{Walahfrid Strabus (\olddag{} 849)}

{
\grechangedim{interwordspacetext}{0.20 cm plus 0.15 cm minus 0.05 cm}{scalable}%
\cuminitiali{I}{temporalia/hym-VitaSanctorum.gtex}
\grechangedim{interwordspacetext}{0.22 cm plus 0.15 cm minus 0.05 cm}{scalable}%
}

\vfill
%\pagebreak

\pars{Versus.} \scriptura{Ps. 36, 30}

% Versus. %%%
\sineinitiali{temporalia/versus-os.gtex}

%\noindent \trVersus

\vfill
\pagebreak

\cantusSineNeumas

\pars{Canticum B. Mariæ V.} \scriptura{\textbf{H326}}

\vspace{-5mm}

\antiphona{I D}{temporalia/ant-magn-vesp2.gtex}

%\trAntMagnificatII

\scriptura{Lc. 1, 46-55}

\cantusSineNeumas
\initiumpsalmi{temporalia/magnificat-initium-isoll-D.gtex}

%\psalmusEtTranslatioT{temporalia/magnificat-comb.tex}{10.3cm}
\input{temporalia/magnificat.tex}

\antiphona{}{temporalia/ant-magn-vesp2.gtex} % repeat the antiphon - new page

\vfill
\pagebreak

\anteOrationem

\pagebreak

%% Oratio. %%%
\pars{Oratio.}

\cuminitiali{}{temporalia/oratio.gtex}
%\trOrationis

\vfill

\rubrica{Hebdomadarius dicit iterum Dominus vobiscum, vel cantor dicit:}

\vspace{2mm}

\sineinitiali{temporalia/domineexaudi.gtex}

\rubrica{Postea cantatur a cantore:}

\vspace{2mm}

\cuminitiali{II}{temporalia/benedicamus-duplex-vesperae.gtex}

\vfill
\pagebreak

\pars{Sequentia.} \scriptura{beatus Notker Balbulus (\olddag{} 912); \textbf{Sg. 381, pag. 455}}

{
\grechangedim{interwordspacetext}{0.10 cm plus 0.15 cm minus 0.05 cm}{scalable}%
\antiphona{}{temporalia/seq-DilecteDeo.gtex}
\grechangedim{interwordspacetext}{0.22 cm plus 0.15 cm minus 0.05 cm}{scalable}%
}

%\trSequentia

\vfill

\end{document}
