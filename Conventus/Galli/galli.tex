% LuaLaTeX

\documentclass[a4paper, twoside, 12pt]{article}
\usepackage[latin]{babel} 
%\usepackage[landscape, left=3cm, right=1.5cm, top=2cm, bottom=1cm]{geometry} % okraje stranky
\usepackage[landscape, a4paper, mag=1166, truedimen, left=2cm, right=1.5cm, top=1.6cm, bottom=0.95cm]{geometry} % okraje stranky

\usepackage{fontspec}
\setmainfont[FeatureFile={junicode.fea}, Ligatures={Common, TeX}, RawFeature=+fixi]{Junicode}
%\setmainfont{Junicode}

% shortcut for Junicode without ligatures (for the Czech texts)
\newfontfamily\nlfont[FeatureFile={junicode.fea}, Ligatures={Common, TeX}, RawFeature=+fixi]{Junicode}

% Hebrew font:
% http://scripts.sil.org/cms/scripts/page.php?site_id=nrsi&id=SILHebrUnic2
\newfontfamily\hebfont[Scale=1]{Ezra SIL}

\usepackage{multicol}
\usepackage{color}
\usepackage{lettrine}
\usepackage{fancyhdr}

% usual packages loading:
\usepackage{luatextra}
\usepackage{graphicx} % support the \includegraphics command and options
\usepackage{gregoriotex} % for gregorio score inclusion
\usepackage{gregoriosyms}
\usepackage{wrapfig} % figures wrapped by the text
\usepackage{parcolumns}
\usepackage[contents={},opacity=1,scale=1,color=black]{background}
\usepackage{tikzpagenodes}
\usepackage{calc}
\usepackage{longtable}
\usetikzlibrary{calc}

\setlength{\headheight}{14.5pt}

\input{conventuscommune.tex} % Often used macros
%%%% Preklady jednotlivych zpevu (nektere se opakuji, a je dobre mit je
% vsechny na jedne hromade)

% HOURS ---

\newcommand{\trAntI}{\translatioCantus{Muž boží měl kožený toulec, pečlivě
zavázaný, jenž mu visel na šíji a~často se ho dotýkal.}}

\newcommand{\trAntII}{\translatioCantus{Klíč od~něho tak dobře střežil, že
dokud žil v~těle, nikdo z~jeho žáků nezvěděl, co je uvnitř.}}

\newcommand{\trAntIII}{\translatioCantus{Ale když se odebral z~tohoto
života, schránku otevřeli a~objevili v~ní žíněné roucho a~měděný řetěz
potřísněný krví.}}

\newcommand{\trAntIV}{\translatioCantus{A když prohlédli mistrovo tělo,
nalezli jeho tělo na čtyřech místech hluboce zbrázděno ranami od řetězu.}}

\newcommand{\trAntV}{\translatioCantus{Krev vytékající z~těch ran, místy
prostoupila i~žíněným rouchem.}}

\newcommand{\trCapituli}{\translatioCantus{
Miláčkovi Boha a~lidí,
Mojžíšovi požehnané paměti,~\gredagger{}
dopřál slávu rovnou slávě svatých~\grestar{}
učinil ho mocným na postrach nepřátelům
a~jeho slovy zastavil divy.}}

\newcommand{\trLectioBrevis}{\translatioCantus{
Pamatujte na své představené,
kteří vám hlásali Boží slovo.
Uvažte, jak oni skončili život, a~napodobujte jejich víru.
Ježíš Kristus je stejný včera i~dnes i~navěky.
Nenechte se svést věelijakými cizími naukami.}}

\newcommand{\trRespLaud}{\translatioCantus{Spravedlivého vodil Hospodin~\grestar{}
po přímých stezkách. \Vbardot{} A~ukázal mu Boží království.}}

\newcommand{\trRespLaudB}{\translatioCantus{Na tvých hradbách, Jeruzaléme,
ustanovil jsem strážné;~\grestar{}
budou bdít nad mým lidem. \Vbardot{} Ani ve dne, ani v~noci nesmějí nikdy
mlčet.}}

\newcommand{\trVersus}{\translatioCantus{\Vbardot{} Ústa spravedlivého šeptají moudrost, aleluja.
\Rbardot{} A~jeho jazyk ohlašuje právo, aleluja.}}

\newcommand{\trAntBenedictus}{\translatioCantus{Když na bujné oře vložili
nosítka a~sňali jim uzdu, vydali se přímo k~cele božího muže.}}

\newcommand{\trPreces}{\translatioCantus{
\noindent S vděčností chvalme Krista, dobrého Pastýře, \gredagger{} který dal život za své ovce, \grestar{} a~pokorně ho prosme: \Rbardot{} Pane, buď pastýřem svého lidu.

\noindent Kriste, ty dáváš církvi pastýře, a~jejich službou se ujímáš svého lidu, \grestar{} dej, ať v~lásce těch, kteří nás vedou, poznáváme, jak nás miluješ. \Rbardot{} Pane, buď pastýřem svého lidu.

\noindent Ty stále konáš skrze své zástupce službu pastýře a~učitele, \grestar{} nepřestávej nás nikdy vést prostřednictvím svých služebníků. \Rbardot{} Pane, buď pastýřem svého lidu.

\noindent Ty prokazuješ svému lidu skrze jeho pastýře službu lékaře duše i~těla, \grestar{} ochraňuj náš život a~veď nás ke svatosti. \Rbardot{} Pane, buď pastýřem svého lidu.

\noindent Ty posíláš své svaté, aby slovem i~příkladem vedli tvůj lid k~tobě, \grestar{} na jejich přímluvu nás posiluj, abychom vytrvali na cestě, která vede k~věčnému životu. \Rbardot{} Pane, buď pastýřem svého lidu.}}

\newcommand{\trOrationis}{\translatioCantus{Bože, jenž nám dopřáváš radovat
se z~výroční slavnosti svatého tvého vyznavače Havla, uděl dobrotivě,
abychom když slavíme jeho narození, též se řídili podobou jeho skutků.
Skrze…}}
 % Czech translations of the proper texts

\newcommand{\annusEditionis}{2016}

\def\hebinitial#1{%
\leavevmode{\newbox\hebbox\setbox\hebbox\hbox{\hebfont{#1}\hskip 1mm}\kern -\wd\hebbox\hbox{\hebfont{#1}\hskip 1mm}}%
}

%%%% Vicekrat opakovane kousky

\newcommand{\anteOrationem}{
  \rubrica{Ante Orationem, cantatur a Superiore:}

  \pars{Supplicatio Litaniæ.}

  \cuminitiali{}{temporalia/supplicatiolitaniae.gtex}

  \pars{Oratio Dominica.}

  \cuminitiali{}{temporalia/oratiodominica.gtex}

  \rubrica{Deinde dicitur ab Hebdomadario:}

  \cuminitiali{}{temporalia/dominusvobiscum-solemnis.gtex}

  \rubrica{In choro monialium loco Dominus vobiscum dicitur:}

  \sineinitiali{temporalia/domineexaudi.gtex}
}

\newcommand{\tuAutem}{
  \vfill

  \sineinitiali{temporalia/tuautem.gtex}
}

\setlength{\columnsep}{30pt} % prostor mezi sloupci

%%%%%%%%%%%%%%%%%%%%%%%%%%%%%%%%%%%%%%%%%%%%%%%%%%%%%%%%%%%%%%%%%%%%%%%%%%%%%%%%%%%%%%%%%%%%%%%%%%%%%%%%%%%%%
\begin{document}

% Here we set the space around the initial.
% Please report to http://home.gna.org/gregorio/gregoriotex/details for more details and options
\grechangedim{afterinitialshift}{2.2mm}{scalable}
\grechangedim{beforeinitialshift}{2.2mm}{scalable}

\grechangedim{interwordspacetext}{0.32 cm plus 0.15 cm minus 0.05 cm}{scalable}%
\grechangedim{annotationraise}{-0.2cm}{scalable}

% Here we set the initial font. Change 38 if you want a bigger initial.
% Emit the initials in red.
\grechangestyle{initial}{\color{red}\fontsize{38}{38}\selectfont}

\pagestyle{empty}

%%%% Titulni stranka
\begin{titulusOfficii}
\dies{Die 16. Octobris.}
\nomenFesti{In Nativitate S. Galli, Confessoris.}
\celebratio{Duplex 2. classis.}
\end{titulusOfficii}

% graphic
\vspace{1.5cm}
\begin{center}
%\includegraphics[width=10cm]{imagines/imago_nativitas.jpg}
\end{center}

\vfill

\begin{center}
Ad usum et secundum consuetudines chori \guillemotright Conventus Choralis\guillemotleft.

Editio Sancti Wolfgangi \annusEditionis
\end{center}

\pagebreak

\renewcommand{\headrulewidth}{0pt} % no horiz. rule at the header
\fancyhf{}
\pagestyle{fancy}

\pars{Oratio ante divinum Officium.}

\lettrine{{\color{red}A}}{peri,} Dómine, os meum ad benedicéndum nomen sanctum tuum:
munda quoque cor meum ab ómnibus vanis, pervérsis, et aliénis
cogitatiónibus:
intelléctum illúmina, afféctum inflámma,
ut digne, atténte ac devóte hoc Offícium recitáre váleam,
et exaudíri mérear ante conspéctum Divínæ Maiestátis tuæ.
Per Christum, Dominum nostrum.
\Rbardot{} Amen.

Dómine, in unióne illíus divínæ intentiónis,
qua ipse in terris laudes Deo persolvísti,
has tibi Horas \rubricatum{(vel \textnormal{hanc tibi Horam})} persólvo.

\trOratioAnteOfficium

\vfill

\pars{Oratio post divinum Officium.}

\rubrica{
  Orationem sequentem devote post Officium recitantibus
  Leo Papa X. defectus, et culpas in eo persolvendo ex humana
  fragilitate contractas, indulsit, et dicitur flexis genibus.
}

\lettrine{{\color{red}S}}{acrosánctæ} et indivíduæ Trinitáti,
crucifíxi Dómini nostri Iesu Christi humanitáti,
beatíssimæ et gloriosíssimæ sempérque Vírginis Maríæ
fecúndæ integritáti, 
et ómnium Sanctórum universitáti
sit sempitérna laus, honor, virtus et glória
ab omni creatúra,
nobísque remíssio ómnium peccatórum,
per infiníta sǽcula sæculórum.
\Rbardot{} Amen.

\noindent \Vbardot{} Beáta víscera Maríæ Virginis, quæ portavérunt
ætérni Patris Fílium.\\
\Rbardot{} Et beáta úbera, quæ lactavérunt Christum Dominum.

\rubrica{Et dicitur secreto \textnormal{Pater noster.} et \textnormal{Ave María.}}

\trOratioPostOfficium

\vfill

\hora{Ad I. Vesperas.} %%%%%%%%%%%%%%%%%%%%%%%%%%%%%%%%%%%%%%%%%%%%%%%%%%%%%
\sideThumbs{I. Vesperæ}

{
\grechangedim{interwordspacetext}{0.18 cm plus 0.15 cm minus 0.05 cm}{scalable}%
\cuminitiali{}{temporalia/deusinadiutorium-communis.gtex}
\grechangedim{interwordspacetext}{0.32 cm plus 0.15 cm minus 0.05 cm}{scalable}%
}

\vfill
\pagebreak

\cantusCumNeumis

\pars{Psalmus 1.} \scriptura{Vita S. Galli XXXII, 1.2; \textbf{H325}}

\vspace{-0.5cm}

\antiphona{VI F}{temporalia/ant1.gtex}

\trAntI

\scriptura{Ps. 144, 10-21}

\initiumpsalmi{temporalia/ps144ii-initium-vi-F-auto.gtex}

\psalmusEtTranslatioT{temporalia/ps144ii-comb.tex}{10cm}

\antiphona{}{temporalia/ant1.gtex} % repeat the antiphon - new page

\vfill
\pagebreak

\pars{Psalmus 2.} \scriptura{Vita S. Galli XXXII, 1; \textbf{H325}}

\vspace{-0.5cm}

\antiphona{VII a}{temporalia/ant2.gtex}

\trAntII

\scriptura{Ps. 145}

\initiumpsalmi{temporalia/ps145-initium-vii-a-auto.gtex}
\psalmusEtTranslatioT{temporalia/ps145-comb.tex}{9cm}

\vfill
\pagebreak

\pars{Psalmus 3.} \scriptura{Vita S. Galli XXXII, 2; \textbf{H326}}

\vspace{-0.5cm}

\antiphona{I g}{temporalia/ant3.gtex}

\trAntIII

\scriptura{Ps. 146}

\initiumpsalmi{temporalia/ps146-initium-i-g-auto.gtex}
\psalmusEtTranslatioT{temporalia/ps146-comb.tex}{10cm}

\antiphona{}{temporalia/ant3.gtex} % repeat the antiphon - new page

\vfill
\pagebreak

\pars{Psalmus 4.} \scriptura{Vita S. Galli XXXII, 2; \textbf{H326}}

\vspace{-0.5cm}

\antiphona{III a}{temporalia/ant4.gtex}

\trAntIV

\scriptura{Ps. 147}

\initiumpsalmi{temporalia/ps147-initium-iii-a-auto.gtex}
\psalmusEtTranslatioT{temporalia/ps147-comb.tex}{10cm}

\vfill
\pagebreak

\raggedcolumns

% Capitulum. %%%
\cantusSineNeumas

\pars{Capitulum.} \scriptura{Sir. 45, 1-2}

\cuminitiali{}{temporalia/capitulum-DilectusDeo.gtex}

% preklad Jeruz. bible
\trCapituli

\vfill
\pars{Responsorium breve.} \scriptura{Ps. 36, 30}

\antiphona{VI}{temporalia/respv.gtex}

\trRespVesp

\vfill
\pagebreak

% Hymnus. %%%
\pars{Hymnus.} \scriptura{Walahfrid Strabus (\gredagger{} 849)}

{
\grechangedim{interwordspacetext}{0.20 cm plus 0.15 cm minus 0.05 cm}{scalable}%
\cuminitiali{I}{temporalia/hym-VitaSanctorum.gtex}
\grechangedim{interwordspacetext}{0.32 cm plus 0.15 cm minus 0.05 cm}{scalable}%
}
%{
%\vspace{-0.3cm}
%\setlength{\columnsep}{7pt} % prostor mezi sloupci
%\begin{translatioMulticol}{3}
Svatých život je cestou a~záchranou\\
Kriste, jenž dáváš mír a~bezúhonnost\\
Původci svému ti hlasem i~myslí\\
zpíváme hymnus.\\
\\
Sžíráni láskou k~tomu, v~němž moci jest\\
plnost zjevena; se vším, co v~moci své\\
mají jen zbožní a~po čem vší silou\\
a~srdcem touží.\\
\\
Pro jeho zbožnost jsi svatého Havla\\
učinil vzorem nebeské jasnosti;\\
abychom jeho učením unikli\\
temnotám mysli. \\
\\
Jako pták zpěvný první se probudil\\
a~skutky svými v~životě dosvědčil\\
to, co mu moudrost jeho učitele\\
ctnostného vlila.\columnbreak

V~slově byl mocný, v~činech úctyhodný,\\
vždy znovu bažil po věčném bohatství;\\
tak zjevně došel odměnou znamení\\
nebeské ctnosti.\\
\\
Prosíme světa, Původce a~spáso\\
na jeho prosby pohlédni teď svaté,\\
rač popřát lidu dobrotivým srdcem,\\
čeho si žádá.\\
\\
Pokojné časy a~víře stálost,\\
churavým zdraví, padlým slitování;\\
všem potom lidem dar nejblaženější\\
v~životě úděl.\\
\\
Milosti Pane, ty jenž vše předvídáš:\\
Ochrana světce ať nikdy neschází\\
těm, jimž jsi dopřál tohoto ochránce\\
za svůj vzor míti.\columnbreak

A~jeho záštitou jistě se stane,\\
Nejvyšší vládce, že tvojí nebude\\
chvály na věčnost žádoucí zbaveno\\
nikdy to místo.\\
\\
Učiň to Synu, milostivý Otče,\\
i~Duchu v~obou, jenž přítomen býváš,\\
tak jako nyní stejně i~na věčné\\
světa okruhy.\\
Amen.
\end{translatioMulticol}

%\setlength{\columnsep}{30pt} % prostor mezi sloupci
%}

\vfill

\pars{Versus.} \scriptura{Ps. 36, 30}

% Versus. %%%
\sineinitiali{temporalia/versus-os.gtex}

\noindent \trVersus

\vfill
\pagebreak

\cantusCumNeumis

\pars{Canticum B. Mariæ V.} \scriptura{Vita S. Galli X, 1; \textbf{H320}}
\vspace{-0.5cm}

\antiphona{VIII G}{temporalia/ant-magn-vesp1.gtex}

\trAntMagnificatI

\scriptura{Lc. 1, 46-55}

\cantusSineNeumas
\initiumpsalmi{temporalia/magnificat-initium-viii-G.gtex}

\vspace{-0.4cm}

\psalmusEtTranslatioT{temporalia/magnificat-comb.tex}{10.3cm}

\antiphona{}{temporalia/ant-magn-vesp1.gtex} % repeat the antiphon - new page

\vfill
\pagebreak

\anteOrationem

\pagebreak

%% Oratio. %%%
\pars{Oratio.}

\cuminitiali{}{temporalia/oratio.gtex}
\trOrationis

\vspace{1cm}
\rubrica{Hebdomadarius dicit iterum Dominus vobiscum. Postea cantatur a cantore:}
\vspace{2mm}

\cuminitiali{II}{temporalia/benedicamus-duplex-vesperae.gtex}

\vfill
\pagebreak

\hora{Ad Completorium.} %%%%%%%%%%%%%%%%%%%%%%%%%%%%%%%%%%%%%%%%%%%%%%%%%%%%%%%%%%
\sideThumbs{{\scriptsize{}Completorium}}

\rubrica{Lector petit benedictionem, dicens:}

\cuminitiali{}{temporalia/jubedomnebenedicere.gtex}

\trJubeDomne

\vfill

\pars{Benedictio.}

\cuminitiali{}{temporalia/benedictio-noctemquietam.gtex}

\trComplBenedictio

\vfill

\pars{Lectio brevis.} \scriptura{1Ptr. 5, 8-9}

\cuminitiali{}{temporalia/lectiobrevis-fratressobrii.gtex}

\trComplLectioBr

\vfill

\noindent \Vbardot{} Adiutórium nostrum in nómine Dómini. \Rbardot{} Qui fecit cælum, et terram.

\vfill

\noindent Pater noster \rubricatum{quod dicitur totum secreto.}

\vfill
\pagebreak

\pars{Confessio.}

\noindent Confíteor Deo omnipoténti, beátæ Maríæ semper Vírgini, beáto
Michaéli Archángelo, beáto Ioánni Baptístæ, sanctis Apóstolis Petro
et Paulo, ómnibus Sanctis, et vobis fratres: quia peccávi nimis cogitatióne,
verbo et ópere: mea culpa, mea culpa, mea máxima culpa.
Ídeo precor beátam Maríam semper Vírginem, beátum Michaélum
Archángelum, beátum Ioánnem Baptístam, sanctos Apóstolos Petrum
et Paulum, omnes Sanctos, et vos fratres, oráre pro me ad Dóminum
Deum nostrum.

\vfill

\noindent \Vbardot{} Misereátur nostri omnípotens Deus, et, dimíssis peccátis nostris, perdúcat
nos ad vitam ætérnam. \Rbardot{} Amen.

\vfill

\noindent \Vbardot{} Indulgéntiam, absolutiónem et remissiónem peccatórum nostrórum tríbuat nobis
omnípotens et miséricors Dóminus. \Rbardot{} Amen.

\vfill

\rubrica{Et facta absolutione dicitur:}

\sineinitiali{temporalia/convertenosdeus.gtex}

\vfill

\cuminitiali{}{temporalia/deusinadiutorium-communis.gtex}

\vfill
\pagebreak

\pars{Psalmus 1.} \scriptura{Ps. 4}

\initiumpsalmi{temporalia/ps4-initium-dir-auto.gtex}

\psalmusEtTranslatioT{temporalia/ps4dir-comb.tex}{10cm}

\vfill
\pagebreak

\pars{Psalmus 2.} \scriptura{Ps. 90}

\psalmusEtTranslatioT{temporalia/ps90-comb.tex}{10cm}

\pagebreak

\pars{Psalmus 3.} \scriptura{Ps. 133}

\psalmusEtTranslatioT{temporalia/ps133-comb.tex}{10cm}

\vfill

\pars{Hymnus.}

\antiphona{II}{temporalia/hym-TeLucis.gtex}
%\input{cantus/amon33/hym-TeLucis-bohtext.tex}

\pagebreak

\pars{Capitulum.} \scriptura{Ier. 14, 9}

\cuminitiali{}{temporalia/capitulum-tuautem.gtex}

% preklad Jeruz. bible
\trComplCapituli

\vfill

\pars{Versus.} \scriptura{Ps. 16, 8}

{
\grechangedim{interwordspacetext}{0.24 cm plus 0.15 cm minus 0.05 cm}{scalable}%
\sineinitiali{temporalia/versus-custodi.gtex}
\grechangedim{interwordspacetext}{0.32 cm plus 0.15 cm minus 0.05 cm}{scalable}%
}

\noindent \trComplVersus

\vfill
\pagebreak

\cantusCumNeumis

\pars{Oratio.}

\cantusSineNeumas

\cuminitiali{}{temporalia/oratio-visita.gtex}

\trComplOrationis

\vfill

\sineinitiali{temporalia/benedicamus-minor.gtex}

\vfill

\pars{Benedictio.}

\noindent Benedícat et custódiat nos omnípotens et miséricors Dóminus, \gredagger{}
Pater, et Fílius, et Spíritus Sanctus. \Rbardot{} Amen.

\vfill
\pagebreak

\pars{Antiphona finalis B. M. V.}

\antiphona{V}{temporalia/ant-salveregina-simplex.gtex}

\trSalveRegina

\vspace{0.5cm}

\sineinitiali{temporalia/versus-orapronobis.gtex}

\trOraProNobis

\vfill
\pagebreak

\hora{Ad Matutinum.} %%%%%%%%%%%%%%%%%%%%%%%%%%%%%%%%%%%%%%%%%%%%%%%%%%%%%%%%%%
\sideThumbs{Matutinum}

\vspace{2mm}

\cuminitiali{}{temporalia/dominelabiamea.gtex}

\vspace{2mm}

\pars{Invitatorium.} \scriptura{\textbf{H320}}

\vspace{-4mm}

\antiphona{III}{temporalia/matinv-RegemConfessorum.gtex}

\trMatInvitatorium

\scriptura{Ps. 94 (Textus antiquus latinus); \textbf{H443}}

\vspace{-5mm}

\antiphona{III}{temporalia/venite3a.gtex}

\trMatVeniteA

\scriptura{Repetitur integrum Invitatorium.}

\antiphona{}{temporalia/venite3b.gtex}

\trMatVeniteB

\scriptura{Repetitur altera pars Invitatorii.}

\rubrica{In sequenti Psalmi versu, ad verba \textnormal{veníte, adorémus et procidámus ante Deum}, genuflectitur.}

\antiphona{}{temporalia/venite3c.gtex}

\trMatVeniteC

\scriptura{Repetitur integrum Invitatorium.}

\antiphona{}{temporalia/venite3d.gtex}

\trMatVeniteD

\scriptura{Repetitur altera pars Invitatorii.}

\vfill
\pagebreak

\antiphona{}{temporalia/venite3e.gtex}

\trMatVeniteE

\scriptura{Repetitur integrum Invitatorium.}

\antiphona{}{temporalia/venite3f.gtex}

\scriptura{Repetitur altera pars Invitatorii. Denique repetitur integrum Invitatorium.}

\antiphona{}{temporalia/matinv-RegemConfessorum.gtex}

\vfill
\pagebreak

\pars{Hymnus.}

\vspace{-0.5cm}

{
\grechangedim{interwordspacetext}{0.30 cm plus 0.15 cm minus 0.05 cm}{scalable}%
\antiphona{II}{temporalia/hym-IsteConfessor.gtex}
\grechangedim{interwordspacetext}{0.32 cm plus 0.15 cm minus 0.05 cm}{scalable}%
}
{
\vspace{0.5cm}
%\setlength{\columnsep}{0pt} % prostor mezi sloupci
%\begin{translatioMulticol}{3}
Tento vyznavač Páně, jehož ctí\\
a~chválí v~úctě národy pod sluncem\\
s~radostí vstoupil v~dnešní den blaženě\\
na trůn milosti.\\
\\
Byl zbožný, moudrý, pokorný a~čistý,\\
bedlivý vedl bez skvrny svůj život\\
až vánkem Ducha oživil nakonec\\
své lidské údy.\columnbreak

A~také často udílí odměnou,\\
že mnohé údy nemocí stíhané\\
napraví, že opět zdraví nabudou\\
a~nemoc zkrotí.\\
\\
Náš zástup nyní jemu se poddává\\
zpívaje chvály vítězi slavnému,\\
aby nám svými prosbami pomáhal\\
po všechny časy.\columnbreak

Buď slavně pozdraven velký ten vládce,\\
jenž sedí v~záři nebeského trůnu\\
a~vládne řádům veškerého světa\\
trojí i~jeden.\\
Amen.
\end{translatioMulticol}

%\setlength{\columnsep}{30pt} % prostor mezi sloupci
}

\vfill
\pagebreak

\subhora{In I. Nocturno}

\pars{Psalmus 1.} \scriptura{Vita S. Galli I, 1; \textbf{H321}}

\vspace{-0.5cm}

\antiphona{I D}{temporalia/matant1.gtex}

\trMatAntI

\scriptura{Psalmus 1.}

\initiumpsalmi{temporalia/ps1-initium-i-D-auto.gtex}

\vspace{-0.4cm}

\psalmusEtTranslatioT{temporalia/ps1-comb.tex}{10cm}

%\antiphona{}{temporalia/matant1.gtex} % repeat the antiphon - new page

\vfill
\pagebreak

\pars{Psalmus 2.} \scriptura{Vita S. Galli I, 1; \textbf{H321}}

\vspace{-0.5cm}

\antiphona{II D}{temporalia/matant2.gtex}

\trMatAntII

\scriptura{Psalmus 2.}

\initiumpsalmi{temporalia/ps2-initium-ii-D-auto.gtex}

\vspace{-0.4cm}

\psalmusEtTranslatioT{temporalia/ps2-comb.tex}{10cm}

%\antiphona{}{temporalia/matant2.gtex} % repeat the antiphon - new page

\vfill
\pagebreak

\pars{Psalmus 3.} \scriptura{Vita S. Galli IX, 1; \textbf{H321}}

\vspace{-0.5cm}

\antiphona{VII a}{temporalia/matant3.gtex}

\trMatAntIII

\scriptura{Psalmus 3.}

\initiumpsalmi{temporalia/ps3-initium-vii-a-auto.gtex}

\psalmusEtTranslatioT{temporalia/ps3-comb.tex}{10cm}

%\antiphona{}{temporalia/matant3.gtex} % repeat the antiphon - new page

\vfill
\pagebreak

\pars{Psalmus 4.} \scriptura{Vita S. Galli IX, 1; \textbf{H321}}

\vspace{-0.5cm}

\antiphona{VIII G2}{temporalia/matant4.gtex}

\trMatAntIV

\scriptura{Psalmus 4.}

\initiumpsalmi{temporalia/ps4-initium-viii-G2-auto.gtex}

\psalmusEtTranslatioT{temporalia/ps4-comb.tex}{10cm}

%\antiphona{}{temporalia/matant4.gtex} % repeat the antiphon - new page

\vfill
\pagebreak

\pars{Psalmus 5.} \scriptura{Vita S. Galli IX, 4; \textbf{H321}}

\vspace{-0.5cm}

\antiphona{IV E}{temporalia/matant5.gtex}

\trMatAntV

\scriptura{Psalmus 5.}

\initiumpsalmi{temporalia/ps5-initium-iv-E-auto.gtex}

\psalmusEtTranslatioT{temporalia/ps5-comb.tex}{10cm}

\antiphona{}{temporalia/matant5.gtex} % repeat the antiphon - new page

\vfill
\pagebreak

\pars{Psalmus 6.} \scriptura{Vita S. Galli IX, 4; \textbf{H321}}

\vspace{-0.5cm}

{
\grechangedim{interwordspacetext}{0.18 cm plus 0.15 cm minus 0.05 cm}{scalable}%
\antiphona{VIII G}{temporalia/matant6.gtex}
\grechangedim{interwordspacetext}{0.32 cm plus 0.15 cm minus 0.05 cm}{scalable}%
}

\trMatAntVI

\scriptura{Psalmus 8.}

\initiumpsalmi{temporalia/ps8-initium-viii-G-auto.gtex}

\psalmusEtTranslatioT{temporalia/ps8-comb.tex}{10cm}

%\antiphona{}{temporalia/matant6.gtex} % repeat the antiphon - new page

\vfill
\pagebreak

\pars{Versus.} \scriptura{Sap. 10, 10}

\sineinitiali{temporalia/versus-justum.gtex}

\noindent \trMatVersusI

\vfill

\sineinitiali{temporalia/oratiodominica-mat.gtex}

\vfill

\pars{Absolutio.}

\cuminitiali{}{temporalia/absolutio-exaudi.gtex}

\trMatAbsolutioI

\vfill
\pagebreak

\cuminitiali{}{temporalia/benedictio-solemn-benedictione.gtex}

\trMatBenedictioI

\vfill

\cuminitiali{}{temporalia/tonus-lectionis-solemnis.gtex}

\vfill

\pars{Lectio I.} \scriptura{Vita S. Galli I, 1}

\noindent Incípit Vitæ San\textit{cti} \textbf{Gal}li.

\textusEtTranslatio{
  Cum præclára sanctíssimi viri Columbáni, conversátio per omnem Hibérniam célebris haberétur,
  paréntes beáti Galli, secúndum sǽculum nobíles, secúndum Deum religiósi,
  fílium suum prímæ ætátis flore niténtem, cum oblatióne Dómino offeréntes,
  íllius magistério commendavérunt, ut in reguláris vitæ profíceret disciplína.
}{\trMatLecI}{10cm}

\tuAutem

\vfill
\pagebreak

\pars{Responsorium 1.} \scriptura{Vita S. Galli I, 1; \textbf{H321}}

\responsorium{II}{temporalia/matresp1.gtex}{\trMatRespI}

\vfill
\pagebreak

\cuminitiali{}{temporalia/benedictio-solemn-unigenitus.gtex}

\trMatBenedictioII

\vfill

\pars{Lectio II.} \scriptura{Vita S. Galli I, 1}

\textusEtTranslatio{
  Cumque bonæ indólis vir charo nutrirétur afféctu, magno virtútum crevit augménto.
  Supérna quoque grátia se præveniénte, tanto stúdio divínas epotávit Scriptúras,
  ut de thesáuro suo nova próferre posset et vétera.
}{\trMatLecII}{10cm}

\tuAutem

\vfill
%\pagebreak

\pars{Responsorium 2.} \scriptura{Vita S. Galli IV, 3; \textbf{H322}}

\responsorium{VII}{temporalia/matresp2.gtex}{\trMatRespII}

\vfill
\pagebreak

\cuminitiali{}{temporalia/benedictio-solemn-spiritus.gtex}

\trMatBenedictioIII

\vfill

\pars{Lectio III.} \scriptura{Vita S. Galli I, 2}

\textusEtTranslatio{
  Grammatícæ étiam régulas, metrórumque subtilitátes capáci consequerétur ingénio.
  Obscúra autem Scripturárum tam sapiénter scire voléntibus reserávit,
  ut cuncti qui eius prudéntiam et sermónes audíerant, admiratióne eum et laude digníssimum iudicárent.
}{\trMatLecIII}{10cm}

\tuAutem

\vfill
\pagebreak

\pars{Responsorium 3.} \scriptura{Vita S. Galli VI, 3; \Vbar{} ibid. VI, 4; \textbf{H322}}

\responsorium{VIII}{temporalia/matresp3.gtex}{\trMatRespIII}

\vfill
\pagebreak

\cuminitiali{}{temporalia/benedictio-solemn-inunitate.gtex}

\trMatBenedictioIV

\vfill

\pars{Lectio IV.} \scriptura{Vita S. Galli I, 2}

\textusEtTranslatio{
  Qua sapiéntiæ maturitáte factum est ut universórum commúni consílio,
  et iussióne Columbáni abbátis, per síngulos sacræ promotiónis gradus ascéndens, invítus sacerdótii suscíperet dignitátem.
}{\trMatLecIV}{10cm}

\tuAutem

\vfill
%\pagebreak

\pars{Responsorium 4.} \scriptura{Vita S. Galli XIV, 2; \Vbar{} ibid. X, 5; \textbf{H322}}

\responsorium{VIII}{temporalia/matresp4.gtex}{\trMatRespIV}

\vfill
\pagebreak

\subhora{In II. Nocturno}

\pars{Psalmus 7.} \scriptura{Vita S. Galli XIV, 3; \textbf{H322}}

\vspace{-0.5cm}

\antiphona{VI F}{temporalia/matant7.gtex}

\trMatAntVII

\scriptura{Psalmus 14.}

\initiumpsalmi{temporalia/ps14-initium-vi-F-auto.gtex}

\psalmusEtTranslatioT{temporalia/ps14-comb.tex}{10cm}

%\antiphona{}{temporalia/matant7.gtex} % repeat the antiphon - new page

\vfill
\pagebreak

\pars{Psalmus 8.} \scriptura{Vita S. Galli XVIII, 4; \textbf{H322}}

\vspace{-0.5cm}

\antiphona{III b}{temporalia/matant8.gtex}

\trMatAntVIII

\scriptura{Psalmus 20.}

\initiumpsalmi{temporalia/ps20-initium-iii-b-auto.gtex}

\psalmusEtTranslatioT{temporalia/ps20-comb.tex}{10cm}

%\antiphona{}{temporalia/matant8.gtex} % repeat the antiphon - new page

\vfill
\pagebreak

\pars{Psalmus 9.} \scriptura{Vita S. Galli XXIV, 2; \textbf{H322}}

\vspace{-0.5cm}

\antiphona{VIII G2}{temporalia/matant9.gtex}

\trMatAntIX

\scriptura{Psalmus 23.}

\initiumpsalmi{temporalia/ps23-initium-viii-G2-auto.gtex}

\psalmusEtTranslatioT{temporalia/ps23-comb.tex}{10cm}

%\antiphona{}{temporalia/matant9.gtex} % repeat the antiphon - new page

\vfill
\pagebreak

\pars{Psalmus 10.} \scriptura{Vita S. Galli XXIV, 3; \textbf{H323}}

\vspace{-0.5cm}

\antiphona{VIII G}{temporalia/matant10.gtex}

\trMatAntX

\scriptura{Psalmus 64.}

\initiumpsalmi{temporalia/ps64-initium-viii-G-auto.gtex}

\psalmusEtTranslatioT{temporalia/ps64-comb.tex}{10cm}

%\antiphona{}{temporalia/matant10.gtex} % repeat the antiphon - new page

\vfill
\pagebreak

\pars{Psalmus 11.} \scriptura{Vita S. Galli XXV, 5; \textbf{H323}}

\vspace{-0.5cm}

\antiphona{I g}{temporalia/matant11.gtex}

\trMatAntXI

\scriptura{Psalmus 74.}

\initiumpsalmi{temporalia/ps74-initium-i-g-auto.gtex}

\psalmusEtTranslatioT{temporalia/ps74-comb.tex}{10cm}

%\antiphona{}{temporalia/matant11.gtex} % repeat the antiphon - new page

\vfill
\pagebreak

\pars{Psalmus 12.} \scriptura{\textbf{H323}}

\vspace{-0.5cm}

\antiphona{III a}{temporalia/matant12.gtex}

\trMatAntXII

\scriptura{Psalmus 91.}

\initiumpsalmi{temporalia/ps91-initium-iii-a-auto.gtex}

\psalmusEtTranslatioT{temporalia/ps91-comb.tex}{10cm}

%\antiphona{}{temporalia/matant12.gtex} % repeat the antiphon - new page

\vfill
\pagebreak

\pars{Versus.}

\sineinitiali{temporalia/versus-amavit.gtex}

\noindent \trMatVersusII

\vfill

\sineinitiali{temporalia/oratiodominica-mat.gtex}

\vfill

\pars{Absolutio.}

\cuminitiali{}{temporalia/absolutio-ipsius.gtex}

\trMatAbsolutioII

\vfill
\pagebreak

\cuminitiali{}{temporalia/benedictio-solemn-deus.gtex}

\trMatBenedictioV

\vfill

\pars{Lectio V.} \scriptura{Vita S. Galli I, 3; ibid. II, 1}

\textusEtTranslatio{
  Ergo dum sacris instáret offíciis, die noctúque précibus Dóminum placávit et lácrymis,
  et supérni inspectóris óculis placere desíderans, pro virtútum et vitæ méritis amabátur ab ómnibus, plácuit univérsis.
  Dum hæc ageréntur, quotídie beátus Columbánus, evangélicam cúpiens asséqui perfectiónem,
  ut videlícet ómnibus quæ habébat relíctis, crucem suam tólleret, et nudus Dóminum sequerétur.
}{\trMatLecV}{10cm}

\tuAutem

\vfill
\pagebreak

\pars{Responsorium 5.} \scriptura{Vita S. Galli XI, 4; \textbf{H323}}

\responsorium{IV}{temporalia/matresp5.gtex}{\trMatRespV}

\vfill
\pagebreak

\cuminitiali{}{temporalia/benedictio-solemn-christus.gtex}

\trMatBenedictioVI

\vfill

\pars{Lectio VI.} \scriptura{Cf. Vita S. Galli II, 2; ibid. IV, 2}

\textusEtTranslatio{
  Ascendéntes ígitur navim, venérunt Británniam,
  et inde ad Gállias transfretárunt, ubi habitácula construéntes,
  et hómines Idololátras íbidem commanéntes docébant adoráre Patrem et Fílium et Spíritum Sanctum.
}{\trMatLecVI}{10cm}

\tuAutem

\vfill
\pagebreak

\pars{Responsorium 6.} \scriptura{Vita S. Galli XI, 5; \textbf{H323}}

\responsorium{II}{temporalia/matresp6.gtex}{\trMatRespVI}

\vfill
\pagebreak

\cuminitiali{}{temporalia/benedictio-solemn-ignem.gtex}

\trMatBenedictioVII

\vfill

\pars{Lectio VII.} \scriptura{Vita S. Galli IV, 3}

\textusEtTranslatio{
  Beátus vero Gallus, zelo pietátis armátus, fana in quibus dæmóniis sacrificábant,
  igni succéndit, et quæcúnque invénit obláta, demérsit in lacum.
  Qua causa permóti ira, et iníto consílio, Gallum perímere voluérunt:
  Columbánum vero flagéllis cæsum et contuméliis afféctum de suis fínibus proturbáre cœpérunt.
}{\trMatLecVII}{10cm}

\tuAutem

\vfill
\pagebreak

\pars{Responsorium 7.} \scriptura{Vita S. Galli XII, 3; \Vbar{} ibid. XI, 5; \textbf{H323}}

\responsorium{VIII}{temporalia/matresp7.gtex}{\trMatRespVII}

\vfill
\pagebreak

\cuminitiali{}{temporalia/benedictio-solemn-acunctis.gtex}

\trMatBenedictioVIII

\vfill

\pars{Lectio VIII.} \scriptura{Cf. Vita S. Galli VIII, 4; ibid. IX, 1}

\textusEtTranslatio{
  His ígitur provocáti iniúriis, Itáliam pétere decrevérunt.
  Igitur cum proficiscéndi tempus instáret, beátum Gallum repentína febris invásit.
  Unde abbátis sui pédibus advolútus, indicávit se infirmitáte veheménti laboráre,
  et ídeo iter propósitum non posse perfícere.
}{\trMatLecVIII}{10cm}

\tuAutem

\vfill
\pagebreak

\pars{Responsorium 8.} \scriptura{Vita S. Galli XXVI, 1.2; \Vbar{} ibid. XXVI, 2; \textbf{H324}}

\responsorium{II}{temporalia/matresp8.gtex}{\trMatRespVIII}

\vfill
\pagebreak

\subhora{In III. Nocturno}

\pars{Ad Cantica.} \scriptura{Vita S. Galli XXIX, 4; \textbf{H324}}

\vspace{-0.5cm}

\antiphona{I g}{temporalia/matant13.gtex}

\trMatAntIX

\scriptura{Canticum Beatitudo Sapientis; Sir. 14, 22; ibid. 15, 3.4.6}

\initiumpsalmi{temporalia/beatitudosapientis-initium-i-g-auto.gtex}

\psalmusEtTranslatioT{temporalia/beatitudosapientis-comb.tex}{10cm}

\vfill

\scriptura{Canticum Ieremiæ; Ier. 17, 7-8}

\psalmusEtTranslatioT{temporalia/jeremiae-comb.tex}{10cm}

\vfill

\scriptura{Canticum Ecclesiasticæ; Sir. 31, 8-11}

\psalmusEtTranslatioT{temporalia/ecclesiasticus31-comb.tex}{10cm}

\antiphona{}{temporalia/matant13.gtex} % repeat the antiphon - new page

\vfill
\pagebreak

\pars{Versus.}

\sineinitiali{temporalia/versus-magna.gtex}

\noindent \trMatVersusIII

\vfill

\sineinitiali{temporalia/oratiodominica-mat.gtex}

\vfill

\pars{Absolutio.}

\cuminitiali{}{temporalia/absolutio-avinculis.gtex}

\trMatAbsolutioIII

\vfill
\pagebreak

\cuminitiali{}{temporalia/benedictio-solemn-evangelica.gtex}

\trMatBenedictioIX

\vfill

% Léctio sancti Evangélii secúndum Lucam.
\pars{Lectio IX.} \scriptura{Lc. 12, 35-40}

\noindent Léctio sancti \textit{E}\textit{van}\textbf{gé}lii~\grestar{} secún\textit{dum} \textbf{Lu}cam.

\textusEtTranslatio{
  In \textit{il}\textit{lo} \textbf{tém}pore:~\grestar{}
  Dixit Iesus discípulis suis: Sint lumbi vestri præcíncti, et lucérnæ ardéntes in má\textit{ni}\textit{bus} \textbf{ves}tris,~\grestar{}
  et vos símiles homínibus exspectántibus dóminum suum, quando revertátur \textit{a} \textbf{núp}tiis.
  \textit{Et} \textbf{ré}liqua.
}{\trMatLecIXa}{10cm}

% Homilía sancti Gregorii Papæ.
\scriptura{Homilia 13. in Evang.}

\noindent Homilía sancti Gregori\textit{i} \textbf{Pa}pæ.

\textusEtTranslatio{
  Sancti Evangélii, fratres charíssimi, apérta vobis est léctio recitáta.
  Sed ne alíquibus ipsa eius planíties alta fortásse videátur,
  eam sub brevitáte transcúrrimus, quátenus eius exposítio ita nesciéntibus fiat cógnita,
  ut tamen sciéntibus non sit onerósa.
  Quia viris luxúria in lumbis sit, féminis in umbílico, testátur Dóminus,
  qui de diábolo ad beátum Iob lóquitur, dícens:
  Virtus eius in lumbis eius, et fortitúdo íllius in umbílico ventris eius.
  A principáli ígitur sexu lumbórum nómine luxúria designátur,
  cum Dóminus dicit: Sint lumbi vestri præcíncti.
  Lumbos enim præcíngimus cum carnis luxúriam per continéntiam coarctámus.
  Sed quia minus est mala non ágere, nisi étiam quisque stúdeat et bonis opéribus insudáre, prótinus ádditur:
  Et lucérnæ ardéntes in mánibus vestris.
  Lucérnas quippe ardéntes in mánibus tenémus cum per bona ópera próximis nostris lucis exémpla monstrámus.
  De quibus profécto opéribus Dóminus dicit: Lúceat lux vestra coram homínibus,
  ut vídeant ópera vestra bona, et gloríficent Patrem vestrum qui in cœlis est.
  Duo autem sunt quæ iubéntur, et lumbos restríngere, et lucérnas ténere,
  ut et mundítia sit castitátis in córpore, et lumen veritátis in operatióne.
  Redemptóri étenim nostro unum sine áltero placére nequáquam potest,
  si aut is qui bona agit adhuc luxúriæ inquinaménta non déserit,
  aut is qui castitáte præéminet necdum se per bona ópera exércet.
  Nec cástitas ergo magna est sine bono ópere, nec opus bonum est áliquod sine castitáte.
  Sed et si utrúmque ágitur, restat ut quisquis ille est spe ad supérnam pátriam tendat,
  et nequáquam se a vítiis pro mundi huius honestáte contíneat.
  Qui etsi quædam bona aliquándo pro honestáte ínchoat, in eius tamen intentióne non debet permanére,
  nec per bona ópera præséntis mundi glóriam quǽrere,
  sed totam spem in Redemptóris sui advéntum constítuat.
  Unde et prótinus súbditur: Et vos símiles homínibus exspectántibus dóminum suum,
  quando revertátur a núptiis.
  Ad núptias quippe Dóminus ábiit, quia resúrgens a mórtuis,
  ascéndens in cœlum, supérnam sibi angelórum multitúdinem novus homo copulávit.
  Qui tunc revértitur, cum nobis iam per iudícium manifestátur.
}{\trMatLecIXb}{10cm}

\tuAutem

\vfill
%\pagebreak

\pars{Responsorium 9.} \scriptura{Vita S. Galli XXIX, 4; \textbf{H324}}

\responsorium{I}{temporalia/matresp9.gtex}{\trMatRespIX}

\vfill
\pagebreak

\cuminitiali{}{temporalia/benedictio-solemn-ille.gtex}

\trMatBenedictioX

\vfill

\pars{Lectio X.} \scriptura{Homilia 13. in Evang.}

\textusEtTranslatio{
  Bene autem de servis exspectántibus súbditur:
  Ut cum venérit et pulsáverit, conféstim apériant ei.
  Venit quippe Dóminus cum ad iudícium próperat, pulsat vero,
  cum iam per ægritúdinis moléstias esse mortem vicínam desígnat. 
  Cui conféstim apérimus, si hunc cum amóre suscípimus.
  Aperíre enim iúdici pulsánti non vult, qui exíre de córpore trépidat,
  et vidére eum quem contempsísse se méminit iúdicem formídat.
  Qui autem de sua spe et operatióne secúrus est,
  pulsánti conféstim apérit, quia lætus iúdicem sustínet;
  et cum tempus propínquæ mortis agnóverit, de glória retributiónis hilaréscit.
  Unde et prótinus súbditur: Beáti sunt servi illi,
  quos cum venérit dóminus, invenérit vigilántes.
  Vígilat qui ad aspéctum veri lúminis mentis óculos apértos tenet,
  vígilat qui servat operándo quod credit, vígilat qui a se torpóris et negligéntiæ ténebras repéllit.
  Hinc étenim Paulus dicit: Evigiláte, iusti, et nolíte peccáre.
  Hinc rursus ait; Hora est iam nos de somno súrgere.
  Sed véniens Dóminus quid servis vigilántibus exhíbeat audiámus:
  Amen dico vobis quod præcínget se, et fáciet eos discúmbere, et tránsiens ministrábit illis.
  Præcínget se, id est ad retributiónem præparábit;
  et fáciet illos discúmbere, id est in ætérna quiéte refóveri.
  Discúmbere quippe nostrum in regno quiéscere est.
  Unde rursum Dóminus dicit: Vénient et recúmbent cum Abraham, Isaac et Iacob.
  Tránsiens autem Dóminus minístrat, quia lucis suæ illustratióne nos sátiat.
  Transíre vero dictum est, cum de iudício ad regnum redit.
  Vel certe Dóminus nobis post iudícium transit,
  quia ab humanitátis forma in divinitátis suæ contemplatiónem nos elévat.
  Et transíre eius est in claritátis suæ speculatiónem nos dúcere, cum eum quem in humanitáte in iudício cérnimus,
  étiam in divinitáte post iudícium vidémus.
  Ad iudícium quippe véniens, in forma servi ómnibus appáret, quia scriptum est:
  Vidébunt in quem transfixérunt.
  Sed cum repróbi in supplícium córruunt, iusti ad claritátis eius glóriam pertrahúntur, sicut scriptum est:
  Tollátur ímpius, ne vídeat glóriam Dei.
}{\trMatLecX}{10cm}

\tuAutem

\vfill
%\pagebreak

\pars{Responsorium 10.} \scriptura{Vita S. Galli XXX, 6; \textbf{H324}}

\responsorium{VII}{temporalia/matresp10.gtex}{\trMatRespX}

\vfill
\pagebreak

\cuminitiali{}{temporalia/benedictio-solemn-cujus.gtex}

\trMatBenedictioXI

\vfill

\pars{Lectio XI.} \scriptura{Homilia 13. in Evang.}

\textusEtTranslatio{
  Sed quid si servi in prima vigília negligéntes exístunt?
  Prima quippe vigília primæ ætatis custódia est.
  Sed neque sic desperándum est, et a bono ópere cessándum.
  Nam longanimitátis suæ patiéntiam insínuans Dóminus, subdit:
  Et si vénerit in secúnda vigília, et si in tértia vígilia vénerit,
  et ita invenérit, beáti sunt servi illi.
  Prima quippe vigília primǽvum tempus est, id est puerítia.
  Secúnda, adolescéntia vel iuvéntus, quæ auctoritáte sacri elóquii unum sunt,
  dicénte Salomóne: Lætáre iúvenis in adolescéntia tua.
  Tértia autem, senéctus accípitur.
  Qui ergo vigiláre in prima vigília nóluit custódiat vel secúndam,
  ut qui convérti a pravitátibus suis in puerítia negléxit ad vias vitæ saltem in témpore iuventútis evígilet.
  Et qui evigiláre in secúnda vigília nóluit tértiæ vigíliæ remédia non amíttat,
  ut qui in iuventúte ad vias vitæ non evígilat saltem in senectúte resipíscat.
  Pensáte, fratres charíssimi, quia conclúsit Dei píetas durítiam nostram.
  Non est iam quid homo excusatiónis invéniat.
  Deus despícitur, et exspéctat; contémni se videt, et revócat;
  iniúriam de contémptu suo súscipit, et tamen quandóque reverténtibus étiam prǽmia promíttit.
  Sed nemo hanc eius longanimitátem négligat, quia tanto districtiórem iustítiam in iudício exíget,
  quanto longiórem patiéntiam ante iudícium prærogávit.
  Hinc étenim Paulus dicit: Ignóras quóniam benígnitas Dei ad pœniténtiam te addúcit?
  Tu autem secúndum durítiam tuam et cor impœ́nitens thesáurizas tibi iram in die iræ et revelatiónis iusti iudícii Dei.
  Hinc Psalmísta ait: Deus iudex iustus, fortis, et longánimis.
  Dictúrus quippe longánimem, præmísit iustum,
  ut quem vides peccáta delinquéntium diu patiénter ferre, scias hunc étiam quandóque distrícte iudicáre.
  Hinc per quemdam sapiéntem dícitur: Altíssimus enim est pátiens rédditor.
  Pátiens enim rédditor dícitur, quia peccáta hóminum et pátitur et reddit.
  Nam quos diu, ut convertántur, tólerat, non convérsos dúrius damnat.
  Ad excutiéndam vero mentis nostræ desídiam, étiam exterióra damna per similitúdinem ad médium deducúntur,
  ut per hæc ánimus ad sui custódiam suscitétur.
  Nam dícitur: Hoc autem scitóte, quia si sciret paterfamílias qua hora fur veníret, vigiláret útique,
  et non síneret pérfodi domum suam.
  Ex qua præmíssa similitúdine étiam exhortátio subinfértur, cum dícitur:
  Et vos estóte paráti, quia qua hora non putátis Fílius hóminis véniet.
  Nesciénte enim patrefamílias fur domum pérfodit, quia dum a sui custódia spíritus dormit,
  improvísa mors véniens carnis nostræ habitáculum irrúmpit,
  et eum quem dóminum domus invénerit dormiéntem necat,
  quia cum ventúra damna spíritus mínime prǽvidet, hunc mors ad supplícium nesciéntem rapit.
  Furi autem resísteret, si vigiláret, quia advéntum iúdicis,
  qui occúlte ánimam rapit, prǽcavens, ei pœniténdo occúrreret, ne impœ́nitens períret.
}{\trMatLecXI}{10cm}

\tuAutem

\vfill
%\pagebreak

\pars{Responsorium 11.} \scriptura{Vita S. Galli XXXI, 1.2; \textbf{H324}}

\responsorium{IV}{temporalia/matresp11.gtex}{\trMatRespXI}

\vfill
\pagebreak

\cuminitiali{}{temporalia/benedictio-solemn-adsocietatem.gtex}

\trMatBenedictioXII

\vfill

\pars{Lectio XII.} \scriptura{Homilia 13. in Evang.}

\textusEtTranslatio{
  Horam vero últimam Dóminus noster idcírco vóluit nobis esse incógnitam,
  ut semper possit esse suspécta, ut dum illam prævidére non póssumus,
  ad illam sine intermissióne præparémur.
  Proínde, fratres mei, in conditióne mortalitátis vestræ mentis óculos fígite,
  veniénti vos iúdici per fletus quotídie et laménta præparáte.
  Et cum certa mors máneat ómnibus, nolíte de temporális vitæ providéntia incérta cogitáre.
  Terrenárum rerum vos cura non aggrávet.
  Quantislíbet enim auri et argénti mólibus circumdétur,
  quibuslíbet pretiósis véstibus induátur caro,
  quid est áliud quam caro?
  Nolíte ergo atténdere quid habétis, sed quid estis.
  Vultis audíre quid estis?
  Prophéta índicat, dicens: Vere fenum est pópulus.
  Si enim fenum pópulus non est, ubi sunt illi qui ea quæ hódie cólimus nobíscum transácto anno beáti Felícis natalítia celebravérunt?
  O quanta et quália de præséntis vitæ provisióne cogitábant, sed, subripiénte mortis artículo,
  repénte in his quæ prævídere nolébant invénti sunt,
  et cuncta simul temporália quæ congregáta quasi stabíliter ténere videbántur amisérunt.
  Si ergo transácta multitúdo géneris humáni per nativitátem víruit in carne,
  per mortem áruit in púlvere, vidélicet fenum fuit.
  Quia ígitur moméntis suis horæ fúgiunt, agíte, fratres charíssimi,
  ut in boni opéris mercéde teneántur.
  Audite quid sápiens Salomon dicat: Quodcúnque potest manus tua fácere, instánter operáre,
  quia nec opus, nec sciéntia, nec rátio, nec sapiéntia erunt apud ínferos, quo tu próperas.
  Quia ergo et ventúræ mortis tempus ignorámus, et post mortem operári non póssumus,
  súperest ut ante mortem témpora indúlta rapiámus.
  Sic enim sic mors ipsa cum vénerit vincétur, si priúsquam véniat semper timeátur.
}{\trMatLecXII}{10cm}

\tuAutem

\vfill
\pagebreak

\pars{Responsorium 12.} \scriptura{\textbf{H325}}

\responsorium{I}{temporalia/matresp12.gtex}{\trMatRespXI}

\vfill
\pagebreak

% Te Deum

\pars{Hymnus Ambrosianus}

{
\grechangedim{interwordspacetext}{0.28 cm plus 0.15 cm minus 0.05 cm}{scalable}%
\cuminitiali{III}{temporalia/tedeum-solemnis.gtex}
\grechangedim{interwordspacetext}{0.32 cm plus 0.15 cm minus 0.05 cm}{scalable}%
}

\trTeDeum

\vfill
\pagebreak

% Evangelium

\cuminitiali{}{temporalia/tonus-evangelii-b.gtex}

\vfill

\scriptura{Lc. 12, 35-40}

\noindent Léctio san\textit{cti} \textit{E}\textit{van}\textbf{gé}lii {\textnormal{\grestar{}}} secúndum \textbf{Lu}cam.

\textusEtTranslatio{
  In illo témpore:
  Dixit Jesus discípulis suis: Sint lumbi vestri præcíncti, et lucérnæ ardéntes in mánibus vestris,
  et vos símiles homínibus exspectántibus dóminum suum,
  quando revertátur a núptiis: ut, cum vénerit, et pulsáverit, conféstim apériant ei.
  Beáti servi illi, quos cum vénerit dóminus, invénerit vigilántes:
  amen dico vobis, quod præcínget se, et fáciet illos discúmbere, et tránsiens ministrábit illis. 
  Et si vénerit in secúnda vigília, et si in tértia vigília vénerit,
  et ita invénerit, beáti sunt, beati sunt servi illi.
  Hoc autem scitóte, quóniam si sciret paterfamílias,
  quia hora fur vénerit, vigiláret útique, et non síneret pérfodi domum suam.
  Et vos estúte paráti, quis qua hora non putátis.
  Fílius hóminis véniet.
}{\trMatEvangelium}{10cm}

\vfill
\cuminitiali{I}{temporalia/tedecetlaus.gtex}

\trTeDecetLaus

\vfill
\pagebreak

\sineinitiali{temporalia/domineexaudi.gtex}

\vfill

\pars{Oratio.}

\cuminitiali{}{temporalia/oratio.gtex}
\trOrationis

\vfill

\noindent \Vbardot{} Dómine, exáudi oratiónem meam.
\Rbardot{} Et clamor meus ad te véniat.

\vfill

% Nocturnale Romanum 2002, p. LXXVI Benedicamus Domino seems to match
% the one from Solemn Laudes.
\cuminitiali{V}{temporalia/benedicamus-solemnis-laud.gtex}

\vfill

\noindent \Vbardot{} Fidélium ánimæ per misericórdiam Dei requiéscant in pace.
\Rbardot{} Amen.

\trFideliumAnimae

\vfill
\pagebreak

\hora{Ad Laudes.} %%%%%%%%%%%%%%%%%%%%%%%%%%%%%%%%%%%%%%%%%%%%%%%%%%%%%%%%%%
\sideThumbs{Laudes}

% Psalmi festivi (AM33, pg. 721):
% 66 // 92, 99, 62, Dan3, 148+149+150

\vspace{1cm}
\cuminitiali{}{temporalia/deusinadiutorium-communis.gtex}
\vspace{1cm}

\cantusSineNeumas

\pars{Psalmus 1.} \scriptura{Ps. 66}

{
\grechangedim{interwordspacetext}{0.28 cm plus 0.15 cm minus 0.05 cm}{scalable}%
\initiumpsalmi{temporalia/ps66-initium-dir-auto.gtex}
\grechangedim{interwordspacetext}{0.32 cm plus 0.15 cm minus 0.05 cm}{scalable}%
}

\psalmusEtTranslatioT{temporalia/ps66-comb.tex}{10cm}

\vfill
\pagebreak

\pars{Psalmus 2.} \scriptura{Vita S. Galli XXXII, 1.2; \textbf{H325}}

\vspace{-0.5cm}

\antiphona{VI F}{temporalia/ant1.gtex}

\trAntI

\scriptura{Ps. 50}

\initiumpsalmi{temporalia/ps50-initium-vi-F-auto.gtex}

\psalmusEtTranslatioT{temporalia/ps50-comb.tex}{10cm}

\antiphona{}{temporalia/ant1.gtex} % repeat the antiphon - new page

\vfill
\pagebreak

\pars{Psalmus 3.} \scriptura{Vita S. Galli XXXII, 1; \textbf{H325}}

\vspace{-0.5cm}

\antiphona{VII a}{temporalia/ant2.gtex}

\trAntII

\scriptura{Ps. 117}

\initiumpsalmi{temporalia/ps117-initium-vii-a-auto.gtex}

\psalmusEtTranslatioT{temporalia/ps117-comb.tex}{10cm}

\antiphona{}{temporalia/ant2.gtex} % repeat the antiphon - new page

\vfill
\pagebreak

\pars{Psalmus 4.} \scriptura{Vita S. Galli XXXII, 2; \textbf{H326}}

\vspace{-0.5cm}

\antiphona{I g}{temporalia/ant3.gtex}

\trAntIII

\scriptura{Ps. 62.}

\initiumpsalmi{temporalia/ps62-initium-i-g-auto.gtex}

\vspace{-0.6cm}

\psalmusEtTranslatioT{temporalia/ps62-comb.tex}{10cm}

%\antiphona{}{temporalia/ant3.gtex} % repeat the antiphon - new page

\vfill
\pagebreak

\pars{Psalmus 5.} \scriptura{Vita S. Galli XXXII, 2; \textbf{H326}}

\vspace{-0.5cm}

\antiphona{III a}{temporalia/ant4.gtex}

\trAntIV

\scriptura{Canticum trium puerorum, Dan. 3, 57-88 et 56}

\initiumpsalmi{temporalia/dan3-initium-iii-a-auto.gtex}

\psalmusEtTranslatioT{temporalia/dan3-comb.tex}{10cm}

\rubrica{Hic non dicitur Gloria Patri, neque Amen.}
\vspace{1cm}

\antiphona{}{temporalia/ant4.gtex} % repeat the antiphon - new page

\vfill
\pagebreak

\pars{Psalmus 6.} \scriptura{Vita S. Galli XXXII, 2; \textbf{H326}}

\vspace{-0.5cm}

\antiphona{I g}{temporalia/ant5.gtex}

\trAntV

\scriptura{Ps. 148}

\initiumpsalmi{temporalia/ps148-initium-i-g-auto.gtex}

\newlength{\psVItransW}
\setlength{\psVItransW}{10.5cm}

\psalmusEtTranslatioT{temporalia/ps148-comb.tex}{10cm}

\vspace{-0.5cm}

\rubrica{Hic non dicitur Gloria Patri.}

\vfill
\pagebreak

%
\scriptura{Ps. 149}

\vspace{-0.3cm}

\psalmusEtTranslatioT{temporalia/ps149-comb.tex}{10cm}

\vspace{-0.8cm}

\rubrica{Hic non dicitur Gloria Patri.}

\vspace{-0.2cm}

%
\scriptura{Ps. 150}

\vspace{-0.3cm}

\psalmusEtTranslatioT{temporalia/ps150-comb.tex}{10cm}

\antiphona{}{temporalia/ant5.gtex} % repeat the antiphon - new page

\vfill
\pagebreak

\cantusSineNeumas

\pars{Capitulum.} \scriptura{Sir. 45, 1-2}

\cuminitiali{}{temporalia/capitulum-DilectusDeo.gtex}

% preklad Jeruz. bible
\trCapituli

\vfill
\pars{Responsorium breve.} \scriptura{Sap. 10, 10}

\antiphona{VI}{temporalia/respl.gtex}

\trRespLaud

\vfill
\pagebreak

% Hymnus. %%%
\pars{Hymnus.} \scriptura{Walahfrid Strabus (\gredagger{} 849)}

{
\grechangedim{interwordspacetext}{0.20 cm plus 0.15 cm minus 0.05 cm}{scalable}%
\cuminitiali{I}{temporalia/hym-VitaSanctorum.gtex}
\grechangedim{interwordspacetext}{0.32 cm plus 0.15 cm minus 0.05 cm}{scalable}%
}
%{
%\vspace{-0.3cm}
%\setlength{\columnsep}{7pt} % prostor mezi sloupci
%\begin{translatioMulticol}{3}
Svatých život je cestou a~záchranou\\
Kriste, jenž dáváš mír a~bezúhonnost\\
Původci svému ti hlasem i~myslí\\
zpíváme hymnus.\\
\\
Sžíráni láskou k~tomu, v~němž moci jest\\
plnost zjevena; se vším, co v~moci své\\
mají jen zbožní a~po čem vší silou\\
a~srdcem touží.\\
\\
Pro jeho zbožnost jsi svatého Havla\\
učinil vzorem nebeské jasnosti;\\
abychom jeho učením unikli\\
temnotám mysli. \\
\\
Jako pták zpěvný první se probudil\\
a~skutky svými v~životě dosvědčil\\
to, co mu moudrost jeho učitele\\
ctnostného vlila.\columnbreak

V~slově byl mocný, v~činech úctyhodný,\\
vždy znovu bažil po věčném bohatství;\\
tak zjevně došel odměnou znamení\\
nebeské ctnosti.\\
\\
Prosíme světa, Původce a~spáso\\
na jeho prosby pohlédni teď svaté,\\
rač popřát lidu dobrotivým srdcem,\\
čeho si žádá.\\
\\
Pokojné časy a~víře stálost,\\
churavým zdraví, padlým slitování;\\
všem potom lidem dar nejblaženější\\
v~životě úděl.\\
\\
Milosti Pane, ty jenž vše předvídáš:\\
Ochrana světce ať nikdy neschází\\
těm, jimž jsi dopřál tohoto ochránce\\
za svůj vzor míti.\columnbreak

A~jeho záštitou jistě se stane,\\
Nejvyšší vládce, že tvojí nebude\\
chvály na věčnost žádoucí zbaveno\\
nikdy to místo.\\
\\
Učiň to Synu, milostivý Otče,\\
i~Duchu v~obou, jenž přítomen býváš,\\
tak jako nyní stejně i~na věčné\\
světa okruhy.\\
Amen.
\end{translatioMulticol}

%\setlength{\columnsep}{30pt} % prostor mezi sloupci
%}

\vfill

\pars{Versus.} \scriptura{Ps. 36, 30}

% Versus. %%%
\sineinitiali{temporalia/versus-os.gtex}

\noindent \trVersus

\vfill
\pagebreak

\cantusCumNeumis

\pars{Canticum Zachariæ.} \scriptura{Cf. Vita S. Galli XXX, 6; ibid. XXXIII, 1; ibid. XVII, 5; \textbf{H326}}

\antiphona{VIII G}{temporalia/ant-ben-laud.gtex}

\trAntBenedictus

\scriptura{Lc. 1, 68-79}

\initiumpsalmi{temporalia/benedictus-initium-viiisoll-G-auto.gtex}

\psalmusEtTranslatioT{temporalia/benedictus-comb.tex}{10cm}

\antiphona{}{temporalia/ant-ben-laud.gtex} % repeat the antiphon - new page

\vfill
\pagebreak

\cantusSineNeumas

\anteOrationem

\pagebreak

% Oratio. %%%
\pars{Oratio.}

\cuminitiali{}{temporalia/oratio.gtex}
\trOrationis

\vspace{1cm}
\rubrica{Hebdomadarius dicit iterum Dominus vobiscum. Postea cantatur a cantore:}
\vspace{2mm}

\cuminitiali{}{temporalia/benedicamus-duplex-laudes.gtex}

\vfill
\pagebreak

\iffalse

\hora{Ad Tertiam.} %%%%%%%%%%%%%%%%%%%%%%%%%%%%%%%%%%%%%%%%%%%%%%%%%%%%%%%%%%
\sideThumbs{Tertia}

\vspace{1cm}
\cuminitiali{}{temporalia/deusinadiutorium-communis.gtex}
\vspace{1cm}

\pars{Hymnus.}

\cuminitiali{II}{temporalia/hym-NuncSancte.gtex}
\begin{translatioMulticol}{3}
Nyní k nám, Duchu přesvatý,\\
s Otcem i Synem přistup v milosti.\\
V jediné chvíli nám s nimi\\
do srdce láskou pronikni.\columnbreak

Ať ústa, jazyk i smysly,\\
mysl a síla vyznají\\
lásku, jež ke všem horlivě\\
i k Tobě v nás zaplane.\columnbreak

To splň nám, dobrý Otče náš,\\
i~ty, jenž rovné božství máš,\\
i~Duchu, který těšíš nás\\
a~vládneš, Bože, v~každý čas.\\
Amen.
\end{translatioMulticol}


\vfill
\pagebreak

\pars{Psalmus.} \scriptura{\textbf{H307}}

\antiphona{VII c}{temporalia/ant2.gtex}

\trAntII

\scriptura{Ps. 118, 33-80}

\initiumpsalmi{temporalia/ps118v_vi-initium-vii-c-auto.gtex}

\psalmusEtTranslatioT{temporalia/ps118v_vi-comb.tex}{10cm}

\vspace{-0.5cm}

\psalmusEtTranslatioT{temporalia/ps118vii_viii-comb.tex}{10cm}

\vspace{-0.5cm}

\psalmusEtTranslatioT{temporalia/ps118ix_x-comb.tex}{10cm}

\vspace{-0.7cm}

\antiphona{}{temporalia/ant2.gtex} % repeat the antiphon - new page

\vfill
\pagebreak

\raggedcolumns

% Capitulum. %%%
\cantusSineNeumas

\pars{Capitulum.} \scriptura{Sir. 24, 14}

\cuminitiali{}{temporalia/capitulum-AbInitio.gtex}

% preklad Jeruz. bible
\trCapituli

\vfill

\pars{Versus.}

{
\grechangedim{interwordspacetext}{0.28 cm plus 0.15 cm minus 0.05 cm}{scalable}%
\sineinitiali{temporalia/versus-specie.gtex}
\grechangedim{interwordspacetext}{0.32 cm plus 0.15 cm minus 0.05 cm}{scalable}%
}

%\noindent \trVersusTertia

\vfill

\rubrica{Ante Orationem, cantatur a Superiore:}

\pars{Supplicatio Litaniæ.}

\cuminitiali{}{temporalia/supplicatiolitaniae.gtex}

\vfill

\pars{Oratio Dominica.}

\sineinitiali{temporalia/oratiodominica-mat.gtex}

\vfill

\sineinitiali{temporalia/domineexaudi.gtex}

\vfill
\pagebreak

% Oratio. %%%
\pars{Oratio.}

\cuminitiali{}{temporalia/oratio.gtex}
\trOrationis

\vfill

\sineinitiali{temporalia/benedicamus-minor.gtex}

\vfill

\noindent \Vbardot{} Fidélium ánimæ per misericórdiam Dei requiéscant in pace.
\Rbardot{} Amen.

\trFideliumAnimae

\vfill
\pagebreak

\hora{Ad Missam - missa IX.} %%%%%%%%%%%%%%%%%%%%%%%%%%%%%%%%%%%%%%%%%%%%%%%%%%%%%
\sideThumbs{Missa}

\vspace{0.3cm}

\pars{Antiphona ad introitum.} \scriptura{Ps. 44, 2; Cf. \textbf{E75}}

\antiphona{I}{temporalia/introitus-GaudeamusOmnes.gtex}

\trIntroitus

\vfill

\pars{Kyrie IX \textit{(Cum iubilo)}.} \scriptura{XII. s.}

\vspace{0.3cm}

\cuminitiali{I}{temporalia/ix-kyrie.gtex}

\vfill
\pagebreak

\pars{Gloria IX.} \scriptura{XI. s.}

\vspace{0.3cm}

\cuminitiali{VII}{temporalia/ix-gloria.gtex}

\vfill
\pagebreak

\pars{Graduale.} \scriptura{Cf. \textbf{C132}}

\antiphona{IV}{temporalia/graduale-BenedictaEtVenerabilis.gtex}

\trGraduale

\vfill

\pars{Alleluia.} \scriptura{Cf. \textbf{E367}}

\antiphona{VII}{temporalia/alleluia-SolemnitasGloriosae.gtex}

\trAlleluia

\vfill
\pagebreak

\pars{Credo III.} \scriptura{XVII. s.}

{
\vspace{-0.25cm}
\vfill
\grechangedim{interwordspacetext}{0.23 cm plus 0.15 cm minus 0.05 cm}{scalable}%
\grechangedim{spacelinestext}{0.40617 cm}{scalable}%
\cuminitiali{V}{temporalia/credo-iii.gtex}
\grechangedim{interwordspacetext}{0.32 cm plus 0.15 cm minus 0.05 cm}{scalable}%
\grechangedim{spacelinestext}{0.50617 cm}{scalable}%
}

\vfill
\pagebreak

\pars{Offertorium.} \scriptura{Cf. \textbf{E210}}

\antiphona{VIII}{temporalia/offertorium-BeataEs.gtex}

\trOffertorium

\vspace{1cm}

\pars{Sanctus IX.} \scriptura{XIV. s.}

\vspace{0.3cm}

\cuminitiali{V}{temporalia/ix-sanctus.gtex}

\vspace{1cm}
\pars{Agnus Dei IX.} \scriptura{(X) XIII. s.}

\vspace{0.3cm}

\cuminitiali{V}{temporalia/ix-agnusdei.gtex}

\vfill
\pagebreak

\pars{Communio.} \scriptura{Lc. 1, 48.49}

\antiphona{VI}{temporalia/communio-BeatamMeDicent.gtex}

\trCommunio

\scriptura{Lc. 1, 46-47.50-55}

\sineinitiali{temporalia/communio-versus-Magnificat.gtex}

\vfill
\pagebreak

\hora{Ad Sextam.} %%%%%%%%%%%%%%%%%%%%%%%%%%%%%%%%%%%%%%%%%%%%%%%%%%%%%%%%%%
\sideThumbs{Sexta}

\vspace{1cm}
\cuminitiali{}{temporalia/deusinadiutorium-communis.gtex}
\vspace{1cm}

\pars{Hymnus.}

\antiphona{II}{temporalia/hym-RectorPotens.gtex}
\input{cantus/amon33/hym-RectorPotens-bohtext.tex}

\vfill
\pagebreak

\pars{Psalmus.} \scriptura{\textbf{H307}}

{
\grechangedim{interwordspacetext}{0.10 cm plus 0.15 cm minus 0.05 cm}{scalable}%
\antiphona{VI F}{temporalia/ant3.gtex}
\grechangedim{interwordspacetext}{0.32 cm plus 0.15 cm minus 0.05 cm}{scalable}%
}

\trAntIII

\scriptura{Ps. 118, 81-128}

\initiumpsalmi{temporalia/ps118xi_xii-initium-vi-F-auto.gtex}

\psalmusEtTranslatioT{temporalia/ps118xi_xii-comb.tex}{10cm}

\vspace{-0.5cm}

\psalmusEtTranslatioT{temporalia/ps118xiii_xiv-comb.tex}{10cm}

\vspace{-0.5cm}

\psalmusEtTranslatioT{temporalia/ps118xv_xvi-comb.tex}{10cm}

\vspace{-0.5cm}

{
\grechangedim{interwordspacetext}{0.10 cm plus 0.15 cm minus 0.05 cm}{scalable}%
\antiphona{}{temporalia/ant3.gtex} % repeat the antiphon - new page
\grechangedim{interwordspacetext}{0.32 cm plus 0.15 cm minus 0.05 cm}{scalable}%
}

\vfill
\pagebreak

\raggedcolumns

% Capitulum. %%%
\cantusSineNeumas

\pars{Capitulum.} \scriptura{Sir. 24, 15-16}

\cuminitiali{}{temporalia/capitulum-EtSic.gtex}

% preklad Jeruz. bible
\trCapituliEtSic

\vfill

\pars{Versus.}

\sineinitiali{temporalia/versus-adjuvabit.gtex}

%\noindent \trVersusSexta

\vfill

\vfill

\rubrica{Ante Orationem, cantatur a Superiore:}

\pars{Supplicatio Litaniæ.}

\cuminitiali{}{temporalia/supplicatiolitaniae.gtex}

\vfill

\pars{Oratio Dominica.}

\sineinitiali{temporalia/oratiodominica-mat.gtex}

\vfill

\sineinitiali{temporalia/domineexaudi.gtex}

\vfill
\pagebreak

% Oratio. %%%
\pars{Oratio.}

\cuminitiali{}{temporalia/oratio.gtex}
\trOrationis

\vfill

\sineinitiali{temporalia/benedicamus-minor.gtex}

\vfill

\noindent \Vbardot{} Fidélium ánimæ per misericórdiam Dei requiéscant in pace.
\Rbardot{} Amen.

\trFideliumAnimae

\vfill
\pagebreak

\hora{Ad Nonam.} %%%%%%%%%%%%%%%%%%%%%%%%%%%%%%%%%%%%%%%%%%%%%%%%%%%%%%%%%%
\sideThumbs{Nona}

\vspace{1cm}
\cuminitiali{}{temporalia/deusinadiutorium-communis.gtex}
\vspace{1cm}

\pars{Hymnus.}

\antiphona{II}{temporalia/hym-RerumDeus.gtex}
\input{cantus/amon33/hym-RerumDeus-bohtext.tex}

\vfill
\pagebreak

\pars{Psalmus.} \scriptura{\textbf{H308}}

{
\grechangedim{interwordspacetext}{0.14 cm plus 0.15 cm minus 0.05 cm}{scalable}%
\antiphona{VII c}{temporalia/ant5.gtex}
\grechangedim{interwordspacetext}{0.32 cm plus 0.15 cm minus 0.05 cm}{scalable}%
}

\trAntV

\scriptura{Ps. 118, 129-176}

\initiumpsalmi{temporalia/ps118xvii_xviii-initium-vii-c-auto.gtex}

\psalmusEtTranslatioT{temporalia/ps118xvii_xviii-comb.tex}{10cm}

\vspace{-0.5cm}

\psalmusEtTranslatioT{temporalia/ps118xix_xx-comb.tex}{10cm}

\vspace{-0.5cm}

\psalmusEtTranslatioT{temporalia/ps118xxi_xxii-comb.tex}{10cm}

\vspace{-0.5cm}

{
\grechangedim{interwordspacetext}{0.14 cm plus 0.15 cm minus 0.05 cm}{scalable}%
\antiphona{}{temporalia/ant5.gtex} % repeat the antiphon - new page
\grechangedim{interwordspacetext}{0.32 cm plus 0.15 cm minus 0.05 cm}{scalable}%
}

\vfill
\pagebreak

\raggedcolumns

% Capitulum. %%%
\cantusSineNeumas

\pars{Capitulum.} \scriptura{Sir. 24, 19-20}

\cuminitiali{}{temporalia/capitulum-InPlateis.gtex}

% preklad Jeruz. bible
\trCapituliInPlateis

\vfill

\pars{Versus.}

\sineinitiali{temporalia/versus-elegit.gtex}

%\noindent \trVersusNona

\vfill

\rubrica{Ante Orationem, cantatur a Superiore:}

\pars{Supplicatio Litaniæ.}

\cuminitiali{}{temporalia/supplicatiolitaniae.gtex}

\vfill

\pars{Oratio Dominica.}

\sineinitiali{temporalia/oratiodominica-mat.gtex}

\vfill

\sineinitiali{temporalia/domineexaudi.gtex}

\vfill
\pagebreak

% Oratio. %%%
\pars{Oratio.}

\cuminitiali{}{temporalia/oratio.gtex}
\trOrationis

\vfill

\sineinitiali{temporalia/benedicamus-minor.gtex}

\vfill

\noindent \Vbardot{} Fidélium ánimæ per misericórdiam Dei requiéscant in pace.
\Rbardot{} Amen.

\trFideliumAnimae

\fi

\newpage
\RemoveSideThumbs
\pagestyle{empty}

%%% COLOPHON

\begin{center}
%\includegraphics[width=4cm]{imagines/giotto.jpg}
\end{center}

\vfill

Fontes. 
Textus et cantus officii divini secundum
Antiphonale Sacrosanctæ Romanæ Eclesiæ Pro Diurnis Horis, Romæ 1912
et Nocturnale Romanum, 2002, præter: psalmi 149 et 150 post
psalmum 148 in Laudibus additi secundum Antiphonale Monasticum pro Diurnis Horis,
Solesmis 1934; lectio sancti Evangelii et hymnus Te Decet Laus post hymnum
Ambrosianum additi secundum ritum monasticum vetum; responsorium breve
in Laudibus et Vesperis additum secundum Antiphonale Monasticum. /
Textus et cantus missæ secundum
Graduale triplex, Solesmis 1979. /
Versus ad communionem secundum
http://media.musicasacra.com/pdf/beatamme.pdf (12.VIII.2012). /
Translatio capituli et lectionis sumpta est ex:
Jeruzalémská bible, Praha-Kostelní Vydří 2009. /
Translationes psalmorum ex
Hejčl Jan: Žaltář čili Kniha žalmů, Praha 1922. /
Neumæ super canto missæ de codicibus Cantatorium, Stiftsbibl. 359 et
Einsiedeln,
Stiftsbibl. 121 et neumæ super canto officii divini de codice Hartker,
Stiftsbibl. 391.

Collaborantes.
Textus latinos cantusque transcripsit et omnem laborem typographicum peregit
Jakub Jelínek. /
Proprios cantus festi in linguam bohemicam Jakub Pavlík et Tereza Hodinová
transtulerunt. Pro erroribus autem prior est accusandus. /
Psalmos in lingua bohemica de libro supra dicto transcripsit
Barbora Maturová. /
Filip Srovnal librum istum præparare mandavit et laborem exprobrationibus
utilissimis comitabatur. Ipse etiam neumas super cantus missæ
de Graduali triplici transcripsit. /
Imaginem, quæ paginam tituli ornat, Klára Jirsová pinxit.

Instrumenta adhibita.
LuaTeX, %http://www.luatex.org / 
Gregorio, %http://home.gna.org/gregorio /
typi Junicode. %http://junicode.sourceforge.net

\begin{center}
Liber hic imprimis ad usum chori 
\guillemotright Conventus Choralis\guillemotleft\ 
paratus est
et secundum eius consuetudines.
http://www.introitus.cz

\vspace{0.8cm}

{\large Editio Sancti Wolfgangi \annusEditionis.}

\vspace{2mm}

Series \guillemotright Conventus\guillemotleft, vol. VIII.

\vspace{0.8cm}

http://stiwolfgangi.xf.cz

\vfill

\today

\end{center}

\end{document}
