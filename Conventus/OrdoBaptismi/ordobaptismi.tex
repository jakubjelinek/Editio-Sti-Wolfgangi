% LuaLaTeX

\documentclass[a4paper, twoside, 12pt]{article}
\usepackage[latin]{babel}
%\usepackage[landscape, left=3cm, right=1.5cm, top=2cm, bottom=1cm]{geometry} % okraje stranky
\usepackage[portrait, a4paper, mag=1300, truedimen, left=0.8cm, right=0.8cm, top=0.8cm, bottom=0.8cm]{geometry} % okraje stranky

\usepackage{fontspec}
\setmainfont[FeatureFile={junicode.fea}, Ligatures={Common, TeX}, RawFeature=+fixi]{Junicode}
%\setmainfont{Junicode}

% shortcut for Junicode without ligatures (for the Czech texts)
\newfontfamily\nlfont[FeatureFile={junicode.fea}, Ligatures={Common, TeX}, RawFeature=+fixi]{Junicode}

\usepackage{multicol}
\usepackage{color}
\usepackage{lettrine}
\usepackage{fancyhdr}

% usual packages loading:
\usepackage{luatextra}
\usepackage{graphicx} % support the \includegraphics command and options
\usepackage{gregoriotex} % for gregorio score inclusion
\usepackage{gregoriosyms}
\usepackage{wrapfig} % figures wrapped by the text
\usepackage{parcolumns}
\usepackage[contents={},opacity=1,scale=1,color=black]{background}
\usepackage{tikzpagenodes}
\usepackage[savepos]{zref}
\usepackage{calc}
\usepackage{longtable}
\newif\ifparvulus
\newif\ifignatius
\parvulustrue
\ignatiustrue

\setlength{\headheight}{12pt}

\input{conventuscommune.tex} % Often used macros
%%%% Preklady jednotlivych zpevu (nektere se opakuji, a je dobre mit je
% vsechny na jedne hromade)

% HOURS ---

\newcommand{\trAntI}{\translatioCantus{Muž boží měl kožený toulec, pečlivě
zavázaný, jenž mu visel na šíji a~často se ho dotýkal.}}

\newcommand{\trAntII}{\translatioCantus{Klíč od~něho tak dobře střežil, že
dokud žil v~těle, nikdo z~jeho žáků nezvěděl, co je uvnitř.}}

\newcommand{\trAntIII}{\translatioCantus{Ale když se odebral z~tohoto
života, schránku otevřeli a~objevili v~ní žíněné roucho a~měděný řetěz
potřísněný krví.}}

\newcommand{\trAntIV}{\translatioCantus{A když prohlédli mistrovo tělo,
nalezli jeho tělo na čtyřech místech hluboce zbrázděno ranami od řetězu.}}

\newcommand{\trAntV}{\translatioCantus{Krev vytékající z~těch ran, místy
prostoupila i~žíněným rouchem.}}

\newcommand{\trCapituli}{\translatioCantus{
Miláčkovi Boha a~lidí,
Mojžíšovi požehnané paměti,~\gredagger{}
dopřál slávu rovnou slávě svatých~\grestar{}
učinil ho mocným na postrach nepřátelům
a~jeho slovy zastavil divy.}}

\newcommand{\trLectioBrevis}{\translatioCantus{
Pamatujte na své představené,
kteří vám hlásali Boží slovo.
Uvažte, jak oni skončili život, a~napodobujte jejich víru.
Ježíš Kristus je stejný včera i~dnes i~navěky.
Nenechte se svést věelijakými cizími naukami.}}

\newcommand{\trRespLaud}{\translatioCantus{Spravedlivého vodil Hospodin~\grestar{}
po přímých stezkách. \Vbardot{} A~ukázal mu Boží království.}}

\newcommand{\trRespLaudB}{\translatioCantus{Na tvých hradbách, Jeruzaléme,
ustanovil jsem strážné;~\grestar{}
budou bdít nad mým lidem. \Vbardot{} Ani ve dne, ani v~noci nesmějí nikdy
mlčet.}}

\newcommand{\trVersus}{\translatioCantus{\Vbardot{} Ústa spravedlivého šeptají moudrost, aleluja.
\Rbardot{} A~jeho jazyk ohlašuje právo, aleluja.}}

\newcommand{\trAntBenedictus}{\translatioCantus{Když na bujné oře vložili
nosítka a~sňali jim uzdu, vydali se přímo k~cele božího muže.}}

\newcommand{\trPreces}{\translatioCantus{
\noindent S vděčností chvalme Krista, dobrého Pastýře, \gredagger{} který dal život za své ovce, \grestar{} a~pokorně ho prosme: \Rbardot{} Pane, buď pastýřem svého lidu.

\noindent Kriste, ty dáváš církvi pastýře, a~jejich službou se ujímáš svého lidu, \grestar{} dej, ať v~lásce těch, kteří nás vedou, poznáváme, jak nás miluješ. \Rbardot{} Pane, buď pastýřem svého lidu.

\noindent Ty stále konáš skrze své zástupce službu pastýře a~učitele, \grestar{} nepřestávej nás nikdy vést prostřednictvím svých služebníků. \Rbardot{} Pane, buď pastýřem svého lidu.

\noindent Ty prokazuješ svému lidu skrze jeho pastýře službu lékaře duše i~těla, \grestar{} ochraňuj náš život a~veď nás ke svatosti. \Rbardot{} Pane, buď pastýřem svého lidu.

\noindent Ty posíláš své svaté, aby slovem i~příkladem vedli tvůj lid k~tobě, \grestar{} na jejich přímluvu nás posiluj, abychom vytrvali na cestě, která vede k~věčnému životu. \Rbardot{} Pane, buď pastýřem svého lidu.}}

\newcommand{\trOrationis}{\translatioCantus{Bože, jenž nám dopřáváš radovat
se z~výroční slavnosti svatého tvého vyznavače Havla, uděl dobrotivě,
abychom když slavíme jeho narození, též se řídili podobou jeho skutků.
Skrze…}}
 % Czech translations of the proper texts

\setlength{\columnsep}{15pt} % prostor mezi sloupci

\makeatletter
\newcounter{litscore}
\newcounter{littabstop}[litscore]
\newcommand{\grealign}{%
	\@bsphack%
	\ifgre@boxing\else%
		\kern\gre@dimen@begindifference%
		\stepcounter{littabstop}%
		\expandafter\zsavepos{stop-\thelitscore-\thelittabstop}%
		\kern-\gre@dimen@begindifference%
	\fi%
	\@esphack%
}

\newcommand{\litsetstops}{%
  \gdef\nstabbing@stops{%
    \hspace*{-\oddsidemargin}\hspace{-1in}%
    \hspace*{\zposx{stop-\thelitscore-1} sp}\=%
  }%
  \count@=\@ne
  \loop\ifnum\count@<\value{littabstop}%
    \begingroup\edef\x{\endgroup
      \noexpand\g@addto@macro\noexpand\nstabbing@stops{%
        \noexpand\hspace{-\noexpand\zposx{stop-\thelitscore-\the\count@} sp}%
        \noexpand\hspace{\noexpand\zposx{stop-\thelitscore-\the\numexpr\count@+1} sp}\noexpand\=%
      }%
    }\x
    \advance\count@\@ne
  \repeat
  \nstabbing@stops\kill
}
\makeatother

\newenvironment{nstabbing}
  {\setlength{\topsep}{0pt}%
   \setlength{\partopsep}{0pt}%
   \tabbing%
   \litsetstops}
  {\endtabbing\stepcounter{litscore}}

\renewcommand{\rubricatum}[1]{\textit{\color{red}#1}}

%%%%%%%%%%%%%%%%%%%%%%%%%%%%%%%%%%%%%%%%%%%%%%%%%%%%%%%%%%%%%%%%%%%%%%%%%%%%%%%%%%%%%%%%%%%%%%%%%%%%%%%%%%%%%
\begin{document}

% Here we set the space around the initial.
% Please report to http://home.gna.org/gregorio/gregoriotex/details for more details and options
\grechangedim{afterinitialshift}{2.2mm}{scalable}
\grechangedim{beforeinitialshift}{2.2mm}{scalable}
\grechangedim{interwordspacetext}{0.20 cm plus 0.15 cm minus 0.05 cm}{scalable}%
\grechangedim{annotationraise}{-0.2cm}{scalable}

% Here we set the initial font. Change 38 if you want a bigger initial.
% Emit the initials in red.
\grechangestyle{initial}{\color{red}\fontsize{38}{38}\selectfont}

\renewcommand{\headrulewidth}{0pt} % no horiz. rule at the header
\pagestyle{empty}

\grechangedim{spaceabovelines}{0.2cm}{scalable}%

\begin{center}
{\LARGE ORDO BAPTISMI PARVULORUM}

\vspace{0.5cm}

\ifparvulus
\ifignatius
{\Large Ignatius Ludovicus Srovnal\\
26. Augusti MMXXII}
\else
{\Large Angelus Benedictus-Ioseph Srovnal\\
19. Ianuarii MMXIX}
\fi
\else
{\Large Myrrha Liesse Srovnalová\\
18. Februarii MMXVII}
\fi

\vspace{1cm}

\ifparvulus
\ifignatius
\includegraphics[width=5cm]{ignatius.jpg}
\else
\includegraphics[width=5cm]{benedictlabre.png}
\fi
\else
\includegraphics[width=5cm]{ordobaptismi.jpg}
\fi

\vspace{0.2cm}

\end{center}

\vfill

\gresetnabcfont{grelaon}{8}

\pars{Antiphona ad introitum} \scriptura{Cf. Is. 55, 1; Ps. 77, 1; \textbf{L75}}

\vspace{-0.6cm}

\antiphona{II}{temporalia/introitus-Sitientes.gtex}

\trIntroitus

\vfill
\pagebreak

\pars{\textsc{Ritus recipiendi parvulum}}

\vspace{-0.3cm}

\psalmusEtTranslatioT{temporalia/ritusrecipiendi-comb.tex}{6cm}

\ifignatius
\pars{\textsc{Ad processionem}}

\gresetnabcfont{gregall}{8}

\antiphona{VI}{temporalia/alleluia.gtex}

\scriptura{Ps. 117, 1-6}

\initiumpsalmi{temporalia/communio-versus-Confitemini-initium.gtex}

\vspace{-1cm}

\psalmusEtTranslatioTS{temporalia/communio-versus-Confitemini-comb.tex}{6cm}

\pars{\textsc{Sacra verbi Dei celebratio}}

\pars{Graduale}

\psalmusEtTranslatioT{temporalia/graduale-comb.tex}{6cm}
\else
\pars{\textsc{Sacra verbi Dei celebratio}}

\gresetnabcfont{gregall}{8}

\pars{Alleluia} \scriptura{Ps. 67, 18.19; \textbf{C116}}

\vspace{-0.6cm}

\antiphona{VIII}{temporalia/alleluia-DominusInSina.gtex}

\trAlleluia
\fi

\gresetnabcfont{grelaon}{8}

\vfill

\vspace{0.4cm}

\psalmusEtTranslatioT{temporalia/evangelium-initium-comb.tex}{6cm}

\ifparvulus
\pars{Léctio sancti Evángelii secúndum Marcum.} \scriptura{Mc. 10, 13-16}

\psalmusEtTranslatioT{temporalia/evangelium2-comb.tex}{6cm}
\else
\pars{Léctio sancti Evángelii secúndum Ioánnem.} \scriptura{Io. 19, 38-42}

\psalmusEtTranslatioT{temporalia/evangelium-comb.tex}{6cm}
\fi

\vfill

\ifignatius
\pagebreak

\pars{Hymnus}

\antiphona{VIII}{temporalia/hym-VeniCreator.gtex}

\vspace{0.3cm}

\vfill
\pagebreak
\fi

\noindent\pars{Oratio fidelium}

\vspace{-0.5cm}

\psalmusEtTranslatioT{temporalia/oratiofidelium-comb.tex}{6cm}

\sineinitiali{temporalia/terogamus.gtex}

\psalmusEtTranslatioT{temporalia/oratiofidelium-2-comb.tex}{6cm}

\sineinitiali{temporalia/litaniae.gtex}

\begin{nstabbing}
\ifparvulus
\>Sancti Angeli \>\>\>\>\textbf{De}-\>\>i, \>orá-\>\>\textit{te} \>\textit{pro} \textbf{no}bis.\\
\fi
\>Sancte Ioánnes \>\>\>Ba-\textbf{ptí}-\>\>\>sta, \>\>o-\>\textit{ra} \>\textit{pro} \textbf{no}bis.\\
\>Sancte \>\>\>\>\textbf{Io}-\>seph, \>\>\>o-\>\textit{ra} \>\textit{pro} \textbf{no}bis.\\
\>Sancti Petre et \>\>\>\>\textbf{Pau}-\>\>le, \>orá-\>\>\textit{te} \>\textit{pro} \textbf{no}bis.\\
\textellipsis
\end{nstabbing}

\noindent\pars{Oratio exorcismi et uncio præbaptismalis}

\vspace{-0.3cm}

\psalmusEtTranslatioT{temporalia/exorcismus-comb.tex}{6cm}

\vfill
\pagebreak

\pars{\textsc{Celebratio baptismi}}

\vspace{-0.3cm}

\psalmusEtTranslatioT{temporalia/celebratio-baptismi-comb.tex}{6cm}

\pars{\textsc{Ritus explanativi}}

\vspace{-0.3cm}

\psalmusEtTranslatioT{temporalia/ritus-explanativi-comb.tex}{6cm}

\pars{\textsc{Conclusio ritus}}

\scriptura{Gal. 3, 27; \textbf{L112}}

\vspace{-0.6cm}

\antiphona{II}{temporalia/communio-OmnesQui.gtex}

\ifparvulus
\scriptura{Ps. 28, 3}

\initiumpsalmi{temporalia/communio-versus-Vox-initium.gtex}
\fi

\trCommunio

\vspace{-0.3cm}

\psalmusEtTranslatioT{temporalia/conclusio-ritus-comb.tex}{6cm}

\vfill
\pagebreak

%\vspace{-0.3cm}

\ifparvulus
\ifignatius
\gresetnabcfont{gregall}{8}

\pars{Pro pace in Ucraina.} \scriptura{Sir. 50, 25; 2 Esdr. 4, 20; \textbf{H416}}

\vspace{-4mm}

\antiphona{II D}{temporalia/ant-dapacemdomine.gtex}

\trDaPacem

\pars{Antiphona B.M.V.}

\antiphona{I}{temporalia/an_salve_regina.gtex}
\else
\antiphona{}{temporalia/svatyvaclave.gtex}
\fi
\else
\antiphona{VI}{temporalia/alleluia.gtex}

\scriptura{Ps. 117, 1-6}

\initiumpsalmi{temporalia/communio-versus-Confitemini-initium.gtex}

\vspace{-1cm}

\psalmusEtTranslatioTS{temporalia/communio-versus-Confitemini-comb.tex}{6cm}
\fi

\vfill

\end{document}
