% LuaLaTeX

\documentclass[a4paper, twoside, 12pt]{article}
\usepackage[latin]{babel}
%\usepackage[landscape, left=3cm, right=1.5cm, top=2cm, bottom=1cm]{geometry} % okraje stranky
\usepackage[portrait, a4paper, mag=1300, truedimen, left=0.8cm, right=0.8cm, top=0.8cm, bottom=0.8cm]{geometry} % okraje stranky

\usepackage{fontspec}
\setmainfont[FeatureFile={junicode.fea}, Ligatures={Common, TeX}, RawFeature=+fixi]{Junicode}
%\setmainfont{Junicode}

% shortcut for Junicode without ligatures (for the Czech texts)
\newfontfamily\nlfont[FeatureFile={junicode.fea}, Ligatures={Common, TeX}, RawFeature=+fixi]{Junicode}

\usepackage{multicol}
\usepackage{color}
\usepackage{lettrine}
\usepackage{fancyhdr}

% usual packages loading:
\usepackage{luatextra}
\usepackage{graphicx} % support the \includegraphics command and options
\usepackage{gregoriotex} % for gregorio score inclusion
\usepackage{gregoriosyms}
\usepackage{wrapfig} % figures wrapped by the text
\usepackage{parcolumns}
\usepackage[contents={},opacity=1,scale=1,color=black]{background}
\usepackage{tikzpagenodes}
\usepackage{calc}
\usepackage{longtable}

\setlength{\headheight}{12pt}

\input{conventuscommune.tex} % Often used macros
%%%% Preklady jednotlivych zpevu (nektere se opakuji, a je dobre mit je
% vsechny na jedne hromade)

% HOURS ---

\newcommand{\trAntI}{\translatioCantus{Muž boží měl kožený toulec, pečlivě
zavázaný, jenž mu visel na šíji a~často se ho dotýkal.}}

\newcommand{\trAntII}{\translatioCantus{Klíč od~něho tak dobře střežil, že
dokud žil v~těle, nikdo z~jeho žáků nezvěděl, co je uvnitř.}}

\newcommand{\trAntIII}{\translatioCantus{Ale když se odebral z~tohoto
života, schránku otevřeli a~objevili v~ní žíněné roucho a~měděný řetěz
potřísněný krví.}}

\newcommand{\trAntIV}{\translatioCantus{A když prohlédli mistrovo tělo,
nalezli jeho tělo na čtyřech místech hluboce zbrázděno ranami od řetězu.}}

\newcommand{\trAntV}{\translatioCantus{Krev vytékající z~těch ran, místy
prostoupila i~žíněným rouchem.}}

\newcommand{\trCapituli}{\translatioCantus{
Miláčkovi Boha a~lidí,
Mojžíšovi požehnané paměti,~\gredagger{}
dopřál slávu rovnou slávě svatých~\grestar{}
učinil ho mocným na postrach nepřátelům
a~jeho slovy zastavil divy.}}

\newcommand{\trLectioBrevis}{\translatioCantus{
Pamatujte na své představené,
kteří vám hlásali Boží slovo.
Uvažte, jak oni skončili život, a~napodobujte jejich víru.
Ježíš Kristus je stejný včera i~dnes i~navěky.
Nenechte se svést věelijakými cizími naukami.}}

\newcommand{\trRespLaud}{\translatioCantus{Spravedlivého vodil Hospodin~\grestar{}
po přímých stezkách. \Vbardot{} A~ukázal mu Boží království.}}

\newcommand{\trRespLaudB}{\translatioCantus{Na tvých hradbách, Jeruzaléme,
ustanovil jsem strážné;~\grestar{}
budou bdít nad mým lidem. \Vbardot{} Ani ve dne, ani v~noci nesmějí nikdy
mlčet.}}

\newcommand{\trVersus}{\translatioCantus{\Vbardot{} Ústa spravedlivého šeptají moudrost, aleluja.
\Rbardot{} A~jeho jazyk ohlašuje právo, aleluja.}}

\newcommand{\trAntBenedictus}{\translatioCantus{Když na bujné oře vložili
nosítka a~sňali jim uzdu, vydali se přímo k~cele božího muže.}}

\newcommand{\trPreces}{\translatioCantus{
\noindent S vděčností chvalme Krista, dobrého Pastýře, \gredagger{} který dal život za své ovce, \grestar{} a~pokorně ho prosme: \Rbardot{} Pane, buď pastýřem svého lidu.

\noindent Kriste, ty dáváš církvi pastýře, a~jejich službou se ujímáš svého lidu, \grestar{} dej, ať v~lásce těch, kteří nás vedou, poznáváme, jak nás miluješ. \Rbardot{} Pane, buď pastýřem svého lidu.

\noindent Ty stále konáš skrze své zástupce službu pastýře a~učitele, \grestar{} nepřestávej nás nikdy vést prostřednictvím svých služebníků. \Rbardot{} Pane, buď pastýřem svého lidu.

\noindent Ty prokazuješ svému lidu skrze jeho pastýře službu lékaře duše i~těla, \grestar{} ochraňuj náš život a~veď nás ke svatosti. \Rbardot{} Pane, buď pastýřem svého lidu.

\noindent Ty posíláš své svaté, aby slovem i~příkladem vedli tvůj lid k~tobě, \grestar{} na jejich přímluvu nás posiluj, abychom vytrvali na cestě, která vede k~věčnému životu. \Rbardot{} Pane, buď pastýřem svého lidu.}}

\newcommand{\trOrationis}{\translatioCantus{Bože, jenž nám dopřáváš radovat
se z~výroční slavnosti svatého tvého vyznavače Havla, uděl dobrotivě,
abychom když slavíme jeho narození, též se řídili podobou jeho skutků.
Skrze…}}
 % Czech translations of the proper texts

\setlength{\columnsep}{15pt} % prostor mezi sloupci

%%%%%%%%%%%%%%%%%%%%%%%%%%%%%%%%%%%%%%%%%%%%%%%%%%%%%%%%%%%%%%%%%%%%%%%%%%%%%%%%%%%%%%%%%%%%%%%%%%%%%%%%%%%%%
\begin{document}

% Here we set the space around the initial.
% Please report to http://home.gna.org/gregorio/gregoriotex/details for more details and options
\grechangedim{afterinitialshift}{2.2mm}{scalable}
\grechangedim{beforeinitialshift}{2.2mm}{scalable}
\grechangedim{interwordspacetext}{0.20 cm plus 0.15 cm minus 0.05 cm}{scalable}%
\grechangedim{annotationraise}{-0.2cm}{scalable}

% Here we set the initial font. Change 38 if you want a bigger initial.
% Emit the initials in red.
\grechangestyle{initial}{\color{red}\fontsize{38}{38}\selectfont}

\renewcommand{\headrulewidth}{0pt} % no horiz. rule at the header
\pagestyle{empty}

\grechangedim{spaceabovelines}{0.2cm}{scalable}%

\begin{center}
{\LARGE ORDO BAPTISMI PARVULORUM}
\end{center}

%\pars{Antiphona ad introitum} \scriptura{Cf. Gal. 6, 14; Ps. 66, 2.3.4; \textbf{E188}}
%
%\vspace{-0.6cm}
%
%\antiphona{IV}{temporalia/introitus-NosAutemp.gtex}
%
%\trIntroitus

\vfill

\setlength{\parindent}{0pt}
\renewcommand{\pars}[1]{{\large\color{red}#1}}
\renewcommand{\rubricatum}[1]{\textnormal{\textit{#1}}}

\bfseries

\pars{\textsc{Ritus recipiendi parvulum}}

\rubricatum{Celebrans primo parentes interrogat:}

Quod nomen infánti vestro imposuístis?

\rubricatum{Parentes:} {\color{red}N.}

\rubricatum{Celebrans:} Quid pétitis ab Ecclésia Dei pro {\color{red}N.}?

\rubricatum{Parentes:} Baptísmum.

\rubricatum{Celebrans:} Baptísmum pro infánte vestro peténtes,
estísne cónscii offícii, quod suscípitis, illum in fide educándi, ut, Dei
mandáta servans, Dóminum et próximum suum díligat sicut Christus nos
edócuit?

\rubricatum{Parentes:} Cónscii sumus.

\rubricatum{Ad patrinos deinde conversus, celebrans his vel similibus
verbis quærit:} Estísne parátus ad paréntes huius infántis in suo múnere
adiuvándos?

\rubricatum{Patrini:} Paráti sumus.

\rubricatum{Deinde celebrans prosequitur dicens:} {\color{red}N.}, magno gáudio commúnitas
christiána te éxcipit. In cuius nómine ego signo te signo crucis; et paréntes
tui post me eódem signo Christi Salvatóris te signábunt.

\rubricatum{Et signat parvulum in fronte, nihil dicens; postea
invitat parentes et, si opportunum videtur, patrinos, ut idem faciant.}


\pars{\textsc{Sacra verbi Dei celebratio}}

\noindent\pars{Oratio fidelium}

\rubricatum{Celebrans:} Fratres caríssimi, pro hoc párvulo, qui grátiam Baptísmi
adeptúrus est, pro paréntibus eius atque patrínis, pro ómnibus baptizátis,
Dómini nostri Iesu Christi mi\-se\-ri\-cór\-di\-am invocémus.

\rubricatum{Lector:} Ut, fulgénte divíno mystério mortis et resurrectiónis tuæ,
hunc párvulum per Baptísmum regeneráre et sanctæ Ecclésiæ aggregáre
dignéris:



\vfill
\pagebreak

\pars{Oratio exorcismi et uncio præbaptismalis}

\rubricatum{Expletis invocationibus, celebrans dicit:}

Dómine Deus omnípotens,
qui Fílium tuum unigénitum misísti,
ut hóminem, peccáti servitúte captívum,
filiórum tuórum libertáte donáres,
te humíllime pro hoc infánte deprecámur:
ut, quos scis huius mundi expertúros illécebras
et contra diáboli insídias pugnatúros,
passiónis et resurrectiónis Fílii tui virtúte
ab originális culpæ labe nunc erípias
et, eiúsdem Christi grátia munítos,
in itínere vitæ suæ sine intermissióne custódias.
Per Christum Dóminum nostrum.

\rubricatum{Omnes:} Amen.

\rubricatum{Prosequitur celebrans:}

Múniat te virtus Christi Salvatóris,
in cuius signum te óleo linímus salútis,
in eódem Christo Dómino nostro,
qui vivit et regnat in sǽcula sæculórum.

\rubricatum{Omnes:} Amen.

\rubricatum{Celebrans infantem linit in pectore oleo catechumenorum.}


\vspace{0.7cm}

\pars{\textsc{Celebratio baptismi}}

\rubricatum{Cum ad fontem pervenerint, celebrans breviter in mentem 
adstantium revocat mirabile Dei consilium,
qui voluit hominis animam et corpus per aquam sanctificare.
Quod his vel similibus verbis facere potest:}

Orémus, fratres dilectíssimi, ut Dóminus Deus omnípotens novam ex aqua et
Spíritu Sancto vitam huic párvulo largiátur.

\rubricatum{Deinde, ad fontem conversus, extra tempus paschale, celebrans
profert benedictionem sequentem:}

Deus,
qui invisíbili poténtia
per sacramentórum signa mirábilem operáris efféctum,
et creatúram aquæ multis modis præparásti,
ut Baptísmi grátiam demonstráret;
Deus, cuius Spíritus
super aquas inter ipsa mundi primórdia ferebátur,
ut iam tunc virtútem sanctificándi aquárum natúra concíperet;
Deus, qui regeneratiónis spéciem
in ipsa dilúvii effusióne signásti,
ut uníus eiusdémque eleménti mystério
et finis esset vítiis et orígo virtútum;
Deus, qui Abrahæ fílios
per mare Rubrum sicco vestígio transíre fecísti,
ut plebs, a Pharaónis servitúte liberáta,
pópulum baptizatórum præfiguráret;
Deus, cuius Fílius, in aqua Iordánis a Ioánne baptizátus,
Sancto Spíritu est inúnctus,
et, in cruce pendens,
una cum sánguine aquam de látere suo prodúxit,
ac, post resurrectiónem suam, discípulis iussit:
«Ite, docéte omnes gentes,
baptizántes eos in nómine Patris et Fílii et Spíritus Sancti»:
Réspice in fáciem Ecclésiæ tuæ,
eíque dignáre fontem Baptísmatis aperíre.
Sumat hæc aqua Unigéniti tui grátiam de Spíritu Sancto,
ut homo, ad imáginem tuam cónditus,
sacraménto Baptísmatis
a cunctis squalóribus vetustátis ablútus,
in novam infántiam
ex aqua et Spíritu Sancto resúrgere mereátur.

\rubricatum{Celebrans manu dextera tangit aquam et pergit:}

Descéndat, qǽsumus, Dómine,
in hanc plenitúdinem fontis
per Fílium tuum virtus Spíritus Sancti,
ut omnes, cum Christo consepúlti
per Baptísmum in mortem,
ad vitam cum ipso resúrgant.
Per Christum Dominum nostrum.

\rubricatum{Omnes:} Amen.

\pars{Abrenuntiatio et professio fidei}

\rubricatum{Celebrans parentes et patrinos his verbis admonet:}

Dilectíssimi paréntes et patríni: per sacraméntum Baptísmi párvulus a vobis
oblátus novam a caritáte Dei vitam acceptúrus est ex aqua et Spíritu Sancto. 
Vos autem ita eos in fide educáre studeátis, ut vita illa divína a peccáti
contagióne præservétur atque de die in diem in ipsis possit augéri.

Si ergo, fide vestra ducti, paráti estis ad hoc munus suscipiéndum, Baptísmi
vestri mémores, peccáto abrenuntiáte et in Christum Iesum profitémini
fidem, quæ est fides Ecclésiæ, in qua párvuli baptizántur.

\rubricatum{Postea eosdem interrogat:} Abrenuntiátis peccáto, ut in libertáte filiórurn Dei vivátis?

\rubricatum{Parentes et patrini:} Abrenúntio.

\rubricatum{Celebrans:} Abrenuntiátis seductiónibus iniquitátis, ne peccátum vobis dominétur?

\rubricatum{Parentes et patrini:} Abrenúntio.

\rubricatum{Celebrans:} Abrenuntiátis Sátanæ, qui est auctor et princeps peccáti?

\rubricatum{Parentes et patrini:} Abrenúntio.

\rubricatum{Deinde celebrans triplicem professionem fidei a parentibus et patrinis exquirit, dicens:}

Créditis in Deum Patrem omnipoténtem, creatórem cæli et terræ?

\rubricatum{Parentes et patrini:} Credo.

\rubricatum{Celebrans:} Créditis in Iesum Christum, Fílium eius únicum, Dóminum nostrum, natum ex
María Vírgine, passum et sepúltum, qui a mórtuis resurréxit et sedet ad déxteram Patris?

\rubricatum{Parentes et patrini:} Credo.

\rubricatum{Celebrans:} Créditis in Spíritum Sanctum, sanctam Ecclésiam cathólicam,
sanctórum communiónem, remissiónem peccatórum, carnis resurrectiónem et vitam ætérnam?

\rubricatum{Parentes et patrini:} Credo.

\rubricatum{Celebrans, dicens:} Hæc est fides nostra. Hæc est fides Ecclésisæ, quam profitéri gloriámur,
in Christo Iesu Dómino nostro.

\rubricatum{Omnes:} Amen.

\pars{Baptismus}

\rubricatum{Celebrans primam familiam invitat, ut accedat ad fontem. Expresso autem nomine infantis, parentes et patrinos
interrogat:}

Vultis ígitur ut {\color{red}N.} in fide Ecclésiæ, quam vobíscum omnes modo proféssi sumus, Baptísmum recípiat?

\rubricatum{Parentes et patrini:} Vólumus.

\rubricatum{Et statim celebrans baptizat infantem, dicens:}

{\color{red}N.}, ego te baptízo in nómine Patris,

\rubricatum{immergit infantem vel infundit aquam primo}

et Fílii,

\rubricatum{immergit illum vel infundit aquam secundo}

et Spíritus Sancti.

\rubricatum{immergit illum vel infundit aquam tertio.}


\vfill
\pagebreak

\pars{\textsc{Ritus explanativi}}

\rubricatum{Deinde celebrans dicit:}

Deus omnípotens, Pater Dómini nostri Iesu Christi, qui te a peccáto liberávit et regenerávit ex aqua et
Spíritu Sancto, ipse te linit chrísmate salútis, ut, eius aggregátus pópulo, Christi sacerdótis,
prophétæ et regis membrum permáneas in vitam ætérnam.

\rubricatum{Omnes:} Amen.

\rubricatum{Postea celebrans infantem sacro chrismate in vertice capitis linit, nihil dicens.}

\pars{Impositio vestis candidæ}

\rubricatum{Celebrans dicit:} {\color{red}N.}, nova creatúra
\ifparvulus
factus
\else
facta
\fi
es et Christum induísti.
Vestis hæc cándida sit tibi signum dignitátis, quam, tuórum verbo et exémplo propinquórum
\ifparvulus
adiútus,
\else
adiúta,
\fi
immaculátam pérferas in vitam ætérnam.

\rubricatum{Omnes:} Amen.

\rubricatum{Et imponitur infanti vestis alba.}

\pars{Traditio cerei accensi}

\rubricatum{Postea celebrans accipit cereum paschalem et dicit:} Lumen Christi accípe.

\rubricatum{Unus (ex. gr. pater vel patrinus) e cereo paschali cereum infantis accendit.}

\rubricatum{Postea celebrans dicit:} Vobis, paréntibus et patrínis, lumen hoc concréditur fovéndum,
ut
\ifparvulus
párvulus iste, a Christo illuminátus, tamquam fílius
\else
párvula ista, a Christo illumináta, tamquam fília
\fi
lucis indesinénter ámbulet et, in fide persevérans, adveniénti Dómino occúrrere váleat
cum ómnibus Sanctis in aula cælésti.



\pars{\textsc{Conclusio ritus}}

\pars{Benedictio et dimissio}

\textit{Deinde celebrans benedicit matrem, infantem suum in brachiis
tenentem, patrem et omnes adstantes dicens:}

\textit{Celebrans:} Dóminus Deus omnípotens, qui per Fílium suum natum ex María Vírgine christiánas
lætíficat matres ætérnæ spe vitæ, quæ suis affúlget infántibus, dignétur matrem huius
benedícere infántis, ut, quæ de sóbole grátias nunc agit accépta, perpétuo cum ipsa
in gratiárum máneat actióne, in Christo Iesu Dómino nostro.

\textit{Omnes:} Amen.

\textit{Celebrans:} Dóminus Deus omnípotens, qui vitam terrénam largítur et cæléstem, patrem huius infántis
benedícat, ut, una cum coniúge sua, verbo et exémplo proli priórem se fídei testes exhíbeat,
in Christo Iesu Dómino nostro.

\textit{Omnes:} Amen.

\textit{Celebrans:} Dóminus Deus omnípotens, qui nos ex aqua et Spíritu Sancto in vitam regenerávit
ætérnam, hos fidéles suos muníficus benedícat, ut semper et ubíque vívida sint membra pópuli sui,
et pacem suam ómnibus hic præséntibus largiátur, in Christo Iesu Dómino nostro.

\textit{Omnes:} Amen.

\textit{Celebrans:} Benedícat vos omnípotens Deus,
Pater, et Fílius \grecross{} et Spíritus Sanctus.

\textit{Omnes:} Amen.


\vfill

\end{document}
