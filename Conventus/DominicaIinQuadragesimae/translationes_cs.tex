%%%% Preklady jednotlivych zpevu (nektere se opakuji, a je dobre mit je
% vsechny na jedne hromade)

\newcommand{\trMatLecI}{\translatioCantus{
A~protože jsme jeho spolupracovníky, také vás napomínáme, abyste nepřijímali milost Boží nadarmo. 
On totiž říká: V~příhodnou chvíli jsem tě vyslyšel; v~den spásy jsem tě podpořil.
Hle, nyní je ta příhodná chvíle, hle nyní je ten den spásy.
Nezavdáváme nikomu nijaký důvod k~pohoršení, aby služba nebyla pohaněna.
Ale ve všem se prokazujeme jako Boží služebníci: velkou stálostí v~souženích, v~nesnázích, v~úzkostech,
pod ranami, v~žalářích, v~nepokojích, v~námahách, v~bděních, v~postech;
čistotou, věděním, trpělivostí, dobrotou, svatým duchem, nepředstíranou láskou,
slovem pravdy, Boží mocí; útočnými i~obrannými zbraněmi spravedlnosti; 
ve cti i~v~potupě, ve špatné i~v~dobré pověsti; jako ti, kteří jsou pokládáni za podvodníky, a~přesto pravdomluvní;
za neznámé, třebaže my jsme velice známí; za umírající, zatímco jsme přece naživu; za trestané, kteří však nejsou vydáváni na smrt; 
za smutné, zatímco my se stále radujeme; za chudé, zatímco my tolik lidí obohacujeme; za ty, kdo nic nemají, zatímco nám patří všechno.}}

\newcommand{\trMatLecII}{\translatioCantus{
Mluvili jsme k~vám naprosto volně, Korinťané; naše srdce se otevřelo dokořán. 
U~nás stísnění nejste; stísněni jste ve svých srdcích.
Oplácejte nám tedy stejným; mluvím k~vám jako ke svým dětem, otevřte i~vy své srdce dokořán.
Nespojujte se v~nesourodé spřežení s~nevěřícími.
Co má vlastně společného spravedlnost s~bezbožností?
Co spojuje světlo s~tmou? 
Jaká je shoda mezi Kristem a~Beliarem?
Jaké sdružování mezi věřícím a~nevěřícím? 
Jaký soulad mezi Božím chrámem a~modlami?
A~právě my jsme chrámem živého Boha, jak to Bůh řekl: 
Budu přebývat uprostřed nich, mezi nimi budu chodit; budu jejich Bůh a~oni budou můj lid.}}

\newcommand{\trMatLecIII}{\translatioCantus{
Jsem vrchovatě naplněn útěchou; ve veškerém našem soužení překypuji radostí. 
Je pravda, že když jsme přišli do Makedonie, nezakusilo naše tělo odpočinku.
Všude samá soužení: navenek boje, uvnitř strach. 
Leč ten, který utěšuje ponížené, Bůh, utěšil Titovým příchodem nás,
a~nejen jeho příchodem, ale zároveň též tou útěchou, kterou jste mu dali vy sami.
On nás zpravil o~vaší horoucí touze, o~vašem hoři, o~vašem zanícení pro mne, takže ve mně radost převládla.
Vskutku, pokud jsem vás zarmoutil svým listem nelituji toho.
A~pokud jsem toho litoval -- vidím, že vás ten list zarmoutil, třebaže jen na okamžik --, nyní se z~toho raduji, ne proto, že jste byli zarmouceni, ale proto, že vás ten zármutek přivedl k~pokání.}}

\newcommand{\trMatLecIV}{\translatioCantus{
Když vám mám, drazí bratři, kázat o velkém svatém postě, jak lépe bych měl
začít, než od apoštolových slov, skrze něž promlouvá Kristus.  Zopakuji
tedy, co bylo již přečteno: „Hle, nyní je čas příhodný, hle, nyní je čas
spásy.“ Třebaže není žádného času, který by nebyl naplněn Božími dary
a~k~milo\-sr\-den\-ství Božímu se nám prostírá cesta vždy, právě nyní, když nás
opětovný návrat dne, kdy jsme byli vykoupeni, vybízí ke skutkům zbožnosti,
jsme znovu vybízeni s~velikou pílí pozdvihnout každou mysl k~duchovním věcem
a~povzbudit ji k větší důvěře, abychom nade vše vyvýšený svátek utrpení
Páně oslavili s~očištěným tělem i~duchem.}}

\newcommand{\trMatLecV}{\translatioCantus{
Těmto mystériím by se ovšem slušelo věnovat neustálou pozornost a~úctu,
abychom zůstali v~očích Páně takoví, jakými nás chce shledat v~onen
velikonoční den. Ale poněvadž k~tomu mají sílu jen nemnozí a~pro slabost
těla přísná zdrženlivost polevuje a~návyk k~ní ochabuje v~obcování se
světem, je nutno srdce zbožných očistit od pozemského prachu. Bylo tedy
dobrotivým Božím ustanovením, že je nám dopřáno pro navrácení čistoty
duši takto cvičit po čtyřicet dní, během nichž budou vykoupena přestoupení
z~jiných dob a~propuká čistý půst.}}

\newcommand{\trMatLecVI}{\translatioCantus{
Když tedy máme vstoupit, nejmilejší, do oněch tajemných dní ustanovených
k~očištění ducha i~těla, spěchejme splnit apoštolské nařízení a~očistit se
od každého přestupku těla i~ducha, aby, potrestavše výstupky obého, duch,
jenž má z~nařízení Božího vládnout tělu, zase obdržel svoji vládu
a~důstojnost, abychom nikomu nezavdávali příčinu pohoršení a~nebyli vystaveni
pomluvám. Kdyby se totiž rozešly mravy postících se s~ideálem dokonalé
čistoty, právem bychom musili strpět výčitky nevěřících, a~pro naši neřest
by si nestoudníci otírali svůj jazyk o~náboženství. Půst jako celek totiž
nespočívá jen ve zdrženlivosti od pokrmů, a~odpírání si jídla ani by
nepřineslo žádoucí užitek, pokud se mysl neodvrátí od špatnosti.}}

\newcommand{\trMatLecVIIa}{\translatioCantus{
Tu byl Ježíš odveden Duchem na poušť, aby byl pokoušen od ďábla. 
Čtyřicet dní a~čtyřicet nocí se postil, potom měl hlad.}}

\newcommand{\trMatLecVIIb}{\translatioCantus{
%Někteří lidé bývají na pochybách, od kterého to ducha byl Ježíš veden na poušť, protože se potom praví:
%Vzal ho ďábel do svatého města.
%A~dále: Vzal ho na horu velmi vysokou.
%Pravdivě však a~beze vší pochyby vhodně se rozhoduje, aby se věřilo, že byl zaveden na poušť Duchem Svatým:
%aby ho vedl jeho Duch tam, kde by ho mohl zlý duch pokoušet.
%Avšak hle, když se praví, že Bůh člověk byl vzat od ďábla ať na vysokou horu nebo do svatého města, mysl couvá, lidské uši se hrozí to slyšeti.
%A~přece, uvážíme-li i~jiné skutečnosti, poznáme, že to není nic neuvěřitelného.
Někteří pochybují, jaký že to duch vedl Ježíše na poušť, protože je
připojeno: „I~vzal ho ďábel do svatého města.“ a~dále: „Vzal ho na
převysokou horu.“ Ale vpravdě a~bez pochyby se přijímá a~věří, že to na
poušť vedl Svatý Duch, takže ho jeho Duch dovedl tam, kde ho nalezl
a~pokoušel duch zlý. Ale když se řekne, že bohočlověka vzal ďábel na vysokou
horu, nebo do svatého města, duše se děsí a~uši se zakrývají strachem.
Zjistíme, že však to není nemožné, když zvážíme a~připočteme k~tomu ještě
další skutky.
}}

\newcommand{\trMatLecVIII}{\translatioCantus{
%Ďábel je zajisté hlavou všech nešlechetníků a~údy této hlavy jsou všichni nešlechetní.
%Zdali nebyl údem ďáblovým Pilát?
%Zda nebyli údy ďáblovými Židé pronásledující a~vojáci křižující Krista?
%Jaký tedy div, že se nechal jím samým vésti na horu, když se nechal od jeho údů i~ukřižovati?
%Není tedy nedůstojno našeho Vykupitele, že chtěl býti pokoušen, když přišel býti zabit.
%Bylo zajisté spravedlivé, aby tak přemohl svým pokušením i~naše pokušení, jako přišel přemoci naši smrt smrtí svou.
Jistě je ďábel hlavou všech bezbožných a~bezbožníci jsou jeho údy.
Nebyl snad údem ďáblovým Pilát?
A~nabyly údy ďáblovými zavilí židé a vojáci, kteří ukřižovali Krista? Není
to tedy žádný div, že se dal na horu vyvést jím, když se dal jeho údy
ukřižovat. Není tedy nehodno našeho Vykupitele dát se pokoušet, když se
přišel dát zabít. Bylo ovšem správné, aby svým pokušením přemohl naše
pokušení, jako přišel přemoci svojí smrtí naši smrt.
}}

\newcommand{\trMatLecIX}{\translatioCantus{
%Jest nám však věděti, že pokušení působí trojím způsobem: ponuknutím, zalíbením a~souhlasem.
%A~když my jsme pokoušeni, obyčejně upadáme v~zalíbení, ba i~v~souhlas;
%neboť jsouce zplozeni z~hříchu těla, neseme též sami v~sobě to, pro co snášíme boje.
%Avšak Bůh, jenž vtěliv se v~lůně Panny, přišel na svět bez hříchu, nenesl v~sobě žádný rozpor.
%Mohl tedy býti pokoušen ponoukáním, avšak zalíbení hříchu neuštklo jeho mysl.
%A~tak bylo všechno to ďáblovo pokoušení vně, nikoli uvnitř.
Je třeba ale vědět, že se pokušení děje trojím způsobem - nabízením,
zalíbením a svolením. A když jsem pokoušeni my, často upadneme v~zalíbení,
nebo až k souhlasu, neboť jsme zrozeni skrze tělesný hřích a~máme tak sami
v~sobě sklon v~boji polevovat. Ale Bůh, jenž se vtělil v~lůno Panny, přišel
na svět bez hříchu a~nestrpěl v~sobě takový rozpor. Mohl být pokoušen
nabídkou, ale jeho mysl nezasáhlo zalíbení v~hříchu. A~proto veškeré toto
ďáblovo pokoušení zůstalo mimo něj, nevstoupilo dovnitř.
}}

\newcommand{\trMatEvangelium}{\translatioCantus{
Tu byl Ježíš odveden Duchem na poušť, aby byl pokoušen od ďábla. 
Čtyřicet dní a~čtyřicet nocí se postil, potom měl hlad. 
A~přistoupil k~němu pokušitel a~řekl mu: ,,Jsi-li Syn Boží, řekni, ať se z~těch kamenů stanou chleby.`` 
On však odpověděl: ,,Je psáno: Netoliko chlebem bude živ člověk, ale každým slovem, jež vychází z~Božích úst.`` 
Tu ho ďábel bere s~sebou do Svatého města a~postavil ho na vrchol Chrámu a~říká mu: ,,Jsi-li Syn Boží, vrhni se dolů; neboť je psáno:
Svým andělům dá příkazy o~tobě a~oni tě ponesou na rukou, abys nenarazil nohou o~kámen.``
Ježíš mu řekl: ,,Také je psáno: Nebudeš pokoušet Pána, svého Boha.``
Dále ho ďábel s~sebou bere na převysokou horu, ukazuje mu všechna království světa i~jejich nádheru a~řekl mu:
,,To vše ti dám, padneš-li na tvář a~budeš se mi klanět.``
Tu mu Ježíš říká: ,,Odstup, Satane! Neboť je psáno: Pánu, svému Bohu, se budeš klanět a~Jeho jediného budeš uctívat.}}
