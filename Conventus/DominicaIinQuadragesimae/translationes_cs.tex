%%%% Preklady jednotlivych zpevu (nektere se opakuji, a je dobre mit je
% vsechny na jedne hromade)

\newcommand{\trMatLecI}{\translatioCantus{
A~protože jsme jeho spolupracovníky, také vás napomínáme, abyste nepřijímali milost Boží nadarmo. 
On totiž říká: V~příhodnou chvíli jsem tě vyslyšel; v~den spásy jsem tě podpořil.
Hle, nyní je ta příhodná chvíle, hle nyní je ten den spásy.
Nezavdáváme nikomu nijaký důvod k~pohoršení, aby služba nebyla pohaněna.
Ale ve všem se prokazujeme jako Boží služebníci: velkou stálostí v~souženích, v~nesnázích, v~úzkostech,
pod ranami, v~žalářích, v~nepokojích, v~námahách, v~bděních, v~postech;
čistotou, věděním, trpělivostí, dobrotou, svatým duchem, nepředstíranou láskou,
slovem pravdy, Boží mocí; útočnými i~obrannými zbraněmi spravedlnosti; 
ve cti i~v~potupě, ve špatné i~v~dobré pověsti; jako ti, kteří jsou pokládáni za podvodníky, a~přesto pravdomluvní;
za neznámé, třebaže my jsme velice známí; za umírající, zatímco jsme přece naživu; za trestané, kteří však nejsou vydáváni na smrt; 
za smutné, zatímco my se stále radujeme; za chudé, zatímco my tolik lidí obohacujeme; za ty, kdo nic nemají, zatímco nám patří všechno.
}}

\newcommand{\trMatLecII}{\translatioCantus{
Mluvili jsme k~vám naprosto volně, Korinťané; naše srdce se otevřelo dokořán. 
U~nás stísnění nejste; stísněni jste ve svých srdcích.
Oplácejte nám tedy stejným; mluvím k~vám jako ke svým dětem, otevřte i~vy své srdce dokořán.
Nespojujte se v~nesourodé spřežení s~nevěřícími.
Co má vlastně společného spravedlnost s~bezbožností?
Co spojuje světlo s~tmou? 
Jaká je shoda mezi Kristem a~Beliarem?
Jaké sdružování mezi věřícím a~nevěřícím? 
Jaký soulad mezi Božím chrámem a~modlami?
A~právě my jsme chrámem živého Boha, jak to Bůh řekl: 
Budu přebývat uprostřed nich, mezi nimi budu chodit; budu jejich Bůh a~oni budou můj lid.
Vyjděte tedy z~jejich středu a~oddělte se, říká Pán. Nedotýkejte se ničeho nečistého, a~já vás přijmu.
Budu vám Otcem a~vy mně budete syny a~dcerami, říká všemohoucí Pán.
}}

\newcommand{\trMatLecIII}{\translatioCantus{
Mám ve vás velkou důvěru, jsem na vás velice hrdý.
Jsem vrchovatě naplněn útěchou; ve veškerém našem soužení překypuji radostí. 
Je pravda, že když jsme přišli do Makedonie, nezakusilo naše tělo odpočinku.
Všude samá soužení: navenek boje, uvnitř strach. 
Leč ten, který utěšuje ponížené, Bůh, utěšil Titovým příchodem nás,
a~nejen jeho příchodem, ale zároveň též tou útěchou, kterou jste mu dali vy sami.
On nás zpravil o~vaší horoucí touze, o~vašem hoři, o~vašem zanícení pro mne, takže ve mně radost převládla.
Vskutku, pokud jsem vás zarmoutil svým listem nelituji toho.
A~pokud jsem toho litoval -- vidím, že vás ten list zarmoutil, třebaže jen na okamžik --, nyní se z~toho raduji, ne proto, že jste byli zarmouceni, ale proto, že vás ten zármutek přivedl k~pokání.
Vždyť jste byli zarmouceni ve shodě s~Bohem, takže od nás jste neutrpěli žádnou škodu.
}}

\newcommand{\trMatLecIV}{\translatioCantus{
}}

\newcommand{\trMatLecV}{\translatioCantus{
}}

\newcommand{\trMatLecVI}{\translatioCantus{
}}

\newcommand{\trMatLecVIIa}{\translatioCantus{
Tu byl Ježíš odveden Duchem na poušť, aby byl pokoušen od ďábla. 
Čtyřicet dní a~čtyřicet nocí se postil, potom měl hlad. 
}}

\newcommand{\trMatLecVIIb}{\translatioCantus{
}}

\newcommand{\trMatLecVIII}{\translatioCantus{
}}

\newcommand{\trMatLecIX}{\translatioCantus{
}}

\newcommand{\trMatEvangelium}{\translatioCantus{
Tu byl Ježíš odveden Duchem na poušť, aby byl pokoušen od ďábla. 
Čtyřicet dní a~čtyřicet nocí se postil, potom měl hlad. 
A~přistoupil k~němu pokušitel a~řekl mu: ,,Jsi-li Syn Boží, řekni, ať se z~těch kamenů stanou chleby.`` 
On však odpověděl: ,,Je psáno: Netoliko chlebem bude živ člověk, ale každým slovem, jež vychází z~Božích úst.`` 
Tu ho ďábel bere s~sebou do Svatého města a~postavil ho na vrchol Chrámu a~říká mu: ,,Jsi-li Syn Boží, vrhni se dolů; neboť je psáno:
Svým andělům dá příkazy o~tobě a~oni tě ponesou na rukou, abys nenarazil nohou o~kámen.``
Ježíš mu řekl: ,,Také je psáno: Nebudeš pokoušet Pána, svého Boha.``
Dále ho ďábel s~sebou bere na převysokou horu, ukazuje mu všechna království světa i~jejich nádheru a~řekl mu:
,,To vše ti dám, padneš-li na tvář a~budeš se mi klanět.``
Tu mu Ježíš říká: ,,Odstup, Satane! Neboť je psáno: Pánu, svému Bohu, se budeš klanět a~Jeho jediného budeš uctívat.
}}
