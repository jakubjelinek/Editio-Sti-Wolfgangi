%%%% Preklady jednotlivych zpevu (nektere se opakuji, a je dobre mit je
% vsechny na jedne hromade)

\newcommand{\trOratioAnteOfficium}{\translatioCantus{Otevři, Pane, má ústa, abych chválil tvé svaté jméno.
Očisti mé srdce od všech marnivých, zvrácených a~jiných myšlenek, osvěť rozum, rozněť cit,
abych mohl důstojně, soustředěně a~zbožně recitovat a~vysloužil si být
vyslyšen před tváří tvé velebnosti. Skrze Krista…}}

\newcommand{\trOratioPostOfficium}{\translatioCantus{\textit{Následující modlitbu
opatřil pro ty, kdo ji zbožně vyřknou po hodinkách, papež Lev X.
odpustky za nedostatky a~provinění vzniklé při konání hodinek z~lidské křehkosti. Říká se
vkleče.}
Svatosvaté a~nerozdílné Trojici, ukřižovanému lidství našeho Pána Ježíše
Krista, přeblažené a~přeslavné plodné neporušenosti vždy Panny Marie
i~souhrnu všech svatých buď ode všeho stvoření věčná chvála, čest a~sláva, nám
pak buď dáno odpuštění všech hříchů, po nekonečné věky věků. Amen.}}

% HOURS ---

\newcommand{\trVespAntI}{\translatioCantus{Před Hospodinem smím chodit~\grestar{}
na zemi mezi živými.}}

\newcommand{\trVespAntII}{\translatioCantus{Běda mně,~\grestar{} pobyt můj v~cizině se
prodloužil!}}

\newcommand{\trVespAntIII}{\translatioCantus{Hospodin~\grestar{} střeží tě všeho
zlého, střeží tvůj život Hospodin.}}

\newcommand{\trVespAntIV}{\translatioCantus{Budeš-li, Pane, hříchy mít na
zřeteli, Hospodine, kdož obstojí?}}

\newcommand{\trVespAntV}{\translatioCantus{Díly~\grestar{} svých rukou, Pane, nepohrdneš!}}

\newcommand{\trRespVesp}{\translatioCantus{Ústa spravedlivého~\grestar{}
šeptají moudrost. \Vbardot{} A~jeho jazyk ohlašuje právo.}}

\newcommand{\trRespLaud}{\translatioCantus{Spravedlivého vodil Hospodin~\grestar{}
po přímých stezkách. \Vbardot{} A~ukázal mu Boží království.}}

\newcommand{\trVersusAudivi}{\translatioCantus{\Vbardot{} Potom jsem uslyšel, jak
mi jakýsi hlas z~nebe říká. ~\Rbardot{} Blahoslavení mrtví, kteří umírají v~Pánu.}}

\newcommand{\trAntMagnificat}{\translatioCantus{Všechno,~\grestar{} co mi dává Otec,
přijde ke mně, a~toho, kdo ke mně přichází, nevyvrhnu ven.}}

\newcommand{\trOratioI}{\translatioCantus{Bože, Stvořiteli a~Vykupiteli
všech věřících: uděl duším všech svých služebníků a~služebnic odpuštění hříchů,~\gredagger{}
aby se jim dosáhli slitování,~\grestar{} jež si vždy přáli. Skrze…}}

\newcommand{\trOratioII}{\translatioCantus{Vysvoboď, prosíme, Pane, duši
tvého služebníka {\color{red}N.} {\color{red}(}tvé služebnice
{\color{red}N.}{\color{red})}, aby po
časné smrti žil{\color{red}(}a{\color{red})} v~tobě;~\gredagger{}
a~co v~lidském obcování ze slabosti spáchal{\color{red}(}a{\color{red})},
zahlaď přemírou svého dobrotivého milosrdenství. Skrze…}}

\newcommand{\trOratioIII}{\translatioCantus{Bože, jemuž je vlastní smilování
a~shovívavost, pokorně tě vzýváme za duši tvého služebníka {\color{red}N.}
{\color{red}(}tvé služebnice {\color{red}N.}{\color{red})}, jemuž
{\color{red}(}jíž{\color{red})} jsi dnes poručil odejít z~tohoto světa:~\gredagger{}
Nevydávej ji do rukou nepříteli a~pamatuj na ni; nařiď andělům, aby ji
přijali a~dovedli do rajské vlasti,~\grestar{}
aby jí nebylo trpět pekelné muky, ale
obdržela věčnou radost, neboť v~tebe doufala a~věřila. Skrze…}}

\newcommand{\trOratioIV}{\translatioCantus{Prosíme, Pane, abys duši tvého
služebníka {\color{red}N.} {\color{red}(}tvé služebnice
{\color{red}N.}{\color{red})}, od je{\color{red}(}jí{\color{red})}hož
pohřbu si připomínáme již tři dny {\color{red}(}{\color{red}nebo} sedm dní
{\color{red}nebo} třicet dní{\color{red})},~\gredagger{} přidružil ke svým
svatým a~vyvoleným~\grestar{} a~svlažil ji rosou tvého věčného
milosrdenství. Skrze…}}

\newcommand{\trOratioV}{\translatioCantus{Bože, Pane slitování,~\gredagger{}
dej duši svého služebníka {\color{red}N.} {\color{red}(}své služebnice
{\color{red}N.} {\color{red}nebo} svých služebníků
a~služebnic{\color{red})}, výročí jehož
{\color{red}(}jejíhož {\color{red}nebo} jejichž{\color{red})} pohřbu si připomínáme,~\grestar{}
usednout a~spočinout v~blaženém míru a~v~jase světla. Skrze…}}

\newcommand{\trOratioVI}{\translatioCantus{Bože, jenž jsi ve svém
nevýslovném úradku chtěl zařadit svého služebníka {\color{red}N.}
mezi své velekněze,~\gredagger{} uděl, prosíme, aby byl připojen k~věčnému sboru tvých
kněží ten, jenž vykonává na zemi pravomoc tvého jednorozeného Syna, skrze
něhož…}}

\newcommand{\trOratioVII}{\translatioCantus{Bože, jenž jsi dopřál svému
služebníku {\color{red}N.} {\color{red}(}svým služebníkům
{\color{red}N.} a~{\color{red}N.}{\color{red})} hodnosti
apoštolské a~kněžské,~\gredagger{} uděl, prosíme,~\grestar{} aby i~na věčnosti
byl{\color{red}(}i{\color{red})} připočten{\color{red}(}i{\color{red})}
k~jejich řádu. Skrze…}}

\newcommand{\trOratioVIII}{\translatioCantus{Bože, jenž jsi dopřál svému
služebníku {\color{red}N.} {\color{red}(}svým služebníkům {\color{red}N.}
a~{\color{red}N.}{\color{red})} hodnosti kněžské,~\gredagger{} uděl,
prosíme,~\grestar{} aby i~na věčnosti byl{\color{red}(}i{\color{red})}
připočten{\color{red}(}i{\color{red})} k~jejich řádu. Skrze…}}

\newcommand{\trOratioIX}{\translatioCantus{Uděl, Pane, prosíme,~\gredagger{}
aby duše tvého služebníka, kněze {\color{red}N.}, kterého jsi za jeho
pobytu v~této časnosti ozdobil svými dary,~\grestar{} se navěky radovala v~nebes
slavném sídle. Skrze…}}

\newcommand{\trOratioX}{\translatioCantus{Nakloň, Pane, své ucho našim
prosbám, jimiž si pokorně vyprošujeme milost,~\gredagger{} abys duši svého
služebníka {\color{red}N.}, jíž jsi poručil odejít z~tohoto časného
světa,~\grestar{} zavedl do místa světla a~míru. Skrze…}}

\newcommand{\trOratioXI}{\translatioCantus{Prosíme, Pane, pro své
milosrdenství se slituj nad svou služebnicí {\color{red}N.},~\gredagger{} jež
vysvlekla poskvrnu smrtelnosti~\grestar{} a~uveď ji do věčné spásy. Skrze…}}

\newcommand{\trOratioXII}{\translatioCantus{Bože, jenž jsi štědrý ve svém
milosrdenství a~toužíš po spáse lidí,~\gredagger{} žádáme
si tvé milosti, abys bratřím, příbuzným a~dobrodincům našeho společenství,
kteří odešli z~tohoto světa,~\grestar{} dal na přímluvu blahoslavené Panny Marie
a~všech svatých přijít do společenství věčné blaženosti. Skrze…}}

\newcommand{\trOratioXIII}{\translatioCantus{Bože, jemuž je vlastní
smilování a~shovívavost,~\gredagger{} smiluj se nad dušemi svých
služebníků a~služebnic a~odpusť jim všechny viny,~\grestar{} aby osvobozeni
z~pout smrtelnosti si zasloužili přijít k~Tobě. Skrze…}}

\newcommand{\trOratioXIV}{\translatioCantus{Duším svých služebníků
a~služebnic, uděl, prosíme, Pane, věčné milosrdenství, aby u~nich
setrvalo na věky a~pomohlo jim, že v~tebe doufali a~věřili. Skrze…}}

\newcommand{\trOratioXV}{\translatioCantus{Bože, jenž jsi nám nakázal ctít
matku i~otce,~\gredagger{} smiluj se laskavě nad dušemi mé matky a~mého otce a~odpusť
jim hříchy~\grestar{} a~dopřej mně též je spatřit v~radosti a~záři věčnosti. Skrze…}}

\newcommand{\trLaudAntI}{\translatioCantus{Zaplesají Hospodinu~\grestar{}
kosti ponížené.}}

\newcommand{\trLaudAntII}{\translatioCantus{Vyslyš, Hospodine,~\grestar{}
mou modlitbu, k~tobě přichází každý člověk.}}

\newcommand{\trLaudAntIII}{\translatioCantus{Drží mne~\grestar{} tvoje
pravice, Hospodine.}}

\newcommand{\trLaudAntIV}{\translatioCantus{Od bran šeolu~\grestar{}
uchraň, Hospodine, mou duši.}}

\newcommand{\trLaudAntV}{\translatioCantus{Vše, co dýchá,~\grestar{}
chval Hospodina!}}

\newcommand{\trAntBenedictus}{\translatioCantus{Já jsem~\grestar{}
vzkříšení a~život. Kdo ve mne věří, i~kdyby zemřel, bude žít
a~každý, kdo žije a~věří ve mne, nezemře navěky.}}

% Matutinum

\newcommand{\trMatInvitatorium}{\translatioCantus{Krále vzývejme, před nímž vše živé je.}}

\newcommand{\trMatVeniteA}{\translatioCantus{Pojďte, chvalme s~radostí Pána,
s~jásotem slavme Boha, svou spásu; předstupme před tvář jeho s~díky, písně plesu pějme jemu.}}

\newcommand{\trMatVeniteB}{\translatioCantus{Neboť Bůh veliký jest Hospodin, a~král nade všecky bohy.
Jsouť v~jeho ruce všecky hlubiny země, temena hor jsou majetek jeho.}}

\newcommand{\trMatVeniteC}{\translatioCantus{Jehoť jest moře, neb on je učinil; i~souš
je dílo jeho rukou. Pojďme, klanějme se, padněme, klekněme před Pánem, svým
tvůrcem. Jeť on Pán, náš Bůh, a~my jsme lid, jejž on vodí a~ovce, jež pase.}}

\newcommand{\trMatVeniteD}{\translatioCantus{Kéž byste poslechli dnes hlasu jeho:
,,Nezatvrzujte svých srdcí jak v~Hádce, jak v~Pokušení na poušti, kde vaši otcové pokoušeli mne,
zkoušeli mne, ač vídali skutky mé.``}}

\newcommand{\trMatVeniteE}{\translatioCantus{Čtyřicet roků mrzel jsem se na to pokolení
a~řekl jsem: ,,Lid je to myslí stále bloudící``! Oni však nechtěli znáti mé cesty, takže jsem
přisáhl ve svém hněvu: ,,Nedojdou odpočinku mého!\mbox{}``}}

\newcommand{\trMatAntI}{\translatioCantus{Urovnej,~\grestar{} Pane, můj Bože, před sebou mou cestu.}}

\newcommand{\trMatAntII}{\translatioCantus{Obrať se,~\grestar{} Pane, a~vytrhni život
můj, neboť, kdo ve smrti je tebe pamětliv.}}

\newcommand{\trMatAntIII}{\translatioCantus{Aby jak lvi mne neroztrhali,
když není, kdo by mne vytrhl, zachránil.}}

\newcommand{\trMatVersusI}{\translatioCantus{\Vbardot{} Od bran šeolu.
\Rbardot{} Uchraň, Hospodine, jejich duše.}}

\newcommand{\trMatLecI}{\translatioCantus{Stravuji se, nebudu žít pořád;
a~tak mě nech, můj život je pouhé dechnutí! Co tedy je člověk, že mu
přikládáš takovou váhu, že na něho upínáš svou pozornost,
že na něho každého rána dohlížíš, že ho každým okamžikem zkoumáš?
Přestaneš se na mne konečně dívat, abych měl čas polknout slinu?
Pokud jsem zhřešil, co jsem tím udělal tobě, ty bedlivý pozorovateli člověka?
Proč sis mě vzal za terč, proč jsem ti na obtíž?
Nemůžeš ode mne strpět urážku, přejít mou vinu? Vždyť brzy
budu ležet v~prachu, budeš mě hledat a~já už nebudu.}}

\newcommand{\trMatRespI}{\translatioCantus{Věřím,~\grestar{}
že můj Vykupitel žije, že on jako poslední povstane nad prachem.~\gredagger{}
A~ve svém těle uzřím Boha, Spasitele svého. \Vbardot{}
Ten, jehož uvidím, bude na mé straně; ten, na nějž budou hledět mé oči,
nebude cizinec.}}

\newcommand{\trMatLecII}{\translatioCantus{Protože se mi oškliví život, dám
volný průchod svému nářku, vyleji hořkost své duše.
Řeknu Bohu: Neodsuzuj mě, prozraď mi, proč mi přičítáš vinu.
Dělá ti dobře, že mi činíš násilí, že pokořuješ dílo svých
rukou a~že podporuješ záměry zlovolných?}}

\newcommand{\trMatLecIIa}{\translatioCantus{Ten člověk zrozený
z~ženy, jenž má život krátký, ale trápení do sytosti.
Podobá se květu, rozkvétá, pak uvadá, bez ustání prchá jako stín.
A~tuto bytost nikdy nespouštíš z~očí, přivádíš ji před sebe na soud!
Kdo však vytěží čisté z~nečistého? Nikdo!
Poněvadž jeho dny jsou sečteny a~počet jeho měsíců závisí od tebe a~ty mu
určuješ nepřekročitelnou mez, odvrať od něho oči a~nechej ho, ať jako
nádeník skončí svůj den.}}

\newcommand{\trMatRespII}{\translatioCantus{Páchnoucího Lazara jsi z~hrobu
vzkřísil,~\gredagger{} avšak jim, Pane, dopřej pokoje
a~místa potěšení. \Vbardot{} Vždyť přijdeš soudit živé i~mrtvé i~celý věk ohněm.}}

\newcommand{\trMatLecIII}{\translatioCantus{Tvé ruce mě ztvárnily, utvořily;
pak sis to rozmyslel a~chtěl bys mě zničit!
Vzpomeň si: udělal jsi mě, jako se hněte hlína, a~pošleš mě nazpět do prachu.
Což jsi mě neslil jako mléko a~nenechal srazit jako sýr,
neoblékl do kůže a~masa, neutkal z~kostí a~šlach?
Pak jsi mě obdařil životem a~starostlivě jsi bděl nad mým dechem.}}

\newcommand{\trMatLecIIIa}{\translatioCantus{Mé maso pod kůží propadá
hnilobě a~kosti se mi obnažují jako zuby.
Slitujte se, slitujte se nade mnou, přátelé moji! Neboť mě zasáhla Boží ruka.
Proč se na mne vrháte jako sám Bůh, aniž se nasytíte mým masem?
Ach! Přál bych si, aby má slova byla sepsána, aby byla vyryta jako nápis,
železným dlátem a~bodcem navěky vytesána do skály!
Já vím, že můj Obhájce žije, že on jako poslední povstane nad prachem.
Po mém probuzení mě postaví vedle sebe a~ve svém těle uzřím Boha.
Ten, jehož uvidím, bude na mé straně; ten, na nějž budou hledět mé oči,
nebude cizinec. A~mé ledví ve mně se stravuje.}}

\newcommand{\trMatRespIII}{\translatioCantus{Kam se ukryji před Tvou tváří,
Pane, až přijdeš soudit zemi?~\gredagger{} Vždyť jsem
chybil ve svém životě. \Vbardot{} Děsím se, co jsem spáchal a~stydím se před
Tebou -- neodsuzuj mne, až přijdeš soudit. \Vbardot{}
Věčné spočinutí jim dej, Pane, a~neustále ať jim světlo svítí.}}

\newcommand{\trMatAntIV}{\translatioCantus{Kde pastvy hojnost,~\grestar{}
tam lehat smím.}}

\newcommand{\trMatAntV}{\translatioCantus{Čím jsem se v~mládí provinil,
nevzpomínej, Hospodine.}}

\newcommand{\trMatAntVI}{\translatioCantus{Věřím a~vidím~\grestar{}
dobrotu Hospodinovu v~zemi živých!}}

\newcommand{\trMatVersusII}{\translatioCantus{\Vbardot{}Posadil je mezi knížata.
\Rbardot{} Mezi knížata svého lidu.}}

\newcommand{\trMatLecIV}{\translatioCantus{Pak začni rokovat a~já odpovím;
nebo spíš budu mluvit já a~odpověď mi dáš ty.
Kolik jsem spáchal provinění a~hříchů? Řekni mi, jaký byl můj přestupek, můj hřích?
Proč ukrýváš svou tvář a~pokládáš mě za svého nepřítele?
Chceš děsit list zmítaný větrem, pronásledovat suché stéblo?
Že proti mně vynášíš hořké rozsudky a~přičítáš mi viny mládí,
žes mi dal nohy do klády, pozoruješ všechny mé stezky a~značíš si stopy mých kroků!
A~on se rozpadá jako červotočivé dřevo nebo jako šat, jejž rozežírá mol.}}

\newcommand{\trMatLecIVa}{\translatioCantus{Z knihy svatého Augustina
o~tom, jak se postarat o~mrtvé.
Uspořádat pohřeb a~zřídit hrobku, jakož i~jejich slavnostnost, to je spíše
útěchou pro živé, nežli pomocí mrtvým. Neříkám, že se mají těla mrtvých,
v~první řadě spravedlivých a~věřících, jež byla duchu jakoby jeho nádobami
a~nástroji, bez ohledu někde povrhnout. Jestliže  otcův oděv a~prsten
a~podobné věci jsou potomkům tím dražší, čím více své rodiče milovali, tím
spíše nebudeme pohrdat tělem, jež je nám ještě mnohem bližší a~více s~námi
spojeno, nežli cokoli, co nosíme. To totiž nemáme jen pro ozdobu a~jako
vnější pomůcku, ale náleží k~samé lidské přirozenosti. Vždyť i~za starých
dob bývaly slavnostně vystrojovány pohřby a~zřizovány hroby spravedlivým;
a~oni sami také to často přikazovali ještě za života svým synům, jak jejich
tělo pohřbít, či kam je přenést.}}

\newcommand{\trMatRespIV}{\translatioCantus{Vzpomeň na mě,~\grestar{} Bože,
že můj život je pouhé dechnutí.~\gredagger{} Unikám každému pohledu.
\Vbardot{} Z~hlubin volám k~tobě, Hospodine: Pane, slyš můj hlas.}}

\newcommand{\trMatLecV}{\trMatLecIIa}

\newcommand{\trMatLecVa}{\translatioCantus{Drazí věřící také vyjadřují
svoji náklonnost k~nim vzpomínkou a~mod\-lit\-bou; ty bezpochyby pomohou těm,
kteří si za svého života v~těle takový prospěch zasloužili. A~vpravdě se
nemají opomíjet prosby za duše zemřelých, i~když z~nějakého důvodu není
možno tělo pohřbít, nebo pohřbít na svatém místě, a~církev tak činí a~ujímá
se všech zemřelých z~celého křesťanského  společenství a~pamatuje na ně
všeobecnou připomínkou i~bez uvádění jmen, aby těm, jež nemají rodiče ani
děti, ani žádné známé či přátele, takto posloužila jejich společná matka.
A~myslím, že kdyby se nedostávalo těchto proseb, jež se v~pravé víře
a~zbožnosti činí za zemřelé, nic by jejim duším neprospělo pochovat jejich
tělo byť na tom nejsvětějším místě.}}

\newcommand{\trMatRespV}{\translatioCantus{Ach, běda,~\grestar{} Pane!
Tolik jsem chybil ve svém životě! Co si mám, bědný, počít?
Kam se uteku, než k~Tobě, Pane?~\gredagger{} Smiluj se nade mnou, až přijdeš
v~poslední den. \Vbardot{} Má duše je velice ztrýzněná, pomoz jí, Pane.}}

\newcommand{\trMatLecVI}{\translatioCantus{Ach! Kdybys mi poskytl přístřeší v~šeolu,
kdybys mě tam ukryl, dokud potrvá tvůj hněv, kdybys mi určil lhůtu, a~pak si na mne vzpomněl:
- vždyť, když už člověk zemře, může zase ožít? - Po všechny
dny své služby bych čekal, až by mě přišli vystřídat.
Ty bys zavolal a~já bych ti odpověděl; zase bys chtěl uzřít dílo svých rukou.
A~zatímco teď počítáš všechny mé kroky, už bys pak nečíhal na můj hřích.}}

\newcommand{\trMatLecVIa}{\translatioCantus{Když je to tedy tak,
nedomnívejme se, že se setkáme s~mrtvými, na nichž nám záleží, jestliže
za ně každý rok nebudeme prosit a~nevykonáme oběť u~oltáře, neobětujeme
modlitby nebo almužny, i~když ty neprospějí všem, za něž se
činí, ale jen těm, kteří jsou ze svého života připraveni, aby jim prospěly.
Ale poněvadž nerozpoznáme, kteří to jsou, je třeba je konat za uzdravení
všech, aby nebyl vynechán nikdo, komu mají a~mohou prospět. Lépe když se
jich dostane nadbytkem i~těm, kterým nepomohou ani neuškodí, než kdyby
scházely těm, kterým mohou prospět. A~každý to pro potřebné učiní raději,
když se mu od svých blízkých dostane téhož. Avšak to, co podnikáme,
abychom pohřbili tělo, není zárukou spásy, ale jen lidskou službou,
vyjádřením citu, poněvadž nikdo nemá v~nenávisti vlastní tělo a~je tedy
třeba postarat se i~o~tělo bližního, jak jen je možno, až odejde ten, který
se o~ně staral. Jestliže tak činí ti, kteří nevěří ve vzkříšení těla, čím
více tak musí činit ti, kteří věří, že je třeba posloužit tělu, jež je sice
mrtvé, ale vstane a~zůstane na věčnost, také na stvrzení této své víry.}}

\newcommand{\trMatRespVI}{\translatioCantus{Nevzpomínej,~\grestar{}
Pane, na mé hříchy,~\gredagger{} až přijdeš soudit věk ohněm. \Vbardot{} Zaveď
můj život, Pane a~Bože, před svoji tvář. \Vbardot{} Věčné spočinutí jim dej,
Pane, a~neustále ať jim světlo svítí.}}

\newcommand{\trMatAntVII}{\translatioCantus{Rač mne, Pane, vysvobodit,
hleď, abys mi přispěl.}}

\newcommand{\trMatAntVIII}{\translatioCantus{Uzdrav mne, Hospodine, neb jsem hřešil
proti tobě.}}

\newcommand{\trMatAntIX}{\translatioCantus{Žízní~\grestar{} duše má po Bohu živém.
Kdy přijdu, abych se ukázal před Boží tváří?}}

\newcommand{\trMatVersusIII}{\translatioCantus{\Vbardot{} Nevydávej zvěři duši své
hrdličky. \Rbardot{} Až do konce nezapomínej na život svých nešťastných.}}

\newcommand{\trMatLecVII}{\translatioCantus{Můj dech se ve mně vyčerpává a~scházejí se mí hrobaři.
Mými druhy jsou jen posměváčci, jejichž tvrdost trýzní mé probděné noci.
Polož si tedy sám před sebe mou záruku, kdo by si totiž se mnou chtěl plácnout?
Mé dny utekly i~s~mými záměry a~struny mého srdce se strhaly.
Z~noci se chce dělat den; prý už je blízko světlo zahánějící temnoty.
Mou nadějí je bydlet v~šeolu, rozprostřít si lože v~temnotách.
Volám na hrob: ,,Jsi můj otec!\mbox{}`` Na červa: ,,Ty jsi má matka a~má
sestra!\mbox{}``
Kde tedy je moje naděje? A~mé štěstí, kdo je spatří?}}

\newcommand{\trMatLecVIIa}{\translatioCantus{Hlásá-li se tedy, že Kristus
vstal z~mrtvých, jak mohou někteří mezi vámi říkat, že vzkříšení z~mrtvých není?
Není-li vzkříšení z~mrtvých, nevstal z~mrtvých ani Kristus.
Ale jestliže Kristus nevstal z~mrtvých, pak je prázdné naše poselství,
prázdná je též vaše víra.
Ukazuje se pak dokonce, že o~Bohu svědčíme falešně, protože jsme proti Bohu tvrdili,
že on vzkřísil Krista, zatímco ho nevzkřísil, je-li pravda, že mrtví nevstávají.
Vždyť nevstávají-li mrtví, ani Kristus z~mrtvých nevstal.
A~jestliže Kristus nevstal z~mrtvých, marná je vaše víra;
jste dosud ve svých hříších.
Pak také ti, kdo usnuli v~Kristu, zahynuli.
Jestliže jsme vložili svou naději v~Krista jen pro tento život
jsme nejvíc politováníhodní ze všech lidí.
Ale ne: Kristus vstal z~mrtvých jako prvotina těch, kdo usnuli.
Protože totiž smrt přišla skrze člověka, přichází skrze člověka také zmrtvýchvstání.
Jako totiž všichni umírají v~Adamovi, tak všichni budou opět oživeni v~Kristu.}}

\newcommand{\trMatRespVII}{\translatioCantus{Každý den, když jsem hřešil,
sužuje mne strach ze smrti~\gredagger{} neboť z~podsvětí není vykoupení;
smiluj se nade mnou Pane a~zachraň mne. \Vbardot{} Sám pro sebe,
Pane, mne zachraň a~svou silou mne vysvoboď.}}

\newcommand{\trMatLecVIII}{\trMatLecIIIa}

\newcommand{\trMatLecVIIIa}{\translatioCantus{Řekne se však, jak vstávají mrtví?
S~jakým tělem zase přijdou?
Blázne! Co ty seješ, nenabývá opět života, pokud to nezemře.
A~co seješ, není tělo, jež má přijít, ale pouze semeno, buď pšenice anebo nějaké jiné rostliny;
a~Bůh mu dává tělo, jak sám chtěl, každému semeni vlastní tělo.
Všechna těla nejsou stejná, leč jiné je tělo lidí, jiné tělo zvířat, jiné tělo ptáků, jiné ryb.
Jsou také nebeská tělesa a~tělesa pozemská, ale jinak září ta nebeská, jinak pozemská.
Jinak září slunce, jinak září měsíc, jinak září hvězdy. V~záření se dokonce liší hvězda od hvězdy.
Tak tomu je se vzkříšením mrtvých: zasévá se v~porušenosti, z~mrtvých se vstává v~neporušitelnosti;
zasévá se v~hanbě, z~mrtvých se vstává ve slávě; zasévá se ve slabosti, z~mrtvých se vstává v~síle;
zasévá se tělo obdařené duší, z~mrtvých vstává tělo duchovní.}}

\newcommand{\trMatRespVIII}{\translatioCantus{Nesuď mne Pane podle mých činů
-- vždyť jsem nic před tvou tváří nevykonal,
a~tak tedy prosím o~milost tvůj majestát,~\gredagger{} abys, Pane, zahladil mou nepravost.
\Vbardot{} Omyj mne od vší nepravosti a~očisť mne od mého provinění.}}

\newcommand{\trMatLecIX}{\translatioCantus{Ach! Proč jsi mi dal vyjít
z~lůna? Tehdy bych byl zahynul: žádné oko by mě neuvidělo,
byl bych, jako bych nikdy nebyl, z~břicha by mě byli přenesli do hrobu.
A~dny mého žití trvají tak krátce! Už mě tedy nepozoruj a~popřej mi trochu radosti,
než nenávratně odejdu do země temnot a~hustého stínu,
kde vládne tma a~nepořádek, kde se sám jas podobá tmavé noci.}}

\newcommand{\trMatLecIXa}{\translatioCantus{Hle, řeknu vám jedno tajemství:
nezemřeme všichni, ale všichni budeme proměněni.
V~jedné chvilce, v~jednom okamžení, za zvuku poslední polnice, neboť ta polnice zazní,
a~mrtví vstanou neporušitelní a~my budeme proměněni.
Je totiž nutné, aby ta porušitelná bytost oblékla neporušitelnost, aby ta smrtelná bytost oblékla nesmrtelnost.
Až tedy ta porušitelná bytost obleče neporušitelnost
a~až ta smrtelná bytost obleče nesmrtelnost, pak se naplní slovo, které je psáno:
\textit{Smrt byla pohlcena ve vítězství.
Kde je, smrti, tvé vítězství? Kde je, smrti, tvůj bodec?}
Bodcem smrti je hřích a~silou hříchu je Zákon.
Ale buď dík Bohu, který nám dává vítězství skrze našeho Pána Ježíše Krista!
Tak tedy, moji milovaní bratři, buďte pevní, nezviklatelní,
stále čiňte pokroky v~díle Páně vědouce, že v~Pánu vaše námaha nebude marná.}}

\newcommand{\trMatRespIX}{\translatioCantus{Vysvoboď mne, Pane,~\grestar{}
od věčné smrti v~onen strašný den,~\gredagger{} kdy se pohnou nebesa
i~zem~\ddag{} až přijdeš soudit věk ohněm. \Vbardot{} Roztřásl jsem se strachem
před rozsudkem, který přijde a~z~nadcházejícího hněvu.
\Vbardot{} Ach ten den, hněvu den! Den pohromy a~neštěstí, veliký a~hořký den.
\Vbardot{} Věčné spočinutí jim dej, Pane, a~neustále ať jim světlo svítí.}}

\newcommand{\trMatRespX}{\translatioCantus{Vysvoboď mne,~\grestar{} Pane, ze stezek
podsvětí, neboť jsi roztříštil bronzové brány a~podsvětí navštívil; a~osvítils je,
aby tě spatřili~\gredagger{} ti, co byli v~pekelných mukách. \Vbardot{}
Volali a~pravili: Tys přece přišel, náš Vykupiteli!
\Vbardot{} Věčné spočinutí jim dej, Pane, a~neustále ať jim světlo svítí.}}

% LITTLE HOURS ---

\newcommand{\trOratioPropitiare}{\translatioCantus{Odpusť, Pane,
duším všech tvých služebníků a~služebnic, za něž vzýváme
pokorně tvůj majestát, aby pomocí těchto přímluv si zasloužily vejít do
věčného spočinutí. Skrze…}}

\newcommand{\trCapituliJustus}{\translatioCantus{Spravedlivý se od rána
celým svým srdcem obrací k~Pánu, svému stvořiteli;~\grestar{}
úpěnlivě prosí před Nejvyšším.}}

\newcommand{\trCapituliJustum}{\translatioCantus{Spravedlivého vodil Hospodin po přímých stezkách,~\gredagger{}
a~ukázal mu Boží království a~dal mu poznání svatých věcí;~\grestar{}
dal mu úspěch v~jeho tvrdých pracích a~dopřál výnos jeho námaze.}}

\newcommand{\trOratioFidelium}{\translatioCantus{Bože, Stvořiteli
a~Vykupiteli všech věřících: uděl duším všech svých služebníků a~služebnic
odpuštění hříchů, aby jím dosáhli slitování, jež
si vždy přáli. Skrze…}}
