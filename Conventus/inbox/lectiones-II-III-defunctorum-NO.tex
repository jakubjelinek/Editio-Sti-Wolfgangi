\documentclass[options]{article}
\begin{document}
	Ex Epístolis sancti Brauliónis Cæsaraugustáni epíscopi
	\begin{flushright}
		(Epist. 19: PL 80, 665-666)
	\end{flushright}
Spes ómnium  credéntium Christus excedéntes a mundo dormiéntes vocat, non mórtuos, dicens;
\emph{Lázarus amícus noster dormit.}\\
Sed et sanctus Apóstolus non vult nos contristári de dormiéntibus, ac per hoc si fides nostra hoc habet, quia omnes credéntes in Christo secúndum vocem evangélicam non moriéntur in ætérnum, fide scimus quia nec ille mórtuus est, nec nos moriémur.\\
Quóniam ipse Dóminus in iussu et in voce archángeli et in tuba Dei descéndet de cælo, et mórtui qui in eo sunt resúrgent.\\
Spes ergo nos resurrectiónis ánimet, quóniam quos hic amíttimus illic revidébimus; tantum est ut in eo bene credámus, præcéptis scílicet eius paréntes, apud quem est summa virtútis suscitáre facílius mórtuos quam nobis somno déditos. Ecce ista dícimus, et tamen afféctu néscio quo in lácrimis retráhimur, et crédulam mentem desidérii frangit afféctus. Heu! miserábilis humána condício, et sine Christo vanum omne quod vívimus.\\
\\
Resp 2 Qui Lazarum (as in defunctorum-brevi)\\
\\
O mors, quas coniúnctos dívidis, et amicítia sociátos dura et crudélis dissócias! iam iam confráctæ sunt vires tuæ. Iam contrítum est ímpium iugum tuum ab illo, qui tibi per Oséæ  rugítus minabátur 
\emph{O mors, ero mors tua. Unde per Apóstolum insultámus: Ubi est, mors, victória tua? Ubi est, mors acúleus tuus?}\\
Ipse qui te vicit nos redémit, qui ánimam suam diléctam trádidit in manus impiórum, ut ex ímpiis fáceret sibi diléctos. Longum quidem est et multum quæ ad consolatiónem commúnem de divínis Scriptúris débeant replicári. Sed suffíciat nobis spes resurrectiónis et oculórum nostrórum diréctio ad glóriam nostri Redemptóris, in quo nos fide iam resurrexísse putámus, dicénte Apóstolo:
\emph{Si enim mórtui sumus cum Christo, crédimus quia simul étiam vívimus cum ipso.}\\
 Non enim sumus nostri, sed eius qui nos redémit, ex cuius voluntáte volúntas semper nostra péndere debet; ob hoc et in oratióne dícimus:
 \emph{Fiat volúntas tua.}
 Quamóbrem necésse est ut cum lob in fúnere dicámus:
\emph{Dóminus dedit, Dóminus ábstulit, sit nomen Dómini benedíctum.}
 Dicámus hoc cum Iob hic, ne dissímiles in causa præsénti ab eo inveniámur illic.
 \\
 \\
 Resp 3 Libera me Domine de morte - new
 
\end{document}