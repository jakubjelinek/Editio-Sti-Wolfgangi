\documentclass[options]{article}
\begin{document}
	\textbf{Ex Prólogo commentariórum sancti Hierónymi presbýteri in Isaíam prophétam (Nn. 1. 2: CCL 73, 1-3)}\\
	
	Reddo quod débeo, ob\'{œ}diens Christi præcéptis, qui ait: 
	\textit{Scrutámini Scriptúras;}
	et: 
	\textit{Qu\'{æ}rite et inveniétis,}
	ne illud áudiam cum Iud\'{æ}is: 
	\textit{Errátis, nesciéntes Scriptúras neque virtútem Dei.}
	Si enim iuxta apóstolum Paulum, Christus Dei virtus est Deíque sapiéntia, et qui nescit Scriptúras nescit Dei virtútem eiúsque sapiéntiam: ignorátio Scripturárum ignorátio Christi est.\\
	Unde imitábor patremfamílias, qui de thesáuro suo profert nova et vétera; et sponsam dicéntem in Cántico canticórum:
	\textit{Nova et vétera, fratruélis meus, servávi tibi;}
	sicque expónam Isaíam, ut illum non solum prophétam, sed evangelístam et apóstolum dóceam. Ipse enim de se et de céteris evangelístis ait: 
	\textit{Quam speciósi pedes evangelizántium bona, evangelizántium pacem.}
	Et ad ipsum quasi ad apóstolum lóquitur Deus: 
	\textit{Quem mittam, et quis ibit ad pópulum istum?}
	Et ille respóndit: 
	\textit{Ecce ego, mitte me.}\\
	Nullúsque putet me volúminis istíus arguméntum brevi cúpere sermóne comprehéndere, cum univérsa Dómini sacraménta præsens Scriptúra contíneat; et tam natus de Vírgine Emmánuel, quam illústrium patrátor óperum atque signórum, mórtuus ac sepúltus, et resúrgens ab ínferis, et Salvátor universárum géntium prædicétur. Quid loquar de phýsica, éthica et lógica? Quidquid sanctárum est Scripturárum, quidquid potest humána lingua proférre et mortálium sensus accípere, isto volúmine continétur. De cuius mystériis testátur ipse qui scripsit:
	\textit{Et erit vobis vísio ómnium, sicut verba libri signáti, quem cum déderint sciénti lítteras, dicent: Lege istum. Et respondébit: Non possum, signátus est enim. Et dábitur liber nesciénti lítteras dicetúrque ei: Lege. Et respondébit: Néscio lítteras.}\\
	 Quod si cui vidétur infírmum, illud eiúsdem Apóstoli áudiat: 
	 \textit{Prophétæ duo aut tres loquántur, et álii diiúdicent; si autem álii fúerit revelátum sedénti, prior táceat.}
	 Qua possunt ratióne reticére, cum in dicióne sit Spíritus qui lóquitur per prophétas, vel tacére vel dícere? Si ergo intellegébant quæ dicébant, cuncta sapiéntiæ rationísque sunt plena. Nec aer voce pulsátus ad aures eórum perveniébat; sed Deus loquebátur in ánimo prophetárum, iuxta illud quod álius Prophéta dicit: 
	 \textit{Angelus qui loquebátur in me, et: Clamántes in córdibus nostris, Abba, Pater, et: Audiam quid loquátur in me Dóminus Deus.}
	
	 
\end{document}