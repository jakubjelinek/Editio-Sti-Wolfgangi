\documentclass[options]{article}
\begin{document}
	Ex Sermónibus sancti Augustíni epíscopi
	\begin{flushright}
		(Sermo 256, 1. 2. 3: PL 38, 1191-1193)
	\end{flushright}
	Hic cantémus Allelúia adhuc sollíciti, ut illic possímus aliquándo cantáre secúri. Quare hic sollíciti? Non vis ut sim sollícitus, quando lego: \emph{Numquid non tentátio est vita humána super terram?} Non vis ut sim sollícitus, quando mihi adhuc dícitur: \emph{Vigiláte et oráte, ne intrétis in tentatiónem?} Non vis ut sim sollícitus, ubi sic abúndat tentátio, ut nobis ipsa præscríbat orátio, quando dícimus: \emph{Dimítte nobis débita nostra, sicut et nos dimíttimus debitóribus nostris?} Cotídie petitóres, cotídie debitóres. Vis ut sim secúrus, ubi cotídie peto indulgéntiam pro peccátis, adiutórium pro perículis? Cum enim díxero propter prætérita peccáta: \emph{Dimítte nobis débita nostra, sicut et nos dimíttimus debitóribus nostris,} contínuo propter futúra perícula addo et adiúngo: \emph{Ne nos ínferas in tentatiónem.} Quómodo est autem pópulus in bono, quando mecum clamat: \emph{Líbera nos a malo?} Et tamen, fratres, in isto adhuc malo cantémus Allelúia Deo bono, qui nos líberat a malo.\\
	Etiam hic inter perícula, inter tentatiónes, et ab áliis et a nobis cantétur Allelúia. \emph{Fidélis enim Deus, qui non permíttet,} inquit, \emph{vos tentári supra id quod potéstis.} Ergo et hic cantémus Allelúia. Adhuc est homo reus, sed fidélis est Deus. Non ait: Non permíttet vos tentári; sed: \emph{Non permíttet vos tentári supra id quod potéstis; sed fáciet cum tentatióne étiam éxitum, ut possítis sustinére.} Intrásti in tentatiónem; sed fáciet Deus étiam éxitum, ne péreas in tentatióne; ut, quómodo vas fíguli, forméris prædicatióne, coquáris tribulatióne. Sed quando intras, éxitum cógita; quia fidélis est Deus: \emph{custódiet Dóminus intróitum tuum et éxitum tuum.}\\
	
Resp 2	resp-murotuoinexpugnabili-CROCHU.gabc\\
	\\
	
	Porro autem, cum factum corpus hoc immortále et incorruptíbile, quando períerit tota tentátio; quia \emph{corpus quidem mórtuum est;} quare mórtuum est? \emph{propter peccátum. Spíritus autem vita est;} quare? \emph{propter iustítiam.} Remíttimus ergo mórtuum corpus? Non, sed audi: \emph{Si autem Spíritus eius qui suscitávit Christum a mórtuis hábitat in vobis, qui suscitávit Christum a mórtuis vivificábit et mortália córpora vestra.} Modo enim corpus animále, tunc spiritále.\\ 
	O felix illic Allelúia! o secúra! o sine adversário! ubi nemo erit inimícus, nemo perit amícus. Ibi laudes Deo, et hic laudes Deo; sed hic a sollícitis, ibi a secúris; hic a moritúris, ibi a semper victúris; hic in spe, ibi in re; hic in via, illic in pátria.\\ 
	Modo ergo, fratres mei, cantémus, non ad delectatiónem quiétis, sed ad solácium labóris. Quómodo solent cantáre viatóres: canta, sed ámbula; labórem consoláre cantándo, pigrítiam noli amáre; canta et ámbula. Quid est, ámbula? Prófice, in bono prófice. Sunt enim, secúndum Apóstolum, quidam proficiéntes in peius. Tu, si próficis, ámbulas; sed in bono prófice, in recta fide prófice, in bonis móribus prófice; canta et ámbula. 
	
	Resp 3 resp-redemitdominus-CROCHU-cumdox.gabc
	
	
\end{document}