\documentclass[options]{article}
\begin{document}
	Ex Expositióne sancti Ambrósii epíscopi in Lucam
	\begin{flushright}
		(Lib. 2, 19. 22-23. 26-27: CCL 14, 39-42)	
	\end{flushright}
	Angelus, cum abscóndita nuntiáret, ut fides astruerétur exémplo, senióris féminæ sterilísque concéptum Vírgini Maríæ nuntiávit, ut possíbile Deo omne quod ei placúerit asséreret.
	Ubi audívit hoc María, non quasi incrédula de oráculo nec quasi incérta de núntio nec quasi dúbitans de exémplo, sed quasi læta pro voto, religiósa pro offício, festína pro gáudio in montána perréxit.
	Quo enim iam Deo plena nisi ad superióra cum festinatióne conténderet? Nescit tarda molímina Sancti Spíritus grátia. Cito quoque advéntus Maríæ et præséntiæ domínicæ benefícia declarántur; \emph{simul enim ut audívit salutatiónem Maríæ Elísabeth, exsultávit infans in útero eius et repléta est Spíritu Sancto.}\\
	Vide distinctiónem singulorúmque verbórum proprietátes. Vocem prior Elísabeth audívit, sed Ioánnes prior grátiam sensit: illa natúræ órdine audívit, iste exsultávit ratióne mystérii; illa Maríæ, iste Dómini sensit advéntum, fémina mulíeris et pignus pígnoris; istæ grátiam loquúntur, illi intus operántur pietatísque mystérium matérnis adoriúntur proféctibus duplicíque miráculo prophétant matres spíritu parvulórum.\\
	\\
	Resp  resp-emitteagnumdomine-CROCHU.gabc\\
	\\
	Exsultávit infans, repléta mater est. Non prius mater repléta quam fílius, sed, cum fílius esset replétus Spíritu Sancto, replévit et matrem. Exsultávit Ioánnes, \emph{exsultávit} et Maríæ \emph{spíritus.} Exsultánte Ioánne replétur Elísabeth, Maríam tamen non repléri Spíritu, sed spíritum eius exsultáre cognóvimus —incomprehensíbilis enim incomprehensibíliter operabátur in matre— et illa post concéptum replétur, ista ante concéptum. \emph{Beáta,} inquit, \emph{quæ credidísti.}
	Sed et vos beáti, qui audístis et credidístis; quæcúmque enim credíderit ánima, et cóncipit et génerat Dei Verbum et ópera eius agnóscit.
	Sit in síngulis Maríæ ánima, ut magníficet Dóminum; sit in síngulis spíritus Maríæ, ut exsúltet in Deo; si secúndum carnem una mater est Christi, secúndum fidem tamen ómnium fructus est Christus; omnis enim ánima áccipit Dei Verbum, si tamen immaculáta et immúnis a vítiis intemeráto castimóniam pudóre custódiat.
	Quæcúmque ígitur talis esse potúerit \emph{ánima magníficat Dóminum,} sicut ánima Maríæ magnificávit Dóminum \emph{et exsultávit spíritus eius in Deo salutári.}\\
	Magnificátur enim Dóminus, sicut et álibi legístis: \emph{Magnificáte Dóminum mecum,} non quo Dómino áliquid humána voce possit adiúngi, sed quia magnificátur in nobis: imágo enim Dei Christus est, et ídeo, si quid iustum religiosúmque fécerit ánima, illam imáginem Dei, ad cuius est similitúdinem creáta, magníficat, et ídeo, dum magníficat eam, magnitúdinis eius quadam participatióne sublímior fit. \\
	\\
	Resp 3   resp-germinaveruntcampi-CROCHU-cumdox.gabc
\end{document}