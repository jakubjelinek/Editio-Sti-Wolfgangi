\documentclass[options]{article}
\begin{document}
	Ex Oratiónibus meditatívis Guillélmi abbátis monastérii Sancti Theodoríci
	\begin{flushright}
		(Orat. 13,1-3. 5 : SC 324, 212. 214)
	\end{flushright}
Dómine Iesu Christe,
\emph{véritas et vita,} qui \emph{veros adorátores}
Patris tui futúros denuntiásti, qui eum
\emph{adorábunt in spíritu et veritáte,}
líbera, óbsecro, ánimam meam ab idolátria; líbera eam ne quærens te íncidat in sodáles tuos, et erráre incípiat post greges eórum in sacrifício oratiónis suæ; sed tecum cubet, de te pascátur in meridiáno fervóre amóris tui. Quia enim naturáli quodam sensu suo a princípio suo, tuam quodámmodo quasi sómniat fáciem, ad cuius imáginem cóndita est; sed vel desuévit vel non assuévit: áliam pro illa recípere non acquiéscit, cum multæ se ófferant in hora oratiónis suæ.\\
\\
Resp 2 resp-egodixidomine-CROCHU.gabc\\
\\
Sed intentiónis suæ áciem collúctans dirígere in eam, nec videns eam, nonnúmquam præveníri étiam ab ea sentit intentiónis ipsíus conátum ; sæpe vero non nisi \emph{in} gravi \emph{sudóre vultus sui comédere} potest \emph{panem suum} in pœnam antíquæ maledictiónis; sæpe vero nec sic, nec sic, sed in domum paupertátis suæ redíre cógitur pauper et famélica. Aut enim cito próficit aut cito déficit.\\
\\
Sed \emph{quid mihi est in cælo et a te quid volo super terram?} Si enim orans quæro te in cælo isto, pulchro quidem sed corpóreo, quod sursum vídeo, pari modo erro, quam si quæram te in terra quam calco, si in áliquo quólibet loco vel extra locum, loco quem creásti inclúdo te, vel exclúdo. Si formam áliquam vel formátum áliquid imáginer mihi pro te Deo meo, idolátra fio. \\
\\
Resp 3 resp-docebitnos-CROCHU-cumdox.gabc

\end{document}