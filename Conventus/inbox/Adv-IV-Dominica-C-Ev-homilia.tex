\documentclass[options]{article}
\usepackage[T1]{fontenc}
\begin{document}
	Léctio sancti Evangélii secúndum Lucam
	\begin{flushright}
		Lc 1, 39-45
	\end{flushright}
	Exsúrgens María in diébus illis ábiit in montána cum festinatióne in civitátem Iudæ et intrávit in domum Zacharíæ et salutávit Elísabeth.
	Et factum est, ut audívit salutatiónem Maríæ Elísabeth, exsultávit infans in útero eius, et repléta est Spíritu Sancto Elísabeth et exclamávit voce magna et dixit:
	«Benedícta tu inter mulíeres, et benedíctus fructus ventris tui. Et unde hoc mihi, ut véniat mater Dómini mei ad me? Ecce enim ut facta est vox salutatiónis tuæ in áuribus meis, exsultávit in gáudio infans in útero meo. Et beáta, quæ crédidit, quóniam perficiéntur ea, quæ dicta sunt ei a Dómino».\\
	\\
	Ex Sermónibus beáti Guerríci abbátis
	\begin{flushright}
		(Sermo 2 de Advéntus, 1. 2-3 : SC 166,104. 108)
	\end{flushright}
	\emph{Ecce venit Rex: occurrámus óbviam Salvatóri nostro.}
	Pulchre Sálomon ait: \emph{Aqua frígida ánimæ sitiénti núntius bonus de terra longínqua.} Bonus útique núntius, qui advéntum Salvatóris núntiat, reconciliatiónem mundi, bona superventúri s\'{æ}culi. \emph{Quam pulchri pedes annuntiántium pacem, annuntiántium bona.} Multi síquidem, non unus, multi, inquam, sed in uno spíritu longa série ab inítio s\'{æ}culi nobis supervenére núntii; et ómnium vox símilis, una senténtia: \emph{Venit, ecce venit.}\\
	\\
	Veni ergo, Dómine, \emph{salvum me fac et salvus ero;} veni et\emph{osténde fáciem tuam et salvi érimus. Te enim exspectávimus; esto salus nostra in témpore tribulatiónis.} Sic prophétæ et iusti desidério et afféctu tanto ante Christo ventúro occurrébant, desiderántes si fíeri posset óculis vidére quod spíritu prævidébant.\\
	\\
	Unde Dóminus discípulis dicébat: \emph{Beáti óculi, qui vident quæ vos vidétis. Dico enim vobis, quod multi prophétæ et iusti voluérunt vidére quæ vos vidétis et non vidérunt.} Abraham quoque pater noster exultávit ut vidéret diem Christi. \emph{Vidit,} sed apud ínferos, \emph{et gavísus est.} In quo útique tepor et durítia cordis nostri sugillátur: si non vidélicet cum gáudio spiritáli Christi nascéntis diem anniversárium exspectámus, qui nobis in próximo vidéndus, Dómino annuénte, promíttitur.\\
	\\
	Hoc sane gáudium nostrum tale vidétur exígere Scriptúra, ut spíritus noster levans se super se Christo veniénti quodámmodo occúrrere géstiat, desidério se exténdens in anterióra, impatiénsque morárum iam vidére conténdat futúra. Ego namque non solum ad secúndum advéntum sed étiam ad primum árbitror pertinére, quod tot locis Scripturárum ei monémur occúrrere. Quómodo ? inquis. Quia vidélicet, sicut secúndo advéntui occurrémus motu et exsultatióne córporis, sic et primo occurréndum est afféctu et exsultatióne cordis.\\
	\\
	Resp 7 resp-intueminiquantus-CROCHU-cumdox.gabc
\end{document}