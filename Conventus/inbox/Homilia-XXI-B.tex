\documentclass[options]{article}
\begin{document}

Ex Tractátibus sancti Augustíni epíscopi in Ioánnem (Tract. 27,5)

Quid est ergo quod adiúngit: 
\textit{Spíritus est qui vivíficat, caro non prodest quidquam?}
Dicámus ei (pátitur enim nos non contradicéntes, sed nosse cupiéntes): O Dómine, magíster bone, quómodo caro non prodest quidquam, cum tu díxeris: 
\textit{Nisi quis manducáverit carnem meam, et bíberit sánguinem meum, non habébit in se vitam?}
 An vita non prodest quidquam? et propter quid sumus quod sumus, nisi ut habeámus vitam ætérnam, quam tua carne promíttis? quid est ergo, 
 \textit{non prodest quidquam caro}?
 Non prodest quidquam, sed quómodo illi intellexérunt: carnem quippe sic intellexérunt, quómodo in cadávere dilaniátur, aut in macéllo vénditur, non quómodo spíritu vegetátur. Proínde sic dictum est: 
 \textit{Caro non prodest quidquam;}
  quómodo dictum est: 
\textit{Sciéntia inflat.}
   Iam ergo debémus odísse sciéntiam? Absit. Et quid est: Sciéntia inflat? Sola, sine caritáte: ídeo adiúnxit: 
   \textit{Cáritas vero ædíficat.}
  Adde ergo sciéntiæ caritátem, et útilis erit sciéntia; non per se, sed per caritátem. Sic étiam nunc, 
  \textit{caro non prodest quidquam,}
 sed sola caro: accédat spíritus ad carnem, quómodo accédit cáritas ad sciéntiam, et prodest plúrimum. Nam si caro nihil prodésset, Verbum caro non fíeret, ut inhabitáret in nobis. Si per carnem nobis multum prófuit Christus, quómodo caro nihil prodest? Sed per carnem Spíritus áliquid pro salúte nostra egit. Caro vas fuit; quod habébat atténde, non quod erat. Apóstoli missi sunt; numquid caro ipsórum nihil nobis prófuit? Si caro Apostolórum nobis prófuit, caro Dómini pótuit nihil prodésse? Unde enim ad nos sonus verbi, nisi per vocem carnis? unde stilus, unde conscríptio? Ista ómnia ópera carnis sunt, sed agitánte spíritu tamquam órganum suum. 
 \textit{Spíritus}
  ergo 
  \textit{est qui vivíficat, caro autem non prodest quidquam:}
   sicut illi intellexérunt carnem, non sic ego do ad manducándum carnem meam.

\end{document}
