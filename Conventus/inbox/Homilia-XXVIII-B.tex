\documentclass[options]{article}
\begin{document}
		Ex Libro 
		\textit{Quis dives salvétur?}
		Cleméntis Alexandríni 
		\begin{flushright}
					(Cap. 5,10 : PG 9, 610)
		\end{flushright}
	Hæc quidem in Marci evangélio scripta sunt; quin et in réliquis plane ómnibus éadem habéntur, paucis fórsitan per síngulos verbis immutátis, sed quæ eámdem ubíque senténtiam contíneant. Decet autem nos qui apérte sciámus, nihil Salvatórem quasi humáno more locútum esse, sed divína mysticáque sapiéntia suos cuncta docuísse; sermónes hos non carnáliter audíre, sed laténtem in eis sensum digna investigatióne mentísque solértia ac sagacitáte perquírere et addíscere.\\
	\\
\emph{Si vis perféctus esse.}
	Nondum ergo perféctus erat; perfécto namque nihil perféctius est. Céterum præcláre illud atque divíne: 
	\emph{Si vis,}
	colloquéntis ánimæ líberam arbítrii facultátem osténdit; in hómine quippe, tamquam líbero, líbera erat voluntátis eléctio; in Deo autem dare, tamquam Dómino atque árbitro. Dat autem voléntibus et summo stúdio adniténtibus et orántibus, ut sic illórum própria exsístat salus. Neque enim Deus cogit — vis enim inimíca est Deo —, sed quæréntibus tríbuit, et peténtibus præbet, ac pulsántibus áperit. 
	\emph{Si vis}
	ígitur, si vere vis, et teípsum non fallis, illud cómpara quo defíceris.\\
	\\
	\emph{Unum tibi deest;}
	 illud 
	 \emph{unum}
	 quod manet, quod bonum est, quod est iam supra legem, quod lex non dat, quod lex non capit, quod vivéntium próprium est. Dénique qui totam legem a iuventúte impléverat, et qui de se sic magna atque supérba locútus erat,
	  \emph{unum}
	 hoc ómnibus paráre nequívit, quod Salvatóris singuláre est, ut 
	 \emph{vitam aetérnam,}
	  cuius eum desidérium incésserat, arríperet.\\
	  \\
	    \emph{Sed tristis ábiit,}
	    vitas mandáto gravátus, cuius grátia supplicátum vénerat. Non enim vere vitam ambiébat, ut verbis proferébat; sed bonæ dumtáxat voluntátis famam aucupabátur; ac quidem circa multa sollícitus esse póterat, 
	 \emph{unum}
	vero illud, opus illud salútis ut perfíceret, non valébat; inque rem supínus et infírmus erat.
\end{document}