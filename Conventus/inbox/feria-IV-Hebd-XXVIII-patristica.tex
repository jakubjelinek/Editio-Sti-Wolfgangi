\documentclass[options]{article}
\begin{document}
	Ex Quæstiónibus sancti Máximi Confessóris abbátis ad Thalássium
	\begin{flushright}
		(Quæst. 63: PG 90, 667-670)
	\end{flushright}
Lampas superimpósita candelábro patérna est veráque lux, 
\emph{quæ illúminat omnem hóminem veniéntem in mundum,}
Dóminus noster Iesus Christus, qui ex nobis nostr\'{æ}que carnis assumptióne lampas et factus est et nuncupátus; id est naturális et Patris sapiéntia sermóque, qui in Ecclésia Dei pia fide prædicátur, vitáque ex virtútis ratiónibus institúta mandatórum observatióne in géntibus exaltátur ac splendéscit, lucétque ómnibus qui in domo sunt (in isto scílicet mundo), quemádmodum ípsemet Deus ac Sermo quodam loco lóquitur: 
\emph{Nemo,}
inquit,
\emph{accéndit lucérnam et ponit eam sub módio, sed super candelábrum; et lucet ómnibus qui in domo sunt:}
plane se ipse lucérnam vocans, quippe qui Deus cum esset per natúram, homo factus sit secúndum dispensatiónem.\\
Atque id, puto, magnus quoque David intéllegens, lucérnam Dóminum vocávit, quibus ait:
\emph{Lucérna pédibus meis lex tua et lumen sémitis meis.}
Talis enim salvátor meus Deúsque est, qui ignorántiæ atque vítii discútiat ténebras, quam étiam ob causam Scriptúra lucérnam ipsum appellávit.\\
Hic nimírum solus, lucérnæ more ignorántiæ discússa calígine, abactísque nequítiæ atque vítii ténebris, cunctis via salútis efféctus est: per virtútem ac sciéntiam ad Patrem eos ducens, qui, ut ipsum tamquam iustítiæ viam per divína mandáta incédant, in ánimum indúcunt. \\
\\
resp-docebitnos-CROCHU.gabc\\
\\
 Candelábrum autem sanctam vocávit Ecclésiam, in qua per prædicatiónem Dei sermo lucens, quotquot hoc in mundo velut in quadam domo versántur, veritátis fulgóribus illústrat, ómnium mentem divína agnitióne adímplens.\\
  Sub quo módio tenéri nullo modo sermo sústinet, cui summo vértice collocári allúbeat ac magnitúdine decóris Ecclésiæ. Quámdiu enim sermo velut módio legis líttera cohibétur, cunctos sempitérna luce privávit; qui nimírum sensum ut seductórem nec nisi erróris capácem eáque vi pr\'{æ}ditum ut affínium dumtáxat córporum labem interitúmque percípiat, exúere satagéntibus, spiritálem contemplatiónem non pr\'{æ}beat; sed candelábro impósitus, Ecclésiæ scílicet, id est rationáli in spíritu cúltui, ut omnes illúminet.\\
   Líttera enim, nisi spiritáliter intellegátur, solum sensum habet, quo eius prolátio circumscríbitur, nec eórum quæ scripta sunt vis ad ánimum transíre sínitur.\\
    Ne ígitur lucérnam (ratiónem scílicet, quæ sciéntiæ lucem accéndit) contemplatiónis ac actiónis cultu accendéntes, sub módio ponámus; ne, velut qui sapiéntiæ inexplicábilem vim líttera circumscribámus, rei peragámur: sed super candelábrum (sanctam nimírum Ecclésiam) quæ, in veræ contemplatiónis celso vértice, divinórum cunctis dógmatum facem præténdat.\\
    \\
    
   
   Resp--In-circuitu-tuo-Domine (new, uploaded)
  


\end{document}