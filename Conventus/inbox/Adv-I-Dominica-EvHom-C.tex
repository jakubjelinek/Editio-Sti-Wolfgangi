\documentclass[options]{article}
\usepackage[T1]{fontenc}
\begin{document}
Léctio sancti Evangélii secúndum Lucam 
\begin{flushright}
Lc 21, 25-28. 34-36	
\end{flushright}
	In illo témpore: Dixit Iesus discípulis suis:
	«Erunt signa in sole et luna et stellis, et super terram pressúra géntium præ confusióne sónitus maris et flúctuum, arescéntibus homínibus præ timóre et exspectatióne eórum, quæ supervénient orbi, nam virtútes cælórum movebúntur.
	Et tunc vidébunt Fílium hóminis veniéntem in nube cum potestáte et glória magna. His autem fíeri incipiéntibus, respícite et leváte cápita vestra, quóniam appropínquat redémptio vestra.
	Atténdite autem vobis, ne forte gravéntur corda vestra in crápula et ebrietáte et curis huius vitæ, et supervéniat in vos repentína dies illa; tamquam láqueus enim supervéniet in omnes, qui sedent super fáciem omnis terræ. Vigiláte ítaque omni témpore orántes, ut possítis fúgere ista ómnia, quæ futúra sunt, et stare ante Fílium hóminis».\\
	\\
	Ex Homíliis sancti Gregórii Magni papæ in Evangélia.
	\begin{flushright}
			(Hom. 1,1. 2. 3. 6 : PL 76,1077-1079.1081)
	\end{flushright}
	Dóminus ac Redémptor noster, fratres caríssimi, parátos nos inveníre desíderans, senescéntem mundum quæ mala sequántur denúntiat, ut nos ab eius amóre compéscat. Appropinquántem eius términum quantæ percussiónes prævéniant innotéscit, ut, si Deum metúere in tranquillitáte nólumus, saltem vicínum eius iudícium vel percussiónibus attríti timeámus. Huic enim lectióni sancti evangélii, quam modo vestra fratérnitas audívit, paulo supérius Dóminus præmísit, dicens: \emph{Exsúrget gens contra gentem et regnum advérsus regnum; et erunt terræ motus magni per loca et pestiléntiæ et fames.} Et quibúsdam interpósitis, hoc quod modo audístis adiúnxit: \emph{Erunt signa in sole et luna et stellis, et in terris pressúra géntium.} Ex quibus profécto ómnibus ália iam facta cérnimus, ália e próximo ventúra formidámus.\\
	Idcírco dícimus, ut ad cautélæ stúdium vestræ mentes evígilent, ne securitáte tórpeant, ne ignorántia languéscant, sed semper eas et timor sollícitet, et in bono ópere sollicitúdo confírmet, pensántes hoc quod Redemptóris nostri voce subiúngitur: \emph{Arescéntibus homínibus præ timóre et exspectatióne quæ supervénient univérso orbi.}\\
	\\
	Sed quia hæc contra réprobos dicta sunt, mox ad electórum consolatiónem verba vertúntur: 
	\emph{His autem fíeri incipiéntibus, respícite et leváte vos cápita vestra, quóniam appropínquat redémptio vestra.} Leváre ítaque cápita est mentes nostras ad gáudia pátriæ cæléstis erígere. Qui ergo Deum díligunt, ex mundi fine gaudére et hilaréscere iubéntur, quia vidélicet eum quem amant mox invéniunt, dum transit is quem non amavérunt. Absit enim ut fidélis quisque qui Deum vidére desíderat de mundi percussiónibus lúgeat, quem finíri eísdem suis percussiónibus non ignórat.\\
	Illum ergo diem, fratres caríssimi, tota intentióne cogitáte, vitam corrígite, mores mutáte, mala lentántia resisténdo víncite, perpetráta autem flétibus puníte. Advéntum namque ætérni iúdicis tanto securióres quandóque vidébitis, quanto nunc districtiónem illíus timéndo prævenítis.\\
	\\
	
	Resp 7 resp-laetenturcaeli-CROCHU-cumdox.gabc
	
\end{document}