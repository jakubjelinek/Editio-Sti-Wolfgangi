\documentclass[options]{article}
\begin{document}
	Ex Confessiónum libris sancti Augustíni epíscopi
	\begin{flushright}
		(Lib. 5, 23-24; 6, 3 : CCL 27, 70.71.75-76)	
	\end{flushright}
	Veni Mediolánum ad Ambrósium epíscopum, in óptimis notum orbi terræ, pium cultórem tuum, cuius tunc elóquia strénue ministrábant \emph{ádipem fruménti tui et lætítiam ólei et sóbriam vini ebrietátem} pópulo tuo. Ad eum autem ducébar abs te nésciens, ut per eum ad te sciens dúcerer. Suscépit me patérne ille homo Dei et peregrinatiónem meam satis episcopáliter diléxit. Et eum amáre cœpi primo quidem non tamquam doctórem veri — quod in Ecclésia tua prorsus desperábam — sed tamquam hóminem benígnum in me. Et studióse audiébam disputántem in pópulo, non intentióne, qua débui, sed quasi explórans eius facúndiam. Cum enim non satágerem díscere quæ dicébat, sed tantum quemádmodum dicébat audíre, veniébant in ánimum meum simul cum verbis, quæ diligébam, res étiam, quas neglegébam.\\
	\\
	Ambrósium felícem quemdam hóminem secúndum s\'{æ}culum opinábar, quem sic tantæ potestátes honorárent: cælibátus tantum eius mihi laboriósus videbátur. Quid autem ille spei géreret, advérsus ipsíus excelléntiæ tentaménta quid luctáminis habéret quidve soláminis in advérsis, et occúltum os eius, quod erat in corde eius, quam sápida gáudia de pane tuo rumináret, nec conícere nóveram nec expértus eram. Nec ille sciébat æstus meos nec fóveam perículi mei. Non enim qu\'{æ}rere ab eo póteram quod volébam, sicut volébam, secludéntibus me ab eius aure atque ore catérvis negotiosórum hóminum, quorum infirmitátibus serviébat.\\
	\\
	Sed cum legébat, óculi ducebántur per páginas et cor intelléctum rimabátur, vox autem et lingua quiescébant. Sæpe, cum adessémus — non enim vetabátur quisquam íngredi aut ei veniéntem nuntiári mos erat — sic eum legéntem vídimus tácite et áliter numquam sedentésque in diutúrno siléntio — quis enim tam inténto esse óneri audéret ? — discedebámus et coniectabámus eum parvo ipso témpore, quod reparándæ mentis suæ nanciscebátur, feriátum ab strépitu causárum alienárum nolle in áliud avocári et cavére fortásse, ne auditóre suspénso et inténto, étiam expónere esset necésse, quamquam et causa servándæ vocis, quæ illi facíllime obtundebátur, póterat esse iústior tácite legéndi. Quólibet tamen ánimo id ágeret, bono útique ille vir agébat.\\
	\\
	Resp 3 IuravitDominus
\end{document}