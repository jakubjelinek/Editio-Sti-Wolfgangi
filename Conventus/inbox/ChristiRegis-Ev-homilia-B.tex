\documentclass[options]{article}
\usepackage[T1]{fontenc}
	\begin{document}
			Léctio sancti Evangélii secúndum Ioánnem 
		\begin{flushright}
			Io 18, 33-37 
		\end{flushright} 
	In illo témpore:
	Dixit Pilátus ad Iesum: «Tu es rex Iudæórum?».\\
	Respóndit Iesus: «A temetípso tu hoc dicis, an álii tibi dixérunt de me?».\\
	Respóndit Pilátus: «Numquid ego Iud\'{æ}us sum? Gens tua et pontífices tradidérunt te mihi; quid fecísti?».\\
	Respóndit Iesus: «Regnum meum non est de mundo hoc; si ex hoc mundo esset regnum meum, minístri mei decertárent, ut non tráderer Iud\'{æ}is; nunc autem meum regnum non est hinc».\\
	Dixit ítaque ei Pilátus: «Ergo rex es tu?».\\
	Respóndit Iesus: «Tu dicis quia rex sum. Ego in hoc natus sum et ad hoc veni in mundum, ut testimónium perhíbeam veritáti; omnis, qui est ex veritáte, audit meam vocem».\\
	
	Ex Tractátibus sancti Augustíni epíscopi in Ioánnem 
	\begin{flushright}
		(Tract. 115,2 : CCL 36, 644-645)
	\end{flushright}
	
	Audíte, ómnia regna terréna: «Non impédio dominatiónem vestram in hoc mundo; \emph{regnum meum non est de hoc mundo.}» Nolíte metúere metu vaníssimo quo Heródes ille maior, cum Christus natus nuntiarétur, expávit, et tot infántes ut ad eum mors perveníret, occídit, timéndo magis quam irascéndo crudélior: \emph{Regnum,} inquit, \emph{meum non est de hoc mundo.} Quid vultis ámplius? Veníte ad regnum quod non est de hoc mundo; veníte credéndo, et nolíte sævíre metuéndo.\\
	
	Quod est enim eius regnum nisi credéntes in eum, quibus dicit: \emph{De mundo non estis, sicut et ego non sum de mundo?} Quamvis eos esse vellet in mundo; propter quod de illis dixit ad Patrem: \emph{Non rogo ut tollas eos de mundo, sed ut serves eos ex malo.} Unde et hic non ait: «\emph{Regnum meum non est} in hoc mundo», sed: \emph{non est de hoc mundo.}\\
	\\
	Et cum hoc probáret dicens: \emph{Si ex hoc mundo esset regnum meum, minístri mei útique decertárent, ut non tráderer Iud\'{æ}is}; non ait: «\emph{Nunc autem regnum meum}  non est hic», sed: \emph{Non est hinc.} Hic est enim regnum eius usque in finem s\'{æ}culi, habens inter se commíxta zizánia usque ad messem; messis enim finis est s\'{æ}culi, quando \emph{messóres} vénient, \emph{id est ángeli}, et \emph{cólligent de regno eius ómnia scándala}; quod útique non fíeret, si \emph{regnum} eius non esset hic. Sed tamen \emph{non est hinc}, quia peregrinátur in mundo.\\
	
	Regno suo quippe dicit: \emph{De mundo non estis, sed ego vos elégi de mundo.} Erant ergo de mundo, quando regnum eius non erant, sed ad mundi príncipem pertinébant. De mundo est ergo quidquid hóminum a vero quidem Deo creátum, sed ex Adam vitiáta atque damnáta stirpe generátum est; factum est autem regnum non iam de mundo, quidquid inde in Christo regenerátum est. Sic enim nos Deus éruit de potestáte tenebrárum et tránstulit in regnum Fílii caritátis suæ; de quo regno dicit: \emph{Regnum meum non est de hoc mundo}; vel: \emph{Regnum meum non est hinc.}\\
	\\
	  resp-redemitdominus-CROCHU-cumdox.gabc
	\end{document}
