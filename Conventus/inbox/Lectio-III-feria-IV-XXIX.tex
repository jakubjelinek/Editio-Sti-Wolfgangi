\documentclass[options]{article}
\begin{document}
	Ex Commentário Ruperti Tuitiensis abbátis in Amos prophétam
	\begin{flushright}
		(Lib. 4 : PL 168, 366-367)
	\end{flushright}
\emph{Vidi Dóminum stantem super altáre.}
Magna vere vísio, et magnæ rei significátio éxstitit. Prophético deínde spíritu vidit, et revéra futúrum erat, quod illi talis vísio præosténdit Qu\'{æ}rimus ergo ubi vel quando factum sit tale quid, cuius in signum stans Dóminus super altáre vidéri debúerit. Quæréntibus autem in toto Christi Evangélio, vel in omni evangélica grátiæ sacraménto, nihil tam magnum, nihil tam évidens secúndum huius visiónis proprietátem nobis occúrrit quam schema vel hábitus Dómini nostri Iesu Christi crucifíxi. Crucifíxus namque et sacrifícium pro nobis factus, super altáre crucis stetit, statióne diffícili, statióne laboriósa sibi.\\
Diligénter animadverténdum, nec umquam óculis mentis nostræ debet abésse statiónis illíus spectáculum. Pendébat ei stabat mánibus ad crucis córnua confíxis, pédibus ligno suppedáneo per clavórum fixúram cohæréntibus in modum stantis. Táliter stans ipse hóstia, crux vero altáre erat.\\
Mirabíliter Dóminus qui super altáre crucis stare habébat, ut iam dictum est, simul cum praeostensióne statiónis illíus omnipoténtiam suam deprómit, mira declamatióne ásserit quod manus suas nullus inimicórum eius effúgere possit. Nunc ut verba hæc mélius perpéndere queámus, simul recordémur illud quod idem homo factus iamque passúrus, sive super aram crucis statúrus dixit: 
\emph{Et ego si exaltátus fúero a terra, ómnia traham ad me.}
\end{document}