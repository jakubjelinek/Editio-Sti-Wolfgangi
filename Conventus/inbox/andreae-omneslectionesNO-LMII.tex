\documentclass[options]{article}
\usepackage[T1]{fontenc}
\begin{document}
	De Epístola prima beáti Pauli apóstoli ad Corínthios
	\begin{flushright}
		1, 18—2, 5
	\end{flushright}
Fratres: Verbum crucis pereúntibus quidem stultítia est, his autem, qui salvi fiunt, id est nobis, virtus Dei est. Scriptum est enim:
\emph{«Perdam sapiéntiam sapiéntium et prudéntiam prudéntium reprobábo».}
Ubi sápiens? Ubi scriba? Ubi conquisítor huius s\'{æ}culi? Nonne stultam fecit Deus sapiéntiam huius mundi? Nam quia in Dei sapiéntia non cognóvit mundus per sapiéntiam Deum, plácuit Deo per stultítiam prædicatiónis salvos fácere credéntes. Quóniam et Iud\'{æ}i signa petunt et Græci sapiéntiam quærunt, nos autem prædicámus Christum crucifíxum, Iud\'{æ}is quidem scándalum, géntibus autem stultítiam; ipsis autem vocátis, Iud\'{æ}is atque Græcis, Christum Dei virtútem et Dei sapiéntiam, quia quod stultum est Dei, sapiéntius est homínibus et, quod infírmum est Dei, fórtius est homínibus.\\
Vidéte enim vocatiónem vestram, fratres, quia non multi sapiéntes secúndum carnem, non multi poténtes, non multi nóbiles; sed, quæ stulta sunt mundi, elégit Deus, ut confúndat sapiéntes, et infírma mundi elégit Deus, ut confúndat fórtia, et ignobília mundi et contemptibília elégit Deus, quæ non sunt, ut ea, quæ sunt, destrúeret, ut non gloriétur omnis caro in conspéctu Dei. Ex ipso autem vos estis in Christo Iesu, qui factus est sapiéntia nobis a Deo et iustítia et sanctificátio et redémptio, ut quemádmodum scriptum est: \emph{«Qui gloriátur, in Dómino gloriétur».}\\
Et, ego, cum veníssem ad vos, fratres, veni non per sublimitátem sermónis aut sapiéntiæ annúntians vobis mystérium Dei. Non enim iudicávi scire me áliquid inter vos nisi Iesum Christum et hunc crucifíxum. Et ego in infirmitáte et timóre et tremóre multo fui apud vos, et sermo meus et prædicátio mea non in persuasibílibus sapiéntiæ verbis, sed in ostensióne Spíritus et virtútis, ut fides vestra non sit in sapiéntia hóminum sed in virtúte Dei.\\
\\
 Resp—DoctorbonusetamicusDei.gabc\\
\\
Ex Sermónibus sancti Bernárdi abbátis
\begin{flushright}
	(Sermo 1 de S. Andréa, 1.5.10 : EC 5,427.430.433)
\end{flushright}
Celebrántes hódie gloriósum beáti Andréæ triúmphum, in verbis grátiæ quæ procedébant de ore eius, exsultávimus et delectáti sumus. Neque enim locus póterat esse tristítiæ, ubi tam veheménter lætabátur et ipse. Nemo ex nobis compássus est sic patiénti, nemo ausus est plángere exsultántem. Dénique cum ducerétur ipse beátus Andréas ad crucem, pópulus, qui sanctum et iustum dolébat iniúste damnári, prohibére vóluit ne punirétur; sed magis ipse instantíssima prece prohíbuit eos, ne non coronarétur, immo ne non paterétur. Desiderábat síquidem dissólvi et cum Christo esse, sed in cruce quam semper amáverat.\\\\
Desiderábat regnum intráre, sed per patíbulum. Quid enim dicit illi amátæ suæ? «Per te, inquit, me recípiat, qui per te me redémit.» Ergo si dilígimus eum, congaudémus ei, non solum quia coronátus, sed et quia crucifíxus, quia desidérium ánimæ eius tríbuit ei Dóminus et \emph{pósuit in cápite eius corónam de lápide pretióso.} Verúmtamen dum congratulámur ei, quod diu desiderátæ crucis frui mereátur ampléxu, mirum valde est, si non ipsum eius mirámur gáudium, cui congratulámur.\\
\\
Resp 2 O bona crux\\
\\
Tríplicem licet consideráre gradum: incipiéntium, proficiéntium, perfectórum. \emph{Inítium} enim \emph{sapiéntiæ, timor Dómini;}  médium, spes; cáritas, plenitúdo. Dénique Apóstolum audi, quia \emph{plenitúdo legis est cáritas.} Qui initiátur a timóre, crucem Christi sústinet patiénter; qui próficit in spe, portat libénter; qui consummátur in caritáte, ampléctitur iam ardénter. Solus iste est qui dícere possit, quia «amátor tuus semper fui et desiderávi amplécti te»\\
\\
Felix ánima, quæ ad hunc caritátis pervénerit statum! Nec sane desperándum nobis, quandóquidem eius qui pervénit, ob hoc máxime memória celebrátur, ut ipsíus et invocémus auxílium et provocémur exémplo. Si dicis beátum Andréam apóstolum esse, non posse te, qui pusíllus es, eius sequi vestígia, púdeat certe vel eos qui tecum sunt non imitári. Nemo repénte fit summus: ascendéndo, non volándo apprehénditur súmmitas scalæ.\\
\\
Resp 3  Salve crux\\
\\
\end{document}