\documentclass[options]{article}

\begin{document}
	Ex Tractátu sancti Bernárdi abbátis De grádibus humilitátis et supérbiæ
	\begin{flushright}
		N. 19: EC 3, 30)
	\end{flushright}
In his tribus, id est in luctu pæniténtiæ, in desidério iustítias, in opéribus misericórdiæ illi qui persevérant, a tribus impediméntis, quæ aut ignorántia, aut infirmitáte, aut stúdio contraxérunt, cordis áciem mundant, quo per contemplatiónem ad tértium veritátis gradum pertránseant. Hæc sunt 
\emph{viæ quæ vidéntur homínibus bonæ,}
illis dumtáxat
\emph{qui lætántur cum male fécerint, et exsúltant in rebus péssimis}



ac se de infirmitáte vel ignorántia tegunt ad excusándas excusatiónes in peccátis.\\
\\
Sed frustra sibi de infirmitáte vel ignorántia blandiúntur, qui ut libérius peccent, libénter ignórant vel infirmántur. Putas primo hómini prófuit, licet ipse non libénter peccávit, quod se per uxórem, tamquam per carnis infirmitátem, deféndit? Aut primi mártyris lapidatóres, quóniam aures suas continuérunt, per ignorántiam excusábiles erunt? \\
\\
Qui ígitur stúdio et amóre peccándi a veritáte se séntiunt alienátos, infirmitáte et ignorántia pressos, stúdium in gémitum, amórem in mærórem convértant, infirmitátem carnis fervóre iustítiæ, ignorántiam liberalitáte repéllant, ne si nunc egéntem, nudam, infírmam veritátem ignórant, cum potestáte magna et virtúte veniéntem, terréntem, arguéntem, sero cum rubóre cognóscant, frustra cum tremóre respóndeant : 
\emph{Quando te vídimus egére et non ministrávimus tibi ?}\\
\\
\emph{Cognoscétur}
certe
\emph{Dóminus iudícia fáciens,}
 qui nunc ignorátur misericórdiam quærens. Dénique
 \emph{vidébunt in quem transfixérunt,}
  simíliter et avári quem contempsérunt. Ab omni ergo labe, infirmitáte, ignorántia, studióve contrácta, flendo, iustítiam esuriéndo, opéribus misericórdiæ insisténdo, mundátur óculus cordis, cui se in sui puritáte vidéndam Véritas promíttit: 
  \emph{Beáti enim mundo corde, quóniam ipsi Deum vidébunt.}
  
  

\end{document}