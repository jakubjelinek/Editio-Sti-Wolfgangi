\documentclass[options]{article}
\begin{document}
	Ex Libro sancti Augustíni epíscopi 
	\emph{De grátia et líbero arbítrio}
	\begin{flushright}
		(Gap. 15 : PL 44, 899-900)
	\end{flushright}
Ne putétur nihil fácere ipsos hómines per líberum arbítrium, ídeo in psalmo dícitur: 
\emph{Nolíte obduráre corda vestra.}
Et per Ezechiélem:
\emph{Proícite a vobis omnes impietátes vestras, quas ímpie egístis in me, et fácite vobis cor novum et spíritum novum, et fácite ómnia mandáta mea. Et convertímini et vivétis.}
Meminérimus eum dícere:
\emph{Et convertímini et vivétis;}
cui dícitur:
\emph{Convérte nos, Deus.}
Meminérimus eum dícere:
\emph{Proícite a vobis impietátes vestras;}
cum ipse
\emph{iustíficet ímpium.}
Meminérimus ipsum dícere:
\emph{Fácite vobis cor novum et spíritum novum; qui dicit: Dabo vobis cor novum, et spíritum novum dabo vobis.}\\
\\
Quómodo ergo qui dicit:
\emph{Fácite vobis;}
 hoc dicit:
 \emph{Dabo vobis?}
 Quare iubet, si ipse datúrus est? Quare dat, si homo factúrus est; nisi quia dat quod iubet, cum ádiuvat ut fáciat cui iubet? Semper est autem in nobis volúntas líbera, sed non semper est bona. Aut enim a iustítia líbera est, quando servit peccáto, et tunc est mala; aut a peccáto líbera est, quando servit iustítiæ, et tunc est bona. Grátia vero Dei semper est bona, et per hanc fit ut sit homo bonæ voluntátis, qui prius fuit voluntátis malæ. Per hanc étiam fit ut ipsa bona volúntas, quæ iam esse cœpit, augeátur, et tam magna fiat, ut possit implére divína mandáta quæ volúerit, cum valde perfectéque volúerit.\\
 \\
 Ad hoc enim valet quod scriptum est:
 \emph{Si volúeris, servábis mandáta;}
 ut homo qui volúerit et non potúerit, nondum se plene velle cognóscat, et oret ut hábeat tantam voluntátem, quanta súfficit ad implénda mandáta. Sic quippe adiuvátur, ut fáciat quod iubétur. Tunc enim útile est velle, cum póssumus; et tunc útile est posse, cum vólumus; nam quid prodest, si quod non póssumus vólumus, aut quod póssumus nólumus ?



\end{document}