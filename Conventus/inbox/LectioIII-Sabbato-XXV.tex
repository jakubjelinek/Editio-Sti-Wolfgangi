\documentclass[options]{article}
\title{LectioIII-Sabbato-XXV}
\begin{document}
	\textbf{Ex Homíliis Sevéri Antiochéni epíscopi}\\
	(Hom. de sancta Dei Matre : Gard. Mai, Spicilégium Románum 10, 212)\\
	Volénti mihi ad vírginem matrem óculos attóllere, perque dictas de ea senténtias suspénso ob reveréntiam vestígio incédere, quædam véluti a Deo vox deférri vidétur, meásque aures clamóre válido impéllere his verbis:
	\textit{Cave ne huc accédas; tolle cálceos de pédibus; locus enim in quo insístis, terra sancta est.}
	Reápse opórtet mortáli quálibet carnalíque phantásia, tamquam cálceis, mentem illam semet exúere, quæ ad divinárum rerum contemplatiónem conscéndere nítitur. Quid vero cogitári augústius vel excélsius potest quam Dei Mater? Certe qui ad eam accédit, ad sanctam véluti terram sic appropínquat, ut ipsum dénique cælum attíngat.\\
	
	Quamquam enim María de terra est, et humánam natúram nobísque consubstantiálem sortíta, áttamen intemeráta est omníque mácula carens; quin ádeo de suis viscéribus, ceu de cælo, Deum prótulit factum hóminem, quem ipsa divínitus concépit ac péperit: non quod divínam natúram ipsi déderit, qui de ea natus est; is enim quólibet princípio caret, omníque re mundiáli vetústior est; sed quia ex se, per ineffábilem arcanúmque Spíritus Sancti descénsum, humánam natúram illi cóntulit, qui nihilóminus sine ulla mutatióne permánsit.\\
	
	Quod si tantæ rei ratiónem scrutári velis, omnem tibi vestigatiónem præclúdit virginitátis signáculum, in partu étiam incólume. Res profécto inexplicábilis, arcána atque ineffábilis, quam qui consíderat, attónitus cum Iacóbo exclámat:
	\textit{Quam terríbilis est locus iste! Hæc útique iánua cæli est.}
	
	
\end{document}