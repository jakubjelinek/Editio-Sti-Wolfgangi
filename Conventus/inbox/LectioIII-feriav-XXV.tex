\documentclass[options]{article}
\title{Léctio III}
\begin{document}
	\textbf{Ex Enarratiónibus sancti Augustíni epíscopi in psalmos}\\
	(En. in ps. 95,4 : CCL 39,1345-1346)
	
	\textit{Quóniam magnus Dóminus et laudábilis nimis.}
	Quis 
	\textit{Dóminus,}
	nisi lesus Christus, 
	\textit{magnus et laudábilis nimis?}
	Nostis certe quia homo appáruit; nostis certe quia in útero féminæ concéptus est, nostis quia ex útero natus est, nostis quia lactátus est, quia mánibus portátus est, quia circumcísus, quia hóstia pro illo obláta est, quia crevit; postrémo nostis quia expalmátus est, conspútus, spinis coronátus, crucifíxus est, mórtuus est, láncea percússus est; nostis quia hæc ómnia passus est: 
	\textit{Magnus}
	est
	\textit{et laudábilis nimis.}\\
	
	Nolíte contémnere parvum, intellégite magnum. Parvus factus est, quia parvi erátis; intellegátur magnus, et in illo magni éritis. Sic enim ædificátur domus, sic erigúntur moles in ipsa domo: crescunt lápides qui ducúntur ad ædifícium. Créscite ergo, intellégite Christum magnum; et parvus magnus est, magnus nimis. Finívit verba; volébat dícere quantum magnus  etsi tota die díceret : "Magnus, magnus", quid díceret? Tota die dicens : "Magnus" finíret aliquándo, quia finítur dies; magnitúdo illíus ante dies, ultra dies, sine die.
	\\
	
	Ergo quid díceret ?
	\textit{Quóniam magnus Dóminus et laudábilis nimis.}
	Quid enim dictúra est lingua parva ad laudándum magnum ? Dicéndo: 
	\textit{nimis,}
	emísit vocem, et dedit cogitatióni quod sápiat; tamquam dicens: " Quod sonáre non possum, tu cógita; et cum cogitáveris, parum erit". Quod cogitátio nullíus éxplicat, lingua alicúius éxplicat? 
	\textit{Magnus Dóminus et laudábilis nimis.}
	Ipse laudétur, ipse prædicétur, eius glória nuntiétur, et ædificátur domus.
	
	
	
\end{document}