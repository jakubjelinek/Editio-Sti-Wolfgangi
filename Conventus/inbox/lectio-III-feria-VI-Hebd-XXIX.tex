\documentclass[options]{article}
\begin{document}
	Ex Cántico spiritáli sancti Ioánnis a Cruce presbýteri.
	\begin{flushright}
		(Red. A, str. 38)
	\end{flushright} 
Anima uníta et in Deum transformáta spirat in Deo ad Deum altíssimam quandam aspiratiónem divínæ símilem, quam Deus in illa manens spirat in semetípso tamquam ipsíus exémplar. Quod quantum cápio, significáre vóluit sanctus Paulus cum dixit: 
\emph{Quóniam autem estis fílii, misit Deus Spíritum Fílii sui in corda vestra clamántem: Abba, Pater.}
Quod in viris perféctis áccidit.\\\\
 Nec mirum est quod rem ádeo sublímem ánima præstáre possit: cum Deus hoc illi cónferat benefícium ut ad deifórmem uniónem in Sanctíssima Trinitáte pervéniat, cur, quæso, incredíbile sit ipsam effícere opus suum intellegéntiæ, notítiæ et amóris in Trinitáte, una cum ipsa Trinitáte, idque máxima cum similitúdine ad ipsam, per participatiónem tamen, idípsum Deo in ipsa operánte?\\\\
  Quómodo autem hoc fiat, nulla ália poténtia nec sapiéntia éxprimi potest quam demonstrándo quemádmodum Dei Fílius hunc nobis sublímem statum et locum impetráverit atque promerúerit ut \emph{fílii Dei simus,} et ita pétiit a Patre: \emph{Pater, quos dedísti mihi, volo, ut ubi sum ego, et illi sint mecum,} efficiéndo nimírum idem opus quod ego fácio, per participatiónem.\\\\
     Quod fit communicándo cum illis eúndem amórem, quem commúnicat cum Fílio, non tamen naturáliter sicut Fílio, sed per unitátem et transformatiónem amóris, sicut nec étiam hic intellégitur quod Fílius dicat Patri ut sancti sint unum essentiáliter et naturáliter, sicut Pater et Fílius unum sunt, sed solum signíficat quod sint unum per uniónem amóris, quemádmodum Pater et Fílius unum in unitáte Amóris. Unde ánimæ éadem bona póssident per participatiónem, quæ illi póssident per natúram: quaprópter revéra dii sunt per participatiónem, símiles atque consórtes eiúsdem Dei. 
     
 

\end{document}