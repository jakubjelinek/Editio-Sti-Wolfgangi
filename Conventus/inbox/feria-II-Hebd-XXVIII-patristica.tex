\documentclass[options]{article}
\begin{document}
	Ex Tractátu sancti Fulgéntii Ruspénsis epíscopi Contra Fabiánum 
	\begin{flushright}
		 (Cap. 28, 16-19: CCL 91 A, 813-814)
	\end{flushright}
Illud implétur in sacrifíciis offeréndis, quod ipsum Salvatórem nostrum præcepísse beátus testátur Apóstolus dicens: 
\emph{Quóniam Dóminus Iesus in qua nocte tradebátur accépit panem et grátias agens fregit et dixit: Hoc est corpus meum pro vobis; hoc fácite in meam commemoratiónem. Simíliter et cálicem postquam cenávit, dicens: Hic calix novum testaméntum est in meo sánguine; hoc fácite, quotiescúmque bibétis, in meam commemoratiónem. Quotiescúmque enim manducábitis panem hunc et cálicem bibétis, mortem Dómini annuntiábitis donec véniat.}\\
 Ideo ígitur sacrifícium offértur, ut mors Dómini annuntiétur et eius fiat commemorátio, qui pro nobis pósuit ánimam suam. Ipse autem ait: 
 \emph{Maiórem hac dilectiónem nemo habet quam ut ánimam suam quis ponat pro amícis suis.}
 Quóniam ergo Christus pro nobis caritáte mórtuus est, cum témpore sacrifícii commemoratiónem mortis eius fácimus, caritátem nobis tríbui per advéntum Sancti Spíritus postulámus; hoc supplíciter exorántes, ut per ipsam caritátem qua pro nobis Christus crucifígi dignátus est, nos quoque grátia Sancti Spíritus accépta, mundum crucifíxum habére et mundo crucifígi possímus; imitantésque Dómini nostri mortem, sicut Christus,
 \emph{quod mórtuus est peccáto, mórtuus est semel, quod autem vivit, vivit Deo, étiam nos in novitáte vitæ ambulémus;}
  et múnere caritátis accépto, moriámur peccáto et vivámus Deo.
  \emph{Cáritas enim Dei diffúsa est in córdibus nostris per Spíritum Sanctum, qui datus est nobis.}\\
  \\
  RESP 2 Vota mea (new)\\
  \\
  Nam et ipsa participátio córporis et sánguinis Dómini, cum eius panem manducámus et cálicem bíbimus, hoc útique nobis insínuat, ut moriámur mundo et vitam nostram abscónditam habeámus cum Christo in Deo, carnémque nostram crucifigámus cum vítiis et concupiscéntiis suis.\\
    Sic fit ut omnes fidéles, qui Deum et próximum díligunt, etiámsi non bibant cálicem corpóreæ passiónis, bibant tamen cálicem domínicæ caritátis; quo inebriáti, membra sua quæ sunt super terram mortíficent et indúti Dóminum Iesum Christum, carnis curam non fáciant in desidériis; neque contempléntur quæ vidéntur, sed quæ non vidéntur. Sic enim calix Dómini bíbitur, dum sancta cáritas custodítur; sine qua, si corpus suum quisquam tradíderit ut árdeat, nihil ei prodest. Dono autem caritátis hoc nobis confértur, ut hoc in veritáte simus, quod in sacrifício mýstice celebrámus.\\
    \\
    RESP 3 : In illo die suscipiam (new) \\
    \\
    	
\end{document}