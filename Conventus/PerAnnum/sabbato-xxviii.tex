\newcommand{\titulus}{\nomenFesti{S. Teresiæ de Avila, Virginis \& Ecclesiæ Doctoris.}
\dies{Die 15. Octobris.}}
\newcommand{\oratio}{\pars{Oratio.}

\noindent Deus, qui per Spíritum tuum beátam Terésiam suscitásti, ut requiréndæ perfectiónis sémitam Ecclésiæ manifestáret, da nobis et cæléstis eius doctrínæ pábulo semper nutríri et veræ sanctitátis desidério accéndi.

\pars{Pro pace in Ucraina.} \scriptura{Sir. 50, 25; 2 Esdr. 4, 20; \textbf{H416}}

\vspace{-4mm}

\antiphona{II D}{temporalia/ant-dapacemdomine.gtex}

\vfill

\noindent Deus, a quo sancta desidéria, recta consília et iusta sunt ópera: da servis tuis illam, quam mundus dare non potest, pacem; ut et corda nostra mandátis tuis dédita, et hóstium subláta formídine, témpora sint tua protectióne tranquílla.

\noindent Per Dóminum nostrum Iesum Christum, Fílium tuum, qui tecum vivit et regnat in unitáte Spíritus Sancti, Deus, per ómnia sǽcula sæculórum.

\noindent \Rbardot{} Amen.}
\newcommand{\invitatorium}{\pars{Invitatorium.}

\vspace{-4mm}

\antiphona{IV}{temporalia/inv-fontemsapientiae.gtex}}
\newcommand{\hymnusmatutinum}{\pars{Hymnus}

\cuminitiali{VI}{temporalia/hym-DulciDepromat.gtex}}
\newcommand{\matversus}{\noindent \Vbardot{} Non cessámus pro vobis orántes et postulántes.

\noindent \Rbardot{} Ut impleámini agnitióne voluntátis Dei.}
\newcommand{\lectioi}{\pars{Lectio I.} \scriptura{Mal. 3, 1-24}

\noindent De libro Malachíæ prophétæ.

\noindent Hæc dicit Dóminus Deus: «Ecce ego mittam ángelum meum, et præparábit viam ante fáciem meam; et statim véniet ad templum suum Dominátor, quem vos quǽritis, et ángelus testaménti, quem vos vultis. Ecce venit, dicit Dóminus exercítuum; et quis póterit sustinére diem advéntus eius, et quis stabit, cum apparébit? Ipse enim quasi ignis conflans et quasi herba fullónum; et sedébit conflans et emúndans argéntum et purgábit fílios Levi et colábit eos quasi aurum et quasi argéntum, et erunt Dómino offeréntes sacrifícia in iustítia. Et placébit Dómino sacrifícium Iudæ et Ierúsalem sicut diébus prístinis et sicut annis antíquis. Et accédam ad vos in iudício; et ero testis velox maléficis et adúlteris et periúris et, qui ópprimunt mercenários, víduas et pupíllos et flectunt ius peregrinórum nec timuérunt me, dicit Dóminus exercítuum.

\noindent Ego enim Dóminus et non mutátus sum; sed vos, fílii Iacob, nondum ad finem pervenístis. A diébus enim patrum vestrórum recessístis a præcéptis legítimis meis et non custodístis ea. Revertímini ad me, et revértar ad vos, dicit Dóminus exercítuum.

\noindent Et dícitis: “In quo revertémur?”. Numquid homo potest defraudáre Deum?

\noindent Sed vos defraudátis me. Et dícitis: “In quo defraudávimus te?”.

\noindent In décimis et in primítiis. Maledictióne vos maledícti estis, quia me vos defraudátis, gens tota. Inférte omnem décimam in hórreum, et sit cibus in domo mea; et probáte me super hoc, dicit Dóminus exercítuum: si non aperúero vobis cataráctas cæli et effúdero vobis benedictiónem usque ad abundántiam et increpábo pro vobis devorántem, et non corrúmpet fructum terræ, nec erit stérilis vobis vínea in agro, dicit Dóminus exercítuum.

\noindent Et beátos vos dicent omnes gentes; éritis enim vos terra desiderábilis, dicit Dóminus exercítuum.

\noindent {\color{gray} Invaluérunt super me verba vestra, dicit Dóminus; et dícitis: “Quid locúti sumus contra te?”. Dícitis: “Vanum est servíre Deo; et, quod emoluméntum, quia custodívimus præcépta eius et quia ambulávimus tristes coram Dómino exercítuum? Ergo nunc beátos dícimus arrogántes; síquidem ædificáti sunt faciéntes impietátem et tentavérunt Deum et salvi facti sunt”. Tunc locúti sunt timéntes Dóminum, unusquísque cum próximo suo. Et atténdit Dóminus et audívit; et scriptus est liber memorabílium coram eo timéntibus Dóminum et cogitántibus nomen eius. Erunt mihi, ait Dóminus exercítuum, in die, qua ego fácio in pecúlium; et parcam eis, sicut parcit vir fílio suo serviénti sibi. Rursum vidébitis quid sit inter iustum et ímpium, inter serviéntem Deo et non serviéntem ei.

\noindent Ecce enim dies véniet succénsa quasi camínus; et erunt omnes supérbi et omnes faciéntes impietátem stípula; et inflammábit eos dies véniens, dicit Dóminus exercítuum, quæ non derelínquet eis radícem et ramum. Et oriétur vobis timéntibus nomen meum sol iustítiæ et sánitas in pennis eius; et egrediémini et saliétis sicut vítuli sagináti et calcábitis ímpios, cum fúerint cinis sub planta pedum vestrórum in die, quam ego fácio, dicit Dóminus exercítuum.

\noindent Mementóte legis Móysi servi mei, cui mandávi in Horeb ad omnem Israel præcépta et iudícia. Ecce ego mittam vobis Elíam prophétam, ántequam véniat dies Dómini magnus et horríbilis; et convértet cor patrum ad fílios et cor filiórum ad patres eórum, ne véniam et percútiam terram anathémate».}}
\newcommand{\responsoriumi}{\pars{Responsorium 1.} \scriptura{\Rbardot{} Cf. Ap. 2, 17 \Vbardot{} Cf. Mal. 3, 1; \textbf{H49}}

\vspace{-5mm}

\responsorium{III}{temporalia/resp-hicquiadvenit-CROCHU.gtex}{}

\rubrica{vel ad libitum:}

\vspace{3mm}

\pars{Responsorium 1.} \scriptura{\Rbar{} Is. 43, 14.15 \Vbar{} Cantor; \textbf{H22}}

\vspace{-5mm}

\responsorium{V}{temporalia/resp-eccevenietdominus-CROCHU.gtex}{}}
\newcommand{\lectioii}{\pars{Lectio II.} \scriptura{Opusc. De libro vitæ, cap. 22, 6-7. 14}

\noindent Ex Opéribus sanctæ Terésiæ a Iesu vírginis.

\noindent Qui Christum Iesum, amícum et tam generósum ducem, apud se habet, ómnia ferre potest: ipse enim nos iuvat, corróborat, néminem destítuit, ac verus et sincérus amícus est. Maniféste enim vídeo, si Deo placére et magnas ab illo grátias accípere velímus, ipsum velle ut per sacratíssimæ illíus humanitátis, in qua maiéstas eius sibi complacére dixit manus illæ nobis obvéniant.

\noindent Sæpíssime hoc per experiéntiam advérti, quin et Dóminus ipse hoc mihi dixit; ad óculum, inquam, vidi per hanc nos portam íngredi debére, si magna quædam a supréma maiestáte arcána et secréta pandi nobis velímus. Alia via, summo etsi iam adépto contemplatiónis vértice, non est quærénda. Hac quippe via tuto et secúre incéditur. Dóminus hic noster is est a quo et per quem ómnia nobis bona obvéniunt: is nos docébit. Illíus si intueámur vitam, mélius ac perféctius quod imitémur exémplum non inveniémus.}
\newcommand{\responsoriumii}{\pars{Responsorium 2.} \scriptura{\Vbar{} Ps. 44, 10; \textbf{H350}}

\vspace{-5mm}

\responsorium{VIII}{temporalia/resp-virgogloriosa-CROCHU.gtex}{}}
\newcommand{\lectioiii}{\pars{Lectio III.}

\noindent Quid vólumus ámplius, quam tam fidum ad latus nostrum amícum habére, qui in labóribus et difficultátibus nos, sicut amíci mundáni fáciunt, non déseret? Beátus ille qui eum sincére et vere amábit et iuxta se semper tenébit. Intueámur gloriósum apóstolum Paulum, qui videbátur Iesum in ore haud tenére non posse, quippe qui illum cordi suo inscúlptum et impréssum habéret. Hoc autem intellécto, diligénter inquisívi et dídici non áliam ac hanc viam ingréssos esse sanctos quosdam mire contemplatívos, ut Francíscum, Antónium Patavínum, Bernárdum, Catharínam Senénsem. Magna cum libertáte per hanc viam ambuláre opórtet, in Dei manus totos nos resignántes. Si ille nos inter suos cubiculários et secretários suos reférre atque evéhere velit, libénter obsequámur.

\noindent Quotiescúmque autem de Christo cogitámus, semper mémores amóris illíus simus, quo ipse actus tantas nobis grátias et benefícia prǽstitit, et quantum Deus amórem nobis osténderit, cum tale nobis amóris illíus quo nos proséquitur pignus déderit: amor enim amórem póstulat. Quámobrem óperam demus ut hoc semper præ óculis habeámus et ita nos ad amórem provocémus. Nam si semel hanc nobis grátiam Deus fáciat, ut amor hic cordi nostro profúnde imprimátur, ómnia nobis facíllima erunt, et brevíssimo témpore et parvo labóre permúltum præstábimus.}
\newcommand{\responsoriumiii}{\pars{Responsorium 3.}

\vspace{-5mm}

\responsorium{VIII}{temporalia/resp-ipsisumdesponsata-CROCHU-cumdox.gtex}{}}
\newcommand{\hymnuslaudes}{\pars{Hymnus}

\cuminitiali{VIII}{temporalia/hym-RegisSuperni.gtex}}
\newcommand{\lectiobrevis}{\pars{Lectio Brevis.} \scriptura{Ct. 8, 7}

\noindent Aquæ multæ non potuérunt exstínguere caritátem, nec flúmina óbruent illam; si déderit homo omnem substántiam domus suæ pro dilectióne, quasi nihil despícient eum.}
\newcommand{\responsoriumbreve}{\pars{Responsorium breve.} \scriptura{Ps. 45, 6}

\cuminitiali{VI}{temporalia/resp-adiuvabiteam.gtex}}
\newcommand{\preces}{\noindent Christo, vírginum sponso et corónæ, lætis vócibus iubilémus \gredagger{} et devóte eum exorémus, dicéntes:

\Rbardot{} Iesu, coróna vírginum, audi nos.

\noindent Christe, quem sanctæ vírgines ut únicum sponsum dilexérunt, \gredagger{} concéde ut nihil a tua caritáte nos séparet.

\Rbardot{} Iesu, coróna vírginum, audi nos.

\noindent Qui Maríam, matrem tuam, regínam vírginum coronásti, \gredagger{} da nos, eius intercessióne, cordis puritáte tibi iúgiter famulári.

\Rbardot{} Iesu, coróna vírginum, audi nos.

\noindent Per intercessiónem ancillárum tuárum, quæ íntegro et indivíso corde semper de te sollícitæ fuérunt, ut essent sanctæ córpore et spíritu, \gredagger{} præsta ut figúra huius mundi, quæ prǽterit, nos a te numquam ábstrahat.

\Rbardot{} Iesu, coróna vírginum, audi nos.

\noindent Dómine Iesu, sponse, quem vírgines sapiéntes adventúrum exspectábant, \gredagger{} concéde nobis, ut in spe vigilántes te præstolémur.

\Rbardot{} Iesu, coróna vírginum, audi nos.

\noindent Per intercessiónem sanctæ Terésiæ, quæ virgo fuit sápiens et de número prudéntum, \gredagger{} præsta nobis innocéntiam vitæ et sapiéntiam.

\Rbardot{} Iesu, coróna vírginum, audi nos.}
\newcommand{\benedictus}{\pars{Canticum Zachariæ.} \scriptura{Cf. Prv. 8, 31}

\vspace{-4mm}

\antiphona{IV e}{temporalia/ant-obeataanima.gtex}

%\vspace{-2mm}

\scriptura{Lc. 1, 68-79}

%\vspace{-2mm}

\cantusSineNeumas
\initiumpsalmi{temporalia/benedictus-initium-iv-e2-auto.gtex}

%\vspace{-1.5mm}

\input{temporalia/benedictus-iv-e2.tex} \Abardot{}}
\newcommand{\precestotum}{\pars{Deprecatio Gelasii}

\vspace{-5mm}

\grechangedim{interwordspacetext}{0.16 cm plus 0.15 cm minus 0.05 cm}{scalable}%
\antiphona{D\textsuperscript{1}}{temporalia/deprecatio4-propace.gtex}
\grechangedim{interwordspacetext}{0.22 cm plus 0.15 cm minus 0.05 cm}{scalable}%

\vfill

\pars{Oratio Dominica.}

\cuminitiali{D}{temporalia/oratiodominica-d.gtex}}
\newcommand{\dominusnosbenedicat}{\antiphona{D}{temporalia/dominusnosbenedicat-d.gtex}}
\newcommand{\benedicamuslaudes}{\cuminitiali{}{temporalia/benedicamus-memoria-laudes.gtex}}
\include{hebdomadaxxviii}
% LuaLaTeX

\documentclass[a4paper, twoside, 12pt]{article}
\usepackage[latin]{babel}
%\usepackage[landscape, left=3cm, right=1.5cm, top=2cm, bottom=1cm]{geometry} % okraje stranky
%\usepackage[landscape, a4paper, mag=1166, truedimen, left=2cm, right=1.5cm, top=1.6cm, bottom=0.95cm]{geometry} % okraje stranky
\usepackage[landscape, a4paper, mag=1400, truedimen, left=0.5cm, right=0.5cm, top=0.5cm, bottom=0.5cm]{geometry} % okraje stranky

\usepackage{fontspec}
\setmainfont[FeatureFile={junicode.fea}, Ligatures={Common, TeX}, RawFeature=+fixi]{Junicode}
%\setmainfont{Junicode}

% shortcut for Junicode without ligatures (for the Czech texts)
\newfontfamily\nlfont[FeatureFile={junicode.fea}, Ligatures={Common, TeX}, RawFeature=+fixi]{Junicode}

% Hebrew font: http://scripts.sil.org/cms/scripts/page.php?site_id=nrsi&id=SILHebrUnic2
\newfontfamily\hebfont[Scale=1]{Ezra SIL}

\usepackage{multicol}
\usepackage{color}
\usepackage{lettrine}
\usepackage{fancyhdr}

% usual packages loading:
\usepackage{luatextra}
\usepackage{graphicx} % support the \includegraphics command and options
\usepackage{gregoriotex} % for gregorio score inclusion
\usepackage{gregoriosyms}
\usepackage{wrapfig} % figures wrapped by the text
\usepackage{parcolumns}
\usepackage[contents={},opacity=1,scale=1,color=black]{background}
\usepackage{tikzpagenodes}
\usepackage{calc}
\usepackage{longtable}
\usetikzlibrary{calc}

\setlength{\headheight}{14.5pt}

\input{conventuscommune.tex} % Often used macros

\newcommand{\annusEditionis}{2022}

\def\hebinitial#1{%
\leavevmode{\newbox\hebbox\setbox\hebbox\hbox{\hebfont{#1}\hskip 1mm}\kern -\wd\hebbox\hbox{\hebfont{#1}\hskip 1mm}}%
}

%%%% Vicekrat opakovane kousky

\newcommand{\anteOrationem}{
  \rubrica{Ante Orationem, cantatur a Superiore:}

  \pars{Supplicatio Litaniæ.}

  \cuminitiali{}{temporalia/supplicatiolitaniae.gtex}

  \pars{Oratio Dominica.}

  \cuminitiali{}{temporalia/oratiodominica.gtex}

  \rubrica{Deinde dicitur ab Hebdomadario:}

  \cuminitiali{}{temporalia/dominusvobiscum-solemnis.gtex}

  \rubrica{In choro monialium loco Dominus vobiscum dicitur:}

  \sineinitiali{temporalia/domineexaudi.gtex}
}

\setlength{\columnsep}{30pt} % prostor mezi sloupci

%%%%%%%%%%%%%%%%%%%%%%%%%%%%%%%%%%%%%%%%%%%%%%%%%%%%%%%%%%%%%%%%%%%%%%%%%%%%%%%%%%%%%%%%%%%%%%%%%%%%%%%%%%%%%
\begin{document}

% Here we set the space around the initial.
% Please report to http://home.gna.org/gregorio/gregoriotex/details for more details and options
\grechangedim{afterinitialshift}{2.2mm}{scalable}
\grechangedim{beforeinitialshift}{2.2mm}{scalable}
\grechangedim{interwordspacetext}{0.22 cm plus 0.15 cm minus 0.05 cm}{scalable}%
\grechangedim{annotationraise}{-0.2cm}{scalable}

% Here we set the initial font. Change 38 if you want a bigger initial.
% Emit the initials in red.
\grechangestyle{initial}{\color{red}\fontsize{38}{38}\selectfont}

\pagestyle{empty}

%%%% Titulni stranka
\begin{titulusOfficii}
\ifx\titulus\undefined
\nomenFesti{Sabbato \hebdomada{}}
\else
\titulus
\fi
\end{titulusOfficii}

\vfill

\begin{center}
%Ad usum et secundum consuetudines chori \guillemotright{}Conventus Choralis\guillemotleft.

%Editio Sancti Wolfgangi \annusEditionis
\end{center}

\scriptura{}

\pars{}

\pagebreak

\renewcommand{\headrulewidth}{0pt} % no horiz. rule at the header
\fancyhf{}
\pagestyle{fancy}

\cantusSineNeumas

\hora{Ad Matutinum.} %%%%%%%%%%%%%%%%%%%%%%%%%%%%%%%%%%%%%%%%%%%%%%%%%%%%%

\vspace{2mm}

\cuminitiali{}{temporalia/dominelabiamea.gtex}

\vfill
%\pagebreak

\vspace{2mm}

\ifx\invitatorium\undefined
\pars{Invitatorium.} \scriptura{Lc. 24, 34; Psalmus 94; \textbf{H232}}

\vspace{-4mm}

\antiphona{VI}{temporalia/inv-surrexitdominusvere.gtex}
\else
\invitatorium
\fi

\vfill
\pagebreak

\ifx\hymnusmatutinum\undefined
\pars{Hymnus.}

\cuminitiali{VIII}{temporalia/hym-LaetareCaelum.gtex}
\else
\hymnusmatutinum
\fi

\vspace{-3mm}

\vfill
\pagebreak

\ifx\matutinum\undefined
\ifx\matua\undefined
\else
% MAT A
\pars{Psalmus 1.}

\vspace{-4mm}

\antiphona{VIII G\textsuperscript{5}}{temporalia/ant-alleluia-turco15.gtex}

\vspace{-3mm}

\scriptura{Ps. 104, 1-15}

\vspace{-2mm}

\initiumpsalmi{temporalia/ps104i-initium-viii-g5.gtex}

\vspace{-1.5mm}

\input{temporalia/ps104i-viii-g.tex}

\vfill
\pagebreak

\pars{Psalmus 2.} \scriptura{Ps. 104, 16-27}

%\vspace{-2mm}

\initiumpsalmi{temporalia/ps104ii-initium-viii-g5.gtex}

\input{temporalia/ps104ii-viii-g.tex}

\vfill
\pagebreak

\pars{Psalmus 3.} \scriptura{Ps. 104, 28-45}

%\vspace{-2mm}

\initiumpsalmi{temporalia/ps104iii-initium-viii-g5.gtex}

\input{temporalia/ps104iii-viii-g.tex}

\vfill

\antiphona{}{temporalia/ant-alleluia-turco15.gtex}

\vfill
\pagebreak
\fi
\ifx\matub\undefined
\else
% MAT B
\pars{Psalmus 1.}

\vspace{-4mm}

\antiphona{t. pereg.}{temporalia/ant-alleluia-turco3.gtex}

%\vspace{-2mm}

\scriptura{Ps. 105, 1-15}

%\vspace{-2mm}

\initiumpsalmi{temporalia/ps105i-initium-per-auto.gtex}

\input{temporalia/ps105i-per.tex}

\vfill
\pagebreak

\pars{Psalmus 2.} \scriptura{Ps. 105, 16-31}

\vspace{-2.5mm}

\initiumpsalmi{temporalia/ps105ii-initium-per-auto.gtex}

\vspace{-1.5mm}

\input{temporalia/ps105ii-per.tex}

\vfill
\pagebreak

\pars{Psalmus 3.} \scriptura{Ps. 105, 32-48}

%\vspace{-2mm}

\initiumpsalmi{temporalia/ps105iii-initium-per-auto.gtex}

\input{temporalia/ps105iii-per.tex}

\vfill

\antiphona{}{temporalia/ant-alleluia-turco3.gtex}

\vfill
\pagebreak
\fi
\ifx\matuc\undefined
\else
% MAT C
\pars{Psalmus 1.} \scriptura{Ps. 106, 8}

\vspace{-4mm}

\antiphona{IV e}{temporalia/ant-alleluia-fo2.gtex}

%\vspace{-2mm}

\scriptura{Ps. 106, 1-14}

%\vspace{-2mm}

\initiumpsalmi{temporalia/ps106i-initium-iv-e2-auto.gtex}

\input{temporalia/ps106i-iv-e2.tex}

\vfill
\pagebreak

\pars{Psalmus 2.} \scriptura{Ps. 106, 15-30}

%\vspace{-2mm}

\initiumpsalmi{temporalia/ps106ii-initium-iv-e2-auto.gtex}

\input{temporalia/ps106ii-iv-e2.tex}

\vfill
\pagebreak

\pars{Psalmus 3.} \scriptura{Ps. 106, 31-43}

%\vspace{-2mm}

\initiumpsalmi{temporalia/ps106iii-initium-iv-e2-auto.gtex}

\input{temporalia/ps106iii-iv-e2.tex}

\vfill
\pagebreak

\antiphona{}{temporalia/ant-alleluia-fo2.gtex}

\vfill
\pagebreak
\fi
\ifx\matud\undefined
\else
% MAT D
\pars{Psalmus 1.}

\vspace{-4mm}

\antiphona{III g}{temporalia/ant-alleluia-turco26.gtex}

%\vspace{-2mm}

\scriptura{Ps. 77, 40-51}

%\vspace{-2mm}

\initiumpsalmi{temporalia/ps77xl_li-initium-iii-g-auto.gtex}

\input{temporalia/ps77xl_li-iii-g.tex}

\vfill
\pagebreak

\pars{Psalmus 2.} \scriptura{Ps. 77, 52-64}

\vspace{-2mm}

\initiumpsalmi{temporalia/ps77lii_lxiv-initium-iii-g-auto.gtex}

\input{temporalia/ps77lii_lxiv-iii-g.tex}

\vfill
\pagebreak

\pars{Psalmus 3.} \scriptura{Ps. 77, 65-72}

%\vspace{-2mm}

\initiumpsalmi{temporalia/ps77lxv_lxxii-initium-iii-g-auto.gtex}

\input{temporalia/ps77lxv_lxxii-iii-g.tex}

\vfill

\antiphona{}{temporalia/ant-alleluia-turco26.gtex}

\vfill
\pagebreak
\fi
\else
\matutinum
\fi

\pars{Versus.}

\ifx\matversus\undefined
\noindent \Vbardot{} Deus regenerávit nos in spem vivam, allelúia.

\noindent \Rbardot{} Per resurrectiónem Iesu Christi ex mórtuis, allelúia.
\else
\matversus
\fi

\vspace{5mm}

\sineinitiali{temporalia/oratiodominica-mat.gtex}

\vspace{5mm}

\pars{Absolutio.}

\cuminitiali{}{temporalia/absolutio-avinculis.gtex}

\vfill
\pagebreak

\cuminitiali{}{temporalia/benedictio-solemn-ille.gtex}

\vspace{7mm}

\lectioi

\noindent \Vbardot{} Tu autem, Dómine, miserére nobis.
\noindent \Rbardot{} Deo grátias.

\vfill
\pagebreak

\responsoriumi

\vfill
\pagebreak

\cuminitiali{}{temporalia/benedictio-solemn-divinum.gtex}

\vspace{7mm}

\lectioii

\noindent \Vbardot{} Tu autem, Dómine, miserére nobis.
\noindent \Rbardot{} Deo grátias.

\vfill
\pagebreak

\responsoriumii

\vfill
\pagebreak

\cuminitiali{}{temporalia/benedictio-solemn-adsocietatem.gtex}

\vspace{7mm}

\lectioiii

\noindent \Vbardot{} Tu autem, Dómine, miserére nobis.
\noindent \Rbardot{} Deo grátias.

\vfill
\pagebreak

\responsoriumiii

\vfill
\pagebreak

\rubrica{Reliqua omittuntur, nisi Laudes separandæ sint.}

\sineinitiali{temporalia/domineexaudi.gtex}

\vfill

\oratio

\vfill

\noindent \Vbardot{} Dómine, exáudi oratiónem meam.
\Rbardot{} Et clamor meus ad te véniat.

\vfill

\noindent \Vbardot{} Benedicámus Dómino.
\noindent \Rbardot{} Deo grátias.

\vfill

\noindent \Vbardot{} Fidélium ánimæ per misericórdiam Dei requiéscant in pace.
\Rbardot{} Amen.

\vfill
\pagebreak

\hora{Ad Laudes.} %%%%%%%%%%%%%%%%%%%%%%%%%%%%%%%%%%%%%%%%%%%%%%%%%%%%%

\cantusSineNeumas

\vspace{0.5cm}
\grechangedim{interwordspacetext}{0.18 cm plus 0.15 cm minus 0.05 cm}{scalable}%
\cuminitiali{}{temporalia/deusinadiutorium-communis.gtex}
\grechangedim{interwordspacetext}{0.22 cm plus 0.15 cm minus 0.05 cm}{scalable}%

\vfill
\pagebreak

\ifx\hymnuslaudes\undefined
\ifx\laudac\undefined
\else
\pars{Hymnus}

\cuminitiali{I}{temporalia/hym-ChorusNovae-praglia.gtex}
\vspace{-3mm}
\fi
\ifx\laudbd\undefined
\else
\pars{Hymnus}

\cuminitiali{I}{temporalia/hym-ChorusNovae.gtex}
\vspace{-3mm}
\fi
\else
\hymnuslaudes
\fi

\vfill
\pagebreak

\ifx\laudes\undefined
\ifx\lauda\undefined
\else
\pars{Psalmus 1.}

\vspace{-4mm}

\antiphona{VII a}{temporalia/ant-alleluia-turco29.gtex}

\scriptura{Psalmus 118, 145-152; \hspace{5mm} \hebinitial{ק}}

\initiumpsalmi{temporalia/ps118xix-initium-vii-a-auto.gtex}

\input{temporalia/ps118xix-vii-a.tex} \Abardot{}

\vfill
\pagebreak

\pars{Psalmus 2.} \scriptura{Ex. 15, 2}

\vspace{-4mm}

\antiphona{IV e}{temporalia/ant-fortitudomeaetlausmea.gtex}

\scriptura{Canticum Moysis, Ex. 15, 1-4a.7b-13.17-19}

\initiumpsalmi{temporalia/moysis1-initium-iv-e2-auto.gtex}

\input{temporalia/moysis1-iv-e2.tex}

\antiphona{}{temporalia/ant-fortitudomeaetlausmea.gtex}

\vfill
\pagebreak

\pars{Psalmus 3.}

\vspace{-4mm}

\antiphona{E}{temporalia/ant-alleluia-praglia-e2.gtex}

\scriptura{Psalmus 116.}

\initiumpsalmi{temporalia/ps116-initium-e-auto.gtex}

\input{temporalia/ps116-e.tex} \Abardot{}

\vfill
\pagebreak
\fi
\ifx\laudb\undefined
\else
\pars{Psalmus 1.}

\vspace{-4.5mm}

\antiphona{E}{temporalia/ant-alleluia-praglia-e2.gtex}

\vspace{-3mm}

\scriptura{Psalmus 91.}

\vspace{-2mm}

\initiumpsalmi{temporalia/ps91-initium-e-auto.gtex}

\vspace{-1.5mm}

\input{temporalia/ps91-e.tex} \Abardot{}

\vfill
\pagebreak

\pars{Psalmus 2.} \scriptura{Eccli. 39, 19}

\vspace{-4mm}

\antiphona{VII c\textsuperscript{2}}{temporalia/ant-effrondeteingratia.gtex}

\vspace{-2mm}

\scriptura{Canticum Moysi, Dt. 32, 1-32}

\vspace{-2mm}

\initiumpsalmi{temporalia/moysis2i_xii-initium-vii-c2-auto.gtex}

\input{temporalia/moysis2i_xii-vii-c2.tex}

\vfill

\antiphona{}{temporalia/ant-effrondeteingratia.gtex}

\vfill
\pagebreak

\pars{Psalmus 3.}

\vspace{-4mm}

\antiphona{I a\textsuperscript{2}}{temporalia/ant-alleluia-turco23.gtex}

%\vspace{-2mm}

\scriptura{Ps. 8}

%\vspace{-2mm}

\initiumpsalmi{temporalia/ps8-initium-i-a2-auto.gtex}

\input{temporalia/ps8-i-a2.tex} \Abardot{}

\vfill
\pagebreak
\fi
\ifx\laudc\undefined
\else
\pars{Psalmus 1.}

\vspace{-4mm}

\antiphona{E}{temporalia/ant-alleluia-praglia-e2.gtex}

%\vspace{-2mm}

\scriptura{Psalmus 118, 145-152.}

%\vspace{-2mm}

\initiumpsalmi{temporalia/ps118xix-initium-e-auto.gtex}

%\vspace{-1.5mm}

\input{temporalia/ps118xix-e.tex} \Abardot{}

\vfill
\pagebreak

\pars{Psalmus 2.}

\vspace{-4mm}

\antiphona{V a}{temporalia/ant-mecumsitdomine-tp.gtex}

%\vspace{-2mm}

\scriptura{Canticum Sapientiæ, Sap. 9, 1-6.9-11}

\initiumpsalmi{temporalia/sapientia-initium-v-a-auto.gtex}

\input{temporalia/sapientia-v-a.tex} \Abardot{}

\vfill
\pagebreak

\pars{Psalmus 3.}

\vspace{-4mm}

\antiphona{II* a}{temporalia/ant-alleluia-turco18.gtex}

%\vspace{-2mm}

\scriptura{Ps. 116}

%\vspace{-2mm}

\initiumpsalmi{temporalia/ps116-initium-ii_-a-auto.gtex}

\input{temporalia/ps116-ii_-a.tex} \Abardot{}

\vfill
\pagebreak
\fi
\ifx\laudd\undefined
\else
\pars{Psalmus 1.}

\vspace{-4.5mm}

\antiphona{VIII G\textsuperscript{2}}{temporalia/ant-alleluia-turco12.gtex}

\vspace{-3mm}

\scriptura{Psalmus 91.}

\vspace{-2mm}

\initiumpsalmi{temporalia/ps91-initium-viii-G5-auto.gtex}

\vspace{-1.5mm}

\input{temporalia/ps91-viii-G5.tex} \Abardot{}

\vfill
\pagebreak

\pars{Psalmus 2.} \scriptura{Heb. 13, 8}

\vspace{-4mm}

\antiphona{II D}{temporalia/ant-iesuschristusheriethodie.gtex}

%\vspace{-2mm}

\scriptura{Canticum Ezechiæ, Ez. 36, 24-28}

\initiumpsalmi{temporalia/ezechiae2-initium-ii-D-auto.gtex}

\input{temporalia/ezechiae2-ii-D.tex} \Abardot{}

\vfill
\pagebreak

\pars{Psalmus 3.}

\vspace{-4mm}

\antiphona{I a\textsuperscript{2}}{temporalia/ant-alleluia-turco23.gtex}

%\vspace{-2mm}

\scriptura{Ps. 8}

%\vspace{-2mm}

\initiumpsalmi{temporalia/ps8-initium-i-a4-auto.gtex}

\input{temporalia/ps8-i-a4.tex} \Abardot{}

\vfill
\pagebreak
\fi
\else
\laudes
\fi

\ifx\lectiobrevis\undefined
\pars{Lectio Brevis.} \scriptura{Rom. 14, 7-9}

\noindent Nemo nostrum sibi vivit et nemo sibi móritur; sive enim vívimus, Dómino vívimus, sive mórimur, Dómino mórimur. Sive ergo vívimus, sive mórimur, Dómini sumus. In hoc enim Christus et mórtuus est et vixit, ut et mortuórum et vivórum dominétur.
\else
\lectiobrevis
\fi

\vfill

\ifx\responsoriumbreve\undefined
\pars{Responsorium breve.} \scriptura{Cf. Mt. 28, 6; Cf. Gal. 3, 13}

\cuminitiali{VI}{temporalia/resp-surrexitdominusdesepulcro.gtex}
\else
\responsoriumbreve
\fi

\vfill
\pagebreak

\benedictus

\vspace{-1cm}

\vfill
\pagebreak

\ifx\precestotum\undefined
\pars{Preces.}

\sineinitiali{}{temporalia/tonusprecumnovum.gtex}

\ifx\preces\undefined
\ifx\lauda\undefined
\else
\noindent Christum, panem vitæ, \gredagger{} qui mensa verbi et córporis sui fruéntes suscitábit in novíssimo die, \grestar{} læti deprecémur:

\Rbardot{} Da nobis, Dómine, pacem et gáudium.

\noindent Fili Dei, qui, suscitátus a mórtuis, princeps es vitæ, \grestar{} nos omnésque fratres tuos bénedic et sanctífica.

\Rbardot{} Da nobis, Dómine, pacem et gáudium.

\noindent Tu, qui pacem et gáudium ómnibus in te credéntibus largíris, \grestar{} da nos sicut fílios lucis ambuláre et de victória tua lætári.

\Rbardot{} Da nobis, Dómine, pacem et gáudium.

\noindent Adáuge fidem Ecclésiæ peregrinántis in terra, \grestar{} ut resurrectiónis tuæ testimónium mundo perhíbeat.

\Rbardot{} Da nobis, Dómine, pacem et gáudium.

\noindent Tu qui, multa passus, \gredagger{} in glóriam Patris intrásti, \grestar{} luctum mæréntium convérte in gáudium.

\Rbardot{} Da nobis, Dómine, pacem et gáudium.
\fi
\ifx\laudb\undefined
\else
\noindent Christum, qui vitam ætérnam nobis manifestávit, \grestar{} devóta mente rogémus, clamántes:

\Rbardot{} Resurréctio tua locuplétet nos grátia, Dómine.

\noindent Pastor ætérne, \gredagger{} réspice gregem tuum e somno surgéntem \grestar{} et pasce nos verbi et panis tui ubérrimo alimónio.

\Rbardot{} Resurréctio tua locuplétet nos grátia, Dómine.

\noindent Ne permíttas nos a lupo rapi vel a mercenário perdi, \grestar{} sed fac, ut vocem tuam fidéliter audiámus.

\Rbardot{} Resurréctio tua locuplétet nos grátia, Dómine.

\noindent Tu, qui cum prædicatóribus ubíque cooperáris eorúmque sermónem confírmas, \grestar{} fac, ut hódie resurrectiónem tuam móribus et vita proclamémus.

\Rbardot{} Resurréctio tua locuplétet nos grátia, Dómine.

\noindent Esto ipse gáudium nostrum, \gredagger{} quod nemo tollat a nobis, \grestar{} ut, reiécta tristítia peccáti, vitam appetámus ætérnam.

\Rbardot{} Resurréctio tua locuplétet nos grátia, Dómine.
\fi
\ifx\laudc\undefined
\else
\noindent Christum, panem vitæ, \gredagger{} qui mensa verbi et córporis sui fruéntes suscitábit in novíssimo die, \grestar{} læti deprecémur:

\Rbardot{} Da nobis, Dómine, pacem et gáudium.

\noindent Fili Dei, qui, suscitátus a mórtuis, princeps es vitæ, \grestar{} nos omnésque fratres tuos bénedic et sanctífica.

\Rbardot{} Da nobis, Dómine, pacem et gáudium.

\noindent Tu, qui pacem et gáudium ómnibus in te credéntibus largíris, \grestar{} da nos sicut fílios lucis ambuláre et de victória tua lætári.

\Rbardot{} Da nobis, Dómine, pacem et gáudium.

\noindent Adáuge fidem Ecclésiæ peregrinántis in terra, \grestar{} ut resurrectiónis tuæ testimónium mundo perhíbeat.

\Rbardot{} Da nobis, Dómine, pacem et gáudium.

\noindent Tu qui, multa passus, \gredagger{} in glóriam Patris intrásti, \grestar{} luctum mæréntium convérte in gáudium.

\Rbardot{} Da nobis, Dómine, pacem et gáudium.
\fi
\ifx\laudd\undefined
\else
\noindent Christum, qui vitam ætérnam nobis manifestávit, \grestar{} devóta mente rogémus, clamántes:

\Rbardot{} Resurréctio tua locuplétet nos grátia, Dómine.

\noindent Pastor ætérne, \gredagger{} réspice gregem tuum e somno surgéntem \grestar{} et pasce nos verbi et panis tui ubérrimo alimónio.

\Rbardot{} Resurréctio tua locuplétet nos grátia, Dómine.

\noindent Ne permíttas nos a lupo rapi vel a mercenário perdi, \grestar{} sed fac, ut vocem tuam fidéliter audiámus.

\Rbardot{} Resurréctio tua locuplétet nos grátia, Dómine.

\noindent Tu, qui cum prædicatóribus ubíque cooperáris eorúmque sermónem confírmas, \grestar{} fac, ut hódie resurrectiónem tuam móribus et vita proclamémus.

\Rbardot{} Resurréctio tua locuplétet nos grátia, Dómine.

\noindent Esto ipse gáudium nostrum, \gredagger{} quod nemo tollat a nobis, \grestar{} ut, reiécta tristítia peccáti, vitam appetámus ætérnam.

\Rbardot{} Resurréctio tua locuplétet nos grátia, Dómine.
\fi
\else
\preces
\fi

\vfill

\pars{Oratio Dominica.}

\cuminitiali{}{temporalia/oratiodominicaalt.gtex}

\vfill
\pagebreak

\rubrica{vel:}

\pars{Deprecatio Gelasii}

\vspace{-5mm}

\grechangedim{interwordspacetext}{0.16 cm plus 0.15 cm minus 0.05 cm}{scalable}%
\antiphona{D\textsuperscript{1}}{temporalia/deprecatio4-propace.gtex}
\grechangedim{interwordspacetext}{0.22 cm plus 0.15 cm minus 0.05 cm}{scalable}%

\vfill

\pars{Oratio Dominica.}

\cuminitiali{D}{temporalia/oratiodominica-d.gtex}
\else
\precestotum
\fi

\vfill
\pagebreak

% Oratio. %%%
\oratio

\vspace{-1mm}

\vfill

\rubrica{Hebdomadarius dicit Dominus vobiscum, vel, absente sacerdote vel diacono, sic concluditur:}

\vspace{2mm}

\ifx\dominusnosbenedicat\undefined
\antiphona{C}{temporalia/dominusnosbenedicat.gtex}
\else
\dominusnosbenedicat
\fi

\rubrica{Postea cantatur a cantore:}

\vspace{2mm}

\cuminitiali{VII}{temporalia/benedicamus-tempore-paschali.gtex}

\vspace{1mm}

\vfill
\pagebreak

\end{document}

