\newcommand{\titulus}{\dies{Dominica ultima per annum.}
\nomenFesti{Domini Nostri Iesu Christi Universorum Regis.}}
\newcommand{\oratio}{\pars{Oratio.}

\noindent Omnípotens sempitérne Deus, qui in dilécto Fílio tuo, universórum rege, ómnia instauráre voluísti, concéde propítius, ut tota creatúra, a servitúte liberáta, tuæ maiestáti desérviat ac te sine fine colláudet.

\pars{Pro pace in universo mundo.} \scriptura{Sir. 50, 25; 2 Esdr. 4, 20; \textbf{H416}}

\vspace{-4mm}

\antiphona{II D}{temporalia/ant-dapacemdomine.gtex}

\vfill

\noindent Deus, a quo sancta desidéria, recta consília et iusta sunt ópera: da servis tuis illam, quam mundus dare non potest, pacem; ut et corda nostra mandátis tuis dédita, et hóstium subláta formídine, témpora sint tua protectióne tranquílla.

\noindent Per Dóminum nostrum Iesum Christum, Fílium tuum, qui tecum vivit et regnat in unitáte Spíritus Sancti, Deus, per ómnia sǽcula sæculórum.

\noindent \Rbardot{} Amen.}
\newcommand{\tedeumsolemnis}{Solemnis}
\newcommand{\invitatorium}{\pars{Invitatorium.}

\vspace{-4mm}

\antiphona{VII}{temporalia/inv-iesumchristum.gtex}}
\newcommand{\hymnusmatutinum}{\pars{Hymnus}

\vspace{-5mm}

\antiphona{D}{temporalia/hym-IesuRex.gtex}}
\newcommand{\nocturnoi}{\pars{Psalmus 1.} \scriptura{Ps. 2, 6}

\vspace{-4mm}

\antiphona{III g}{temporalia/ant-egoautemconstitutussum.gtex}

\vspace{-2mm}

\scriptura{Ps. 2}

\vspace{-2mm}

\initiumpsalmi{temporalia/ps2-initium-iii-g-auto.gtex}

\input{temporalia/ps2-iii-g.tex} \Abardot{}

\vfill
\pagebreak

\pars{Psalmus 2.} \scriptura{Ps. 8, 6; Heb. 2, 7}

\vspace{-4mm}

\antiphona{VIII G\textsuperscript{2}}{temporalia/ant-gloriaethonore.gtex}

\vspace{-1mm}

\scriptura{Ps. 8}

\vspace{-2mm}

\initiumpsalmi{temporalia/ps8-initium-viii-G2-auto.gtex}

\input{temporalia/ps8-viii-G2.tex} \Abardot{}

\vfill
\pagebreak

\pars{Psalmus 3.} \scriptura{Ps. 23, 7.9; \textbf{H44}}

\vspace{-4mm}

\antiphona{V a}{temporalia/ant-elevamini.gtex}

%\vspace{-2mm}

\scriptura{Ps. 23}

\initiumpsalmi{temporalia/ps23-initium-v-a-auto.gtex}

\input{temporalia/ps23-v-a.tex} \Abardot{}

\vfill
\pagebreak
}
\newcommand{\nocturnoii}{\vspace{-4mm}

\pars{Psalmus 4.} \scriptura{Ps. 46, 7-8; \textbf{H72}}

\vspace{-4mm}

\antiphona{I a}{temporalia/ant-psallitedeo.gtex}

%\vspace{-2mm}

\scriptura{Ps. 46}

%\vspace{-2mm}

\initiumpsalmi{temporalia/ps46-initium-i-a.gtex}

\input{temporalia/ps46-i-a.tex} \Abardot{}

\vfill
\pagebreak

\pars{Psalmus 5.} \scriptura{Ps. 137, 4; Ier. 27, 7}

\vspace{-4mm}

\antiphona{VIII G\textsuperscript{5}}{temporalia/ant-adorabunteum.gtex}

%\vspace{-2mm}

\scriptura{Ps. 71, 1-11}

\initiumpsalmi{temporalia/ps71i-initium-viii-G6-auto.gtex}

\input{temporalia/ps71i-viii-G6.tex} \Abardot{}

\vfill
\pagebreak

\pars{Psalmus 6.} \scriptura{Ps. 71, 17}

\vspace{-4mm}

\antiphona{II* a}{temporalia/ant-benediceturinipso.gtex}

%\vspace{-5mm}

\scriptura{Ps. 71, 12-19}

%\vspace{-2mm}

\initiumpsalmi{temporalia/ps71ii-initium-ii_-a-auto.gtex}

\input{temporalia/ps71ii-ii_-a.tex} \Abardot{}

\vfill
\pagebreak}
\newcommand{\nocturnoiii}{\pars{Cantica.} \scriptura{1 Par. 29, 11}

\vspace{-4mm}

\antiphona{VIII G\textsuperscript{2}}{temporalia/ant-tuaestpotentia.gtex}

%\vspace{-2mm}

\scriptura{Canticum David, 1 Par. 29, 10-13}

%\vspace{-2mm}

\initiumpsalmi{temporalia/david-initium-viii-g2-auto.gtex}

\input{temporalia/david-viii-g2.tex}

\rubrica{Hic non dicitur antiphona.}

\vfill
\pagebreak

\scriptura{Canticum Isaiæ Prophetæ, Is. 12, 1-7}

\initiumpsalmi{temporalia/isaiae-initium-viii-g2-auto.gtex}

\input{temporalia/isaiae-viii-g2.tex}

\vfill
\pagebreak

\scriptura{Canticum Isaiaæ, Is. 61, 10-11; 62, 1-7}

\vspace{-2mm}

\initiumpsalmi{temporalia/isaiae4-initium-viii-g2-auto.gtex}

\input{temporalia/isaiae4-viii-g2.tex}

\vfill

\vspace{-2mm}

\antiphona{}{temporalia/ant-tuaestpotentia.gtex}

\vfill
\pagebreak}
\newcommand{\matversusi}{\pars{Versus.} \scriptura{Mt. 28, 17}

\sineinitiali{temporalia/versus-dataest-communis.gtex}}
\newcommand{\matversusii}{\pars{Versus.} \scriptura{1 Paral. 16, 28}

\sineinitiali{temporalia/versus-afferte.gtex}}
\newcommand{\lectioi}{\pars{Lectio I.} \scriptura{Ap. 1, 4-6}

\noindent De libro Apocalýpsis beáti Ioánnis Apóstoli.

\noindent Grátia vobis et pax ab eo qui est et qui erat et qui ventúrus est,

\noindent et a septem spirítibus qui in conspéctu throni eius sunt,

\noindent et a Iesu Christo, qui est testis fidélis, primogénitus mortuórum et princeps regum terræ,

\noindent qui diléxit nos et lavit nos a peccátis nostris in sánguine suo,

\noindent et fecit nos regnum et sacerdótes Deo et Patri suo:

\noindent ipsi glória et impérium in sǽcula sæculórum. Amen.}
\newcommand{\responsoriumi}{\pars{Responsorium 1.} \scriptura{\Rbardot{} Esth. 13, 9 \Vbardot{} ibid. 13, 17; \textbf{H411}}

\vspace{-5mm}

\responsorium{II}{temporalia/resp-dominerexomnipotens-CROCHU.gtex}{}}
\newcommand{\lectioii}{\pars{Lectio II.} \scriptura{Ap. 1, 10.12-18}

\noindent Fui in spíritu in domínica die, et audívi post me vocem magnam tamquam tubæ.

\noindent Et convérsus sum ut vidérem vocem quæ loquebátur mecum;

\noindent et convérsus vidi septem candelábra áurea, et in médio septem candelabrórum aureórum símilem fílio hóminis vestítum podére et præcínctum ad mamíllas zona áurea.

\noindent Caput autem eius et capílli erant cándidi tamquam lana alba et tamquam nix, et óculi eius tamquam flamma ignis,

\noindent et pedes eius símiles aurichálco, sicut in camíno ardénti, et vox illíus tamquam vox aquárum multárum.

\noindent Et habébat in déxtera sua stellas septem, et de ore eius gládius utráque parte acútus exíbat, et fácies eius sicut sol lucet in virtúte sua.

\noindent Et cum vidíssem eum, cécidi ad pedes eius tamquam mórtuus;

\noindent et pósuit déxteram suam super me, dicens: ,,Noli timére; ego sum primus et novíssimus, et vivus et fui mórtuus, et ecce sum vivens in sǽcula sæculórum; et hábeo claves mortis et inférni.``}
\newcommand{\responsoriumii}{\pars{Responsorium 2.} \scriptura{\Rbardot{} Is. 6, 1 \Vbardot{} Is. 6, 2; \textbf{H416}}

\vspace{-5mm}

\responsorium{I}{temporalia/resp-vididominumsedentem-CROCHU-sinedox.gtex}{}}
\newcommand{\lectioiii}{\pars{Lectio III.} \scriptura{Ap. 2, 26.28; 3, 5.12.20-21}

\noindent ,,Qui vícerit et custodíerit usque in finem ópera mea, dabo illi potestátem super gentes, sicut et ego accépi a Patre meo; et dabo illi stellam matutínam.

\noindent Et non delébo nomen eius de libro vitæ, et confitébor nomen eius coram Patre meo et coram ángelis eius.

\noindent Fáciam illum colúmnam in templo Dei mei, et foras non egrediétur ámplius;

\noindent et scribam super eum nomen Dei mei et nomen civitátis Dei mei, novæ Ierúsalem, quæ descéndit de cælo a Deo meo, et nomen meum novum.

\noindent Ecce sto ad óstium et pulso; si quis audíerit vocem meam et aperúerit mihi iánuam, intrábo ad illum et cenábo cum illo, et ipse mecum.

\noindent Qui vícerit, dabo ei sedére mecum in throno meo, sicut et ego vici et sedi cum Patre meo in throno eius.``}
\newcommand{\responsoriumiii}{\pars{Responsorium 3.} \scriptura{\Rbardot{} Idt. 9, 17 \Vbardot{} ibid. 9, 16; \textbf{H410}}

\vspace{-5mm}

\responsorium{II}{temporalia/resp-dominatordomine-CROCHU-cumdox.gtex}{}}
\newcommand{\lectioiv}{\pars{Lectio IV.} \scriptura{Cap. 25: PG 11, 495-499}

\noindent Ex Libéllo Orígenis presbýteri \emph{De oratióne}.

\noindent Si \emph{regnum Dei}, iuxta verbum Dómini et Servatóris nostri, \emph{cum observatióne non venit, neque dicent: Ecce hic aut ecce illic; sed regnum Dei intra} nos \emph{est, nam prope est verbum valde in ore} nostro et \emph{in corde} nostro: procul dúbio is qui regnum Dei adveníre precátur, de eo quod in se habet regno Dei recte orat, ut oriátur et fructus ferat et perficiátur. Nam in quólibet sanctórum Deus regnat et quílibet sanctus spiritálibus obséquitur légibus Dei, qui in ipso hábitat ut in recte administráta civitáte. Præsens ei Pater adest et conrégnat Patri Christus in illa ánima perfécta iuxta illud: \emph{Ad eum veniémus et mansiónem apud eum faciémus}.}
\newcommand{\responsoriumiv}{\pars{Responsorium 4.} \scriptura{\Rbardot{} 2 Mac. 14, 35-36 \Vbardot{} Ps. 79, 2; \textbf{H415}}

\vspace{-5mm}

\responsorium{II}{temporalia/resp-tudomineuniversorum-CROCHU.gtex}{}}
\newcommand{\lectiov}{\pars{Lectio V.}

\noindent Tunc ergo id quod in nobis est regnum Dei perpétuo procedéntibus nobis ad summum pervéniet, cum illud implétum fúerit quod Apóstolus ait, Christum subiéctis sibi ómnibus inimícis traditúrum \emph{regnum Deo et Patri, ut sit Deus ómnia in ómnibus.} Propter hoc indesinénter orántes ea ánimi affectióne, quæ Verbo divína fiat, dicámus Patri nostro, qui in cælis est: \emph{Sanctificétur nomen tuum, advéniat regnum tuum.}

\noindent Id quoque de regno Dei percipiéndum est: sicut non est \emph{participátio iustítiæ cum iniquitáte,} neque \emph{socíetas lucis ad ténebras}, neque \emph{convéntio Christi ad Bélial}: sic regnum Dei cum regno peccáti stare non posse.}
\newcommand{\responsoriumv}{\pars{Responsorium 5.} \scriptura{\Rbardot{} Dan. 3, 55; Is. 40, 12 \Vbardot{} Ps. 79, 2; Dan. 9, 18; \textbf{H419}}

\vspace{-5mm}

\responsorium{II}{temporalia/resp-quicaelorumcontinesthronos-CROCHU-sinedox.gtex}{}}
\newcommand{\lectiovi}{\pars{Lectio VI.}

\noindent Ergo si Deum in nobis regnáre vólumus, nullo modo \emph{regnet peccátum in nostro mortáli córpore,} sed mortificémus \emph{membra} nostra, \emph{quæ sunt super terram} et fructificémus Spíritu; ut in nobis, quasi in spiritáli paradíso, Deus obámbulet regnétque solus in nobis cum Christo suo, qui sédeat in nobis a dextris virtútis illíus spiritális, quam optámus accípere: sedeátque donec inimíci eius omnes, qui in nobis sunt, fiant \emph{scabéllum pedum} eius et evacuétur in nobis omnis principátus et potéstas et virtus. Possunt enim hæc in unoquóque nostrum fíeri et \emph{novíssima inimíca} déstrui \emph{mors}; ut et in nobis Christus dicat: \emph{Ubi est, mors, stímulus tuus? Ubi est, inférne, victória tua?} Iam nunc ígitur \emph{corruptíbile nostrum} induátur sanctitátem et \emph{incorruptiónem}; \emph{et mortále,} evacuáta morte, patérnam induátur \emph{immortalitátem,} ut in nobis, regnánte Deo, in regeneratiónis iam resurrectionísque bonis versémur.}
\newcommand{\responsoriumvi}{\pars{Responsorium 6.} \scriptura{\Rbardot{} 1 Paral. 29, 11 \Vbardot{} 2 Mac. 1, 24; \textbf{H424}}

\vspace{-5mm}

\responsorium{II}{temporalia/resp-tuaestpotentia-CROCHU-cumdox.gtex}{}}
\newcommand{\evangelium}{
\pars{Versus.} \scriptura{Ps. 71, 11}

\sineinitiali{temporalia/versus-adorabunt.gtex}

\vspace{5mm}

\sineinitiali{temporalia/oratiodominica-mat.gtex}

\vspace{5mm}

\pars{Absolutio.}

\cuminitiali{}{temporalia/absolutio-avinculis.gtex}

\vfill
\pagebreak

\cuminitiali{}{temporalia/benedictio-solemn-evangelica.gtex}

\vspace{7mm}

\pars{Evangelium} \scriptura{Lc. 23, 35-43}

\noindent Léctio sancti Evangélii secúndum Lucam.

\noindent In illo témpore:

\noindent Deridébant Iesum príncipes dicéntes: «Alios salvos fecit; se salvum fáciat, si hic est Christus Dei eléctus!».

\noindent Illudébant autem ei et mílites accedéntes, acétum offeréntes illi et dicéntes: «Si tu es rex Iudæórum, salvum te fac!». Erat autem et superscríptio super illum: «Hic est rex Iudæórum».

\noindent Unus autem de his, qui pendébant, latrónibus blasphemábat eum dicens: «Nonne tu es Christus? Salvum fac temetípsum et nos!».

\noindent Respóndens autem alter increpábat illum dicens: «Neque tu times Deum, quod in eádem damnatióne es? Et nos quidem iuste, nam digna factis recípimus! Hic vero nihil mali gessit». Et dicébat: «Iesu, meménto mei, cum véneris in regnum tuum».

\noindent Et dixit illi: «Amen dico tibi: Hódie mecum eris in paradíso».

\scriptura{Hom. 1 de cruce, 3 : PG 49,403-404}

\noindent Ex Homilíis sancti Ioánnis Chrysóstomi epíscopi.

\noindent \emph{Meménto mei,} Dómine, \emph{in regno tuo.} Non prius ausus est latro dícere: \emph{Meménto mei in regno tuo,} quam per confessiónem peccatórum sárcinam deposuísset.

\noindent Viden' quanta res sit conféssio? Conféssus est, et paradísum apéruit. Conféssus est, et tantam accépit fidúciam, ut a latrocínio regnum péteret.

\noindent Viden' quantórum nobis bonórum crux causa fúerit? Regnum petis? Quid vides huiúsmodi? Clavi et crux in conspéctu sunt: «Verum ea ipsa crux, inquit, est sýmbolum regni. Ideo ipsum regem voco, quia vídeo crucifíxum: regis enim est pro súbditis mori. Hic ipse dixit: \emph{Bonus pastor ánimam suam ponit pro óvibus;} ergo et bonus rex ánimam suam ponit pro súbditis. Quóniam ígitur ánimam suam pósuit, ídeo regem illum voco: \emph{Meménto mei,} Dómine, \emph{in regno tuo.}»

\noindent Viden' quómodo crux sit regni sýmbolum? Visne et aliúnde illud díscere? Non relíquit eum in terra, sed attráxit eum et in cælum dedúxit. Unde hoc palam est? Quia ventúrus est cum illo in secúndo et glorióso eius advéntu, ut discas crucem esse rem honorábilem; quare et glóriam illam vocávit.

\noindent At videámus quómodo cum cruce véniat; necésse quippe est demonstratiónem in médium redúcere. \emph{Si díxerint,} inquit, \emph{ecce in penetrálibus est Christus, ecce in desérto est; ne abeátis;} de secúndo et glorióso suo advéntu sic loquens propter falsos christos, propter falsos prophétas, propter Antichrístum, ne quis sedúctus in illum incíderet.

\noindent \emph{Sicut fulgur exit ab oriénte et appáret usque ad occidéntem, sic erit et advéntus Fílii hóminis.} Conféstim ómnibus apparébit, neminíque interrogátu opus erit num hic, num illic sit Christus. Quemádmodum enim apparénte fúlgure, non opus est perquírere an apparúerit; sic et in advéntu Christi non opus erit quǽrere an Christus advénerit. At quod quǽritur est, an cum cruce ventúrus sit; non enim promíssi oblíti sumus. Audi ígitur ea quæ sequúntur. \emph{Tunc,} inquit; tunc, quandónam? \emph{Cum vénerit Fílius hóminis, sol obscurábitur et luna non dabit lucem suam. Tunc stellæ cadent, tunc signum Fílii hóminis apparébit in cælo.} Viden' quanta sit virtus signi crucis?

\vfill
\pagebreak

\pars{Responsorium 7.} \scriptura{\Rbardot{} Ier. 31, 11.12 \Vbardot{} Ps. 4, 8; \textbf{H418}}

\vspace{-5mm}

\responsorium{III}{temporalia/resp-redemitdominus-CROCHU-cumdox.gtex}{}

\vfill
\pagebreak
}
\newcommand{\deusinadiutorium}{\grechangedim{interwordspacetext}{0.18 cm plus 0.15 cm minus 0.05 cm}{scalable}%
\cuminitiali{}{temporalia/deusinadiutorium-alter.gtex}}
\newcommand{\hymnuslaudes}{\pars{Hymnus.} \scriptura{Victorius Genovesi (\olddag{} 1967)}

\cuminitiali{IV}{temporalia/hym-AEternaImago.gtex}}
\newcommand{\laudes}{\pars{Psalmus 1.} \scriptura{Zach. 6, 12.13; 9, 10}

\vspace{-4mm}

\antiphona{VII c\textsuperscript{2}}{temporalia/ant-ecceviroriens.gtex}

%\vspace{-2mm}

\scriptura{Psalmus 62.}

%\vspace{-1mm}

\initiumpsalmi{temporalia/ps62-initium-vii-c2-auto.gtex}

%\vspace{-1.5mm}

\input{temporalia/ps62-vii-c2.tex} \Abardot{}

\vfill
\pagebreak

\pars{Psalmus 2.} \scriptura{Mich. 5, 4-5}

\vspace{-4mm}

\antiphona{VII a}{temporalia/ant-magnificabitur.gtex}

\vspace{-2mm}

\scriptura{Canticum trium puerorum, Dan. 3, 57-88 et 56}

\vspace{-2mm}

\initiumpsalmi{temporalia/dan3-initium-vii-a-auto.gtex}

\input{temporalia/dan3-vii-a-sinedox.tex}

\rubrica{Hic non dicitur Gloria Patri, neque Amen.}

\vfill

\antiphona{}{temporalia/ant-magnificabitur.gtex}

\vfill
\pagebreak

\pars{Psalmus 3.} \scriptura{Dan. 7, 14}

\antiphona{I f}{temporalia/ant-deditei.gtex}

%\vspace{-2mm}

\scriptura{Psalmus 149}

%\vspace{-2mm}

\initiumpsalmi{temporalia/ps149-initium-i-f-auto.gtex}

\input{temporalia/ps149-i-f.tex} \Abardot{}

\vfill
\pagebreak}
\newcommand{\lectiobrevis}{\pars{Lectio Brevis.} \scriptura{Eph. 4, 15-16}

\noindent Veritátem autem faciéntes in caritáte, crescámus in illo per ómnia, qui est caput Christus: ex quo totum corpus compáctum et connéxum per omnem iunctúram subministratiónis, secúndum operatiónem in mensúram uniuscuiúsque membri, augméntum córporis facit in ædificatiónem sui in caritáte.}
\newcommand{\responsoriumbreve}{\pars{Responsorium breve.} \scriptura{Ps. 144, 10.11}

\antiphona{VI}{temporalia/resp-sanctitui.gtex}}
\newcommand{\benedictus}{\pars{Canticum Zachariæ.} \scriptura{Ap. 1, 6.5}

\vspace{-4mm}

\antiphona{V a}{temporalia/ant-fecitnos.gtex}

\vspace{-1mm}

\scriptura{Lc. 1, 68-79}

\vspace{-3mm}

\cantusSineNeumas
\initiumpsalmi{temporalia/benedictus-initium-vsoll-a2-auto.gtex}

%\vspace{-1.5mm}

\input{temporalia/benedictus-vsoll-a2.tex} \Abardot{}}
\newcommand{\preces}{\noindent Fratres, Christum regem,~\gredagger{} qui est ante ómnia et in quo ómnia constant,~\grestar{} orémus clamántes:

\Rbardot{} Advéniat regnum tuum, Dómine.

\noindent Christe salvátor, tu qui es Dóminus Deus noster,~\gredagger{} rex et pastor noster,~\grestar{} duc pópulum tuum in páscua vitæ.

\Rbardot{} Advéniat regnum tuum, Dómine.

\noindent Pastor bone, qui ánimam tuam posuísti pro óvibus tuis,~\grestar{} rege nos, et nihil nobis déerit.

\Rbardot{} Advéniat regnum tuum, Dómine.

\noindent Redémptor noster, qui constitútus es rex super omnem terram,~\grestar{} fac, ut ómnia in te instauréntur.

\Rbardot{} Advéniat regnum tuum, Dómine.

\noindent Rex universórum, qui venísti in mundum, ut testimónium perhíbeas veritáti,~\grestar{} agnóscant omnes primátum tuum in ómnibus.

\Rbardot{} Advéniat regnum tuum, Dómine.

\noindent Exémplar et magíster noster, qui in regnum tuum nos transtulísti,~\grestar{} tríbue, ut sanctos, immaculátos et irreprehensíbiles coram te nos hódie exhibeámus.

\Rbardot{} Advéniat regnum tuum, Dómine.}
\newcommand{\benedicamuslaudes}{\cuminitiali{II}{temporalia/benedicamus-solemnism-laud.gtex}}
\newcommand{\hebdomada}{infra Hebdom. XXIV per Annum.}
\newcommand{\hiemalis}{Hiemalis}
\newcommand{\matub}{Matutinum Hebdomadae B}
\newcommand{\matubd}{Matutinum Hebdomadae B vel D}
\newcommand{\laudb}{Laudes Hebdomadae B}
\newcommand{\laudbd}{Laudes Hebdomadae B vel D}

% LuaLaTeX

\documentclass[a4paper, twoside, 12pt]{article}
\usepackage[latin]{babel}
%\usepackage[landscape, left=3cm, right=1.5cm, top=2cm, bottom=1cm]{geometry} % okraje stranky
%\usepackage[landscape, a4paper, mag=1166, truedimen, left=2cm, right=1.5cm, top=1.6cm, bottom=0.95cm]{geometry} % okraje stranky
\usepackage[landscape, a4paper, mag=1400, truedimen, left=0.5cm, right=0.5cm, top=0.5cm, bottom=0.5cm]{geometry} % okraje stranky

\usepackage{fontspec}
\setmainfont[FeatureFile={junicode.fea}, Ligatures={Common, TeX}, RawFeature=+fixi]{Junicode}
%\setmainfont{Junicode}

% shortcut for Junicode without ligatures (for the Czech texts)
\newfontfamily\nlfont[FeatureFile={junicode.fea}, Ligatures={Common, TeX}, RawFeature=+fixi]{Junicode}

\usepackage{multicol}
\usepackage{color}
\usepackage{lettrine}
\usepackage{fancyhdr}

% usual packages loading:
\usepackage{luatextra}
\usepackage{graphicx} % support the \includegraphics command and options
\usepackage{gregoriotex} % for gregorio score inclusion
\usepackage{gregoriosyms}
\usepackage{wrapfig} % figures wrapped by the text
\usepackage{parcolumns}
\usepackage[contents={},opacity=1,scale=1,color=black]{background}
\usepackage{tikzpagenodes}
\usepackage{calc}
\usepackage{longtable}
\usetikzlibrary{calc}

\setlength{\headheight}{14.5pt}

\input{conventuscommune.tex} % Often used macros
%%%% Preklady jednotlivych zpevu (nektere se opakuji, a je dobre mit je
% vsechny na jedne hromade)

% HOURS ---

\newcommand{\trAntI}{\translatioCantus{Muž boží měl kožený toulec, pečlivě
zavázaný, jenž mu visel na šíji a~často se ho dotýkal.}}

\newcommand{\trAntII}{\translatioCantus{Klíč od~něho tak dobře střežil, že
dokud žil v~těle, nikdo z~jeho žáků nezvěděl, co je uvnitř.}}

\newcommand{\trAntIII}{\translatioCantus{Ale když se odebral z~tohoto
života, schránku otevřeli a~objevili v~ní žíněné roucho a~měděný řetěz
potřísněný krví.}}

\newcommand{\trAntIV}{\translatioCantus{A když prohlédli mistrovo tělo,
nalezli jeho tělo na čtyřech místech hluboce zbrázděno ranami od řetězu.}}

\newcommand{\trAntV}{\translatioCantus{Krev vytékající z~těch ran, místy
prostoupila i~žíněným rouchem.}}

\newcommand{\trCapituli}{\translatioCantus{
Miláčkovi Boha a~lidí,
Mojžíšovi požehnané paměti,~\gredagger{}
dopřál slávu rovnou slávě svatých~\grestar{}
učinil ho mocným na postrach nepřátelům
a~jeho slovy zastavil divy.}}

\newcommand{\trLectioBrevis}{\translatioCantus{
Pamatujte na své představené,
kteří vám hlásali Boží slovo.
Uvažte, jak oni skončili život, a~napodobujte jejich víru.
Ježíš Kristus je stejný včera i~dnes i~navěky.
Nenechte se svést věelijakými cizími naukami.}}

\newcommand{\trRespLaud}{\translatioCantus{Spravedlivého vodil Hospodin~\grestar{}
po přímých stezkách. \Vbardot{} A~ukázal mu Boží království.}}

\newcommand{\trRespLaudB}{\translatioCantus{Na tvých hradbách, Jeruzaléme,
ustanovil jsem strážné;~\grestar{}
budou bdít nad mým lidem. \Vbardot{} Ani ve dne, ani v~noci nesmějí nikdy
mlčet.}}

\newcommand{\trVersus}{\translatioCantus{\Vbardot{} Ústa spravedlivého šeptají moudrost, aleluja.
\Rbardot{} A~jeho jazyk ohlašuje právo, aleluja.}}

\newcommand{\trAntBenedictus}{\translatioCantus{Když na bujné oře vložili
nosítka a~sňali jim uzdu, vydali se přímo k~cele božího muže.}}

\newcommand{\trPreces}{\translatioCantus{
\noindent S vděčností chvalme Krista, dobrého Pastýře, \gredagger{} který dal život za své ovce, \grestar{} a~pokorně ho prosme: \Rbardot{} Pane, buď pastýřem svého lidu.

\noindent Kriste, ty dáváš církvi pastýře, a~jejich službou se ujímáš svého lidu, \grestar{} dej, ať v~lásce těch, kteří nás vedou, poznáváme, jak nás miluješ. \Rbardot{} Pane, buď pastýřem svého lidu.

\noindent Ty stále konáš skrze své zástupce službu pastýře a~učitele, \grestar{} nepřestávej nás nikdy vést prostřednictvím svých služebníků. \Rbardot{} Pane, buď pastýřem svého lidu.

\noindent Ty prokazuješ svému lidu skrze jeho pastýře službu lékaře duše i~těla, \grestar{} ochraňuj náš život a~veď nás ke svatosti. \Rbardot{} Pane, buď pastýřem svého lidu.

\noindent Ty posíláš své svaté, aby slovem i~příkladem vedli tvůj lid k~tobě, \grestar{} na jejich přímluvu nás posiluj, abychom vytrvali na cestě, která vede k~věčnému životu. \Rbardot{} Pane, buď pastýřem svého lidu.}}

\newcommand{\trOrationis}{\translatioCantus{Bože, jenž nám dopřáváš radovat
se z~výroční slavnosti svatého tvého vyznavače Havla, uděl dobrotivě,
abychom když slavíme jeho narození, též se řídili podobou jeho skutků.
Skrze…}}
 % Czech translations of the proper texts

\newcommand{\annusEditionis}{2020}

%%%% Vicekrat opakovane kousky

\newcommand{\anteOrationem}{
  \rubrica{Ante Orationem, cantatur a Superiore:}

  \pars{Supplicatio Litaniæ.}

  \cuminitiali{}{temporalia/supplicatiolitaniae.gtex}

  \pars{Oratio Dominica.}

  \cuminitiali{}{temporalia/oratiodominica.gtex}

  \rubrica{Deinde dicitur ab Hebdomadario:}

  \cuminitiali{}{temporalia/dominusvobiscum-solemnis.gtex}

  \rubrica{In choro monialium loco Dominus vobiscum dicitur:}

  \sineinitiali{temporalia/domineexaudi.gtex}
}

\setlength{\columnsep}{30pt} % prostor mezi sloupci

%%%%%%%%%%%%%%%%%%%%%%%%%%%%%%%%%%%%%%%%%%%%%%%%%%%%%%%%%%%%%%%%%%%%%%%%%%%%%%%%%%%%%%%%%%%%%%%%%%%%%%%%%%%%%
\begin{document}

% Here we set the space around the initial.
% Please report to http://home.gna.org/gregorio/gregoriotex/details for more details and options
\grechangedim{afterinitialshift}{2.2mm}{scalable}
\grechangedim{beforeinitialshift}{2.2mm}{scalable}
\grechangedim{interwordspacetext}{0.22 cm plus 0.15 cm minus 0.05 cm}{scalable}%
\grechangedim{annotationraise}{-0.2cm}{scalable}

% Here we set the initial font. Change 38 if you want a bigger initial.
% Emit the initials in red.
\grechangestyle{initial}{\color{red}\fontsize{38}{38}\selectfont}

\pagestyle{empty}

%%%% Titulni stranka
\begin{titulusOfficii}
\titulus{}
\end{titulusOfficii}

% graphic
%\vspace{1.5cm}
%\begin{center}
%\includegraphics[width=8cm]{emmaus.jpg}
%\end{center}

\vfill

\begin{center}
%Ad usum et secundum consuetudines chori \guillemotright{}Conventus Choralis\guillemotleft.

%Editio Sancti Wolfgangi \annusEditionis
\end{center}

\pagebreak

\renewcommand{\headrulewidth}{0pt} % no horiz. rule at the header
\fancyhf{}
\pagestyle{fancy}

\pars{Oratio ante divinum Officium.}

\lettrine{{\color{red}A}}{peri,} Dómine, os meum ad benedicéndum nomen sanctum tuum:
munda quoque cor meum ab ómnibus vanis, pervérsis, et aliénis
cogitatiónibus:
intelléctum illúmina, afféctum inflámma,
ut digne, atténte ac devóte hoc Offícium recitáre váleam,
et exaudíri mérear ante conspéctum Divínæ Maiestátis tuæ.
Per Christum, Dóminum nostrum.
\Rbardot{} Amen.

Dómine, in unióne illíus divínæ intentiónis,
qua ipse in terris laudes Deo persolvísti,
has tibi Horas \rubricatum{(vel \textnormal{hanc tibi Horam})} persólvo.

%\trOratioAnteOfficium

\vfill

\pars{Oratio post divinum Officium.}

\rubrica{
  Orationem sequentem devote post Officium recitantibus
  Leo Papa X. defectus, et culpas in eo persolvendo ex humana
  fragilitate contractas, indulsit, et dicitur flexis genibus.
}

\lettrine{{\color{red}S}}{acrosánctæ} et indivíduæ Trinitáti,
crucifíxi Dómini nostri Iesu Christi humanitáti,
beatíssimæ et gloriosíssimæ sempérque Vírginis Maríæ
fecúndæ integritáti, 
et ómnium Sanctórum universitáti
sit sempitérna laus, honor, virtus et glória
ab omni creatúra,
nobísque remíssio ómnium peccatórum,
per infiníta sǽcula sæculórum.
\Rbardot{} Amen.

\noindent \Vbardot{} Beáta víscera Maríæ Virginis, quæ portavérunt
ætérni Patris Fílium.\\
\Rbardot{} Et beáta úbera, quæ lactavérunt Christum Dominum.

\rubrica{Et dicitur secreto \textnormal{Pater noster.} et \textnormal{Ave María.}}

%\trOratioPostOfficium

\vfill

\hora{Ad I. Vesperas.} %%%%%%%%%%%%%%%%%%%%%%%%%%%%%%%%%%%%%%%%%%%%%%%%%%%%%
%\sideThumbs{I. Vesperæ}

\cantusSineNeumas

\vspace{0.5cm}
\grechangedim{interwordspacetext}{0.18 cm plus 0.15 cm minus 0.05 cm}{scalable}%
\cuminitiali{}{temporalia/deusinadiutorium-solemnis.gtex}
\grechangedim{interwordspacetext}{0.22 cm plus 0.15 cm minus 0.05 cm}{scalable}%

\vfill
\pagebreak

\pars{Psalmus 1.} \scriptura{Ps. 144, 13; \textbf{H100}}

\vspace{-4mm}

\antiphona{VII c\textsuperscript{2}}{temporalia/ant-regnumtuum.gtex}

\scriptura{Psalmus 144, 10-21.}

\initiumpsalmi{temporalia/ps144ii-initium-vii-c2-auto.gtex}

%\psalmusEtTranslatioT{temporalia/ps144ii-VII-comb.tex}{10cm}
\input{temporalia/ps144ii-VII.tex} \Abardot{}

\vspace{-1cm}

\vfill
\pagebreak

\pars{Psalmus 2.} \scriptura{Ps. 145, 2; \textbf{H100}}

\vspace{-4mm}

\antiphona{IV E}{temporalia/ant-laudabodeum.gtex}

\scriptura{Psalmus 145.}

\initiumpsalmi{temporalia/ps145-initium-iv-E-auto.gtex}

%\psalmusEtTranslatioT{temporalia/ps145-VII-comb.tex}{10cm}
\input{temporalia/ps145-VII.tex} \Abardot{}

\vfill
\pagebreak

\pars{Psalmus 3.} \scriptura{Ps. 146, 1; \textbf{H101}}

\vspace{-4mm}

\antiphona{VIII a}{temporalia/ant-deonostro.gtex}

\scriptura{Psalmus 146.}

\initiumpsalmi{temporalia/ps146-initium-viii-A-auto.gtex}

%\psalmusEtTranslatioT{temporalia/ps146-VII-comb.tex}{10cm}
\input{temporalia/ps146-VII.tex} \Abardot{}

\vfill
\pagebreak

\pars{Psalmus 4.} \scriptura{Ps. 147, 1}

\vspace{-4mm}

\antiphona{E}{temporalia/ant-laudajerusalem.gtex}

\scriptura{Psalmus 147.}

\initiumpsalmi{temporalia/ps147-initium-e-auto.gtex}

%\psalmusEtTranslatioT{temporalia/ps147-VII-comb.tex}{10cm}
\input{temporalia/ps147-VII.tex} \Abardot{}

\vfill
\pagebreak

\pars{Capitulum.} \scriptura{Rom. 11, 33}

\grechangedim{interwordspacetext}{0.12 cm plus 0.15 cm minus 0.05 cm}{scalable}%
\cuminitiali{}{temporalia/capitulum-OAltitudo.gtex}
\grechangedim{interwordspacetext}{0.22 cm plus 0.15 cm minus 0.05 cm}{scalable}

% preklad Jeruz. bible
%\trCapituliI

\vfill

\pars{Responsorium breve.} \scriptura{Ps. 146, 5}

\cuminitiali{VI}{temporalia/resp-magnusdominusnoster.gtex}

%\trResp

\vfill
\pagebreak

\pars{Hymnus} \scriptura{Ambrosius (\olddag{} 397)}

\cuminitiali{I}{temporalia/hym-OLuxBeata-aestivalis.gtex}
\vspace{-3mm}
%\input{hym-OLuxBeata-bohtext.tex}

\vfill
%\pagebreak

\pars{Versus.}

% Versus. %%%
\sineinitiali{temporalia/versus-vespertina.gtex}

%\noindent \trVersus

\vfill
\pagebreak

\magnificati

\vfill
\pagebreak

%\sideThumbs{{\scriptsize{}Fine horarum}}

\anteOrationem

\pagebreak

% Oratio. %%%
\oratioLaudes

\vspace{-1mm}
%\trOrationisI

\vfill

\rubrica{Hebdomadarius dicit iterum Dominus vobiscum, vel cantor dicit:}

\vspace{2mm}

\sineinitiali{temporalia/domineexaudi.gtex}

\rubrica{Postea cantatur a cantore:}

\vspace{2mm}

\cuminitiali{I}{temporalia/benedicamus-dominica-perannum.gtex}

\vspace{1mm}

\vfill
\pagebreak

\hora{Ad Matutinum.} %%%%%%%%%%%%%%%%%%%%%%%%%%%%%%%%%%%%%%%%%%%%%%%%%%%%%
%\sideThumbs{Matutinum}

\vspace{2mm}

\cuminitiali{}{temporalia/dominelabiamea.gtex}

\vspace{2mm}

\pars{Invitatorium.} \scriptura{Ps. 94, 1; Psalmus 94}

\vspace{-6mm}

\antiphona{E}{temporalia/inv-veniteexsultemus.gtex}

\vfill
\pagebreak

\pars{Hymnus.} \scriptura{Adamus Sancti Victoris (\olddag 1146)}

\vspace{-5mm}

\antiphona{VII}{temporalia/hym-SalveDies.gtex}

\scriptura{Non dicitur \textnormal{Amen} in fine.}
%{
%\vspace{-5mm}
%\setlength{\columnsep}{0pt} % prostor mezi sloupci
%\input{hym-SalveDies-bohtext.tex}
%\setlength{\columnsep}{30pt} % prostor mezi sloupci
%}

\vfill
\pagebreak

\subhora{In I. Nocturno}

\pars{Psalmus 1.} \scriptura{Ps. 1, 1}

\vspace{-4mm}

\antiphona{VIII G}{temporalia/ant-beatusvir.gtex}

%\vspace{-5mm}

\scriptura{Ps. 1}

%\vspace{-2mm}

\initiumpsalmi{temporalia/ps1-initium-viii-G-auto.gtex}

%\psalmusEtTranslatioT{temporalia/ps1-I-comb.tex}{10cm}
\input{temporalia/ps1-I.tex} \Abardot{}

\vfill
\pagebreak

\pars{Psalmus 2.} \scriptura{Ps. 2, 11; \textbf{H93}}

\vspace{-4mm}

\antiphona{VII a}{temporalia/ant-servitedomino.gtex}

\vspace{-3mm}

\scriptura{Ps. 2}

\vspace{-2mm}

\initiumpsalmi{temporalia/ps2-initium-vii-a-auto.gtex}

%\psalmusEtTranslatioT{temporalia/ps2-I-comb.tex}{10cm}
\input{temporalia/ps2-I.tex} \Abardot{}

\vfill
\pagebreak

\pars{Psalmus 3.} \scriptura{Ps. 3, 7}

\vspace{-4mm}

\antiphona{VI F}{temporalia/ant-exsurgedominesalvum.gtex}

%\vspace{-5mm}

\scriptura{Ps. 3}

\initiumpsalmi{temporalia/ps3-initium-vi-F-auto.gtex}

%\psalmusEtTranslatioT{temporalia/ps3-I-comb.tex}{10cm}
\input{temporalia/ps3-I.tex} \Abardot{}

\vfill
\pagebreak

\pars{Versus.} \scriptura{Ps. 118, 55}

% Versus. %%%
\sineinitiali{temporalia/versus-memorfui.gtex}

\vspace{5mm}

\sineinitiali{temporalia/oratiodominica-mat.gtex}

\vspace{5mm}

\pars{Absolutio.}

\cuminitiali{}{temporalia/absolutio-exaudi.gtex}

\vfill
\pagebreak

\cuminitiali{}{temporalia/benedictio-solemn-benedictione.gtex}

\vspace{7mm}

\lectioi

\noindent \Vbardot{} Tu autem, Dómine, miserére nobis.
\noindent \Rbardot{} Deo grátias.

\vfill
\pagebreak

\responsoriumi

\vfill
\pagebreak

\cuminitiali{}{temporalia/benedictio-solemn-unigenitus.gtex}

\vspace{7mm}

\lectioii

\noindent \Vbardot{} Tu autem, Dómine, miserére nobis.
\noindent \Rbardot{} Deo grátias.

\vfill
\pagebreak

\responsoriumii

\vfill
\pagebreak

\cuminitiali{}{temporalia/benedictio-solemn-spiritus.gtex}

\vspace{7mm}

\lectioiii

\noindent \Vbardot{} Tu autem, Dómine, miserére nobis.
\noindent \Rbardot{} Deo grátias.

\vfill
\pagebreak

\responsoriumiii

\vfill
\pagebreak

\subhora{In II. Nocturno}

\pars{Psalmus 4.} \scriptura{Ps. 8, 2}

\vspace{-4mm}

\antiphona{I g}{temporalia/ant-quamadmirabileest.gtex}

%\vspace{-5mm}

\scriptura{Ps. 8}

%A\vspace{-2mm}

\initiumpsalmi{temporalia/ps8-initium-i-g-auto.gtex}

%\psalmusEtTranslatioT{temporalia/ps8-I-comb.tex}{10cm}
\input{temporalia/ps8-I.tex} \Abardot{}

\vfill
\pagebreak

\pars{Psalmus 5.} \scriptura{Ps. 9, 5}

\vspace{-4mm}

\antiphona{VIII G}{temporalia/ant-sedistisuperthronum.gtex}

%\vspace{-5mm}

\scriptura{Ps. 9, 2-11}

\initiumpsalmi{temporalia/ps9ii_xi-initium-viii-G-auto.gtex}

%\psalmusEtTranslatioT{temporalia/ps9ii_xi-I-comb.tex}{10cm}
\input{temporalia/ps9ii_xi-I.tex} \Abardot{}

\vfill
\pagebreak

\pars{Psalmus 6.} \scriptura{Ps. 9, 20}

\vspace{-4mm}

\antiphona{I g\textsuperscript{3}}{temporalia/ant-exsurgedominenon.gtex}

%\vspace{-5mm}

\scriptura{Ps. 9, 12-21}

\initiumpsalmi{temporalia/ps9xii_xxi-initium-i-g3-auto.gtex}

%\psalmusEtTranslatioT{temporalia/ps9xii_xxi-I-comb.tex}{10cm}
\input{temporalia/ps9xii_xxi-I.tex} \Abardot{}

\vfill
\pagebreak

\pars{Versus.} \scriptura{Ps. 118, 62}

% Versus. %%%
\sineinitiali{temporalia/versus-medianocte.gtex}

\vspace{5mm}

\sineinitiali{temporalia/oratiodominica-mat.gtex}

\vspace{5mm}

\pars{Absolutio.}

\cuminitiali{}{temporalia/absolutio-ipsius.gtex}

\vfill
\pagebreak

\cuminitiali{}{temporalia/benedictio-solemn-deus.gtex}

\vspace{7mm}

\lectioiv

\noindent \Vbardot{} Tu autem, Dómine, miserére nobis.
\noindent \Rbardot{} Deo grátias.

\vfill
\pagebreak

\responsoriumiv

\vfill
\pagebreak

\cuminitiali{}{temporalia/benedictio-solemn-christus.gtex}

\vspace{7mm}

\lectiov

\noindent \Vbardot{} Tu autem, Dómine, miserére nobis.
\noindent \Rbardot{} Deo grátias.

\vfill
\pagebreak

\responsoriumv

\vfill
\pagebreak

\cuminitiali{}{temporalia/benedictio-solemn-ignem.gtex}

\vspace{7mm}

\lectiovi

\noindent \Vbardot{} Tu autem, Dómine, miserére nobis.
\noindent \Rbardot{} Deo grátias.

\vfill
\pagebreak

\responsoriumvi

\vfill
\pagebreak

\subhora{In III. Nocturno}

\pars{Psalmus 7.} \scriptura{Ps. 9, 22}

\vspace{-4mm}

\antiphona{II D}{temporalia/ant-utquiddomine.gtex}

\vspace{-4mm}

\scriptura{Ps. 9, 22-32}

%\vspace{-2mm}

\initiumpsalmi{temporalia/ps9xxii_xxxii-initium-ii-D-auto.gtex}

%\psalmusEtTranslatioT{temporalia/ps9xxii_xxxii-I-comb.tex}{10cm}
\input{temporalia/ps9xxii_xxxii-I.tex} \Abardot{}

\vfill
\pagebreak

\pars{Psalmus 8.}\scriptura{Ex. 15, 18}

\vspace{-4mm}

\antiphona{IV* e}{temporalia/ant-inaeternum.gtex}

%\vspace{-4mm}

\scriptura{Ps. 9, 33-39}

\initiumpsalmi{temporalia/ps9xxxiii_xxxix-initium-iv_-e-auto.gtex}

%\psalmusEtTranslatioT{temporalia/ps9xxxiii_xxxix-I-comb.tex}{10cm}
\input{temporalia/ps9xxxiii_xxxix-I.tex} \Abardot{}

\vfill
\pagebreak

\pars{Psalmus 9.} \scriptura{Ps. 10, 8}

\vspace{-4mm}

\antiphona{II* f}{temporalia/ant-justusdominus.gtex}

%\vspace{-4mm}

\scriptura{Ps. 10}

%\initiumpsalmi{temporalia/ps10-initium-iv-c-auto.gtex}
\initiumpsalmi{temporalia/ps10-initium-ii_-f.gtex}

%\psalmusEtTranslatioT{temporalia/ps10-I-comb.tex}{10cm}
\input{temporalia/ps10-I.tex} \Abardot{}

\vfill
\pagebreak

\pars{Versus.} \scriptura{Ps. 118, 148}

% Versus. %%%
\sineinitiali{temporalia/versus-praevenerunt.gtex}

\vspace{5mm}

\sineinitiali{temporalia/oratiodominica-mat.gtex}

\vspace{5mm}

\pars{Absolutio.}

\cuminitiali{}{temporalia/absolutio-avinculis.gtex}

\vfill
\pagebreak

\cuminitiali{}{temporalia/benedictio-solemn-evangelica.gtex}

\vspace{7mm}

\lectiovii

\noindent \Vbardot{} Tu autem, Dómine, miserére nobis.
\noindent \Rbardot{} Deo grátias.

\vfill
\pagebreak

\responsoriumvii

\vfill
\pagebreak

\cuminitiali{}{temporalia/benedictio-solemn-divinum.gtex}

\vspace{7mm}

\lectioviii

\noindent \Vbardot{} Tu autem, Dómine, miserére nobis.
\noindent \Rbardot{} Deo grátias.

\vfill
\pagebreak

\responsoriumviii

\vfill
\pagebreak

\cuminitiali{}{temporalia/benedictio-solemn-adsocietatem.gtex}

\vspace{7mm}

\lectioix

\noindent \Vbardot{} Tu autem, Dómine, miserére nobis.
\noindent \Rbardot{} Deo grátias.

\vfill
\pagebreak

% Te Deum

{
\pars{Hymnus Ambrosianus} \scriptura{Tonus Solemnis}

\vspace{-2mm}

\grechangedim{interwordspacetext}{0.26 cm plus 0.15 cm minus 0.05 cm}{scalable}%
\cuminitiali{III}{temporalia/tedeum-solemnis-gn.gtex}
\grechangedim{interwordspacetext}{0.22 cm plus 0.15 cm minus 0.05 cm}{scalable}%
}

\vfill
\pagebreak

\rubrica{Reliqua omittuntur, nisi Laudes separandæ sint.}

\pars{Oratio}

\noindent \Vbardot{} Dómine, exáudi oratiónem meam.

\noindent \Rbardot{} Et clamor meus ad te véniat.

Orémus:

\oratioLaudes

\vspace{7mm}

\pars{Conclusio}

\noindent \Vbardot{} Dómine, exáudi oratiónem meam.

\noindent \Rbardot{} Et clamor meus ad te véniat.

\noindent \Vbardot{} Benedicámus Dómino, allelúia, allelúia.

\noindent \Rbardot{} Deo grátias, allelúia, allelúia.

\noindent \Vbardot{} Fidélium ánimæ per misericórdiam Dei requiéscant in pace.

\noindent \Rbardot{} Amen.

\vfill
\pagebreak

\hora{Ad Laudes.} %%%%%%%%%%%%%%%%%%%%%%%%%%%%%%%%%%%%%%%%%%%%%%%%%%%%%
%\sideThumbs{Laudes}

\cantusSineNeumas

\vspace{0.5cm}
\grechangedim{interwordspacetext}{0.18 cm plus 0.15 cm minus 0.05 cm}{scalable}%
\cuminitiali{}{temporalia/deusinadiutorium-alter.gtex}
\grechangedim{interwordspacetext}{0.22 cm plus 0.15 cm minus 0.05 cm}{scalable}%

\vfill
%\pagebreak

\pars{Psalmus 1.}

\vspace{-4mm}

\antiphona{VI F}{temporalia/ant-alleluia1.gtex}

\scriptura{Psalmus 50.}

\initiumpsalmi{temporalia/ps50-initium-vi-F-auto.gtex}

%\psalmusEtTranslatioT{temporalia/ps50-I-comb.tex}{10cm}
\input{temporalia/ps50-I.tex}

\vfill
\pagebreak

\pars{Psalmus 2.}

\scriptura{Psalmus 117.}

\initiumpsalmi{temporalia/ps117-initium-vi-F-auto.gtex}

%\psalmusEtTranslatioT{temporalia/ps117-I-comb.tex}{10cm}
\input{temporalia/ps117-I.tex}

\vfill
\pagebreak

\pars{Psalmus 3.}

\scriptura{Psalmus 62.}

\initiumpsalmi{temporalia/ps62-initium-vi-F-auto.gtex}

%\psalmusEtTranslatioT{temporalia/ps62-I-comb.tex}{10cm}
\input{temporalia/ps62-I.tex}

\vfill

\vspace{-6mm}

\antiphona{}{temporalia/ant-alleluia1.gtex} % repeat the antiphon - new page

\vfill
\pagebreak

\pars{Psalmus 4.} \scriptura{Dan. 3, 22-26; \textbf{H422}}

\vspace{-4mm}

\antiphona{VIII G}{temporalia/ant-trespueri.gtex}

\scriptura{Canticum trium puerorum, Dan. 3, 57-88 et 56}

\initiumpsalmi{temporalia/dan3-initium-viii-G-auto.gtex}

%\psalmusEtTranslatioT{temporalia/dan3-comb.tex}{10cm}
\input{temporalia/dan3.tex}

\rubrica{Hic non dicitur Gloria Patri, neque Amen.}

\vfill

\vspace{-6mm}

\antiphona{}{temporalia/ant-trespueri.gtex} % repeat the antiphon - new page

\vfill
\pagebreak

\pars{Psalmus 5.}

\vspace{-4mm}

\antiphona{VIII G}{temporalia/ant-alleluia2.gtex}

\scriptura{Psalmus 148.}

\initiumpsalmi{temporalia/ps148-initium-viii-G-auto.gtex}

%\psalmusEtTranslatioT{temporalia/ps148-I-comb.tex}{10cm}
\input{temporalia/ps148-I.tex}

\rubrica{Hic non dicitur Gloria Patri.}

\vfill
\pagebreak

%
\scriptura{Psalmus 149.}

\initiumpsalmi{temporalia/ps149-initium-viii-G-auto.gtex}

%\psalmusEtTranslatioT{temporalia/ps149-I-comb.tex}{10cm}
\input{temporalia/ps149-I.tex}

\rubrica{Hic non dicitur Gloria Patri.}

\vfill
\pagebreak

%
\scriptura{Psalmus 150.}

\initiumpsalmi{temporalia/ps150-initium-viii-G-auto.gtex}

%\psalmusEtTranslatioT{temporalia/ps150-I-comb.tex}{10cm}
\input{temporalia/ps150-I.tex}

\vfill

\vspace{-6mm}

\antiphona{}{temporalia/ant-alleluia2.gtex} % repeat the antiphon - new page

\vfill
\pagebreak

\pars{Capitulum.} \scriptura{Ac. 7, 12}

\grechangedim{interwordspacetext}{0.12 cm plus 0.15 cm minus 0.05 cm}{scalable}%
\cuminitiali{}{temporalia/capitulum-Benedictio.gtex}
\grechangedim{interwordspacetext}{0.22 cm plus 0.15 cm minus 0.05 cm}{scalable}

% preklad Jeruz. bible
%\trCapituliI

\vfill

\pars{Responsorium breve.} \scriptura{Ps. 118, 36-37}

\cuminitiali{IV}{temporalia/resp-inclinacormeum.gtex}

%\trResp

\vfill
\pagebreak

\pars{Hymnus} \scriptura{Gregorius Magnus (\olddag{} 604)}

\cuminitiali{IV}{temporalia/hym-EcceJamNoctis.gtex}
\vspace{-3mm}
%\input{hym-EcceJamNocis-bohtext.tex}

\vfill
%\pagebreak

\pars{Versus.} \scriptura{Ps. 92, 1}

% Versus. %%%
\sineinitiali{temporalia/versus-dominusregnavit.gtex}

%\noindent \trVersus

\vfill
\pagebreak

\benedictus

\vspace{-1cm}

\vfill
\pagebreak

%\sideThumbs{{\scriptsize{}Fine horarum}}

\anteOrationem

\pagebreak

% Oratio. %%%
\oratioLaudes

\vspace{-1mm}
%\trOrationisI

\vfill

\rubrica{Hebdomadarius dicit iterum Dominus vobiscum, vel cantor dicit:}

\vspace{2mm}

\sineinitiali{temporalia/domineexaudi.gtex}

\rubrica{Postea cantatur a cantore:}

\vspace{2mm}

\cuminitiali{I}{temporalia/benedicamus-dominica-perannum.gtex}

\vspace{1mm}

\vfill
\pagebreak

\hora{Ad II. Vesperas.} %%%%%%%%%%%%%%%%%%%%%%%%%%%%%%%%%%%%%%%%%%%%%%%%%%%%%
%\sideThumbs{II. Vesperæ}

\cantusSineNeumas

%\vspace{0.5cm}
\grechangedim{interwordspacetext}{0.18 cm plus 0.15 cm minus 0.05 cm}{scalable}%
\cuminitiali{}{temporalia/deusinadiutorium-solemnis.gtex}
\grechangedim{interwordspacetext}{0.22 cm plus 0.15 cm minus 0.05 cm}{scalable}%

\vfill
%\pagebreak

\vspace{-2mm}

\pars{Psalmus 1.} \scriptura{Ps. 109, 1; \textbf{H91}}

\vspace{-4mm}

\antiphona{VII c\textsuperscript{2}}{temporalia/ant-dixitdominus.gtex}

\vspace{-4mm}

\scriptura{Psalmus 109.}

\initiumpsalmi{temporalia/ps109-initium-vii-c2-auto.gtex}

%\psalmusEtTranslatioT{temporalia/ps109-I-comb.tex}{10cm}
\input{temporalia/ps109-I.tex} \Abardot{}

\vspace{-1cm}

\vfill
\pagebreak

\pars{Psalmus 2.} \scriptura{Ps. 110, 8; \textbf{H91}}

\vspace{-4mm}

\antiphona{IV g}{temporalia/ant-fideliaomnia.gtex}

\scriptura{Psalmus 110.}

\initiumpsalmi{temporalia/ps110-initium-iv-g-auto.gtex}

%\psalmusEtTranslatioT{temporalia/ps110-I-comb.tex}{10cm}
\input{temporalia/ps110-I.tex} \Abardot{}

\vfill
\pagebreak

\pars{Psalmus 3.} \scriptura{Ps. 111, 1; \textbf{H92}}

\vspace{-4mm}

\antiphona{IV a}{temporalia/ant-inmandatis.gtex}

\scriptura{Psalmus 111.}

\initiumpsalmi{temporalia/ps111-initium-iv-a-auto.gtex}

%\psalmusEtTranslatioT{temporalia/ps111-I-comb.tex}{10cm}
\input{temporalia/ps111-I.tex} \Abardot{}

\vfill
\pagebreak

\pars{Psalmus 4.} \scriptura{Ps. 112, 2; \textbf{H92}}

\vspace{-4mm}

\antiphona{VII c}{temporalia/ant-sitnomendomini.gtex}

\scriptura{Psalmus 112.}

\initiumpsalmi{temporalia/ps112-initium-vii-c-auto.gtex}

%\psalmusEtTranslatioT{temporalia/ps112-I-comb.tex}{10cm}
\input{temporalia/ps112-I.tex} \Abardot{}

\vfill
\pagebreak

\pars{Capitulum.} \scriptura{2 Cor. 1, 3-4}

\grechangedim{interwordspacetext}{0.12 cm plus 0.15 cm minus 0.05 cm}{scalable}%
\cuminitiali{}{temporalia/capitulum-BenedictusDeus.gtex}
\grechangedim{interwordspacetext}{0.22 cm plus 0.15 cm minus 0.05 cm}{scalable}

% preklad Jeruz. bible
%\trCapituliI

\vfill

\pars{Responsorium breve.} \scriptura{Ps. 103, 24}

\cuminitiali{VI}{temporalia/resp-quammagnificata.gtex}

%\trResp

\vfill
\pagebreak

\pars{Hymnus} \scriptura{Gregorius Magnus (\olddag{} 604)}

\cuminitiali{I}{temporalia/hym-LucisCreator-aestivalis.gtex}
\vspace{-3mm}
%\begin{translatioMulticol}{3}
Tvůrce světa předobrý,\\
tys ustanovil denní řád\\
a proudy světla rozhodil,\\
když světu základy jsi klad.\\
\\
A spojils ráno s večerem\\
a dnem tu dobu nazýváš;\\
hle padá temné noci stín -\\
slyš prosbu, vyslyš nářek náš.\columnbreak

Ach, nedej, by nás stihla smrt,\\
když svědomí nám tíží hřích,\\
když nemyslíme na věčnost\\
v té síti hříchů šalebných.\\
\\
Vzbuď naši touhu po nebi,\\
kde věčný život čeká nás,\\
a pomoz odložit vše zlé\\
a smýti z duše každý kaz.\columnbreak

To splň nám, dobrý Otče náš,\\
i ty, jenž rovné božství máš,\\
i Duchu, který těšíš nás\\
a vládneš, Bože, v každý čas.\\
Amen. 
\end{translatioMulticol}


\vfill
%\pagebreak

\pars{Versus.} \scriptura{Ps. 140, 2}

% Versus. %%%
\sineinitiali{temporalia/versus-dirigatur.gtex}

%\noindent \trVersus

\vfill
\pagebreak

\magnificatii

\vfill
\pagebreak

%\sideThumbs{{\scriptsize{}Fine horarum}}

\anteOrationem

\pagebreak

% Oratio. %%%
\oratioLaudes

\vspace{-1mm}
%\trOrationisI

\vfill

\rubrica{Hebdomadarius dicit iterum Dominus vobiscum, vel cantor dicit:}

\vspace{2mm}

\sineinitiali{temporalia/domineexaudi.gtex}

\rubrica{Postea cantatur a cantore:}

\vspace{2mm}

\cuminitiali{I}{temporalia/benedicamus-dominica-perannum.gtex}

\vspace{1mm}

\end{document}

