\newcommand{\titulus}{\nomenFesti{Beatæ Mariæ Virginis Ecclesiæ Matris.}
\dies{Feria II post Pentecosten.}}
\newcommand{\oratio}{\pars{Oratio.}

\noindent Deus, misericordiárum Pater, cuius Unigénitus, cruci affíxus, beátam Maríam Vírginem, Genetrícem suam, Matrem quoque nostram constítuit, concéde, quǽsumus, ut, eius cooperánte caritáte, Ecclésia tua, in dies fecúndior, prolis sanctitáte exsúltet et in grémium suum cunctas áttrahat famílias populórum.

\pars{Pro pace in universo mundo.} \scriptura{Sir. 50, 25; 2 Esdr. 4, 20; \textbf{H416}}

\vspace{-4mm}

\antiphona{II D}{temporalia/ant-dapacemdomine.gtex}

\vfill

\noindent Deus, a quo sancta desidéria, recta consília et iusta sunt ópera: da servis tuis illam, quam mundus dare non potest, pacem; ut et corda nostra mandátis tuis dédita, et hóstium subláta formídine, témpora sint tua protectióne tranquílla.

\noindent Per Dóminum nostrum Iesum Christum, Fílium tuum, qui tecum vivit et regnat in unitáte Spíritus Sancti, Deus, per ómnia sǽcula sæculórum.

\noindent \Rbardot{} Amen.}
\newcommand{\invitatorium}{\pars{Invitatorium.} \scriptura{Lc. 1, 28; Psalmus 94}

\vspace{-4mm}

\antiphona{VII}{temporalia/inv-avemaria-simplex.gtex}}
\newcommand{\hymnusmatutinum}{\pars{Hymnus.}

\vspace{-5mm}

\antiphona{VIII}{temporalia/hym-QuemTerra-simplex.gtex}}
\newcommand{\matversus}{\noindent \Vbardot{} Quam dúlcia fáucibus meis elóquia tua, Dómine.

\noindent \Rbardot{} Super mel ori meo.}
\newcommand{\lectioi}{\pars{Lectio I.} \scriptura{Ios. 1, 1-9}

\noindent Incipit liber Iósue.

\noindent Factum est, ut post mortem Móysi servi Dómini loquerétur Dóminus ad Iósue fílium Nun minístrum Móysi et díceret ei:

\noindent «Móyses servus meus mórtuus est; nunc ígitur surge et transi Iordánem istum, tu et omnis pópulus iste in terram, quam ego dabo fíliis Israel.

\noindent Omnem locum, quem calcáverit vestígium pedis vestri, vobis trádidi, sicut locútus sum Móysi. A desérto et Líbano isto usque ad flúvium magnum Euphráten, omnis terra Hetthæórum usque ad mare Magnum contra solis occásum erit términus vester.

\noindent Nullus tibi póterit resístere cunctis diébus vitæ tuæ: sicut fui cum Móyse, ero et tecum; non dimíttam, nec derelínquam te.

\noindent Confortáre et esto robústus; tu enim sorte dívides pópulo huic terram, pro qua iurávi pátribus suis, ut tráderem eam illis.

\noindent Confortáre tantum et esto robústus valde, ut custódias et fácias iuxta omnem legem, quam præcépit tibi Móyses servus meus.

\noindent Ne declínes ab ea ad déxteram vel ad sinístram, ut prosperéris in ómnibus, ad quæcúmque perréxeris.

\noindent Non recédat hoc volúmen legis de ore tuo, sed meditáberis in eo diébus ac nóctibus, ut custódias et fácias iuxta ómnia, quæ scripta sunt in eo: tunc óptime díriges viam tuam et tunc prosperáberis.

\noindent Nonne præcépi tibi: Confortáre et esto robústus, noli metúere et noli timére, quóniam tecum est Dóminus Deus tuus in ómnibus, ad quæcúmque perréxeris?».

\noindent Præcepítque Iósue præféctis pópuli dicens: «Transíte per médium castrórum et imperáte pópulo ac dícite: Præparáte vobis cibária, quóniam post diem tértium transíbitis Iordánem hunc et intrábitis ad possidéndam terram, quam Dóminus Deus vester datúrus est vobis».

\noindent {\color{gray} Rubenítis quoque et Gadítis et dimídiæ tríbui Manásse ait: «Mementóte sermónis, quem præcépit vobis Móyses fámulus Dómini dicens: “Dóminus Deus vester dedit vobis réquiem et terram hanc”.

\noindent Uxóres vestræ et fílii ac iuménta manébunt in terra, quam trádidit vobis Móyses trans Iordánem; vos autem transíbitis armáti ante fratres vestros, omnes viri fortes, et adiuvábitis eos, donec det réquiem Dóminus frátribus vestris, sicut et vobis dedit, et possídeant ipsi quoque terram, quam Dóminus Deus vester datúrus est eis. Et sic revertémini in terram possessiónis vestræ et habitábitis in ea, quam vobis dedit Móyses fámulus Dómini trans Iordánem contra solis ortum».

\noindent Responderúntque ad Iósue atque dixérunt: «Omnia, quæ præcepísti nobis, faciémus et, quocúmque míseris, íbimus. Sicut obœdívimus in cunctis Móysi, ita obœdiémus et tibi; tantum sit Dóminus Deus tuus tecum, sicut fuit cum Móyse. Quicúmque contradíxerit ori tuo et non obœdíerit cunctis sermónibus, quos præcéperis ei, moriátur; tu tantum confortáre et viríliter age».}}
\newcommand{\responsoriumi}{\pars{Responsorium 1.} \scriptura{\Rbardot{} Ios. 1, 5-6; \textbf{H161}}

\vspace{-5mm}

\responsorium{II}{temporalia/resp-sicutfuicummoyse-CROCHU.gtex}{}}
\newcommand{\lectioii}{\pars{Lectio II.} \scriptura{Sermo 25, 7-8 : PL 46,937-939}

\noindent Ex Sermónibus sancti Augustíni epíscopi.

\noindent Ecce illud magis atténdite, caríssimi fratres mei, illud magis atténdite, óbsecro vos, quod ait Dóminus Christus exténdens manum super discípulos suos: \emph{Hæc est mater mea, et fratres mei.} Et: \emph{Qui fécerit voluntátem Patris mei qui me misit, ipse mihi et frater et soror et mater est.}

\noindent Numquid non fecit voluntátem Patris Virgo María, quæ fide crédidit, fide concépit, elécta est, de qua nobis salus inter hómines nascerétur, creáta est a Christo, ántequam in illa Christus crearétur?

\noindent Fecit, fecit plane voluntátem Patris sancta María, et ídeo plus est Maríæ discípulam fuísse Christi quam matrem fuísse Christi.

\noindent Plus est felícius discípulam fuísse Christi quam matrem fuísse Christi. Ideo María beáta erat, quia, et ántequam páreret, magístrum in útero portávit. María beáta, quia audívit verbum Dei et custodívit.

\noindent Plus mente custodívit veritátem quam útero carnem. Véritas Christus, caro Christus. Véritas Christus in mente Maríæ, caro Christus in ventre Maríæ.

\noindent Plus est quod est in mente quam quod portátur in ventre. Sancta María, beáta María, sed mélior est Ecclésia quam Virgo María.

\noindent Quare? Quia María pórtio est Ecclésiæ, sanctum membrum, excéllens membrum, superéminens membrum, sed tamen tótius córporis membrum. Si tótius córporis, plus est prolécto corpus quam membrum.

\noindent Caput Dóminus, et totus Christus caput et corpus. Quid dicam? Divínum caput habémus, Deum caput habémus.}
\newcommand{\responsoriumii}{\pars{Responsorium 2.} \scriptura{\Rbardot{} Sir. 24, 17 \Vbardot{} ibid. 20; \textbf{H296}}

\vspace{-5mm}

\responsorium{IV}{temporalia/resp-sicutcedrus.gtex}{}}
\newcommand{\lectioiii}{\pars{Lectio III.}

\noindent Ergo, caríssimi, vos atténdite. Et vos membra Christi estis, et vos corpus Christi estis. Audémus nos dícere matres Christi.

\noindent Dixi enim vos fratres eius omnes, et matrem suam non audérem? Sed multo minus áudeo, quod Christus dixit, negáre.

\noindent Eia, caríssimi, inténdite quómodo sit Ecclésia, quod maniféstum est, coniux Christi, quod difficílius intellégitur, sed tamen verum est, mater Christi. In ipsíus typo María Virgo præcéssit.

\noindent Unde rogo vos, María mater est Christi, nisi quia péperit membra Christi? Vos, quibus loquor, membra estis Christi.

\noindent Quis vos péperit? Audio vocem cordis vestri: «Mater Ecclésia.» Mater ista sancta, honoráta, Maríæ símilis et parit et virgo est.

\noindent Quia parit, per vos probo. Ex illa nati estis, et Christum parit; nam membra Christi estis. Ubi corrúpta est Eva sermóne serpéntis, ibi debet esse virgo Ecclésia dono Omnipoténtis.

\noindent Ergo in mente páriant membra Christi, sicut María in ventre Virgo péperit Christum, et sic éritis matres Christi.}
\newcommand{\responsoriumiii}{\pars{Responsorium 3.} \scriptura{\Vbardot{} Ct. 3, 6; \textbf{H296}}

%\vspace{-5mm}

\cuminitiali{III}{temporalia/resp-vidispeciosam.gtex}}
\newcommand{\hymnuslaudes}{\pars{Hymnus.}

\cuminitiali{II}{temporalia/hym-OGloriosaDomina-praglia-lh.gtex}}
\newcommand{\lectiobrevis}{\pars{Lectio brevis.} \scriptura{Cf. Is. 61, 10}

\noindent Gaudens gaudébo in Dómino, et exsultábit ánima mea in Deo meo, quia índuit me vestiméntis salútis et induménto iustítiæ circúmdedit me, quasi sponsam ornátam monílibus suis.}
\newcommand{\responsoriumbreve}{\pars{Responsorium breve.} \scriptura{Ps. 44, 3}

\antiphona{VI}{temporalia/resp-diffusaest.gtex}}
\newcommand{\benedictus}{\pars{Canticum Zachariæ.} \scriptura{Ac. 1, 14}

\vspace{-4mm}

\antiphona{VIII G}{temporalia/ant-erantomnes.gtex}

\scriptura{Lc. 1, 68-79}

%\vspace{-2mm}

\initiumpsalmi{temporalia/benedictus-initium-viii-G-auto.gtex}

%\vspace{-1.5mm}

\input{temporalia/benedictus-viii-G.tex} \Abardot{}}
\newcommand{\preces}{\noindent Salvatórem nostrum celebrántes,~\gredagger{} qui ex María Vírgine nasci dignátus est,~\grestar{} exorémus dicéntes:

\Rbardot{} Intercédat pro nobis mater tua, Dómine.

\noindent Salvátor mundi,~\gredagger{} qui redemptiónis tuæ virtúte ab omni peccáti labe matrem tuam præservásti,~\grestar{} serva nos mundos a peccáto.

\Rbardot{} Intercédat pro nobis mater tua, Dómine.

\noindent Redémptor noster, ~\gredagger{} qui Vírginem Maríam thálamum puríssimum habitatiónis tuæ et Spíritus Sancti fecísti sacrárium,~\grestar{} nos templum tui Spíritus fac perénne.

\Rbardot{} Intercédat pro nobis mater tua, Dómine.

\noindent Verbum ætérnum,~\gredagger{} quod matrem tuam docuísti óptimam sibi partem elígere,~\grestar{} tríbue nobis eam imitári, cibum quæréntes, qui permáneat in vitam ætérnam.

\Rbardot{} Intercédat pro nobis mater tua, Dómine.

\noindent Rex regum,~\gredagger{} qui matrem tuam córpore et ánima tecum voluísti in cælum assúmptam,~\grestar{} fac ut quæ sursum sunt semper cogitémus.

\Rbardot{} Intercédat pro nobis mater tua, Dómine.

\noindent Dómine cæli et terræ,~\gredagger{} qui Maríam regínam a dextris tuis astáre fecísti,~\grestar{} tríbue nos eiúsdem glóriæ meréri consórtium.

\Rbardot{} Intercédat pro nobis mater tua, Dómine.}
\newcommand{\benedicamuslaudes}{\cuminitiali{I}{temporalia/benedicamus-festis-bmv.gtex}}
\newcommand{\hebdomada}{infra Hebdom. X post Pentecosten.}
\newcommand{\oratioLaudes}{\cuminitiali{}{temporalia/oratio10.gtex}}

% LuaLaTeX

\documentclass[a4paper, twoside, 12pt]{article}
\usepackage[latin]{babel}
%\usepackage[landscape, left=3cm, right=1.5cm, top=2cm, bottom=1cm]{geometry} % okraje stranky
%\usepackage[landscape, a4paper, mag=1166, truedimen, left=2cm, right=1.5cm, top=1.6cm, bottom=0.95cm]{geometry} % okraje stranky
\usepackage[landscape, a4paper, mag=1400, truedimen, left=0.5cm, right=0.5cm, top=0.5cm, bottom=0.5cm]{geometry} % okraje stranky

\usepackage{fontspec}
\setmainfont[FeatureFile={junicode.fea}, Ligatures={Common, TeX}, RawFeature=+fixi]{Junicode}
%\setmainfont{Junicode}

% shortcut for Junicode without ligatures (for the Czech texts)
\newfontfamily\nlfont[FeatureFile={junicode.fea}, Ligatures={Common, TeX}, RawFeature=+fixi]{Junicode}

\usepackage{multicol}
\usepackage{color}
\usepackage{lettrine}
\usepackage{fancyhdr}

% usual packages loading:
\usepackage{luatextra}
\usepackage{graphicx} % support the \includegraphics command and options
\usepackage{gregoriotex} % for gregorio score inclusion
\usepackage{gregoriosyms}
\usepackage{wrapfig} % figures wrapped by the text
\usepackage{parcolumns}
\usepackage[contents={},opacity=1,scale=1,color=black]{background}
\usepackage{tikzpagenodes}
\usepackage{calc}
\usepackage{longtable}
\usetikzlibrary{calc}

\setlength{\headheight}{14.5pt}

\input{conventuscommune.tex} % Often used macros

\newcommand{\annusEditionis}{2021}

%%%% Vicekrat opakovane kousky

\newcommand{\anteOrationem}{
  \rubrica{Ante Orationem, cantatur a Superiore:}

  \pars{Supplicatio Litaniæ.}

  \cuminitiali{}{temporalia/supplicatiolitaniae.gtex}

  \pars{Oratio Dominica.}

  \cuminitiali{}{temporalia/oratiodominica.gtex}

  \rubrica{Deinde dicitur ab Hebdomadario:}

  \cuminitiali{}{temporalia/dominusvobiscum-solemnis.gtex}

  \rubrica{In choro monialium loco Dominus vobiscum dicitur:}

  \sineinitiali{temporalia/domineexaudi.gtex}
}

\setlength{\columnsep}{30pt} % prostor mezi sloupci

%%%%%%%%%%%%%%%%%%%%%%%%%%%%%%%%%%%%%%%%%%%%%%%%%%%%%%%%%%%%%%%%%%%%%%%%%%%%%%%%%%%%%%%%%%%%%%%%%%%%%%%%%%%%%
\begin{document}

% Here we set the space around the initial.
% Please report to http://home.gna.org/gregorio/gregoriotex/details for more details and options
\grechangedim{afterinitialshift}{2.2mm}{scalable}
\grechangedim{beforeinitialshift}{2.2mm}{scalable}
\grechangedim{interwordspacetext}{0.22 cm plus 0.15 cm minus 0.05 cm}{scalable}%
\grechangedim{annotationraise}{-0.2cm}{scalable}

% Here we set the initial font. Change 38 if you want a bigger initial.
% Emit the initials in red.
\grechangestyle{initial}{\color{red}\fontsize{38}{38}\selectfont}

\pagestyle{empty}

%%%% Titulni stranka
\begin{titulusOfficii}
\ifx\titulus\undefined
\nomenFesti{Feria II \hebdomada{}}
\else
\titulus
\fi
\end{titulusOfficii}

\vfill

\begin{center}
%Ad usum et secundum consuetudines chori \guillemotright{}Conventus Choralis\guillemotleft.

%Editio Sancti Wolfgangi \annusEditionis
\end{center}

\scriptura{}

\pars{}

\pagebreak

\renewcommand{\headrulewidth}{0pt} % no horiz. rule at the header
\fancyhf{}
\pagestyle{fancy}

\cantusSineNeumas

\ifx\oratio\undefined
\ifx\laudb\undefined
\else
\newcommand{\oratio}{\pars{Oratio.}

\noindent Dómine Deus omnípotens, qui ad princípium huius diéi nos perveníre fecísti, tua nos hódie salva virtúte, ut in hac die ad nullum declinémus peccátum, sed semper ad tuam iustítiam faciéndam nostra procédant elóquia, dirigántur cogitatiónes et ópera.

\noindent Per Dóminum nostrum Iesum Christum, Fílium tuum, qui tecum vivit et regnat in unitáte Spíritus Sancti, Deus, per ómnia sǽcula sæculórum.

\noindent \Rbardot{} Amen.}
\fi
\fi

\hora{Ad Matutinum.} %%%%%%%%%%%%%%%%%%%%%%%%%%%%%%%%%%%%%%%%%%%%%%%%%%%%%
%\sideThumbs{Matutinum}

\vspace{2mm}

\cuminitiali{}{temporalia/dominelabiamea.gtex}

\vfill
%\pagebreak

\vspace{2mm}

\ifx\invitatorium\undefined
\pars{Invitatorium.} \scriptura{Ps. 94, 1; Psalmus 94; \textbf{H451}}

\vspace{-6mm}

\antiphona{VI}{temporalia/inv-jubilemusdeo.gtex}\else
\invitatorium
\fi

\vfill
\pagebreak

\ifx\hymnusmatutinum\undefined
\ifx\matua\undefined
\else
\pars{Hymnus.}

{
\grechangedim{interwordspacetext}{0.10 cm plus 0.15 cm minus 0.05 cm}{scalable}%
\antiphona{II}{temporalia/hym-IpsumNunc.gtex}
\grechangedim{interwordspacetext}{0.22 cm plus 0.15 cm minus 0.05 cm}{scalable}%
}
\fi
\else
\hymnusmatutinum
\fi

\vspace{-3mm}

\vfill
\pagebreak

\ifx\matub\undefined
\else
% MAT B
\pars{Psalmus 1.} \scriptura{Ps. 30, 2; \textbf{H90}}

\vspace{-4mm}

\antiphona{VIII G}{temporalia/ant-intuaiustitia.gtex}

%\vspace{-2mm}

\scriptura{Ps. 30, 2-9}

%\vspace{-2mm}

\initiumpsalmi{temporalia/ps30i-initium-viii-G-auto.gtex}

\vspace{-1.5mm}

\input{temporalia/ps30i-viii-G.tex} \Abardot{}

\vfill
\pagebreak

\pars{Psalmus 2.} \scriptura{Ps. 66, 2}

\vspace{-4mm}

\antiphona{E}{temporalia/ant-illuminadomine.gtex}

%\vspace{-2mm}

\scriptura{Ps. 30, 10-17}

%\vspace{-2mm}

\initiumpsalmi{temporalia/ps30ii-initium-e-a-auto.gtex}

\input{temporalia/ps30ii-e-a.tex} \Abardot{}

\vfill
\pagebreak

\pars{Psalmus 3.} \scriptura{Ps. 30, 24}

\vspace{-4mm}

\antiphona{II D}{temporalia/ant-diligitedominum.gtex}

%\vspace{-5mm}

\scriptura{Ps. 30, 20-25}

%\vspace{-2mm}

\initiumpsalmi{temporalia/ps30iii-initium-ii-D-auto.gtex}

\input{temporalia/ps30iii-ii-D.tex} \Abardot{}

\vfill
\pagebreak
\fi

\pars{Versus.}

\ifx\matversus\undefined
\ifx\matub\undefined
\else
\noindent \Vbardot{} Dírige me, Dómine, in veritáte tua, et doce me.

\noindent \Rbardot{} Quia tu es Deus salútis meæ.
\fi
\else
\matversus
\fi

\vspace{5mm}

\sineinitiali{temporalia/oratiodominica-mat.gtex}

\vspace{5mm}

\pars{Absolutio.}

\cuminitiali{}{temporalia/absolutio-exaudi.gtex}

\vfill
\pagebreak

\cuminitiali{}{temporalia/benedictio-solemn-benedictione.gtex}

\vspace{7mm}

\lectioi

\noindent \Vbardot{} Tu autem, Dómine, miserére nobis.
\noindent \Rbardot{} Deo grátias.

\vfill
\pagebreak

\responsoriumi

\vfill
\pagebreak

\cuminitiali{}{temporalia/benedictio-solemn-unigenitus.gtex}

\vspace{7mm}

\lectioii

\noindent \Vbardot{} Tu autem, Dómine, miserére nobis.
\noindent \Rbardot{} Deo grátias.

\vfill
\pagebreak

\responsoriumii

\vfill
\pagebreak

\cuminitiali{}{temporalia/benedictio-solemn-spiritus.gtex}

\vspace{7mm}

\lectioiii

\noindent \Vbardot{} Tu autem, Dómine, miserére nobis.
\noindent \Rbardot{} Deo grátias.

\vfill
\pagebreak

\responsoriumiii

\vfill
\pagebreak

\rubrica{Reliqua omittuntur, nisi Laudes separandæ sint.}

\sineinitiali{temporalia/domineexaudi.gtex}

\vfill

\oratio

\vfill

\noindent \Vbardot{} Dómine, exáudi oratiónem meam.
\Rbardot{} Et clamor meus ad te véniat.

\vfill

\noindent \Vbardot{} Benedicámus Dómino.
\noindent \Rbardot{} Deo grátias.

\vfill

\noindent \Vbardot{} Fidélium ánimæ per misericórdiam Dei requiéscant in pace.
\Rbardot{} Amen.

\vfill
\pagebreak

\hora{Ad Laudes.} %%%%%%%%%%%%%%%%%%%%%%%%%%%%%%%%%%%%%%%%%%%%%%%%%%%%%
%\sideThumbs{Laudes}

\cantusSineNeumas

\vspace{0.5cm}
\grechangedim{interwordspacetext}{0.18 cm plus 0.15 cm minus 0.05 cm}{scalable}%
\cuminitiali{}{temporalia/deusinadiutorium-communis.gtex}
\grechangedim{interwordspacetext}{0.22 cm plus 0.15 cm minus 0.05 cm}{scalable}%

\vfill
\pagebreak

\ifx\hymnuslaudes\undefined
\ifx\laudbd\undefined
\else
\pars{Hymnus} \scriptura{Hilarius (\olddag{} 367)}

\grechangedim{interwordspacetext}{0.16 cm plus 0.15 cm minus 0.05 cm}{scalable}%
\cuminitiali{IV}{temporalia/hym-LucisLargitor.gtex}
\grechangedim{interwordspacetext}{0.22 cm plus 0.15 cm minus 0.05 cm}{scalable}%
\vspace{-3mm}
\fi
\else
\hymnuslaudes
\fi

\vfill
\pagebreak

\ifx\laudb\undefined
\else
\pars{Psalmus 1.} \scriptura{Ps. 41, 3; \textbf{H391}}

\vspace{-4mm}

\antiphona{II D}{temporalia/ant-sitivitanima.gtex}

%\vspace{-2mm}

\scriptura{Psalmus 41}

%\vspace{-2mm}

\initiumpsalmi{temporalia/ps41-initium-ii-D-auto.gtex}

%\vspace{-1.5mm}

\input{temporalia/ps41-ii-D.tex}

\vfill

\antiphona{}{temporalia/ant-sitivitanima.gtex}

\vfill
\pagebreak

\pars{Psalmus 2.}

\vspace{-4mm}

\antiphona{III a}{temporalia/ant-ostendenobisdomine.gtex}

%\vspace{-2mm}

\scriptura{Canticum Ecclesiastici, Sir. 36, 1-7.13-16}

%\vspace{-3mm}

\initiumpsalmi{temporalia/ecclesiastici-initium-iii-a-auto.gtex}

\input{temporalia/ecclesiastici-iii-a.tex} \Abardot{}

\vfill
\pagebreak

\pars{Psalmus 3.}

\vspace{-4mm}

\antiphona{II D}{temporalia/ant-operamanuumeius.gtex}

\scriptura{Psalmus 18, 1-7}

\initiumpsalmi{temporalia/ps18i-initium-ii-D-auto.gtex}

\input{temporalia/ps18i-ii-D.tex} \Abardot{}

\vfill
\pagebreak
\fi

\ifx\lectiobrevis\undefined
\ifx\laudb\undefined
\else
\pars{Lectio Brevis.} \scriptura{Ier. 15, 16}

\noindent Invénti sunt sermónes tui, et comédi eos, et factum est mihi verbum tuum in gáudium et in lætítiam cordis mei, quóniam invocátum est nomen tuum super me, Dómine Deus exercítuum.
\fi
\else
\lectiobrevis
\fi

\vfill

\ifx\responsoriumbreve\undefined
\ifx\laudbd\undefined
\else
\pars{Responsorium breve.} \scriptura{Ps. 32, 1.3}

\cuminitiali{VI}{temporalia/resp-exsultateiusti.gtex}
\fi
\else
\responsoriumbreve
\fi

\vfill
\pagebreak

\ifx\benedictus\undefined
\ifx\laudbd\undefined
\else
\pars{Canticum Zachariæ.} \scriptura{Lc. 1, 68; \textbf{H422}}

\vspace{-4mm}

{
\grechangedim{interwordspacetext}{0.18 cm plus 0.15 cm minus 0.05 cm}{scalable}%
\antiphona{IV E}{temporalia/ant-benedictusdominus.gtex}
\grechangedim{interwordspacetext}{0.22 cm plus 0.15 cm minus 0.05 cm}{scalable}%
}

%\vspace{-3mm}

\scriptura{Lc. 1, 68-79}

%\vspace{-2mm}

\cantusSineNeumas
\initiumpsalmi{temporalia/benedictus-initium-iv-E-auto.gtex}

%\vspace{-1.5mm}

\input{temporalia/benedictus-iv-E.tex} \Abardot{}
\fi
\else
\benedictus
\fi

\vspace{-1cm}

\vfill
\pagebreak

%\sideThumbs{{\scriptsize{}Fine horarum}}

\pars{Preces.}

\sineinitiali{}{temporalia/tonusprecum.gtex}

\ifx\preces\undefined
\ifx\laudb\undefined
\else
\noindent Salvátor noster fecit nos regnum et sacerdótium, ut hóstias Deo acceptábiles offerámus. \gredagger{} Grati ígitur eum invocémus:

\Rbardot{} Serva nos in tuo ministério, Dómine.

\noindent Christe, sacérdos ætérne, qui sanctum pópulo tuo sacerdótium concessísti, \gredagger{} concéde, ut spiritáles hóstias Deo acceptábiles iúgiter offerámus.

\Rbardot{} Serva nos in tuo ministério, Dómine.

\noindent Spíritus tui fructus nobis largíre propítius, \gredagger{} patiéntiam, benignitátem et mansuetúdinem.

\Rbardot{} Serva nos in tuo ministério, Dómine.

\noindent Da nobis te amáre, ut te, qui es cáritas, possideámus, \gredagger{} et bene ágere, ut per vitam étiam nostram te laudémus.

\Rbardot{} Serva nos in tuo ministério, Dómine.

\noindent Quæ frátribus nostris sunt utília, nos quǽrere concéde, \gredagger{} ut salútem facílius consequántur.

\Rbardot{} Serva nos in tuo ministério, Dómine.
\fi
\else
\preces
\fi

\vfill

\pars{Oratio Dominica.}

\cuminitiali{}{temporalia/oratiodominicaalt.gtex}

\vfill
\pagebreak

\rubrica{vel:}

\pars{Supplicatio Litaniæ.}

\cuminitiali{}{temporalia/supplicatiolitaniae.gtex}

\vfill

\pars{Oratio Dominica.}

\cuminitiali{}{temporalia/oratiodominica.gtex}

\vfill
\pagebreak

% Oratio. %%%
\oratio

\vspace{-1mm}

\vfill

\rubrica{Hebdomadarius dicit Dominus vobiscum, vel, absente sacerdote vel diacono, sic concluditur:}

\vspace{2mm}

\antiphona{C}{temporalia/dominusnosbenedicat.gtex}

\rubrica{Postea cantatur a cantore:}

\vspace{2mm}

\cuminitiali{IV}{temporalia/benedicamus-feria-laudes.gtex}

\vspace{1mm}

\vfill
\pagebreak

\end{document}

