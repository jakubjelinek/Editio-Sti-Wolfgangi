\newcommand{\titulus}{\nomenFesti{S. Francisci Assisiensis.}
\dies{Die 4. Octobris.}}
\newcommand{\oratio}{\pars{Oratio.}

\noindent Deus, qui beáto Francísco paupertáte et humilitáte Christo configurári tribuísti, concéde, ut, per illíus sémitas gradiéntes, Fílium tuum sequi et tibi coniúngi læta valeámus caritáte.

\noindent Per Dóminum nostrum Iesum Christum, Fílium tuum, qui tecum vivit et regnat in unitáte Spíritus Sancti, Deus, per ómnia sǽcula sæculórum.

\noindent \Rbardot{} Amen.}
\newcommand{\invitatorium}{\pars{Invitatorium.}

\vspace{-4mm}

\antiphona{IV}{temporalia/inv-mirabileminsanctis.gtex}}
\newcommand{\hymnusmatutinum}{\pars{Hymnus}

\cuminitiali{VIII}{temporalia/hym-InCaelestiCollegio.gtex}}
\newcommand{\matversus}{\noindent \Vbardot{} Audi, pópule meus, et loquar. 

\noindent \Rbardot{} Deus, Deus tuus ego sum.}
\newcommand{\lectioi}{\pars{Lectio I.} \scriptura{1 Tim. 2, 1-15}

\noindent De Epístola prima beáti Pauli apóstoli ad Timótheum.

\noindent Caríssime: Obsecro ígitur primo ómnium fíeri obsecratiónes, oratiónes, postulatiónes, gratiárum actiónes pro ómnibus homínibus, pro régibus et ómnibus, qui in sublimitáte sunt, ut quiétam et tranquíllam vitam agámus in omni pietáte et castitáte. Hoc bonum est et accéptum coram salvatóre nostro Deo, qui omnes hómines vult salvos fíeri et ad agnitiónem veritátis veníre. Unus enim Deus, unus et mediátor Dei et hóminum, homo Christus Iesus, qui dedit redemptiónem semetípsum pro ómnibus, testimónium tempóribus suis; in quod pósitus sum ego prædicátor et apóstolus — veritátem dico, non méntior — doctor géntium in fide et veritáte.

\noindent Volo ergo viros oráre in omni loco levántes puras manus sine ira et disceptatióne; simíliter et mulíeres in hábitu ornáto cum verecúndia et sobrietáte ornántes se, non in tortis crínibus et auro aut margarítis vel veste pretiósa, sed, quod decet mulíeres, profiténtes pietátem per ópera bona.

\noindent Múlier in tranquillitáte discat cum omni subiectióne; docére autem mulíeri non permítto, neque dominári in virum, sed esse in tranquillitáte. Adam enim primus formátus est, deínde Eva; et Adam non est sedúctus, múlier autem sedúcta in prævaricatióne fuit. Salvábitur autem per filiórum generatiónem, si permánserint in fide et dilectióne et sanctificatióne cum sobrietáte.}
\newcommand{\responsoriumi}{\pars{Responsorium 1.} \scriptura{\Rbardot{} Ps. 36, 3 \Vbardot{} ibid., 5; \textbf{H86}}

\vspace{-5mm}

\responsorium{V}{temporalia/resp-delectareindomino-CROCHU.gtex}{}}
\newcommand{\lectioii}{\pars{Lectio II.} \scriptura{Opuscula, edit. Quaracchi 1949, 87-94}

\noindent E Lítteris a sancto Francísco Asisiénsi ómnibus fidélibus scriptis.

\noindent Verbum Patris tam dignum, tam sanctum et gloriósum nuntiávit altíssimus Pater de cælo ventúrum per sanctum Gabriélem archángelum suum sanctæ et gloriósæ Vírgini Maríæ, ex cuius útero veram recépit carnem humanitátis et fragilitátis nostræ. Qui, cum dives esset super ómnia, vóluit tamen ipse cum beatíssima matre sua elígere paupertátem. Et prope passiónem suam celebrávit Pascha cum discípulis suis. Deínde orávit ad Patrem dicens: \emph{Pater, si fíeri potest, tránseat a me calix iste.}

\noindent Pósuit tamen voluntátem suam in voluntáte Patris. Cuius Patris talis fuit volúntas, ut Fílius eius benedíctus et gloriósus, quem dedit nobis et natus fuit pro nobis, se ipsum per próprium sánguinem suum sacrifícium et hóstiam in ara crucis offérret, non pro se, per quem facta sunt ómnia, sed pro peccátis nostris, relínquens nobis exémplum, ut sequámur vestígia eius. Et vult ut omnes salvémur per ipsum et recipiámus ipsum puro corde et casto córpore.}
\newcommand{\responsoriumii}{\pars{Responsorium 2.}

\vspace{-5mm}

\responsorium{I}{temporalia/resp-franciscusutinpublicum.gtex}{}}
\newcommand{\lectioiii}{\pars{Lectio III.}

\noindent O quam beáti et benedícti sunt illi, qui Dóminum díligunt et fáciunt sicut dicit ipse Dóminus in Evangélio: \emph{Díliges Dóminum Deum tuum ex toto corde tuo et ex tota ánima tua et próximum tuum sicut te ipsum!} Diligámus ígitur Deum et adorémus eum puro corde et pura mente, quia ipse super ómnia quærens hoc dicit: \emph{Veri adoratóres adorábunt Patrem in spíritu et veritáte.} Omnes enim, qui adórant eum, in spíritu veritátis opórtet eum adoráre. Et dirigámus ei laudes et oratiónes die ac nocte dicéndo: \emph{Pater noster, qui es in cælis; quia opórtet nos semper oráre et non defícere.}

\noindent Faciámus ínsuper fructus dignos pæniténtiæ. Et diligámus próximos sicut nosmetípsos. Habeámus caritátem et humilitátem et faciámus eleemósynas, quia ipsæ lavant ánimas a sórdibus peccatórum. Hómines enim ómnia perdunt, quæ in hoc sǽculo relínquunt; secum tamen portant caritátis mercédem et eleemósynas quas fecérunt, de quibus habébunt a Dómino prǽmium et dignam remuneratiónem.

\noindent Non debémus secúndum carnem esse sapiéntes atque prudéntes, sed magis debémus esse símplices, húmiles et puri. Numquam debémus desideráre esse super álios, sed magis debémus esse servi et súbditi omni humánæ creatúræ propter Deum. Et omnes, qui tália fécerint et perseveráverint usque in finem, requiéscet super eos Spíritus Dómini et fáciet in eis habitáculum et mansiónem et erunt fílii Patris cæléstis, cuius ópera fáciunt, et sunt sponsi, fratres et matres Dómini nostri Iesu Christi.}
\newcommand{\responsoriumiii}{\pars{Responsorium 3.} \scriptura{\Rbardot{} Mt. 5, 3.5-6 \Vbardot{} ibid., 7; \textbf{H333}}

\vspace{-5mm}

\responsorium{VII}{temporalia/resp-beatipauperesspiritu-CROCHU.gtex}{}}
\newcommand{\hymnuslaudes}{\pars{Hymnus}

\cuminitiali{III}{temporalia/hym-PlaudeTurba.gtex}}
\newcommand{\laudes}{\pars{Psalmus 1.}

\vspace{-4mm}

\antiphona{II D}{temporalia/ant-sanctusfranciscus.gtex}

%\vspace{-2mm}

\scriptura{Psalmus 62.}

%\vspace{-1mm}

\initiumpsalmi{temporalia/ps62-initium-ii-D.gtex}

%\vspace{-1.5mm}

%\input{temporalia/ps62-ii-D.tex} \Abardot{}
\input{temporalia/ps62-viii-C2.tex} \Abardot{}

\vfill
\pagebreak

\pars{Psalmus 2.}

\vspace{-4mm}

\antiphona{V g}{temporalia/ant-doctusdoctrinaegratia.gtex}

\vspace{-2mm}

%\trAntIV

\scriptura{Canticum trium puerorum, Dan. 3, 57-88 et 56}

\vspace{-2mm}

\initiumpsalmi{temporalia/dan3-initium-v-g.gtex}

\input{temporalia/dan3-v-g-sinedox.tex}

\rubrica{Hic non dicitur Gloria Patri, neque Amen.}

\vfill

\antiphona{}{temporalia/ant-doctusdoctrinaegratia.gtex}

\vfill
\pagebreak

\pars{Psalmus 3.}

\vspace{-4mm}

\antiphona{VI F}{temporalia/ant-laudanslaudare.gtex}

%\vspace{-2mm}

\scriptura{Psalmus 149}

%\vspace{-2mm}

\initiumpsalmi{temporalia/ps149-initium-vi-F-auto.gtex}

\input{temporalia/ps149-vi-F.tex} \Abardot{}

\vfill
\pagebreak}
\newcommand{\lectiobrevis}{\pars{Lectio Brevis.} \scriptura{Rom. 12, 1-2}

\noindent Obsecro vos, fratres, per misericórdiam Dei, ut exhibeátis córpora vestra hóstiam vivéntem, sanctam, Deo placéntem, rationábile obséquium vestrum; et nolíte conformári huic sǽculo, sed transformámini renovatióne mentis, ut probétis quid sit volúntas Dei, quid bonum et bene placens et perféctum.}
\newcommand{\responsoriumbreve}{\pars{Responsorium breve.} \scriptura{Ps. 67, 36}

\cuminitiali{VI}{temporalia/resp-mirabilisdeus.gtex}}
\newcommand{\preces}{\noindent Christum Deum sanctum, fratres, exaltémus, orántes ut serviámus illi in sanctitáte et iustítia coram ipso ómnibus diébus nostris, \gredagger{} et acclamémus:

\Rbardot{} Tu solus sanctus, Dómine.

\noindent Qui tentári voluísti per ómnia pro similitúdine nostra absque peccáto, \gredagger{} miserére nostri, Dómine Iesu.

\Rbardot{} Tu solus sanctus, Dómine.

\noindent Qui nos omnes ad perfectiónem caritátis vocásti, \gredagger{} sanctífica nos, Dómine Iesu.

\Rbardot{} Tu solus sanctus, Dómine.

\noindent Qui nos iussísti esse salem terræ et lucem mundi, \gredagger{} illúmina nos, Dómine Iesu.

\Rbardot{} Tu solus sanctus, Dómine.

\noindent Qui voluísti ministráre, non ministrári, \gredagger{} fac nos tibi et frátribus humíliter servíre, Dómine Iesu.

\Rbardot{} Tu solus sanctus, Dómine.

\noindent Tu, splendor glóriæ Patris et figúra substántiæ eius, \gredagger{} fac ut in glória vultum tuum respiciámus, Dómine Iesu.

\Rbardot{} Tu solus sanctus, Dómine.}
\newcommand{\benedictus}{\pars{Canticum Zachariæ.}

\vspace{-4mm}

\antiphona{VII a}{temporalia/ant-omartyrdesiderii.gtex}

%\vspace{-3mm}

\scriptura{Lc. 1, 68-79}

%\vspace{-2mm}

\cantusSineNeumas
\initiumpsalmi{temporalia/benedictus-initium-vii-a-auto.gtex}

%\vspace{-1.5mm}

\input{temporalia/benedictus-vii-a.tex}

\antiphona{}{temporalia/ant-omartyrdesiderii.gtex}}
\newcommand{\benedicamuslaudes}{\cuminitiali{VIII}{temporalia/benedicamus-adlaudes-festis.gtex}}
\include{hebdomadaxxvii}
% LuaLaTeX

\documentclass[a4paper, twoside, 12pt]{article}
\usepackage[latin]{babel}
%\usepackage[landscape, left=3cm, right=1.5cm, top=2cm, bottom=1cm]{geometry} % okraje stranky
%\usepackage[landscape, a4paper, mag=1166, truedimen, left=2cm, right=1.5cm, top=1.6cm, bottom=0.95cm]{geometry} % okraje stranky
\usepackage[landscape, a4paper, mag=1400, truedimen, left=0.5cm, right=0.5cm, top=0.5cm, bottom=0.5cm]{geometry} % okraje stranky

\usepackage{fontspec}
\setmainfont[FeatureFile={junicode.fea}, Ligatures={Common, TeX}, RawFeature=+fixi]{Junicode}
%\setmainfont{Junicode}

% shortcut for Junicode without ligatures (for the Czech texts)
\newfontfamily\nlfont[FeatureFile={junicode.fea}, Ligatures={Common, TeX}, RawFeature=+fixi]{Junicode}

\usepackage{multicol}
\usepackage{color}
\usepackage{lettrine}
\usepackage{fancyhdr}

% usual packages loading:
\usepackage{luatextra}
\usepackage{graphicx} % support the \includegraphics command and options
\usepackage{gregoriotex} % for gregorio score inclusion
\usepackage{gregoriosyms}
\usepackage{wrapfig} % figures wrapped by the text
\usepackage{parcolumns}
\usepackage[contents={},opacity=1,scale=1,color=black]{background}
\usepackage{tikzpagenodes}
\usepackage{calc}
\usepackage{longtable}
\usetikzlibrary{calc}

\setlength{\headheight}{14.5pt}

\input{conventuscommune.tex} % Often used macros

\newcommand{\annusEditionis}{2021}

%%%% Vicekrat opakovane kousky

\newcommand{\anteOrationem}{
  \rubrica{Ante Orationem, cantatur a Superiore:}

  \pars{Supplicatio Litaniæ.}

  \cuminitiali{}{temporalia/supplicatiolitaniae.gtex}

  \pars{Oratio Dominica.}

  \cuminitiali{}{temporalia/oratiodominica.gtex}

  \rubrica{Deinde dicitur ab Hebdomadario:}

  \cuminitiali{}{temporalia/dominusvobiscum-solemnis.gtex}

  \rubrica{In choro monialium loco Dominus vobiscum dicitur:}

  \sineinitiali{temporalia/domineexaudi.gtex}
}

\setlength{\columnsep}{30pt} % prostor mezi sloupci

%%%%%%%%%%%%%%%%%%%%%%%%%%%%%%%%%%%%%%%%%%%%%%%%%%%%%%%%%%%%%%%%%%%%%%%%%%%%%%%%%%%%%%%%%%%%%%%%%%%%%%%%%%%%%
\begin{document}

% Here we set the space around the initial.
% Please report to http://home.gna.org/gregorio/gregoriotex/details for more details and options
\grechangedim{afterinitialshift}{2.2mm}{scalable}
\grechangedim{beforeinitialshift}{2.2mm}{scalable}
\grechangedim{interwordspacetext}{0.22 cm plus 0.15 cm minus 0.05 cm}{scalable}%
\grechangedim{annotationraise}{-0.2cm}{scalable}

% Here we set the initial font. Change 38 if you want a bigger initial.
% Emit the initials in red.
\grechangestyle{initial}{\color{red}\fontsize{38}{38}\selectfont}

\pagestyle{empty}

%%%% Titulni stranka
\begin{titulusOfficii}
\ifx\titulus\undefined
\nomenFesti{Feria II \hebdomada{}}
\else
\titulus
\fi
\end{titulusOfficii}

\vfill

\begin{center}
%Ad usum et secundum consuetudines chori \guillemotright{}Conventus Choralis\guillemotleft.

%Editio Sancti Wolfgangi \annusEditionis
\end{center}

\scriptura{}

\pars{}

\pagebreak

\renewcommand{\headrulewidth}{0pt} % no horiz. rule at the header
\fancyhf{}
\pagestyle{fancy}

\cantusSineNeumas

\ifx\oratio\undefined
\ifx\laudb\undefined
\else
\newcommand{\oratio}{\pars{Oratio.}

\noindent Dómine Deus omnípotens, qui ad princípium huius diéi nos perveníre fecísti, tua nos hódie salva virtúte, ut in hac die ad nullum declinémus peccátum, sed semper ad tuam iustítiam faciéndam nostra procédant elóquia, dirigántur cogitatiónes et ópera.

\noindent Per Dóminum nostrum Iesum Christum, Fílium tuum, qui tecum vivit et regnat in unitáte Spíritus Sancti, Deus, per ómnia sǽcula sæculórum.

\noindent \Rbardot{} Amen.}
\fi
\fi

\hora{Ad Matutinum.} %%%%%%%%%%%%%%%%%%%%%%%%%%%%%%%%%%%%%%%%%%%%%%%%%%%%%
%\sideThumbs{Matutinum}

\vspace{2mm}

\cuminitiali{}{temporalia/dominelabiamea.gtex}

\vfill
%\pagebreak

\vspace{2mm}

\ifx\invitatorium\undefined
\pars{Invitatorium.} \scriptura{Ps. 94, 1; Psalmus 94; \textbf{H451}}

\vspace{-6mm}

\antiphona{VI}{temporalia/inv-jubilemusdeo.gtex}\else
\invitatorium
\fi

\vfill
\pagebreak

\ifx\hymnusmatutinum\undefined
\ifx\matua\undefined
\else
\pars{Hymnus.}

{
\grechangedim{interwordspacetext}{0.10 cm plus 0.15 cm minus 0.05 cm}{scalable}%
\antiphona{II}{temporalia/hym-IpsumNunc.gtex}
\grechangedim{interwordspacetext}{0.22 cm plus 0.15 cm minus 0.05 cm}{scalable}%
}
\fi
\else
\hymnusmatutinum
\fi

\vspace{-3mm}

\vfill
\pagebreak

\ifx\matub\undefined
\else
% MAT B
\pars{Psalmus 1.} \scriptura{Ps. 30, 2; \textbf{H90}}

\vspace{-4mm}

\antiphona{VIII G}{temporalia/ant-intuaiustitia.gtex}

%\vspace{-2mm}

\scriptura{Ps. 30, 2-9}

%\vspace{-2mm}

\initiumpsalmi{temporalia/ps30i-initium-viii-G-auto.gtex}

\vspace{-1.5mm}

\input{temporalia/ps30i-viii-G.tex} \Abardot{}

\vfill
\pagebreak

\pars{Psalmus 2.} \scriptura{Ps. 66, 2}

\vspace{-4mm}

\antiphona{E}{temporalia/ant-illuminadomine.gtex}

%\vspace{-2mm}

\scriptura{Ps. 30, 10-17}

%\vspace{-2mm}

\initiumpsalmi{temporalia/ps30ii-initium-e-a-auto.gtex}

\input{temporalia/ps30ii-e-a.tex} \Abardot{}

\vfill
\pagebreak

\pars{Psalmus 3.} \scriptura{Ps. 30, 24}

\vspace{-4mm}

\antiphona{II D}{temporalia/ant-diligitedominum.gtex}

%\vspace{-5mm}

\scriptura{Ps. 30, 20-25}

%\vspace{-2mm}

\initiumpsalmi{temporalia/ps30iii-initium-ii-D-auto.gtex}

\input{temporalia/ps30iii-ii-D.tex} \Abardot{}

\vfill
\pagebreak
\fi

\pars{Versus.}

\ifx\matversus\undefined
\ifx\matub\undefined
\else
\noindent \Vbardot{} Dírige me, Dómine, in veritáte tua, et doce me.

\noindent \Rbardot{} Quia tu es Deus salútis meæ.
\fi
\else
\matversus
\fi

\vspace{5mm}

\sineinitiali{temporalia/oratiodominica-mat.gtex}

\vspace{5mm}

\pars{Absolutio.}

\cuminitiali{}{temporalia/absolutio-exaudi.gtex}

\vfill
\pagebreak

\cuminitiali{}{temporalia/benedictio-solemn-benedictione.gtex}

\vspace{7mm}

\lectioi

\noindent \Vbardot{} Tu autem, Dómine, miserére nobis.
\noindent \Rbardot{} Deo grátias.

\vfill
\pagebreak

\responsoriumi

\vfill
\pagebreak

\cuminitiali{}{temporalia/benedictio-solemn-unigenitus.gtex}

\vspace{7mm}

\lectioii

\noindent \Vbardot{} Tu autem, Dómine, miserére nobis.
\noindent \Rbardot{} Deo grátias.

\vfill
\pagebreak

\responsoriumii

\vfill
\pagebreak

\cuminitiali{}{temporalia/benedictio-solemn-spiritus.gtex}

\vspace{7mm}

\lectioiii

\noindent \Vbardot{} Tu autem, Dómine, miserére nobis.
\noindent \Rbardot{} Deo grátias.

\vfill
\pagebreak

\responsoriumiii

\vfill
\pagebreak

\rubrica{Reliqua omittuntur, nisi Laudes separandæ sint.}

\sineinitiali{temporalia/domineexaudi.gtex}

\vfill

\oratio

\vfill

\noindent \Vbardot{} Dómine, exáudi oratiónem meam.
\Rbardot{} Et clamor meus ad te véniat.

\vfill

\noindent \Vbardot{} Benedicámus Dómino.
\noindent \Rbardot{} Deo grátias.

\vfill

\noindent \Vbardot{} Fidélium ánimæ per misericórdiam Dei requiéscant in pace.
\Rbardot{} Amen.

\vfill
\pagebreak

\hora{Ad Laudes.} %%%%%%%%%%%%%%%%%%%%%%%%%%%%%%%%%%%%%%%%%%%%%%%%%%%%%
%\sideThumbs{Laudes}

\cantusSineNeumas

\vspace{0.5cm}
\grechangedim{interwordspacetext}{0.18 cm plus 0.15 cm minus 0.05 cm}{scalable}%
\cuminitiali{}{temporalia/deusinadiutorium-communis.gtex}
\grechangedim{interwordspacetext}{0.22 cm plus 0.15 cm minus 0.05 cm}{scalable}%

\vfill
\pagebreak

\ifx\hymnuslaudes\undefined
\ifx\laudbd\undefined
\else
\pars{Hymnus} \scriptura{Hilarius (\olddag{} 367)}

\grechangedim{interwordspacetext}{0.16 cm plus 0.15 cm minus 0.05 cm}{scalable}%
\cuminitiali{IV}{temporalia/hym-LucisLargitor.gtex}
\grechangedim{interwordspacetext}{0.22 cm plus 0.15 cm minus 0.05 cm}{scalable}%
\vspace{-3mm}
\fi
\else
\hymnuslaudes
\fi

\vfill
\pagebreak

\ifx\laudb\undefined
\else
\pars{Psalmus 1.} \scriptura{Ps. 41, 3; \textbf{H391}}

\vspace{-4mm}

\antiphona{II D}{temporalia/ant-sitivitanima.gtex}

%\vspace{-2mm}

\scriptura{Psalmus 41}

%\vspace{-2mm}

\initiumpsalmi{temporalia/ps41-initium-ii-D-auto.gtex}

%\vspace{-1.5mm}

\input{temporalia/ps41-ii-D.tex}

\vfill

\antiphona{}{temporalia/ant-sitivitanima.gtex}

\vfill
\pagebreak

\pars{Psalmus 2.}

\vspace{-4mm}

\antiphona{III a}{temporalia/ant-ostendenobisdomine.gtex}

%\vspace{-2mm}

\scriptura{Canticum Ecclesiastici, Sir. 36, 1-7.13-16}

%\vspace{-3mm}

\initiumpsalmi{temporalia/ecclesiastici-initium-iii-a-auto.gtex}

\input{temporalia/ecclesiastici-iii-a.tex} \Abardot{}

\vfill
\pagebreak

\pars{Psalmus 3.}

\vspace{-4mm}

\antiphona{II D}{temporalia/ant-operamanuumeius.gtex}

\scriptura{Psalmus 18, 1-7}

\initiumpsalmi{temporalia/ps18i-initium-ii-D-auto.gtex}

\input{temporalia/ps18i-ii-D.tex} \Abardot{}

\vfill
\pagebreak
\fi

\ifx\lectiobrevis\undefined
\ifx\laudb\undefined
\else
\pars{Lectio Brevis.} \scriptura{Ier. 15, 16}

\noindent Invénti sunt sermónes tui, et comédi eos, et factum est mihi verbum tuum in gáudium et in lætítiam cordis mei, quóniam invocátum est nomen tuum super me, Dómine Deus exercítuum.
\fi
\else
\lectiobrevis
\fi

\vfill

\ifx\responsoriumbreve\undefined
\ifx\laudbd\undefined
\else
\pars{Responsorium breve.} \scriptura{Ps. 32, 1.3}

\cuminitiali{VI}{temporalia/resp-exsultateiusti.gtex}
\fi
\else
\responsoriumbreve
\fi

\vfill
\pagebreak

\ifx\benedictus\undefined
\ifx\laudbd\undefined
\else
\pars{Canticum Zachariæ.} \scriptura{Lc. 1, 68; \textbf{H422}}

\vspace{-4mm}

{
\grechangedim{interwordspacetext}{0.18 cm plus 0.15 cm minus 0.05 cm}{scalable}%
\antiphona{IV E}{temporalia/ant-benedictusdominus.gtex}
\grechangedim{interwordspacetext}{0.22 cm plus 0.15 cm minus 0.05 cm}{scalable}%
}

%\vspace{-3mm}

\scriptura{Lc. 1, 68-79}

%\vspace{-2mm}

\cantusSineNeumas
\initiumpsalmi{temporalia/benedictus-initium-iv-E-auto.gtex}

%\vspace{-1.5mm}

\input{temporalia/benedictus-iv-E.tex} \Abardot{}
\fi
\else
\benedictus
\fi

\vspace{-1cm}

\vfill
\pagebreak

%\sideThumbs{{\scriptsize{}Fine horarum}}

\pars{Preces.}

\sineinitiali{}{temporalia/tonusprecum.gtex}

\ifx\preces\undefined
\ifx\laudb\undefined
\else
\noindent Salvátor noster fecit nos regnum et sacerdótium, ut hóstias Deo acceptábiles offerámus. \gredagger{} Grati ígitur eum invocémus:

\Rbardot{} Serva nos in tuo ministério, Dómine.

\noindent Christe, sacérdos ætérne, qui sanctum pópulo tuo sacerdótium concessísti, \gredagger{} concéde, ut spiritáles hóstias Deo acceptábiles iúgiter offerámus.

\Rbardot{} Serva nos in tuo ministério, Dómine.

\noindent Spíritus tui fructus nobis largíre propítius, \gredagger{} patiéntiam, benignitátem et mansuetúdinem.

\Rbardot{} Serva nos in tuo ministério, Dómine.

\noindent Da nobis te amáre, ut te, qui es cáritas, possideámus, \gredagger{} et bene ágere, ut per vitam étiam nostram te laudémus.

\Rbardot{} Serva nos in tuo ministério, Dómine.

\noindent Quæ frátribus nostris sunt utília, nos quǽrere concéde, \gredagger{} ut salútem facílius consequántur.

\Rbardot{} Serva nos in tuo ministério, Dómine.
\fi
\else
\preces
\fi

\vfill

\pars{Oratio Dominica.}

\cuminitiali{}{temporalia/oratiodominicaalt.gtex}

\vfill
\pagebreak

\rubrica{vel:}

\pars{Supplicatio Litaniæ.}

\cuminitiali{}{temporalia/supplicatiolitaniae.gtex}

\vfill

\pars{Oratio Dominica.}

\cuminitiali{}{temporalia/oratiodominica.gtex}

\vfill
\pagebreak

% Oratio. %%%
\oratio

\vspace{-1mm}

\vfill

\rubrica{Hebdomadarius dicit Dominus vobiscum, vel, absente sacerdote vel diacono, sic concluditur:}

\vspace{2mm}

\antiphona{C}{temporalia/dominusnosbenedicat.gtex}

\rubrica{Postea cantatur a cantore:}

\vspace{2mm}

\cuminitiali{IV}{temporalia/benedicamus-feria-laudes.gtex}

\vspace{1mm}

\vfill
\pagebreak

\end{document}

