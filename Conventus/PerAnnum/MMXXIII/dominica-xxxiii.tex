\newcommand{\titulus}{\nomenFesti{Dominica XXXIII per Annum.}}
\newcommand{\oratio}{\pars{Oratio.}

\noindent Da nobis, quǽsumus, Dómine Deus noster, in tua semper devotióne gaudére, quia perpétua est et plena felícitas, si bonórum ómnium iúgiter serviámus auctóri.

\pars{Pro pace in universo mundo.} \scriptura{Sir. 50, 25; 2 Esdr. 4, 20; \textbf{H416}}

\vspace{-4mm}

\antiphona{II D}{temporalia/ant-dapacemdomine.gtex}

\vfill

\noindent Deus, a quo sancta desidéria, recta consília et iusta sunt ópera: da servis tuis illam, quam mundus dare non potest, pacem; ut et corda nostra mandátis tuis dédita, et hóstium subláta formídine, témpora sint tua protectióne tranquílla.

\noindent Per Dóminum nostrum Iesum Christum, Fílium tuum, qui tecum vivit et regnat in unitáte Spíritus Sancti, Deus, per ómnia sǽcula sæculórum.

\noindent \Rbardot{} Amen.}
\newcommand{\tedeummonasticum}{Monasticum}
\newcommand{\hymnusmatutinum}{\pars{Hymnus.}

\vspace{-5mm}

\antiphona{IV}{temporalia/hym-PrimoDierum-worcester.gtex}}
\newcommand{\nocturnoi}{\pars{Psalmus 1.} \scriptura{Ps. 1, 2; \textbf{H372}}

\vspace{-4mm}

\antiphona{I d}{temporalia/ant-inlegedominifuit.gtex}

%\vspace{-2mm}

\scriptura{Ps. 1}

%\vspace{-2mm}

\initiumpsalmi{temporalia/ps1-initium-i-d-auto.gtex}

\input{temporalia/ps1-i-d.tex} \Abardot{}

\vfill
\pagebreak

\pars{Psalmus 2.} \scriptura{Ps. 2, 6; \textbf{H65}}

\vspace{-4mm}

\antiphona{I a\textsuperscript{2}}{temporalia/ant-praedicanspraeceptum.gtex}

\vspace{-1mm}

\scriptura{Ps. 2}

\vspace{-2mm}

\initiumpsalmi{temporalia/ps2-initium-i-a4-auto.gtex}

\input{temporalia/ps2-i-a4.tex} \Abardot{}

\vfill
\pagebreak

\pars{Psalmus 3.} \scriptura{Ps. 3, 4-5; \textbf{H376}}

\vspace{-4mm}

\antiphona{VIII G\textsuperscript{2}}{temporalia/ant-tuesgloriamea.gtex}

%\vspace{-2mm}

\scriptura{Ps. 3}

\initiumpsalmi{temporalia/ps3-initium-viii-G5-auto.gtex}

\input{temporalia/ps3-viii-G5.tex} \Abardot{}

\vfill
\pagebreak
}
\newcommand{\nocturnoii}{\vspace{-4mm}

\pars{Psalmus 4.} \scriptura{Ps. 102, 1; \textbf{H99}}

\vspace{-4mm}

\antiphona{VIII c}{temporalia/ant-benedicanimamea.gtex}

%\vspace{-2mm}

\scriptura{Ps. 102, 1-7}

%\vspace{-2mm}

\initiumpsalmi{temporalia/ps102i-initium-viii-C-auto.gtex}

\input{temporalia/ps102i-viii-C.tex} \Abardot{}

\vfill
\pagebreak

\pars{Psalmus 5.} \scriptura{Ps. 102, 11}

\vspace{-4mm}

\antiphona{I d}{temporalia/ant-supertimentesdominum.gtex}

%\vspace{-2mm}

\scriptura{Ps. 102, 8-16}

%\vspace{-2mm}

\initiumpsalmi{temporalia/ps102ii-initium-i-d-auto.gtex}

\input{temporalia/ps102ii-i-d.tex} \Abardot{}

\vfill
\pagebreak

\pars{Psalmus 6.} \scriptura{Ps. 102, 20; \textbf{H332}}

\vspace{-4mm}

\antiphona{III g (VIII*)}{temporalia/ant-benedicitedomino.gtex}

%\vspace{-5mm}

\scriptura{Ps. 102, 17-22}

%\vspace{-2mm}

\initiumpsalmi{temporalia/ps102iii-initium-iii-g.gtex}

\input{temporalia/ps102iii-iii-g.tex} \Abardot{}

\vfill
\pagebreak}
\newcommand{\nocturnoiii}{\pars{Cantica.}

\vspace{-4mm}

\antiphona{D}{temporalia/ant-eccedeusnoster.gtex}

%\vspace{-2mm}

\scriptura{Canticum Isaiæ, Is. 33, 2-10}

%\vspace{-2mm}

\initiumpsalmi{temporalia/isaiae7-initium-d-g-auto.gtex}

\input{temporalia/isaiae7-d-g.tex} \hfill \rubrica{Hic non dicitur antiphona.}

\vfill
\pagebreak

\scriptura{Canticum Isaiæ, Is. 33, 13-17}

%\vspace{-2mm}

\initiumpsalmi{temporalia/isaiae8-initium-d-g-auto.gtex}

\input{temporalia/isaiae8-d-g.tex}

\vfill
\pagebreak

\scriptura{Canticum Ecclesiastici, Sir. 36, 14-19}

%\vspace{-2mm}

\initiumpsalmi{temporalia/ecclesiasticus36-initium-d-g-auto.gtex}

\input{temporalia/ecclesiasticus36-d-g.tex}

\vfill

\antiphona{}{temporalia/ant-eccedeusnoster.gtex}

\vfill
\pagebreak}
\newcommand{\lectioi}{\pars{Lectio I.} \scriptura{Ioel 2, 21-22}

\noindent De libro Ioélis prophétæ.

\noindent Hæc dicit Dóminus: «Noli timére, terra; exsúlta et lætáre, quóniam magna Dóminus operátus est. Nolíte timére, animália regiónis, quia germinavérunt páscua desérti, quia lignum áttulit fructum suum, ficus et vínea dedérunt divítias suas.}
\newcommand{\responsoriumi}{\pars{Responsorium 1.} \scriptura{\Rbardot{} Is. 6, 1 \Vbardot{} Is. 6, 2; \textbf{H416}}

\vspace{-5mm}

\responsorium{I}{temporalia/resp-vididominumsedentem-CROCHU-sinedox.gtex}{}}
\newcommand{\lectioii}{\pars{Lectio II.} \scriptura{Ioel 2, 23-27}

\noindent Et, fílii Sion, exsultáte et lætámini in Dómino Deo vestro, quia dedit vobis plúviam iustítiæ et descéndere fecit ad vos imbrem matutínum et serótinum sicut prius. Et implebúntur áreæ fruménto, et redundábunt torculária vino et óleo; et reddam vobis annos, quos comédit locústa, bruchus et gryllos et erúca, exércitus meus magnus, quem misi in vos. Et comedétis vescéntes et saturabímini et laudábitis nomen Dómini Dei vestri, qui mirabília fecit vobíscum; et non confundétur pópulus meus in sempitérnum. Et sciétis quia in médio Israel ego sum, et ego Dóminus Deus vester, et non est ámplius; et non confundétur pópulus meus in ætérnum».}
\newcommand{\responsoriumii}{\pars{Responsorium 2.} \scriptura{\Rbardot{} Ier. 31, 11.12 \Vbardot{} Ps. 4, 8; \textbf{H418}}

\vspace{-5mm}

\responsorium{III}{temporalia/resp-redemitdominus-CROCHU.gtex}{}}
\newcommand{\lectioiii}{\pars{Lectio III.} \scriptura{Ioel 3, 1-5}

\noindent Et erit post hæc: effúndam spíritum meum super omnem carnem, et prophetábunt fílii vestri et fíliæ vestræ, senes vestri sómnia somniábunt, et iúvenes vestri visiónes vidébunt; sed et super servos meos et ancíllas in diébus illis effúndam spíritum meum. Et dabo prodígia in cælo et in terra, sánguinem et ignem et colúmnas fumi; sol convertétur in ténebras et luna in sánguinem, ántequam véniat dies Dómini magnus et horríbilis. Et erit: omnis, qui invocáverit nomen Dómini, salvus erit, quia in monte Sion et in Ierúsalem erit salvátio, sicut dixit Dóminus, et in resíduis, quos Dóminus vocáverit.}
\newcommand{\responsoriumiii}{\pars{Responsorium 3.} \scriptura{\Rbardot{} Lam. 1, 1-2; \textbf{H416}}

\vspace{-5mm}

\responsorium{I}{temporalia/resp-aspicedominequiafactaest-CROCHU-cumdox.gtex}{}}
\newcommand{\lectioiv}{\pars{Lectio IV.} \scriptura{Ps 95, 14.15 : CCL 39, 1351-1353}

\noindent Ex Enarratiónibus sancti Augustíni epíscopi in Psalmos.

\noindent \emph{Tunc exsultábunt ómnia ligna silvárum ante fáciem Dómini, quóniam venit, quóniam venit iudicáre terram.} Primo venit, et póstea ventúrus est. Hæc primo vox ipsíus sónuit in Evangélio: \emph{Amodo vidébitis Fílium hóminis veniéntem in núbibus.} Quid est: \emph{Amodo?} Nonne Dóminus póstea ventúrus est, quando se plangent omnes tribus terræ? Prius venit in prædicatóribus suis, et implévit totum orbem terrárum. Non resistámus primo advéntui, ut non expavescámus secúndum.

\noindent Quid ergo debet fácere christiánus? Uti mundo non servíre mundo. Quid est hoc? Habéntes tamquam non habéntes. Sic dicit Apóstolus: \emph{De cétero fratres, tempus breve est: réliquum est ut et hi qui habent uxóres, tamquam non habéntes sint; et qui flent, tamquam non flentes; et qui gaudent, tamquam non gaudéntes; et qui emunt, quasi non tenéntes; et qui utúntur hoc mundo, tamquam non uténtes: prǽterit enim figúra huius mundi. Volo vos sine sollicitúdine esse.} Qui sine sollicitúdine est, secúrus exspéctat quando véniat Dóminus ipsíus. Nam qualis amor est Christi, timére ne véniat? Fratres, non erubéscimus? Amámus, et timémus ne véniat. Certe amámus? an peccáta nostra plus amámus? Ergo ipsa odérimus peccáta, et amémus eum qui ventúrus est ad puniénda peccáta. Véniet, velímus, nolímus; non enim quia modo non venit, ídeo ventúrus non est. Véniet, et quando nescis; et si parátum te invénerit, nihil tibi obest quia nescis.}
\newcommand{\responsoriumiv}{\pars{Responsorium 4.} \scriptura{\Rbardot{} Michææ 6, 8 \Vbardot{} Ps. 36, 6; \textbf{H418}}

\vspace{-5mm}

\responsorium{V}{temporalia/resp-indicabotibihomo-CROCHU.gtex}{}}
\newcommand{\lectiov}{\pars{Lectio V.}

\noindent \emph{Et exsultábunt ómnia ligna silvárum.} Venit primo, et póstea iudicáre terram: exsultántes invéniet eos, qui primo eius advéntui credidérunt, \emph{quóniam venit.}

\noindent \emph{Iudicábit orbem terrárum in æquitáte et pópulos in veritáte sua.} Quæ est ǽquitas et véritas? Congregábit secum eléctos suos ad iudicándum, céteros autem separábit ab ínvicem; positúrus est enim álios ad déxteram, álios ad sinístram. Quid autem ǽquius, quid vérius, quam ut non exspéctent misericórdiam de iúdice, qui noluérunt fácere misericórdiam, ántequam veníret iudex? Qui autem voluérunt fácere misericórdiam, cum misericórdia iudicabúntur. Dicétur enim eis ad déxteram pósitis: \emph{Veníte, benedícti Patris mei, percípite regnum quod vobis parátum est ab orígine mundi.} Et ímputat ópera misericórdiæ: \emph{Esurívi enim, et dedístis mihi manducáre; sitívi et potástis me,} et cétera.}
\newcommand{\responsoriumv}{\pars{Responsorium 5.} \scriptura{\Rbardot{} Is. 19, 25 \Vbardot{} Ps. 32, 12; \textbf{H418}}

\vspace{-5mm}

\responsorium{VIII}{temporalia/resp-laudabilispopulus-CROCHU.gtex}{}}
\newcommand{\lectiovi}{\pars{Lectio VI.}

\noindent Rursus ad sinístram pósitis quid imputátur? Quia noluérunt fácere misericórdiam. Et quo ibunt? \emph{Ite in ignem ætérnum.} Iste audítus malus magnum gémitum fáciet. Sed quid dixit álius psalmus? \emph{In memória ætérna erit iustus; ab audítu malo non timébit.} Quid est, audítus malus? \emph{Ite in ignem ætérnum, qui parátus est diábolo et ángelis eius.} Qui gaudébit ad audítum bonum, non timébit ab audítu malo. Hæc est ǽquitas, hæc véritas.

\noindent An quia tu iniústus es, iustus non erit iudex? aut quia tu mendax es, verax non erit véritas? Sed si vis habére misericórdem, esto miséricors ántequam véniat: dimítte si quid in te commíssum est, da ex eo quo abúndas. Ex de cuius das, nisi de illíus? Si de tuo dares, largítio esset; cum de illíus das, reddítio est. \emph{Quid enim habes, quod non accepísti?} Hæ sunt hóstiæ Deo gratíssimæ: misericórdia, humílitas, conféssio, pax, cáritas. Has apportémus, et secúri exspectábimus advéntum iúdicis, qui \emph{iudicábit orbem terrárum in æquitáte et pópulos in veritáte sua.}}
\newcommand{\responsoriumvi}{\pars{Responsorium 6.} \scriptura{\Rbardot{} Ier. 14, 19.20 \Vbardot{} Bar. 2, 12; \textbf{H417}}

\vspace{-5mm}

\responsorium{VIII}{temporalia/resp-sustinuimuspacemetnonvenit-CROCHU-cumdox.gtex}{}}
\newcommand{\evangelium}{
\pars{Versus.} \scriptura{Ps. 118, 148}

% Versus. %%%
\sineinitiali{temporalia/versus-praevenerunt.gtex}

\vspace{5mm}

\sineinitiali{temporalia/oratiodominica-mat.gtex}

\vspace{5mm}

\pars{Absolutio.}

\cuminitiali{}{temporalia/absolutio-avinculis.gtex}

\vfill
\pagebreak

\cuminitiali{}{temporalia/benedictio-solemn-evangelica.gtex}

\vspace{7mm}

\pars{Evangelium} \scriptura{Mt. 25, 14-20}

\noindent Léctio sancti sancti Evangélii secúndum Matthǽum.

\noindent In illo témpore: Dixit Iesus discípulis suis parábolam hanc:

\noindent «Homo péregre proficíscens vocávit servos suos et trádidit illis bona sua. Et uni dedit quinque talénta, álii autem duo, álii vero unum, unicuíque secúndum própriam virtútem, et proféctus est.

\noindent {\color{gray} Statim ábiit, qui quinque talénta accéperat, et operátus est in eis et lucrátus est ália quinque; simíliter qui duo accéperat, lucrátus est ália duo. Qui autem unum accéperat, ábiens fodit in terra et abscóndit pecúniam dómini sui.}

\noindent Post multum vero témporis venit dóminus servórum illórum et ponit ratiónem cum eis.

\noindent Et accédens, qui quinque talénta accéperat, óbtulit ália quinque talénta dicens: “Dómine, quinque talénta tradidísti mihi; ecce ália quinque superlucrátus sum”.

\noindent {\color{gray} Ait illi dóminus eius: “Euge, serve bone et fidélis. Super pauca fuísti fidélis; supra multa te constítuam: intra in gáudium dómini tui”.

\noindent Accéssit autem et qui duo talénta accéperat, et ait: “Dómine, duo talénta tradidísti mihi; ecce ália duo lucrátus sum”.

\noindent Ait illi dóminus eius: “Euge, serve bone et fidélis. Super pauca fuísti fidélis; supra multa te constítuam: intra in gáudium dómini tui”.

\noindent Accédens autem et qui unum taléntum accéperat, ait: “Dómine, novi te quia homo durus es: metis, ubi non seminásti, et cóngregas, ubi non sparsísti; et timens ábii et abscóndi taléntum tuum in terra. Ecce habes, quod tuum est”.

\noindent Respóndens autem dóminus eius dixit ei: “Serve male et piger! Sciébas quia meto, ubi non seminávi, et cóngrego, ubi non sparsi? Opórtuit ergo te míttere pecúniam meam nummuláriis, et véniens ego recepíssem, quod meum est cum usúra. Tóllite ítaque ab eo taléntum et date ei, qui habet decem talénta: omni enim habénti dábitur, et abundábit; ei autem, qui non habet, et quod habet, auferétur ab eo. Et inútilem servum eícite in ténebras exterióres: illic erit fletus et stridor déntium”».}

\scriptura{Hom. 9,1 : PL 76,1106}

\noindent Ex Homíliis sancti Gregórii Magni papæ in Evangélia.

\noindent Léctio sancti evangélii, fratres caríssimi, sollícite consideráre nos ádmonet, ne nos, qui plus céteris in hoc mundo accepísse áliquid cérnimur ab auctóre mundi, grávius inde iudicémur. Cum enim augéntur dona, ratiónes étiam crescunt donórum. Tanto ergo esse humílior atque ad serviéndum prómptior quisque debet ex múnere, quanto se obligatiórem esse cónspicit in reddénda ratióne.

\noindent Quis ítaque iste homo est qui péregre proficíscitur, nisi Redémptor noster qui in ea carne quam assúmpserat ábiit in cælum? Carnis enim locus próprius terra est, quæ quasi ad peregrína dúcitur, dum per Redemptórem nostrum in cælo collocátur. Sed \emph{homo} iste \emph{péregre proficíscens, servis suis bona sua trádidit,} quia fidélibus suis spiritália dona concéssit.

\emph{Et uni} quidem \emph{quinque talénta, álii duo, álii vero} commísit \emph{unum.} Quinque étenim sunt córporis sensus; quinque ergo taléntis, donum quinque sénsuum, id est exteriórum sciéntia, exprímitur. Duóbus vero intelléctus et operátio designátur. Uníus autem talénti nómine intelléctus tantúmmodo designátur.

Sed \emph{is qui quinque talénta accéperat ália quinque lucrátus est,} quia sunt nonnúlli qui, etsi intérna ac mýstica penetráre nésciunt, pro intentióne tamen supérnæ pátriæ docent recta quos possunt de ipsis exterióribus quæ accepérunt; dumque se a carnis petulántia et a terrenárum rerum ámbitu, atque a visibílium voluptáte, custódiunt, ab his étiam álios admonéndo compéscunt. Et sunt nonnúlli qui, quasi duóbus taléntis ditáti, intelléctum atque operatiónem percípiunt, subtília de intérnis intéllegunt, mira in exterióribus operántur.

\vfill
\pagebreak

\pars{Responsorium 7.} \scriptura{\Rbardot{} Mt. 25, 21 \Vbardot{} ibid., 20; \textbf{H377}}

\vspace{-5mm}

\responsorium{VII}{temporalia/resp-eugeservebone-CROCHU-cumdox.gtex}{}

\vfill
\pagebreak
}
\newcommand{\responsoriumbreve}{\pars{Responsorium breve.}

\cuminitiali{VI}{temporalia/resp-christefilidei.gtex}}
\newcommand{\benedictus}{\pars{Canticum Zachariæ.} \scriptura{Mt. 25, 20; \textbf{H380}}

\vspace{-4mm}

{
\grechangedim{interwordspacetext}{0.18 cm plus 0.15 cm minus 0.05 cm}{scalable}%
\antiphona{I g\textsuperscript{5}}{temporalia/ant-dominequinquetalenta.gtex}
\grechangedim{interwordspacetext}{0.22 cm plus 0.15 cm minus 0.05 cm}{scalable}%
}

%\trAntIMagnificat

\vspace{-1mm}

\scriptura{Lc. 1, 68-79}

\vspace{-1mm}

\cantusSineNeumas
\initiumpsalmi{temporalia/benedictus-initium-isoll-g5-auto.gtex}

%\vspace{-1.5mm}

\input{temporalia/benedictus-isoll-g5.tex} \Abardot{}}
\newcommand{\hebdomada}{infra Hebdom. XXXIII per Annum.}
\newcommand{\hiemalis}{Hiemalis}
\newcommand{\matua}{Matutinum Hebdomadae A}
\newcommand{\matuac}{Matutinum Hebdomadae A vel C}
\newcommand{\lauda}{Laudes Hebdomadae A}
\newcommand{\laudac}{Laudes Hebdomadae A vel C}

% LuaLaTeX

\documentclass[a4paper, twoside, 12pt]{article}
\usepackage[latin]{babel}
%\usepackage[landscape, left=3cm, right=1.5cm, top=2cm, bottom=1cm]{geometry} % okraje stranky
%\usepackage[landscape, a4paper, mag=1166, truedimen, left=2cm, right=1.5cm, top=1.6cm, bottom=0.95cm]{geometry} % okraje stranky
\usepackage[landscape, a4paper, mag=1400, truedimen, left=0.5cm, right=0.5cm, top=0.5cm, bottom=0.5cm]{geometry} % okraje stranky

\usepackage{fontspec}
\setmainfont[FeatureFile={junicode.fea}, Ligatures={Common, TeX}, RawFeature=+fixi]{Junicode}
%\setmainfont{Junicode}

% shortcut for Junicode without ligatures (for the Czech texts)
\newfontfamily\nlfont[FeatureFile={junicode.fea}, Ligatures={Common, TeX}, RawFeature=+fixi]{Junicode}

\usepackage{multicol}
\usepackage{color}
\usepackage{lettrine}
\usepackage{fancyhdr}

% usual packages loading:
\usepackage{luatextra}
\usepackage{graphicx} % support the \includegraphics command and options
\usepackage{gregoriotex} % for gregorio score inclusion
\usepackage{gregoriosyms}
\usepackage{wrapfig} % figures wrapped by the text
\usepackage{parcolumns}
\usepackage[contents={},opacity=1,scale=1,color=black]{background}
\usepackage{tikzpagenodes}
\usepackage{calc}
\usepackage{longtable}
\usetikzlibrary{calc}

\setlength{\headheight}{14.5pt}

\input{conventuscommune.tex} % Often used macros
%%%% Preklady jednotlivych zpevu (nektere se opakuji, a je dobre mit je
% vsechny na jedne hromade)

% HOURS ---

\newcommand{\trAntI}{\translatioCantus{Muž boží měl kožený toulec, pečlivě
zavázaný, jenž mu visel na šíji a~často se ho dotýkal.}}

\newcommand{\trAntII}{\translatioCantus{Klíč od~něho tak dobře střežil, že
dokud žil v~těle, nikdo z~jeho žáků nezvěděl, co je uvnitř.}}

\newcommand{\trAntIII}{\translatioCantus{Ale když se odebral z~tohoto
života, schránku otevřeli a~objevili v~ní žíněné roucho a~měděný řetěz
potřísněný krví.}}

\newcommand{\trAntIV}{\translatioCantus{A když prohlédli mistrovo tělo,
nalezli jeho tělo na čtyřech místech hluboce zbrázděno ranami od řetězu.}}

\newcommand{\trAntV}{\translatioCantus{Krev vytékající z~těch ran, místy
prostoupila i~žíněným rouchem.}}

\newcommand{\trCapituli}{\translatioCantus{
Miláčkovi Boha a~lidí,
Mojžíšovi požehnané paměti,~\gredagger{}
dopřál slávu rovnou slávě svatých~\grestar{}
učinil ho mocným na postrach nepřátelům
a~jeho slovy zastavil divy.}}

\newcommand{\trLectioBrevis}{\translatioCantus{
Pamatujte na své představené,
kteří vám hlásali Boží slovo.
Uvažte, jak oni skončili život, a~napodobujte jejich víru.
Ježíš Kristus je stejný včera i~dnes i~navěky.
Nenechte se svést věelijakými cizími naukami.}}

\newcommand{\trRespLaud}{\translatioCantus{Spravedlivého vodil Hospodin~\grestar{}
po přímých stezkách. \Vbardot{} A~ukázal mu Boží království.}}

\newcommand{\trRespLaudB}{\translatioCantus{Na tvých hradbách, Jeruzaléme,
ustanovil jsem strážné;~\grestar{}
budou bdít nad mým lidem. \Vbardot{} Ani ve dne, ani v~noci nesmějí nikdy
mlčet.}}

\newcommand{\trVersus}{\translatioCantus{\Vbardot{} Ústa spravedlivého šeptají moudrost, aleluja.
\Rbardot{} A~jeho jazyk ohlašuje právo, aleluja.}}

\newcommand{\trAntBenedictus}{\translatioCantus{Když na bujné oře vložili
nosítka a~sňali jim uzdu, vydali se přímo k~cele božího muže.}}

\newcommand{\trPreces}{\translatioCantus{
\noindent S vděčností chvalme Krista, dobrého Pastýře, \gredagger{} který dal život za své ovce, \grestar{} a~pokorně ho prosme: \Rbardot{} Pane, buď pastýřem svého lidu.

\noindent Kriste, ty dáváš církvi pastýře, a~jejich službou se ujímáš svého lidu, \grestar{} dej, ať v~lásce těch, kteří nás vedou, poznáváme, jak nás miluješ. \Rbardot{} Pane, buď pastýřem svého lidu.

\noindent Ty stále konáš skrze své zástupce službu pastýře a~učitele, \grestar{} nepřestávej nás nikdy vést prostřednictvím svých služebníků. \Rbardot{} Pane, buď pastýřem svého lidu.

\noindent Ty prokazuješ svému lidu skrze jeho pastýře službu lékaře duše i~těla, \grestar{} ochraňuj náš život a~veď nás ke svatosti. \Rbardot{} Pane, buď pastýřem svého lidu.

\noindent Ty posíláš své svaté, aby slovem i~příkladem vedli tvůj lid k~tobě, \grestar{} na jejich přímluvu nás posiluj, abychom vytrvali na cestě, která vede k~věčnému životu. \Rbardot{} Pane, buď pastýřem svého lidu.}}

\newcommand{\trOrationis}{\translatioCantus{Bože, jenž nám dopřáváš radovat
se z~výroční slavnosti svatého tvého vyznavače Havla, uděl dobrotivě,
abychom když slavíme jeho narození, též se řídili podobou jeho skutků.
Skrze…}}
 % Czech translations of the proper texts

\newcommand{\annusEditionis}{2020}

%%%% Vicekrat opakovane kousky

\newcommand{\anteOrationem}{
  \rubrica{Ante Orationem, cantatur a Superiore:}

  \pars{Supplicatio Litaniæ.}

  \cuminitiali{}{temporalia/supplicatiolitaniae.gtex}

  \pars{Oratio Dominica.}

  \cuminitiali{}{temporalia/oratiodominica.gtex}

  \rubrica{Deinde dicitur ab Hebdomadario:}

  \cuminitiali{}{temporalia/dominusvobiscum-solemnis.gtex}

  \rubrica{In choro monialium loco Dominus vobiscum dicitur:}

  \sineinitiali{temporalia/domineexaudi.gtex}
}

\setlength{\columnsep}{30pt} % prostor mezi sloupci

%%%%%%%%%%%%%%%%%%%%%%%%%%%%%%%%%%%%%%%%%%%%%%%%%%%%%%%%%%%%%%%%%%%%%%%%%%%%%%%%%%%%%%%%%%%%%%%%%%%%%%%%%%%%%
\begin{document}

% Here we set the space around the initial.
% Please report to http://home.gna.org/gregorio/gregoriotex/details for more details and options
\grechangedim{afterinitialshift}{2.2mm}{scalable}
\grechangedim{beforeinitialshift}{2.2mm}{scalable}
\grechangedim{interwordspacetext}{0.22 cm plus 0.15 cm minus 0.05 cm}{scalable}%
\grechangedim{annotationraise}{-0.2cm}{scalable}

% Here we set the initial font. Change 38 if you want a bigger initial.
% Emit the initials in red.
\grechangestyle{initial}{\color{red}\fontsize{38}{38}\selectfont}

\pagestyle{empty}

%%%% Titulni stranka
\begin{titulusOfficii}
\titulus{}
\end{titulusOfficii}

% graphic
%\vspace{1.5cm}
%\begin{center}
%\includegraphics[width=8cm]{emmaus.jpg}
%\end{center}

\vfill

\begin{center}
%Ad usum et secundum consuetudines chori \guillemotright{}Conventus Choralis\guillemotleft.

%Editio Sancti Wolfgangi \annusEditionis
\end{center}

\pagebreak

\renewcommand{\headrulewidth}{0pt} % no horiz. rule at the header
\fancyhf{}
\pagestyle{fancy}

\pars{Oratio ante divinum Officium.}

\lettrine{{\color{red}A}}{peri,} Dómine, os meum ad benedicéndum nomen sanctum tuum:
munda quoque cor meum ab ómnibus vanis, pervérsis, et aliénis
cogitatiónibus:
intelléctum illúmina, afféctum inflámma,
ut digne, atténte ac devóte hoc Offícium recitáre váleam,
et exaudíri mérear ante conspéctum Divínæ Maiestátis tuæ.
Per Christum, Dóminum nostrum.
\Rbardot{} Amen.

Dómine, in unióne illíus divínæ intentiónis,
qua ipse in terris laudes Deo persolvísti,
has tibi Horas \rubricatum{(vel \textnormal{hanc tibi Horam})} persólvo.

%\trOratioAnteOfficium

\vfill

\pars{Oratio post divinum Officium.}

\rubrica{
  Orationem sequentem devote post Officium recitantibus
  Leo Papa X. defectus, et culpas in eo persolvendo ex humana
  fragilitate contractas, indulsit, et dicitur flexis genibus.
}

\lettrine{{\color{red}S}}{acrosánctæ} et indivíduæ Trinitáti,
crucifíxi Dómini nostri Iesu Christi humanitáti,
beatíssimæ et gloriosíssimæ sempérque Vírginis Maríæ
fecúndæ integritáti, 
et ómnium Sanctórum universitáti
sit sempitérna laus, honor, virtus et glória
ab omni creatúra,
nobísque remíssio ómnium peccatórum,
per infiníta sǽcula sæculórum.
\Rbardot{} Amen.

\noindent \Vbardot{} Beáta víscera Maríæ Virginis, quæ portavérunt
ætérni Patris Fílium.\\
\Rbardot{} Et beáta úbera, quæ lactavérunt Christum Dominum.

\rubrica{Et dicitur secreto \textnormal{Pater noster.} et \textnormal{Ave María.}}

%\trOratioPostOfficium

\vfill

\hora{Ad I. Vesperas.} %%%%%%%%%%%%%%%%%%%%%%%%%%%%%%%%%%%%%%%%%%%%%%%%%%%%%
%\sideThumbs{I. Vesperæ}

\cantusSineNeumas

\vspace{0.5cm}
\grechangedim{interwordspacetext}{0.18 cm plus 0.15 cm minus 0.05 cm}{scalable}%
\cuminitiali{}{temporalia/deusinadiutorium-solemnis.gtex}
\grechangedim{interwordspacetext}{0.22 cm plus 0.15 cm minus 0.05 cm}{scalable}%

\vfill
\pagebreak

\pars{Psalmus 1.} \scriptura{Ps. 144, 13; \textbf{H100}}

\vspace{-4mm}

\antiphona{VII c\textsuperscript{2}}{temporalia/ant-regnumtuum.gtex}

\scriptura{Psalmus 144, 10-21.}

\initiumpsalmi{temporalia/ps144ii-initium-vii-c2-auto.gtex}

%\psalmusEtTranslatioT{temporalia/ps144ii-VII-comb.tex}{10cm}
\input{temporalia/ps144ii-VII.tex} \Abardot{}

\vspace{-1cm}

\vfill
\pagebreak

\pars{Psalmus 2.} \scriptura{Ps. 145, 2; \textbf{H100}}

\vspace{-4mm}

\antiphona{IV E}{temporalia/ant-laudabodeum.gtex}

\scriptura{Psalmus 145.}

\initiumpsalmi{temporalia/ps145-initium-iv-E-auto.gtex}

%\psalmusEtTranslatioT{temporalia/ps145-VII-comb.tex}{10cm}
\input{temporalia/ps145-VII.tex} \Abardot{}

\vfill
\pagebreak

\pars{Psalmus 3.} \scriptura{Ps. 146, 1; \textbf{H101}}

\vspace{-4mm}

\antiphona{VIII a}{temporalia/ant-deonostro.gtex}

\scriptura{Psalmus 146.}

\initiumpsalmi{temporalia/ps146-initium-viii-A-auto.gtex}

%\psalmusEtTranslatioT{temporalia/ps146-VII-comb.tex}{10cm}
\input{temporalia/ps146-VII.tex} \Abardot{}

\vfill
\pagebreak

\pars{Psalmus 4.} \scriptura{Ps. 147, 1}

\vspace{-4mm}

\antiphona{E}{temporalia/ant-laudajerusalem.gtex}

\scriptura{Psalmus 147.}

\initiumpsalmi{temporalia/ps147-initium-e-auto.gtex}

%\psalmusEtTranslatioT{temporalia/ps147-VII-comb.tex}{10cm}
\input{temporalia/ps147-VII.tex} \Abardot{}

\vfill
\pagebreak

\pars{Capitulum.} \scriptura{Rom. 11, 33}

\grechangedim{interwordspacetext}{0.12 cm plus 0.15 cm minus 0.05 cm}{scalable}%
\cuminitiali{}{temporalia/capitulum-OAltitudo.gtex}
\grechangedim{interwordspacetext}{0.22 cm plus 0.15 cm minus 0.05 cm}{scalable}

% preklad Jeruz. bible
%\trCapituliI

\vfill

\pars{Responsorium breve.} \scriptura{Ps. 146, 5}

\cuminitiali{VI}{temporalia/resp-magnusdominusnoster.gtex}

%\trResp

\vfill
\pagebreak

\pars{Hymnus} \scriptura{Ambrosius (\olddag{} 397)}

\cuminitiali{I}{temporalia/hym-OLuxBeata-aestivalis.gtex}
\vspace{-3mm}
%\input{hym-OLuxBeata-bohtext.tex}

\vfill
%\pagebreak

\pars{Versus.}

% Versus. %%%
\sineinitiali{temporalia/versus-vespertina.gtex}

%\noindent \trVersus

\vfill
\pagebreak

\magnificati

\vfill
\pagebreak

%\sideThumbs{{\scriptsize{}Fine horarum}}

\anteOrationem

\pagebreak

% Oratio. %%%
\oratioLaudes

\vspace{-1mm}
%\trOrationisI

\vfill

\rubrica{Hebdomadarius dicit iterum Dominus vobiscum, vel cantor dicit:}

\vspace{2mm}

\sineinitiali{temporalia/domineexaudi.gtex}

\rubrica{Postea cantatur a cantore:}

\vspace{2mm}

\cuminitiali{I}{temporalia/benedicamus-dominica-perannum.gtex}

\vspace{1mm}

\vfill
\pagebreak

\hora{Ad Matutinum.} %%%%%%%%%%%%%%%%%%%%%%%%%%%%%%%%%%%%%%%%%%%%%%%%%%%%%
%\sideThumbs{Matutinum}

\vspace{2mm}

\cuminitiali{}{temporalia/dominelabiamea.gtex}

\vspace{2mm}

\pars{Invitatorium.} \scriptura{Ps. 94, 1; Psalmus 94}

\vspace{-6mm}

\antiphona{E}{temporalia/inv-veniteexsultemus.gtex}

\vfill
\pagebreak

\pars{Hymnus.} \scriptura{Adamus Sancti Victoris (\olddag 1146)}

\vspace{-5mm}

\antiphona{VII}{temporalia/hym-SalveDies.gtex}

\scriptura{Non dicitur \textnormal{Amen} in fine.}
%{
%\vspace{-5mm}
%\setlength{\columnsep}{0pt} % prostor mezi sloupci
%\input{hym-SalveDies-bohtext.tex}
%\setlength{\columnsep}{30pt} % prostor mezi sloupci
%}

\vfill
\pagebreak

\subhora{In I. Nocturno}

\pars{Psalmus 1.} \scriptura{Ps. 1, 1}

\vspace{-4mm}

\antiphona{VIII G}{temporalia/ant-beatusvir.gtex}

%\vspace{-5mm}

\scriptura{Ps. 1}

%\vspace{-2mm}

\initiumpsalmi{temporalia/ps1-initium-viii-G-auto.gtex}

%\psalmusEtTranslatioT{temporalia/ps1-I-comb.tex}{10cm}
\input{temporalia/ps1-I.tex} \Abardot{}

\vfill
\pagebreak

\pars{Psalmus 2.} \scriptura{Ps. 2, 11; \textbf{H93}}

\vspace{-4mm}

\antiphona{VII a}{temporalia/ant-servitedomino.gtex}

\vspace{-3mm}

\scriptura{Ps. 2}

\vspace{-2mm}

\initiumpsalmi{temporalia/ps2-initium-vii-a-auto.gtex}

%\psalmusEtTranslatioT{temporalia/ps2-I-comb.tex}{10cm}
\input{temporalia/ps2-I.tex} \Abardot{}

\vfill
\pagebreak

\pars{Psalmus 3.} \scriptura{Ps. 3, 7}

\vspace{-4mm}

\antiphona{VI F}{temporalia/ant-exsurgedominesalvum.gtex}

%\vspace{-5mm}

\scriptura{Ps. 3}

\initiumpsalmi{temporalia/ps3-initium-vi-F-auto.gtex}

%\psalmusEtTranslatioT{temporalia/ps3-I-comb.tex}{10cm}
\input{temporalia/ps3-I.tex} \Abardot{}

\vfill
\pagebreak

\pars{Versus.} \scriptura{Ps. 118, 55}

% Versus. %%%
\sineinitiali{temporalia/versus-memorfui.gtex}

\vspace{5mm}

\sineinitiali{temporalia/oratiodominica-mat.gtex}

\vspace{5mm}

\pars{Absolutio.}

\cuminitiali{}{temporalia/absolutio-exaudi.gtex}

\vfill
\pagebreak

\cuminitiali{}{temporalia/benedictio-solemn-benedictione.gtex}

\vspace{7mm}

\lectioi

\noindent \Vbardot{} Tu autem, Dómine, miserére nobis.
\noindent \Rbardot{} Deo grátias.

\vfill
\pagebreak

\responsoriumi

\vfill
\pagebreak

\cuminitiali{}{temporalia/benedictio-solemn-unigenitus.gtex}

\vspace{7mm}

\lectioii

\noindent \Vbardot{} Tu autem, Dómine, miserére nobis.
\noindent \Rbardot{} Deo grátias.

\vfill
\pagebreak

\responsoriumii

\vfill
\pagebreak

\cuminitiali{}{temporalia/benedictio-solemn-spiritus.gtex}

\vspace{7mm}

\lectioiii

\noindent \Vbardot{} Tu autem, Dómine, miserére nobis.
\noindent \Rbardot{} Deo grátias.

\vfill
\pagebreak

\responsoriumiii

\vfill
\pagebreak

\subhora{In II. Nocturno}

\pars{Psalmus 4.} \scriptura{Ps. 8, 2}

\vspace{-4mm}

\antiphona{I g}{temporalia/ant-quamadmirabileest.gtex}

%\vspace{-5mm}

\scriptura{Ps. 8}

%A\vspace{-2mm}

\initiumpsalmi{temporalia/ps8-initium-i-g-auto.gtex}

%\psalmusEtTranslatioT{temporalia/ps8-I-comb.tex}{10cm}
\input{temporalia/ps8-I.tex} \Abardot{}

\vfill
\pagebreak

\pars{Psalmus 5.} \scriptura{Ps. 9, 5}

\vspace{-4mm}

\antiphona{VIII G}{temporalia/ant-sedistisuperthronum.gtex}

%\vspace{-5mm}

\scriptura{Ps. 9, 2-11}

\initiumpsalmi{temporalia/ps9ii_xi-initium-viii-G-auto.gtex}

%\psalmusEtTranslatioT{temporalia/ps9ii_xi-I-comb.tex}{10cm}
\input{temporalia/ps9ii_xi-I.tex} \Abardot{}

\vfill
\pagebreak

\pars{Psalmus 6.} \scriptura{Ps. 9, 20}

\vspace{-4mm}

\antiphona{I g\textsuperscript{3}}{temporalia/ant-exsurgedominenon.gtex}

%\vspace{-5mm}

\scriptura{Ps. 9, 12-21}

\initiumpsalmi{temporalia/ps9xii_xxi-initium-i-g3-auto.gtex}

%\psalmusEtTranslatioT{temporalia/ps9xii_xxi-I-comb.tex}{10cm}
\input{temporalia/ps9xii_xxi-I.tex} \Abardot{}

\vfill
\pagebreak

\pars{Versus.} \scriptura{Ps. 118, 62}

% Versus. %%%
\sineinitiali{temporalia/versus-medianocte.gtex}

\vspace{5mm}

\sineinitiali{temporalia/oratiodominica-mat.gtex}

\vspace{5mm}

\pars{Absolutio.}

\cuminitiali{}{temporalia/absolutio-ipsius.gtex}

\vfill
\pagebreak

\cuminitiali{}{temporalia/benedictio-solemn-deus.gtex}

\vspace{7mm}

\lectioiv

\noindent \Vbardot{} Tu autem, Dómine, miserére nobis.
\noindent \Rbardot{} Deo grátias.

\vfill
\pagebreak

\responsoriumiv

\vfill
\pagebreak

\cuminitiali{}{temporalia/benedictio-solemn-christus.gtex}

\vspace{7mm}

\lectiov

\noindent \Vbardot{} Tu autem, Dómine, miserére nobis.
\noindent \Rbardot{} Deo grátias.

\vfill
\pagebreak

\responsoriumv

\vfill
\pagebreak

\cuminitiali{}{temporalia/benedictio-solemn-ignem.gtex}

\vspace{7mm}

\lectiovi

\noindent \Vbardot{} Tu autem, Dómine, miserére nobis.
\noindent \Rbardot{} Deo grátias.

\vfill
\pagebreak

\responsoriumvi

\vfill
\pagebreak

\subhora{In III. Nocturno}

\pars{Psalmus 7.} \scriptura{Ps. 9, 22}

\vspace{-4mm}

\antiphona{II D}{temporalia/ant-utquiddomine.gtex}

\vspace{-4mm}

\scriptura{Ps. 9, 22-32}

%\vspace{-2mm}

\initiumpsalmi{temporalia/ps9xxii_xxxii-initium-ii-D-auto.gtex}

%\psalmusEtTranslatioT{temporalia/ps9xxii_xxxii-I-comb.tex}{10cm}
\input{temporalia/ps9xxii_xxxii-I.tex} \Abardot{}

\vfill
\pagebreak

\pars{Psalmus 8.}\scriptura{Ex. 15, 18}

\vspace{-4mm}

\antiphona{IV* e}{temporalia/ant-inaeternum.gtex}

%\vspace{-4mm}

\scriptura{Ps. 9, 33-39}

\initiumpsalmi{temporalia/ps9xxxiii_xxxix-initium-iv_-e-auto.gtex}

%\psalmusEtTranslatioT{temporalia/ps9xxxiii_xxxix-I-comb.tex}{10cm}
\input{temporalia/ps9xxxiii_xxxix-I.tex} \Abardot{}

\vfill
\pagebreak

\pars{Psalmus 9.} \scriptura{Ps. 10, 8}

\vspace{-4mm}

\antiphona{II* f}{temporalia/ant-justusdominus.gtex}

%\vspace{-4mm}

\scriptura{Ps. 10}

%\initiumpsalmi{temporalia/ps10-initium-iv-c-auto.gtex}
\initiumpsalmi{temporalia/ps10-initium-ii_-f.gtex}

%\psalmusEtTranslatioT{temporalia/ps10-I-comb.tex}{10cm}
\input{temporalia/ps10-I.tex} \Abardot{}

\vfill
\pagebreak

\pars{Versus.} \scriptura{Ps. 118, 148}

% Versus. %%%
\sineinitiali{temporalia/versus-praevenerunt.gtex}

\vspace{5mm}

\sineinitiali{temporalia/oratiodominica-mat.gtex}

\vspace{5mm}

\pars{Absolutio.}

\cuminitiali{}{temporalia/absolutio-avinculis.gtex}

\vfill
\pagebreak

\cuminitiali{}{temporalia/benedictio-solemn-evangelica.gtex}

\vspace{7mm}

\lectiovii

\noindent \Vbardot{} Tu autem, Dómine, miserére nobis.
\noindent \Rbardot{} Deo grátias.

\vfill
\pagebreak

\responsoriumvii

\vfill
\pagebreak

\cuminitiali{}{temporalia/benedictio-solemn-divinum.gtex}

\vspace{7mm}

\lectioviii

\noindent \Vbardot{} Tu autem, Dómine, miserére nobis.
\noindent \Rbardot{} Deo grátias.

\vfill
\pagebreak

\responsoriumviii

\vfill
\pagebreak

\cuminitiali{}{temporalia/benedictio-solemn-adsocietatem.gtex}

\vspace{7mm}

\lectioix

\noindent \Vbardot{} Tu autem, Dómine, miserére nobis.
\noindent \Rbardot{} Deo grátias.

\vfill
\pagebreak

% Te Deum

{
\pars{Hymnus Ambrosianus} \scriptura{Tonus Solemnis}

\vspace{-2mm}

\grechangedim{interwordspacetext}{0.26 cm plus 0.15 cm minus 0.05 cm}{scalable}%
\cuminitiali{III}{temporalia/tedeum-solemnis-gn.gtex}
\grechangedim{interwordspacetext}{0.22 cm plus 0.15 cm minus 0.05 cm}{scalable}%
}

\vfill
\pagebreak

\rubrica{Reliqua omittuntur, nisi Laudes separandæ sint.}

\pars{Oratio}

\noindent \Vbardot{} Dómine, exáudi oratiónem meam.

\noindent \Rbardot{} Et clamor meus ad te véniat.

Orémus:

\oratioLaudes

\vspace{7mm}

\pars{Conclusio}

\noindent \Vbardot{} Dómine, exáudi oratiónem meam.

\noindent \Rbardot{} Et clamor meus ad te véniat.

\noindent \Vbardot{} Benedicámus Dómino, allelúia, allelúia.

\noindent \Rbardot{} Deo grátias, allelúia, allelúia.

\noindent \Vbardot{} Fidélium ánimæ per misericórdiam Dei requiéscant in pace.

\noindent \Rbardot{} Amen.

\vfill
\pagebreak

\hora{Ad Laudes.} %%%%%%%%%%%%%%%%%%%%%%%%%%%%%%%%%%%%%%%%%%%%%%%%%%%%%
%\sideThumbs{Laudes}

\cantusSineNeumas

\vspace{0.5cm}
\grechangedim{interwordspacetext}{0.18 cm plus 0.15 cm minus 0.05 cm}{scalable}%
\cuminitiali{}{temporalia/deusinadiutorium-alter.gtex}
\grechangedim{interwordspacetext}{0.22 cm plus 0.15 cm minus 0.05 cm}{scalable}%

\vfill
%\pagebreak

\pars{Psalmus 1.}

\vspace{-4mm}

\antiphona{VI F}{temporalia/ant-alleluia1.gtex}

\scriptura{Psalmus 50.}

\initiumpsalmi{temporalia/ps50-initium-vi-F-auto.gtex}

%\psalmusEtTranslatioT{temporalia/ps50-I-comb.tex}{10cm}
\input{temporalia/ps50-I.tex}

\vfill
\pagebreak

\pars{Psalmus 2.}

\scriptura{Psalmus 117.}

\initiumpsalmi{temporalia/ps117-initium-vi-F-auto.gtex}

%\psalmusEtTranslatioT{temporalia/ps117-I-comb.tex}{10cm}
\input{temporalia/ps117-I.tex}

\vfill
\pagebreak

\pars{Psalmus 3.}

\scriptura{Psalmus 62.}

\initiumpsalmi{temporalia/ps62-initium-vi-F-auto.gtex}

%\psalmusEtTranslatioT{temporalia/ps62-I-comb.tex}{10cm}
\input{temporalia/ps62-I.tex}

\vfill

\vspace{-6mm}

\antiphona{}{temporalia/ant-alleluia1.gtex} % repeat the antiphon - new page

\vfill
\pagebreak

\pars{Psalmus 4.} \scriptura{Dan. 3, 22-26; \textbf{H422}}

\vspace{-4mm}

\antiphona{VIII G}{temporalia/ant-trespueri.gtex}

\scriptura{Canticum trium puerorum, Dan. 3, 57-88 et 56}

\initiumpsalmi{temporalia/dan3-initium-viii-G-auto.gtex}

%\psalmusEtTranslatioT{temporalia/dan3-comb.tex}{10cm}
\input{temporalia/dan3.tex}

\rubrica{Hic non dicitur Gloria Patri, neque Amen.}

\vfill

\vspace{-6mm}

\antiphona{}{temporalia/ant-trespueri.gtex} % repeat the antiphon - new page

\vfill
\pagebreak

\pars{Psalmus 5.}

\vspace{-4mm}

\antiphona{VIII G}{temporalia/ant-alleluia2.gtex}

\scriptura{Psalmus 148.}

\initiumpsalmi{temporalia/ps148-initium-viii-G-auto.gtex}

%\psalmusEtTranslatioT{temporalia/ps148-I-comb.tex}{10cm}
\input{temporalia/ps148-I.tex}

\rubrica{Hic non dicitur Gloria Patri.}

\vfill
\pagebreak

%
\scriptura{Psalmus 149.}

\initiumpsalmi{temporalia/ps149-initium-viii-G-auto.gtex}

%\psalmusEtTranslatioT{temporalia/ps149-I-comb.tex}{10cm}
\input{temporalia/ps149-I.tex}

\rubrica{Hic non dicitur Gloria Patri.}

\vfill
\pagebreak

%
\scriptura{Psalmus 150.}

\initiumpsalmi{temporalia/ps150-initium-viii-G-auto.gtex}

%\psalmusEtTranslatioT{temporalia/ps150-I-comb.tex}{10cm}
\input{temporalia/ps150-I.tex}

\vfill

\vspace{-6mm}

\antiphona{}{temporalia/ant-alleluia2.gtex} % repeat the antiphon - new page

\vfill
\pagebreak

\pars{Capitulum.} \scriptura{Ac. 7, 12}

\grechangedim{interwordspacetext}{0.12 cm plus 0.15 cm minus 0.05 cm}{scalable}%
\cuminitiali{}{temporalia/capitulum-Benedictio.gtex}
\grechangedim{interwordspacetext}{0.22 cm plus 0.15 cm minus 0.05 cm}{scalable}

% preklad Jeruz. bible
%\trCapituliI

\vfill

\pars{Responsorium breve.} \scriptura{Ps. 118, 36-37}

\cuminitiali{IV}{temporalia/resp-inclinacormeum.gtex}

%\trResp

\vfill
\pagebreak

\pars{Hymnus} \scriptura{Gregorius Magnus (\olddag{} 604)}

\cuminitiali{IV}{temporalia/hym-EcceJamNoctis.gtex}
\vspace{-3mm}
%\input{hym-EcceJamNocis-bohtext.tex}

\vfill
%\pagebreak

\pars{Versus.} \scriptura{Ps. 92, 1}

% Versus. %%%
\sineinitiali{temporalia/versus-dominusregnavit.gtex}

%\noindent \trVersus

\vfill
\pagebreak

\benedictus

\vspace{-1cm}

\vfill
\pagebreak

%\sideThumbs{{\scriptsize{}Fine horarum}}

\anteOrationem

\pagebreak

% Oratio. %%%
\oratioLaudes

\vspace{-1mm}
%\trOrationisI

\vfill

\rubrica{Hebdomadarius dicit iterum Dominus vobiscum, vel cantor dicit:}

\vspace{2mm}

\sineinitiali{temporalia/domineexaudi.gtex}

\rubrica{Postea cantatur a cantore:}

\vspace{2mm}

\cuminitiali{I}{temporalia/benedicamus-dominica-perannum.gtex}

\vspace{1mm}

\vfill
\pagebreak

\hora{Ad II. Vesperas.} %%%%%%%%%%%%%%%%%%%%%%%%%%%%%%%%%%%%%%%%%%%%%%%%%%%%%
%\sideThumbs{II. Vesperæ}

\cantusSineNeumas

%\vspace{0.5cm}
\grechangedim{interwordspacetext}{0.18 cm plus 0.15 cm minus 0.05 cm}{scalable}%
\cuminitiali{}{temporalia/deusinadiutorium-solemnis.gtex}
\grechangedim{interwordspacetext}{0.22 cm plus 0.15 cm minus 0.05 cm}{scalable}%

\vfill
%\pagebreak

\vspace{-2mm}

\pars{Psalmus 1.} \scriptura{Ps. 109, 1; \textbf{H91}}

\vspace{-4mm}

\antiphona{VII c\textsuperscript{2}}{temporalia/ant-dixitdominus.gtex}

\vspace{-4mm}

\scriptura{Psalmus 109.}

\initiumpsalmi{temporalia/ps109-initium-vii-c2-auto.gtex}

%\psalmusEtTranslatioT{temporalia/ps109-I-comb.tex}{10cm}
\input{temporalia/ps109-I.tex} \Abardot{}

\vspace{-1cm}

\vfill
\pagebreak

\pars{Psalmus 2.} \scriptura{Ps. 110, 8; \textbf{H91}}

\vspace{-4mm}

\antiphona{IV g}{temporalia/ant-fideliaomnia.gtex}

\scriptura{Psalmus 110.}

\initiumpsalmi{temporalia/ps110-initium-iv-g-auto.gtex}

%\psalmusEtTranslatioT{temporalia/ps110-I-comb.tex}{10cm}
\input{temporalia/ps110-I.tex} \Abardot{}

\vfill
\pagebreak

\pars{Psalmus 3.} \scriptura{Ps. 111, 1; \textbf{H92}}

\vspace{-4mm}

\antiphona{IV a}{temporalia/ant-inmandatis.gtex}

\scriptura{Psalmus 111.}

\initiumpsalmi{temporalia/ps111-initium-iv-a-auto.gtex}

%\psalmusEtTranslatioT{temporalia/ps111-I-comb.tex}{10cm}
\input{temporalia/ps111-I.tex} \Abardot{}

\vfill
\pagebreak

\pars{Psalmus 4.} \scriptura{Ps. 112, 2; \textbf{H92}}

\vspace{-4mm}

\antiphona{VII c}{temporalia/ant-sitnomendomini.gtex}

\scriptura{Psalmus 112.}

\initiumpsalmi{temporalia/ps112-initium-vii-c-auto.gtex}

%\psalmusEtTranslatioT{temporalia/ps112-I-comb.tex}{10cm}
\input{temporalia/ps112-I.tex} \Abardot{}

\vfill
\pagebreak

\pars{Capitulum.} \scriptura{2 Cor. 1, 3-4}

\grechangedim{interwordspacetext}{0.12 cm plus 0.15 cm minus 0.05 cm}{scalable}%
\cuminitiali{}{temporalia/capitulum-BenedictusDeus.gtex}
\grechangedim{interwordspacetext}{0.22 cm plus 0.15 cm minus 0.05 cm}{scalable}

% preklad Jeruz. bible
%\trCapituliI

\vfill

\pars{Responsorium breve.} \scriptura{Ps. 103, 24}

\cuminitiali{VI}{temporalia/resp-quammagnificata.gtex}

%\trResp

\vfill
\pagebreak

\pars{Hymnus} \scriptura{Gregorius Magnus (\olddag{} 604)}

\cuminitiali{I}{temporalia/hym-LucisCreator-aestivalis.gtex}
\vspace{-3mm}
%\begin{translatioMulticol}{3}
Tvůrce světa předobrý,\\
tys ustanovil denní řád\\
a proudy světla rozhodil,\\
když světu základy jsi klad.\\
\\
A spojils ráno s večerem\\
a dnem tu dobu nazýváš;\\
hle padá temné noci stín -\\
slyš prosbu, vyslyš nářek náš.\columnbreak

Ach, nedej, by nás stihla smrt,\\
když svědomí nám tíží hřích,\\
když nemyslíme na věčnost\\
v té síti hříchů šalebných.\\
\\
Vzbuď naši touhu po nebi,\\
kde věčný život čeká nás,\\
a pomoz odložit vše zlé\\
a smýti z duše každý kaz.\columnbreak

To splň nám, dobrý Otče náš,\\
i ty, jenž rovné božství máš,\\
i Duchu, který těšíš nás\\
a vládneš, Bože, v každý čas.\\
Amen. 
\end{translatioMulticol}


\vfill
%\pagebreak

\pars{Versus.} \scriptura{Ps. 140, 2}

% Versus. %%%
\sineinitiali{temporalia/versus-dirigatur.gtex}

%\noindent \trVersus

\vfill
\pagebreak

\magnificatii

\vfill
\pagebreak

%\sideThumbs{{\scriptsize{}Fine horarum}}

\anteOrationem

\pagebreak

% Oratio. %%%
\oratioLaudes

\vspace{-1mm}
%\trOrationisI

\vfill

\rubrica{Hebdomadarius dicit iterum Dominus vobiscum, vel cantor dicit:}

\vspace{2mm}

\sineinitiali{temporalia/domineexaudi.gtex}

\rubrica{Postea cantatur a cantore:}

\vspace{2mm}

\cuminitiali{I}{temporalia/benedicamus-dominica-perannum.gtex}

\vspace{1mm}

\end{document}

