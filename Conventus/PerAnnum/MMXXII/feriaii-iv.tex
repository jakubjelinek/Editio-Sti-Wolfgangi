\newcommand{\titulus}{\nomenFesti{S. Ioannis Bosco, Presbyteri.}
\dies{Die 31. Ianuarii.}}
\newcommand{\oratio}{\pars{Oratio.}

\noindent Deus, qui beátum Ioánnem, presbýterum, adulescéntium patrem et magístrum excitásti, concéde, quǽsumus, ut, eódem caritátis igne succénsi, ánimas quǽrere tibíque soli servíre valeámus.

\noindent Per Dóminum nostrum Iesum Christum, Fílium tuum, qui tecum vivit et regnat in unitáte Spíritus Sancti, Deus, per ómnia sǽcula sæculórum.

\noindent \Rbardot{} Amen.}
\newcommand{\invitatorium}{\pars{Invitatorium.}

\vspace{-4mm}

\antiphona{IV}{temporalia/inv-mirabileminsanctis.gtex}}
\newcommand{\hymnusmatutinum}{\pars{Hymnus}

\cuminitiali{VIII}{temporalia/hym-IesuRedemptor.gtex}}
\newcommand{\matversus}{\noindent \Vbardot{} Quam dúlcia fáucibus meis elóquia tua, Dómine.

\noindent \Rbardot{} Super mel ori meo.}
\newcommand{\lectioi}{\pars{Lectio I.} \scriptura{1 Th. 2, 13-19}

\noindent De Epístola prima beáti Pauli apóstoli ad Thessalonicénses.

\noindent Fratres: Grátias ágimus Deo sine intermissióne, quóniam cum accepissétis a nobis verbum audítus Dei, accepístis non ut verbum hóminum sed, sicut est vere, verbum Dei, quod et operátur in vobis, qui créditis. Vos enim imitatóres facti estis, fratres, ecclesiárum Dei, quæ sunt in Iudǽa in Christo Iesu, qui éadem passi estis et vos a contribúlibus vestris, sicut et ipsi a Iudǽis, qui et Dóminum occidérunt Iesum et prophétas et nos persecúti sunt et Deo non placent et ómnibus homínibus adversántur, prohibéntes nos géntibus loqui, ut salvæ fiant, ut ímpleant peccáta sua semper. Pervénit autem ira Dei super illos usque in finem.

\noindent Nos autem, fratres, desoláti a vobis ad tempus horæ, fácie non corde, abundántius festinávimus fáciem vestram vidére cum multo desidério. Propter quod volúimus veníre ad vos, ego quidem Paulus et semel et íterum, et impedívit nos Sátanas. Quæ est enim nostra spes aut gáudium aut coróna glóriæ —nonne et vos— ante Dóminum nostrum Iesum in advéntu eius? Vos enim estis glória nostra et gáudium.}
\newcommand{\responsoriumi}{\pars{Responsorium 1.} \scriptura{\Rbardot{} Ps. 36, 3 \Vbardot{} ibid., 5; \textbf{H86}}

\vspace{-5mm}

\responsorium{V}{temporalia/resp-delectareindomino-CROCHU.gtex}{}}
\newcommand{\lectioii}{\pars{Lectio II.} \scriptura{1 Th. 3, 1-12}

\noindent Propter quod non sustinéntes ámplius, plácuit nobis, ut relinquerémur Athénis soli, et mísimus Timótheum, fratrem nostrum et cooperatórem Dei in evangélio Christi, ad confirmándos vos et exhortándos pro fide vestra, ut nemo turbétur in tribulatiónibus istis. Ipsi enim scitis quod in hoc pósiti sumus; nam et cum apud vos essémus, prædicebámus vobis passúros nos tribulatiónes, sicut et factum est et scitis. Proptérea et ego ámplius non sústinens, misi ad cognoscéndam fidem vestram, ne forte tentáverit vos is qui tentat, et inánis fiat labor noster.

\noindent Nunc autem veniénte Timótheo ad nos a vobis et annuntiánte nobis fidem et caritátem vestram et quia memóriam nostri habétis bonam semper, desiderántes nos vidére, sicut nos quoque vos, ídeo consoláti sumus, fratres, propter vos in omni necessitáte et tribulatióne nostra per vestram fidem, quóniam nunc vívimus, si vos statis in Dómino. Quam enim gratiárum actiónem póssumus Deo retribúere pro vobis in omni gáudio, quo gaudémus propter vos ante Deum nostrum, nocte et die abundántius orántes ut videámus fáciem vestram et compleámus ea, quæ desunt fídei vestræ?

\noindent Ipse autem Deus et Pater noster et Dóminus noster Iesus dírigat viam nostram ad vos; vos autem Dóminus abundáre et superabundáre fáciat caritáte in ínvicem et in omnes, quemádmodum et nos in vos, ad confirmánda corda vestra sine queréla in sanctitáte ante Deum et Patrem nostrum, in advéntu Dómini nostri Iesu cum ómnibus sanctis eius. Amen.}
\newcommand{\responsoriumii}{\pars{Responsorium 2.} \scriptura{\Rbardot{} Ps. 38, 13 \Vbardot{} ibid., 2; \textbf{H86}}

\vspace{-5mm}

\responsorium{II}{temporalia/resp-auribuspercipedomine-CROCHU.gtex}{}}
\newcommand{\lectioiii}{\pars{Lectio III.} \scriptura{Epistolario, Torino 1959, 4, 201-203}

\noindent Ex Epístolis sancti Ioánnis Bosco presbýteri.

\noindent {\color{gray}In primis, si vidéri cúpimus sollíciti veræ beatitúdinis nostrórum alumnórum quo facílius eos inducámus ad própria offícia adimplénda, opórtet ne umquam obliviscámini vos vices ágere paréntum iúvenum dilectórum, pro quibus semper amánter adlaborávi, stúdui et sacerdotália múnera exércui, et non ego solus, sed tota Salesiána socíetas.}

\noindent Quóties, filíoli mei, in meo non brevi currículo mihi persuadéndum fuit de huiúsmodi magna veritáte! Facílius est irásci quam sustinére, púero minári quam persuadére; dicam immo commódius esse nostræ impatiéntiæ et supérbiæ pœna affícere pervicáces pótius quam eos corrígere, fírmiter suavitérque tolerándo.

\noindent Caritátem tamen Pauli vobis comméndo, qua ipse se gerébat in neóphytos, quæ sæpe ad lácrimas eum adducébat et ad supplicatiónem, quando eos parum dóciles et suæ dilectióni obniténtes inveniébat.

\noindent Cavéte ne quis iúdicet vos veheménti ánimi ímpetu commovéri. Diffícile est in puniéndo illam ánimi constántiam serváre, quæ necessária est, ne dúbium exoriátur nos ad ostendéndam nostram auctoritátem ágere vel ad ánimi ímpetum effundéndum.

\noindent Ut fílios aspiciámus eos, in quos áliqua potéstas est nobis exercénda. Constituámus nos quasi in eórum famulátum, quemádmodum Iesus, qui ad obœdiéndum venit, non ad imperándum, pudeátque nos ipsíus speciéi dominándi; nec dominémur in ipsis, nisi ad mélius eísdem serviéndum.

\noindent Huiúsmodi erat agéndi rátio Iesu cum Apóstolis, qui eos, ignorántes et rudes, immo et parum fidéles sustinébat, et in peccatóres se ea cum benignitáte et familiári amicítia gerébat, ut álii stupórem, álii vero scándalum concíperent, aliíque dénique spem a Deo impetrándi véniam. Ideóque nobis mandávit ut essémus mites et húmiles corde.

\noindent Nostri sunt fílii, quaprópter cum eórum erróres compéscimus, omnem iram deponámus vel ádeo temperémus quasi omníno exstinxérimus.

\noindent Non in ánimo concitátio, non in óculis contémptio, non in ore contumélia, sed misericórdia in præsénti, spes futúri témporis, ut veros decet patres, qui veræ stúdeant correctióni et emendatióni.

\noindent In gravíssimis rerum adiúnctis præstat Deum supplíciter et humíliter exoráre, quam verbórum flumen emíttere, quæ, dum audiéntium ánimos offéndunt, nullam sóntibus áfferunt utilitátem.}
\newcommand{\responsoriumiii}{\pars{Responsorium 3.} \scriptura{\Vbardot{} Sap. 10, 10; \textbf{H379}}

\vspace{-5mm}

\responsorium{VIII}{temporalia/resp-istehomoperfecit-CROCHU-cumdox.gtex}{}}
\newcommand{\hymnuslaudes}{\pars{Hymnus}

\cuminitiali{VIII}{temporalia/hym-IesuCorona.gtex}}
\newcommand{\lectiobrevis}{\pars{Lectio Brevis.} \scriptura{Rom. 12, 1-2}

\noindent Obsecro vos, fratres, per misericórdiam Dei, ut exhibeátis córpora vestra hóstiam vivéntem, sanctam, Deo placéntem, rationábile obséquium vestrum; et nolíte conformári huic sǽculo, sed transformámini renovatióne mentis, ut probétis quid sit volúntas Dei, quid bonum et bene placens et perféctum.}
\newcommand{\responsoriumbreve}{\pars{Responsorium breve.} \scriptura{Ps. 36, 31}

\cuminitiali{VI}{temporalia/resp-lexdeiejus.gtex}}
\newcommand{\preces}{\noindent Christum Deum sanctum, fratres, exaltémus,~\gredagger{} orántes ut serviámus illi in sanctitáte et iustítia coram ipso ómnibus diébus nostris,~\grestar{} et acclamémus: 

\Rbardot{} Tu solus sanctus, Dómine.

\noindent Qui tentári voluísti per ómnia pro similitúdine nostra absque peccáto,~\grestar{} miserére nostri, Dómine Iesu.

\Rbardot{} Tu solus sanctus, Dómine.

\noindent Qui nos omnes ad perfectiónem caritátis vocásti,~\grestar{} sanctífica nos, Dómine Iesu.

\Rbardot{} Tu solus sanctus, Dómine.

\noindent Qui nos iussísti esse salem terræ et lucem mundi,~\grestar{} illúmina nos, Dómine Iesu.

\Rbardot{} Tu solus sanctus, Dómine.

\noindent Qui voluísti ministráre,~\gredagger{} non ministrári,~\grestar{} fac nos tibi et frátribus humíliter servíre, Dómine Iesu.

\Rbardot{} Tu solus sanctus, Dómine.

\noindent Tu, splendor glóriæ Patris et figúra substántiæ eius,~\grestar{} fac ut in glória vultum tuum respiciámus, Dómine Iesu.

\Rbardot{} Tu solus sanctus, Dómine.}
\newcommand{\benedictus}{\pars{Canticum Zachariæ.} \scriptura{Prv. 23, 26; \textbf{H402}}

%\vspace{-4mm}

{
\grechangedim{interwordspacetext}{0.18 cm plus 0.15 cm minus 0.05 cm}{scalable}%
\antiphona{IV e*}{temporalia/ant-praebefilicortuum.gtex}
\grechangedim{interwordspacetext}{0.22 cm plus 0.15 cm minus 0.05 cm}{scalable}%
}

%\vspace{-3mm}

\scriptura{Lc. 1, 68-79}

%\vspace{-2mm}

\cantusSineNeumas
\initiumpsalmi{temporalia/benedictus-initium-iv-e-auto.gtex}

%\vspace{-1.5mm}

\input{temporalia/benedictus-iv-e.tex} \Abardot{}}
\newcommand{\benedicamuslaudes}{\cuminitiali{}{temporalia/benedicamus-memoria-laudes.gtex}}
\newcommand{\hebdomada}{infra Hebdom. IV post Pentecosten.}
\newcommand{\oratioLaudes}{\cuminitiali{}{temporalia/oratio4.gtex}}

% LuaLaTeX

\documentclass[a4paper, twoside, 12pt]{article}
\usepackage[latin]{babel}
%\usepackage[landscape, left=3cm, right=1.5cm, top=2cm, bottom=1cm]{geometry} % okraje stranky
%\usepackage[landscape, a4paper, mag=1166, truedimen, left=2cm, right=1.5cm, top=1.6cm, bottom=0.95cm]{geometry} % okraje stranky
\usepackage[landscape, a4paper, mag=1400, truedimen, left=0.5cm, right=0.5cm, top=0.5cm, bottom=0.5cm]{geometry} % okraje stranky

\usepackage{fontspec}
\setmainfont[FeatureFile={junicode.fea}, Ligatures={Common, TeX}, RawFeature=+fixi]{Junicode}
%\setmainfont{Junicode}

% shortcut for Junicode without ligatures (for the Czech texts)
\newfontfamily\nlfont[FeatureFile={junicode.fea}, Ligatures={Common, TeX}, RawFeature=+fixi]{Junicode}

\usepackage{multicol}
\usepackage{color}
\usepackage{lettrine}
\usepackage{fancyhdr}

% usual packages loading:
\usepackage{luatextra}
\usepackage{graphicx} % support the \includegraphics command and options
\usepackage{gregoriotex} % for gregorio score inclusion
\usepackage{gregoriosyms}
\usepackage{wrapfig} % figures wrapped by the text
\usepackage{parcolumns}
\usepackage[contents={},opacity=1,scale=1,color=black]{background}
\usepackage{tikzpagenodes}
\usepackage{calc}
\usepackage{longtable}
\usetikzlibrary{calc}

\setlength{\headheight}{14.5pt}

\input{conventuscommune.tex} % Often used macros

\newcommand{\annusEditionis}{2021}

%%%% Vicekrat opakovane kousky

\newcommand{\anteOrationem}{
  \rubrica{Ante Orationem, cantatur a Superiore:}

  \pars{Supplicatio Litaniæ.}

  \cuminitiali{}{temporalia/supplicatiolitaniae.gtex}

  \pars{Oratio Dominica.}

  \cuminitiali{}{temporalia/oratiodominica.gtex}

  \rubrica{Deinde dicitur ab Hebdomadario:}

  \cuminitiali{}{temporalia/dominusvobiscum-solemnis.gtex}

  \rubrica{In choro monialium loco Dominus vobiscum dicitur:}

  \sineinitiali{temporalia/domineexaudi.gtex}
}

\setlength{\columnsep}{30pt} % prostor mezi sloupci

%%%%%%%%%%%%%%%%%%%%%%%%%%%%%%%%%%%%%%%%%%%%%%%%%%%%%%%%%%%%%%%%%%%%%%%%%%%%%%%%%%%%%%%%%%%%%%%%%%%%%%%%%%%%%
\begin{document}

% Here we set the space around the initial.
% Please report to http://home.gna.org/gregorio/gregoriotex/details for more details and options
\grechangedim{afterinitialshift}{2.2mm}{scalable}
\grechangedim{beforeinitialshift}{2.2mm}{scalable}
\grechangedim{interwordspacetext}{0.22 cm plus 0.15 cm minus 0.05 cm}{scalable}%
\grechangedim{annotationraise}{-0.2cm}{scalable}

% Here we set the initial font. Change 38 if you want a bigger initial.
% Emit the initials in red.
\grechangestyle{initial}{\color{red}\fontsize{38}{38}\selectfont}

\pagestyle{empty}

%%%% Titulni stranka
\begin{titulusOfficii}
\ifx\titulus\undefined
\nomenFesti{Feria II \hebdomada{}}
\else
\titulus
\fi
\end{titulusOfficii}

\vfill

\begin{center}
%Ad usum et secundum consuetudines chori \guillemotright{}Conventus Choralis\guillemotleft.

%Editio Sancti Wolfgangi \annusEditionis
\end{center}

\scriptura{}

\pars{}

\pagebreak

\renewcommand{\headrulewidth}{0pt} % no horiz. rule at the header
\fancyhf{}
\pagestyle{fancy}

\cantusSineNeumas

\ifx\oratio\undefined
\ifx\laudb\undefined
\else
\newcommand{\oratio}{\pars{Oratio.}

\noindent Dómine Deus omnípotens, qui ad princípium huius diéi nos perveníre fecísti, tua nos hódie salva virtúte, ut in hac die ad nullum declinémus peccátum, sed semper ad tuam iustítiam faciéndam nostra procédant elóquia, dirigántur cogitatiónes et ópera.

\noindent Per Dóminum nostrum Iesum Christum, Fílium tuum, qui tecum vivit et regnat in unitáte Spíritus Sancti, Deus, per ómnia sǽcula sæculórum.

\noindent \Rbardot{} Amen.}
\fi
\fi

\hora{Ad Matutinum.} %%%%%%%%%%%%%%%%%%%%%%%%%%%%%%%%%%%%%%%%%%%%%%%%%%%%%
%\sideThumbs{Matutinum}

\vspace{2mm}

\cuminitiali{}{temporalia/dominelabiamea.gtex}

\vfill
%\pagebreak

\vspace{2mm}

\ifx\invitatorium\undefined
\pars{Invitatorium.} \scriptura{Ps. 94, 1; Psalmus 94; \textbf{H451}}

\vspace{-6mm}

\antiphona{VI}{temporalia/inv-jubilemusdeo.gtex}\else
\invitatorium
\fi

\vfill
\pagebreak

\ifx\hymnusmatutinum\undefined
\ifx\matua\undefined
\else
\pars{Hymnus.}

{
\grechangedim{interwordspacetext}{0.10 cm plus 0.15 cm minus 0.05 cm}{scalable}%
\antiphona{II}{temporalia/hym-IpsumNunc.gtex}
\grechangedim{interwordspacetext}{0.22 cm plus 0.15 cm minus 0.05 cm}{scalable}%
}
\fi
\else
\hymnusmatutinum
\fi

\vspace{-3mm}

\vfill
\pagebreak

\ifx\matub\undefined
\else
% MAT B
\pars{Psalmus 1.} \scriptura{Ps. 30, 2; \textbf{H90}}

\vspace{-4mm}

\antiphona{VIII G}{temporalia/ant-intuaiustitia.gtex}

%\vspace{-2mm}

\scriptura{Ps. 30, 2-9}

%\vspace{-2mm}

\initiumpsalmi{temporalia/ps30i-initium-viii-G-auto.gtex}

\vspace{-1.5mm}

\input{temporalia/ps30i-viii-G.tex} \Abardot{}

\vfill
\pagebreak

\pars{Psalmus 2.} \scriptura{Ps. 66, 2}

\vspace{-4mm}

\antiphona{E}{temporalia/ant-illuminadomine.gtex}

%\vspace{-2mm}

\scriptura{Ps. 30, 10-17}

%\vspace{-2mm}

\initiumpsalmi{temporalia/ps30ii-initium-e-a-auto.gtex}

\input{temporalia/ps30ii-e-a.tex} \Abardot{}

\vfill
\pagebreak

\pars{Psalmus 3.} \scriptura{Ps. 30, 24}

\vspace{-4mm}

\antiphona{II D}{temporalia/ant-diligitedominum.gtex}

%\vspace{-5mm}

\scriptura{Ps. 30, 20-25}

%\vspace{-2mm}

\initiumpsalmi{temporalia/ps30iii-initium-ii-D-auto.gtex}

\input{temporalia/ps30iii-ii-D.tex} \Abardot{}

\vfill
\pagebreak
\fi

\pars{Versus.}

\ifx\matversus\undefined
\ifx\matub\undefined
\else
\noindent \Vbardot{} Dírige me, Dómine, in veritáte tua, et doce me.

\noindent \Rbardot{} Quia tu es Deus salútis meæ.
\fi
\else
\matversus
\fi

\vspace{5mm}

\sineinitiali{temporalia/oratiodominica-mat.gtex}

\vspace{5mm}

\pars{Absolutio.}

\cuminitiali{}{temporalia/absolutio-exaudi.gtex}

\vfill
\pagebreak

\cuminitiali{}{temporalia/benedictio-solemn-benedictione.gtex}

\vspace{7mm}

\lectioi

\noindent \Vbardot{} Tu autem, Dómine, miserére nobis.
\noindent \Rbardot{} Deo grátias.

\vfill
\pagebreak

\responsoriumi

\vfill
\pagebreak

\cuminitiali{}{temporalia/benedictio-solemn-unigenitus.gtex}

\vspace{7mm}

\lectioii

\noindent \Vbardot{} Tu autem, Dómine, miserére nobis.
\noindent \Rbardot{} Deo grátias.

\vfill
\pagebreak

\responsoriumii

\vfill
\pagebreak

\cuminitiali{}{temporalia/benedictio-solemn-spiritus.gtex}

\vspace{7mm}

\lectioiii

\noindent \Vbardot{} Tu autem, Dómine, miserére nobis.
\noindent \Rbardot{} Deo grátias.

\vfill
\pagebreak

\responsoriumiii

\vfill
\pagebreak

\rubrica{Reliqua omittuntur, nisi Laudes separandæ sint.}

\sineinitiali{temporalia/domineexaudi.gtex}

\vfill

\oratio

\vfill

\noindent \Vbardot{} Dómine, exáudi oratiónem meam.
\Rbardot{} Et clamor meus ad te véniat.

\vfill

\noindent \Vbardot{} Benedicámus Dómino.
\noindent \Rbardot{} Deo grátias.

\vfill

\noindent \Vbardot{} Fidélium ánimæ per misericórdiam Dei requiéscant in pace.
\Rbardot{} Amen.

\vfill
\pagebreak

\hora{Ad Laudes.} %%%%%%%%%%%%%%%%%%%%%%%%%%%%%%%%%%%%%%%%%%%%%%%%%%%%%
%\sideThumbs{Laudes}

\cantusSineNeumas

\vspace{0.5cm}
\grechangedim{interwordspacetext}{0.18 cm plus 0.15 cm minus 0.05 cm}{scalable}%
\cuminitiali{}{temporalia/deusinadiutorium-communis.gtex}
\grechangedim{interwordspacetext}{0.22 cm plus 0.15 cm minus 0.05 cm}{scalable}%

\vfill
\pagebreak

\ifx\hymnuslaudes\undefined
\ifx\laudbd\undefined
\else
\pars{Hymnus} \scriptura{Hilarius (\olddag{} 367)}

\grechangedim{interwordspacetext}{0.16 cm plus 0.15 cm minus 0.05 cm}{scalable}%
\cuminitiali{IV}{temporalia/hym-LucisLargitor.gtex}
\grechangedim{interwordspacetext}{0.22 cm plus 0.15 cm minus 0.05 cm}{scalable}%
\vspace{-3mm}
\fi
\else
\hymnuslaudes
\fi

\vfill
\pagebreak

\ifx\laudb\undefined
\else
\pars{Psalmus 1.} \scriptura{Ps. 41, 3; \textbf{H391}}

\vspace{-4mm}

\antiphona{II D}{temporalia/ant-sitivitanima.gtex}

%\vspace{-2mm}

\scriptura{Psalmus 41}

%\vspace{-2mm}

\initiumpsalmi{temporalia/ps41-initium-ii-D-auto.gtex}

%\vspace{-1.5mm}

\input{temporalia/ps41-ii-D.tex}

\vfill

\antiphona{}{temporalia/ant-sitivitanima.gtex}

\vfill
\pagebreak

\pars{Psalmus 2.}

\vspace{-4mm}

\antiphona{III a}{temporalia/ant-ostendenobisdomine.gtex}

%\vspace{-2mm}

\scriptura{Canticum Ecclesiastici, Sir. 36, 1-7.13-16}

%\vspace{-3mm}

\initiumpsalmi{temporalia/ecclesiastici-initium-iii-a-auto.gtex}

\input{temporalia/ecclesiastici-iii-a.tex} \Abardot{}

\vfill
\pagebreak

\pars{Psalmus 3.}

\vspace{-4mm}

\antiphona{II D}{temporalia/ant-operamanuumeius.gtex}

\scriptura{Psalmus 18, 1-7}

\initiumpsalmi{temporalia/ps18i-initium-ii-D-auto.gtex}

\input{temporalia/ps18i-ii-D.tex} \Abardot{}

\vfill
\pagebreak
\fi

\ifx\lectiobrevis\undefined
\ifx\laudb\undefined
\else
\pars{Lectio Brevis.} \scriptura{Ier. 15, 16}

\noindent Invénti sunt sermónes tui, et comédi eos, et factum est mihi verbum tuum in gáudium et in lætítiam cordis mei, quóniam invocátum est nomen tuum super me, Dómine Deus exercítuum.
\fi
\else
\lectiobrevis
\fi

\vfill

\ifx\responsoriumbreve\undefined
\ifx\laudbd\undefined
\else
\pars{Responsorium breve.} \scriptura{Ps. 32, 1.3}

\cuminitiali{VI}{temporalia/resp-exsultateiusti.gtex}
\fi
\else
\responsoriumbreve
\fi

\vfill
\pagebreak

\ifx\benedictus\undefined
\ifx\laudbd\undefined
\else
\pars{Canticum Zachariæ.} \scriptura{Lc. 1, 68; \textbf{H422}}

\vspace{-4mm}

{
\grechangedim{interwordspacetext}{0.18 cm plus 0.15 cm minus 0.05 cm}{scalable}%
\antiphona{IV E}{temporalia/ant-benedictusdominus.gtex}
\grechangedim{interwordspacetext}{0.22 cm plus 0.15 cm minus 0.05 cm}{scalable}%
}

%\vspace{-3mm}

\scriptura{Lc. 1, 68-79}

%\vspace{-2mm}

\cantusSineNeumas
\initiumpsalmi{temporalia/benedictus-initium-iv-E-auto.gtex}

%\vspace{-1.5mm}

\input{temporalia/benedictus-iv-E.tex} \Abardot{}
\fi
\else
\benedictus
\fi

\vspace{-1cm}

\vfill
\pagebreak

%\sideThumbs{{\scriptsize{}Fine horarum}}

\pars{Preces.}

\sineinitiali{}{temporalia/tonusprecum.gtex}

\ifx\preces\undefined
\ifx\laudb\undefined
\else
\noindent Salvátor noster fecit nos regnum et sacerdótium, ut hóstias Deo acceptábiles offerámus. \gredagger{} Grati ígitur eum invocémus:

\Rbardot{} Serva nos in tuo ministério, Dómine.

\noindent Christe, sacérdos ætérne, qui sanctum pópulo tuo sacerdótium concessísti, \gredagger{} concéde, ut spiritáles hóstias Deo acceptábiles iúgiter offerámus.

\Rbardot{} Serva nos in tuo ministério, Dómine.

\noindent Spíritus tui fructus nobis largíre propítius, \gredagger{} patiéntiam, benignitátem et mansuetúdinem.

\Rbardot{} Serva nos in tuo ministério, Dómine.

\noindent Da nobis te amáre, ut te, qui es cáritas, possideámus, \gredagger{} et bene ágere, ut per vitam étiam nostram te laudémus.

\Rbardot{} Serva nos in tuo ministério, Dómine.

\noindent Quæ frátribus nostris sunt utília, nos quǽrere concéde, \gredagger{} ut salútem facílius consequántur.

\Rbardot{} Serva nos in tuo ministério, Dómine.
\fi
\else
\preces
\fi

\vfill

\pars{Oratio Dominica.}

\cuminitiali{}{temporalia/oratiodominicaalt.gtex}

\vfill
\pagebreak

\rubrica{vel:}

\pars{Supplicatio Litaniæ.}

\cuminitiali{}{temporalia/supplicatiolitaniae.gtex}

\vfill

\pars{Oratio Dominica.}

\cuminitiali{}{temporalia/oratiodominica.gtex}

\vfill
\pagebreak

% Oratio. %%%
\oratio

\vspace{-1mm}

\vfill

\rubrica{Hebdomadarius dicit Dominus vobiscum, vel, absente sacerdote vel diacono, sic concluditur:}

\vspace{2mm}

\antiphona{C}{temporalia/dominusnosbenedicat.gtex}

\rubrica{Postea cantatur a cantore:}

\vspace{2mm}

\cuminitiali{IV}{temporalia/benedicamus-feria-laudes.gtex}

\vspace{1mm}

\vfill
\pagebreak

\end{document}

