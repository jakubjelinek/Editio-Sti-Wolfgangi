\newcommand{\titulus}{\nomenFesti{In Nativitate S. Ioannis Baptistæ.}
\dies{Die 24. Iunii.}}
\newcommand{\tedeumsolemnis}{Solemnis}
\newcommand{\oratio}{\pars{Oratio.}

\noindent Deus, qui beátum Ioánnem Baptístam suscitásti, ut perféctam plebem Christo Dómino præparáret, da pópulis tuis spiritálium grátiam gaudiórum et ómnium fidélium mentes dírige in viam salútis et pacis.

\pars{Pro pace in Ucraina.} \scriptura{Sir. 50, 25; 2 Esdr. 4, 20; \textbf{H416}}

\vspace{-4mm}

\antiphona{II D}{temporalia/ant-dapacemdomine.gtex}

\vfill

\noindent Deus, a quo sancta desidéria, recta consília et iusta sunt ópera: da servis tuis illam, quam mundus dare non potest, pacem; ut et corda nostra mandátis tuis dédita, et hóstium subláta formídine, témpora sint tua protectióne tranquílla.

\noindent Per Dóminum nostrum Iesum Christum, Fílium tuum, qui tecum vivit et regnat in unitáte Spíritus Sancti, Deus, per ómnia sǽcula sæculórum.

\noindent \Rbardot{} Amen.}
\newcommand{\invitatorium}{\pars{Invitatorium.} \scriptura{Cantor; Psalmus 94; \textbf{H273} \& \textbf{H447}}

\vspace{-6mm}

\antiphona{V}{temporalia/inv-regempraecursoris-FKP.gtex}}
\newcommand{\hymnusmatutinum}{\pars{Hymnus} \scriptura{Paulus Diaconus (\olddag{} 799)}

\cuminitiali{IV}{temporalia/hym-AntraDeserti.gtex}}
\newcommand{\nocturnoi}{\pars{Psalmus 1.} \scriptura{Ier. 1, 5; \textbf{H274}}

\vspace{-4mm}

\antiphona{VIII G}{temporalia/ant-priusquamteformarem.gtex}

%\vspace{-5mm}

\scriptura{Ps. 1}

%\vspace{-2mm}

\initiumpsalmi{temporalia/ps1-initium-viii-g-auto.gtex}

\input{temporalia/ps1-viii-g.tex} \Abardot{}

\vfill
\pagebreak

\pars{Psalmus 2.} \scriptura{Ier. 1, 7; \textbf{H274}}

\vspace{-4mm}

\antiphona{VIII G}{temporalia/ant-adomniaquaemittamte-FKP.gtex}

%\vspace{-5mm}

\scriptura{Ps. 2}

\initiumpsalmi{temporalia/ps2-initium-viii-G-auto.gtex}

\input{temporalia/ps2-viii-G.tex} \Abardot{}

\vfill
\pagebreak

\pars{Psalmus 3.} \scriptura{Ier. 1, 5; \textbf{H274}}

\vspace{-4mm}

\antiphona{VIII G}{temporalia/ant-netimeasafacieeorum-FKP.gtex}

%\vspace{-2mm}

\scriptura{Ps. 3}

%\vspace{-2mm}

\initiumpsalmi{temporalia/ps3-initium-viii-G-auto.gtex}

\input{temporalia/ps3-viii-G.tex} \Abardot{}

\vfill
\pagebreak}
\newcommand{\matversusi}{\pars{Versus.} \scriptura{Ps. 8, 6-7}

\noindent \Vbardot{} Glória et honóre coronásti eum, Dómine.

\noindent \Rbardot{} Et constituísti eum super ópera mánuum tuárum.}
\newcommand{\lectioi}{\pars{Lectio I.} \scriptura{Ier. 1, 1-4}

\noindent De libro Ieremíæ prophétæ.

\noindent Et factum est verbum Dómini ad me dicens: «Priúsquam te formárem in útero, novi te et, ántequam exíres de vulva, sanctificávi te et prophétam géntibus dedi te».

\noindent Et dixi: «Heu, Dómine Deus! Ecce néscio loqui, quia puer ego sum».}
\newcommand{\responsoriumi}{\pars{Responsorium 1.} \scriptura{\Rbardot{} Io. 1, 6.7 \Vbardot{} Mc. 1, 4; \textbf{H274}}

\vspace{-5mm}

\responsorium{VII}{temporalia/resp-fuithomo-FKP.gtex}{}}
\newcommand{\lectioii}{\pars{Lectio II.} \scriptura{Ier. 1, 7-10}

\noindent Et dixit Dóminus ad me: «Noli dícere: “Puer sum”, quóniam, ad quoscúmque mittam te, ibis et univérsa, quæcúmque mandávero tibi, loquéris. Ne tímeas a fácie eórum, quia tecum ego sum, ut éruam te», dicit Dóminus.

\noindent Et misit Dóminus manum suam et tétigit os meum et dixit Dóminus ad me: «Ecce dedi verba mea in ore tuo; ecce constítui te hódie super gentes et super regna, ut evéllas et déstruas et dispérdas et díssipes et ædífices et plantes.}
\newcommand{\responsoriumii}{\pars{Responsorium 2.} \scriptura{\Rbardot{} Cantor \Vbardot{} Io. 1, 6; \textbf{H274}}

\vspace{-5mm}

\responsorium{VII}{temporalia/resp-elisabethzachariae-FKP.gtex}{}}
\newcommand{\lectioiii}{\pars{Lectio III.} \scriptura{Ier. 1, 17-19}

\noindent Tu ergo accínge lumbos tuos et surge et lóquere ad eos ómnia, quæ ego præcípio tibi: ne tímeas a fácie eórum, alióquin timére te fáciam vultum eórum.

\noindent Ego quippe dedi te hódie in civitátem munítam et in colúmnam férream et in murum ǽreum contra omnem terram régibus Iudæ, princípibus eius et sacerdótibus et pópulo terræ; et bellábunt advérsum te et non prævalébunt, quia tecum ego sum, ait Dóminus, ut erípiam te».}
\newcommand{\responsoriumiii}{\pars{Responsorium 3.} \scriptura{\Rbardot{} Ier. 1, 5 \Vbardot{} Eccli. 45, 1; \textbf{H275}}

\vspace{-5mm}

\responsorium{VII}{temporalia/resp-priusquamteformarem-FKP.gtex}{}}
\newcommand{\nocturnoii}{\pars{Psalmus 4.} \scriptura{Is. 49, 1; \textbf{H274}}

\vspace{-4mm}

\antiphona{VIII G}{temporalia/ant-dominusabutero-FKP.gtex}

\vspace{-2mm}

\scriptura{Ps. 20}

\vspace{-2mm}

\initiumpsalmi{temporalia/ps20-initium-viii-G-auto.gtex}

\vspace{-1.5mm}

\input{temporalia/ps20-viii-G.tex} \Abardot{}

\vfill
\pagebreak

\pars{Psalmus 5.} \scriptura{Is. 49, 2; \textbf{H275}}

\vspace{-4mm}

\antiphona{VII c}{temporalia/ant-posuitosmeum-FKP.gtex}

%\vspace{-4mm}

\scriptura{Ps. 91, 2-9}

%\vspace{-2mm}

\initiumpsalmi{temporalia/ps91i-initium-vii-c-auto.gtex}

\input{temporalia/ps91i-vii-c.tex} \Abardot{}

\vfill
\pagebreak

\pars{Psalmus 6.} \scriptura{Io. 1, 15; Io. 1, 19.27.30.32; \textbf{H31}}

\vspace{-4mm}

\antiphona{I f}{temporalia/ant-hocesttestimonium.gtex}

\scriptura{Ps. 91, 10-16}

%\vspace{-2mm}

\initiumpsalmi{temporalia/ps91ii-initium-i-f-auto.gtex}

\input{temporalia/ps91ii-i-f.tex} \Abardot{}

\vfill
\pagebreak}
\newcommand{\matversusii}{\pars{Versus.} \pars{Versus.} \scriptura{Ps. 20, 4}

\noindent \Vbardot{} Posuísti, Dómine, super caput eius.

\noindent \Rbardot{} Corónam de lápide pretióso.}
\newcommand{\lectioiv}{\pars{Lectio IV.} \scriptura{Sermo 293, 1-3: PL 38, 1327-1328}

\noindent Ex Sermónibus sancti Augustíni epíscopi.

\noindent Nativitátem Ioánnis quodámmodo consecrátam obsérvat Ecclésia: nec invenítur ullus in pátribus, cuius nativitátem sollémniter celebrémus; celebrámus Ioánnis, celebrámus et Christi: hoc vacáre non potest, et si forte a nobis pro tantæ rei dignitáte minus explicátur, fructuósius tamen et áltius cogitátur. Náscitur Ioánnes de anícula stérili, náscitur Christus de iuvéncula vírgine.

\noindent Non créditur Ioánnes nascitúrus, et fit pater mutus; créditur Christus, et fide concípitur. Proposúimus inquirénda, et discutiénda prædíximus; sed hoc prælocútus sum, et si ómnibus tanti mystérii sínibus perscrutándis non suffícimus vel facultáte vel témpore; mélius vos docébit qui lóquitur in vobis, étiam abséntibus nobis, quem pie cogitátis, quem corde suscepístis, cuius templa facti estis.}
\newcommand{\responsoriumiv}{\pars{Responsorium 4.} \scriptura{\Rbardot{} Lc. 1, 13 \Vbardot{} ibid. 1, 12.13; \textbf{H275}}

\vspace{-5mm}

\responsorium{VII}{temporalia/resp-descenditangelus-FKP.gtex}{}}
\newcommand{\lectiov}{\pars{Lectio V.}

\noindent Vidétur ergo Ioánnes interiéctus quidam limes testamentórum duórum, véteris et novi. Nam eum esse quodámmodo límitem, Dóminus ipse testátur dicens: \emph{Lex et prophétæ usque ad Ioánnem Baptístam.} Sústinet ergo persónam vetustátis, et præcónium novitátis. Propter persónam vetustátis, de sénibus náscitur; propter persónam novitátis, in viscéribus matris prophéta declarátur. Nondum enim natus ad Sanctæ Maríæ advéntum, exsultávit in útero matris. Iam ibi designátus erat, designátus ántequam natus; cuius præcúrsor esset osténditur, ántequam ab eo viderétur. Divína sunt hæc, et mensúram humánæ fragilitátis excédunt. Postrémo náscitur, áccipit nomen, lingua sólvitur patris. Refer quod factum est ad significántem imáginem rerum.}
\newcommand{\responsoriumv}{\pars{Responsorium 5.} \scriptura{\Rbardot{} Io. 5, 35 \Vbardot{} Lc. 1, 17; \textbf{H276}}

\vspace{-5mm}

\responsorium{VII}{temporalia/resp-hicpraecursor-FKP.gtex}{}}
\newcommand{\lectiovi}{\pars{Lectio VI.}

\noindent Zacharías tacet et amíttit vocem, donec Ioánnes nascerétur, præcúrsor Dómini, et aperíret vocem. Quid est siléntium Zacharíæ, nisi prophetía latens, et ante prædicatiónem Christi quodam modo occúlta et clausa? Aperítur illíus advéntu, clara fit ventúro eo qui prophetabátur. Hoc est apértio vocis Zacharíæ in nativitáte Ioánnis, quod est discíssio veli in cruce Christi. Ioánnes si seípsum nuntiáret, Zacharíæ os non aperíret. Sólvitur lingua, quia náscitur vox; nam Ioánni iam prænuntiánti Dóminum dictum est: \emph{Tu quis es?} Et respóndit: \emph{Ego sum vox clamántis in erémo.} Vox Ioánnes, Dóminus autem \emph{in princípio erat Verbum.}  Ioánnes vox ad tempus, Christus Verbum in princípio ætérnum.}
\newcommand{\responsoriumvi}{\pars{Responsorium 6.} \scriptura{\Rbardot{} Lc. 1, 62-64 \Vbardot{} ibid. 1, 64.67; \textbf{H276}}

\vspace{-5mm}

\responsorium{II}{temporalia/resp-innuebantpatrieius-FKP.gtex}{}}
\newcommand{\nocturnoiii}{\pars{Cantica.} \scriptura{Is. 49, 7; \textbf{H275}}

\vspace{-4mm}

\antiphona{VIII c}{temporalia/ant-regesvidebunt-FKP.gtex}

%\vspace{-2mm}

\scriptura{Canticum Ieremiæ, Ier. 17, 7-8}

\initiumpsalmi{temporalia/jeremiae-initium-viii-c-auto.gtex}

%\vspace{-2mm}

\input{temporalia/jeremiae-viii-c.tex} \hfill \rubrica{Hic non dicitur antiphona.}

\vfill
\pagebreak

\scriptura{Canticum Beatitudo Sapientis; Sir. 14, 22; ibid. 15, 3.4.6}

\initiumpsalmi{temporalia/beatitudosapientis-initium-viii-c-auto.gtex}

\input{temporalia/beatitudosapientis-viii-c.tex}

\vfill
\pagebreak

\scriptura{Canticum Ecclesiasticæ; Sir. 31, 8-11}

\initiumpsalmi{temporalia/ecclesiasticus31-initium-viii-c-auto.gtex}

\input{temporalia/ecclesiasticus31-viii-c.tex}

\antiphona{}{temporalia/ant-regesvidebunt-FKP.gtex}

\vfill
\pagebreak}
\newcommand{\evangelium}{
\pars{Versus.} \scriptura{Cf. Io. 1, 6}

\noindent \Vbardot{} Elísabeth Zacharíæ magnum virum génuit.

\noindent \Rbardot{} Ioánnem Baptístam, præcursórem Dómini.

\vspace{5mm}

\sineinitiali{temporalia/oratiodominica-mat.gtex}

\vspace{5mm}

\pars{Absolutio.}

\cuminitiali{}{temporalia/absolutio-avinculis.gtex}

\vfill
\pagebreak

\cuminitiali{}{temporalia/benedictio-solemn-evangelica.gtex}

\vspace{7mm}

\pars{Evangelium} \scriptura{Lc. 1, 57-66,80}

\noindent Léctio sancti Evangélii secúndum Lucam.

\noindent Elísabeth implétum est tempus pariéndi, et péperit fílium. Et audiérunt vicíni et cognáti eius quia magnificávit Dóminus misericórdiam suam cum illa, et congratulabántur ei.

\noindent Et factum est, in die octávo venérunt circumcídere púerum et vocábant eum nómine patris eius, Zacharíam. Et respóndens mater eius dixit: «Nequáquam, sed vocábitur Ioánnes».

\noindent Et dixérunt ad illam: «Nemo est in cognatióne tua, qui vocétur hoc nómine».

\noindent Innuébant autem patri eius quem vellet vocári eum. Et póstulans pugillárem scripsit dicens: «Ioánnes est nomen eius». Et miráti sunt univérsi. Apértum est autem ílico os eius et lingua eius, et loquebátur benedícens Deum.

\noindent Et factus est timor super omnes vicínos eórum, et super ómnia montána Iudǽæ divulgabántur ómnia verba hæc. Et posuérunt omnes, qui audíerant, in corde suo dicéntes: «Quid putas puer iste erit?». Etenim manus Dómini erat cum illo.

\noindent Puer autem crescebat et confortabatur spiritu et erat in deserto usque in diem ostentionis suæ ad Israel.

\scriptura{Sermo 216 : CCL 104, 858-859}

\noindent Ex Sermónibus sancti Cæsárii Arelaténsis epíscopi.

\noindent Natálem sancti Ioánnis, fratres caríssimi, hódie celebrámus; quod nulli umquam sanctórum légimus fuísse concéssum.

\noindent Solíus enim Dómini et beáti Ioánnis dies nativitátis in univérso mundo celebrátur et cólitur; illum enim stérilis péperit, istum virgo concépit; in Elísabeth sterílitas víncitur, in beáta María conceptiónis consuetúdo mutátur.

\noindent Elísabeth virum cognoscéndo fílium génuit; María ángelo crédidit, et concépit.

\noindent Hóminem concépit Elísabeth, et hóminem María, sed Elísabeth solum hóminem, María Deum et hóminem.

\noindent Quid sibi vult ergo Ioánnes? Unde interpósitus? Unde præmíssus? Magnus ígitur Ioánnes, cuius magnitúdini étiam Salvátor testimónium pérhibet dicens: \emph{Non surréxit inter natos mulíerum maior Ioánne Baptísta.}

\noindent Præcéllit cunctis, éminet univérsis; antecédit prophétas, supergréditur patriárchas; et quisquis de mulíere natus est, inférior est Ioánne.

\noindent Dicit fortásse áliquis: «Si inter natos mulíerum Ioánnes maior est, maior est Salvatóre.» Absit.

\noindent Ioánnes enim natus mulíeris, Christus autem vírginis natus est; ille corruptíbilis úteri sínibus effúsus est, iste impollútæ vulvæ flore progénitus.

\noindent Ideo autem cum Ioánnis nativitáte Dómini generátio deputátur, ne Dóminus extra veritátem videátur conditiónis humánæ.

\noindent Si comparétur homínibus Ioánnes, omnes súperat ille homo, non eum vincit nisi Deus homo. Ioánnes præmíssus est ante Deum. Tanta in illo excelléntia erat, tanta grátia, ut ipse putátus sit Christus.

\noindent Quid ergo dixit de Christo? \emph{Nos omnes de plenitúdine eius accépimus.} Quid est \emph{nos omnes}? Prophétæ, patriárchæ, apóstoli, quotquot sancti et ante incarnatiónem præmíssi vel ab incarnáto missi, omnes nos de plenitúdine eius accépimus: nos vasa sumus, ille fons est.

\noindent Si ergo intelléximus mystérium, fratres mei, Ioánnes homo est, Christus Deus est.

\noindent Humiliétur homo et exaltétur Deus, secúndum illud quod de Dómino ipse Ioánnes dixit: \emph{Illum opórtet créscere, me autem mínui.}

\noindent Ut humiliarétur homo, eo die natus est Ioánnes, quo incípiunt decréscere dies; ut exaltétur Deus, eo die natus est Christus, quo incípiunt créscere dies.

\noindent Magnum sacraméntum, fratres dilectíssimi, ídeo celebrámus natálem Ioánnis sicut et Christi, quia ipsa natívitas plena est mystério. Quo mystério, nisi humilitátis nostræ, sicut natívitas Christi plena est mystério altitúdinis nostræ?

\noindent Ergo in hómine minuámur, ut in Deo crescámus; in nobis humiliémur, ut in illo exaltémur.

\vfill
\pagebreak

\pars{Responsorium 7.} \scriptura{\Rbardot{} Mt. 11, 11 \Vbardot{} Io. 1, 6; \textbf{H276}}

\vspace{-5mm}

\responsorium{I}{temporalia/resp-internatosmulierum-FKP.gtex}{}

\vfill

\rubrica{vel ad libitum:}

\vspace{3mm}

\pars{Responsorium 7.} \scriptura{\Rbardot{} Lc. 1, 76 \Vbardot{} ibid., 77; \textbf{H82} \& \textbf{E269}}

\vspace{-5mm}

\responsorium{I}{temporalia/resp-tupuerpropheta-CROCHU-cumdox.gtex}{}

\vfill
\pagebreak
}
\newcommand{\hymnuslaudes}{\pars{Hymnus}

\cuminitiali{IV}{temporalia/hym-ONimis.gtex}}
\newcommand{\laudes}{\pars{Psalmus 1.} \scriptura{Lc. 1, 63.14; \textbf{H277}}

\vspace{-0.4cm}

\antiphona{I f}{temporalia/ant-joannesvocabitur.gtex}

\scriptura{Psalmus 62.}

\initiumpsalmi{temporalia/ps62-initium-i-f-auto.gtex}

\input{temporalia/ps62-i-f.tex} \Abardot{}

\vfill
\pagebreak

\pars{Psalmus 2.} \scriptura{Cf. Lc. 1, 17; \textbf{H275}}

\vspace{-0.4cm}

\antiphona{VII a}{temporalia/ant-ipsepraeibitanteillum.gtex}

\scriptura{Canticum trium puerorum, Dan. 3, 57-88 et 56}

\initiumpsalmi{temporalia/dan3-initium-vii-a-auto.gtex}

\input{temporalia/dan3-vii-a-sinedox.tex}

\rubrica{Hic non dicitur Gloria Patri, neque Amen.}

\antiphona{}{temporalia/ant-ipsepraeibitanteillum.gtex}

\vfill
\pagebreak

\pars{Psalmus 3.} \scriptura{Lc. 1, 76; \textbf{H277}}

\vspace{-0.4cm}

\antiphona{III b}{temporalia/ant-tupuerpropheta.gtex}

\scriptura{Psalmus 149.}

\initiumpsalmi{temporalia/ps149-initium-iii-b-auto.gtex}

\input{temporalia/ps149-iii-b.tex} \Abardot{}

\vfill
\pagebreak}
\newcommand{\lectiobrevis}{\pars{Lectio Brevis.} \scriptura{Mal. 3, 23-24}

\noindent Ecce ego mittam vobis Elíam prophétam, ántequam véniat dies Dómini magnus et horríbilis; et convértet cor patrum ad fílios et cor filiórum ad patres eórum, ne véniam et percútiam terram anathémate.}
\newcommand{\responsoriumbreve}{\pars{Responsorium breve.} \scriptura{Cf. Lc. 7, 28}

\cuminitiali{VI}{temporalia/resp-internatos.gtex}}
\newcommand{\benedictus}{\pars{Canticum Zachariæ.} \scriptura{Lc. 1, 64.67.68; \textbf{H277}}

\vspace{-4mm}

{
\grechangedim{interwordspacetext}{0.18 cm plus 0.15 cm minus 0.05 cm}{scalable}%
\antiphona{VIII G\textsuperscript{2}}{temporalia/ant-apertumestoszachariae.gtex}
\grechangedim{interwordspacetext}{0.22 cm plus 0.15 cm minus 0.05 cm}{scalable}%
}

\vspace{-2mm}

\scriptura{Lc. 1, 68-79}

\vspace{-1mm}

\cantusSineNeumas
\initiumpsalmi{temporalia/benedictus-initium-viiisoll-g5-auto.gtex}

%\vspace{-1.5mm}

\input{temporalia/benedictus-viiisoll-g5.tex} \Abardot{}}
\newcommand{\preces}{\noindent Christum, qui Ioánnem præcursórem ante fáciem suam misit, \gredagger{} ut viam Dómini paráret, \grestar{} fidénter deprecémur:

\Rbardot{} Vísita nos, Oriens ex alto.

\noindent Tu, qui Ioánnem exsultáre fecísti in sinu Elísabeth, \grestar{} de tuo advéntu in hunc mundum fac nos semper gaudére.

\Rbardot{} Vísita nos, Oriens ex alto.

\noindent Tu, qui Baptístæ ore et vita viam pæniténtiæ nobis indicásti, \grestar{} convérte corda nostra ad mandáta regni tui.

\Rbardot{} Vísita nos, Oriens ex alto.

\noindent Tu, qui ore hóminis te prædicári voluísti \grestar{} mitte in orbem totum Evangélii tui præcónes.

\Rbardot{} Vísita nos, Oriens ex alto.

\noindent Tu, qui in Iordáne a Ioánne baptizári voluísti \gredagger{} ut omnis iustítia implerétur, \grestar{} fac nos iustítiæ regni tui adlaboráre.

\Rbardot{} Vísita nos, Oriens ex alto.}
\newcommand{\benedicamuslaudes}{\cuminitiali{II}{temporalia/benedicamus-solemnism-laud.gtex}}
\newcommand{\hebdomada}{infra Hebdom. XI post Pentecosten.}
\newcommand{\oratioLaudes}{\cuminitiali{}{temporalia/oratio11.gtex}}

% LuaLaTeX

\documentclass[a4paper, twoside, 12pt]{article}
\usepackage[latin]{babel}
%\usepackage[landscape, left=3cm, right=1.5cm, top=2cm, bottom=1cm]{geometry} % okraje stranky
%\usepackage[landscape, a4paper, mag=1166, truedimen, left=2cm, right=1.5cm, top=1.6cm, bottom=0.95cm]{geometry} % okraje stranky
\usepackage[landscape, a4paper, mag=1400, truedimen, left=0.5cm, right=0.5cm, top=0.5cm, bottom=0.5cm]{geometry} % okraje stranky

\usepackage{fontspec}
\setmainfont[FeatureFile={junicode.fea}, Ligatures={Common, TeX}, RawFeature=+fixi]{Junicode}
%\setmainfont{Junicode}

% shortcut for Junicode without ligatures (for the Czech texts)
\newfontfamily\nlfont[FeatureFile={junicode.fea}, Ligatures={Common, TeX}, RawFeature=+fixi]{Junicode}

\usepackage{multicol}
\usepackage{color}
\usepackage{lettrine}
\usepackage{fancyhdr}

% usual packages loading:
\usepackage{luatextra}
\usepackage{graphicx} % support the \includegraphics command and options
\usepackage{gregoriotex} % for gregorio score inclusion
\usepackage{gregoriosyms}
\usepackage{wrapfig} % figures wrapped by the text
\usepackage{parcolumns}
\usepackage[contents={},opacity=1,scale=1,color=black]{background}
\usepackage{tikzpagenodes}
\usepackage{calc}
\usepackage{longtable}
\usetikzlibrary{calc}

\setlength{\headheight}{14.5pt}

\input{conventuscommune.tex} % Often used macros
%%%% Preklady jednotlivych zpevu (nektere se opakuji, a je dobre mit je
% vsechny na jedne hromade)

% HOURS ---

\newcommand{\trAntI}{\translatioCantus{Muž boží měl kožený toulec, pečlivě
zavázaný, jenž mu visel na šíji a~často se ho dotýkal.}}

\newcommand{\trAntII}{\translatioCantus{Klíč od~něho tak dobře střežil, že
dokud žil v~těle, nikdo z~jeho žáků nezvěděl, co je uvnitř.}}

\newcommand{\trAntIII}{\translatioCantus{Ale když se odebral z~tohoto
života, schránku otevřeli a~objevili v~ní žíněné roucho a~měděný řetěz
potřísněný krví.}}

\newcommand{\trAntIV}{\translatioCantus{A když prohlédli mistrovo tělo,
nalezli jeho tělo na čtyřech místech hluboce zbrázděno ranami od řetězu.}}

\newcommand{\trAntV}{\translatioCantus{Krev vytékající z~těch ran, místy
prostoupila i~žíněným rouchem.}}

\newcommand{\trCapituli}{\translatioCantus{
Miláčkovi Boha a~lidí,
Mojžíšovi požehnané paměti,~\gredagger{}
dopřál slávu rovnou slávě svatých~\grestar{}
učinil ho mocným na postrach nepřátelům
a~jeho slovy zastavil divy.}}

\newcommand{\trLectioBrevis}{\translatioCantus{
Pamatujte na své představené,
kteří vám hlásali Boží slovo.
Uvažte, jak oni skončili život, a~napodobujte jejich víru.
Ježíš Kristus je stejný včera i~dnes i~navěky.
Nenechte se svést věelijakými cizími naukami.}}

\newcommand{\trRespLaud}{\translatioCantus{Spravedlivého vodil Hospodin~\grestar{}
po přímých stezkách. \Vbardot{} A~ukázal mu Boží království.}}

\newcommand{\trRespLaudB}{\translatioCantus{Na tvých hradbách, Jeruzaléme,
ustanovil jsem strážné;~\grestar{}
budou bdít nad mým lidem. \Vbardot{} Ani ve dne, ani v~noci nesmějí nikdy
mlčet.}}

\newcommand{\trVersus}{\translatioCantus{\Vbardot{} Ústa spravedlivého šeptají moudrost, aleluja.
\Rbardot{} A~jeho jazyk ohlašuje právo, aleluja.}}

\newcommand{\trAntBenedictus}{\translatioCantus{Když na bujné oře vložili
nosítka a~sňali jim uzdu, vydali se přímo k~cele božího muže.}}

\newcommand{\trPreces}{\translatioCantus{
\noindent S vděčností chvalme Krista, dobrého Pastýře, \gredagger{} který dal život za své ovce, \grestar{} a~pokorně ho prosme: \Rbardot{} Pane, buď pastýřem svého lidu.

\noindent Kriste, ty dáváš církvi pastýře, a~jejich službou se ujímáš svého lidu, \grestar{} dej, ať v~lásce těch, kteří nás vedou, poznáváme, jak nás miluješ. \Rbardot{} Pane, buď pastýřem svého lidu.

\noindent Ty stále konáš skrze své zástupce službu pastýře a~učitele, \grestar{} nepřestávej nás nikdy vést prostřednictvím svých služebníků. \Rbardot{} Pane, buď pastýřem svého lidu.

\noindent Ty prokazuješ svému lidu skrze jeho pastýře službu lékaře duše i~těla, \grestar{} ochraňuj náš život a~veď nás ke svatosti. \Rbardot{} Pane, buď pastýřem svého lidu.

\noindent Ty posíláš své svaté, aby slovem i~příkladem vedli tvůj lid k~tobě, \grestar{} na jejich přímluvu nás posiluj, abychom vytrvali na cestě, která vede k~věčnému životu. \Rbardot{} Pane, buď pastýřem svého lidu.}}

\newcommand{\trOrationis}{\translatioCantus{Bože, jenž nám dopřáváš radovat
se z~výroční slavnosti svatého tvého vyznavače Havla, uděl dobrotivě,
abychom když slavíme jeho narození, též se řídili podobou jeho skutků.
Skrze…}}
 % Czech translations of the proper texts

\newcommand{\annusEditionis}{2020}

%%%% Vicekrat opakovane kousky

\newcommand{\anteOrationem}{
  \rubrica{Ante Orationem, cantatur a Superiore:}

  \pars{Supplicatio Litaniæ.}

  \cuminitiali{}{temporalia/supplicatiolitaniae.gtex}

  \pars{Oratio Dominica.}

  \cuminitiali{}{temporalia/oratiodominica.gtex}

  \rubrica{Deinde dicitur ab Hebdomadario:}

  \cuminitiali{}{temporalia/dominusvobiscum-solemnis.gtex}

  \rubrica{In choro monialium loco Dominus vobiscum dicitur:}

  \sineinitiali{temporalia/domineexaudi.gtex}
}

\setlength{\columnsep}{30pt} % prostor mezi sloupci

%%%%%%%%%%%%%%%%%%%%%%%%%%%%%%%%%%%%%%%%%%%%%%%%%%%%%%%%%%%%%%%%%%%%%%%%%%%%%%%%%%%%%%%%%%%%%%%%%%%%%%%%%%%%%
\begin{document}

% Here we set the space around the initial.
% Please report to http://home.gna.org/gregorio/gregoriotex/details for more details and options
\grechangedim{afterinitialshift}{2.2mm}{scalable}
\grechangedim{beforeinitialshift}{2.2mm}{scalable}
\grechangedim{interwordspacetext}{0.22 cm plus 0.15 cm minus 0.05 cm}{scalable}%
\grechangedim{annotationraise}{-0.2cm}{scalable}

% Here we set the initial font. Change 38 if you want a bigger initial.
% Emit the initials in red.
\grechangestyle{initial}{\color{red}\fontsize{38}{38}\selectfont}

\pagestyle{empty}

%%%% Titulni stranka
\begin{titulusOfficii}
\titulus{}
\end{titulusOfficii}

% graphic
%\vspace{1.5cm}
%\begin{center}
%\includegraphics[width=8cm]{emmaus.jpg}
%\end{center}

\vfill

\begin{center}
%Ad usum et secundum consuetudines chori \guillemotright{}Conventus Choralis\guillemotleft.

%Editio Sancti Wolfgangi \annusEditionis
\end{center}

\pagebreak

\renewcommand{\headrulewidth}{0pt} % no horiz. rule at the header
\fancyhf{}
\pagestyle{fancy}

\pars{Oratio ante divinum Officium.}

\lettrine{{\color{red}A}}{peri,} Dómine, os meum ad benedicéndum nomen sanctum tuum:
munda quoque cor meum ab ómnibus vanis, pervérsis, et aliénis
cogitatiónibus:
intelléctum illúmina, afféctum inflámma,
ut digne, atténte ac devóte hoc Offícium recitáre váleam,
et exaudíri mérear ante conspéctum Divínæ Maiestátis tuæ.
Per Christum, Dóminum nostrum.
\Rbardot{} Amen.

Dómine, in unióne illíus divínæ intentiónis,
qua ipse in terris laudes Deo persolvísti,
has tibi Horas \rubricatum{(vel \textnormal{hanc tibi Horam})} persólvo.

%\trOratioAnteOfficium

\vfill

\pars{Oratio post divinum Officium.}

\rubrica{
  Orationem sequentem devote post Officium recitantibus
  Leo Papa X. defectus, et culpas in eo persolvendo ex humana
  fragilitate contractas, indulsit, et dicitur flexis genibus.
}

\lettrine{{\color{red}S}}{acrosánctæ} et indivíduæ Trinitáti,
crucifíxi Dómini nostri Iesu Christi humanitáti,
beatíssimæ et gloriosíssimæ sempérque Vírginis Maríæ
fecúndæ integritáti, 
et ómnium Sanctórum universitáti
sit sempitérna laus, honor, virtus et glória
ab omni creatúra,
nobísque remíssio ómnium peccatórum,
per infiníta sǽcula sæculórum.
\Rbardot{} Amen.

\noindent \Vbardot{} Beáta víscera Maríæ Virginis, quæ portavérunt
ætérni Patris Fílium.\\
\Rbardot{} Et beáta úbera, quæ lactavérunt Christum Dominum.

\rubrica{Et dicitur secreto \textnormal{Pater noster.} et \textnormal{Ave María.}}

%\trOratioPostOfficium

\vfill

\hora{Ad I. Vesperas.} %%%%%%%%%%%%%%%%%%%%%%%%%%%%%%%%%%%%%%%%%%%%%%%%%%%%%
%\sideThumbs{I. Vesperæ}

\cantusSineNeumas

\vspace{0.5cm}
\grechangedim{interwordspacetext}{0.18 cm plus 0.15 cm minus 0.05 cm}{scalable}%
\cuminitiali{}{temporalia/deusinadiutorium-solemnis.gtex}
\grechangedim{interwordspacetext}{0.22 cm plus 0.15 cm minus 0.05 cm}{scalable}%

\vfill
\pagebreak

\pars{Psalmus 1.} \scriptura{Ps. 144, 13; \textbf{H100}}

\vspace{-4mm}

\antiphona{VII c\textsuperscript{2}}{temporalia/ant-regnumtuum.gtex}

\scriptura{Psalmus 144, 10-21.}

\initiumpsalmi{temporalia/ps144ii-initium-vii-c2-auto.gtex}

%\psalmusEtTranslatioT{temporalia/ps144ii-VII-comb.tex}{10cm}
\input{temporalia/ps144ii-VII.tex} \Abardot{}

\vspace{-1cm}

\vfill
\pagebreak

\pars{Psalmus 2.} \scriptura{Ps. 145, 2; \textbf{H100}}

\vspace{-4mm}

\antiphona{IV E}{temporalia/ant-laudabodeum.gtex}

\scriptura{Psalmus 145.}

\initiumpsalmi{temporalia/ps145-initium-iv-E-auto.gtex}

%\psalmusEtTranslatioT{temporalia/ps145-VII-comb.tex}{10cm}
\input{temporalia/ps145-VII.tex} \Abardot{}

\vfill
\pagebreak

\pars{Psalmus 3.} \scriptura{Ps. 146, 1; \textbf{H101}}

\vspace{-4mm}

\antiphona{VIII a}{temporalia/ant-deonostro.gtex}

\scriptura{Psalmus 146.}

\initiumpsalmi{temporalia/ps146-initium-viii-A-auto.gtex}

%\psalmusEtTranslatioT{temporalia/ps146-VII-comb.tex}{10cm}
\input{temporalia/ps146-VII.tex} \Abardot{}

\vfill
\pagebreak

\pars{Psalmus 4.} \scriptura{Ps. 147, 1}

\vspace{-4mm}

\antiphona{E}{temporalia/ant-laudajerusalem.gtex}

\scriptura{Psalmus 147.}

\initiumpsalmi{temporalia/ps147-initium-e-auto.gtex}

%\psalmusEtTranslatioT{temporalia/ps147-VII-comb.tex}{10cm}
\input{temporalia/ps147-VII.tex} \Abardot{}

\vfill
\pagebreak

\pars{Capitulum.} \scriptura{Rom. 11, 33}

\grechangedim{interwordspacetext}{0.12 cm plus 0.15 cm minus 0.05 cm}{scalable}%
\cuminitiali{}{temporalia/capitulum-OAltitudo.gtex}
\grechangedim{interwordspacetext}{0.22 cm plus 0.15 cm minus 0.05 cm}{scalable}

% preklad Jeruz. bible
%\trCapituliI

\vfill

\pars{Responsorium breve.} \scriptura{Ps. 146, 5}

\cuminitiali{VI}{temporalia/resp-magnusdominusnoster.gtex}

%\trResp

\vfill
\pagebreak

\pars{Hymnus} \scriptura{Ambrosius (\olddag{} 397)}

\cuminitiali{I}{temporalia/hym-OLuxBeata-aestivalis.gtex}
\vspace{-3mm}
%\input{hym-OLuxBeata-bohtext.tex}

\vfill
%\pagebreak

\pars{Versus.}

% Versus. %%%
\sineinitiali{temporalia/versus-vespertina.gtex}

%\noindent \trVersus

\vfill
\pagebreak

\magnificati

\vfill
\pagebreak

%\sideThumbs{{\scriptsize{}Fine horarum}}

\anteOrationem

\pagebreak

% Oratio. %%%
\oratioLaudes

\vspace{-1mm}
%\trOrationisI

\vfill

\rubrica{Hebdomadarius dicit iterum Dominus vobiscum, vel cantor dicit:}

\vspace{2mm}

\sineinitiali{temporalia/domineexaudi.gtex}

\rubrica{Postea cantatur a cantore:}

\vspace{2mm}

\cuminitiali{I}{temporalia/benedicamus-dominica-perannum.gtex}

\vspace{1mm}

\vfill
\pagebreak

\hora{Ad Matutinum.} %%%%%%%%%%%%%%%%%%%%%%%%%%%%%%%%%%%%%%%%%%%%%%%%%%%%%
%\sideThumbs{Matutinum}

\vspace{2mm}

\cuminitiali{}{temporalia/dominelabiamea.gtex}

\vspace{2mm}

\pars{Invitatorium.} \scriptura{Ps. 94, 1; Psalmus 94}

\vspace{-6mm}

\antiphona{E}{temporalia/inv-veniteexsultemus.gtex}

\vfill
\pagebreak

\pars{Hymnus.} \scriptura{Adamus Sancti Victoris (\olddag 1146)}

\vspace{-5mm}

\antiphona{VII}{temporalia/hym-SalveDies.gtex}

\scriptura{Non dicitur \textnormal{Amen} in fine.}
%{
%\vspace{-5mm}
%\setlength{\columnsep}{0pt} % prostor mezi sloupci
%\input{hym-SalveDies-bohtext.tex}
%\setlength{\columnsep}{30pt} % prostor mezi sloupci
%}

\vfill
\pagebreak

\subhora{In I. Nocturno}

\pars{Psalmus 1.} \scriptura{Ps. 1, 1}

\vspace{-4mm}

\antiphona{VIII G}{temporalia/ant-beatusvir.gtex}

%\vspace{-5mm}

\scriptura{Ps. 1}

%\vspace{-2mm}

\initiumpsalmi{temporalia/ps1-initium-viii-G-auto.gtex}

%\psalmusEtTranslatioT{temporalia/ps1-I-comb.tex}{10cm}
\input{temporalia/ps1-I.tex} \Abardot{}

\vfill
\pagebreak

\pars{Psalmus 2.} \scriptura{Ps. 2, 11; \textbf{H93}}

\vspace{-4mm}

\antiphona{VII a}{temporalia/ant-servitedomino.gtex}

\vspace{-3mm}

\scriptura{Ps. 2}

\vspace{-2mm}

\initiumpsalmi{temporalia/ps2-initium-vii-a-auto.gtex}

%\psalmusEtTranslatioT{temporalia/ps2-I-comb.tex}{10cm}
\input{temporalia/ps2-I.tex} \Abardot{}

\vfill
\pagebreak

\pars{Psalmus 3.} \scriptura{Ps. 3, 7}

\vspace{-4mm}

\antiphona{VI F}{temporalia/ant-exsurgedominesalvum.gtex}

%\vspace{-5mm}

\scriptura{Ps. 3}

\initiumpsalmi{temporalia/ps3-initium-vi-F-auto.gtex}

%\psalmusEtTranslatioT{temporalia/ps3-I-comb.tex}{10cm}
\input{temporalia/ps3-I.tex} \Abardot{}

\vfill
\pagebreak

\pars{Versus.} \scriptura{Ps. 118, 55}

% Versus. %%%
\sineinitiali{temporalia/versus-memorfui.gtex}

\vspace{5mm}

\sineinitiali{temporalia/oratiodominica-mat.gtex}

\vspace{5mm}

\pars{Absolutio.}

\cuminitiali{}{temporalia/absolutio-exaudi.gtex}

\vfill
\pagebreak

\cuminitiali{}{temporalia/benedictio-solemn-benedictione.gtex}

\vspace{7mm}

\lectioi

\noindent \Vbardot{} Tu autem, Dómine, miserére nobis.
\noindent \Rbardot{} Deo grátias.

\vfill
\pagebreak

\responsoriumi

\vfill
\pagebreak

\cuminitiali{}{temporalia/benedictio-solemn-unigenitus.gtex}

\vspace{7mm}

\lectioii

\noindent \Vbardot{} Tu autem, Dómine, miserére nobis.
\noindent \Rbardot{} Deo grátias.

\vfill
\pagebreak

\responsoriumii

\vfill
\pagebreak

\cuminitiali{}{temporalia/benedictio-solemn-spiritus.gtex}

\vspace{7mm}

\lectioiii

\noindent \Vbardot{} Tu autem, Dómine, miserére nobis.
\noindent \Rbardot{} Deo grátias.

\vfill
\pagebreak

\responsoriumiii

\vfill
\pagebreak

\subhora{In II. Nocturno}

\pars{Psalmus 4.} \scriptura{Ps. 8, 2}

\vspace{-4mm}

\antiphona{I g}{temporalia/ant-quamadmirabileest.gtex}

%\vspace{-5mm}

\scriptura{Ps. 8}

%A\vspace{-2mm}

\initiumpsalmi{temporalia/ps8-initium-i-g-auto.gtex}

%\psalmusEtTranslatioT{temporalia/ps8-I-comb.tex}{10cm}
\input{temporalia/ps8-I.tex} \Abardot{}

\vfill
\pagebreak

\pars{Psalmus 5.} \scriptura{Ps. 9, 5}

\vspace{-4mm}

\antiphona{VIII G}{temporalia/ant-sedistisuperthronum.gtex}

%\vspace{-5mm}

\scriptura{Ps. 9, 2-11}

\initiumpsalmi{temporalia/ps9ii_xi-initium-viii-G-auto.gtex}

%\psalmusEtTranslatioT{temporalia/ps9ii_xi-I-comb.tex}{10cm}
\input{temporalia/ps9ii_xi-I.tex} \Abardot{}

\vfill
\pagebreak

\pars{Psalmus 6.} \scriptura{Ps. 9, 20}

\vspace{-4mm}

\antiphona{I g\textsuperscript{3}}{temporalia/ant-exsurgedominenon.gtex}

%\vspace{-5mm}

\scriptura{Ps. 9, 12-21}

\initiumpsalmi{temporalia/ps9xii_xxi-initium-i-g3-auto.gtex}

%\psalmusEtTranslatioT{temporalia/ps9xii_xxi-I-comb.tex}{10cm}
\input{temporalia/ps9xii_xxi-I.tex} \Abardot{}

\vfill
\pagebreak

\pars{Versus.} \scriptura{Ps. 118, 62}

% Versus. %%%
\sineinitiali{temporalia/versus-medianocte.gtex}

\vspace{5mm}

\sineinitiali{temporalia/oratiodominica-mat.gtex}

\vspace{5mm}

\pars{Absolutio.}

\cuminitiali{}{temporalia/absolutio-ipsius.gtex}

\vfill
\pagebreak

\cuminitiali{}{temporalia/benedictio-solemn-deus.gtex}

\vspace{7mm}

\lectioiv

\noindent \Vbardot{} Tu autem, Dómine, miserére nobis.
\noindent \Rbardot{} Deo grátias.

\vfill
\pagebreak

\responsoriumiv

\vfill
\pagebreak

\cuminitiali{}{temporalia/benedictio-solemn-christus.gtex}

\vspace{7mm}

\lectiov

\noindent \Vbardot{} Tu autem, Dómine, miserére nobis.
\noindent \Rbardot{} Deo grátias.

\vfill
\pagebreak

\responsoriumv

\vfill
\pagebreak

\cuminitiali{}{temporalia/benedictio-solemn-ignem.gtex}

\vspace{7mm}

\lectiovi

\noindent \Vbardot{} Tu autem, Dómine, miserére nobis.
\noindent \Rbardot{} Deo grátias.

\vfill
\pagebreak

\responsoriumvi

\vfill
\pagebreak

\subhora{In III. Nocturno}

\pars{Psalmus 7.} \scriptura{Ps. 9, 22}

\vspace{-4mm}

\antiphona{II D}{temporalia/ant-utquiddomine.gtex}

\vspace{-4mm}

\scriptura{Ps. 9, 22-32}

%\vspace{-2mm}

\initiumpsalmi{temporalia/ps9xxii_xxxii-initium-ii-D-auto.gtex}

%\psalmusEtTranslatioT{temporalia/ps9xxii_xxxii-I-comb.tex}{10cm}
\input{temporalia/ps9xxii_xxxii-I.tex} \Abardot{}

\vfill
\pagebreak

\pars{Psalmus 8.}\scriptura{Ex. 15, 18}

\vspace{-4mm}

\antiphona{IV* e}{temporalia/ant-inaeternum.gtex}

%\vspace{-4mm}

\scriptura{Ps. 9, 33-39}

\initiumpsalmi{temporalia/ps9xxxiii_xxxix-initium-iv_-e-auto.gtex}

%\psalmusEtTranslatioT{temporalia/ps9xxxiii_xxxix-I-comb.tex}{10cm}
\input{temporalia/ps9xxxiii_xxxix-I.tex} \Abardot{}

\vfill
\pagebreak

\pars{Psalmus 9.} \scriptura{Ps. 10, 8}

\vspace{-4mm}

\antiphona{II* f}{temporalia/ant-justusdominus.gtex}

%\vspace{-4mm}

\scriptura{Ps. 10}

%\initiumpsalmi{temporalia/ps10-initium-iv-c-auto.gtex}
\initiumpsalmi{temporalia/ps10-initium-ii_-f.gtex}

%\psalmusEtTranslatioT{temporalia/ps10-I-comb.tex}{10cm}
\input{temporalia/ps10-I.tex} \Abardot{}

\vfill
\pagebreak

\pars{Versus.} \scriptura{Ps. 118, 148}

% Versus. %%%
\sineinitiali{temporalia/versus-praevenerunt.gtex}

\vspace{5mm}

\sineinitiali{temporalia/oratiodominica-mat.gtex}

\vspace{5mm}

\pars{Absolutio.}

\cuminitiali{}{temporalia/absolutio-avinculis.gtex}

\vfill
\pagebreak

\cuminitiali{}{temporalia/benedictio-solemn-evangelica.gtex}

\vspace{7mm}

\lectiovii

\noindent \Vbardot{} Tu autem, Dómine, miserére nobis.
\noindent \Rbardot{} Deo grátias.

\vfill
\pagebreak

\responsoriumvii

\vfill
\pagebreak

\cuminitiali{}{temporalia/benedictio-solemn-divinum.gtex}

\vspace{7mm}

\lectioviii

\noindent \Vbardot{} Tu autem, Dómine, miserére nobis.
\noindent \Rbardot{} Deo grátias.

\vfill
\pagebreak

\responsoriumviii

\vfill
\pagebreak

\cuminitiali{}{temporalia/benedictio-solemn-adsocietatem.gtex}

\vspace{7mm}

\lectioix

\noindent \Vbardot{} Tu autem, Dómine, miserére nobis.
\noindent \Rbardot{} Deo grátias.

\vfill
\pagebreak

% Te Deum

{
\pars{Hymnus Ambrosianus} \scriptura{Tonus Solemnis}

\vspace{-2mm}

\grechangedim{interwordspacetext}{0.26 cm plus 0.15 cm minus 0.05 cm}{scalable}%
\cuminitiali{III}{temporalia/tedeum-solemnis-gn.gtex}
\grechangedim{interwordspacetext}{0.22 cm plus 0.15 cm minus 0.05 cm}{scalable}%
}

\vfill
\pagebreak

\rubrica{Reliqua omittuntur, nisi Laudes separandæ sint.}

\pars{Oratio}

\noindent \Vbardot{} Dómine, exáudi oratiónem meam.

\noindent \Rbardot{} Et clamor meus ad te véniat.

Orémus:

\oratioLaudes

\vspace{7mm}

\pars{Conclusio}

\noindent \Vbardot{} Dómine, exáudi oratiónem meam.

\noindent \Rbardot{} Et clamor meus ad te véniat.

\noindent \Vbardot{} Benedicámus Dómino, allelúia, allelúia.

\noindent \Rbardot{} Deo grátias, allelúia, allelúia.

\noindent \Vbardot{} Fidélium ánimæ per misericórdiam Dei requiéscant in pace.

\noindent \Rbardot{} Amen.

\vfill
\pagebreak

\hora{Ad Laudes.} %%%%%%%%%%%%%%%%%%%%%%%%%%%%%%%%%%%%%%%%%%%%%%%%%%%%%
%\sideThumbs{Laudes}

\cantusSineNeumas

\vspace{0.5cm}
\grechangedim{interwordspacetext}{0.18 cm plus 0.15 cm minus 0.05 cm}{scalable}%
\cuminitiali{}{temporalia/deusinadiutorium-alter.gtex}
\grechangedim{interwordspacetext}{0.22 cm plus 0.15 cm minus 0.05 cm}{scalable}%

\vfill
%\pagebreak

\pars{Psalmus 1.}

\vspace{-4mm}

\antiphona{VI F}{temporalia/ant-alleluia1.gtex}

\scriptura{Psalmus 50.}

\initiumpsalmi{temporalia/ps50-initium-vi-F-auto.gtex}

%\psalmusEtTranslatioT{temporalia/ps50-I-comb.tex}{10cm}
\input{temporalia/ps50-I.tex}

\vfill
\pagebreak

\pars{Psalmus 2.}

\scriptura{Psalmus 117.}

\initiumpsalmi{temporalia/ps117-initium-vi-F-auto.gtex}

%\psalmusEtTranslatioT{temporalia/ps117-I-comb.tex}{10cm}
\input{temporalia/ps117-I.tex}

\vfill
\pagebreak

\pars{Psalmus 3.}

\scriptura{Psalmus 62.}

\initiumpsalmi{temporalia/ps62-initium-vi-F-auto.gtex}

%\psalmusEtTranslatioT{temporalia/ps62-I-comb.tex}{10cm}
\input{temporalia/ps62-I.tex}

\vfill

\vspace{-6mm}

\antiphona{}{temporalia/ant-alleluia1.gtex} % repeat the antiphon - new page

\vfill
\pagebreak

\pars{Psalmus 4.} \scriptura{Dan. 3, 22-26; \textbf{H422}}

\vspace{-4mm}

\antiphona{VIII G}{temporalia/ant-trespueri.gtex}

\scriptura{Canticum trium puerorum, Dan. 3, 57-88 et 56}

\initiumpsalmi{temporalia/dan3-initium-viii-G-auto.gtex}

%\psalmusEtTranslatioT{temporalia/dan3-comb.tex}{10cm}
\input{temporalia/dan3.tex}

\rubrica{Hic non dicitur Gloria Patri, neque Amen.}

\vfill

\vspace{-6mm}

\antiphona{}{temporalia/ant-trespueri.gtex} % repeat the antiphon - new page

\vfill
\pagebreak

\pars{Psalmus 5.}

\vspace{-4mm}

\antiphona{VIII G}{temporalia/ant-alleluia2.gtex}

\scriptura{Psalmus 148.}

\initiumpsalmi{temporalia/ps148-initium-viii-G-auto.gtex}

%\psalmusEtTranslatioT{temporalia/ps148-I-comb.tex}{10cm}
\input{temporalia/ps148-I.tex}

\rubrica{Hic non dicitur Gloria Patri.}

\vfill
\pagebreak

%
\scriptura{Psalmus 149.}

\initiumpsalmi{temporalia/ps149-initium-viii-G-auto.gtex}

%\psalmusEtTranslatioT{temporalia/ps149-I-comb.tex}{10cm}
\input{temporalia/ps149-I.tex}

\rubrica{Hic non dicitur Gloria Patri.}

\vfill
\pagebreak

%
\scriptura{Psalmus 150.}

\initiumpsalmi{temporalia/ps150-initium-viii-G-auto.gtex}

%\psalmusEtTranslatioT{temporalia/ps150-I-comb.tex}{10cm}
\input{temporalia/ps150-I.tex}

\vfill

\vspace{-6mm}

\antiphona{}{temporalia/ant-alleluia2.gtex} % repeat the antiphon - new page

\vfill
\pagebreak

\pars{Capitulum.} \scriptura{Ac. 7, 12}

\grechangedim{interwordspacetext}{0.12 cm plus 0.15 cm minus 0.05 cm}{scalable}%
\cuminitiali{}{temporalia/capitulum-Benedictio.gtex}
\grechangedim{interwordspacetext}{0.22 cm plus 0.15 cm minus 0.05 cm}{scalable}

% preklad Jeruz. bible
%\trCapituliI

\vfill

\pars{Responsorium breve.} \scriptura{Ps. 118, 36-37}

\cuminitiali{IV}{temporalia/resp-inclinacormeum.gtex}

%\trResp

\vfill
\pagebreak

\pars{Hymnus} \scriptura{Gregorius Magnus (\olddag{} 604)}

\cuminitiali{IV}{temporalia/hym-EcceJamNoctis.gtex}
\vspace{-3mm}
%\input{hym-EcceJamNocis-bohtext.tex}

\vfill
%\pagebreak

\pars{Versus.} \scriptura{Ps. 92, 1}

% Versus. %%%
\sineinitiali{temporalia/versus-dominusregnavit.gtex}

%\noindent \trVersus

\vfill
\pagebreak

\benedictus

\vspace{-1cm}

\vfill
\pagebreak

%\sideThumbs{{\scriptsize{}Fine horarum}}

\anteOrationem

\pagebreak

% Oratio. %%%
\oratioLaudes

\vspace{-1mm}
%\trOrationisI

\vfill

\rubrica{Hebdomadarius dicit iterum Dominus vobiscum, vel cantor dicit:}

\vspace{2mm}

\sineinitiali{temporalia/domineexaudi.gtex}

\rubrica{Postea cantatur a cantore:}

\vspace{2mm}

\cuminitiali{I}{temporalia/benedicamus-dominica-perannum.gtex}

\vspace{1mm}

\vfill
\pagebreak

\hora{Ad II. Vesperas.} %%%%%%%%%%%%%%%%%%%%%%%%%%%%%%%%%%%%%%%%%%%%%%%%%%%%%
%\sideThumbs{II. Vesperæ}

\cantusSineNeumas

%\vspace{0.5cm}
\grechangedim{interwordspacetext}{0.18 cm plus 0.15 cm minus 0.05 cm}{scalable}%
\cuminitiali{}{temporalia/deusinadiutorium-solemnis.gtex}
\grechangedim{interwordspacetext}{0.22 cm plus 0.15 cm minus 0.05 cm}{scalable}%

\vfill
%\pagebreak

\vspace{-2mm}

\pars{Psalmus 1.} \scriptura{Ps. 109, 1; \textbf{H91}}

\vspace{-4mm}

\antiphona{VII c\textsuperscript{2}}{temporalia/ant-dixitdominus.gtex}

\vspace{-4mm}

\scriptura{Psalmus 109.}

\initiumpsalmi{temporalia/ps109-initium-vii-c2-auto.gtex}

%\psalmusEtTranslatioT{temporalia/ps109-I-comb.tex}{10cm}
\input{temporalia/ps109-I.tex} \Abardot{}

\vspace{-1cm}

\vfill
\pagebreak

\pars{Psalmus 2.} \scriptura{Ps. 110, 8; \textbf{H91}}

\vspace{-4mm}

\antiphona{IV g}{temporalia/ant-fideliaomnia.gtex}

\scriptura{Psalmus 110.}

\initiumpsalmi{temporalia/ps110-initium-iv-g-auto.gtex}

%\psalmusEtTranslatioT{temporalia/ps110-I-comb.tex}{10cm}
\input{temporalia/ps110-I.tex} \Abardot{}

\vfill
\pagebreak

\pars{Psalmus 3.} \scriptura{Ps. 111, 1; \textbf{H92}}

\vspace{-4mm}

\antiphona{IV a}{temporalia/ant-inmandatis.gtex}

\scriptura{Psalmus 111.}

\initiumpsalmi{temporalia/ps111-initium-iv-a-auto.gtex}

%\psalmusEtTranslatioT{temporalia/ps111-I-comb.tex}{10cm}
\input{temporalia/ps111-I.tex} \Abardot{}

\vfill
\pagebreak

\pars{Psalmus 4.} \scriptura{Ps. 112, 2; \textbf{H92}}

\vspace{-4mm}

\antiphona{VII c}{temporalia/ant-sitnomendomini.gtex}

\scriptura{Psalmus 112.}

\initiumpsalmi{temporalia/ps112-initium-vii-c-auto.gtex}

%\psalmusEtTranslatioT{temporalia/ps112-I-comb.tex}{10cm}
\input{temporalia/ps112-I.tex} \Abardot{}

\vfill
\pagebreak

\pars{Capitulum.} \scriptura{2 Cor. 1, 3-4}

\grechangedim{interwordspacetext}{0.12 cm plus 0.15 cm minus 0.05 cm}{scalable}%
\cuminitiali{}{temporalia/capitulum-BenedictusDeus.gtex}
\grechangedim{interwordspacetext}{0.22 cm plus 0.15 cm minus 0.05 cm}{scalable}

% preklad Jeruz. bible
%\trCapituliI

\vfill

\pars{Responsorium breve.} \scriptura{Ps. 103, 24}

\cuminitiali{VI}{temporalia/resp-quammagnificata.gtex}

%\trResp

\vfill
\pagebreak

\pars{Hymnus} \scriptura{Gregorius Magnus (\olddag{} 604)}

\cuminitiali{I}{temporalia/hym-LucisCreator-aestivalis.gtex}
\vspace{-3mm}
%\begin{translatioMulticol}{3}
Tvůrce světa předobrý,\\
tys ustanovil denní řád\\
a proudy světla rozhodil,\\
když světu základy jsi klad.\\
\\
A spojils ráno s večerem\\
a dnem tu dobu nazýváš;\\
hle padá temné noci stín -\\
slyš prosbu, vyslyš nářek náš.\columnbreak

Ach, nedej, by nás stihla smrt,\\
když svědomí nám tíží hřích,\\
když nemyslíme na věčnost\\
v té síti hříchů šalebných.\\
\\
Vzbuď naši touhu po nebi,\\
kde věčný život čeká nás,\\
a pomoz odložit vše zlé\\
a smýti z duše každý kaz.\columnbreak

To splň nám, dobrý Otče náš,\\
i ty, jenž rovné božství máš,\\
i Duchu, který těšíš nás\\
a vládneš, Bože, v každý čas.\\
Amen. 
\end{translatioMulticol}


\vfill
%\pagebreak

\pars{Versus.} \scriptura{Ps. 140, 2}

% Versus. %%%
\sineinitiali{temporalia/versus-dirigatur.gtex}

%\noindent \trVersus

\vfill
\pagebreak

\magnificatii

\vfill
\pagebreak

%\sideThumbs{{\scriptsize{}Fine horarum}}

\anteOrationem

\pagebreak

% Oratio. %%%
\oratioLaudes

\vspace{-1mm}
%\trOrationisI

\vfill

\rubrica{Hebdomadarius dicit iterum Dominus vobiscum, vel cantor dicit:}

\vspace{2mm}

\sineinitiali{temporalia/domineexaudi.gtex}

\rubrica{Postea cantatur a cantore:}

\vspace{2mm}

\cuminitiali{I}{temporalia/benedicamus-dominica-perannum.gtex}

\vspace{1mm}

\end{document}

