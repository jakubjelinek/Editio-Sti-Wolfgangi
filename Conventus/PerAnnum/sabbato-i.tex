\newcommand{\titulus}{\nomenFesti{S. Hilarii, Episcopi \& Ecclesiæ Doctoris.}
\dies{Die 13. Ianuarii.}}
\newcommand{\sineobmv}{Sine officium B.M.V.}
\newcommand{\oratio}{\pars{Oratio.}

\noindent Præsta, quǽsumus, omnípotens Deus, ut divinitátem Fílii tui, quam beátus Hilárius, epíscopus, constánter asséruit, et conveniénter intellégere valeámus et veráciter profitéri.

\pars{Pro pace in universo mundo.} \scriptura{Sir. 50, 25; 2 Esdr. 4, 20; \textbf{H416}}

\vspace{-4mm}

\antiphona{II D}{temporalia/ant-dapacemdomine.gtex}

\vfill

\noindent Deus, a quo sancta desidéria, recta consília et iusta sunt ópera: da servis tuis illam, quam mundus dare non potest, pacem; ut et corda nostra mandátis tuis dédita, et hóstium subláta formídine, témpora sint tua protectióne tranquílla.

\noindent Per Dóminum nostrum Iesum Christum, Fílium tuum, qui tecum vivit et regnat in unitáte Spíritus Sancti, Deus, per ómnia sǽcula sæculórum.

\noindent \Rbardot{} Amen.}
\newcommand{\invitatorium}{\pars{Invitatorium.}

\vspace{-4mm}

\antiphona{IV}{temporalia/inv-fontemsapientiae.gtex}}
\newcommand{\hymnusmatutinum}{\pars{Hymnus}

\cuminitiali{IV}{temporalia/hym-AEterneSol.gtex}}
\newcommand{\lectioi}{\pars{Lectio I.} \scriptura{Sir. 44, 1b-2.16-24}

\noindent De libro Ecclesiástici.

\noindent Laudémus viros gloriósos et paréntes nostros in generatióne sua.

\noindent Multam glóriam fecit Dóminus, magnificéntiam suam a sǽculo.

\noindent Henoch plácuit Deo et translátus est in paradísum, ut det géntibus pæniténtiam.

\noindent Noe invéntus est perféctus iustus, et in témpore iracúndiæ factus est reconciliátio; propter eum dimíssum est réliquum terræ, cum factum est dilúvium: testaménta sǽculi pósita sunt apud illum, ne deléri posset dilúvio omnis caro.

\noindent Abraham magnus pater multitúdinis géntium, et non est invénta mácula in glória eius; qui conservávit legem Excélsi et fuit in testaménto cum illo.

\noindent In carne eius stare fecit testaméntum, et in tentatióne invéntus est fidélis.

\noindent Ideo iure iurándo státuit illi benedíci gentes in sémine eius, créscere illum quasi terræ cúmulum et ut stellas exaltáre semen eius et hereditáre illos a mari usque ad mare et a Flúmine usque ad términos terræ.}

\newcommand{\responsoriumi}{\pars{Responsorium 1.} \scriptura{\Rbardot{} Is. 19, 25 \Vbardot{} Ps. 32, 12; \textbf{H418}}

\vspace{-5mm}

\responsorium{VIII}{temporalia/resp-laudabilispopulus-CROCHU.gtex}{}}
\newcommand{\lectioii}{\pars{Lectio II.} \scriptura{Sir. 44, 25-27; 45, 1-6}

\noindent Et in Isaac eódem modo státuit propter Abraham patrem eius.

\noindent Benedictiónem ómnium géntium dedit illi Dóminus et testaméntum confirmávit super caput Iacob.

\noindent Agnóvit eum in benedictiónibus suis et dedit illi hereditátem et divísit illi partem in tríbubus duódecim.

\noindent Et edúxit ex illo hóminem misericórdiæ inveniéntem grátiam in óculis omnis carnis, diléctum a Deo et homínibus Móysen, cuius memória in benedictióne est.

\noindent Símilem illum fecit in glória sanctórum et magnificávit eum in timóre inimicórum et in verbis suis signa accelerávit.

\noindent Glorificávit illum in conspéctu regum et ius dedit illi ad pópulum suum et osténdit illi glóriam suam.

\noindent In fide et lenitáte ipsíus sanctum fecit illum et elégit eum ex omni carne.

\noindent Audítam fecit illi vocem suam et indúxit illum in nubem; et dedit illi coram præcépta et legem vitæ et disciplínæ, docére Iacob testaméntum suum et iudícia sua Israel.}
\newcommand{\responsoriumii}{\pars{Responsorium 2.} \scriptura{\Rbardot{} Dt. 4, 1; 27, 3 \Vbardot{} Ps. 80, 9-10; \textbf{H161}}

\vspace{-5mm}

\responsorium{II}{temporalia/resp-audiisraelpraecepta-CROCHU.gtex}{}}
\newcommand{\lectioiii}{\pars{Lectio III.} \scriptura{Lib. 1, 37-38: PL 10, 48-49}

\noindent Ex Tractátu sancti Hilárii epíscopi \emph{De Trinitáte.}

\noindent Ego quidem hoc vel præcípuum vitæ meæ offícium debére me tibi, Pater omnípotens Deus, cónscius sum, ut te omnis sermo meus et sensus loquátur.

\noindent Neque enim ullum áliud maius prǽmium hic ipse usus mihi a te concéssus loquéndi potest reférre, quam ut prædicándo te tibi sérviat, teque quod es, Patrem, Patrem scílicet unigéniti Dei, aut ignoránti sǽculo, aut negánti hærético demónstret.

\noindent Et in hoc quidem tantum voluntátis meæ proféssio est, céterum auxílii et misericórdiæ tuæ munus orándum est, ut exténsa tibi fídei nostræ confessionísque vela flatu Spíritus tui ímpleas, nosque in cursum prædicatiónis ínitæ propéllas. Non enim nobis infidélis sponsiónis istíus auctor est, dicens: \emph{Pétite, et dábitur vobis; quǽrite, et inveniétis; pulsáte, et aperiétur vobis.}

\noindent Nos quidem ínopes ea quibus egémus precábimur, et in scrutándis prophetárum tuórum Apostolorúmque dictis stúdium pérvicax afferémus, et omnes observátæ intellegéntiæ áditus pulsábimus; sed tuum est, et orátum tribúere, et quæsítum adésse, et patére pulsátum.

\noindent Torpémus enim quodam natúræ nostræ pigro stupóre, et ad res tuas intellegéndas intra ignorántiæ necessitátem ingénii nostri imbecillitáte cohibémur; sed doctrínæ tuæ stúdia ad sensum nos divínæ cognitiónis instítuunt, et ultra naturálem opiniónem fídei obœdiéntia próvehit.

\noindent Exspectámus ergo ut trépidi huius cœpti exórdia íncites, et proféctu accrescénte confírmes, et ad consórtium vel prophetális vel apostólici spíritus voces: ut dicta eórum non álio quam ipsi locúti sunt sensu apprehendámus, verborúmque proprietátes iísdem rerum significatiónibus exsequámur.

\noindent Locutúri enim sumus quæ ab iis in sacraménto prædicáta sunt: te ætérnum Deum, ætérni unigéniti Dei Patrem; et unum te sine nativitáte, et unum Dóminum Iesum Christum ex te nativitátis ætérnæ, non in deórum númerum veritátis diversitáte referéndum, neque non ex te génitum, qui Deus unus es, prædicándum, neque áliud quam Deum verum, qui ex te Deo vero Patre natus est, confiténdum.

\noindent Tríbue ergo nobis verbórum significatiónem, intellegéntiæ lumen, dictórum honórem, veritátis fidem; et præsta, ut quod crédimus, et loquámur, scílicet ut contíngat nobis, unum te Deum Patrem et unum Dóminum Iesum Christum de prophétis atque Apóstolis cognoscéntibus, nunc advérsum negántes hæréticos, ita Deum et te celebráre, ne solum, et eum prædicáre, ne falsum.}
\newcommand{\responsoriumiii}{\pars{Responsorium 3.} \scriptura{\textbf{H378}}

\vspace{-5mm}

\responsorium{VIII}{temporalia/resp-isteestquiantedeum-CROCHU-cumdox.gtex}{}

\rubrica{vel ad libitum:}

\vspace{3mm}

\pars{Responsorium 3.} \scriptura{\Rbardot{} Sir. 15, 5 \Vbardot{} Ier. 1, 9; \textbf{H63}}

\vspace{-5mm}

\responsorium{I}{temporalia/resp-inmedioecclesiae-CROCHU-cumdox.gtex}{}}
\newcommand{\hymnuslaudes}{\pars{Hymnus.}

\cuminitiali{VIII}{temporalia/hym-DoctorAEternus.gtex}}
\newcommand{\lectiobrevis}{\pars{Lectio Brevis.} \scriptura{Sap. 7, 13-14}

\noindent Sapiéntiam sine fictióne dídici et sine invídia commúnico; divítias illíus non abscóndo. Infinítus enim thesáurus est homínibus; quem qui acquisiérunt, ad amicítiam in Deum se paravérunt propter disciplínæ dona commendáti.}
\newcommand{\responsoriumbreve}{\pars{Responsorium breve.} \scriptura{Eccli. 44, 15}

\cuminitiali{VI}{temporalia/resp-sapientiamsanctorum.gtex}}
\newcommand{\benedictus}{\pars{Canticum Zachariæ.} \scriptura{Mt. 10, 20}

\vspace{-4mm}

\antiphona{VIII G}{temporalia/ant-nonenimvosestis.gtex}

%\vspace{-3mm}

\scriptura{Lc. 1, 68-79}

%\vspace{-2mm}

\cantusSineNeumas
\initiumpsalmi{temporalia/benedictus-initium-viii-g-auto.gtex}

%\vspace{-1.5mm}

\input{temporalia/benedictus-viii-g.tex} \Abardot{}}
\newcommand{\preces}{\noindent Christo, bono pastóri,~\gredagger{} qui pro suis óvibus ánimam pósuit,~\grestar{} laudes grati exsolvámus et supplicémus, dicéntes:

\Rbardot{} Pasce pópulum tuum, Dómine.

\noindent Christe, qui in sanctis pastóribus misericórdiam et dilectiónem tuam dignátus es osténdere,~\grestar{} numquam désinas per eos nobíscum misericórditer ágere.

\Rbardot{} Pasce pópulum tuum, Dómine.

\noindent Qui múnere pastóris animárum fungi per tuos vicários pergis,~\grestar{} ne destíteris nos ipse per rectóres nostros dirígere.

\Rbardot{} Pasce pópulum tuum, Dómine.

\noindent Qui in sanctis tuis, populórum dúcibus, córporum animarúmque médicus exstitísti,~\grestar{} numquam cesses ministérium in nos vitæ et sanctitátis perágere.

\Rbardot{} Pasce pópulum tuum, Dómine.

\noindent Qui, prudéntia et caritáte sanctórum, tuum gregem erudísti,~\grestar{} nos in sanctitáte iúgiter per pastóres nostros ædífica.

\Rbardot{} Pasce pópulum tuum, Dómine.}
\newcommand{\hebdomada}{I}
\newcommand{\oratioMatutinum}{\noindent Præsta, quǽsumus, omnípotens Deus: \gredagger{} ut qui paschália festa perégimus, \grestar{} hæc, te largiénte, móribus et vita teneámus. Per Dóminum.}
\newcommand{\oratioLaudes}{\cuminitiali{}{temporalia/oratio.gtex}}


% LuaLaTeX

\documentclass[a4paper, twoside, 12pt]{article}
\usepackage[latin]{babel}
%\usepackage[landscape, left=3cm, right=1.5cm, top=2cm, bottom=1cm]{geometry} % okraje stranky
%\usepackage[landscape, a4paper, mag=1166, truedimen, left=2cm, right=1.5cm, top=1.6cm, bottom=0.95cm]{geometry} % okraje stranky
\usepackage[landscape, a4paper, mag=1400, truedimen, left=0.5cm, right=0.5cm, top=0.5cm, bottom=0.5cm]{geometry} % okraje stranky

\usepackage{fontspec}
\setmainfont[FeatureFile={junicode.fea}, Ligatures={Common, TeX}, RawFeature=+fixi]{Junicode}
%\setmainfont{Junicode}

% shortcut for Junicode without ligatures (for the Czech texts)
\newfontfamily\nlfont[FeatureFile={junicode.fea}, Ligatures={Common, TeX}, RawFeature=+fixi]{Junicode}

% Hebrew font: http://scripts.sil.org/cms/scripts/page.php?site_id=nrsi&id=SILHebrUnic2
\newfontfamily\hebfont[Scale=1]{Ezra SIL}

\usepackage{multicol}
\usepackage{color}
\usepackage{lettrine}
\usepackage{fancyhdr}

% usual packages loading:
\usepackage{luatextra}
\usepackage{graphicx} % support the \includegraphics command and options
\usepackage{gregoriotex} % for gregorio score inclusion
\usepackage{gregoriosyms}
\usepackage{wrapfig} % figures wrapped by the text
\usepackage{parcolumns}
\usepackage[contents={},opacity=1,scale=1,color=black]{background}
\usepackage{tikzpagenodes}
\usepackage{calc}
\usepackage{longtable}
\usetikzlibrary{calc}

\setlength{\headheight}{14.5pt}

\input{conventuscommune.tex} % Often used macros

\newcommand{\annusEditionis}{2022}

\def\hebinitial#1{%
\leavevmode{\newbox\hebbox\setbox\hebbox\hbox{\hebfont{#1}\hskip 1mm}\kern -\wd\hebbox\hbox{\hebfont{#1}\hskip 1mm}}%
}

%%%% Vicekrat opakovane kousky

\newcommand{\anteOrationem}{
  \rubrica{Ante Orationem, cantatur a Superiore:}

  \pars{Supplicatio Litaniæ.}

  \cuminitiali{}{temporalia/supplicatiolitaniae.gtex}

  \pars{Oratio Dominica.}

  \cuminitiali{}{temporalia/oratiodominica.gtex}

  \rubrica{Deinde dicitur ab Hebdomadario:}

  \cuminitiali{}{temporalia/dominusvobiscum-solemnis.gtex}

  \rubrica{In choro monialium loco Dominus vobiscum dicitur:}

  \sineinitiali{temporalia/domineexaudi.gtex}
}

\setlength{\columnsep}{30pt} % prostor mezi sloupci

%%%%%%%%%%%%%%%%%%%%%%%%%%%%%%%%%%%%%%%%%%%%%%%%%%%%%%%%%%%%%%%%%%%%%%%%%%%%%%%%%%%%%%%%%%%%%%%%%%%%%%%%%%%%%
\begin{document}

% Here we set the space around the initial.
% Please report to http://home.gna.org/gregorio/gregoriotex/details for more details and options
\grechangedim{afterinitialshift}{2.2mm}{scalable}
\grechangedim{beforeinitialshift}{2.2mm}{scalable}
\grechangedim{interwordspacetext}{0.22 cm plus 0.15 cm minus 0.05 cm}{scalable}%
\grechangedim{annotationraise}{-0.2cm}{scalable}

% Here we set the initial font. Change 38 if you want a bigger initial.
% Emit the initials in red.
\grechangestyle{initial}{\color{red}\fontsize{38}{38}\selectfont}

\pagestyle{empty}

%%%% Titulni stranka
\begin{titulusOfficii}
\ifx\titulus\undefined
\nomenFesti{Sabbato \hebdomada{}}
\else
\titulus
\fi
\end{titulusOfficii}

\vfill

\begin{center}
%Ad usum et secundum consuetudines chori \guillemotright{}Conventus Choralis\guillemotleft.

%Editio Sancti Wolfgangi \annusEditionis
\end{center}

\scriptura{}

\pars{}

\pagebreak

\renewcommand{\headrulewidth}{0pt} % no horiz. rule at the header
\fancyhf{}
\pagestyle{fancy}

\cantusSineNeumas

\hora{Ad Matutinum.} %%%%%%%%%%%%%%%%%%%%%%%%%%%%%%%%%%%%%%%%%%%%%%%%%%%%%

\vspace{2mm}

\cuminitiali{}{temporalia/dominelabiamea.gtex}

\vfill
%\pagebreak

\vspace{2mm}

\ifx\invitatorium\undefined
\pars{Invitatorium.} \scriptura{Lc. 24, 34; Psalmus 94; \textbf{H232}}

\vspace{-4mm}

\antiphona{VI}{temporalia/inv-surrexitdominusvere.gtex}
\else
\invitatorium
\fi

\vfill
\pagebreak

\ifx\hymnusmatutinum\undefined
\pars{Hymnus.}

\cuminitiali{VIII}{temporalia/hym-LaetareCaelum.gtex}
\else
\hymnusmatutinum
\fi

\vspace{-3mm}

\vfill
\pagebreak

\ifx\matutinum\undefined
\ifx\matua\undefined
\else
% MAT A
\pars{Psalmus 1.}

\vspace{-4mm}

\antiphona{VIII G\textsuperscript{5}}{temporalia/ant-alleluia-turco15.gtex}

\vspace{-3mm}

\scriptura{Ps. 104, 1-15}

\vspace{-2mm}

\initiumpsalmi{temporalia/ps104i-initium-viii-g5.gtex}

\vspace{-1.5mm}

\input{temporalia/ps104i-viii-g.tex}

\vfill
\pagebreak

\pars{Psalmus 2.} \scriptura{Ps. 104, 16-27}

%\vspace{-2mm}

\initiumpsalmi{temporalia/ps104ii-initium-viii-g5.gtex}

\input{temporalia/ps104ii-viii-g.tex}

\vfill
\pagebreak

\pars{Psalmus 3.} \scriptura{Ps. 104, 28-45}

%\vspace{-2mm}

\initiumpsalmi{temporalia/ps104iii-initium-viii-g5.gtex}

\input{temporalia/ps104iii-viii-g.tex}

\vfill

\antiphona{}{temporalia/ant-alleluia-turco15.gtex}

\vfill
\pagebreak
\fi
\ifx\matub\undefined
\else
% MAT B
\pars{Psalmus 1.}

\vspace{-4mm}

\antiphona{t. pereg.}{temporalia/ant-alleluia-turco3.gtex}

%\vspace{-2mm}

\scriptura{Ps. 105, 1-15}

%\vspace{-2mm}

\initiumpsalmi{temporalia/ps105i-initium-per-auto.gtex}

\input{temporalia/ps105i-per.tex}

\vfill
\pagebreak

\pars{Psalmus 2.} \scriptura{Ps. 105, 16-31}

\vspace{-2.5mm}

\initiumpsalmi{temporalia/ps105ii-initium-per-auto.gtex}

\vspace{-1.5mm}

\input{temporalia/ps105ii-per.tex}

\vfill
\pagebreak

\pars{Psalmus 3.} \scriptura{Ps. 105, 32-48}

%\vspace{-2mm}

\initiumpsalmi{temporalia/ps105iii-initium-per-auto.gtex}

\input{temporalia/ps105iii-per.tex}

\vfill

\antiphona{}{temporalia/ant-alleluia-turco3.gtex}

\vfill
\pagebreak
\fi
\ifx\matuc\undefined
\else
% MAT C
\pars{Psalmus 1.} \scriptura{Ps. 106, 8}

\vspace{-4mm}

\antiphona{IV e}{temporalia/ant-alleluia-fo2.gtex}

%\vspace{-2mm}

\scriptura{Ps. 106, 1-14}

%\vspace{-2mm}

\initiumpsalmi{temporalia/ps106i-initium-iv-e2-auto.gtex}

\input{temporalia/ps106i-iv-e2.tex}

\vfill
\pagebreak

\pars{Psalmus 2.} \scriptura{Ps. 106, 15-30}

%\vspace{-2mm}

\initiumpsalmi{temporalia/ps106ii-initium-iv-e2-auto.gtex}

\input{temporalia/ps106ii-iv-e2.tex}

\vfill
\pagebreak

\pars{Psalmus 3.} \scriptura{Ps. 106, 31-43}

%\vspace{-2mm}

\initiumpsalmi{temporalia/ps106iii-initium-iv-e2-auto.gtex}

\input{temporalia/ps106iii-iv-e2.tex}

\vfill
\pagebreak

\antiphona{}{temporalia/ant-alleluia-fo2.gtex}

\vfill
\pagebreak
\fi
\ifx\matud\undefined
\else
% MAT D
\pars{Psalmus 1.}

\vspace{-4mm}

\antiphona{III g}{temporalia/ant-alleluia-turco26.gtex}

%\vspace{-2mm}

\scriptura{Ps. 77, 40-51}

%\vspace{-2mm}

\initiumpsalmi{temporalia/ps77xl_li-initium-iii-g-auto.gtex}

\input{temporalia/ps77xl_li-iii-g.tex}

\vfill
\pagebreak

\pars{Psalmus 2.} \scriptura{Ps. 77, 52-64}

\vspace{-2mm}

\initiumpsalmi{temporalia/ps77lii_lxiv-initium-iii-g-auto.gtex}

\input{temporalia/ps77lii_lxiv-iii-g.tex}

\vfill
\pagebreak

\pars{Psalmus 3.} \scriptura{Ps. 77, 65-72}

%\vspace{-2mm}

\initiumpsalmi{temporalia/ps77lxv_lxxii-initium-iii-g-auto.gtex}

\input{temporalia/ps77lxv_lxxii-iii-g.tex}

\vfill

\antiphona{}{temporalia/ant-alleluia-turco26.gtex}

\vfill
\pagebreak
\fi
\else
\matutinum
\fi

\pars{Versus.}

\ifx\matversus\undefined
\noindent \Vbardot{} Deus regenerávit nos in spem vivam, allelúia.

\noindent \Rbardot{} Per resurrectiónem Iesu Christi ex mórtuis, allelúia.
\else
\matversus
\fi

\vspace{5mm}

\sineinitiali{temporalia/oratiodominica-mat.gtex}

\vspace{5mm}

\pars{Absolutio.}

\cuminitiali{}{temporalia/absolutio-avinculis.gtex}

\vfill
\pagebreak

\cuminitiali{}{temporalia/benedictio-solemn-ille.gtex}

\vspace{7mm}

\lectioi

\noindent \Vbardot{} Tu autem, Dómine, miserére nobis.
\noindent \Rbardot{} Deo grátias.

\vfill
\pagebreak

\responsoriumi

\vfill
\pagebreak

\cuminitiali{}{temporalia/benedictio-solemn-divinum.gtex}

\vspace{7mm}

\lectioii

\noindent \Vbardot{} Tu autem, Dómine, miserére nobis.
\noindent \Rbardot{} Deo grátias.

\vfill
\pagebreak

\responsoriumii

\vfill
\pagebreak

\cuminitiali{}{temporalia/benedictio-solemn-adsocietatem.gtex}

\vspace{7mm}

\lectioiii

\noindent \Vbardot{} Tu autem, Dómine, miserére nobis.
\noindent \Rbardot{} Deo grátias.

\vfill
\pagebreak

\responsoriumiii

\vfill
\pagebreak

\rubrica{Reliqua omittuntur, nisi Laudes separandæ sint.}

\sineinitiali{temporalia/domineexaudi.gtex}

\vfill

\oratio

\vfill

\noindent \Vbardot{} Dómine, exáudi oratiónem meam.
\Rbardot{} Et clamor meus ad te véniat.

\vfill

\noindent \Vbardot{} Benedicámus Dómino.
\noindent \Rbardot{} Deo grátias.

\vfill

\noindent \Vbardot{} Fidélium ánimæ per misericórdiam Dei requiéscant in pace.
\Rbardot{} Amen.

\vfill
\pagebreak

\hora{Ad Laudes.} %%%%%%%%%%%%%%%%%%%%%%%%%%%%%%%%%%%%%%%%%%%%%%%%%%%%%

\cantusSineNeumas

\vspace{0.5cm}
\grechangedim{interwordspacetext}{0.18 cm plus 0.15 cm minus 0.05 cm}{scalable}%
\cuminitiali{}{temporalia/deusinadiutorium-communis.gtex}
\grechangedim{interwordspacetext}{0.22 cm plus 0.15 cm minus 0.05 cm}{scalable}%

\vfill
\pagebreak

\ifx\hymnuslaudes\undefined
\ifx\laudac\undefined
\else
\pars{Hymnus}

\cuminitiali{I}{temporalia/hym-ChorusNovae-praglia.gtex}
\vspace{-3mm}
\fi
\ifx\laudbd\undefined
\else
\pars{Hymnus}

\cuminitiali{I}{temporalia/hym-ChorusNovae.gtex}
\vspace{-3mm}
\fi
\else
\hymnuslaudes
\fi

\vfill
\pagebreak

\ifx\laudes\undefined
\ifx\lauda\undefined
\else
\pars{Psalmus 1.}

\vspace{-4mm}

\antiphona{VII a}{temporalia/ant-alleluia-turco29.gtex}

\scriptura{Psalmus 118, 145-152; \hspace{5mm} \hebinitial{ק}}

\initiumpsalmi{temporalia/ps118xix-initium-vii-a-auto.gtex}

\input{temporalia/ps118xix-vii-a.tex} \Abardot{}

\vfill
\pagebreak

\pars{Psalmus 2.} \scriptura{Ex. 15, 2}

\vspace{-4mm}

\antiphona{IV e}{temporalia/ant-fortitudomeaetlausmea.gtex}

\scriptura{Canticum Moysis, Ex. 15, 1-4a.7b-13.17-19}

\initiumpsalmi{temporalia/moysis1-initium-iv-e2-auto.gtex}

\input{temporalia/moysis1-iv-e2.tex}

\antiphona{}{temporalia/ant-fortitudomeaetlausmea.gtex}

\vfill
\pagebreak

\pars{Psalmus 3.}

\vspace{-4mm}

\antiphona{E}{temporalia/ant-alleluia-praglia-e2.gtex}

\scriptura{Psalmus 116.}

\initiumpsalmi{temporalia/ps116-initium-e-auto.gtex}

\input{temporalia/ps116-e.tex} \Abardot{}

\vfill
\pagebreak
\fi
\ifx\laudb\undefined
\else
\pars{Psalmus 1.}

\vspace{-4.5mm}

\antiphona{E}{temporalia/ant-alleluia-praglia-e2.gtex}

\vspace{-3mm}

\scriptura{Psalmus 91.}

\vspace{-2mm}

\initiumpsalmi{temporalia/ps91-initium-e-auto.gtex}

\vspace{-1.5mm}

\input{temporalia/ps91-e.tex} \Abardot{}

\vfill
\pagebreak

\pars{Psalmus 2.} \scriptura{Eccli. 39, 19}

\vspace{-4mm}

\antiphona{VII c\textsuperscript{2}}{temporalia/ant-effrondeteingratia.gtex}

\vspace{-2mm}

\scriptura{Canticum Moysi, Dt. 32, 1-32}

\vspace{-2mm}

\initiumpsalmi{temporalia/moysis2i_xii-initium-vii-c2-auto.gtex}

\input{temporalia/moysis2i_xii-vii-c2.tex}

\vfill

\antiphona{}{temporalia/ant-effrondeteingratia.gtex}

\vfill
\pagebreak

\pars{Psalmus 3.}

\vspace{-4mm}

\antiphona{I a\textsuperscript{2}}{temporalia/ant-alleluia-turco23.gtex}

%\vspace{-2mm}

\scriptura{Ps. 8}

%\vspace{-2mm}

\initiumpsalmi{temporalia/ps8-initium-i-a2-auto.gtex}

\input{temporalia/ps8-i-a2.tex} \Abardot{}

\vfill
\pagebreak
\fi
\ifx\laudc\undefined
\else
\pars{Psalmus 1.}

\vspace{-4mm}

\antiphona{E}{temporalia/ant-alleluia-praglia-e2.gtex}

%\vspace{-2mm}

\scriptura{Psalmus 118, 145-152.}

%\vspace{-2mm}

\initiumpsalmi{temporalia/ps118xix-initium-e-auto.gtex}

%\vspace{-1.5mm}

\input{temporalia/ps118xix-e.tex} \Abardot{}

\vfill
\pagebreak

\pars{Psalmus 2.}

\vspace{-4mm}

\antiphona{V a}{temporalia/ant-mecumsitdomine-tp.gtex}

%\vspace{-2mm}

\scriptura{Canticum Sapientiæ, Sap. 9, 1-6.9-11}

\initiumpsalmi{temporalia/sapientia-initium-v-a-auto.gtex}

\input{temporalia/sapientia-v-a.tex} \Abardot{}

\vfill
\pagebreak

\pars{Psalmus 3.}

\vspace{-4mm}

\antiphona{II* a}{temporalia/ant-alleluia-turco18.gtex}

%\vspace{-2mm}

\scriptura{Ps. 116}

%\vspace{-2mm}

\initiumpsalmi{temporalia/ps116-initium-ii_-a-auto.gtex}

\input{temporalia/ps116-ii_-a.tex} \Abardot{}

\vfill
\pagebreak
\fi
\ifx\laudd\undefined
\else
\pars{Psalmus 1.}

\vspace{-4.5mm}

\antiphona{VIII G\textsuperscript{2}}{temporalia/ant-alleluia-turco12.gtex}

\vspace{-3mm}

\scriptura{Psalmus 91.}

\vspace{-2mm}

\initiumpsalmi{temporalia/ps91-initium-viii-G5-auto.gtex}

\vspace{-1.5mm}

\input{temporalia/ps91-viii-G5.tex} \Abardot{}

\vfill
\pagebreak

\pars{Psalmus 2.} \scriptura{Heb. 13, 8}

\vspace{-4mm}

\antiphona{II D}{temporalia/ant-iesuschristusheriethodie.gtex}

%\vspace{-2mm}

\scriptura{Canticum Ezechiæ, Ez. 36, 24-28}

\initiumpsalmi{temporalia/ezechiae2-initium-ii-D-auto.gtex}

\input{temporalia/ezechiae2-ii-D.tex} \Abardot{}

\vfill
\pagebreak

\pars{Psalmus 3.}

\vspace{-4mm}

\antiphona{I a\textsuperscript{2}}{temporalia/ant-alleluia-turco23.gtex}

%\vspace{-2mm}

\scriptura{Ps. 8}

%\vspace{-2mm}

\initiumpsalmi{temporalia/ps8-initium-i-a4-auto.gtex}

\input{temporalia/ps8-i-a4.tex} \Abardot{}

\vfill
\pagebreak
\fi
\else
\laudes
\fi

\ifx\lectiobrevis\undefined
\pars{Lectio Brevis.} \scriptura{Rom. 14, 7-9}

\noindent Nemo nostrum sibi vivit et nemo sibi móritur; sive enim vívimus, Dómino vívimus, sive mórimur, Dómino mórimur. Sive ergo vívimus, sive mórimur, Dómini sumus. In hoc enim Christus et mórtuus est et vixit, ut et mortuórum et vivórum dominétur.
\else
\lectiobrevis
\fi

\vfill

\ifx\responsoriumbreve\undefined
\pars{Responsorium breve.} \scriptura{Cf. Mt. 28, 6; Cf. Gal. 3, 13}

\cuminitiali{VI}{temporalia/resp-surrexitdominusdesepulcro.gtex}
\else
\responsoriumbreve
\fi

\vfill
\pagebreak

\benedictus

\vspace{-1cm}

\vfill
\pagebreak

\ifx\precestotum\undefined
\pars{Preces.}

\sineinitiali{}{temporalia/tonusprecumnovum.gtex}

\ifx\preces\undefined
\ifx\lauda\undefined
\else
\noindent Christum, panem vitæ, \gredagger{} qui mensa verbi et córporis sui fruéntes suscitábit in novíssimo die, \grestar{} læti deprecémur:

\Rbardot{} Da nobis, Dómine, pacem et gáudium.

\noindent Fili Dei, qui, suscitátus a mórtuis, princeps es vitæ, \grestar{} nos omnésque fratres tuos bénedic et sanctífica.

\Rbardot{} Da nobis, Dómine, pacem et gáudium.

\noindent Tu, qui pacem et gáudium ómnibus in te credéntibus largíris, \grestar{} da nos sicut fílios lucis ambuláre et de victória tua lætári.

\Rbardot{} Da nobis, Dómine, pacem et gáudium.

\noindent Adáuge fidem Ecclésiæ peregrinántis in terra, \grestar{} ut resurrectiónis tuæ testimónium mundo perhíbeat.

\Rbardot{} Da nobis, Dómine, pacem et gáudium.

\noindent Tu qui, multa passus, \gredagger{} in glóriam Patris intrásti, \grestar{} luctum mæréntium convérte in gáudium.

\Rbardot{} Da nobis, Dómine, pacem et gáudium.
\fi
\ifx\laudb\undefined
\else
\noindent Christum, qui vitam ætérnam nobis manifestávit, \grestar{} devóta mente rogémus, clamántes:

\Rbardot{} Resurréctio tua locuplétet nos grátia, Dómine.

\noindent Pastor ætérne, \gredagger{} réspice gregem tuum e somno surgéntem \grestar{} et pasce nos verbi et panis tui ubérrimo alimónio.

\Rbardot{} Resurréctio tua locuplétet nos grátia, Dómine.

\noindent Ne permíttas nos a lupo rapi vel a mercenário perdi, \grestar{} sed fac, ut vocem tuam fidéliter audiámus.

\Rbardot{} Resurréctio tua locuplétet nos grátia, Dómine.

\noindent Tu, qui cum prædicatóribus ubíque cooperáris eorúmque sermónem confírmas, \grestar{} fac, ut hódie resurrectiónem tuam móribus et vita proclamémus.

\Rbardot{} Resurréctio tua locuplétet nos grátia, Dómine.

\noindent Esto ipse gáudium nostrum, \gredagger{} quod nemo tollat a nobis, \grestar{} ut, reiécta tristítia peccáti, vitam appetámus ætérnam.

\Rbardot{} Resurréctio tua locuplétet nos grátia, Dómine.
\fi
\ifx\laudc\undefined
\else
\noindent Christum, panem vitæ, \gredagger{} qui mensa verbi et córporis sui fruéntes suscitábit in novíssimo die, \grestar{} læti deprecémur:

\Rbardot{} Da nobis, Dómine, pacem et gáudium.

\noindent Fili Dei, qui, suscitátus a mórtuis, princeps es vitæ, \grestar{} nos omnésque fratres tuos bénedic et sanctífica.

\Rbardot{} Da nobis, Dómine, pacem et gáudium.

\noindent Tu, qui pacem et gáudium ómnibus in te credéntibus largíris, \grestar{} da nos sicut fílios lucis ambuláre et de victória tua lætári.

\Rbardot{} Da nobis, Dómine, pacem et gáudium.

\noindent Adáuge fidem Ecclésiæ peregrinántis in terra, \grestar{} ut resurrectiónis tuæ testimónium mundo perhíbeat.

\Rbardot{} Da nobis, Dómine, pacem et gáudium.

\noindent Tu qui, multa passus, \gredagger{} in glóriam Patris intrásti, \grestar{} luctum mæréntium convérte in gáudium.

\Rbardot{} Da nobis, Dómine, pacem et gáudium.
\fi
\ifx\laudd\undefined
\else
\noindent Christum, qui vitam ætérnam nobis manifestávit, \grestar{} devóta mente rogémus, clamántes:

\Rbardot{} Resurréctio tua locuplétet nos grátia, Dómine.

\noindent Pastor ætérne, \gredagger{} réspice gregem tuum e somno surgéntem \grestar{} et pasce nos verbi et panis tui ubérrimo alimónio.

\Rbardot{} Resurréctio tua locuplétet nos grátia, Dómine.

\noindent Ne permíttas nos a lupo rapi vel a mercenário perdi, \grestar{} sed fac, ut vocem tuam fidéliter audiámus.

\Rbardot{} Resurréctio tua locuplétet nos grátia, Dómine.

\noindent Tu, qui cum prædicatóribus ubíque cooperáris eorúmque sermónem confírmas, \grestar{} fac, ut hódie resurrectiónem tuam móribus et vita proclamémus.

\Rbardot{} Resurréctio tua locuplétet nos grátia, Dómine.

\noindent Esto ipse gáudium nostrum, \gredagger{} quod nemo tollat a nobis, \grestar{} ut, reiécta tristítia peccáti, vitam appetámus ætérnam.

\Rbardot{} Resurréctio tua locuplétet nos grátia, Dómine.
\fi
\else
\preces
\fi

\vfill

\pars{Oratio Dominica.}

\cuminitiali{}{temporalia/oratiodominicaalt.gtex}

\vfill
\pagebreak

\rubrica{vel:}

\pars{Deprecatio Gelasii}

\vspace{-5mm}

\grechangedim{interwordspacetext}{0.16 cm plus 0.15 cm minus 0.05 cm}{scalable}%
\antiphona{D\textsuperscript{1}}{temporalia/deprecatio4-propace.gtex}
\grechangedim{interwordspacetext}{0.22 cm plus 0.15 cm minus 0.05 cm}{scalable}%

\vfill

\pars{Oratio Dominica.}

\cuminitiali{D}{temporalia/oratiodominica-d.gtex}
\else
\precestotum
\fi

\vfill
\pagebreak

% Oratio. %%%
\oratio

\vspace{-1mm}

\vfill

\rubrica{Hebdomadarius dicit Dominus vobiscum, vel, absente sacerdote vel diacono, sic concluditur:}

\vspace{2mm}

\ifx\dominusnosbenedicat\undefined
\antiphona{C}{temporalia/dominusnosbenedicat.gtex}
\else
\dominusnosbenedicat
\fi

\rubrica{Postea cantatur a cantore:}

\vspace{2mm}

\cuminitiali{VII}{temporalia/benedicamus-tempore-paschali.gtex}

\vspace{1mm}

\vfill
\pagebreak

\end{document}

