\newcommand{\titulus}{\nomenFesti{Beatæ Mariæ Virginis Perdolentis.}
\dies{Die 15. Septembris.}}
\newcommand{\oratio}{\pars{Oratio.}

\noindent Deus, qui Fílio tuo in cruce exaltáto compatiéntem Matrem astáre voluísti, da Ecclésiæ tuæ, ut, Christi passiónis cum ipsa consors effécta, eiúsdem resurrectiónis párticeps esse mereátur.

\pars{Pro pace in universo mundo.} \scriptura{Sir. 50, 25; 2 Esdr. 4, 20; \textbf{H416}}

\vspace{-4mm}

\antiphona{II D}{temporalia/ant-dapacemdomine.gtex}

\vfill

\noindent Deus, a quo sancta desidéria, recta consília et iusta sunt ópera: da servis tuis illam, quam mundus dare non potest, pacem; ut et corda nostra mandátis tuis dédita, et hóstium subláta formídine, témpora sint tua protectióne tranquílla.

\noindent Per Dóminum nostrum Iesum Christum, Fílium tuum, qui tecum vivit et regnat in unitáte Spíritus Sancti, Deus, per ómnia sǽcula sæculórum.

\noindent \Rbardot{} Amen.}
\newcommand{\invitatorium}{\pars{Invitatorium.}

\vspace{-4mm}

\antiphona{IV}{temporalia/inv-salvatoremmundi.gtex}}
\newcommand{\hymnusmatutinum}{\pars{Hymnus.} \scriptura{Iacobus de Benedetti (\olddag{} 1306)}

\cuminitiali{II}{temporalia/hym-StabatMater.gtex}}
\newcommand{\matutinum}{\pars{Psalmus 1.} \scriptura{Lam. 1, 18; \textbf{H225}}

\vspace{-4mm}

\antiphona{VII d\textsuperscript{2}}{temporalia/ant-attenditeuniversi.gtex}

%\vspace{-2mm}

\scriptura{Ps. 23}

%\vspace{-2mm}

\initiumpsalmi{temporalia/ps23-initium-vii-d2-auto.gtex}

%\vspace{-1.5mm}

\input{temporalia/ps23-vii-d2.tex} \Abardot{}

\vfill
\pagebreak

\pars{Psalmus 2.} \scriptura{Ct. 1, 12}

\vspace{-4mm}

\antiphona{II D}{temporalia/ant-fasciculusmyrrhae.gtex}

\vspace{-2mm}

\scriptura{Ps. 45}

\vspace{-2mm}

\initiumpsalmi{temporalia/ps45-initium-ii-D-auto.gtex}

%\vspace{-1.5mm}

\input{temporalia/ps45-ii-D.tex} \Abardot{}

\vfill
\pagebreak

\pars{Psalmus 3.} \scriptura{Ct. 8, 7; \textbf{H300}}

\vspace{-4mm}

\antiphona{VIII C\textsuperscript{2}}{temporalia/ant-aquaemultae.gtex}

%\vspace{-2mm}

\scriptura{Ps. 86}

%\vspace{-2mm}

\initiumpsalmi{temporalia/ps86-initium-viii-C2-auto.gtex}

\input{temporalia/ps86-viii-C2.tex} \Abardot{}

\vfill
\pagebreak}
\newcommand{\matversus}{\noindent \Vbardot{} María conservábat ómnia verba hæc.

\noindent \Rbardot{} Cónferens in corde suo.}
\newcommand{\absolutio}{\cuminitiali{}{temporalia/absolutio-precibus.gtex}}
\newcommand{\benedictioi}{\cuminitiali{}{temporalia/benedictio-solemn-noscum.gtex}}
\newcommand{\benedictioii}{\cuminitiali{}{temporalia/benedictio-solemn-ipsavirgo.gtex}}
\newcommand{\benedictioiii}{\cuminitiali{}{temporalia/benedictio-solemn-pervirginem.gtex}}
\newcommand{\lectioi}{\pars{Lectio I.} \scriptura{Tob. 2, 1-5.7-10a}

\noindent De libro Thobis.

\noindent Et sub Asarháddon rege descéndi in domum meam, et réddita est mihi uxor mea Anna et fílius meus Thobías. In Pentecóste, die festo nostro, qui est sanctus a Septimánis, factum est mihi prándium bonum, et discúbui, ut prandérem. Et appósita est mihi mensa, et vidi pulmentária complúra.

\noindent Et dixi Thobíae fílio meo: “Vade et, quemcúmque páuperem invéneris ex frátribus nostris, qui sunt captívi in Ninéve, qui in mente habet Dóminum in toto corde suo, hunc adduc, et manducábit páriter mecum; ecce sustíneo te, fili, donec vénias”.

\noindent Et ábiit Thobías quǽrere áliquem páuperem ex frátribus nostris et revérsus dixit mihi: “Pater!”. Et ego dixi illi: “Ecce ego, fili”.

\noindent Et respóndens ait: “Ecce unus ex natióne nostra occísus est et proiéctus est in foro et nunc íbidem láqueo suffocátus est”.

\noindent Et exsíliens relíqui prándium, ántequam ex illo gustárem, et sústuli eum de platéa in unam domum, donec sol cáderet, et illum sepelírem.

\noindent Et revérsus lavi et manducávi panem meum cum luctu. Et lacrimátus sum. Et, postquam sol occídit, ábii et fódiens sepelívi illum.

\noindent Et próximi mei deridébant me dicéntes: “Non timet adhuc hic homo; iam enim inquisítus est huius rei causa, ut occiderétur, et fugit et ecce íterum sépelit mórtuos”.

\noindent Et lavi ea nocte, postquam illum sepelívi, et introívi in átrium meum et obdormívi circa paríetem átrii, et fácies mea nuda erat propter æstum.

\noindent Et ignorábam quóniam pásseres in paríete super me erant, quorum stércora insedérunt in óculos meos cálida et induxérunt albúgines.

\noindent Et ibam ad médicos, ut curárer, et, quanto inunxérunt me medicaméntis, tanto magis óculi mei excæcabántur máculis, donec perexcaecátus sum.

\noindent Et eram inútilis meis óculis annis quáttuor, et omnes fratres mei dolébant pro me.}
\newcommand{\responsoriumi}{\pars{Responsorium 1.} \scriptura{\Rbardot{} Tob. 3, 15 \Vbardot{} ibid. 3, 2; \textbf{H407}}

\vspace{-5mm}

\responsorium{I}{temporalia/resp-petodomine-CROCHU.gtex}{}

\vfill

\rubrica{vel ad libitum:}

\vspace{3mm}

\pars{Responsorium 1.} \scriptura{\Rbardot{} Tob. 4, 20 \Vbardot{} ibid. 14, 10.11; \textbf{H407}}

\vspace{-5mm}

\responsorium{VIII}{temporalia/resp-omnitempore-CROCHU.gtex}{}}
\newcommand{\lectioii}{\pars{Lectio II.} \scriptura{Sermo in dom. infra oct. Assumptionis, 14-15: Opera omnia, Edit. Cisterc. 5 [1968], 273-274}

\noindent Ex Sermónibus sancti Bernárdi abbátis.

\noindent Martýrium Vírginis, tam in Simeónis prophetía quam in ipsa domínicæ passiónis história commendátur. Pósitus est hic, ait sanctus senex de párvulo Iesu, in signum cui contradicétur, et tuam ipsíus ánimam, ad Maríam autem dicébat, pertransíbit gládius.

\noindent Vere tuam, o Beáta mater, ánimam gládius pertransívit. Alióquin nónnisi eam pertránsiens, carnem Fílii penetráret. Et quidem posteáquam emísit spíritum tuus ille Iesus, ómnium quidem sed speciáliter tuus, ipsíus plane non áttigit ánimam crudélis láncea, quæ ipsíus, nec mórtuo parcens cui nocére non posset, apéruit latus, sed tuam útique ánimam pertransívit. Ipsíus nimírum ánima iam ibi non erat, sed tua plane nequíbat avélli. Tuam ergo pertransívit ánimam vis dolóris, ut plus quam mártyrem non immérito prædicémus, in qua nimírum corpóreæ sensum passiónis excésserit compassiónis efféctus.}
\newcommand{\responsoriumii}{\pars{Responsorium 2.} \scriptura{Lam. 1, 12; \textbf{H223}}

\vspace{-5mm}

\responsorium{VIII}{temporalia/resp-ovosomnes.gtex}{}}
\newcommand{\lectioiii}{\pars{Lectio III.}

\noindent An non tibi plus quam gládius fuit sermo ille, revéra pertránsiens ánimam, et pertíngens usque ad divisiónem ánimæ et spíritus: Múlier, ecce fílius tuus? O commutatiónem! Ioánnes tibi pro Iesu tráditur, servus pro Dómino, discípulus pro Magístro, fílius Zebedǽi pro Fílio Dei, homo purus pro Deo vero. Quómodo non tuam affectuosíssimam ánimam pertransíret hæc audítio, quando et nostra, licet sáxea, licet férrea péctora, sola recordátio scindit?

\noindent Non mirémini, fratres, quod María martyr in ánima fuísse dicátur. Mirétur qui non memínerit audísse se Paulum inter máxima géntium crímina memorántem quod sine affectióne fuíssent. Longe id fuit a Maríæ viscéribus, longe sit et a sérvulis suis.

\noindent Sed forte quis dicat: «Numquid non eum præscíerat moritúrum?». Et indubitánter. «Numquid non sperábat contínuo resurrectúrum?». Et fidénter. «Super hæc dóluit crucifíxum?». Et veheménter. Alióquin, quisnam tu, frater, aut unde tibi hæc sapiéntia, ut miréris plus Maríam compatiéntem quam Maríæ Fílium patiéntem? Ille étiam córpore mori pótuit, ista cómmori corde non pótuit? Fecit illud cáritas, qua maiórem nemo hábuit; fecit et hoc cáritas, cui post illam símilis áltera non fuit.}
\newcommand{\responsoriumiii}{\pars{Responsorium 3.} \scriptura{\Rbardot{} Iob 16, 17 \Vbardot{} Lam. 1, 12; \textbf{H219}}

\vspace{-5mm}

\responsorium{V}{temporalia/resp-caligaverunt-cumdox.gtex}{}}
\newcommand{\hymnuslaudes}{\pars{Hymnus.}

\cuminitiali{II}{temporalia/hym-EiaMater.gtex}}
\newcommand{\laudes}{\pars{Psalmus 1.} \scriptura{Ps. 62, 9}

\vspace{-4mm}

\antiphona{VIII G}{temporalia/ant-adhaesitanimamea.gtex}

%\vspace{-2mm}

\scriptura{Psalmus 62}

%\vspace{-2mm}

\initiumpsalmi{temporalia/ps62-initium-viii-G-auto.gtex}

%\vspace{-1.5mm}

\input{temporalia/ps62-viii-G.tex} \Abardot{}

\vfill
\pagebreak

\pars{Psalmus 2.} \scriptura{Phil. 1, 21; Gal. 6, 14; \textbf{H284}}

\vspace{-4mm}

\antiphona{I g}{temporalia/ant-mihivivere.gtex}

%\vspace{-2mm}

\scriptura{Canticum trium puerorum, Dan. 3, 57-88 et 56}

\initiumpsalmi{temporalia/dan3-initium-i-g-auto.gtex}

\input{temporalia/dan3-i-g-sinedox.tex}

\rubrica{Hic non dicitur Gloria Patri, neque Amen.}

\vfill

\antiphona{}{temporalia/ant-mihivivere.gtex}

\vfill
\pagebreak

\pars{Psalmus 3.}

\vspace{-4mm}

\antiphona{I d}{temporalia/ant-mortuussum.gtex}

%\vspace{-2mm}

\scriptura{Psalmus 149}

%\vspace{-2mm}

\initiumpsalmi{temporalia/ps149-initium-i-d-auto.gtex}

\input{temporalia/ps149-i-d.tex} \Abardot{}

\vfill
\pagebreak}
\newcommand{\lectiobrevis}{\pars{Lectio Brevis.} \scriptura{Col. 1, 24-25}

\noindent Nunc gáudeo in passiónibus pro vobis et adímpleo ea, quæ desunt passiónum Christi, in carne mea pro córpore eius, quod est ecclésia, cuius factus sum ego miníster secúndum dispensatiónem Dei, quæ data est mihi in vos, ut ímpleam verbum Dei.}
\newcommand{\responsoriumbreve}{\pars{Responsorium breve.}

\cuminitiali{VI}{temporalia/resp-pertesalutem.gtex}}
\newcommand{\benedictus}{\pars{Canticum Zachariæ.} \scriptura{Lc. 2, 35}

\vspace{-4mm}

\antiphona{VI F}{temporalia/ant-tuamipsiusanimam.gtex}

\vspace{-2mm}

\scriptura{Lc. 1, 68-79}

\vspace{-2mm}

\cantusSineNeumas
\initiumpsalmi{temporalia/benedictus-initium-vi-F-auto.gtex}

%\vspace{-1.5mm}

\input{temporalia/benedictus-vi-F.tex} \Abardot{}}
\newcommand{\benedicamuslaudes}{\cuminitiali{I}{temporalia/benedicamus-festis-bmv.gtex}}
\newcommand{\hebdomada}{infra Hebdom. XXIV et ultima post Pentecosten.}
\newcommand{\oratioLaudes}{\cuminitiali{}{temporalia/oratio24.gtex}}
\newcommand{\hiemalis}{Hiemalis.}

% LuaLaTeX

\documentclass[a4paper, twoside, 12pt]{article}
\usepackage[latin]{babel}
%\usepackage[landscape, left=3cm, right=1.5cm, top=2cm, bottom=1cm]{geometry} % okraje stranky
%\usepackage[landscape, a4paper, mag=1166, truedimen, left=2cm, right=1.5cm, top=1.6cm, bottom=0.95cm]{geometry} % okraje stranky
\usepackage[landscape, a4paper, mag=1400, truedimen, left=0.5cm, right=0.5cm, top=0.5cm, bottom=0.5cm]{geometry} % okraje stranky

\usepackage{fontspec}
\setmainfont[FeatureFile={junicode.fea}, Ligatures={Common, TeX}, RawFeature=+fixi]{Junicode}
%\setmainfont{Junicode}

% shortcut for Junicode without ligatures (for the Czech texts)
\newfontfamily\nlfont[FeatureFile={junicode.fea}, Ligatures={Common, TeX}, RawFeature=+fixi]{Junicode}

\usepackage{multicol}
\usepackage{color}
\usepackage{lettrine}
\usepackage{fancyhdr}

% usual packages loading:
\usepackage{luatextra}
\usepackage{graphicx} % support the \includegraphics command and options
\usepackage{gregoriotex} % for gregorio score inclusion
\usepackage{gregoriosyms}
\usepackage{wrapfig} % figures wrapped by the text
\usepackage{parcolumns}
\usepackage[contents={},opacity=1,scale=1,color=black]{background}
\usepackage{tikzpagenodes}
\usepackage{calc}
\usepackage{longtable}
\usetikzlibrary{calc}

\setlength{\headheight}{14.5pt}

\input{conventuscommune.tex} % Often used macros

\newcommand{\annusEditionis}{2021}

%%%% Vicekrat opakovane kousky

\newcommand{\anteOrationem}{
  \rubrica{Ante Orationem, cantatur a Superiore:}

  \pars{Supplicatio Litaniæ.}

  \cuminitiali{}{temporalia/supplicatiolitaniae.gtex}

  \pars{Oratio Dominica.}

  \cuminitiali{}{temporalia/oratiodominica.gtex}

  \rubrica{Deinde dicitur ab Hebdomadario:}

  \cuminitiali{}{temporalia/dominusvobiscum-solemnis.gtex}

  \rubrica{In choro monialium loco Dominus vobiscum dicitur:}

  \sineinitiali{temporalia/domineexaudi.gtex}
}

\setlength{\columnsep}{30pt} % prostor mezi sloupci

%%%%%%%%%%%%%%%%%%%%%%%%%%%%%%%%%%%%%%%%%%%%%%%%%%%%%%%%%%%%%%%%%%%%%%%%%%%%%%%%%%%%%%%%%%%%%%%%%%%%%%%%%%%%%
\begin{document}

% Here we set the space around the initial.
% Please report to http://home.gna.org/gregorio/gregoriotex/details for more details and options
\grechangedim{afterinitialshift}{2.2mm}{scalable}
\grechangedim{beforeinitialshift}{2.2mm}{scalable}
\grechangedim{interwordspacetext}{0.22 cm plus 0.15 cm minus 0.05 cm}{scalable}%
\grechangedim{annotationraise}{-0.2cm}{scalable}

% Here we set the initial font. Change 38 if you want a bigger initial.
% Emit the initials in red.
\grechangestyle{initial}{\color{red}\fontsize{38}{38}\selectfont}

\pagestyle{empty}

%%%% Titulni stranka
\begin{titulusOfficii}
\ifx\titulus\undefined
\nomenFesti{Feria II \hebdomada{}}
\else
\titulus
\fi
\end{titulusOfficii}

\vfill

\begin{center}
%Ad usum et secundum consuetudines chori \guillemotright{}Conventus Choralis\guillemotleft.

%Editio Sancti Wolfgangi \annusEditionis
\end{center}

\scriptura{}

\pars{}

\pagebreak

\renewcommand{\headrulewidth}{0pt} % no horiz. rule at the header
\fancyhf{}
\pagestyle{fancy}

\cantusSineNeumas

\ifx\oratio\undefined
\ifx\laudb\undefined
\else
\newcommand{\oratio}{\pars{Oratio.}

\noindent Dómine Deus omnípotens, qui ad princípium huius diéi nos perveníre fecísti, tua nos hódie salva virtúte, ut in hac die ad nullum declinémus peccátum, sed semper ad tuam iustítiam faciéndam nostra procédant elóquia, dirigántur cogitatiónes et ópera.

\noindent Per Dóminum nostrum Iesum Christum, Fílium tuum, qui tecum vivit et regnat in unitáte Spíritus Sancti, Deus, per ómnia sǽcula sæculórum.

\noindent \Rbardot{} Amen.}
\fi
\fi

\hora{Ad Matutinum.} %%%%%%%%%%%%%%%%%%%%%%%%%%%%%%%%%%%%%%%%%%%%%%%%%%%%%
%\sideThumbs{Matutinum}

\vspace{2mm}

\cuminitiali{}{temporalia/dominelabiamea.gtex}

\vfill
%\pagebreak

\vspace{2mm}

\ifx\invitatorium\undefined
\pars{Invitatorium.} \scriptura{Ps. 94, 1; Psalmus 94; \textbf{H451}}

\vspace{-6mm}

\antiphona{VI}{temporalia/inv-jubilemusdeo.gtex}\else
\invitatorium
\fi

\vfill
\pagebreak

\ifx\hymnusmatutinum\undefined
\ifx\matua\undefined
\else
\pars{Hymnus.}

{
\grechangedim{interwordspacetext}{0.10 cm plus 0.15 cm minus 0.05 cm}{scalable}%
\antiphona{II}{temporalia/hym-IpsumNunc.gtex}
\grechangedim{interwordspacetext}{0.22 cm plus 0.15 cm minus 0.05 cm}{scalable}%
}
\fi
\else
\hymnusmatutinum
\fi

\vspace{-3mm}

\vfill
\pagebreak

\ifx\matub\undefined
\else
% MAT B
\pars{Psalmus 1.} \scriptura{Ps. 30, 2; \textbf{H90}}

\vspace{-4mm}

\antiphona{VIII G}{temporalia/ant-intuaiustitia.gtex}

%\vspace{-2mm}

\scriptura{Ps. 30, 2-9}

%\vspace{-2mm}

\initiumpsalmi{temporalia/ps30i-initium-viii-G-auto.gtex}

\vspace{-1.5mm}

\input{temporalia/ps30i-viii-G.tex} \Abardot{}

\vfill
\pagebreak

\pars{Psalmus 2.} \scriptura{Ps. 66, 2}

\vspace{-4mm}

\antiphona{E}{temporalia/ant-illuminadomine.gtex}

%\vspace{-2mm}

\scriptura{Ps. 30, 10-17}

%\vspace{-2mm}

\initiumpsalmi{temporalia/ps30ii-initium-e-a-auto.gtex}

\input{temporalia/ps30ii-e-a.tex} \Abardot{}

\vfill
\pagebreak

\pars{Psalmus 3.} \scriptura{Ps. 30, 24}

\vspace{-4mm}

\antiphona{II D}{temporalia/ant-diligitedominum.gtex}

%\vspace{-5mm}

\scriptura{Ps. 30, 20-25}

%\vspace{-2mm}

\initiumpsalmi{temporalia/ps30iii-initium-ii-D-auto.gtex}

\input{temporalia/ps30iii-ii-D.tex} \Abardot{}

\vfill
\pagebreak
\fi

\pars{Versus.}

\ifx\matversus\undefined
\ifx\matub\undefined
\else
\noindent \Vbardot{} Dírige me, Dómine, in veritáte tua, et doce me.

\noindent \Rbardot{} Quia tu es Deus salútis meæ.
\fi
\else
\matversus
\fi

\vspace{5mm}

\sineinitiali{temporalia/oratiodominica-mat.gtex}

\vspace{5mm}

\pars{Absolutio.}

\cuminitiali{}{temporalia/absolutio-exaudi.gtex}

\vfill
\pagebreak

\cuminitiali{}{temporalia/benedictio-solemn-benedictione.gtex}

\vspace{7mm}

\lectioi

\noindent \Vbardot{} Tu autem, Dómine, miserére nobis.
\noindent \Rbardot{} Deo grátias.

\vfill
\pagebreak

\responsoriumi

\vfill
\pagebreak

\cuminitiali{}{temporalia/benedictio-solemn-unigenitus.gtex}

\vspace{7mm}

\lectioii

\noindent \Vbardot{} Tu autem, Dómine, miserére nobis.
\noindent \Rbardot{} Deo grátias.

\vfill
\pagebreak

\responsoriumii

\vfill
\pagebreak

\cuminitiali{}{temporalia/benedictio-solemn-spiritus.gtex}

\vspace{7mm}

\lectioiii

\noindent \Vbardot{} Tu autem, Dómine, miserére nobis.
\noindent \Rbardot{} Deo grátias.

\vfill
\pagebreak

\responsoriumiii

\vfill
\pagebreak

\rubrica{Reliqua omittuntur, nisi Laudes separandæ sint.}

\sineinitiali{temporalia/domineexaudi.gtex}

\vfill

\oratio

\vfill

\noindent \Vbardot{} Dómine, exáudi oratiónem meam.
\Rbardot{} Et clamor meus ad te véniat.

\vfill

\noindent \Vbardot{} Benedicámus Dómino.
\noindent \Rbardot{} Deo grátias.

\vfill

\noindent \Vbardot{} Fidélium ánimæ per misericórdiam Dei requiéscant in pace.
\Rbardot{} Amen.

\vfill
\pagebreak

\hora{Ad Laudes.} %%%%%%%%%%%%%%%%%%%%%%%%%%%%%%%%%%%%%%%%%%%%%%%%%%%%%
%\sideThumbs{Laudes}

\cantusSineNeumas

\vspace{0.5cm}
\grechangedim{interwordspacetext}{0.18 cm plus 0.15 cm minus 0.05 cm}{scalable}%
\cuminitiali{}{temporalia/deusinadiutorium-communis.gtex}
\grechangedim{interwordspacetext}{0.22 cm plus 0.15 cm minus 0.05 cm}{scalable}%

\vfill
\pagebreak

\ifx\hymnuslaudes\undefined
\ifx\laudbd\undefined
\else
\pars{Hymnus} \scriptura{Hilarius (\olddag{} 367)}

\grechangedim{interwordspacetext}{0.16 cm plus 0.15 cm minus 0.05 cm}{scalable}%
\cuminitiali{IV}{temporalia/hym-LucisLargitor.gtex}
\grechangedim{interwordspacetext}{0.22 cm plus 0.15 cm minus 0.05 cm}{scalable}%
\vspace{-3mm}
\fi
\else
\hymnuslaudes
\fi

\vfill
\pagebreak

\ifx\laudb\undefined
\else
\pars{Psalmus 1.} \scriptura{Ps. 41, 3; \textbf{H391}}

\vspace{-4mm}

\antiphona{II D}{temporalia/ant-sitivitanima.gtex}

%\vspace{-2mm}

\scriptura{Psalmus 41}

%\vspace{-2mm}

\initiumpsalmi{temporalia/ps41-initium-ii-D-auto.gtex}

%\vspace{-1.5mm}

\input{temporalia/ps41-ii-D.tex}

\vfill

\antiphona{}{temporalia/ant-sitivitanima.gtex}

\vfill
\pagebreak

\pars{Psalmus 2.}

\vspace{-4mm}

\antiphona{III a}{temporalia/ant-ostendenobisdomine.gtex}

%\vspace{-2mm}

\scriptura{Canticum Ecclesiastici, Sir. 36, 1-7.13-16}

%\vspace{-3mm}

\initiumpsalmi{temporalia/ecclesiastici-initium-iii-a-auto.gtex}

\input{temporalia/ecclesiastici-iii-a.tex} \Abardot{}

\vfill
\pagebreak

\pars{Psalmus 3.}

\vspace{-4mm}

\antiphona{II D}{temporalia/ant-operamanuumeius.gtex}

\scriptura{Psalmus 18, 1-7}

\initiumpsalmi{temporalia/ps18i-initium-ii-D-auto.gtex}

\input{temporalia/ps18i-ii-D.tex} \Abardot{}

\vfill
\pagebreak
\fi

\ifx\lectiobrevis\undefined
\ifx\laudb\undefined
\else
\pars{Lectio Brevis.} \scriptura{Ier. 15, 16}

\noindent Invénti sunt sermónes tui, et comédi eos, et factum est mihi verbum tuum in gáudium et in lætítiam cordis mei, quóniam invocátum est nomen tuum super me, Dómine Deus exercítuum.
\fi
\else
\lectiobrevis
\fi

\vfill

\ifx\responsoriumbreve\undefined
\ifx\laudbd\undefined
\else
\pars{Responsorium breve.} \scriptura{Ps. 32, 1.3}

\cuminitiali{VI}{temporalia/resp-exsultateiusti.gtex}
\fi
\else
\responsoriumbreve
\fi

\vfill
\pagebreak

\ifx\benedictus\undefined
\ifx\laudbd\undefined
\else
\pars{Canticum Zachariæ.} \scriptura{Lc. 1, 68; \textbf{H422}}

\vspace{-4mm}

{
\grechangedim{interwordspacetext}{0.18 cm plus 0.15 cm minus 0.05 cm}{scalable}%
\antiphona{IV E}{temporalia/ant-benedictusdominus.gtex}
\grechangedim{interwordspacetext}{0.22 cm plus 0.15 cm minus 0.05 cm}{scalable}%
}

%\vspace{-3mm}

\scriptura{Lc. 1, 68-79}

%\vspace{-2mm}

\cantusSineNeumas
\initiumpsalmi{temporalia/benedictus-initium-iv-E-auto.gtex}

%\vspace{-1.5mm}

\input{temporalia/benedictus-iv-E.tex} \Abardot{}
\fi
\else
\benedictus
\fi

\vspace{-1cm}

\vfill
\pagebreak

%\sideThumbs{{\scriptsize{}Fine horarum}}

\pars{Preces.}

\sineinitiali{}{temporalia/tonusprecum.gtex}

\ifx\preces\undefined
\ifx\laudb\undefined
\else
\noindent Salvátor noster fecit nos regnum et sacerdótium, ut hóstias Deo acceptábiles offerámus. \gredagger{} Grati ígitur eum invocémus:

\Rbardot{} Serva nos in tuo ministério, Dómine.

\noindent Christe, sacérdos ætérne, qui sanctum pópulo tuo sacerdótium concessísti, \gredagger{} concéde, ut spiritáles hóstias Deo acceptábiles iúgiter offerámus.

\Rbardot{} Serva nos in tuo ministério, Dómine.

\noindent Spíritus tui fructus nobis largíre propítius, \gredagger{} patiéntiam, benignitátem et mansuetúdinem.

\Rbardot{} Serva nos in tuo ministério, Dómine.

\noindent Da nobis te amáre, ut te, qui es cáritas, possideámus, \gredagger{} et bene ágere, ut per vitam étiam nostram te laudémus.

\Rbardot{} Serva nos in tuo ministério, Dómine.

\noindent Quæ frátribus nostris sunt utília, nos quǽrere concéde, \gredagger{} ut salútem facílius consequántur.

\Rbardot{} Serva nos in tuo ministério, Dómine.
\fi
\else
\preces
\fi

\vfill

\pars{Oratio Dominica.}

\cuminitiali{}{temporalia/oratiodominicaalt.gtex}

\vfill
\pagebreak

\rubrica{vel:}

\pars{Supplicatio Litaniæ.}

\cuminitiali{}{temporalia/supplicatiolitaniae.gtex}

\vfill

\pars{Oratio Dominica.}

\cuminitiali{}{temporalia/oratiodominica.gtex}

\vfill
\pagebreak

% Oratio. %%%
\oratio

\vspace{-1mm}

\vfill

\rubrica{Hebdomadarius dicit Dominus vobiscum, vel, absente sacerdote vel diacono, sic concluditur:}

\vspace{2mm}

\antiphona{C}{temporalia/dominusnosbenedicat.gtex}

\rubrica{Postea cantatur a cantore:}

\vspace{2mm}

\cuminitiali{IV}{temporalia/benedicamus-feria-laudes.gtex}

\vspace{1mm}

\vfill
\pagebreak

\end{document}

