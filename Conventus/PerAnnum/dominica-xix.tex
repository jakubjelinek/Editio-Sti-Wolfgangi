\newcommand{\titulus}{\nomenFesti{Dominica XIX per Annum.}}
\newcommand{\tedeummonasticum}{Monasticum}
\newcommand{\oratio}{\pars{Oratio.}

\noindent Omnípotens sempitérne Deus, quem patérno nómine invocáre præsúmimus, pérfice in córdibus nostris spíritum adoptiónis filiórum, ut promíssam hereditátem íngredi mereámur.

\pars{Pro pace in Ucraina.} \scriptura{Sir. 50, 25; 2 Esdr. 4, 20; \textbf{H416}}

\vspace{-4mm}

\antiphona{II D}{temporalia/ant-dapacemdomine.gtex}

\vfill

\noindent Deus, a quo sancta desidéria, recta consília et iusta sunt ópera: da servis tuis illam, quam mundus dare non potest, pacem; ut et corda nostra mandátis tuis dédita, et hóstium subláta formídine, témpora sint tua protectióne tranquílla.

\noindent Per Dóminum nostrum Iesum Christum, Fílium tuum, qui tecum vivit et regnat in unitáte Spíritus Sancti, Deus, per ómnia sǽcula sæculórum.

\noindent \Rbardot{} Amen.}
\newcommand{\nocturnoi}{\pars{Psalmus 1.} \scriptura{Ps. 1, 2; \textbf{H372}}

\vspace{-4mm}

\antiphona{I d}{temporalia/ant-inlegedominifuit.gtex}

%\vspace{-2mm}

\scriptura{Ps. 1}

%\vspace{-2mm}

\initiumpsalmi{temporalia/ps1-initium-i-d-auto.gtex}

\input{temporalia/ps1-i-d.tex} \Abardot{}

\vfill
\pagebreak

\pars{Psalmus 2.} \scriptura{Ps. 2, 6; \textbf{H372}}

\vspace{-4mm}

\antiphona{III a}{temporalia/ant-praedicanspraeceptumdomini.gtex}

\vspace{-1mm}

\scriptura{Ps. 2}

\vspace{-2mm}

\initiumpsalmi{temporalia/ps2-initium-iii-a-auto.gtex}

\input{temporalia/ps2-iii-a.tex} \Abardot{}

\vfill
\pagebreak

\pars{Psalmus 3.} \scriptura{Ps. 3, 5; \textbf{H372}}

\vspace{-4mm}

\antiphona{VII c trans.}{temporalia/ant-vocemeaaddominum.gtex}

%\vspace{-2mm}

\scriptura{Ps. 3}

\initiumpsalmi{temporalia/ps3-initium-vii-c-trans.gtex}

\input{temporalia/ps3-vii-c.tex} \Abardot{}

\vfill
\pagebreak
}
\newcommand{\nocturnoii}{\vspace{-4mm}

\pars{Psalmus 4.} \scriptura{Ex. 15, 2; \textbf{H39}}

\vspace{-4mm}

\antiphona{VIII G}{temporalia/ant-eccedeusmeus.gtex}

%\vspace{-2mm}

\scriptura{Ps. 144, 1-9}

%\vspace{-2mm}

\initiumpsalmi{temporalia/ps144i-initium-viii-G-auto.gtex}

\input{temporalia/ps144i-viii-G.tex} \Abardot{}

\vfill
\pagebreak

\pars{Psalmus 5.} \scriptura{Ps. 144, 13; \textbf{H100}}

\vspace{-4mm}

\antiphona{VII c}{temporalia/ant-regnumtuumdomine.gtex}

%\vspace{-2mm}

\scriptura{Ps. 144, 10-13}

\initiumpsalmi{temporalia/ps144x_xiii-initium-vii-c-auto.gtex}

\input{temporalia/ps144x_xiii-vii-c.tex} \Abardot{}

\vfill
\pagebreak

\pars{Psalmus 6.} \scriptura{\textbf{H99}}

\vspace{-4mm}

\antiphona{VIII a}{temporalia/ant-inaeternumet.gtex}

%\vspace{-4mm}

\scriptura{Ps. 144, 14-21}

%\vspace{-2mm}

\initiumpsalmi{temporalia/ps144xiv_xxi-initium-viii-a-auto.gtex}

%\vspace{-1.5mm}

\input{temporalia/ps144xiv_xxi-viii-a.tex} \Abardot{}

\vfill
\pagebreak}
\newcommand{\nocturnoiii}{\pars{Cantica.}

\vspace{-4mm}

\antiphona{D}{temporalia/ant-eccedeusnoster.gtex}

%\vspace{-2mm}

\scriptura{Canticum Isaiæ, Is. 33, 2-10}

%\vspace{-2mm}

\initiumpsalmi{temporalia/isaiae7-initium-d-g-auto.gtex}

\input{temporalia/isaiae7-d-g.tex} \hfill \rubrica{Hic non dicitur antiphona.}

\vfill
\pagebreak

\scriptura{Canticum Isaiæ, Is. 33, 13-17}

%\vspace{-2mm}

\initiumpsalmi{temporalia/isaiae8-initium-d-g-auto.gtex}

\input{temporalia/isaiae8-d-g.tex}

\vfill
\pagebreak

\scriptura{Canticum Ecclesiastici, Sir. 36, 14-19}

%\vspace{-2mm}

\initiumpsalmi{temporalia/ecclesiasticus36-initium-d-g-auto.gtex}

\input{temporalia/ecclesiasticus36-d-g.tex}

\vfill

\antiphona{}{temporalia/ant-eccedeusnoster.gtex}

\vfill
\pagebreak}
\newcommand{\lectioi}{\pars{Lectio I.} \scriptura{Eccl. 1, 1-7}

\noindent Incipit liber Ecclesiástes.

\noindent Verba Ecclesiástes fílii David regis Ierúsalem. «Vánitas vanitátum, dixit Ecclesiástes, vánitas vanitátum et ómnia vánitas». Quid lucri est hómini de univérso labóre suo, quo labórat sub sole? Generátio prǽterit, et generátio ádvenit, terra autem in ætérnum stat. Oritur sol, et óccidit sol et ad locum suum anhélat ibíque renáscitur. Gyrat per merídiem et fléctitur ad aquilónem, lustrans univérsa in circúitu pergit spíritus et in círculos suos revértitur. Omnia flúmina pergunt ad mare, et mare non redúndat; ad locum, unde éxeunt, flúmina illuc revertúntur in cursu suo.}
\newcommand{\responsoriumi}{\pars{Responsorium 1.} \scriptura{\Rbardot{} Eccli. 24, 8-11 \Vbardot{} ibid. 24, 7; \textbf{H399}}

\vspace{-5mm}

\responsorium{VI}{temporalia/resp-gyrumcaeli-CROCHU.gtex}{}}
\newcommand{\lectioii}{\pars{Lectio II.} \scriptura{Eccl. 1, 8-11}

\noindent Cunctæ res diffíciles; non potest eas homo explicáre sermóne. Non saturátur óculus visu, nec auris audítu implétur. Quod fuit, ipsum est, quod futúrum est. Quod factum est, ipsum est, quod faciéndum est: nihil sub sole novum. Si de quadam re dícitur: «Ecce hoc novum est», iam enim præcéssit in sǽculis, quæ fuérunt ante nos. Non est priórum memória, sed nec eórum quidem, qui póstea futúri sunt, erit recordátio apud eos, qui futúri sunt in novíssimo.}
\newcommand{\responsoriumii}{\pars{Responsorium 2.} \scriptura{\Rbardot{} 3 Reg. 8, 39 \& Ps. 138, 16 \Vbardot{} 1 Paral. 28, 9; \textbf{H400}}

\vspace{-5mm}

\responsorium{II}{temporalia/resp-quaesuntincorde-CROCHU-sinedox.gtex}{}}
\newcommand{\lectioiii}{\pars{Lectio III.} \scriptura{Eccl. 1, 12-18}

\noindent Ego Ecclesiástes fui rex Israel in Ierúsalem et propósui in ánimo meo quǽrere et investigáre sapiénter de ómnibus, quæ fiunt sub sole. Hanc occupatiónem péssimam dedit Deus fíliis hóminum, ut occuparéntur in ea. Vidi cuncta, quæ fiunt sub sole; et ecce univérsa vánitas et afflíctio spíritus. Quod est curvum, rectum fíeri non potest; et, quod defíciens est, numerári non potest. Locútus sum ego in corde meo dicens: «Ecce ego magnificávi et appósui sapiéntiam super omnes, qui fuérunt ante me in Ierúsalem; et mens mea contempláta est multam sapiéntiam et sciéntiam». Dedíque cor meum, ut scirem sapiéntiam et sciéntiam, insipiéntiam et stultítiam. Et agnóvi quod in his quoque esset afflíctio spíritus, eo quod in multa sapiéntia multus sit mæror; et qui addit sciéntiam, addit et labórem.}
\newcommand{\responsoriumiii}{\pars{Responsorium 3.} \scriptura{\Rbardot{} Prv. 31, 8; \Vbardot{} ibid. 31, 7; \textbf{H400}}

\vspace{-5mm}

\responsorium{I}{temporalia/resp-verbuminiquum-CROCHU.gtex}{}}
\newcommand{\lectioiv}{\pars{Lectio IV.} \scriptura{Centuria 1, cap. 1, 4-5. 16-17. 23-24. 26-28. 30-40: PG 90, 962-967}

\noindent Ex Capítibus sancti Máximi Confessóris abbátis De caritáte.

\noindent Cáritas, bona est mentis afféctio, quæ divínæ cognitióni nihil antepónit. Eius autem caritátis hábitum nemo umquam ássequi possit, qui terrénis ullis rebus ánimo devínctus sit.

\noindent Qui Deum díligit, cunctis ab eo cónditis illíus cognitiónem ac sciéntiam antepónit; eíque ánimi desidério ac amóre incessánter incúmbit.

\noindent Cum ómnia quæ sunt, Deo auctóre ac propter Deum cóndita sint, sitque Deus, iis quæ ipso auctóre cóndita sunt, longe præstántior; qui Deum citra omnem comparatiónem meliórem relínquens, deterióribus addícitur, minóris se Deum fácere osténdit, quam quæ illo auctóre cóndita sunt.

\noindent \emph{Qui díligit me,} dicit Dóminus, \emph{mandáta mea servábit. Hoc est autem,} inquit, \emph{mandátum meum, ut diligátis ínvicem.} Itaque qui non díligit próximum, mandátum non servat. Qui autem mandátum non servat, is neque potest dilígere Dóminum.}
\newcommand{\responsoriumiv}{\pars{Responsorium 4.} \scriptura{\Rbardot{} Sap. 9, 10 \Vbardot{} ibid. 9, 4; \textbf{H398}}

\vspace{-5mm} 

\responsorium{I}{temporalia/resp-emittedominesapientiam-CROCHU-sinedox.gtex}{}}
\newcommand{\lectiov}{\pars{Lectio V.}

\noindent Beátus homo, qui omnem perǽque hóminem dilígere potis est.

\noindent Qui Deum díligit, omníno étiam próximum díligit: qui autem eiúsmodi est, pecúnias serváre non potest, sed divíne eas dispénsat, síngulis qui égeant tríbuens.

\noindent Qui Dei imitatióne eleemósynam facit, boni ac mali, iusti et iniústi, in córporis necessáriis discrímen non novit, sed cunctis æquáliter pro iustæ necessitátis modo distríbuit; tamétsi propter bonam voluntátem, eum qui virtúte nitet ac studiósus est, ímprobo antepónit.

\noindent Non solum pecuniárum elargitióne próditur caritátis afféctio; sed multo magis impartitióne divínæ doctrínæ ac corporális obséquii exhibitióne.

\noindent Qui sǽculi rebus vere exque ánimo núntium remíttens citra omnem simulatiónem caritátis obséquiis próximo óperam impéndit, omni cito liberátus afféctu atque vítio, divínæ párticeps dilectiónis scientiǽque evádit.}
\newcommand{\responsoriumv}{\pars{Responsorium 5.} \scriptura{\Rbardot{} Ps. 110, 10 \Vbardot{} Sap. 6, 19 \& Cantor; \textbf{H399}}

\vspace{-5mm}

\responsorium{II}{temporalia/resp-initiumsapientiae-CROCHU.gtex}{}}
\newcommand{\lectiovi}{\pars{Lectio VI.}

\noindent Qui divínam in se nactus est caritátem, non lassátur aut fatíscit sequéndo post Dóminum Deum suum, iuxta divínum Ieremíam, sed omnem labórem opprobriúmque et iniúriam forti ánimo suffert, nulli prorsus cógitans malum.

\noindent \emph{Nolíte dícere,} ait divínus Ieremías, \emph{templum Dómini sumus.} Tu quoque ne díxeris: «Nuda ac sola fides in Dóminum nostrum Iesum Christum salútem mihi potest præstáre». Non enim hoc fíeri potest nisi, étiam caritátem in ipsum per ópera comparáveris. Ad símplicem enim fidem quod áttinet: \emph{Etiam dǽmones credunt, et contremíscunt.}

\noindent Caritátis opus est, benefícii in próximum ex ánimo collocátio, et longanímitas et patiéntia; itémque recta cum ratióne rebus uti.}
\newcommand{\responsoriumvi}{\pars{Responsorium 6.} \scriptura{\Rbardot{} Prov. 3, 11 \Vbardot{} ibid. 6, 20; \textbf{H401}}

\vspace{-5mm}

\responsorium{IV}{temporalia/resp-filinolideficere-CROCHU-cumdox.gtex}{}}
\newcommand{\evangelium}{
\pars{Versus.} \scriptura{Ps. 118, 148}

% Versus. %%%
\sineinitiali{temporalia/versus-praevenerunt.gtex}

\vspace{5mm}

\sineinitiali{temporalia/oratiodominica-mat.gtex}

\vspace{5mm}

\pars{Absolutio.}

\cuminitiali{}{temporalia/absolutio-avinculis.gtex}

\vfill
\pagebreak

\cuminitiali{}{temporalia/benedictio-solemn-evangelica.gtex}

\vspace{7mm}

\pars{Evangelium} \scriptura{Mt. 14, 22-33}

\noindent Léctio sancti Evangélii secúndum Matthǽum.

\noindent [Postquam saturáta est turba,] statim iussit Iesus discípulos ascéndere in navículam et præcédere eum trans fretum, donec dimítteret turbas. Et dimíssis turbis, ascéndit in montem solus oráre. Véspere autem facto, solus erat ibi. Navícula autem iam multis stádiis a terra distábat, flúctibus iactáta; erat enim contrárius ventus.

\noindent Quarta autem vigília noctis venit ad eos ámbulans supra mare. Discípuli autem, vidéntes eum supra mare ambulántem, turbáti sunt dicéntes: «Phantásma est», et præ timóre clamavérunt.

\noindent Statímque Iesus locútus est eis dicens: «Habéte fidúciam, ego sum; nolíte timére!».

\noindent Respóndens autem ei Petrus dixit: «Dómine, si tu es, iube me veníre ad te super aquas».

\noindent At ipse ait: «Veni!».

\noindent Et descéndens Petrus de navícula ambulávit super aquas et venit ad Iesum. Videns vero ventum válidum tímuit et, cum cœpísset mergi, clamávit dicens: «Dómine, salvum me fac!».

\noindent Contínuo autem Iesus exténdens manum apprehéndit eum et ait illi: «Módicæ fídei, quare dubitásti?». Et cum ascendíssent in navículam, cessávit ventus.

\noindent Qui autem in navícula erant, adoravérunt eum dicéntes: «Vere Fílius Dei es!».

\vspace{5mm}

\scriptura{Sermo 76,1. 3. 4. 5. 8 : PL 38, 479-482}

\noindent Ex Sermónibus sancti Augustíni epíscopi.

\noindent Evangélium quod recitátum est de Dómino Christo, qui super aquas maris ambulávit; et de apóstolo Petro, qui ámbulans timéndo titubávit, et diffidéndo mersus, confiténdo rursus emérsit; ádmonet nos intellégere mare præsens sǽculum esse, Petrum vero apóstolum Ecclésiæ únicæ typum. Ipse enim Petrus in apostolórum órdine primus, in Christi amóre promptíssimus, sæpe unus respóndet pro ómnibus.
   
\noindent Hoc intuéntes nos Ecclésiæ membrum, discernámus quid de Dei, quid de nostro. Tunc enim non titubábimus, tunc in petra fundábimur, fixi et stábiles érimus advérsus ventos, imbres, flúmina, tentatiónes vidélicet præséntis sǽculi. Illum tamen vidéte Petrum qui tunc erat figúra nostra: modo fidit, modo títubat; modo immortálem confitétur, modo timet ne moriátur.

\noindent Proínde quia Ecclésia Christi habet firmos, habet et infírmos; nec sine firmis potest esse, nec sine infírmis. In eo quod dixit Petrus: \emph{Tu es Christus, Fílius Dei vivi,} firmos signíficat; in eo quod trépidat et títubat, et Christum pati non vult, mortem timéndo, vitam non agnoscéndo, infírmos Ecclésiæ signíficat. In illo ergo uno apóstolo, id est Petro, in órdine apostolórum primo et præcípuo, in quo figurabátur Ecclésia, utrúmque genus significándum fuit, id est firmi et infírmi; quia sine utróque non est Ecclésia.

\noindent Hinc est ergo et quod modo lectum est: \emph{Dómine, si tu es, iube me veníre ad te super aquas.} Ergo ambulávit Petrus super aquas in iussu Dómini, sciens hoc se a se habére non posse. Fide váluit quod humána infírmitas non valéret. Hi sunt firmi Ecclésiæ. Iúbeat Deus homo ut possit quod non potest homo. \emph{Veni,} inquit. Et descéndit, et cœpit ambuláre super aquas: et pótuit Petrus, quia iússerat petra.

\vfill
\pagebreak

\pars{Responsorium 7.} \scriptura{\Rbardot{} Mt. 14, 28.31 \Vbardot{} ibid. 14, 30; \textbf{H281}}

\vspace{-5mm}

\responsorium{VIII}{temporalia/resp-dominesitues-CROCHU-cumdox.gtex}{}

\vfill
\pagebreak
}
\newcommand{\responsoriumbreve}{\pars{Responsorium breve.}

\cuminitiali{VI}{temporalia/resp-christefilidei.gtex}}
\newcommand{\benedictus}{\pars{Canticum Zachariæ.} \scriptura{Mt. 14, 30}

\vspace{-4mm}

{
\grechangedim{interwordspacetext}{0.18 cm plus 0.15 cm minus 0.05 cm}{scalable}%
\antiphona{II D}{temporalia/ant-cumquevidisset.gtex}
\grechangedim{interwordspacetext}{0.22 cm plus 0.15 cm minus 0.05 cm}{scalable}%
}

%\trAntIMagnificat

\vspace{-4mm}

\scriptura{Lc. 1, 68-79}

\vspace{-2mm}

\cantusSineNeumas
\initiumpsalmi{temporalia/benedictus-initium-iisoll-D-auto.gtex}

%\vspace{-1.5mm}

\input{temporalia/benedictus-iisoll-D.tex} \Abardot{}

\vspace{-10mm}}
\newcommand{\hebdomada}{infra Hebdom. XIX post Pentecosten.}
\newcommand{\oratioLaudes}{\cuminitiali{}{temporalia/oratio19.gtex}}
\newcommand{\hiemalis}{Hiemalis.}

% LuaLaTeX

\documentclass[a4paper, twoside, 12pt]{article}
\usepackage[latin]{babel}
%\usepackage[landscape, left=3cm, right=1.5cm, top=2cm, bottom=1cm]{geometry} % okraje stranky
%\usepackage[landscape, a4paper, mag=1166, truedimen, left=2cm, right=1.5cm, top=1.6cm, bottom=0.95cm]{geometry} % okraje stranky
\usepackage[landscape, a4paper, mag=1400, truedimen, left=0.5cm, right=0.5cm, top=0.5cm, bottom=0.5cm]{geometry} % okraje stranky

\usepackage{fontspec}
\setmainfont[FeatureFile={junicode.fea}, Ligatures={Common, TeX}, RawFeature=+fixi]{Junicode}
%\setmainfont{Junicode}

% shortcut for Junicode without ligatures (for the Czech texts)
\newfontfamily\nlfont[FeatureFile={junicode.fea}, Ligatures={Common, TeX}, RawFeature=+fixi]{Junicode}

\usepackage{multicol}
\usepackage{color}
\usepackage{lettrine}
\usepackage{fancyhdr}

% usual packages loading:
\usepackage{luatextra}
\usepackage{graphicx} % support the \includegraphics command and options
\usepackage{gregoriotex} % for gregorio score inclusion
\usepackage{gregoriosyms}
\usepackage{wrapfig} % figures wrapped by the text
\usepackage{parcolumns}
\usepackage[contents={},opacity=1,scale=1,color=black]{background}
\usepackage{tikzpagenodes}
\usepackage{calc}
\usepackage{longtable}
\usetikzlibrary{calc}

\setlength{\headheight}{14.5pt}

\input{conventuscommune.tex} % Often used macros
%%%% Preklady jednotlivych zpevu (nektere se opakuji, a je dobre mit je
% vsechny na jedne hromade)

% HOURS ---

\newcommand{\trAntI}{\translatioCantus{Muž boží měl kožený toulec, pečlivě
zavázaný, jenž mu visel na šíji a~často se ho dotýkal.}}

\newcommand{\trAntII}{\translatioCantus{Klíč od~něho tak dobře střežil, že
dokud žil v~těle, nikdo z~jeho žáků nezvěděl, co je uvnitř.}}

\newcommand{\trAntIII}{\translatioCantus{Ale když se odebral z~tohoto
života, schránku otevřeli a~objevili v~ní žíněné roucho a~měděný řetěz
potřísněný krví.}}

\newcommand{\trAntIV}{\translatioCantus{A když prohlédli mistrovo tělo,
nalezli jeho tělo na čtyřech místech hluboce zbrázděno ranami od řetězu.}}

\newcommand{\trAntV}{\translatioCantus{Krev vytékající z~těch ran, místy
prostoupila i~žíněným rouchem.}}

\newcommand{\trCapituli}{\translatioCantus{
Miláčkovi Boha a~lidí,
Mojžíšovi požehnané paměti,~\gredagger{}
dopřál slávu rovnou slávě svatých~\grestar{}
učinil ho mocným na postrach nepřátelům
a~jeho slovy zastavil divy.}}

\newcommand{\trLectioBrevis}{\translatioCantus{
Pamatujte na své představené,
kteří vám hlásali Boží slovo.
Uvažte, jak oni skončili život, a~napodobujte jejich víru.
Ježíš Kristus je stejný včera i~dnes i~navěky.
Nenechte se svést věelijakými cizími naukami.}}

\newcommand{\trRespLaud}{\translatioCantus{Spravedlivého vodil Hospodin~\grestar{}
po přímých stezkách. \Vbardot{} A~ukázal mu Boží království.}}

\newcommand{\trRespLaudB}{\translatioCantus{Na tvých hradbách, Jeruzaléme,
ustanovil jsem strážné;~\grestar{}
budou bdít nad mým lidem. \Vbardot{} Ani ve dne, ani v~noci nesmějí nikdy
mlčet.}}

\newcommand{\trVersus}{\translatioCantus{\Vbardot{} Ústa spravedlivého šeptají moudrost, aleluja.
\Rbardot{} A~jeho jazyk ohlašuje právo, aleluja.}}

\newcommand{\trAntBenedictus}{\translatioCantus{Když na bujné oře vložili
nosítka a~sňali jim uzdu, vydali se přímo k~cele božího muže.}}

\newcommand{\trPreces}{\translatioCantus{
\noindent S vděčností chvalme Krista, dobrého Pastýře, \gredagger{} který dal život za své ovce, \grestar{} a~pokorně ho prosme: \Rbardot{} Pane, buď pastýřem svého lidu.

\noindent Kriste, ty dáváš církvi pastýře, a~jejich službou se ujímáš svého lidu, \grestar{} dej, ať v~lásce těch, kteří nás vedou, poznáváme, jak nás miluješ. \Rbardot{} Pane, buď pastýřem svého lidu.

\noindent Ty stále konáš skrze své zástupce službu pastýře a~učitele, \grestar{} nepřestávej nás nikdy vést prostřednictvím svých služebníků. \Rbardot{} Pane, buď pastýřem svého lidu.

\noindent Ty prokazuješ svému lidu skrze jeho pastýře službu lékaře duše i~těla, \grestar{} ochraňuj náš život a~veď nás ke svatosti. \Rbardot{} Pane, buď pastýřem svého lidu.

\noindent Ty posíláš své svaté, aby slovem i~příkladem vedli tvůj lid k~tobě, \grestar{} na jejich přímluvu nás posiluj, abychom vytrvali na cestě, která vede k~věčnému životu. \Rbardot{} Pane, buď pastýřem svého lidu.}}

\newcommand{\trOrationis}{\translatioCantus{Bože, jenž nám dopřáváš radovat
se z~výroční slavnosti svatého tvého vyznavače Havla, uděl dobrotivě,
abychom když slavíme jeho narození, též se řídili podobou jeho skutků.
Skrze…}}
 % Czech translations of the proper texts

\newcommand{\annusEditionis}{2020}

%%%% Vicekrat opakovane kousky

\newcommand{\anteOrationem}{
  \rubrica{Ante Orationem, cantatur a Superiore:}

  \pars{Supplicatio Litaniæ.}

  \cuminitiali{}{temporalia/supplicatiolitaniae.gtex}

  \pars{Oratio Dominica.}

  \cuminitiali{}{temporalia/oratiodominica.gtex}

  \rubrica{Deinde dicitur ab Hebdomadario:}

  \cuminitiali{}{temporalia/dominusvobiscum-solemnis.gtex}

  \rubrica{In choro monialium loco Dominus vobiscum dicitur:}

  \sineinitiali{temporalia/domineexaudi.gtex}
}

\setlength{\columnsep}{30pt} % prostor mezi sloupci

%%%%%%%%%%%%%%%%%%%%%%%%%%%%%%%%%%%%%%%%%%%%%%%%%%%%%%%%%%%%%%%%%%%%%%%%%%%%%%%%%%%%%%%%%%%%%%%%%%%%%%%%%%%%%
\begin{document}

% Here we set the space around the initial.
% Please report to http://home.gna.org/gregorio/gregoriotex/details for more details and options
\grechangedim{afterinitialshift}{2.2mm}{scalable}
\grechangedim{beforeinitialshift}{2.2mm}{scalable}
\grechangedim{interwordspacetext}{0.22 cm plus 0.15 cm minus 0.05 cm}{scalable}%
\grechangedim{annotationraise}{-0.2cm}{scalable}

% Here we set the initial font. Change 38 if you want a bigger initial.
% Emit the initials in red.
\grechangestyle{initial}{\color{red}\fontsize{38}{38}\selectfont}

\pagestyle{empty}

%%%% Titulni stranka
\begin{titulusOfficii}
\titulus{}
\end{titulusOfficii}

% graphic
%\vspace{1.5cm}
%\begin{center}
%\includegraphics[width=8cm]{emmaus.jpg}
%\end{center}

\vfill

\begin{center}
%Ad usum et secundum consuetudines chori \guillemotright{}Conventus Choralis\guillemotleft.

%Editio Sancti Wolfgangi \annusEditionis
\end{center}

\pagebreak

\renewcommand{\headrulewidth}{0pt} % no horiz. rule at the header
\fancyhf{}
\pagestyle{fancy}

\pars{Oratio ante divinum Officium.}

\lettrine{{\color{red}A}}{peri,} Dómine, os meum ad benedicéndum nomen sanctum tuum:
munda quoque cor meum ab ómnibus vanis, pervérsis, et aliénis
cogitatiónibus:
intelléctum illúmina, afféctum inflámma,
ut digne, atténte ac devóte hoc Offícium recitáre váleam,
et exaudíri mérear ante conspéctum Divínæ Maiestátis tuæ.
Per Christum, Dóminum nostrum.
\Rbardot{} Amen.

Dómine, in unióne illíus divínæ intentiónis,
qua ipse in terris laudes Deo persolvísti,
has tibi Horas \rubricatum{(vel \textnormal{hanc tibi Horam})} persólvo.

%\trOratioAnteOfficium

\vfill

\pars{Oratio post divinum Officium.}

\rubrica{
  Orationem sequentem devote post Officium recitantibus
  Leo Papa X. defectus, et culpas in eo persolvendo ex humana
  fragilitate contractas, indulsit, et dicitur flexis genibus.
}

\lettrine{{\color{red}S}}{acrosánctæ} et indivíduæ Trinitáti,
crucifíxi Dómini nostri Iesu Christi humanitáti,
beatíssimæ et gloriosíssimæ sempérque Vírginis Maríæ
fecúndæ integritáti, 
et ómnium Sanctórum universitáti
sit sempitérna laus, honor, virtus et glória
ab omni creatúra,
nobísque remíssio ómnium peccatórum,
per infiníta sǽcula sæculórum.
\Rbardot{} Amen.

\noindent \Vbardot{} Beáta víscera Maríæ Virginis, quæ portavérunt
ætérni Patris Fílium.\\
\Rbardot{} Et beáta úbera, quæ lactavérunt Christum Dominum.

\rubrica{Et dicitur secreto \textnormal{Pater noster.} et \textnormal{Ave María.}}

%\trOratioPostOfficium

\vfill

\hora{Ad I. Vesperas.} %%%%%%%%%%%%%%%%%%%%%%%%%%%%%%%%%%%%%%%%%%%%%%%%%%%%%
%\sideThumbs{I. Vesperæ}

\cantusSineNeumas

\vspace{0.5cm}
\grechangedim{interwordspacetext}{0.18 cm plus 0.15 cm minus 0.05 cm}{scalable}%
\cuminitiali{}{temporalia/deusinadiutorium-solemnis.gtex}
\grechangedim{interwordspacetext}{0.22 cm plus 0.15 cm minus 0.05 cm}{scalable}%

\vfill
\pagebreak

\pars{Psalmus 1.} \scriptura{Ps. 144, 13; \textbf{H100}}

\vspace{-4mm}

\antiphona{VII c\textsuperscript{2}}{temporalia/ant-regnumtuum.gtex}

\scriptura{Psalmus 144, 10-21.}

\initiumpsalmi{temporalia/ps144ii-initium-vii-c2-auto.gtex}

%\psalmusEtTranslatioT{temporalia/ps144ii-VII-comb.tex}{10cm}
\input{temporalia/ps144ii-VII.tex} \Abardot{}

\vspace{-1cm}

\vfill
\pagebreak

\pars{Psalmus 2.} \scriptura{Ps. 145, 2; \textbf{H100}}

\vspace{-4mm}

\antiphona{IV E}{temporalia/ant-laudabodeum.gtex}

\scriptura{Psalmus 145.}

\initiumpsalmi{temporalia/ps145-initium-iv-E-auto.gtex}

%\psalmusEtTranslatioT{temporalia/ps145-VII-comb.tex}{10cm}
\input{temporalia/ps145-VII.tex} \Abardot{}

\vfill
\pagebreak

\pars{Psalmus 3.} \scriptura{Ps. 146, 1; \textbf{H101}}

\vspace{-4mm}

\antiphona{VIII a}{temporalia/ant-deonostro.gtex}

\scriptura{Psalmus 146.}

\initiumpsalmi{temporalia/ps146-initium-viii-A-auto.gtex}

%\psalmusEtTranslatioT{temporalia/ps146-VII-comb.tex}{10cm}
\input{temporalia/ps146-VII.tex} \Abardot{}

\vfill
\pagebreak

\pars{Psalmus 4.} \scriptura{Ps. 147, 1}

\vspace{-4mm}

\antiphona{E}{temporalia/ant-laudajerusalem.gtex}

\scriptura{Psalmus 147.}

\initiumpsalmi{temporalia/ps147-initium-e-auto.gtex}

%\psalmusEtTranslatioT{temporalia/ps147-VII-comb.tex}{10cm}
\input{temporalia/ps147-VII.tex} \Abardot{}

\vfill
\pagebreak

\pars{Capitulum.} \scriptura{Rom. 11, 33}

\grechangedim{interwordspacetext}{0.12 cm plus 0.15 cm minus 0.05 cm}{scalable}%
\cuminitiali{}{temporalia/capitulum-OAltitudo.gtex}
\grechangedim{interwordspacetext}{0.22 cm plus 0.15 cm minus 0.05 cm}{scalable}

% preklad Jeruz. bible
%\trCapituliI

\vfill

\pars{Responsorium breve.} \scriptura{Ps. 146, 5}

\cuminitiali{VI}{temporalia/resp-magnusdominusnoster.gtex}

%\trResp

\vfill
\pagebreak

\pars{Hymnus} \scriptura{Ambrosius (\olddag{} 397)}

\cuminitiali{I}{temporalia/hym-OLuxBeata-aestivalis.gtex}
\vspace{-3mm}
%\input{hym-OLuxBeata-bohtext.tex}

\vfill
%\pagebreak

\pars{Versus.}

% Versus. %%%
\sineinitiali{temporalia/versus-vespertina.gtex}

%\noindent \trVersus

\vfill
\pagebreak

\magnificati

\vfill
\pagebreak

%\sideThumbs{{\scriptsize{}Fine horarum}}

\anteOrationem

\pagebreak

% Oratio. %%%
\oratioLaudes

\vspace{-1mm}
%\trOrationisI

\vfill

\rubrica{Hebdomadarius dicit iterum Dominus vobiscum, vel cantor dicit:}

\vspace{2mm}

\sineinitiali{temporalia/domineexaudi.gtex}

\rubrica{Postea cantatur a cantore:}

\vspace{2mm}

\cuminitiali{I}{temporalia/benedicamus-dominica-perannum.gtex}

\vspace{1mm}

\vfill
\pagebreak

\hora{Ad Matutinum.} %%%%%%%%%%%%%%%%%%%%%%%%%%%%%%%%%%%%%%%%%%%%%%%%%%%%%
%\sideThumbs{Matutinum}

\vspace{2mm}

\cuminitiali{}{temporalia/dominelabiamea.gtex}

\vspace{2mm}

\pars{Invitatorium.} \scriptura{Ps. 94, 1; Psalmus 94}

\vspace{-6mm}

\antiphona{E}{temporalia/inv-veniteexsultemus.gtex}

\vfill
\pagebreak

\pars{Hymnus.} \scriptura{Adamus Sancti Victoris (\olddag 1146)}

\vspace{-5mm}

\antiphona{VII}{temporalia/hym-SalveDies.gtex}

\scriptura{Non dicitur \textnormal{Amen} in fine.}
%{
%\vspace{-5mm}
%\setlength{\columnsep}{0pt} % prostor mezi sloupci
%\input{hym-SalveDies-bohtext.tex}
%\setlength{\columnsep}{30pt} % prostor mezi sloupci
%}

\vfill
\pagebreak

\subhora{In I. Nocturno}

\pars{Psalmus 1.} \scriptura{Ps. 1, 1}

\vspace{-4mm}

\antiphona{VIII G}{temporalia/ant-beatusvir.gtex}

%\vspace{-5mm}

\scriptura{Ps. 1}

%\vspace{-2mm}

\initiumpsalmi{temporalia/ps1-initium-viii-G-auto.gtex}

%\psalmusEtTranslatioT{temporalia/ps1-I-comb.tex}{10cm}
\input{temporalia/ps1-I.tex} \Abardot{}

\vfill
\pagebreak

\pars{Psalmus 2.} \scriptura{Ps. 2, 11; \textbf{H93}}

\vspace{-4mm}

\antiphona{VII a}{temporalia/ant-servitedomino.gtex}

\vspace{-3mm}

\scriptura{Ps. 2}

\vspace{-2mm}

\initiumpsalmi{temporalia/ps2-initium-vii-a-auto.gtex}

%\psalmusEtTranslatioT{temporalia/ps2-I-comb.tex}{10cm}
\input{temporalia/ps2-I.tex} \Abardot{}

\vfill
\pagebreak

\pars{Psalmus 3.} \scriptura{Ps. 3, 7}

\vspace{-4mm}

\antiphona{VI F}{temporalia/ant-exsurgedominesalvum.gtex}

%\vspace{-5mm}

\scriptura{Ps. 3}

\initiumpsalmi{temporalia/ps3-initium-vi-F-auto.gtex}

%\psalmusEtTranslatioT{temporalia/ps3-I-comb.tex}{10cm}
\input{temporalia/ps3-I.tex} \Abardot{}

\vfill
\pagebreak

\pars{Versus.} \scriptura{Ps. 118, 55}

% Versus. %%%
\sineinitiali{temporalia/versus-memorfui.gtex}

\vspace{5mm}

\sineinitiali{temporalia/oratiodominica-mat.gtex}

\vspace{5mm}

\pars{Absolutio.}

\cuminitiali{}{temporalia/absolutio-exaudi.gtex}

\vfill
\pagebreak

\cuminitiali{}{temporalia/benedictio-solemn-benedictione.gtex}

\vspace{7mm}

\lectioi

\noindent \Vbardot{} Tu autem, Dómine, miserére nobis.
\noindent \Rbardot{} Deo grátias.

\vfill
\pagebreak

\responsoriumi

\vfill
\pagebreak

\cuminitiali{}{temporalia/benedictio-solemn-unigenitus.gtex}

\vspace{7mm}

\lectioii

\noindent \Vbardot{} Tu autem, Dómine, miserére nobis.
\noindent \Rbardot{} Deo grátias.

\vfill
\pagebreak

\responsoriumii

\vfill
\pagebreak

\cuminitiali{}{temporalia/benedictio-solemn-spiritus.gtex}

\vspace{7mm}

\lectioiii

\noindent \Vbardot{} Tu autem, Dómine, miserére nobis.
\noindent \Rbardot{} Deo grátias.

\vfill
\pagebreak

\responsoriumiii

\vfill
\pagebreak

\subhora{In II. Nocturno}

\pars{Psalmus 4.} \scriptura{Ps. 8, 2}

\vspace{-4mm}

\antiphona{I g}{temporalia/ant-quamadmirabileest.gtex}

%\vspace{-5mm}

\scriptura{Ps. 8}

%A\vspace{-2mm}

\initiumpsalmi{temporalia/ps8-initium-i-g-auto.gtex}

%\psalmusEtTranslatioT{temporalia/ps8-I-comb.tex}{10cm}
\input{temporalia/ps8-I.tex} \Abardot{}

\vfill
\pagebreak

\pars{Psalmus 5.} \scriptura{Ps. 9, 5}

\vspace{-4mm}

\antiphona{VIII G}{temporalia/ant-sedistisuperthronum.gtex}

%\vspace{-5mm}

\scriptura{Ps. 9, 2-11}

\initiumpsalmi{temporalia/ps9ii_xi-initium-viii-G-auto.gtex}

%\psalmusEtTranslatioT{temporalia/ps9ii_xi-I-comb.tex}{10cm}
\input{temporalia/ps9ii_xi-I.tex} \Abardot{}

\vfill
\pagebreak

\pars{Psalmus 6.} \scriptura{Ps. 9, 20}

\vspace{-4mm}

\antiphona{I g\textsuperscript{3}}{temporalia/ant-exsurgedominenon.gtex}

%\vspace{-5mm}

\scriptura{Ps. 9, 12-21}

\initiumpsalmi{temporalia/ps9xii_xxi-initium-i-g3-auto.gtex}

%\psalmusEtTranslatioT{temporalia/ps9xii_xxi-I-comb.tex}{10cm}
\input{temporalia/ps9xii_xxi-I.tex} \Abardot{}

\vfill
\pagebreak

\pars{Versus.} \scriptura{Ps. 118, 62}

% Versus. %%%
\sineinitiali{temporalia/versus-medianocte.gtex}

\vspace{5mm}

\sineinitiali{temporalia/oratiodominica-mat.gtex}

\vspace{5mm}

\pars{Absolutio.}

\cuminitiali{}{temporalia/absolutio-ipsius.gtex}

\vfill
\pagebreak

\cuminitiali{}{temporalia/benedictio-solemn-deus.gtex}

\vspace{7mm}

\lectioiv

\noindent \Vbardot{} Tu autem, Dómine, miserére nobis.
\noindent \Rbardot{} Deo grátias.

\vfill
\pagebreak

\responsoriumiv

\vfill
\pagebreak

\cuminitiali{}{temporalia/benedictio-solemn-christus.gtex}

\vspace{7mm}

\lectiov

\noindent \Vbardot{} Tu autem, Dómine, miserére nobis.
\noindent \Rbardot{} Deo grátias.

\vfill
\pagebreak

\responsoriumv

\vfill
\pagebreak

\cuminitiali{}{temporalia/benedictio-solemn-ignem.gtex}

\vspace{7mm}

\lectiovi

\noindent \Vbardot{} Tu autem, Dómine, miserére nobis.
\noindent \Rbardot{} Deo grátias.

\vfill
\pagebreak

\responsoriumvi

\vfill
\pagebreak

\subhora{In III. Nocturno}

\pars{Psalmus 7.} \scriptura{Ps. 9, 22}

\vspace{-4mm}

\antiphona{II D}{temporalia/ant-utquiddomine.gtex}

\vspace{-4mm}

\scriptura{Ps. 9, 22-32}

%\vspace{-2mm}

\initiumpsalmi{temporalia/ps9xxii_xxxii-initium-ii-D-auto.gtex}

%\psalmusEtTranslatioT{temporalia/ps9xxii_xxxii-I-comb.tex}{10cm}
\input{temporalia/ps9xxii_xxxii-I.tex} \Abardot{}

\vfill
\pagebreak

\pars{Psalmus 8.}\scriptura{Ex. 15, 18}

\vspace{-4mm}

\antiphona{IV* e}{temporalia/ant-inaeternum.gtex}

%\vspace{-4mm}

\scriptura{Ps. 9, 33-39}

\initiumpsalmi{temporalia/ps9xxxiii_xxxix-initium-iv_-e-auto.gtex}

%\psalmusEtTranslatioT{temporalia/ps9xxxiii_xxxix-I-comb.tex}{10cm}
\input{temporalia/ps9xxxiii_xxxix-I.tex} \Abardot{}

\vfill
\pagebreak

\pars{Psalmus 9.} \scriptura{Ps. 10, 8}

\vspace{-4mm}

\antiphona{II* f}{temporalia/ant-justusdominus.gtex}

%\vspace{-4mm}

\scriptura{Ps. 10}

%\initiumpsalmi{temporalia/ps10-initium-iv-c-auto.gtex}
\initiumpsalmi{temporalia/ps10-initium-ii_-f.gtex}

%\psalmusEtTranslatioT{temporalia/ps10-I-comb.tex}{10cm}
\input{temporalia/ps10-I.tex} \Abardot{}

\vfill
\pagebreak

\pars{Versus.} \scriptura{Ps. 118, 148}

% Versus. %%%
\sineinitiali{temporalia/versus-praevenerunt.gtex}

\vspace{5mm}

\sineinitiali{temporalia/oratiodominica-mat.gtex}

\vspace{5mm}

\pars{Absolutio.}

\cuminitiali{}{temporalia/absolutio-avinculis.gtex}

\vfill
\pagebreak

\cuminitiali{}{temporalia/benedictio-solemn-evangelica.gtex}

\vspace{7mm}

\lectiovii

\noindent \Vbardot{} Tu autem, Dómine, miserére nobis.
\noindent \Rbardot{} Deo grátias.

\vfill
\pagebreak

\responsoriumvii

\vfill
\pagebreak

\cuminitiali{}{temporalia/benedictio-solemn-divinum.gtex}

\vspace{7mm}

\lectioviii

\noindent \Vbardot{} Tu autem, Dómine, miserére nobis.
\noindent \Rbardot{} Deo grátias.

\vfill
\pagebreak

\responsoriumviii

\vfill
\pagebreak

\cuminitiali{}{temporalia/benedictio-solemn-adsocietatem.gtex}

\vspace{7mm}

\lectioix

\noindent \Vbardot{} Tu autem, Dómine, miserére nobis.
\noindent \Rbardot{} Deo grátias.

\vfill
\pagebreak

% Te Deum

{
\pars{Hymnus Ambrosianus} \scriptura{Tonus Solemnis}

\vspace{-2mm}

\grechangedim{interwordspacetext}{0.26 cm plus 0.15 cm minus 0.05 cm}{scalable}%
\cuminitiali{III}{temporalia/tedeum-solemnis-gn.gtex}
\grechangedim{interwordspacetext}{0.22 cm plus 0.15 cm minus 0.05 cm}{scalable}%
}

\vfill
\pagebreak

\rubrica{Reliqua omittuntur, nisi Laudes separandæ sint.}

\pars{Oratio}

\noindent \Vbardot{} Dómine, exáudi oratiónem meam.

\noindent \Rbardot{} Et clamor meus ad te véniat.

Orémus:

\oratioLaudes

\vspace{7mm}

\pars{Conclusio}

\noindent \Vbardot{} Dómine, exáudi oratiónem meam.

\noindent \Rbardot{} Et clamor meus ad te véniat.

\noindent \Vbardot{} Benedicámus Dómino, allelúia, allelúia.

\noindent \Rbardot{} Deo grátias, allelúia, allelúia.

\noindent \Vbardot{} Fidélium ánimæ per misericórdiam Dei requiéscant in pace.

\noindent \Rbardot{} Amen.

\vfill
\pagebreak

\hora{Ad Laudes.} %%%%%%%%%%%%%%%%%%%%%%%%%%%%%%%%%%%%%%%%%%%%%%%%%%%%%
%\sideThumbs{Laudes}

\cantusSineNeumas

\vspace{0.5cm}
\grechangedim{interwordspacetext}{0.18 cm plus 0.15 cm minus 0.05 cm}{scalable}%
\cuminitiali{}{temporalia/deusinadiutorium-alter.gtex}
\grechangedim{interwordspacetext}{0.22 cm plus 0.15 cm minus 0.05 cm}{scalable}%

\vfill
%\pagebreak

\pars{Psalmus 1.}

\vspace{-4mm}

\antiphona{VI F}{temporalia/ant-alleluia1.gtex}

\scriptura{Psalmus 50.}

\initiumpsalmi{temporalia/ps50-initium-vi-F-auto.gtex}

%\psalmusEtTranslatioT{temporalia/ps50-I-comb.tex}{10cm}
\input{temporalia/ps50-I.tex}

\vfill
\pagebreak

\pars{Psalmus 2.}

\scriptura{Psalmus 117.}

\initiumpsalmi{temporalia/ps117-initium-vi-F-auto.gtex}

%\psalmusEtTranslatioT{temporalia/ps117-I-comb.tex}{10cm}
\input{temporalia/ps117-I.tex}

\vfill
\pagebreak

\pars{Psalmus 3.}

\scriptura{Psalmus 62.}

\initiumpsalmi{temporalia/ps62-initium-vi-F-auto.gtex}

%\psalmusEtTranslatioT{temporalia/ps62-I-comb.tex}{10cm}
\input{temporalia/ps62-I.tex}

\vfill

\vspace{-6mm}

\antiphona{}{temporalia/ant-alleluia1.gtex} % repeat the antiphon - new page

\vfill
\pagebreak

\pars{Psalmus 4.} \scriptura{Dan. 3, 22-26; \textbf{H422}}

\vspace{-4mm}

\antiphona{VIII G}{temporalia/ant-trespueri.gtex}

\scriptura{Canticum trium puerorum, Dan. 3, 57-88 et 56}

\initiumpsalmi{temporalia/dan3-initium-viii-G-auto.gtex}

%\psalmusEtTranslatioT{temporalia/dan3-comb.tex}{10cm}
\input{temporalia/dan3.tex}

\rubrica{Hic non dicitur Gloria Patri, neque Amen.}

\vfill

\vspace{-6mm}

\antiphona{}{temporalia/ant-trespueri.gtex} % repeat the antiphon - new page

\vfill
\pagebreak

\pars{Psalmus 5.}

\vspace{-4mm}

\antiphona{VIII G}{temporalia/ant-alleluia2.gtex}

\scriptura{Psalmus 148.}

\initiumpsalmi{temporalia/ps148-initium-viii-G-auto.gtex}

%\psalmusEtTranslatioT{temporalia/ps148-I-comb.tex}{10cm}
\input{temporalia/ps148-I.tex}

\rubrica{Hic non dicitur Gloria Patri.}

\vfill
\pagebreak

%
\scriptura{Psalmus 149.}

\initiumpsalmi{temporalia/ps149-initium-viii-G-auto.gtex}

%\psalmusEtTranslatioT{temporalia/ps149-I-comb.tex}{10cm}
\input{temporalia/ps149-I.tex}

\rubrica{Hic non dicitur Gloria Patri.}

\vfill
\pagebreak

%
\scriptura{Psalmus 150.}

\initiumpsalmi{temporalia/ps150-initium-viii-G-auto.gtex}

%\psalmusEtTranslatioT{temporalia/ps150-I-comb.tex}{10cm}
\input{temporalia/ps150-I.tex}

\vfill

\vspace{-6mm}

\antiphona{}{temporalia/ant-alleluia2.gtex} % repeat the antiphon - new page

\vfill
\pagebreak

\pars{Capitulum.} \scriptura{Ac. 7, 12}

\grechangedim{interwordspacetext}{0.12 cm plus 0.15 cm minus 0.05 cm}{scalable}%
\cuminitiali{}{temporalia/capitulum-Benedictio.gtex}
\grechangedim{interwordspacetext}{0.22 cm plus 0.15 cm minus 0.05 cm}{scalable}

% preklad Jeruz. bible
%\trCapituliI

\vfill

\pars{Responsorium breve.} \scriptura{Ps. 118, 36-37}

\cuminitiali{IV}{temporalia/resp-inclinacormeum.gtex}

%\trResp

\vfill
\pagebreak

\pars{Hymnus} \scriptura{Gregorius Magnus (\olddag{} 604)}

\cuminitiali{IV}{temporalia/hym-EcceJamNoctis.gtex}
\vspace{-3mm}
%\input{hym-EcceJamNocis-bohtext.tex}

\vfill
%\pagebreak

\pars{Versus.} \scriptura{Ps. 92, 1}

% Versus. %%%
\sineinitiali{temporalia/versus-dominusregnavit.gtex}

%\noindent \trVersus

\vfill
\pagebreak

\benedictus

\vspace{-1cm}

\vfill
\pagebreak

%\sideThumbs{{\scriptsize{}Fine horarum}}

\anteOrationem

\pagebreak

% Oratio. %%%
\oratioLaudes

\vspace{-1mm}
%\trOrationisI

\vfill

\rubrica{Hebdomadarius dicit iterum Dominus vobiscum, vel cantor dicit:}

\vspace{2mm}

\sineinitiali{temporalia/domineexaudi.gtex}

\rubrica{Postea cantatur a cantore:}

\vspace{2mm}

\cuminitiali{I}{temporalia/benedicamus-dominica-perannum.gtex}

\vspace{1mm}

\vfill
\pagebreak

\hora{Ad II. Vesperas.} %%%%%%%%%%%%%%%%%%%%%%%%%%%%%%%%%%%%%%%%%%%%%%%%%%%%%
%\sideThumbs{II. Vesperæ}

\cantusSineNeumas

%\vspace{0.5cm}
\grechangedim{interwordspacetext}{0.18 cm plus 0.15 cm minus 0.05 cm}{scalable}%
\cuminitiali{}{temporalia/deusinadiutorium-solemnis.gtex}
\grechangedim{interwordspacetext}{0.22 cm plus 0.15 cm minus 0.05 cm}{scalable}%

\vfill
%\pagebreak

\vspace{-2mm}

\pars{Psalmus 1.} \scriptura{Ps. 109, 1; \textbf{H91}}

\vspace{-4mm}

\antiphona{VII c\textsuperscript{2}}{temporalia/ant-dixitdominus.gtex}

\vspace{-4mm}

\scriptura{Psalmus 109.}

\initiumpsalmi{temporalia/ps109-initium-vii-c2-auto.gtex}

%\psalmusEtTranslatioT{temporalia/ps109-I-comb.tex}{10cm}
\input{temporalia/ps109-I.tex} \Abardot{}

\vspace{-1cm}

\vfill
\pagebreak

\pars{Psalmus 2.} \scriptura{Ps. 110, 8; \textbf{H91}}

\vspace{-4mm}

\antiphona{IV g}{temporalia/ant-fideliaomnia.gtex}

\scriptura{Psalmus 110.}

\initiumpsalmi{temporalia/ps110-initium-iv-g-auto.gtex}

%\psalmusEtTranslatioT{temporalia/ps110-I-comb.tex}{10cm}
\input{temporalia/ps110-I.tex} \Abardot{}

\vfill
\pagebreak

\pars{Psalmus 3.} \scriptura{Ps. 111, 1; \textbf{H92}}

\vspace{-4mm}

\antiphona{IV a}{temporalia/ant-inmandatis.gtex}

\scriptura{Psalmus 111.}

\initiumpsalmi{temporalia/ps111-initium-iv-a-auto.gtex}

%\psalmusEtTranslatioT{temporalia/ps111-I-comb.tex}{10cm}
\input{temporalia/ps111-I.tex} \Abardot{}

\vfill
\pagebreak

\pars{Psalmus 4.} \scriptura{Ps. 112, 2; \textbf{H92}}

\vspace{-4mm}

\antiphona{VII c}{temporalia/ant-sitnomendomini.gtex}

\scriptura{Psalmus 112.}

\initiumpsalmi{temporalia/ps112-initium-vii-c-auto.gtex}

%\psalmusEtTranslatioT{temporalia/ps112-I-comb.tex}{10cm}
\input{temporalia/ps112-I.tex} \Abardot{}

\vfill
\pagebreak

\pars{Capitulum.} \scriptura{2 Cor. 1, 3-4}

\grechangedim{interwordspacetext}{0.12 cm plus 0.15 cm minus 0.05 cm}{scalable}%
\cuminitiali{}{temporalia/capitulum-BenedictusDeus.gtex}
\grechangedim{interwordspacetext}{0.22 cm plus 0.15 cm minus 0.05 cm}{scalable}

% preklad Jeruz. bible
%\trCapituliI

\vfill

\pars{Responsorium breve.} \scriptura{Ps. 103, 24}

\cuminitiali{VI}{temporalia/resp-quammagnificata.gtex}

%\trResp

\vfill
\pagebreak

\pars{Hymnus} \scriptura{Gregorius Magnus (\olddag{} 604)}

\cuminitiali{I}{temporalia/hym-LucisCreator-aestivalis.gtex}
\vspace{-3mm}
%\begin{translatioMulticol}{3}
Tvůrce světa předobrý,\\
tys ustanovil denní řád\\
a proudy světla rozhodil,\\
když světu základy jsi klad.\\
\\
A spojils ráno s večerem\\
a dnem tu dobu nazýváš;\\
hle padá temné noci stín -\\
slyš prosbu, vyslyš nářek náš.\columnbreak

Ach, nedej, by nás stihla smrt,\\
když svědomí nám tíží hřích,\\
když nemyslíme na věčnost\\
v té síti hříchů šalebných.\\
\\
Vzbuď naši touhu po nebi,\\
kde věčný život čeká nás,\\
a pomoz odložit vše zlé\\
a smýti z duše každý kaz.\columnbreak

To splň nám, dobrý Otče náš,\\
i ty, jenž rovné božství máš,\\
i Duchu, který těšíš nás\\
a vládneš, Bože, v každý čas.\\
Amen. 
\end{translatioMulticol}


\vfill
%\pagebreak

\pars{Versus.} \scriptura{Ps. 140, 2}

% Versus. %%%
\sineinitiali{temporalia/versus-dirigatur.gtex}

%\noindent \trVersus

\vfill
\pagebreak

\magnificatii

\vfill
\pagebreak

%\sideThumbs{{\scriptsize{}Fine horarum}}

\anteOrationem

\pagebreak

% Oratio. %%%
\oratioLaudes

\vspace{-1mm}
%\trOrationisI

\vfill

\rubrica{Hebdomadarius dicit iterum Dominus vobiscum, vel cantor dicit:}

\vspace{2mm}

\sineinitiali{temporalia/domineexaudi.gtex}

\rubrica{Postea cantatur a cantore:}

\vspace{2mm}

\cuminitiali{I}{temporalia/benedicamus-dominica-perannum.gtex}

\vspace{1mm}

\end{document}

