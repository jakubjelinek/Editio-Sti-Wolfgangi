\newcommand{\hymnusmatutinum}{\pars{Hymnus}

\cuminitiali{IV}{temporalia/hym-SomnoRefectis.gtex}}
\newcommand{\lectioi}{\pars{Lectio I.} \scriptura{1 Sam. 31, 1-4; 2 Sam. 1, 1-4}

\noindent De libris Samuélis.

\noindent In diébus illis: Philísthim pugnábant advérsum Israel; et fugérunt viri Israel ante fáciem Philísthim et cecidérunt interfécti in monte Gélboe. Irruerúntque Philísthim in Saul et fílios eius et percussérunt Iónathan et Abínadab et Melchísua fílios Saul.

\noindent Totúmque pondus prœ́lii versum est in Saul; et consecúti sunt eum viri arcu, et vulnerátus est veheménter a sagittáriis. Dixítque Saul ad armígerum suum: «Evagína gládium tuum et pércute me, ne forte véniant incircumcísi isti et confódiant me et illúdant mihi». Et nóluit ármiger eius; erat enim nímio timóre pertérritus. Arrípuit ítaque Saul gládium et írruit super eum.

\noindent Factum est autem, postquam mórtuus est Saul, ut David reverterétur a cæde Amalec et manéret in Síceleg dies duos. In die autem tértia appáruit homo véniens de castris Saul veste conscíssa et púlvere aspérsus caput; et, ut venit ad David, cécidit super fáciem suam et adorávit. Dixítque ad eum David: «Unde venis?». Qui ait ad eum: «De castris Israel fugi». Et dixit ad eum David: «Quid enim factum est? Indica mihi». Qui ait: «Fugit pópulus ex prœ́lio, et multi corruéntes e pópulo mórtui sunt; sed et Saul et Iónathan fílius eius interiérunt».}
\newcommand{\responsoriumi}{\pars{Responsorium 1.} \scriptura{\Rbardot{} 1 Sam. 18, 7 \Vbardot{} ibid.}

\vspace{-5mm}

\responsorium{VIII}{temporalia/resp-percussitsaulmille2.gtex}{}}
\newcommand{\lectioii}{\pars{Lectio II.} \scriptura{2 Sam. 1, 5-16}

\noindent Dixítque David ad adulescéntem, qui nuntiábat ei: «Unde scis quia mórtuus est Saul et Iónathan fílius eius?». Et ait aduléscens, qui narrábat ei: «Casu veni in montem Gélboe, et Saul incumbébat super hastam suam. Porro currus et équites appropinquábant ei, et convérsus post tergum suum vidénsque me vocávit. Cui cum respondíssem: Adsum, dixit mihi: “Quisnam es tu?”. Et dixi ad eum: Amalecítes ego sum. Et locútus est mihi: “Sta super me et intérfice me, quóniam tenent me angústiæ, et adhuc tota ánima mea in me est”. Stansque super eum occídi illum; sciébam enim quod vívere non póterat post ruínam. Et tuli diadéma, quod erat in cápite eius, et armíllam de bráchio illíus et áttuli ad te dóminum meum huc».

\noindent Apprehéndens autem David vestiménta sua scidit omnésque viri, qui erant cum eo; et planxérunt et flevérunt et ieiunavérunt usque ad vésperam super Saul et super Iónathan fílium eius et super pópulum Dómini et super domum Israel, quod corruíssent gládio.

\noindent Dixítque David ad iúvenem, qui nuntiáverat ei: «Unde es?». Qui respóndit: «Fílius hóminis ádvenæ Amalecítæ ego sum». Et ait ad eum David: «Quare non timuísti míttere manum tuam, ut occíderes christum Dómini?». Vocánsque David unum de púeris ait: «Accédens írrue in eum». Qui percússit illum, et mórtuus est. Et ait ad eum David: «Sanguis tuus super caput tuum; os enim tuum locútum est advérsum te dicens: “Ego interféci christum Dómini”».}
\newcommand{\responsoriumii}{\pars{Responsorium 2.} \scriptura{\Rbardot{} 2 Sam. 1, 21; \textbf{H395}}

\vspace{-5mm}

\responsorium{VIII}{temporalia/resp-montesgelboe.gtex}{}}
\newcommand{\lectioiii}{\pars{Lectio III.} \scriptura{Serm. 47, 1. 2. 3. 6, De ovibus: CCL 41, 572-573. 575-576}

\noindent Ex Sermónibus sancti Augustíni epíscopi.

\noindent Verba quæ cantávimus cóntinent professiónem nostram, quia oves Dei sumus: \emph{Quóniam ipse est Dóminus Deus noster, qui fecit nos.} Ille Deus noster; \emph{nos pópulus páscuæ eius et oves mánuum eius.} Pastóres hómines, oves quas habent non ipsi fecérunt, oves quas pascunt non ipsi creavérunt. Noster autem Dóminus Deus, quia Deus et creátor est, fecit sibi oves quas hábeat et quas pascat. Nec alter instítuit quas ipse pascit, nec quas ipse instítuit alter pascit.

\noindent Proféssi ergo in hoc cántico quia oves eius sumus, pópulus páscuæ eius, oves mánuum eius, audiámus quid ad nos loquátur tamquam ad oves suas. Pridem pastóribus loquebátur; nunc autem óvibus lóquitur. In illis ergo eius verbis nos cum tremóre audiebámus, vos cum securitáte. Quid ergo in istis verbis hodiérnis? numquid vicíssim nos cum securitáte, vos cum tremóre? Non útique. Primo, quia et si pastóres sumus, pastor, non solum quod dícitur ad pastóres, cum tremóre audit, sed étiam quod dícitur ad oves. Si enim secúrus audit quod ad oves dícitur, non est illi cura de óvibus. Deínde iam et tunc díximus caritáti vestræ, duo quædam in nobis esse consideránda: unum quod christiáni sumus, álterum quod præpósiti sumus. Quod ergo præpósiti sumus, inter pastóres deputámur, si boni sumus; quod autem christiáni sumus, et nos vobíscum oves sumus. Sive ergo Dóminus pastóribus loquátur sive óvibus, nos ómnia cum tremóre opórtet audíre, nec recédat sollicitúdo de córdibus nostris.

\noindent Audiámus ítaque, fratres, unde Dóminus oves ímprobas corrípiat, et quid promíttat óvibus suis. \emph{Et vos,} inquit, \emph{oves meæ.} Primo, quanta felícitas est esse gregem Dei, si quis cógitet, fratres, étiam in istis lácrimis et in istis tribulatiónibus magnum gáudium cóncipit. Illi enim dictum est: \emph{Qui pascis Israel,} de quo dictum est: \emph{Non dormitábit nec dórmiet qui custódit Israel.} Vígilat ergo ille super nos vigilántibus nobis, vígilat et dormiéntibus nobis. Si ergo de pastóre hómine secúrum est pecus hóminis, quanta esse debet nostra secúritas pascénte Deo, non tantum quia pascit nos, sed étiam quia fecit nos.

\noindent \emph{Et vos,} inquit, \emph{oves meæ, hæc dicit Dóminus Deus: Ecce ego iúdico inter ovem et ovem, et aríetes et hircos.} Quid hic fáciunt hirci in grege Dei? In eísdem páscuis, in eísdem fóntibus, et hirci tamen sinístræ destináti dextris miscéntur, et prius tolerántur qui separabúntur. Et hic exercétur óvium patiéntia ad similitúdinem patiéntiæ Dei. Separátio enim ab illo erit, aliórum ad sinístram, aliórum ad déxteram.}
\newcommand{\responsoriumiii}{\pars{Responsorium 3.} \scriptura{\Rbardot{} 2 Reg. 7, 8.9.12.13; \Vbardot{} ibid., 7, 9.1; \textbf{H395}}

\vspace{-5mm}

\responsorium{I}{temporalia/resp-egotetuli-CROCHU-cumdox.gtex}{}}
\newcommand{\hymnuslaudes}{\pars{Hymnus} \scriptura{Ambrosius (\olddag{} 397)}

\grechangedim{interwordspacetext}{0.16 cm plus 0.15 cm minus 0.05 cm}{scalable}%
\cuminitiali{VIII}{temporalia/hym-SplendorPaternae-einsiedeln.gtex}
\grechangedim{interwordspacetext}{0.22 cm plus 0.15 cm minus 0.05 cm}{scalable}}
\newcommand{\hebdomada}{infra Hebdom. XIII per Annum.}
\newcommand{\matua}{Matutinum Hebdomadae A}
\newcommand{\matuac}{Matutinum Hebdomadae A vel C}
\newcommand{\lauda}{Laudes Hebdomadae A}
\newcommand{\laudac}{Laudes Hebdomadae A vel C}

% LuaLaTeX

\documentclass[a4paper, twoside, 12pt]{article}
\usepackage[latin]{babel}
%\usepackage[landscape, left=3cm, right=1.5cm, top=2cm, bottom=1cm]{geometry} % okraje stranky
%\usepackage[landscape, a4paper, mag=1166, truedimen, left=2cm, right=1.5cm, top=1.6cm, bottom=0.95cm]{geometry} % okraje stranky
\usepackage[landscape, a4paper, mag=1400, truedimen, left=0.5cm, right=0.5cm, top=0.5cm, bottom=0.5cm]{geometry} % okraje stranky

\usepackage{fontspec}
\setmainfont[FeatureFile={junicode.fea}, Ligatures={Common, TeX}, RawFeature=+fixi]{Junicode}
%\setmainfont{Junicode}

% shortcut for Junicode without ligatures (for the Czech texts)
\newfontfamily\nlfont[FeatureFile={junicode.fea}, Ligatures={Common, TeX}, RawFeature=+fixi]{Junicode}

\usepackage{multicol}
\usepackage{color}
\usepackage{lettrine}
\usepackage{fancyhdr}

% usual packages loading:
\usepackage{luatextra}
\usepackage{graphicx} % support the \includegraphics command and options
\usepackage{gregoriotex} % for gregorio score inclusion
\usepackage{gregoriosyms}
\usepackage{wrapfig} % figures wrapped by the text
\usepackage{parcolumns}
\usepackage[contents={},opacity=1,scale=1,color=black]{background}
\usepackage{tikzpagenodes}
\usepackage{calc}
\usepackage{longtable}
\usetikzlibrary{calc}

\setlength{\headheight}{14.5pt}

\input{conventuscommune.tex} % Often used macros

\newcommand{\annusEditionis}{2021}

%%%% Vicekrat opakovane kousky

\newcommand{\anteOrationem}{
  \rubrica{Ante Orationem, cantatur a Superiore:}

  \pars{Supplicatio Litaniæ.}

  \cuminitiali{}{temporalia/supplicatiolitaniae.gtex}

  \pars{Oratio Dominica.}

  \cuminitiali{}{temporalia/oratiodominica.gtex}

  \rubrica{Deinde dicitur ab Hebdomadario:}

  \cuminitiali{}{temporalia/dominusvobiscum-solemnis.gtex}

  \rubrica{In choro monialium loco Dominus vobiscum dicitur:}

  \sineinitiali{temporalia/domineexaudi.gtex}
}

\setlength{\columnsep}{30pt} % prostor mezi sloupci

%%%%%%%%%%%%%%%%%%%%%%%%%%%%%%%%%%%%%%%%%%%%%%%%%%%%%%%%%%%%%%%%%%%%%%%%%%%%%%%%%%%%%%%%%%%%%%%%%%%%%%%%%%%%%
\begin{document}

% Here we set the space around the initial.
% Please report to http://home.gna.org/gregorio/gregoriotex/details for more details and options
\grechangedim{afterinitialshift}{2.2mm}{scalable}
\grechangedim{beforeinitialshift}{2.2mm}{scalable}
\grechangedim{interwordspacetext}{0.22 cm plus 0.15 cm minus 0.05 cm}{scalable}%
\grechangedim{annotationraise}{-0.2cm}{scalable}

% Here we set the initial font. Change 38 if you want a bigger initial.
% Emit the initials in red.
\grechangestyle{initial}{\color{red}\fontsize{38}{38}\selectfont}

\pagestyle{empty}

%%%% Titulni stranka
\begin{titulusOfficii}
\ifx\titulus\undefined
\nomenFesti{Feria II \hebdomada{}}
\else
\titulus
\fi
\end{titulusOfficii}

\vfill

\begin{center}
%Ad usum et secundum consuetudines chori \guillemotright{}Conventus Choralis\guillemotleft.

%Editio Sancti Wolfgangi \annusEditionis
\end{center}

\scriptura{}

\pars{}

\pagebreak

\renewcommand{\headrulewidth}{0pt} % no horiz. rule at the header
\fancyhf{}
\pagestyle{fancy}

\cantusSineNeumas

\ifx\oratio\undefined
\ifx\laudb\undefined
\else
\newcommand{\oratio}{\pars{Oratio.}

\noindent Dómine Deus omnípotens, qui ad princípium huius diéi nos perveníre fecísti, tua nos hódie salva virtúte, ut in hac die ad nullum declinémus peccátum, sed semper ad tuam iustítiam faciéndam nostra procédant elóquia, dirigántur cogitatiónes et ópera.

\noindent Per Dóminum nostrum Iesum Christum, Fílium tuum, qui tecum vivit et regnat in unitáte Spíritus Sancti, Deus, per ómnia sǽcula sæculórum.

\noindent \Rbardot{} Amen.}
\fi
\fi

\hora{Ad Matutinum.} %%%%%%%%%%%%%%%%%%%%%%%%%%%%%%%%%%%%%%%%%%%%%%%%%%%%%
%\sideThumbs{Matutinum}

\vspace{2mm}

\cuminitiali{}{temporalia/dominelabiamea.gtex}

\vfill
%\pagebreak

\vspace{2mm}

\ifx\invitatorium\undefined
\pars{Invitatorium.} \scriptura{Ps. 94, 1; Psalmus 94; \textbf{H451}}

\vspace{-6mm}

\antiphona{VI}{temporalia/inv-jubilemusdeo.gtex}\else
\invitatorium
\fi

\vfill
\pagebreak

\ifx\hymnusmatutinum\undefined
\ifx\matua\undefined
\else
\pars{Hymnus.}

{
\grechangedim{interwordspacetext}{0.10 cm plus 0.15 cm minus 0.05 cm}{scalable}%
\antiphona{II}{temporalia/hym-IpsumNunc.gtex}
\grechangedim{interwordspacetext}{0.22 cm plus 0.15 cm minus 0.05 cm}{scalable}%
}
\fi
\else
\hymnusmatutinum
\fi

\vspace{-3mm}

\vfill
\pagebreak

\ifx\matub\undefined
\else
% MAT B
\pars{Psalmus 1.} \scriptura{Ps. 30, 2; \textbf{H90}}

\vspace{-4mm}

\antiphona{VIII G}{temporalia/ant-intuaiustitia.gtex}

%\vspace{-2mm}

\scriptura{Ps. 30, 2-9}

%\vspace{-2mm}

\initiumpsalmi{temporalia/ps30i-initium-viii-G-auto.gtex}

\vspace{-1.5mm}

\input{temporalia/ps30i-viii-G.tex} \Abardot{}

\vfill
\pagebreak

\pars{Psalmus 2.} \scriptura{Ps. 66, 2}

\vspace{-4mm}

\antiphona{E}{temporalia/ant-illuminadomine.gtex}

%\vspace{-2mm}

\scriptura{Ps. 30, 10-17}

%\vspace{-2mm}

\initiumpsalmi{temporalia/ps30ii-initium-e-a-auto.gtex}

\input{temporalia/ps30ii-e-a.tex} \Abardot{}

\vfill
\pagebreak

\pars{Psalmus 3.} \scriptura{Ps. 30, 24}

\vspace{-4mm}

\antiphona{II D}{temporalia/ant-diligitedominum.gtex}

%\vspace{-5mm}

\scriptura{Ps. 30, 20-25}

%\vspace{-2mm}

\initiumpsalmi{temporalia/ps30iii-initium-ii-D-auto.gtex}

\input{temporalia/ps30iii-ii-D.tex} \Abardot{}

\vfill
\pagebreak
\fi

\pars{Versus.}

\ifx\matversus\undefined
\ifx\matub\undefined
\else
\noindent \Vbardot{} Dírige me, Dómine, in veritáte tua, et doce me.

\noindent \Rbardot{} Quia tu es Deus salútis meæ.
\fi
\else
\matversus
\fi

\vspace{5mm}

\sineinitiali{temporalia/oratiodominica-mat.gtex}

\vspace{5mm}

\pars{Absolutio.}

\cuminitiali{}{temporalia/absolutio-exaudi.gtex}

\vfill
\pagebreak

\cuminitiali{}{temporalia/benedictio-solemn-benedictione.gtex}

\vspace{7mm}

\lectioi

\noindent \Vbardot{} Tu autem, Dómine, miserére nobis.
\noindent \Rbardot{} Deo grátias.

\vfill
\pagebreak

\responsoriumi

\vfill
\pagebreak

\cuminitiali{}{temporalia/benedictio-solemn-unigenitus.gtex}

\vspace{7mm}

\lectioii

\noindent \Vbardot{} Tu autem, Dómine, miserére nobis.
\noindent \Rbardot{} Deo grátias.

\vfill
\pagebreak

\responsoriumii

\vfill
\pagebreak

\cuminitiali{}{temporalia/benedictio-solemn-spiritus.gtex}

\vspace{7mm}

\lectioiii

\noindent \Vbardot{} Tu autem, Dómine, miserére nobis.
\noindent \Rbardot{} Deo grátias.

\vfill
\pagebreak

\responsoriumiii

\vfill
\pagebreak

\rubrica{Reliqua omittuntur, nisi Laudes separandæ sint.}

\sineinitiali{temporalia/domineexaudi.gtex}

\vfill

\oratio

\vfill

\noindent \Vbardot{} Dómine, exáudi oratiónem meam.
\Rbardot{} Et clamor meus ad te véniat.

\vfill

\noindent \Vbardot{} Benedicámus Dómino.
\noindent \Rbardot{} Deo grátias.

\vfill

\noindent \Vbardot{} Fidélium ánimæ per misericórdiam Dei requiéscant in pace.
\Rbardot{} Amen.

\vfill
\pagebreak

\hora{Ad Laudes.} %%%%%%%%%%%%%%%%%%%%%%%%%%%%%%%%%%%%%%%%%%%%%%%%%%%%%
%\sideThumbs{Laudes}

\cantusSineNeumas

\vspace{0.5cm}
\grechangedim{interwordspacetext}{0.18 cm plus 0.15 cm minus 0.05 cm}{scalable}%
\cuminitiali{}{temporalia/deusinadiutorium-communis.gtex}
\grechangedim{interwordspacetext}{0.22 cm plus 0.15 cm minus 0.05 cm}{scalable}%

\vfill
\pagebreak

\ifx\hymnuslaudes\undefined
\ifx\laudbd\undefined
\else
\pars{Hymnus} \scriptura{Hilarius (\olddag{} 367)}

\grechangedim{interwordspacetext}{0.16 cm plus 0.15 cm minus 0.05 cm}{scalable}%
\cuminitiali{IV}{temporalia/hym-LucisLargitor.gtex}
\grechangedim{interwordspacetext}{0.22 cm plus 0.15 cm minus 0.05 cm}{scalable}%
\vspace{-3mm}
\fi
\else
\hymnuslaudes
\fi

\vfill
\pagebreak

\ifx\laudb\undefined
\else
\pars{Psalmus 1.} \scriptura{Ps. 41, 3; \textbf{H391}}

\vspace{-4mm}

\antiphona{II D}{temporalia/ant-sitivitanima.gtex}

%\vspace{-2mm}

\scriptura{Psalmus 41}

%\vspace{-2mm}

\initiumpsalmi{temporalia/ps41-initium-ii-D-auto.gtex}

%\vspace{-1.5mm}

\input{temporalia/ps41-ii-D.tex}

\vfill

\antiphona{}{temporalia/ant-sitivitanima.gtex}

\vfill
\pagebreak

\pars{Psalmus 2.}

\vspace{-4mm}

\antiphona{III a}{temporalia/ant-ostendenobisdomine.gtex}

%\vspace{-2mm}

\scriptura{Canticum Ecclesiastici, Sir. 36, 1-7.13-16}

%\vspace{-3mm}

\initiumpsalmi{temporalia/ecclesiastici-initium-iii-a-auto.gtex}

\input{temporalia/ecclesiastici-iii-a.tex} \Abardot{}

\vfill
\pagebreak

\pars{Psalmus 3.}

\vspace{-4mm}

\antiphona{II D}{temporalia/ant-operamanuumeius.gtex}

\scriptura{Psalmus 18, 1-7}

\initiumpsalmi{temporalia/ps18i-initium-ii-D-auto.gtex}

\input{temporalia/ps18i-ii-D.tex} \Abardot{}

\vfill
\pagebreak
\fi

\ifx\lectiobrevis\undefined
\ifx\laudb\undefined
\else
\pars{Lectio Brevis.} \scriptura{Ier. 15, 16}

\noindent Invénti sunt sermónes tui, et comédi eos, et factum est mihi verbum tuum in gáudium et in lætítiam cordis mei, quóniam invocátum est nomen tuum super me, Dómine Deus exercítuum.
\fi
\else
\lectiobrevis
\fi

\vfill

\ifx\responsoriumbreve\undefined
\ifx\laudbd\undefined
\else
\pars{Responsorium breve.} \scriptura{Ps. 32, 1.3}

\cuminitiali{VI}{temporalia/resp-exsultateiusti.gtex}
\fi
\else
\responsoriumbreve
\fi

\vfill
\pagebreak

\ifx\benedictus\undefined
\ifx\laudbd\undefined
\else
\pars{Canticum Zachariæ.} \scriptura{Lc. 1, 68; \textbf{H422}}

\vspace{-4mm}

{
\grechangedim{interwordspacetext}{0.18 cm plus 0.15 cm minus 0.05 cm}{scalable}%
\antiphona{IV E}{temporalia/ant-benedictusdominus.gtex}
\grechangedim{interwordspacetext}{0.22 cm plus 0.15 cm minus 0.05 cm}{scalable}%
}

%\vspace{-3mm}

\scriptura{Lc. 1, 68-79}

%\vspace{-2mm}

\cantusSineNeumas
\initiumpsalmi{temporalia/benedictus-initium-iv-E-auto.gtex}

%\vspace{-1.5mm}

\input{temporalia/benedictus-iv-E.tex} \Abardot{}
\fi
\else
\benedictus
\fi

\vspace{-1cm}

\vfill
\pagebreak

%\sideThumbs{{\scriptsize{}Fine horarum}}

\pars{Preces.}

\sineinitiali{}{temporalia/tonusprecum.gtex}

\ifx\preces\undefined
\ifx\laudb\undefined
\else
\noindent Salvátor noster fecit nos regnum et sacerdótium, ut hóstias Deo acceptábiles offerámus. \gredagger{} Grati ígitur eum invocémus:

\Rbardot{} Serva nos in tuo ministério, Dómine.

\noindent Christe, sacérdos ætérne, qui sanctum pópulo tuo sacerdótium concessísti, \gredagger{} concéde, ut spiritáles hóstias Deo acceptábiles iúgiter offerámus.

\Rbardot{} Serva nos in tuo ministério, Dómine.

\noindent Spíritus tui fructus nobis largíre propítius, \gredagger{} patiéntiam, benignitátem et mansuetúdinem.

\Rbardot{} Serva nos in tuo ministério, Dómine.

\noindent Da nobis te amáre, ut te, qui es cáritas, possideámus, \gredagger{} et bene ágere, ut per vitam étiam nostram te laudémus.

\Rbardot{} Serva nos in tuo ministério, Dómine.

\noindent Quæ frátribus nostris sunt utília, nos quǽrere concéde, \gredagger{} ut salútem facílius consequántur.

\Rbardot{} Serva nos in tuo ministério, Dómine.
\fi
\else
\preces
\fi

\vfill

\pars{Oratio Dominica.}

\cuminitiali{}{temporalia/oratiodominicaalt.gtex}

\vfill
\pagebreak

\rubrica{vel:}

\pars{Supplicatio Litaniæ.}

\cuminitiali{}{temporalia/supplicatiolitaniae.gtex}

\vfill

\pars{Oratio Dominica.}

\cuminitiali{}{temporalia/oratiodominica.gtex}

\vfill
\pagebreak

% Oratio. %%%
\oratio

\vspace{-1mm}

\vfill

\rubrica{Hebdomadarius dicit Dominus vobiscum, vel, absente sacerdote vel diacono, sic concluditur:}

\vspace{2mm}

\antiphona{C}{temporalia/dominusnosbenedicat.gtex}

\rubrica{Postea cantatur a cantore:}

\vspace{2mm}

\cuminitiali{IV}{temporalia/benedicamus-feria-laudes.gtex}

\vspace{1mm}

\vfill
\pagebreak

\end{document}

