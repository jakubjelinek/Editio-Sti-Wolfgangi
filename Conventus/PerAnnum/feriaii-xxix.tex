\newcommand{\hymnusmatutinum}{\pars{Hymnus}

\cuminitiali{IV}{temporalia/hym-SomnoRefectis-worcester.gtex}}
\newcommand{\lectioi}{\pars{Lectio I.} \scriptura{Esth. 3, 1-6}

\noindent De libro Esther.

\noindent In diébus illis: Rex Asuérus exaltávit Aman fílium Amadáthi, qui erat de stirpe Agag, et pósuit sólium eius super omnes príncipes, quos habébat. Cunctíque servi regis, qui in fóribus palátii versabántur, flectébant génua et adorábant Aman; sic enim præcéperat rex pro illo. Solus Mardochǽus non flectébat genu, neque adorábat eum. Cui dixérunt púeri regis, qui ad fores palátii præsidébant: «Cur non obsérvas mandátum regis?». Cumque hoc crébrius dícerent, et ille nollet audíre, nuntiavérunt Aman scire cupiéntes utrum perseveráret in senténtia; díxerat enim eis se esse Iudǽum.

\noindent Cumque Aman experiménto probásset quod Mardochǽus non sibi flécteret genu nec se adoráret, irátus est valde et pro níhilo duxit in unum Mardochǽum míttere manus suas — audíerat enim quod esset gentis Iudǽæ — magísque vóluit omnem Iudæórum, qui erant in regno Asuéri, pérdere natiónem.}
\newcommand{\responsoriumi}{\pars{Responsorium 1.} \scriptura{\Rbardot{} Esth. 13, 9 \Vbardot{} ibid. 13, 17; \textbf{H411}}

\vspace{-5mm}

\responsorium{II}{temporalia/resp-dominerexomnipotens-CROCHU.gtex}{}}
\newcommand{\lectioii}{\pars{Lectio II.} \scriptura{Esth. 3, 7-15}

\noindent Mense primo, cuius vocábulum est Nisan, anno duodécimo regni Asuéri, missa est in urnam sors, quæ dícitur Phur, coram Aman, quo die et quo mense gens Iudæórum debéret intérfici; et exívit dies tértia décima mensis duodécimi, qui vocátur Adar. Dixítque Aman regi Asuéro: «Est pópulus per omnes províncias regni tui dispérsus, segregátus inter pópulos alienísque utens légibus, quas céteri non cognóscunt, ínsuper et regis scita contémnens; non éxpedit regi, ut det illis réquiem. Si tibi placet, scriptis decérne, ut péreat, et decem mília talentórum argénti appéndam arcáriis gazæ tuæ». Tulit ergo rex ánulum, quo utebátur, de manu sua et dedit eum Aman fílio Amadáthi de progénie Agag, hosti Iudæórum. Dixítque ad eum: «Argéntum, quod pollicéris, tuum sit; de pópulo age, quod tibi placet».

\noindent Vocatíque sunt scribæ regis mense primo, tértia décima die eius, et scriptum est, ut iússerat Aman, ad omnes sátrapas regis et duces provinciárum et príncipes diversárum géntium, ut quæque gens légere póterat et audíre pro varietáte linguárum, ex nómine regis Asuéri; et lítteræ ipsíus signátæ ánulo. Missæ sunt epístulæ per cursóres ad univérsas províncias regis, ut pérderent, occíderent atque delérent omnes Iudǽos, a púero usque ad senem, párvulos et mulíeres uno die, hoc est tértio décimo mensis duodécimi, qui vocátur Adar, et bona eórum diríperent.

\noindent Exémplar autem epistulárum ut lex in ómnibus provínciis promulgándum erat, ut scirent omnes pópuli et parárent se ad prædíctam diem. Festinábant cursóres, qui missi erant, regis impérium explére; statímque in Susan pepéndit edíctum, rege et Aman celebránte convívium, dum cívitas ipsa esset conturbáta.}
\newcommand{\responsoriumii}{\pars{Responsorium 2.} \scriptura{\Rbardot{} Esth. 14, 12.13 \Vbardot{} ibid. 14, 12.11; \textbf{H411}}

\vspace{-5mm}

\responsorium{II}{temporalia/resp-confortamerexsanctorum-CROCHU.gtex}{}}
\newcommand{\lectioiii}{\pars{Lectio III.} \scriptura{Ep. 130, 9, 18 – 10, 20: CSEL 44, 60-63}

Ex Epístola sancti Augustíni epíscopi ad Probam

\noindent Semper vitam beátam a Dómino Deo desiderémus et semper orémus. Sed ídeo ab áliis curis atque negótiis, quibus ipsum desidérium quodámmodo tepéscit, certis horis ad negótium orándi mentem revocámus, verbis oratiónis nos admonéntes in id quod desiderámus inténdere, ne quod tepéscere cœ́perat, omníno frigéscat et pénitus exstinguátur, nisi crébrius inflammétur.

\noindent Unde et illud quod Apóstolus ait: \emph{Postulatiónes vestræ innotéscant apud Deum,} non sic accipiéndum est, tamquam Deo innotéscant, qui eas et ántequam essent útique nóverat, sed nobis innotéscant apud Deum per tolerántiam, non apud hómines per iactántiam.

\noindent Quæ cum ita sint, étiam diu oráre cum vacat, id est cum ália bonárum et necessariárum actiónum non impediúntur offícia, quamvis et in eis, ut dixi, desidério illo semper orándum sit, non est ímprobum nec inútile. Neque enim, ut nonnúlli putant, hoc est oráre in multilóquio, si diútius orétur. Aliud est sermo multus, áliud diutúrnus afféctus. Nam et de ipso Dómino scriptum est quod pernoctáverit in orándo, et quod prolíxius oráverit: ubi quid áliud quam nobis præbébat exémplum, in témpore precátor opportúnus, cum Patre exaudítor ætérnus?

\noindent Dicúntur fratres in Ægýpto crebras quidem habére oratiónes, sed eas tamen brevíssimas et raptim quodámmodo iaculátas, ne illa vigilánter erécta, quæ oránti plúrimum necessária est, per productióres moras evanéscat atque hebetétur inténtio. Ac per hoc étiam ipsi satis osténdunt, hanc intentiónem sicut non est obtundénda, si perduráre non potest, ita si perduráverit, non cito esse rumpéndam.

\noindent Absit enim ab oratióne multa locútio, sed non desit multa precátio, si fervens persevérat inténtio. Nam multum loqui est in orándo rem necessáriam supérfluis ágere verbis. Multum autem precári est ad eum, quem precámur, diutúrna et pia cordis excitatióne pulsáre. Nam plerúmque hoc negótium plus gemítibus quam sermónibus ágitur, plus fletu quam affátu. Ponit autem \emph{lácrimas} nostras \emph{in conspéctu} suo, et \emph{gémitus noster non est abscónditus} ab eo, qui ómnia per Verbum cóndidit et humána verba non quærit.}
\newcommand{\responsoriumiii}{\pars{Responsorium 3.} \scriptura{Esth. 13, 15-17; \textbf{H412}}

\vspace{-5mm}

\responsorium{II}{temporalia/resp-misereredominepopulotuo-CROCHU-cumdox.gtex}{}}
\include{hebdomadaxxix}
% LuaLaTeX

\documentclass[a4paper, twoside, 12pt]{article}
\usepackage[latin]{babel}
%\usepackage[landscape, left=3cm, right=1.5cm, top=2cm, bottom=1cm]{geometry} % okraje stranky
%\usepackage[landscape, a4paper, mag=1166, truedimen, left=2cm, right=1.5cm, top=1.6cm, bottom=0.95cm]{geometry} % okraje stranky
\usepackage[landscape, a4paper, mag=1400, truedimen, left=0.5cm, right=0.5cm, top=0.5cm, bottom=0.5cm]{geometry} % okraje stranky

\usepackage{fontspec}
\setmainfont[FeatureFile={junicode.fea}, Ligatures={Common, TeX}, RawFeature=+fixi]{Junicode}
%\setmainfont{Junicode}

% shortcut for Junicode without ligatures (for the Czech texts)
\newfontfamily\nlfont[FeatureFile={junicode.fea}, Ligatures={Common, TeX}, RawFeature=+fixi]{Junicode}

\usepackage{multicol}
\usepackage{color}
\usepackage{lettrine}
\usepackage{fancyhdr}

% usual packages loading:
\usepackage{luatextra}
\usepackage{graphicx} % support the \includegraphics command and options
\usepackage{gregoriotex} % for gregorio score inclusion
\usepackage{gregoriosyms}
\usepackage{wrapfig} % figures wrapped by the text
\usepackage{parcolumns}
\usepackage[contents={},opacity=1,scale=1,color=black]{background}
\usepackage{tikzpagenodes}
\usepackage{calc}
\usepackage{longtable}
\usetikzlibrary{calc}

\setlength{\headheight}{14.5pt}

\input{conventuscommune.tex} % Often used macros

\newcommand{\annusEditionis}{2021}

%%%% Vicekrat opakovane kousky

\newcommand{\anteOrationem}{
  \rubrica{Ante Orationem, cantatur a Superiore:}

  \pars{Supplicatio Litaniæ.}

  \cuminitiali{}{temporalia/supplicatiolitaniae.gtex}

  \pars{Oratio Dominica.}

  \cuminitiali{}{temporalia/oratiodominica.gtex}

  \rubrica{Deinde dicitur ab Hebdomadario:}

  \cuminitiali{}{temporalia/dominusvobiscum-solemnis.gtex}

  \rubrica{In choro monialium loco Dominus vobiscum dicitur:}

  \sineinitiali{temporalia/domineexaudi.gtex}
}

\setlength{\columnsep}{30pt} % prostor mezi sloupci

%%%%%%%%%%%%%%%%%%%%%%%%%%%%%%%%%%%%%%%%%%%%%%%%%%%%%%%%%%%%%%%%%%%%%%%%%%%%%%%%%%%%%%%%%%%%%%%%%%%%%%%%%%%%%
\begin{document}

% Here we set the space around the initial.
% Please report to http://home.gna.org/gregorio/gregoriotex/details for more details and options
\grechangedim{afterinitialshift}{2.2mm}{scalable}
\grechangedim{beforeinitialshift}{2.2mm}{scalable}
\grechangedim{interwordspacetext}{0.22 cm plus 0.15 cm minus 0.05 cm}{scalable}%
\grechangedim{annotationraise}{-0.2cm}{scalable}

% Here we set the initial font. Change 38 if you want a bigger initial.
% Emit the initials in red.
\grechangestyle{initial}{\color{red}\fontsize{38}{38}\selectfont}

\pagestyle{empty}

%%%% Titulni stranka
\begin{titulusOfficii}
\ifx\titulus\undefined
\nomenFesti{Feria II \hebdomada{}}
\else
\titulus
\fi
\end{titulusOfficii}

\vfill

\begin{center}
%Ad usum et secundum consuetudines chori \guillemotright{}Conventus Choralis\guillemotleft.

%Editio Sancti Wolfgangi \annusEditionis
\end{center}

\scriptura{}

\pars{}

\pagebreak

\renewcommand{\headrulewidth}{0pt} % no horiz. rule at the header
\fancyhf{}
\pagestyle{fancy}

\cantusSineNeumas

\ifx\oratio\undefined
\ifx\laudb\undefined
\else
\newcommand{\oratio}{\pars{Oratio.}

\noindent Dómine Deus omnípotens, qui ad princípium huius diéi nos perveníre fecísti, tua nos hódie salva virtúte, ut in hac die ad nullum declinémus peccátum, sed semper ad tuam iustítiam faciéndam nostra procédant elóquia, dirigántur cogitatiónes et ópera.

\noindent Per Dóminum nostrum Iesum Christum, Fílium tuum, qui tecum vivit et regnat in unitáte Spíritus Sancti, Deus, per ómnia sǽcula sæculórum.

\noindent \Rbardot{} Amen.}
\fi
\fi

\hora{Ad Matutinum.} %%%%%%%%%%%%%%%%%%%%%%%%%%%%%%%%%%%%%%%%%%%%%%%%%%%%%
%\sideThumbs{Matutinum}

\vspace{2mm}

\cuminitiali{}{temporalia/dominelabiamea.gtex}

\vfill
%\pagebreak

\vspace{2mm}

\ifx\invitatorium\undefined
\pars{Invitatorium.} \scriptura{Ps. 94, 1; Psalmus 94; \textbf{H451}}

\vspace{-6mm}

\antiphona{VI}{temporalia/inv-jubilemusdeo.gtex}\else
\invitatorium
\fi

\vfill
\pagebreak

\ifx\hymnusmatutinum\undefined
\ifx\matua\undefined
\else
\pars{Hymnus.}

{
\grechangedim{interwordspacetext}{0.10 cm plus 0.15 cm minus 0.05 cm}{scalable}%
\antiphona{II}{temporalia/hym-IpsumNunc.gtex}
\grechangedim{interwordspacetext}{0.22 cm plus 0.15 cm minus 0.05 cm}{scalable}%
}
\fi
\else
\hymnusmatutinum
\fi

\vspace{-3mm}

\vfill
\pagebreak

\ifx\matub\undefined
\else
% MAT B
\pars{Psalmus 1.} \scriptura{Ps. 30, 2; \textbf{H90}}

\vspace{-4mm}

\antiphona{VIII G}{temporalia/ant-intuaiustitia.gtex}

%\vspace{-2mm}

\scriptura{Ps. 30, 2-9}

%\vspace{-2mm}

\initiumpsalmi{temporalia/ps30i-initium-viii-G-auto.gtex}

\vspace{-1.5mm}

\input{temporalia/ps30i-viii-G.tex} \Abardot{}

\vfill
\pagebreak

\pars{Psalmus 2.} \scriptura{Ps. 66, 2}

\vspace{-4mm}

\antiphona{E}{temporalia/ant-illuminadomine.gtex}

%\vspace{-2mm}

\scriptura{Ps. 30, 10-17}

%\vspace{-2mm}

\initiumpsalmi{temporalia/ps30ii-initium-e-a-auto.gtex}

\input{temporalia/ps30ii-e-a.tex} \Abardot{}

\vfill
\pagebreak

\pars{Psalmus 3.} \scriptura{Ps. 30, 24}

\vspace{-4mm}

\antiphona{II D}{temporalia/ant-diligitedominum.gtex}

%\vspace{-5mm}

\scriptura{Ps. 30, 20-25}

%\vspace{-2mm}

\initiumpsalmi{temporalia/ps30iii-initium-ii-D-auto.gtex}

\input{temporalia/ps30iii-ii-D.tex} \Abardot{}

\vfill
\pagebreak
\fi

\pars{Versus.}

\ifx\matversus\undefined
\ifx\matub\undefined
\else
\noindent \Vbardot{} Dírige me, Dómine, in veritáte tua, et doce me.

\noindent \Rbardot{} Quia tu es Deus salútis meæ.
\fi
\else
\matversus
\fi

\vspace{5mm}

\sineinitiali{temporalia/oratiodominica-mat.gtex}

\vspace{5mm}

\pars{Absolutio.}

\cuminitiali{}{temporalia/absolutio-exaudi.gtex}

\vfill
\pagebreak

\cuminitiali{}{temporalia/benedictio-solemn-benedictione.gtex}

\vspace{7mm}

\lectioi

\noindent \Vbardot{} Tu autem, Dómine, miserére nobis.
\noindent \Rbardot{} Deo grátias.

\vfill
\pagebreak

\responsoriumi

\vfill
\pagebreak

\cuminitiali{}{temporalia/benedictio-solemn-unigenitus.gtex}

\vspace{7mm}

\lectioii

\noindent \Vbardot{} Tu autem, Dómine, miserére nobis.
\noindent \Rbardot{} Deo grátias.

\vfill
\pagebreak

\responsoriumii

\vfill
\pagebreak

\cuminitiali{}{temporalia/benedictio-solemn-spiritus.gtex}

\vspace{7mm}

\lectioiii

\noindent \Vbardot{} Tu autem, Dómine, miserére nobis.
\noindent \Rbardot{} Deo grátias.

\vfill
\pagebreak

\responsoriumiii

\vfill
\pagebreak

\rubrica{Reliqua omittuntur, nisi Laudes separandæ sint.}

\sineinitiali{temporalia/domineexaudi.gtex}

\vfill

\oratio

\vfill

\noindent \Vbardot{} Dómine, exáudi oratiónem meam.
\Rbardot{} Et clamor meus ad te véniat.

\vfill

\noindent \Vbardot{} Benedicámus Dómino.
\noindent \Rbardot{} Deo grátias.

\vfill

\noindent \Vbardot{} Fidélium ánimæ per misericórdiam Dei requiéscant in pace.
\Rbardot{} Amen.

\vfill
\pagebreak

\hora{Ad Laudes.} %%%%%%%%%%%%%%%%%%%%%%%%%%%%%%%%%%%%%%%%%%%%%%%%%%%%%
%\sideThumbs{Laudes}

\cantusSineNeumas

\vspace{0.5cm}
\grechangedim{interwordspacetext}{0.18 cm plus 0.15 cm minus 0.05 cm}{scalable}%
\cuminitiali{}{temporalia/deusinadiutorium-communis.gtex}
\grechangedim{interwordspacetext}{0.22 cm plus 0.15 cm minus 0.05 cm}{scalable}%

\vfill
\pagebreak

\ifx\hymnuslaudes\undefined
\ifx\laudbd\undefined
\else
\pars{Hymnus} \scriptura{Hilarius (\olddag{} 367)}

\grechangedim{interwordspacetext}{0.16 cm plus 0.15 cm minus 0.05 cm}{scalable}%
\cuminitiali{IV}{temporalia/hym-LucisLargitor.gtex}
\grechangedim{interwordspacetext}{0.22 cm plus 0.15 cm minus 0.05 cm}{scalable}%
\vspace{-3mm}
\fi
\else
\hymnuslaudes
\fi

\vfill
\pagebreak

\ifx\laudb\undefined
\else
\pars{Psalmus 1.} \scriptura{Ps. 41, 3; \textbf{H391}}

\vspace{-4mm}

\antiphona{II D}{temporalia/ant-sitivitanima.gtex}

%\vspace{-2mm}

\scriptura{Psalmus 41}

%\vspace{-2mm}

\initiumpsalmi{temporalia/ps41-initium-ii-D-auto.gtex}

%\vspace{-1.5mm}

\input{temporalia/ps41-ii-D.tex}

\vfill

\antiphona{}{temporalia/ant-sitivitanima.gtex}

\vfill
\pagebreak

\pars{Psalmus 2.}

\vspace{-4mm}

\antiphona{III a}{temporalia/ant-ostendenobisdomine.gtex}

%\vspace{-2mm}

\scriptura{Canticum Ecclesiastici, Sir. 36, 1-7.13-16}

%\vspace{-3mm}

\initiumpsalmi{temporalia/ecclesiastici-initium-iii-a-auto.gtex}

\input{temporalia/ecclesiastici-iii-a.tex} \Abardot{}

\vfill
\pagebreak

\pars{Psalmus 3.}

\vspace{-4mm}

\antiphona{II D}{temporalia/ant-operamanuumeius.gtex}

\scriptura{Psalmus 18, 1-7}

\initiumpsalmi{temporalia/ps18i-initium-ii-D-auto.gtex}

\input{temporalia/ps18i-ii-D.tex} \Abardot{}

\vfill
\pagebreak
\fi

\ifx\lectiobrevis\undefined
\ifx\laudb\undefined
\else
\pars{Lectio Brevis.} \scriptura{Ier. 15, 16}

\noindent Invénti sunt sermónes tui, et comédi eos, et factum est mihi verbum tuum in gáudium et in lætítiam cordis mei, quóniam invocátum est nomen tuum super me, Dómine Deus exercítuum.
\fi
\else
\lectiobrevis
\fi

\vfill

\ifx\responsoriumbreve\undefined
\ifx\laudbd\undefined
\else
\pars{Responsorium breve.} \scriptura{Ps. 32, 1.3}

\cuminitiali{VI}{temporalia/resp-exsultateiusti.gtex}
\fi
\else
\responsoriumbreve
\fi

\vfill
\pagebreak

\ifx\benedictus\undefined
\ifx\laudbd\undefined
\else
\pars{Canticum Zachariæ.} \scriptura{Lc. 1, 68; \textbf{H422}}

\vspace{-4mm}

{
\grechangedim{interwordspacetext}{0.18 cm plus 0.15 cm minus 0.05 cm}{scalable}%
\antiphona{IV E}{temporalia/ant-benedictusdominus.gtex}
\grechangedim{interwordspacetext}{0.22 cm plus 0.15 cm minus 0.05 cm}{scalable}%
}

%\vspace{-3mm}

\scriptura{Lc. 1, 68-79}

%\vspace{-2mm}

\cantusSineNeumas
\initiumpsalmi{temporalia/benedictus-initium-iv-E-auto.gtex}

%\vspace{-1.5mm}

\input{temporalia/benedictus-iv-E.tex} \Abardot{}
\fi
\else
\benedictus
\fi

\vspace{-1cm}

\vfill
\pagebreak

%\sideThumbs{{\scriptsize{}Fine horarum}}

\pars{Preces.}

\sineinitiali{}{temporalia/tonusprecum.gtex}

\ifx\preces\undefined
\ifx\laudb\undefined
\else
\noindent Salvátor noster fecit nos regnum et sacerdótium, ut hóstias Deo acceptábiles offerámus. \gredagger{} Grati ígitur eum invocémus:

\Rbardot{} Serva nos in tuo ministério, Dómine.

\noindent Christe, sacérdos ætérne, qui sanctum pópulo tuo sacerdótium concessísti, \gredagger{} concéde, ut spiritáles hóstias Deo acceptábiles iúgiter offerámus.

\Rbardot{} Serva nos in tuo ministério, Dómine.

\noindent Spíritus tui fructus nobis largíre propítius, \gredagger{} patiéntiam, benignitátem et mansuetúdinem.

\Rbardot{} Serva nos in tuo ministério, Dómine.

\noindent Da nobis te amáre, ut te, qui es cáritas, possideámus, \gredagger{} et bene ágere, ut per vitam étiam nostram te laudémus.

\Rbardot{} Serva nos in tuo ministério, Dómine.

\noindent Quæ frátribus nostris sunt utília, nos quǽrere concéde, \gredagger{} ut salútem facílius consequántur.

\Rbardot{} Serva nos in tuo ministério, Dómine.
\fi
\else
\preces
\fi

\vfill

\pars{Oratio Dominica.}

\cuminitiali{}{temporalia/oratiodominicaalt.gtex}

\vfill
\pagebreak

\rubrica{vel:}

\pars{Supplicatio Litaniæ.}

\cuminitiali{}{temporalia/supplicatiolitaniae.gtex}

\vfill

\pars{Oratio Dominica.}

\cuminitiali{}{temporalia/oratiodominica.gtex}

\vfill
\pagebreak

% Oratio. %%%
\oratio

\vspace{-1mm}

\vfill

\rubrica{Hebdomadarius dicit Dominus vobiscum, vel, absente sacerdote vel diacono, sic concluditur:}

\vspace{2mm}

\antiphona{C}{temporalia/dominusnosbenedicat.gtex}

\rubrica{Postea cantatur a cantore:}

\vspace{2mm}

\cuminitiali{IV}{temporalia/benedicamus-feria-laudes.gtex}

\vspace{1mm}

\vfill
\pagebreak

\end{document}

