\newcommand{\titulus}{\nomenFesti{S. Bonaventuræ, Episcopi et Ecclesiæ Doctoris.}
\dies{Die 15. Iulii.}}
\newcommand{\oratio}{\pars{Oratio.}

\noindent Da, quǽsumus, omnípotens Deus, ut, beáti Bonaventúræ, epíscopi, natalícia celebrántes, et ipsíus proficiámus eruditióne præclára et caritátis ardórem iúgiter æmulémur.

\pars{Pro pace in Ucraina.} \scriptura{Sir. 50, 25; 2 Esdr. 4, 20; \textbf{H416}}

\vspace{-4mm}

\antiphona{II D}{temporalia/ant-dapacemdomine.gtex}

\vfill

\noindent Deus, a quo sancta desidéria, recta consília et iusta sunt ópera: da servis tuis illam, quam mundus dare non potest, pacem; ut et corda nostra mandátis tuis dédita, et hóstium subláta formídine, témpora sint tua protectióne tranquílla.

\noindent Per Dóminum nostrum Iesum Christum, Fílium tuum, qui tecum vivit et regnat in unitáte Spíritus Sancti, Deus, per ómnia sǽcula sæculórum.

\noindent \Rbardot{} Amen.}
\newcommand{\invitatorium}{\pars{Invitatorium.}

\vspace{-4mm}

\antiphona{IV}{temporalia/inv-mirabileminsanctis.gtex}}
\newcommand{\hymnusmatutinum}{\pars{Hymnus}

\cuminitiali{IV}{temporalia/hym-LaetiColentes.gtex}}
\newcommand{\matversus}{\noindent \Vbardot{} Iustum dedúxit Dóminus per vias rectas.

\noindent \Rbardot{} Et osténdit illi regnum Dei.}
\newcommand{\lectioi}{\pars{Lectio I.} \scriptura{2 Chr. 20, 1-9.13}

\noindent De libro secúndo Chronicórum.

\noindent In diébus illis: Congregáti sunt fílii Moab et fílii Ammon et cum eis de Maonítis ad Iósaphat, ut pugnárent contra eum. Venerúntque núntii et indicavérunt Iósaphat dicéntes: «Venit contra te multitúdo magna de his locis, quæ trans mare sunt, de Edom, et ecce consístunt in Asasónthamar, quæ est Engáddi».

\noindent Iósaphat autem timóre pertérritus totum se cóntulit ad rogándum Dóminum et prædicávit ieiúnium univérso Iudæ. Congregatúsque est Iuda ad precándum Dóminum; sed et de ómnibus úrbibus suis venérunt ad obsecrándum eum.

\noindent Cumque stetísset Iósaphat in médio cœtu Iudæ et Ierúsalem in domo Dómini ante átrium novum, ait: «Dómine, Deus patrum nostrórum, tu es Deus in cælo et domináris cunctis regnis géntium; in manu tua est fortitúdo et poténtia, nec quisquam tibi potest resístere. Nonne tu, Deus noster, expulísti habitatóres terræ huius coram pópulo tuo Israel et dedísti eam sémini Abraham amíci tui in sempitérnum? Habitaverúntque in ea et exstruxérunt in illa sanctuárium nómini tuo dicéntes: “Si irrúerint super nos mala, gládius iudícii, pestiléntia et fames, stábimus coram domo hac in conspéctu tuo, quia nomen tuum est in domo hac, et clamábimus ad te in tribulatiónibus nostris, et exáudies salvósque fácies”». Omnis vero Iuda stabat coram Dómino cum párvulis et uxóribus et líberis suis.}
\newcommand{\responsoriumi}{\pars{Responsorium 1.} \scriptura{\Rbardot{} Cantor \Vbardot{} ibidem; \textbf{H412}}

\vspace{-5mm}

\responsorium{II}{temporalia/resp-speminalionumquamhabui-CROCHU.gtex}{}}
\newcommand{\lectioii}{\pars{Lectio II.} \scriptura{2 Chr. 20, 14-24}

\noindent Erat autem Iaháziel fílius Zacharíæ fílii Banaíæ fílii Iéhiel fílii Matthaníæ Levítes de fíliis Asaph, super quem factus est spíritus Dómini in médio congregatiónis, et ait: «Atténdite, omnis Iuda et, qui habitátis Ierúsalem et tu rex Iósaphat: Hæc dicit Dóminus vobis: Nolíte timére nec paveátis hanc multitúdinem magnam; non est enim vestra pugna, sed Dei. Cras descendétis contra eos; ascensúri enim sunt per clivum nómine Sis, et inveniétis illos in summitáte torréntis, qui est contra solitúdinem Iéruel. Non éritis vos, qui dimicábitis; sed tantúmmodo confidénter state et vidébitis auxílium Dómini super vos, o Iuda et Ierúsalem. Nolíte timére, nec paveátis: cras egredímini contra eos, et Dóminus erit vobíscum». Iósaphat ergo inclinávit se super fáciem suam in terra et omnis Iuda et habitatóres Ierúsalem cecidérunt coram Dómino et adoravérunt eum. Porro Levítæ de fíliis Caath, de fíliis Core scílicet, surrexérunt et laudábant Dóminum, Deum Israel voce magna in excélsum.

\noindent Cumque mane surrexíssent, egréssi sunt ad desértum Thécue; profectísque eis, stans Iósaphat in médio eórum dixit: «Audíte me, Iuda et habitatóres Ierúsalem! Crédite in Dómino Deo vestro et permanébitis; crédite prophétis eius, et cuncta evénient vobis próspera». Habuítque consílium cum pópulo et státuit cantóres Dómini, ut laudárent eum in ornátu sancto et antecéderent exércitum ac voce cónsona dícerent: «Confitémini Dómino, quóniam in ætérnum misericórdia eius».

\noindent Cumque cœpíssent laudes cánere, vertit Dóminus insídias eórum contra fílios Ammon et Moab et montem Seir, qui egréssi fúerant, ut pugnárent contra Iudam, et percússi sunt. Et fílii Ammon et Moab consurrexérunt advérsum habitatóres montis Seir, ut interfícerent et delérent eos; cumque hoc ópere perpetrássent, étiam in semetípsos versi mútuis concidére vulnéribus.

\noindent Porro Iuda, cum venísset ad spéculam, quæ réspicit solitúdinem, vidit procul omnem late regiónem plenam cadavéribus, nec superésse quemquam, qui necem potuísset evádere.}
\newcommand{\responsoriumii}{\pars{Responsorium 2.} \scriptura{\Rbardot{} Ps. 19, 10 \& Cantor \Vbardot{} Ps. 85, 7; \textbf{H170}}

\vspace{-5mm}

\responsorium{I}{temporalia/resp-indiequainvocavi.gtex}{}}
\newcommand{\lectioiii}{\pars{Lectio III.} \scriptura{Cap. 7, 1. 2. 4. 6: Opera omnia, 5, 312-313}

\noindent Ex Opúsculo sancti Bonaventúræ epíscopi De itinerário mentis in Deum.

\noindent Christus est via et óstium. Christus est scala et vehículum tamquam \emph{propitiatórium super arcam Dei collocátum et sacraméntum a sǽculis abscónditum.} Ad quod propitiatórium qui áspicit plena conversióne vultus, aspiciéndo eum in cruce suspénsum, per fidem, spem et caritátem, devotiónem, admiratiónem, exsultatiónem, appretiatiónem, laudem et iubilatiónem; \emph{pascha,} hoc est tránsitum, cum eo facit, ut per virgam crucis tránseat mare Rubrum, ab Ægýpto intrans desértum, ubi gustat manna abscónditum, et cum Christo requiéscat in túmulo quasi extérius mórtuus, séntiens tamen, quantum possíbile est secúndum statum viæ, quod in cruce dictum est latróni cohærénti Christo: \emph{Hódie mecum eris in paradíso.}

\noindent In hoc autem tránsitu, si sit perféctus, opórtet quod relinquántur omnes intellectuáles operatiónes, et apex afféctus totus transferátur et transformétur in Deum. Hoc autem est mýsticum et secretíssimum, quod nemo novit, nisi qui áccipit, nec áccipit nisi qui desíderat, nec desíderat nisi quem ignis Spíritus Sancti medúllitus inflámmat, quem Christus misit in terram. Et ídeo dicit Apóstolus, hanc mýsticam sapiéntiam esse per Spíritum Sanctum revelátam.

\noindent Si autem quæras quómodo hæc fiant, intérroga grátiam, non doctrínam; desidérium, non intelléctum; gémitum oratiónis, non stúdium lectiónis; sponsum, non magístrum; Deum, non hóminem; calíginem, non claritátem; non lucem, sed ignem totáliter inflammántem et in Deum excessívis unctiónibus et ardentíssimis affectiónibus transferéntem. Qui quidem ignis Deus est, et hic camínus est in Ierúsalem, et Christus hunc accéndit in fervóre suæ ardentíssimæ passiónis, quem solus ille vere pércipit, qui dicit: \emph{Suspéndium elégit ánima mea, et mortem ossa mea.} Quam mortem qui díligit, vidére potest Deum, quia indubitánter verum est: \emph{Non vidébit me homo et vivet.} Moriámur ígitur et ingrediámur in calíginem, imponámus siléntium sollicitudínibus, concupiscéntiis et phantasmátibus; transeámus cum Christo crucifíxo \emph{ex hoc mundo ad Patrem,} ut, osténso nobis Patre, dicámus cum Philíppo: \emph{Súfficit nobis}; audiámus cum Paulo: \emph{Súfficit tibi grátia mea}; exsultémus cum David dicéntes: \emph{Déficit caro mea et cor meum, Deus cordis mei et pars mea Deus in ætérnum. Benedíctus Dóminus in ætérnum et dicat omnis pópulus: Fiat, Fiat.}}
\newcommand{\responsoriumiii}{\pars{Responsorium 3.} \scriptura{\Rbardot{} Sir. 15, 3 \Vbardot{} Ps. 131, 18; \textbf{H62}}

\vspace{-5mm}

\responsorium{VII}{temporalia/resp-cibavitillum-CROCHU-cumdox.gtex}{}}
\newcommand{\hymnuslaudes}{\pars{Hymnus}

\cuminitiali{VIII}{temporalia/hym-ORedemptorisPietas.gtex}}
\newcommand{\lectiobrevis}{\pars{Lectio Brevis.} \scriptura{Sap. 7, 13-14}

\noindent Sapiéntiam sine fictióne dídici et sine invídia commúnico; divítias illíus non abscóndo. Infinítus enim thesáurus est homínibus; quem qui acquisiérunt, ad amicítiam in Deum se paravérunt propter disciplínæ dona commendáti.}
\newcommand{\responsoriumbreve}{\pars{Responsorium breve.} \scriptura{Sir. 45, 9}

\antiphona{VI}{temporalia/resp-amaviteum.gtex}}
\newcommand{\preces}{\noindent Christum Deum sanctum, fratres, exaltémus, \gredagger{} orántes ut serviámus illi in sanctitáte et iustítia coram ipso ómnibus diébus nostris, \grestar{} et acclamémus:

\Rbardot{} Tu solus sanctus, Dómine.

\noindent Qui tentári voluísti per ómnia pro similitúdine nostra absque peccáto, \grestar{} miserére nostri, Dómine Iesu.

\Rbardot{} Tu solus sanctus, Dómine.

\noindent Qui nos omnes ad perfectiónem caritátis vocásti, \grestar{} sanctífica nos, Dómine Iesu.

\Rbardot{} Tu solus sanctus, Dómine.

\noindent Qui nos iussísti esse salem terræ et lucem mundi, \grestar{} illúmina nos, Dómine Iesu.

\Rbardot{} Tu solus sanctus, Dómine.

\noindent Qui voluísti ministráre, \gredagger{} non ministrári, \grestar{} fac nos tibi et frátribus humíliter servíre, Dómine Iesu.

\Rbardot{} Tu solus sanctus, Dómine.

\noindent Tu, splendor glóriæ Patris et figúra substántiæ eius, \grestar{} fac ut in glória vultum tuum respiciámus, Dómine Iesu.

\Rbardot{} Tu solus sanctus, Dómine.}
\newcommand{\benedictus}{\pars{Canticum Zachariæ.} \scriptura{Eccli. 15, 5; \textbf{H61}}

%\vspace{-4mm}

{
\grechangedim{interwordspacetext}{0.18 cm plus 0.15 cm minus 0.05 cm}{scalable}%
\antiphona{II D}{temporalia/ant-spiritusapientiae.gtex}
\grechangedim{interwordspacetext}{0.22 cm plus 0.15 cm minus 0.05 cm}{scalable}%
}

%\vspace{-3mm}

\scriptura{Lc. 1, 68-79}

%\vspace{-2mm}

\cantusSineNeumas
\initiumpsalmi{temporalia/benedictus-initium-ii-D-auto.gtex}

%\vspace{-1.5mm}

\input{temporalia/benedictus-ii-D.tex} \Abardot{}}
\newcommand{\precestotum}{\pars{Deprecatio Gelasii}

\vspace{-5mm}

\grechangedim{interwordspacetext}{0.16 cm plus 0.15 cm minus 0.05 cm}{scalable}%
\antiphona{D\textsuperscript{1}}{temporalia/deprecatio4-propace.gtex}
\grechangedim{interwordspacetext}{0.22 cm plus 0.15 cm minus 0.05 cm}{scalable}%

\vfill

\pars{Oratio Dominica.}

\cuminitiali{D}{temporalia/oratiodominica-d.gtex}}
\newcommand{\dominusnosbenedicat}{\antiphona{D}{temporalia/dominusnosbenedicat-d.gtex}}
\newcommand{\hebdomada}{infra Hebdom. XV per Annum.}
\newcommand{\matuc}{Matutinum Hebdomadae C}
\newcommand{\matuac}{Matutinum Hebdomadae A vel C}
\newcommand{\laudc}{Laudes Hebdomadae C}
\newcommand{\laudac}{Laudes Hebdomadae A vel C}

% LuaLaTeX

\documentclass[a4paper, twoside, 12pt]{article}
\usepackage[latin]{babel}
%\usepackage[landscape, left=3cm, right=1.5cm, top=2cm, bottom=1cm]{geometry} % okraje stranky
%\usepackage[landscape, a4paper, mag=1166, truedimen, left=2cm, right=1.5cm, top=1.6cm, bottom=0.95cm]{geometry} % okraje stranky
\usepackage[landscape, a4paper, mag=1400, truedimen, left=0.5cm, right=0.5cm, top=0.5cm, bottom=0.5cm]{geometry} % okraje stranky

\usepackage{fontspec}
\setmainfont[FeatureFile={junicode.fea}, Ligatures={Common, TeX}, RawFeature=+fixi]{Junicode}
%\setmainfont{Junicode}

% shortcut for Junicode without ligatures (for the Czech texts)
\newfontfamily\nlfont[FeatureFile={junicode.fea}, Ligatures={Common, TeX}, RawFeature=+fixi]{Junicode}

\usepackage{multicol}
\usepackage{color}
\usepackage{lettrine}
\usepackage{fancyhdr}

% usual packages loading:
\usepackage{luatextra}
\usepackage{graphicx} % support the \includegraphics command and options
\usepackage{gregoriotex} % for gregorio score inclusion
\usepackage{gregoriosyms}
\usepackage{wrapfig} % figures wrapped by the text
\usepackage{parcolumns}
\usepackage[contents={},opacity=1,scale=1,color=black]{background}
\usepackage{tikzpagenodes}
\usepackage{calc}
\usepackage{longtable}
\usetikzlibrary{calc}

\setlength{\headheight}{14.5pt}

\input{conventuscommune.tex} % Often used macros

\newcommand{\annusEditionis}{2021}

%%%% Vicekrat opakovane kousky

\newcommand{\anteOrationem}{
  \rubrica{Ante Orationem, cantatur a Superiore:}

  \pars{Supplicatio Litaniæ.}

  \cuminitiali{}{temporalia/supplicatiolitaniae.gtex}

  \pars{Oratio Dominica.}

  \cuminitiali{}{temporalia/oratiodominica.gtex}

  \rubrica{Deinde dicitur ab Hebdomadario:}

  \cuminitiali{}{temporalia/dominusvobiscum-solemnis.gtex}

  \rubrica{In choro monialium loco Dominus vobiscum dicitur:}

  \sineinitiali{temporalia/domineexaudi.gtex}
}

\setlength{\columnsep}{30pt} % prostor mezi sloupci

%%%%%%%%%%%%%%%%%%%%%%%%%%%%%%%%%%%%%%%%%%%%%%%%%%%%%%%%%%%%%%%%%%%%%%%%%%%%%%%%%%%%%%%%%%%%%%%%%%%%%%%%%%%%%
\begin{document}

% Here we set the space around the initial.
% Please report to http://home.gna.org/gregorio/gregoriotex/details for more details and options
\grechangedim{afterinitialshift}{2.2mm}{scalable}
\grechangedim{beforeinitialshift}{2.2mm}{scalable}
\grechangedim{interwordspacetext}{0.22 cm plus 0.15 cm minus 0.05 cm}{scalable}%
\grechangedim{annotationraise}{-0.2cm}{scalable}

% Here we set the initial font. Change 38 if you want a bigger initial.
% Emit the initials in red.
\grechangestyle{initial}{\color{red}\fontsize{38}{38}\selectfont}

\pagestyle{empty}

%%%% Titulni stranka
\begin{titulusOfficii}
\ifx\titulus\undefined
\nomenFesti{Feria VI \hebdomada{}}
\else
\titulus
\fi
\end{titulusOfficii}

\vfill

\begin{center}
%Ad usum et secundum consuetudines chori \guillemotright{}Conventus Choralis\guillemotleft.

%Editio Sancti Wolfgangi \annusEditionis
\end{center}

\scriptura{}

\pars{}

\pagebreak

\renewcommand{\headrulewidth}{0pt} % no horiz. rule at the header
\fancyhf{}
\pagestyle{fancy}

\cantusSineNeumas

\hora{Ad Matutinum.} %%%%%%%%%%%%%%%%%%%%%%%%%%%%%%%%%%%%%%%%%%%%%%%%%%%%%

\vspace{2mm}

\cuminitiali{}{temporalia/dominelabiamea.gtex}

\vfill
%\pagebreak

\vspace{2mm}

\ifx\invitatorium\undefined
\pars{Invitatorium.} \scriptura{Lc. 24, 34; Psalmus 94; \textbf{H232}}

\antiphona{VI}{temporalia/inv-surrexitdominusvere.gtex}
\else
\invitatorium
\fi

\vfill
\pagebreak

\ifx\hymnusmatutinum\undefined
\pars{Hymnus.}

\cuminitiali{VIII}{temporalia/hym-LaetareCaelum.gtex}
\else
\hymnusmatutinum
\fi

\vspace{-3mm}

\vfill
\pagebreak

\ifx\matutinum\undefined
\ifx\matua\undefined
\else
% MAT A
\pars{Psalmus 1.}

\vspace{-4mm}

\antiphona{I a\textsuperscript{3}}{temporalia/ant-alleluia-turco24.gtex}

%\vspace{-2mm}

\scriptura{Ps. 34, 1-10}

%\vspace{-2mm}

\initiumpsalmi{temporalia/ps34i-initium-i-a5-auto.gtex}

\input{temporalia/ps34i-i-a5.tex}

\vfill
\pagebreak

\pars{Psalmus 2.} \scriptura{Ps. 34, 11-17}

%\vspace{-2mm}

\initiumpsalmi{temporalia/ps34ii-initium-i-a5-auto.gtex}

\input{temporalia/ps34ii-i-a5.tex}

\vfill
\pagebreak

\pars{Psalmus 3.} \scriptura{Ps. 34, 18-28}

\vspace{-2mm}

\initiumpsalmi{temporalia/ps34iii-initium-i-a5-auto.gtex}

\input{temporalia/ps34iii-i-a5.tex}

\vfill

\antiphona{}{temporalia/ant-alleluia-turco24.gtex}

\vfill
\pagebreak
\fi
\ifx\matub\undefined
\else
% MAT B
\pars{Psalmus 1.}

\vspace{-4mm}

\antiphona{D}{temporalia/ant-alleluia-turco2.gtex}

%\vspace{-2mm}

\scriptura{Ps. 37, 2-5}

%\vspace{-2mm}

\initiumpsalmi{temporalia/ps37ii_v-initium-d-g-auto.gtex}

\input{temporalia/ps37ii_v-d-g.tex}

\vfill
\pagebreak

\pars{Psalmus 2.}

\scriptura{Ps. 37, 6-13}

%\vspace{-2mm}

\initiumpsalmi{temporalia/ps37vi_xiii-initium-d-g-auto.gtex}

\input{temporalia/ps37vi_xiii-d-g.tex}

\vfill
\pagebreak

\pars{Psalmus 3.}

\scriptura{Ps. 37, 14-23}

%\vspace{-2mm}

\initiumpsalmi{temporalia/ps37xiv_xxiii-initium-d-g-auto.gtex}

\input{temporalia/ps37xiv_xxiii-d-g.tex}

\vfill

\antiphona{}{temporalia/ant-alleluia-turco2.gtex}

\vfill
\pagebreak
\fi
\ifx\matuc\undefined
\else
% MAT C
\pars{Psalmus 1.}

\vspace{-4mm}

\antiphona{I d\textsuperscript{3}}{temporalia/ant-alleluia-auglx5.gtex}

%\vspace{-3mm}

\scriptura{Ps. 68, 2-13}

%\vspace{-2mm}

\initiumpsalmi{temporalia/ps68ii_xiii-initium-i-d-auto.gtex}

%\vspace{-1.5mm}

\input{temporalia/ps68ii_xiii-i-d.tex}

\vfill
\pagebreak

\pars{Psalmus 2.}

\scriptura{Ps. 68, 14-22}

%\vspace{-2mm}

\initiumpsalmi{temporalia/ps68xiv_xxii-initium-i-d-auto.gtex}

\input{temporalia/ps68xiv_xxii-i-d.tex}

\vfill
\pagebreak

\pars{Psalmus 3.}

\scriptura{Ps. 68, 30-37}

%\vspace{-2mm}

\initiumpsalmi{temporalia/ps68iii-initium-i-d-auto.gtex}

\input{temporalia/ps68iii-i-d.tex}

\vfill

\antiphona{}{temporalia/ant-alleluia-auglx5.gtex}

\vfill
\pagebreak
\fi
\ifx\matud\undefined
\else
% MAT D
\pars{Psalmus 1.}

\vspace{-4mm}

\antiphona{I a\textsuperscript{2}}{temporalia/ant-alleluia-turco24.gtex}

%\vspace{-3mm}

\scriptura{Ps. 77, 1-16}

%\vspace{-2mm}

\initiumpsalmi{temporalia/ps77i_xvi-initium-i-a4-auto.gtex}

\input{temporalia/ps77i_xvi-i-a4.tex}

\vfill
\pagebreak

\pars{Psalmus 2.} \scriptura{Ps. 77, 17-31}

%\vspace{-2mm}

\initiumpsalmi{temporalia/ps77iii-initium-i-a4-auto.gtex}

\input{temporalia/ps77iii-i-a4.tex}

\vfill
\pagebreak

\pars{Psalmus 3.} \scriptura{Ps. 77, 32-39}

%\vspace{-2mm}

\initiumpsalmi{temporalia/ps77xxxii_xxxix-initium-i-a4-auto.gtex}

\input{temporalia/ps77xxxii_xxxix-i-a4.tex}

\vfill

\antiphona{}{temporalia/ant-alleluia-turco24.gtex}

\vfill
\pagebreak
\fi
\else
\matutinum
\fi

\pars{Versus.}

\ifx\matversus\undefined
\noindent \Vbardot{} In resurrectióne tua, Christe, allelúia.

\noindent \Rbardot{} Cæli et terra læténtur, allelúia.
\else
\matversus
\fi

\vspace{5mm}

\sineinitiali{temporalia/oratiodominica-mat.gtex}

\vspace{5mm}

\pars{Absolutio.}

\cuminitiali{}{temporalia/absolutio-ipsius.gtex}

\vfill
\pagebreak

\cuminitiali{}{temporalia/benedictio-solemn-deus.gtex}

\vspace{7mm}

\lectioi

\noindent \Vbardot{} Tu autem, Dómine, miserére nobis.
\noindent \Rbardot{} Deo grátias.

\vfill
\pagebreak

\responsoriumi

\vfill
\pagebreak

\cuminitiali{}{temporalia/benedictio-solemn-christus.gtex}

\vspace{7mm}

\lectioii

\noindent \Vbardot{} Tu autem, Dómine, miserére nobis.
\noindent \Rbardot{} Deo grátias.

\vfill
\pagebreak

\responsoriumii

\vfill
\pagebreak

\cuminitiali{}{temporalia/benedictio-solemn-ignem.gtex}

\vspace{7mm}

\lectioiii

\noindent \Vbardot{} Tu autem, Dómine, miserére nobis.
\noindent \Rbardot{} Deo grátias.

\vfill
\pagebreak

\responsoriumiii

\vfill
\pagebreak

\rubrica{Reliqua omittuntur, nisi Laudes separandæ sint.}

\sineinitiali{temporalia/domineexaudi.gtex}

\vfill

\oratio

\vfill

\noindent \Vbardot{} Dómine, exáudi oratiónem meam.
\Rbardot{} Et clamor meus ad te véniat.

\vfill

\noindent \Vbardot{} Benedicámus Dómino.
\noindent \Rbardot{} Deo grátias.

\vfill

\noindent \Vbardot{} Fidélium ánimæ per misericórdiam Dei requiéscant in pace.
\Rbardot{} Amen.

\vfill
\pagebreak

\hora{Ad Laudes.} %%%%%%%%%%%%%%%%%%%%%%%%%%%%%%%%%%%%%%%%%%%%%%%%%%%%%

\cantusSineNeumas

\vspace{0.5cm}
\grechangedim{interwordspacetext}{0.18 cm plus 0.15 cm minus 0.05 cm}{scalable}%
\cuminitiali{}{temporalia/deusinadiutorium-communis.gtex}
\grechangedim{interwordspacetext}{0.22 cm plus 0.15 cm minus 0.05 cm}{scalable}%

\vfill
\pagebreak

\ifx\hymnuslaudes\undefined
\ifx\laudac\undefined
\else
\pars{Hymnus}

\cuminitiali{I}{temporalia/hym-ChorusNovae-praglia.gtex}
\vspace{-3mm}
\fi
\ifx\laudbd\undefined
\else
\pars{Hymnus}

\cuminitiali{I}{temporalia/hym-ChorusNovae.gtex}
\vspace{-3mm}
\fi
\else
\hymnuslaudes
\fi

\vfill
\pagebreak

\ifx\laudes\undefined
\ifx\lauda\undefined
\else
\pars{Psalmus 1.}

\vspace{-4mm}

\antiphona{VI F}{temporalia/ant-alleluia-turco6.gtex}

\scriptura{Psalmus 50.}

\initiumpsalmi{temporalia/ps50-initium-vi-F-auto.gtex}

\input{temporalia/ps50-vi-F.tex}

\vfill

\antiphona{}{temporalia/ant-alleluia-turco6.gtex}

\vfill
\pagebreak

\pars{Psalmus 2.} \scriptura{Is. 45, 25}

\vspace{-4mm}

\antiphona{V a}{temporalia/ant-indominoiustificabitur-tp.gtex}

\scriptura{Canticum Isaiæ, Is. 45, 15-30}

%\vspace{-2mm}

\initiumpsalmi{temporalia/isaiae2-initium-v-a-auto.gtex}

\input{temporalia/isaiae2-v-a.tex}

\vfill

\antiphona{}{temporalia/ant-indominoiustificabitur-tp.gtex}

\vfill
\pagebreak

\pars{Psalmus 3.}

\vspace{-4mm}

\antiphona{IV* e}{temporalia/ant-alleluia-turco9.gtex}

\scriptura{Psalmus 99.}

\initiumpsalmi{temporalia/ps99-initium-iv_-e-auto.gtex}

\input{temporalia/ps99-iv_-e.tex} \Abardot{}

\vfill
\pagebreak
\fi
\ifx\laudb\undefined
\else
\pars{Psalmus 1.}

\vspace{-4mm}

\antiphona{VII a}{temporalia/ant-alleluia-turco29.gtex}

\scriptura{Psalmus 50.}

\initiumpsalmi{temporalia/ps50-initium-vii-a-auto.gtex}

\input{temporalia/ps50-vii-a.tex}

\vfill

\antiphona{}{temporalia/ant-alleluia-turco29.gtex}

\vfill
\pagebreak

\pars{Psalmus 2.} \scriptura{Hab. 3, 2; \textbf{H99}}

\vspace{-6mm}

\antiphona{IV* e}{temporalia/ant-domineaudivi-tp.gtex}

\vspace{-2mm}

\scriptura{Canticum Habacuc, Hab. 3, 2-19}

%\vspace{-2mm}

%\initiumpsalmi{temporalia/habacuc-initium-iv_-e-auto.gtex}
\initiumpsalmi{temporalia/habacuc-initium-iv_-e.gtex}

\input{temporalia/habacuc-iv_-e.tex}

\vfill

\antiphona{}{temporalia/ant-domineaudivi-tp.gtex}

\vfill
\pagebreak

\pars{Psalmus 3.}

\vspace{-4mm}

\antiphona{E}{temporalia/ant-alleluia-turco4.gtex}

\vspace{-2mm}

\scriptura{Psalmus 147.}

%\vspace{-3mm}

%\initiumpsalmi{temporalia/ps147-initium-e-auto.gtex}
\initiumpsalmi{temporalia/ps147-initium-e.gtex}

\input{temporalia/ps147-e.tex} \Abardot{}

\vfill
\pagebreak
\fi
\ifx\laudc\undefined
\else
\pars{Psalmus 1.}

\vspace{-4mm}

\antiphona{VIII G\textsuperscript{2}}{temporalia/ant-alleluia-turco13.gtex}

\scriptura{Psalmus 50.}

\initiumpsalmi{temporalia/ps50-initium-viii-G5-auto.gtex}

\input{temporalia/ps50-viii-G5.tex}

\vfill

\antiphona{}{temporalia/ant-alleluia-turco13.gtex}

\vfill
\pagebreak

\pars{Psalmus 2.}

\vspace{-4mm}

\antiphona{VIII G}{temporalia/ant-nonnosderelinquas-tp.gtex}

%\vspace{-2mm}

\scriptura{Canticum Ieremiæ, Ier. 14, 17-31}

%\vspace{-2mm}

\initiumpsalmi{temporalia/jeremiae2-initium-viii-G.gtex}

\input{temporalia/jeremiae2-viii-G.tex} \Abardot{}

\vfill
\pagebreak

\pars{Psalmus 3.}

\vspace{-4mm}

\antiphona{E}{temporalia/ant-alleluia-praglia-e2.gtex}

\vspace{-2mm}

\scriptura{Psalmus 99.}

%\vspace{-2mm}

\initiumpsalmi{temporalia/ps99-initium-e-auto.gtex}

\input{temporalia/ps99-e.tex} \Abardot{}

\vfill
\pagebreak
\fi
\ifx\laudd\undefined
\else
\pars{Psalmus 1.}

\vspace{-4mm}

\antiphona{I f}{temporalia/ant-alleluia-turco20.gtex}

\scriptura{Psalmus 50.}

\initiumpsalmi{temporalia/ps50-initium-i-f-auto.gtex}

\input{temporalia/ps50-i-f.tex}

\vfill

\antiphona{}{temporalia/ant-alleluia-turco20.gtex}

\vfill
\pagebreak

\pars{Psalmus 2.} \scriptura{Ac. 22, 14}

\vspace{-4mm}

\antiphona{VIII G}{temporalia/ant-beatiquilavantstolas.gtex}

%\vspace{-2mm}

\scriptura{Canticum Tobiæ, Tob. 13, 10-18}

%\vspace{-2mm}

\initiumpsalmi{temporalia/tobiae2-initium-viii-G-auto.gtex}

\input{temporalia/tobiae2-viii-G.tex} \Abardot{}

\vfill
\pagebreak

\pars{Psalmus 3.}

\vspace{-4mm}

\antiphona{VI F}{temporalia/ant-alleluia-turco5.gtex}

\vspace{-2mm}

\scriptura{Psalmus 147.}

%\vspace{-2mm}

\initiumpsalmi{temporalia/ps147-initium-vi-F-auto.gtex}

\input{temporalia/ps147-vi-F.tex} \Abardot{}

\vfill
\pagebreak
\fi
\else
\laudes
\fi

\ifx\lectiobrevis\undefined
\pars{Lectio Brevis.} \scriptura{Ac. 5, 30-32}

\noindent Deus patrum nostrórum suscitávit Iesum, quem vos interemístis suspendéntes in ligno; hunc Deus Príncipem et Salvatórem exaltávit déxtera sua ad dandam pæniténtiam Israel et remissiónem peccatórum. Et nos sumus testes horum verbórum, et Spíritus Sanctus, quem dedit Deus obœdiéntibus sibi.
\else
\lectiobrevis
\fi

\vfill

\ifx\responsoriumbreve\undefined
\pars{Responsorium breve.} \scriptura{Cf. Mt. 28, 6; Cf. Gal. 3, 13}

\cuminitiali{VI}{temporalia/resp-surrexitdominusdesepulcro.gtex}
\else
\responsoriumbreve
\fi

\vfill
\pagebreak

\benedictus

\vspace{-1cm}

\vfill
\pagebreak

\pars{Preces.}

\sineinitiali{}{temporalia/tonusprecum.gtex}

\ifx\preces\undefined
\ifx\lauda\undefined
\else
\noindent Deum Patrem, qui vitam novam per Christi resurrectiónem cóntulit nobis,~\gredagger{} súpplices exorémus:

\Rbardot{} Clarífica nos claritáte Christi.

\noindent Deus, qui opéribus tuis antíquam dispensatiónem manifestásti, terram creásti et fidélis es in ómnibus generatiónibus,~\gredagger{} exáudi nos, clementíssime Pater.

\Rbardot{} Clarífica nos claritáte Christi.

\noindent Purífica nos puritáte veritátis tuæ, et gressus nostros dírige in cordis sanctitáte,~\gredagger{} ut quod iustum est tibíque plácitum agámus.

\Rbardot{} Clarífica nos claritáte Christi.

\noindent Illúmina vultum tuum super nos,~\gredagger{} ut a peccáto liberáti bonis domus tuæ repleámur.

\Rbardot{} Clarífica nos claritáte Christi.

\noindent Qui per Christum nos tibi reconciliásti,~\gredagger{} pacem nobis largíre omnibúsque in orbe terrárum degéntibus.

\Rbardot{} Clarífica nos claritáte Christi.
\fi
\ifx\laudb\undefined
\else
\noindent Deus Pater Christum per Spíritum suscitávit, et étiam mortália córpora nostra vivificábit.~\gredagger{} Quare clamémus:

\Rbardot{} Dómine, vivífica nos Spíritu Sancto tuo.

\noindent Pater sancte, qui accepísti holocáustum Fílii tui, resúscitans eum ex mórtuis,~\gredagger{} súscipe hodiérnam nostram oblatiónem et perduc nos in vitam ætérnam.

\Rbardot{} Dómine, vivífica nos Spíritu Sancto tuo.

\noindent Opera nostra hódie propítius intuére,~\gredagger{} ut fiant ad glóriam tuam et ad ómnium sanctificatiónem.

\Rbardot{} Dómine, vivífica nos Spíritu Sancto tuo.

\noindent Opus nostrum hódie non sit vanum, sed univérsis homínibus insérviat~\gredagger{} et sic operántes ad regnum tuum fac nos perveníre.

\Rbardot{} Dómine, vivífica nos Spíritu Sancto tuo.

\noindent Aperi hódie óculos nostros et cor nostrum ad fratres,~\gredagger{} ut nos ínvicem amémus nobísque serviámus.

\Rbardot{} Dómine, vivífica nos Spíritu Sancto tuo.
\fi
\ifx\laudc\undefined
\else
\noindent Deum Patrem, qui vitam novam per Christi resurrectiónem cóntulit nobis,~\gredagger{} súpplices exorémus:

\Rbardot{} Clarífica nos claritáte Christi.

\noindent Deus, qui opéribus tuis antíquam dispensatiónem manifestásti, terram creásti et fidélis es in ómnibus generatiónibus,~\gredagger{} exáudi nos, clementíssime Pater.

\Rbardot{} Clarífica nos claritáte Christi.

\noindent Purífica nos puritáte veritátis tuæ, et gressus nostros dírige in cordis sanctitáte,~\gredagger{} ut quod iustum est tibíque plácitum agámus.

\Rbardot{} Clarífica nos claritáte Christi.

\noindent Illúmina vultum tuum super nos,~\gredagger{} ut a peccáto liberáti bonis domus tuæ repleámur.

\Rbardot{} Clarífica nos claritáte Christi.

\noindent Qui per Christum nos tibi reconciliásti,~\gredagger{} pacem nobis largíre omnibúsque in orbe terrárum degéntibus.

\Rbardot{} Clarífica nos claritáte Christi.
\fi
\ifx\laudd\undefined
\else
\noindent Deus Pater Christum per Spíritum suscitávit, et étiam mortália córpora nostra vivificábit.~\gredagger{} Quare clamémus:

\Rbardot{} Dómine, vivífica nos Spíritu Sancto tuo.

\noindent Pater sancte, qui accepísti holocáustum Fílii tui, resúscitans eum ex mórtuis,~\gredagger{} súscipe hodiérnam nostram oblatiónem et perduc nos in vitam ætérnam.

\Rbardot{} Dómine, vivífica nos Spíritu Sancto tuo.

\noindent Opera nostra hódie propítius intuére,~\gredagger{} ut fiant ad glóriam tuam et ad ómnium sanctificatiónem.

\Rbardot{} Dómine, vivífica nos Spíritu Sancto tuo.

\noindent Opus nostrum hódie non sit vanum, sed univérsis homínibus insérviat~\gredagger{} et sic operántes ad regnum tuum fac nos perveníre.

\Rbardot{} Dómine, vivífica nos Spíritu Sancto tuo.

\noindent Aperi hódie óculos nostros et cor nostrum ad fratres,~\gredagger{} ut nos ínvicem amémus nobísque serviámus.

\Rbardot{} Dómine, vivífica nos Spíritu Sancto tuo.
\fi 
\else
\preces
\fi

\vfill

\pars{Oratio Dominica.}

\cuminitiali{}{temporalia/oratiodominicaalt.gtex}

\vfill
\pagebreak

\rubrica{vel:}

\pars{Supplicatio Litaniæ.}

\cuminitiali{}{temporalia/supplicatiolitaniae.gtex}

\vfill

\pars{Oratio Dominica.}

\cuminitiali{}{temporalia/oratiodominica.gtex}

\vfill
\pagebreak

% Oratio. %%%
\oratio

\vspace{-1mm}

\vfill

\rubrica{Hebdomadarius dicit Dominus vobiscum, vel, absente sacerdote vel diacono, sic concluditur:}

\vspace{2mm}

\antiphona{C}{temporalia/dominusnosbenedicat.gtex}

\rubrica{Postea cantatur a cantore:}

\vspace{2mm}

\cuminitiali{VII}{temporalia/benedicamus-tempore-paschali.gtex}

\vspace{1mm}

\vfill
\pagebreak

\end{document}

