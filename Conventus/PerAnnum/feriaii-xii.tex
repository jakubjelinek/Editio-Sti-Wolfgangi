\newcommand{\hymnusmatutinum}{\pars{Hymnus} \scriptura{Walahfrid Strabus (\olddag 849)}

\cuminitiali{VIII}{temporalia/hym-VitaSanctorum-ver.gtex}}
\newcommand{\lectioi}{\pars{Lectio I.} \scriptura{1 Sam. 17, 1-10.32-40}

\noindent De libro primo Samuélis.

\noindent In diébus illis: Congregántes Philísthim ágmina sua in prœ́lium, convenérunt in Socho Iudæ et castrametáti sunt inter Socho et Azéca in Aphesdómmim. Porro Saul et viri Israel congregáti venérunt in vallem Terebínthi et instruxérunt áciem ad pugnándum contra Philísthim. Et Philísthim stabant super montem ex hac parte, et Israel stabat super montem ex áltera parte; vallísque erat inter eos.

\noindent Et egréssus est vir propugnátor de castris Philisthinórum nómine Góliath de Geth altitúdinis sex cubitórum et palmi. Et cassis ǽrea super caput eius, et loríca squamáta induebátur; porro pondus lorícæ eius quinque mília siclórum æris. Et ócreas ǽreas habébat in crúribus et acínaces ǽreus erat inter úmeros eius. Hastíle autem hastæ eius erat quasi liciatórium texéntium, ipsum autem ferrum hastæ eius sescéntos siclos habébat ferri; et ármiger eius antecedébat eum. Stansque clamábat advérsum ágmina Israel et dicébat eis: «Quare venítis paráti ad prœ́lium? Numquid ego non sum Philisthǽus, et vos servi Saul? Elígite ex vobis virum, et descéndat ad singuláre certámen! Si quíverit pugnáre mecum et percússerit me, érimus vobis servi; si autem ego prævalúero et percússero eum, vos servi éritis et serviétis nobis». Et aiébat Philisthǽus: «Ego exprobrávi agmínibus Israel hódie: Date mihi virum, et íneat mecum singuláre certámen!».

\noindent Cum David fuísset addúctus ad Saul, locútus est ei: «Non cóncidat cor cuiúsquam in eo; ego servus tuus vadam et pugnábo advérsus Philisthǽum istum».

\noindent Et ait Saul ad David: “Non vales resístere Philisthǽo isti nec pugnáre advérsus eum, quia puer es; hic autem vir bellátor ab adulescéntia sua”.

\noindent Dixítque David ad Saul: “Pascébat servus tuus patris sui gregem, et veniébat leo vel ursus tollebátque aríetem de médio gregis.
Et sequébar eos et percutiébam eruebámque de ore eórum; et illi consurgébant advérsum me, et apprehendébam mentum eórum et percutiébam interficiebámque eos.
Nam et leónem et ursum interfécit servus tuus; erit ígitur et Philisthǽus hic incircumcísus quasi unus ex eis, quia ausus est maledícere exércitum Dei vivéntis”.

\noindent Et ait David: “Dóminus, qui éruit me de manu leónis et de manu ursi, ipse liberábit me de manu Philisthǽi huius”. Dixit autem Saul ad David: “Vade, et Dóminus tecum sit”.

\noindent Et índuit Saul David vestiméntis suis et impósuit gáleam ǽream super caput eius et vestívit eum loríca. Accínctus ergo David gládio eius super vestem suam cœpit tentáre, si armátus posset incédere; non enim habébat consuetúdinem. Dixítque David ad Saul: «Non possum sic incédere, quia nec usum hábeo». Et depósuit ea et tulit báculum suum in manu sua; et elégit sibi quinque levíssimos lápides de torrénte et misit eos in peram pastorálem, qua ut sácculo lápidum utebátur, et fundam manu tulit et procéssit advérsum Philisthǽum.}
\newcommand{\responsoriumi}{\pars{Responsorium 1.} \scriptura{\Rbardot{} Cantor \Vbardot{} 1 Reg. 17, 37; \textbf{H395}}

\vspace{-5mm}

\responsorium{I}{temporalia/resp-deusomniumexauditorest-sinedox.gtex}{}}
\newcommand{\lectioii}{\pars{Lectio II.} \scriptura{1 Sam. 17, 41-51}

\noindent Ibat autem Philisthǽus incédens et appropínquans advérsum David, et ármiger eius ante eum. Cumque inspexísset Philisthǽus et vidísset David, despéxit eum; erat enim aduléscens rufus et pulcher aspéctu. Et dixit Philisthǽus ad David: «Numquid ego canis sum, quod tu venis ad me cum báculo?». Et maledíxit Philisthǽus David in diis suis; dixítque ad David: «Veni ad me, et dabo carnes tuas volatílibus cæli et béstiis terræ».

\noindent Dixit autem David ad Philisthǽum: «Tu venis ad me cum gládio et hasta et acínace; ego autem vénio ad te in nómine Dómini exercítuum, Dei ágminum Israel, quibus exprobrásti. Hódie dabit te Dóminus in manu mea, et percútiam te et áuferam caput tuum a te; et dabo cadáver tuum et cadávera castrórum Philísthim hódie volatílibus cæli et béstiis terræ, ut sciat omnis terra quia est Deus in Israel, et nóverit univérsa ecclésia hæc quia non in gládio nec in hasta salvat Dóminus: ipsíus enim est bellum, et tradet vos in manus nostras».

\noindent Cum ergo surrexísset Philisthǽus et veníret et appropinquáret contra David, festinávit David et cucúrrit ad pugnam advérsum Philisthǽum. Et misit manum suam in peram tulítque unum lápidem et funda iecit; et percússit Philisthǽum in fronte, et infíxus est lapis in fronte eius, et cécidit in fáciem suam super terram. Prævaluítque David advérsum Philisthǽum in funda et in lápide; percussúmque Philisthǽum interfécit. Cumque gládium non habéret in manu David, cucúrrit et stetit super Philisthǽum; et tulit gládium eius et edúxit eum de vagína sua et interfécit eum præcidítque caput eius.}
\newcommand{\responsoriumii}{\pars{Responsorium 2.} \scriptura{\Rbardot{} 1 Reg. 17, 37 \Vbardot{} Ps. 56, 4.5; \textbf{H395}}

\vspace{-5mm}

\responsorium{VIII}{temporalia/resp-dominusquieripuitme-sinedox.gtex}{}}
\newcommand{\lectioiii}{\pars{Lectio III.} \scriptura{PG 46, 254-255}

\noindent Ex Tractátu sancti Gregórii Nysséni epíscopi De perfécta christiáni forma.

\noindent Paulus {\color{gray} máxime ómnium exquisíte, et qui Christus sit novit, et qualem esse opórteat, qui ab eo nomen accépit, ex iis quæ gessit ipse, declarávit: nam ádeo accuráte illum imitátus est, ut se in Dóminum ipsum expréssum osténderit, quippe qui diligentíssima imitatióne formam ánimi sui ita tránstulit in ipsum exémplar, ut non ámplius, qui loquebátur, Paulus, sed Christus esse viderétur, quemádmodum ípsemet dicit, qui própria bona pulchre sentiébat: \emph{Quóniam experiméntum,} inquit, \emph{quǽritis eius qui in me lóquitur, Christus.} Et: \emph{Vivo ego, iam non ego, vivit autem in me Christus.}

\noindent Hic ígitur} nobis et quam vim nomen hoc Christus hábeat patefécit, cum díceret Christum esse Dei virtútem et Dei sapiéntiam, cumque et pacem ipsum nomináret et lucem inaccessíbilem, in qua Deus inhábitat, expiatiónem et redemptiónem et sacerdótem magnum et Pascha et propitiatiónem animárum, splendórem glóriæ et figúram substántiæ et effectórem sæculórum, cibum ac potum spiritálem, petram et aquam, fundaméntum fídei, et ánguli caput, et Dei invisíbilis imáginem, et magnum Deum, caput córporis Ecclésiæ, et novæ creatúræ primogénitum, et primítias eórum qui dormiérunt, et primogénitum ex mórtuis, et primogénitum in multis frátribus, et mediatórem Dei et hóminum, et Fílium unigénitum glória et honóre coronátum, et Dóminum glóriæ, et rerum princípium, et regem iustítiæ, ad hæc et regem pacis, et regem ómnium, impérium regni nullis términis circumscríptum obtinéntem.

\noindent His et áliis id genus nomínibus eum appellávit, quæ tam multa sunt, ut præ multitúdine haud fácile número comprehéndi possint. Quæ quidem ómnia si inter se componántur et singulórum colligántur significatiónes, mirábilem nobis huius nóminis Christi vim apérient et maiestátis illíus quæ verbis explicári nequit, tantum osténdent, quantum ánimis et cogitatióne cápere valuérimus.

\noindent Quámobrem cum ómnium máximum et diviníssimum et primum nomen Dómini nostri bónitas nobis impertíverit, ut Christi cognómine decoráti appellémur «christiáni», necésse est ut ómnia nómina, quæ vocem hanc interpretántur, in nobis item conspiciántur expréssa, ne falso vocáti «christiáni» videámur, sed ex vita testimónium habeámus.}
\newcommand{\responsoriumiii}{\pars{Responsorium 3.} \scriptura{\Rbardot{} 1 Reg. 8, 28.29 \Vbardot{} Dt. 27, 15; \textbf{H396}}

\vspace{-5mm}

\responsorium{I}{temporalia/resp-audidominehymnum-CROCHU-cumdox.gtex}{}}
\newcommand{\hymnuslaudes}{\pars{Hymnus} \scriptura{Hilarius (\olddag{} 367)}

\grechangedim{interwordspacetext}{0.16 cm plus 0.15 cm minus 0.05 cm}{scalable}%
\cuminitiali{IV}{temporalia/hym-LucisLargitor.gtex}
\grechangedim{interwordspacetext}{0.22 cm plus 0.15 cm minus 0.05 cm}{scalable}%
\vspace{-3mm}}
\newcommand{\hebdomada}{infra Hebdom. XII post Pentecosten.}
\newcommand{\oratioLaudes}{\cuminitiali{}{temporalia/oratio12.gtex}}

% LuaLaTeX

\documentclass[a4paper, twoside, 12pt]{article}
\usepackage[latin]{babel}
%\usepackage[landscape, left=3cm, right=1.5cm, top=2cm, bottom=1cm]{geometry} % okraje stranky
%\usepackage[landscape, a4paper, mag=1166, truedimen, left=2cm, right=1.5cm, top=1.6cm, bottom=0.95cm]{geometry} % okraje stranky
\usepackage[landscape, a4paper, mag=1400, truedimen, left=0.5cm, right=0.5cm, top=0.5cm, bottom=0.5cm]{geometry} % okraje stranky

\usepackage{fontspec}
\setmainfont[FeatureFile={junicode.fea}, Ligatures={Common, TeX}, RawFeature=+fixi]{Junicode}
%\setmainfont{Junicode}

% shortcut for Junicode without ligatures (for the Czech texts)
\newfontfamily\nlfont[FeatureFile={junicode.fea}, Ligatures={Common, TeX}, RawFeature=+fixi]{Junicode}

\usepackage{multicol}
\usepackage{color}
\usepackage{lettrine}
\usepackage{fancyhdr}

% usual packages loading:
\usepackage{luatextra}
\usepackage{graphicx} % support the \includegraphics command and options
\usepackage{gregoriotex} % for gregorio score inclusion
\usepackage{gregoriosyms}
\usepackage{wrapfig} % figures wrapped by the text
\usepackage{parcolumns}
\usepackage[contents={},opacity=1,scale=1,color=black]{background}
\usepackage{tikzpagenodes}
\usepackage{calc}
\usepackage{longtable}
\usetikzlibrary{calc}

\setlength{\headheight}{14.5pt}

\input{conventuscommune.tex} % Often used macros

\newcommand{\annusEditionis}{2021}

%%%% Vicekrat opakovane kousky

\newcommand{\anteOrationem}{
  \rubrica{Ante Orationem, cantatur a Superiore:}

  \pars{Supplicatio Litaniæ.}

  \cuminitiali{}{temporalia/supplicatiolitaniae.gtex}

  \pars{Oratio Dominica.}

  \cuminitiali{}{temporalia/oratiodominica.gtex}

  \rubrica{Deinde dicitur ab Hebdomadario:}

  \cuminitiali{}{temporalia/dominusvobiscum-solemnis.gtex}

  \rubrica{In choro monialium loco Dominus vobiscum dicitur:}

  \sineinitiali{temporalia/domineexaudi.gtex}
}

\setlength{\columnsep}{30pt} % prostor mezi sloupci

%%%%%%%%%%%%%%%%%%%%%%%%%%%%%%%%%%%%%%%%%%%%%%%%%%%%%%%%%%%%%%%%%%%%%%%%%%%%%%%%%%%%%%%%%%%%%%%%%%%%%%%%%%%%%
\begin{document}

% Here we set the space around the initial.
% Please report to http://home.gna.org/gregorio/gregoriotex/details for more details and options
\grechangedim{afterinitialshift}{2.2mm}{scalable}
\grechangedim{beforeinitialshift}{2.2mm}{scalable}
\grechangedim{interwordspacetext}{0.22 cm plus 0.15 cm minus 0.05 cm}{scalable}%
\grechangedim{annotationraise}{-0.2cm}{scalable}

% Here we set the initial font. Change 38 if you want a bigger initial.
% Emit the initials in red.
\grechangestyle{initial}{\color{red}\fontsize{38}{38}\selectfont}

\pagestyle{empty}

%%%% Titulni stranka
\begin{titulusOfficii}
\ifx\titulus\undefined
\nomenFesti{Feria II \hebdomada{}}
\else
\titulus
\fi
\end{titulusOfficii}

\vfill

\begin{center}
%Ad usum et secundum consuetudines chori \guillemotright{}Conventus Choralis\guillemotleft.

%Editio Sancti Wolfgangi \annusEditionis
\end{center}

\scriptura{}

\pars{}

\pagebreak

\renewcommand{\headrulewidth}{0pt} % no horiz. rule at the header
\fancyhf{}
\pagestyle{fancy}

\cantusSineNeumas

\ifx\oratio\undefined
\ifx\laudb\undefined
\else
\newcommand{\oratio}{\pars{Oratio.}

\noindent Dómine Deus omnípotens, qui ad princípium huius diéi nos perveníre fecísti, tua nos hódie salva virtúte, ut in hac die ad nullum declinémus peccátum, sed semper ad tuam iustítiam faciéndam nostra procédant elóquia, dirigántur cogitatiónes et ópera.

\noindent Per Dóminum nostrum Iesum Christum, Fílium tuum, qui tecum vivit et regnat in unitáte Spíritus Sancti, Deus, per ómnia sǽcula sæculórum.

\noindent \Rbardot{} Amen.}
\fi
\fi

\hora{Ad Matutinum.} %%%%%%%%%%%%%%%%%%%%%%%%%%%%%%%%%%%%%%%%%%%%%%%%%%%%%
%\sideThumbs{Matutinum}

\vspace{2mm}

\cuminitiali{}{temporalia/dominelabiamea.gtex}

\vfill
%\pagebreak

\vspace{2mm}

\ifx\invitatorium\undefined
\pars{Invitatorium.} \scriptura{Ps. 94, 1; Psalmus 94; \textbf{H451}}

\vspace{-6mm}

\antiphona{VI}{temporalia/inv-jubilemusdeo.gtex}\else
\invitatorium
\fi

\vfill
\pagebreak

\ifx\hymnusmatutinum\undefined
\ifx\matua\undefined
\else
\pars{Hymnus.}

{
\grechangedim{interwordspacetext}{0.10 cm plus 0.15 cm minus 0.05 cm}{scalable}%
\antiphona{II}{temporalia/hym-IpsumNunc.gtex}
\grechangedim{interwordspacetext}{0.22 cm plus 0.15 cm minus 0.05 cm}{scalable}%
}
\fi
\else
\hymnusmatutinum
\fi

\vspace{-3mm}

\vfill
\pagebreak

\ifx\matub\undefined
\else
% MAT B
\pars{Psalmus 1.} \scriptura{Ps. 30, 2; \textbf{H90}}

\vspace{-4mm}

\antiphona{VIII G}{temporalia/ant-intuaiustitia.gtex}

%\vspace{-2mm}

\scriptura{Ps. 30, 2-9}

%\vspace{-2mm}

\initiumpsalmi{temporalia/ps30i-initium-viii-G-auto.gtex}

\vspace{-1.5mm}

\input{temporalia/ps30i-viii-G.tex} \Abardot{}

\vfill
\pagebreak

\pars{Psalmus 2.} \scriptura{Ps. 66, 2}

\vspace{-4mm}

\antiphona{E}{temporalia/ant-illuminadomine.gtex}

%\vspace{-2mm}

\scriptura{Ps. 30, 10-17}

%\vspace{-2mm}

\initiumpsalmi{temporalia/ps30ii-initium-e-a-auto.gtex}

\input{temporalia/ps30ii-e-a.tex} \Abardot{}

\vfill
\pagebreak

\pars{Psalmus 3.} \scriptura{Ps. 30, 24}

\vspace{-4mm}

\antiphona{II D}{temporalia/ant-diligitedominum.gtex}

%\vspace{-5mm}

\scriptura{Ps. 30, 20-25}

%\vspace{-2mm}

\initiumpsalmi{temporalia/ps30iii-initium-ii-D-auto.gtex}

\input{temporalia/ps30iii-ii-D.tex} \Abardot{}

\vfill
\pagebreak
\fi

\pars{Versus.}

\ifx\matversus\undefined
\ifx\matub\undefined
\else
\noindent \Vbardot{} Dírige me, Dómine, in veritáte tua, et doce me.

\noindent \Rbardot{} Quia tu es Deus salútis meæ.
\fi
\else
\matversus
\fi

\vspace{5mm}

\sineinitiali{temporalia/oratiodominica-mat.gtex}

\vspace{5mm}

\pars{Absolutio.}

\cuminitiali{}{temporalia/absolutio-exaudi.gtex}

\vfill
\pagebreak

\cuminitiali{}{temporalia/benedictio-solemn-benedictione.gtex}

\vspace{7mm}

\lectioi

\noindent \Vbardot{} Tu autem, Dómine, miserére nobis.
\noindent \Rbardot{} Deo grátias.

\vfill
\pagebreak

\responsoriumi

\vfill
\pagebreak

\cuminitiali{}{temporalia/benedictio-solemn-unigenitus.gtex}

\vspace{7mm}

\lectioii

\noindent \Vbardot{} Tu autem, Dómine, miserére nobis.
\noindent \Rbardot{} Deo grátias.

\vfill
\pagebreak

\responsoriumii

\vfill
\pagebreak

\cuminitiali{}{temporalia/benedictio-solemn-spiritus.gtex}

\vspace{7mm}

\lectioiii

\noindent \Vbardot{} Tu autem, Dómine, miserére nobis.
\noindent \Rbardot{} Deo grátias.

\vfill
\pagebreak

\responsoriumiii

\vfill
\pagebreak

\rubrica{Reliqua omittuntur, nisi Laudes separandæ sint.}

\sineinitiali{temporalia/domineexaudi.gtex}

\vfill

\oratio

\vfill

\noindent \Vbardot{} Dómine, exáudi oratiónem meam.
\Rbardot{} Et clamor meus ad te véniat.

\vfill

\noindent \Vbardot{} Benedicámus Dómino.
\noindent \Rbardot{} Deo grátias.

\vfill

\noindent \Vbardot{} Fidélium ánimæ per misericórdiam Dei requiéscant in pace.
\Rbardot{} Amen.

\vfill
\pagebreak

\hora{Ad Laudes.} %%%%%%%%%%%%%%%%%%%%%%%%%%%%%%%%%%%%%%%%%%%%%%%%%%%%%
%\sideThumbs{Laudes}

\cantusSineNeumas

\vspace{0.5cm}
\grechangedim{interwordspacetext}{0.18 cm plus 0.15 cm minus 0.05 cm}{scalable}%
\cuminitiali{}{temporalia/deusinadiutorium-communis.gtex}
\grechangedim{interwordspacetext}{0.22 cm plus 0.15 cm minus 0.05 cm}{scalable}%

\vfill
\pagebreak

\ifx\hymnuslaudes\undefined
\ifx\laudbd\undefined
\else
\pars{Hymnus} \scriptura{Hilarius (\olddag{} 367)}

\grechangedim{interwordspacetext}{0.16 cm plus 0.15 cm minus 0.05 cm}{scalable}%
\cuminitiali{IV}{temporalia/hym-LucisLargitor.gtex}
\grechangedim{interwordspacetext}{0.22 cm plus 0.15 cm minus 0.05 cm}{scalable}%
\vspace{-3mm}
\fi
\else
\hymnuslaudes
\fi

\vfill
\pagebreak

\ifx\laudb\undefined
\else
\pars{Psalmus 1.} \scriptura{Ps. 41, 3; \textbf{H391}}

\vspace{-4mm}

\antiphona{II D}{temporalia/ant-sitivitanima.gtex}

%\vspace{-2mm}

\scriptura{Psalmus 41}

%\vspace{-2mm}

\initiumpsalmi{temporalia/ps41-initium-ii-D-auto.gtex}

%\vspace{-1.5mm}

\input{temporalia/ps41-ii-D.tex}

\vfill

\antiphona{}{temporalia/ant-sitivitanima.gtex}

\vfill
\pagebreak

\pars{Psalmus 2.}

\vspace{-4mm}

\antiphona{III a}{temporalia/ant-ostendenobisdomine.gtex}

%\vspace{-2mm}

\scriptura{Canticum Ecclesiastici, Sir. 36, 1-7.13-16}

%\vspace{-3mm}

\initiumpsalmi{temporalia/ecclesiastici-initium-iii-a-auto.gtex}

\input{temporalia/ecclesiastici-iii-a.tex} \Abardot{}

\vfill
\pagebreak

\pars{Psalmus 3.}

\vspace{-4mm}

\antiphona{II D}{temporalia/ant-operamanuumeius.gtex}

\scriptura{Psalmus 18, 1-7}

\initiumpsalmi{temporalia/ps18i-initium-ii-D-auto.gtex}

\input{temporalia/ps18i-ii-D.tex} \Abardot{}

\vfill
\pagebreak
\fi

\ifx\lectiobrevis\undefined
\ifx\laudb\undefined
\else
\pars{Lectio Brevis.} \scriptura{Ier. 15, 16}

\noindent Invénti sunt sermónes tui, et comédi eos, et factum est mihi verbum tuum in gáudium et in lætítiam cordis mei, quóniam invocátum est nomen tuum super me, Dómine Deus exercítuum.
\fi
\else
\lectiobrevis
\fi

\vfill

\ifx\responsoriumbreve\undefined
\ifx\laudbd\undefined
\else
\pars{Responsorium breve.} \scriptura{Ps. 32, 1.3}

\cuminitiali{VI}{temporalia/resp-exsultateiusti.gtex}
\fi
\else
\responsoriumbreve
\fi

\vfill
\pagebreak

\ifx\benedictus\undefined
\ifx\laudbd\undefined
\else
\pars{Canticum Zachariæ.} \scriptura{Lc. 1, 68; \textbf{H422}}

\vspace{-4mm}

{
\grechangedim{interwordspacetext}{0.18 cm plus 0.15 cm minus 0.05 cm}{scalable}%
\antiphona{IV E}{temporalia/ant-benedictusdominus.gtex}
\grechangedim{interwordspacetext}{0.22 cm plus 0.15 cm minus 0.05 cm}{scalable}%
}

%\vspace{-3mm}

\scriptura{Lc. 1, 68-79}

%\vspace{-2mm}

\cantusSineNeumas
\initiumpsalmi{temporalia/benedictus-initium-iv-E-auto.gtex}

%\vspace{-1.5mm}

\input{temporalia/benedictus-iv-E.tex} \Abardot{}
\fi
\else
\benedictus
\fi

\vspace{-1cm}

\vfill
\pagebreak

%\sideThumbs{{\scriptsize{}Fine horarum}}

\pars{Preces.}

\sineinitiali{}{temporalia/tonusprecum.gtex}

\ifx\preces\undefined
\ifx\laudb\undefined
\else
\noindent Salvátor noster fecit nos regnum et sacerdótium, ut hóstias Deo acceptábiles offerámus. \gredagger{} Grati ígitur eum invocémus:

\Rbardot{} Serva nos in tuo ministério, Dómine.

\noindent Christe, sacérdos ætérne, qui sanctum pópulo tuo sacerdótium concessísti, \gredagger{} concéde, ut spiritáles hóstias Deo acceptábiles iúgiter offerámus.

\Rbardot{} Serva nos in tuo ministério, Dómine.

\noindent Spíritus tui fructus nobis largíre propítius, \gredagger{} patiéntiam, benignitátem et mansuetúdinem.

\Rbardot{} Serva nos in tuo ministério, Dómine.

\noindent Da nobis te amáre, ut te, qui es cáritas, possideámus, \gredagger{} et bene ágere, ut per vitam étiam nostram te laudémus.

\Rbardot{} Serva nos in tuo ministério, Dómine.

\noindent Quæ frátribus nostris sunt utília, nos quǽrere concéde, \gredagger{} ut salútem facílius consequántur.

\Rbardot{} Serva nos in tuo ministério, Dómine.
\fi
\else
\preces
\fi

\vfill

\pars{Oratio Dominica.}

\cuminitiali{}{temporalia/oratiodominicaalt.gtex}

\vfill
\pagebreak

\rubrica{vel:}

\pars{Supplicatio Litaniæ.}

\cuminitiali{}{temporalia/supplicatiolitaniae.gtex}

\vfill

\pars{Oratio Dominica.}

\cuminitiali{}{temporalia/oratiodominica.gtex}

\vfill
\pagebreak

% Oratio. %%%
\oratio

\vspace{-1mm}

\vfill

\rubrica{Hebdomadarius dicit Dominus vobiscum, vel, absente sacerdote vel diacono, sic concluditur:}

\vspace{2mm}

\antiphona{C}{temporalia/dominusnosbenedicat.gtex}

\rubrica{Postea cantatur a cantore:}

\vspace{2mm}

\cuminitiali{IV}{temporalia/benedicamus-feria-laudes.gtex}

\vspace{1mm}

\vfill
\pagebreak

\end{document}

