\newcommand{\titulus}{\nomenFesti{In Dedicatione Basilicæ S. Mariæ.}
\dies{Die 5. Augusti.}}
\newcommand{\sineabsolutio}{Absolutio de B.M.V.}
\newcommand{\oratio}{\pars{Oratio.}

\noindent Famulórum tuórum, quǽsumus, Dómine, delíctis ignósce, ut, qui tibi placére de áctibus nostris non valémus, Genetrícis Fílii tui intercessióne salvémur.

\pars{Pro pace in universo mundo.} \scriptura{Sir. 50, 25; 2 Esdr. 4, 20; \textbf{H416}}

\vspace{-4mm}

\antiphona{II D}{temporalia/ant-dapacemdomine.gtex}

\vfill

\noindent Deus, a quo sancta desidéria, recta consília et iusta sunt ópera: da servis tuis illam, quam mundus dare non potest, pacem; ut et corda nostra mandátis tuis dédita, et hóstium subláta formídine, témpora sint tua protectióne tranquílla.

\noindent Per Dóminum nostrum Iesum Christum, Fílium tuum, qui tecum vivit et regnat in unitáte Spíritus Sancti, Deus, per ómnia sǽcula sæculórum.

\noindent \Rbardot{} Amen.}
\newcommand{\invitatorium}{\pars{Invitatorium.}

\vspace{-4mm}

\antiphona{V}{temporalia/inv-christummariaefilium.gtex}}
\newcommand{\hymnusmatutinum}{\pars{Hymnus.}

\vspace{-5mm}

\antiphona{II}{temporalia/hym-QuemTerra-alt.gtex}}
\newcommand{\matversus}{\noindent \Vbardot{} María conservábat ómnia verba hæc.

\noindent \Rbardot{} Cónferens in corde suo.}
\newcommand{\lectioi}{\vspace{-2mm}

\sineinitiali{temporalia/oratiodominica-mat.gtex}

\vspace{5mm}

\pars{Absolutio.}

\cuminitiali{}{temporalia/absolutio-precibus.gtex}

\vfill
\pagebreak

\cuminitiali{}{temporalia/benedictio-solemn-noscum.gtex}

\vspace{7mm}

\pars{Lectio I.} \scriptura{Prov. 8, 1-5.12-22}

\noindent De libro Proverbiórum.

\noindent Numquid non sapiéntia clámitat, et prudéntia dat vocem suam?

\noindent In summis vertícibus supra viam in médiis sémitis stans, iuxta portas ad intróitum civitátis, in ipsis fóribus conclámat:

\noindent «O viri, ad vos clámito, et vox mea ad fílios hóminum.

\noindent Intellégite, párvuli, astútiam, et, insipiéntes, animadvértite; ego sapiéntia hábito cum prudéntia et artem excogitándi invénio.

\noindent Timor Dómini odísse malum; arrogántiam et supérbiam et viam pravam et os bilíngue detéstor.

\noindent Meum est consílium et prudéntia, mea est prudéntia, mea est fortitúdo.

\noindent Per me reges regnant, et príncipes iusta decérnunt; per me duces ímperant, et poténtes decérnunt iustítiam.

\noindent Ego diligéntes me díligo; et qui mane vígilant ad me, invénient me.

\noindent Mecum sunt divítiæ et glória, opes supérbiæ et iustítia.

\noindent Mélior est enim fructus meus auro et obrýzo, et genímina mea argénto elécto.

\noindent In viis iustítiæ ámbulo, in médio semitárum iudícii, ut ditem diligéntes me et thesáuros eórum répleam.

\noindent Dóminus possédit me in inítio viárum suárum, ántequam quidquam fáceret a princípio; ab ætérno ordináta sum et ex antíquis, ántequam terra fíeret.

\noindent Nondum erant abýssi, et ego iam concépta eram, necdum fontes graves aquis, priúsquam montes demergeréntur, ante colles ego parturiébar.

\noindent Adhuc terram non fécerat et campos et inítium glebæ orbis terræ.

\noindent Quando præparábat cælos, áderam,

\noindent quando certa lege et gyro vallábat abýssos,

\noindent quando nubes firmábat sursum, et prævaluérunt fontes abýssi,

\noindent quando circúmdabat mari términum suum et aquis, ne transírent fines suos,

\noindent quando iecit fundaménta terræ, cum eo eram ut ártifex:

\noindent delectátio eius per síngulos dies ludens coram eo omni témpore, ludens in orbe terrárum, et delíciæ meæ esse cum fíliis hóminum.

\noindent Nunc ergo, fílii, audíte me:

\noindent beáti, qui custódiunt vias meas; audíte disciplínam et estóte sapiéntes et nolíte abícere eam.

\noindent Beátus homo, qui audit me et qui vígilat ad fores meas cotídie et obsérvat ad postes óstii mei.

\noindent Qui me invénerit, invéniet vitam et háuriet delícias a Dómino.

\noindent Qui autem in me peccáverit, lædet ánimam suam: omnes, qui me odérunt, díligunt mortem».

\noindent \Vbardot{} Tu autem, Dómine, miserére nobis.
\noindent \Rbardot{} Deo grátias.}
\newcommand{\responsoriumi}{\pars{Responsorium 1.} \scriptura{\Rbardot{} Prv. 8, 23-25 \Vbardot{} ibid. 8, 26.30; \textbf{H398}}

\vspace{-5mm}

\responsorium{I}{temporalia/resp-inprincipiodeusantequam-CROCHU.gtex}{}

\rubrica{vel ad libitum:}

\vspace{3mm}

\pars{Responsorium 1.} \scriptura{\Rbardot{} Eccli. 24, 8-11 \Vbardot{} ibid. 24, 7; \textbf{H399}}

\vspace{-5mm}

\responsorium{VI}{temporalia/resp-gyrumcaeli-CROCHU.gtex}{}}
\newcommand{\lectioii}{\vspace{-7mm}

\cuminitiali{}{temporalia/benedictio-solemn-ipsavirgo.gtex}

\vspace{7mm}

\pars{Lectio II.} \scriptura{Hom. 4: PG 77, 991. 995-996}

\noindent Ex Homília sancti Cyrílli Alexandríni epíscopi in Concílio Ephesíno hábita.

\noindent Sanctórum cœtum, qui a sancta et Deípara sempérque Vírgine María invitáti prompto ánimo huc confluxérunt, lætum erectúmque conspício. Quare licet multa prémerer mæstítia, áttamen hic sanctórum patrum conspéctus lætítiam mihi prǽbuit. Nunc dulce illud hymnógraphi Dávidis verbum apud nos implétum est: \emph{Ecce quid bonum aut quid iucúndum, nisi habitáre fratres in unum?}

\noindent Salve ítaque a nobis, Sancta mýstica Trínitas, quæ nos omnes in hanc Sanctæ Maríæ Deíparæ ecclésiam convocásti.

\noindent Salve a nobis, Deípara María, venerándus totíus orbis thesáurus, lampas inexstinguíbilis, coróna virginitátis, sceptrum rectæ doctrínæ, templum indissolúbile, locus eius qui loco capi non potest, mater et virgo, per quam is benedíctus in sanctis Evangéliis nominátur, \emph{qui venit in nómine Dómini.}

\noindent Salve, quæ imménsum incomprehensúmque in sancto virgíneo útero comprehendísti; per quam Sancta Trínitas glorificátur et adorátur; per quam pretiósa crux celebrátur et in univérso orbe adorátur; per quam cælum exsúltat; per quam ángeli et archángeli lætántur; per quam dǽmones fugántur; per quam tentátor diábolus cælo décidit; per quam prolápsa creatúra in cælum assúmitur; per quam univérsa creatúra, idolórum vesánia deténta, ad veritátis agnitiónem pérvenit; per quam sanctum baptísma obtíngit credéntibus; per quam exsultatiónis óleum, per quam toto terrárum orbe fundátæ sunt Ecclésiæ, per quam gentes adducúntur ad pæniténtiam.

\noindent \Vbardot{} Tu autem, Dómine, miserére nobis.
\noindent \Rbardot{} Deo grátias.}
\newcommand{\responsoriumii}{\pars{Responsorium 2.} \scriptura{\Vbardot{} Lc. 1, 28; \textbf{H47}}

\vspace{-5mm}

\responsorium{II}{temporalia/resp-sanctaetimmaculata-CROCHU.gtex}{}}
\newcommand{\lectioiii}{\vspace{-7mm}

\cuminitiali{}{temporalia/benedictio-solemn-pervirginem.gtex}

\vspace{7mm}

\pars{Lectio III.}

\noindent Et quid plura dicam? per quam unigénitus Dei Fílius iis, \emph{qui in ténebris et in umbra mortis sedébant,} lux resplénduit; per quam prophétæ prænuntiárunt; per quam Apóstoli salútem géntibus prædicárunt; per quam mórtui exsuscitántur; per quam reges regnant, per Sanctam Trinitátem.

\noindent Ecquis hóminum laudabilíssimam illam Maríam pro dignitáte celebráre queat? Ipsa et mater et virgo est; o rem admirándam! Miráculum hoc me in stupórem rapit. Quis umquam audívit ædificatórem prohibéri ne próprium templum, quod ipse constrúxerit, inhabitáret? Quis ob id ignomíniæ sit obnóxius quod própriam fámulam in matrem ascíscat?

\noindent Ecce ígitur ómnia gaudent; contíngat autem nobis ut uniónem revereámur et adorémus, ac indivísam Trinitátem tremámus et colámus, Maríam semper Vírginem, sanctum vidélicet Dei templum, eiusdémque Fílium et sponsum immaculátum láudibus celebrántes: quóniam ipsi glória in sǽcula sæculórum. Amen.

\noindent \Vbardot{} Tu autem, Dómine, miserére nobis.
\noindent \Rbardot{} Deo grátias.}
\newcommand{\responsoriumiii}{\pars{Responsorium 3.} \scriptura{\Vbardot{} Sedulius; \textbf{H48}}

\vspace{-5mm}

\responsorium{VII}{temporalia/resp-congratulaminiquiacum-CROCHU-cumdox.gtex}{}}
\newcommand{\hymnuslaudes}{\pars{Hymnus}

\cuminitiali{I}{temporalia/hym-AveMarisStella.gtex}}
\newcommand{\lectiobrevis}{\pars{Lectio Brevis.} \scriptura{Cf. Is. 61, 10}

\noindent Gaudens gaudébo in Dómino, et exsultábit ánima mea in Deo meo, quia índuit me vestiméntis salútis et induménto iustítiæ circúmdedit me, quasi sponsam ornátam monílibus suis.}
\newcommand{\responsoriumbreve}{\pars{Responsorium breve.} \scriptura{Lc. 1, 28}

\cuminitiali{VI}{temporalia/resp-avemaria-alt.gtex}}
\newcommand{\preces}{\noindent Salvatórem nostrum celebrántes,~\gredagger{} qui ex María Vírgine nasci dignátus est,~\grestar{} exorémus dicéntes:

\Rbardot{} Intercédat pro nobis mater tua, Dómine.

\noindent Salvátor mundi,~\gredagger{} qui redemptiónis tuæ virtúte ab omni peccáti labe matrem tuam præservásti,~\grestar{} serva nos mundos a peccáto.

\Rbardot{} Intercédat pro nobis mater tua, Dómine.

\noindent Redémptor noster,~\gredagger{} qui Vírginem Maríam thálamum puríssimum habitatiónis tuæ et Spíritus Sancti fecísti sacrárium,~\grestar{} nos templum tui Spíritus fac perénne.

\Rbardot{} Intercédat pro nobis mater tua, Dómine.

\noindent Verbum ætérnum, quod matrem tuam docuísti óptimam sibi partem elígere,~\grestar{} tríbue nobis eam imitári, cibum quæréntes, qui permáneat in vitam ætérnam.

\Rbardot{} Intercédat pro nobis mater tua, Dómine.

\noindent Rex regum, qui matrem tuam córpore et ánima tecum voluísti in cælum assúmptam,~\grestar{} fac ut quæ sursum sunt semper cogitémus.

\Rbardot{} Intercédat pro nobis mater tua, Dómine.

\noindent Dómine cæli et terræ, qui Maríam regínam a dextris tuis astáre fecísti,~\grestar{} tríbue nos eiúsdem glóriæ meréri consórtium.

\Rbardot{} Intercédat pro nobis mater tua, Dómine.}
\newcommand{\benedictus}{\pars{Canticum Zachariæ.}

\vspace{-4mm}

\antiphona{VI C}{temporalia/ant-mentesnostrascastifica.gtex}

\vspace{-2mm}

\scriptura{Lc. 1, 68-79}

\vspace{-2mm}

\cantusSineNeumas
\initiumpsalmi{temporalia/benedictus-initium-vi-C-auto.gtex}

%\vspace{-1.5mm}

\input{temporalia/benedictus-vi-C.tex} \Abardot{}}
\newcommand{\benedicamuslaudes}{\cuminitiali{VIII}{temporalia/benedicamus-officium-bmv.gtex}}
\newcommand{\hebdomada}{infra Hebdom. XVIII post Pentecosten.}
\newcommand{\oratioLaudes}{\cuminitiali{}{temporalia/oratio18.gtex}}
\newcommand{\hiemalis}{Hiemalis.}

% LuaLaTeX

\documentclass[a4paper, twoside, 12pt]{article}
\usepackage[latin]{babel}
%\usepackage[landscape, left=3cm, right=1.5cm, top=2cm, bottom=1cm]{geometry} % okraje stranky
%\usepackage[landscape, a4paper, mag=1166, truedimen, left=2cm, right=1.5cm, top=1.6cm, bottom=0.95cm]{geometry} % okraje stranky
\usepackage[landscape, a4paper, mag=1400, truedimen, left=0.5cm, right=0.5cm, top=0.5cm, bottom=0.5cm]{geometry} % okraje stranky

\usepackage{fontspec}
\setmainfont[FeatureFile={junicode.fea}, Ligatures={Common, TeX}, RawFeature=+fixi]{Junicode}
%\setmainfont{Junicode}

% shortcut for Junicode without ligatures (for the Czech texts)
\newfontfamily\nlfont[FeatureFile={junicode.fea}, Ligatures={Common, TeX}, RawFeature=+fixi]{Junicode}

\usepackage{multicol}
\usepackage{color}
\usepackage{lettrine}
\usepackage{fancyhdr}

% usual packages loading:
\usepackage{luatextra}
\usepackage{graphicx} % support the \includegraphics command and options
\usepackage{gregoriotex} % for gregorio score inclusion
\usepackage{gregoriosyms}
\usepackage{wrapfig} % figures wrapped by the text
\usepackage{parcolumns}
\usepackage[contents={},opacity=1,scale=1,color=black]{background}
\usepackage{tikzpagenodes}
\usepackage{calc}
\usepackage{longtable}
\usetikzlibrary{calc}

\setlength{\headheight}{14.5pt}

\input{conventuscommune.tex} % Often used macros

\newcommand{\annusEditionis}{2021}

%%%% Vicekrat opakovane kousky

\newcommand{\anteOrationem}{
  \rubrica{Ante Orationem, cantatur a Superiore:}

  \pars{Supplicatio Litaniæ.}

  \cuminitiali{}{temporalia/supplicatiolitaniae.gtex}

  \pars{Oratio Dominica.}

  \cuminitiali{}{temporalia/oratiodominica.gtex}

  \rubrica{Deinde dicitur ab Hebdomadario:}

  \cuminitiali{}{temporalia/dominusvobiscum-solemnis.gtex}

  \rubrica{In choro monialium loco Dominus vobiscum dicitur:}

  \sineinitiali{temporalia/domineexaudi.gtex}
}

\setlength{\columnsep}{30pt} % prostor mezi sloupci

%%%%%%%%%%%%%%%%%%%%%%%%%%%%%%%%%%%%%%%%%%%%%%%%%%%%%%%%%%%%%%%%%%%%%%%%%%%%%%%%%%%%%%%%%%%%%%%%%%%%%%%%%%%%%
\begin{document}

% Here we set the space around the initial.
% Please report to http://home.gna.org/gregorio/gregoriotex/details for more details and options
\grechangedim{afterinitialshift}{2.2mm}{scalable}
\grechangedim{beforeinitialshift}{2.2mm}{scalable}
\grechangedim{interwordspacetext}{0.22 cm plus 0.15 cm minus 0.05 cm}{scalable}%
\grechangedim{annotationraise}{-0.2cm}{scalable}

% Here we set the initial font. Change 38 if you want a bigger initial.
% Emit the initials in red.
\grechangestyle{initial}{\color{red}\fontsize{38}{38}\selectfont}

\pagestyle{empty}

%%%% Titulni stranka
\begin{titulusOfficii}
\ifx\titulus\undefined
\nomenFesti{Feria III \hebdomada{}}
\else
\titulus
\fi
\end{titulusOfficii}

\vfill

\begin{center}
%Ad usum et secundum consuetudines chori \guillemotright{}Conventus Choralis\guillemotleft.

%Editio Sancti Wolfgangi \annusEditionis
\end{center}

\scriptura{}

\pars{}

\pagebreak

\renewcommand{\headrulewidth}{0pt} % no horiz. rule at the header
\fancyhf{}
\pagestyle{fancy}

\cantusSineNeumas

\ifx\oratio\undefined
\ifx\laudb\undefined
\else
\newcommand{\oratio}{\pars{Oratio.}

\noindent Dómine Iesu Christe, lux vera, qui omnes hómines illúminas ad salútem, nobis, quǽsumus, concéde virtútem, ut ante te vias pacis et iustítiæ præparémus.

\noindent Qui vivis et regnas cum Deo Patre in unitáte Spíritus Sancti, Deus, per ómnia sǽcula sæculórum.

\noindent \Rbardot{} Amen.}
\fi
\fi

\hora{Ad Matutinum.} %%%%%%%%%%%%%%%%%%%%%%%%%%%%%%%%%%%%%%%%%%%%%%%%%%%%%

\vspace{2mm}

\cuminitiali{}{temporalia/dominelabiamea.gtex}

\vfill
%\pagebreak

\vspace{2mm}

\ifx\invitatorium\undefined
\ifx\matuac\undefined
\else
\pars{Invitatorium.} \scriptura{Ps. 94, 1; Psalmus 94; \textbf{H451}}

\vspace{-6mm}

\antiphona{VI}{temporalia/inv-jubilemusdeo.gtex}
\fi
\ifx\matubd\undefined
\else
\pars{Invitatorium.} \scriptura{Cantor; Psalmus 94; \textbf{H449}}

\vspace{-6mm}

\antiphona{E}{temporalia/inv-regemmagnum.gtex}
\fi
\else
\invitatorium
\fi

\vfill
\pagebreak

\ifx\hymnusmatutinum\undefined
\ifx\matuac\undefined
\else
\pars{Hymnus}

\cuminitiali{IV}{temporalia/hym-SomnoRefectis.gtex}
\fi
\ifx\matubd\undefined
\else
\pars{Hymnus.} \scriptura{Gregorius Magnus (\olddag{} 604)}

{
\grechangedim{interwordspacetext}{0.10 cm plus 0.15 cm minus 0.05 cm}{scalable}%
\antiphona{I}{temporalia/hym-NocteSurgentes.gtex}
\grechangedim{interwordspacetext}{0.22 cm plus 0.15 cm minus 0.05 cm}{scalable}%
}
\fi
\else
\hymnusmatutinum
\fi

\vspace{-3mm}

\vfill
\pagebreak

\ifx\matub\undefined
\else
% MAT B
\pars{Psalmus 1.} \scriptura{Ps. 36, 5; \textbf{H93}}

\vspace{-4mm}

\antiphona{VI F}{temporalia/ant-reveladomino.gtex}

%\vspace{-2mm}

\scriptura{Ps. 36, 1-11}

%\vspace{-2mm}

\initiumpsalmi{temporalia/ps36i_xi-initium-vi-F-auto.gtex}

\input{temporalia/ps36i_xi-vi-F.tex} \Abardot{}

\vfill
\pagebreak

\pars{Psalmus 2.}

\vspace{-4mm}

\antiphona{II D}{temporalia/ant-iuniorfui.gtex}

\vspace{-2mm}

\scriptura{Ps. 36, 12-29}

\vspace{-2mm}

\initiumpsalmi{temporalia/ps36xii_xxix-initium-ii-D-auto.gtex}

\input{temporalia/ps36xii_xxix-ii-D.tex}

\vfill

\antiphona{}{temporalia/ant-iuniorfui.gtex}

\vfill
\pagebreak

\pars{Psalmus 3.} \scriptura{Ps. 36, 3}

\vspace{-4mm}

\antiphona{VI F}{temporalia/ant-speraindomino.gtex}

%\vspace{-2mm}

\scriptura{Ps. 36, 30-40}

%\vspace{-2mm}

\initiumpsalmi{temporalia/ps36iii-initium-vi-F-auto.gtex}

\input{temporalia/ps36iii-vi-F.tex} \Abardot{}

\vfill
\pagebreak
\fi
\ifx\matuc\undefined
\else
% MAT C
\pars{Psalmus 1.} \scriptura{Ps. 67, 2}

\vspace{-4mm}

\antiphona{VII a}{temporalia/ant-exsurgatdeus.gtex}

%\vspace{-2mm}

\scriptura{Ps. 67, 2-11}

\initiumpsalmi{temporalia/ps67i-initium-vii-a-auto.gtex}

\input{temporalia/ps67i-vii-a.tex} \Abardot{}

\vfill
\pagebreak

\pars{Psalmus 2.}

\vspace{-4mm}

\antiphona{I f}{temporalia/ant-deusnosterdeussalvos.gtex}

%\vspace{-2mm}

\scriptura{Ps. 67, 12-24}

%\vspace{-2mm}

\initiumpsalmi{temporalia/ps67ii-initium-i-f-auto.gtex}

\input{temporalia/ps67ii-i-f.tex} \Abardot{}

\vfill
\pagebreak

\pars{Psalmus 3.} \scriptura{Ps. 67, 27; \textbf{H96}}

\vspace{-4mm}

\antiphona{D}{temporalia/ant-inecclesiis.gtex}

%\vspace{-2mm}

\scriptura{Ps. 67, 25-36}

\initiumpsalmi{temporalia/ps67iii-initium-d-g2-auto.gtex}

\input{temporalia/ps67iii-d-g2.tex} \Abardot{}

\vfill
\pagebreak
\fi

\pars{Versus.}

\ifx\matversus\undefined
\ifx\matub\undefined
\else
\noindent \Vbardot{} Bonitátem et prudéntiam et sciéntiam doce me.

\noindent \Rbardot{} Quia præcéptis tuis crédidi.
\fi
\ifx\matuc\undefined
\else
\noindent \Vbardot{} Audiam quid loquátur Dóminus Deus.

\noindent \Rbardot{} Loquétur pacem ad plebem suam.
\fi
\else
\matversus
\fi

\vspace{5mm}

\sineinitiali{temporalia/oratiodominica-mat.gtex}

\vspace{5mm}

\pars{Absolutio.}

\cuminitiali{}{temporalia/absolutio-ipsius.gtex}

\vfill
\pagebreak

\cuminitiali{}{temporalia/benedictio-solemn-deus.gtex}

\vspace{7mm}

\lectioi

\noindent \Vbardot{} Tu autem, Dómine, miserére nobis.
\noindent \Rbardot{} Deo grátias.

\vfill
\pagebreak

\responsoriumi

\vfill
\pagebreak

\cuminitiali{}{temporalia/benedictio-solemn-christus.gtex}

\vspace{7mm}

\lectioii

\noindent \Vbardot{} Tu autem, Dómine, miserére nobis.
\noindent \Rbardot{} Deo grátias.

\vfill
\pagebreak

\responsoriumii

\vfill
\pagebreak

\cuminitiali{}{temporalia/benedictio-solemn-ignem.gtex}

\vspace{7mm}

\lectioiii

\noindent \Vbardot{} Tu autem, Dómine, miserére nobis.
\noindent \Rbardot{} Deo grátias.

\vfill
\pagebreak

\responsoriumiii

\vfill
\pagebreak

\rubrica{Reliqua omittuntur, nisi Laudes separandæ sint.}

\sineinitiali{temporalia/domineexaudi.gtex}

\vfill

\oratio

\vfill

\noindent \Vbardot{} Dómine, exáudi oratiónem meam.
\Rbardot{} Et clamor meus ad te véniat.

\vfill

\noindent \Vbardot{} Benedicámus Dómino.
\noindent \Rbardot{} Deo grátias.

\vfill

\noindent \Vbardot{} Fidélium ánimæ per misericórdiam Dei requiéscant in pace.
\Rbardot{} Amen.

\vfill
\pagebreak

\hora{Ad Laudes.} %%%%%%%%%%%%%%%%%%%%%%%%%%%%%%%%%%%%%%%%%%%%%%%%%%%%%

\cantusSineNeumas

\vspace{0.5cm}
\grechangedim{interwordspacetext}{0.18 cm plus 0.15 cm minus 0.05 cm}{scalable}%
\cuminitiali{}{temporalia/deusinadiutorium-communis.gtex}
\grechangedim{interwordspacetext}{0.22 cm plus 0.15 cm minus 0.05 cm}{scalable}%

\vfill
\pagebreak

\ifx\hymnuslaudes\undefined
\ifx\laudac\undefined
\else
\pars{Hymnus} \scriptura{Ambrosius (\olddag{} 397)}

\cuminitiali{I}{temporalia/hym-SplendorPaternae-hiemalis.gtex}
\fi
\ifx\laudbd\undefined
\else
\pars{Hymnus}

\grechangedim{interwordspacetext}{0.16 cm plus 0.15 cm minus 0.05 cm}{scalable}%
\cuminitiali{IV}{temporalia/hym-AEterneLucis.gtex}
\grechangedim{interwordspacetext}{0.22 cm plus 0.15 cm minus 0.05 cm}{scalable}%
\vspace{-3mm}
\fi
\else
\hymnuslaudes
\fi

\vfill
\pagebreak

\ifx\laudb\undefined
\else
\pars{Psalmus 1.} \scriptura{Ps. 42, 5; \textbf{H95}}

\vspace{-4mm}

\antiphona{VI F}{temporalia/ant-salutarevultusmei.gtex}

\scriptura{Psalmus 42.}

\initiumpsalmi{temporalia/ps42-initium-vi-F-auto.gtex}

\input{temporalia/ps42-vi-F.tex} \Abardot{}

\vfill
\pagebreak

\pars{Psalmus 2.} \scriptura{Is. 38, 20; \textbf{H95}}

\vspace{-7mm}

\antiphona{E}{temporalia/ant-cunctisdiebus.gtex}

\vspace{-4mm}

\scriptura{Canticum Ezechiæ, Is. 38, 10-20}

\vspace{-3mm}

\initiumpsalmi{temporalia/ezechiae-initium-e-auto.gtex}

\input{temporalia/ezechiae-e.tex} \Abardot{}

\vfill
\pagebreak

\pars{Psalmus 3.} \scriptura{Ps. 64, 2; \textbf{H96}}

\vspace{-4mm}

\antiphona{VIII a}{temporalia/ant-tedecet.gtex}

\vspace{-2mm}

\scriptura{Psalmus 64.}

\vspace{-2mm}

\initiumpsalmi{temporalia/ps64-initium-viii-A-auto.gtex}

\input{temporalia/ps64-viii-A.tex} \Abardot{}

\vfill
\pagebreak
\fi
\ifx\laudc\undefined
\else
\pars{Psalmus 1.} \scriptura{Ps. 83, 5}

\vspace{-4mm}

\antiphona{VIII G}{temporalia/ant-beatiquihabitant.gtex}

\vspace{-2mm}

\scriptura{Psalmus 84.}

\vspace{-2mm}

\initiumpsalmi{temporalia/ps84-initium-viii-G-auto.gtex}

\input{temporalia/ps84-viii-G.tex} \Abardot{}

\vfill
\pagebreak

\pars{Psalmus 2.}

\vspace{-4mm}

\antiphona{VII d}{temporalia/ant-denoctespiritusmeus.gtex}

\vspace{-2mm}

\scriptura{Canticum Isaiæ, Is. 26, 1-12}

\vspace{-2mm}

\initiumpsalmi{temporalia/isaiae3-initium-vii-d.gtex}

\input{temporalia/isaiae3-vii-d.tex} \Abardot{}

\vfill
\pagebreak

\pars{Psalmus 3.} \scriptura{Ps. 66, 2}

\vspace{-4mm}

\antiphona{E}{temporalia/ant-illuminadomine.gtex}

%\vspace{-2mm}

\scriptura{Psalmus 66.}

%\vspace{-2mm}

\initiumpsalmi{temporalia/ps66-initium-e.gtex}

\input{temporalia/ps66-e.tex} \Abardot{}

\vfill
\pagebreak
\fi

\ifx\lectiobrevis\undefined
\ifx\laudb\undefined
\else
\pars{Lectio Brevis.} \scriptura{1 Th. 5, 4-5}

\noindent Vos, fratres, non estis in ténebris, ut vos dies ille tamquam fur comprehéndat; omnes enim vos fílii lucis estis et fílii diéi. Non sumus noctis neque tenebrárum.
\fi
\ifx\laudc\undefined
\else
\pars{Lectio Brevis.} \scriptura{1 Io. 4, 14-15}

\noindent Nos vídimus et testificámur quóniam Pater misit Fílium salvatórem mundi. Quisque conféssus fúerit: Iesus est Fílius Dei, Deus in ipso manet, et ipse in Deo.
\fi
\else
\lectiobrevis
\fi

\vfill

\ifx\responsoriumbreve\undefined
\ifx\laudac\undefined
\else
\pars{Responsorium breve.}

\cuminitiali{VI}{temporalia/resp-benedictusdominus.gtex}
\fi
\ifx\laudbd\undefined
\else
\pars{Responsorium breve.} \scriptura{Ps. 118, 149.147}

\cuminitiali{VI}{temporalia/resp-vocemmeamaudi.gtex}
\fi
\else
\responsoriumbreve
\fi

\vfill
\pagebreak

\ifx\benedictus\undefined
\ifx\laudbd\undefined
\else
\pars{Canticum Zachariæ.} \scriptura{Lc. 1, 71; \textbf{H423}}

\vspace{-5mm}

{
\grechangedim{interwordspacetext}{0.18 cm plus 0.15 cm minus 0.05 cm}{scalable}%
\antiphona{I g\textsuperscript{5}}{temporalia/ant-demanuomnium.gtex}
\grechangedim{interwordspacetext}{0.22 cm plus 0.15 cm minus 0.05 cm}{scalable}%
}

%\vspace{-3mm}

\scriptura{Lc. 1, 68-79}

%\vspace{-1mm}

\initiumpsalmi{temporalia/benedictus-initium-i-g5-auto.gtex}

\input{temporalia/benedictus-i-g5.tex} \Abardot{}
\fi
\else
\benedictus
\fi

\vspace{-1cm}

\vfill
\pagebreak

\pars{Preces.}

\sineinitiali{}{temporalia/tonusprecum.gtex}

\ifx\preces\undefined
\ifx\laudb\undefined
\else
\noindent Salvatóri nostro benedicámus, qui sua resurrectióne mundum clarificávit, \gredagger{} et humíliter invocémus eum dicéntes:

\Rbardot{} Salva nos, Dómine, in sémita tua.

\noindent Resurrectiónem tuam, Dómine Iesu, oratióne cólimus matutína, \gredagger{} spes glóriæ tuæ diem nostrum illúminet.

\Rbardot{} Salva nos, Dómine, in sémita tua.

\noindent Súscipe, Dómine, vota et propósita nostra, \gredagger{} tamquam diéi nostri primítias.

\Rbardot{} Salva nos, Dómine, in sémita tua.

\noindent Tríbue in dilectióne tua nos hódie profícere, \gredagger{} ut ómnia in nostrum omniúmque bonum cooperéntur.

\Rbardot{} Salva nos, Dómine, in sémita tua.

\noindent Da, Dómine, sic lucére lucem nostram coram homínibus, \gredagger{} ut vídeant ópera nostra bona et Patrem gloríficent.

\Rbardot{} Salva nos, Dómine, in sémita tua.
\fi
\else
\preces
\fi

\vfill

\pars{Oratio Dominica.}

\cuminitiali{}{temporalia/oratiodominicaalt.gtex}

\vfill
\pagebreak

\rubrica{vel:}

\pars{Supplicatio Litaniæ.}

\cuminitiali{}{temporalia/supplicatiolitaniae.gtex}

\vfill

\pars{Oratio Dominica.}

\cuminitiali{}{temporalia/oratiodominica.gtex}

\vfill
\pagebreak

% Oratio. %%%
\oratio

\vspace{-1mm}

\vfill

\rubrica{Hebdomadarius dicit Dominus vobiscum, vel, absente sacerdote vel diacono, sic concluditur:}

\vspace{2mm}

\antiphona{C}{temporalia/dominusnosbenedicat.gtex}

\rubrica{Postea cantatur a cantore:}

\vspace{2mm}

\cuminitiali{IV}{temporalia/benedicamus-feria-laudes.gtex}

\vspace{1mm}

\vfill
\pagebreak

\end{document}

