\newcommand{\titulus}{\nomenFesti{Ss. Pauli Miki \& Sociorum, Martyrum.}
\dies{Die 6. Februarii.}}
\newcommand{\oratio}{\pars{Oratio.}

\noindent Deus, ómnium fortitúdo sanctórum, qui beátos mártyres Paulum eiúsque sócios per crucem ad vitam vocáre dignátus es, præsta, quǽsumus, ut, eórum intercessióne, fidem quam profitémur usque ad mortem fórtiter teneámus.

\pars{Pro pace in Ucraina.} \scriptura{Sir. 50, 25; 2 Esdr. 4, 20; \textbf{H416}}

\vspace{-4mm}

\antiphona{II D}{temporalia/ant-dapacemdomine.gtex}

\vfill

\noindent Deus, a quo sancta desidéria, recta consília et iusta sunt ópera: da servis tuis illam, quam mundus dare non potest, pacem; ut et corda nostra mandátis tuis dédita, et hóstium subláta formídine, témpora sint tua protectióne tranquílla.

\noindent Per Dóminum nostrum Iesum Christum, Fílium tuum, qui tecum vivit et regnat in unitáte Spíritus Sancti, Deus, per ómnia sǽcula sæculórum.

\noindent \Rbardot{} Amen.}
\newcommand{\invitatorium}{\pars{Invitatorium.}

\vspace{-4mm}

\antiphona{E}{temporalia/inv-regemmartyrumsimplex.gtex}}
\newcommand{\hymnusmatutinum}{\pars{Hymnus}

\cuminitiali{IV}{temporalia/hym-RexGloriose.gtex}}
\newcommand{\lectioi}{\pars{Lectio I.} \scriptura{Gn. 2, 4b-10.15-17}

\noindent De libro Génesis.

\noindent In die quo fecit Dóminus Deus terram et cælum - omne virgúltum agri, ántequam orirétur in terra, omnísque herba regiónis, priúsquam germináret; non enim plúerat Dóminus Deus super terram, et homo non erat, qui operarétur humum, sed fons ascendébat e terra írrigans univérsam superfíciem terræ - tunc formávit Dóminus Deus hóminem púlverem de humo et inspirávit in nares eius spiráculum vitæ, et factus est homo in ánimam vivéntem.

\noindent Et plantávit Dóminus Deus paradísum in Eden ad oriéntem, in quo pósuit hóminem quem formáverat. Produxítque Dóminus Deus de humo omne lignum pulchrum visu et ad vescéndum suáve, lignum étiam vitæ in médio paradísi lignúmque sciéntiæ boni et mali. Et flúvius egrediebátur ex Eden ad irrigándum paradísum, qui inde divíditur in quáttuor cápita.

\noindent Tulit ergo Dóminus Deus hóminem et pósuit eum in paradíso Eden, ut operarétur et custodíret illum; præcipítque Dóminus Deus hómini dicens: «Ex omni ligno paradísi cómede; de ligno autem sciéntiæ boni et mali ne cómedas; in quocúmque enim die coméderis ex eo, morte moriéris.»}
\newcommand{\responsoriumi}{\pars{Responsorium 1.} \scriptura{\Rbardot{} Gn. 2, 15 \Vbardot{} ibid., 7; \textbf{H136}}

\vspace{-5mm}

\responsorium{VIII}{temporalia/resp-tulitergodominus-CROCHU.gtex}{}}
\newcommand{\lectioii}{\pars{Lectio II.} \scriptura{Gn. 2, 18-25}

\noindent Dixit quoque Dóminus Deus: «Non est bonum esse hóminem solum; fáciam ei adiutórium símile sui.» Formátis ígitur Dóminus Deus de humo cunctis animántibus agri et univérsis volatílibus cæli, addúxit ea ad Adæ, ut vidéret quid vocáret ea; omne enim quod vocávit Adam ánimæ vivéntis, ipsum est nomen eius. Appellavítque Adam nomínibus suis cuncta pécora et univérsa volatília cæli et omnes béstias agri; Adae vero non inveniebátur adiútor símilis eius.

\noindent Immísit ergo Dóminus Deus sopórem in Adam. Cumque obdormísset, tulit unam de costis eius et replévit carnem pro ea; et ædificávit Dóminus Deus costam quam túlerat de Adam, in mulíerem et addúxit eam ad Adam.

\noindent Dixítque Adam: «Hæc nunc os ex óssibus meis et caro de carne mea! Hæc vocábitur Virágo, quóniam de viro sumpta est hæc.»

\noindent Quam ob rem relínquet vir patrem suum et matrem et adhærébit uxóri suæ; et erunt in carnem unam. Erant autem utérque nudi, Adam scílicet et uxor eius, et non erubescébant.}
\newcommand{\responsoriumii}{\pars{Responsorium 2.} \scriptura{\Rbardot{} Gn. 2, 18 \Vbardot{} ibid., 20; \textbf{H136}}

\vspace{-5mm}

\responsorium{V}{temporalia/resp-dixitdominusdeusnonest-CROCHU.gtex}{}

\rubrica{vel ad libitum:}

\vspace{3mm}

\pars{Responsorium 2.} \scriptura{\Rbardot{} Gn. 2, 21 \Vbardot{} ibid., 23; \textbf{H137}}

\vspace{-5mm}

\responsorium{V}{temporalia/resp-immisitdominussoporem-CROCHU.gtex}{}}
\newcommand{\lectioiii}{\pars{Lectio III.} \scriptura{Cap. 14, 109-110: Acta Sanctorum, Febr. 1, 769}

\noindent Ex História martýrii sanctórum Pauli Miki eiúsque sociórum ab auctóre coǽvo conscrípta.

\noindent {\color{gray} Crúcibus defíxis, mirábile fuit vidére ómnium constántiam, ad quam eos hortabántur, partim Pater Pásius, partim Pater Rodríguez. Pater Commissárius semper cónstitit quasi immótus óculis in cælum coniéctis. Frater Martínus ad agéndum divínæ bonitáti grátias concinébat quosdam psalmos, adiúncto versículo \emph{In manus tuas, Dómine.} Frater Francíscus Blancus étiam clara voce Deo agébat grátias. Frater Gunsálvus ádmodum eláta voce dicébat domínicam oratiónem et salutatiónem angélicam.}

\noindent Paulus Miki frater noster, videns se consístere in suggéstu ómnium, quos umquam habúerat, honoratíssimo, primum circumstántibus apéruit se Iapónium et de Societáte Iesu esse, moríque propter annuntiátum Evangélium, et grátias Deo ob benefícium tam exímium ágere, deínde hæc verba subiécit: «Cum ad hoc témporis punctum pervénerim e vobis néminem esse opínor, qui me credat velle párcere veritáti. Decláro ítaque vobis nullam esse ad salútem viam, nisi quam tenent christiáni. Quæ cum dóceat me ignóscere inimícis, et ómnibus qui me offendérunt, ego libénter regi omnibúsque meæ mortis auctóribus ignósco, eósque rogo ut christiáno baptísmo initiári velint».

\noindent Hinc óculis in sócios convérsis cœpit eos in hoc extrémo conflíctu animáre; in quorum ómnium vúltibus lætítia quædam apparébat, sed in Ludovíco singuláris; cui cum álius quidam christiánus acclamáret brevi eum futúrum in paradíso, digitórum totiúsque córporis gestu gáudii pleno, ómnium in se spectántium óculos convértit.

\noindent Antónius, qui Ludovíci claudébat latus, óculis cælo infíxis, post invocátum sanctíssimum nomen Iesu et Maríæ occínuit psalmum \emph{Laudáte, púeri, Dóminum,} quem didícerat Nagasáki in institutióne catechéseos; in ea enim dédita ópera púeris quidam psalmi hanc in rem ediscéndi tradúntur.

\noindent Alii dénique repetébant: «Iesu, María», seréna fácie; nonnúlli étiam exhortabántur circumstántes ad vitam christiáno dignam; hisque et áliis simílibus actiónibus promptitúdinem moriéndi demonstrábant.

\noindent {\color{gray} Tum carnífices quáttuor cœpére vagínis (quæ Iapóniis usitátæ sunt) exímere lánceas, ad quarum horríficum conspéctum omnes fidéles acclamárunt: «Iesu, María» et, quod plus, est complorátio secúta miserábilis ipsos fériens cælos. Carnífices quemque eórum uno alteróque ictu, brevíssimo témpore, exanimárunt.}}
\newcommand{\responsoriumiii}{\pars{Responsorium 3.} \scriptura{\Vbardot{} Cf. Eph. 4, 4-5; \textbf{H367}}

\vspace{-5mm}

\responsorium{VIII}{temporalia/resp-virisancti-CROCHU-cumdox.gtex}{}}
\newcommand{\hymnuslaudes}{\pars{Hymnus}

\cuminitiali{VIII}{temporalia/hym-AEternaChristi.gtex}}
\newcommand{\lectiobrevis}{\pars{Lectio Brevis.} \scriptura{2 Cor. 1, 3-5}

\noindent Benedíctus Deus et Pater Dómini nostri Iesu Christi, Pater misericordiárum et Deus totíus consolatiónis, qui consolátur nos in omni tribulatióne nostra, ut possímus et ipsi consolári eos, qui in omni pressúra sunt, per exhortatiónem, qua exhortámur et ipsi a Deo; quóniam, sicut abúndant passiónes Christi in nobis, ita per Christum abúndat et consolátio nostra.}
\newcommand{\responsoriumbreve}{\pars{Responsorium breve.} \scriptura{Ex. 15, 2}

\cuminitiali{VI}{temporalia/resp-fortitudomeaetlausmea.gtex}}
\newcommand{\preces}{\noindent Fratres, Salvatórem nostrum, testem fidélem, \gredagger{} per mártyres interféctos propter verbum Dei, \grestar{} celebrémus, clamántes:

\Rbardot{} Redemísti nos Deo in sánguine tuo.

\noindent Per mártyres tuos, qui líbere mortem in testimónium fídei sunt ampléxi, \grestar{} da nobis, Dómine, veram spíritus libertátem.

\Rbardot{} Redemísti nos Deo in sánguine tuo.

\noindent Per mártyres tuos, qui fidem usque ad sánguinem sunt conféssi, \grestar{} da nobis, Dómine, puritátem fideíque constántiam.

\Rbardot{} Redemísti nos Deo in sánguine tuo.

\noindent Per mártyres tuos, qui, sustinéntes crucem, tua vestígia sunt secúti, \grestar{} da nobis, Dómine, ærúmnas vitæ fórtiter sustinére.

\Rbardot{} Redemísti nos Deo in sánguine tuo.

\noindent Per mártyres tuos, qui stolas suas lavérunt in sánguine Agni, \grestar{} da nobis, Dómine, omnes insídias carnis mundíque devíncere.

\Rbardot{} Redemísti nos Deo in sánguine tuo.}
\newcommand{\benedictus}{\pars{Canticum Zachariæ.} \scriptura{Mt. 5, 10; \textbf{H363}}

\vspace{-4mm}

\antiphona{VIII G}{temporalia/ant-beatiquipersecutionem.gtex}

\vspace{-2mm}

\scriptura{Lc. 1, 68-79}

\vspace{-2mm}

\cantusSineNeumas
\initiumpsalmi{temporalia/benedictus-initium-viii-G-auto.gtex}

%\vspace{-1.5mm}

\input{temporalia/benedictus-viii-G.tex} \Abardot{}}
\newcommand{\benedicamuslaudes}{\cuminitiali{}{temporalia/benedicamus-memoria-laudes.gtex}}
\newcommand{\hebdomada}{infra Hebdom. V post Pentecosten.}
\newcommand{\oratioLaudes}{\cuminitiali{}{temporalia/oratio5.gtex}}

% LuaLaTeX

\documentclass[a4paper, twoside, 12pt]{article}
\usepackage[latin]{babel}
%\usepackage[landscape, left=3cm, right=1.5cm, top=2cm, bottom=1cm]{geometry} % okraje stranky
%\usepackage[landscape, a4paper, mag=1166, truedimen, left=2cm, right=1.5cm, top=1.6cm, bottom=0.95cm]{geometry} % okraje stranky
\usepackage[landscape, a4paper, mag=1400, truedimen, left=0.5cm, right=0.5cm, top=0.5cm, bottom=0.5cm]{geometry} % okraje stranky

\usepackage{fontspec}
\setmainfont[FeatureFile={junicode.fea}, Ligatures={Common, TeX}, RawFeature=+fixi]{Junicode}
%\setmainfont{Junicode}

% shortcut for Junicode without ligatures (for the Czech texts)
\newfontfamily\nlfont[FeatureFile={junicode.fea}, Ligatures={Common, TeX}, RawFeature=+fixi]{Junicode}

\usepackage{multicol}
\usepackage{color}
\usepackage{lettrine}
\usepackage{fancyhdr}

% usual packages loading:
\usepackage{luatextra}
\usepackage{graphicx} % support the \includegraphics command and options
\usepackage{gregoriotex} % for gregorio score inclusion
\usepackage{gregoriosyms}
\usepackage{wrapfig} % figures wrapped by the text
\usepackage{parcolumns}
\usepackage[contents={},opacity=1,scale=1,color=black]{background}
\usepackage{tikzpagenodes}
\usepackage{calc}
\usepackage{longtable}
\usetikzlibrary{calc}

\setlength{\headheight}{14.5pt}

\input{conventuscommune.tex} % Often used macros

\newcommand{\annusEditionis}{2021}

%%%% Vicekrat opakovane kousky

\newcommand{\anteOrationem}{
  \rubrica{Ante Orationem, cantatur a Superiore:}

  \pars{Supplicatio Litaniæ.}

  \cuminitiali{}{temporalia/supplicatiolitaniae.gtex}

  \pars{Oratio Dominica.}

  \cuminitiali{}{temporalia/oratiodominica.gtex}

  \rubrica{Deinde dicitur ab Hebdomadario:}

  \cuminitiali{}{temporalia/dominusvobiscum-solemnis.gtex}

  \rubrica{In choro monialium loco Dominus vobiscum dicitur:}

  \sineinitiali{temporalia/domineexaudi.gtex}
}

\setlength{\columnsep}{30pt} % prostor mezi sloupci

%%%%%%%%%%%%%%%%%%%%%%%%%%%%%%%%%%%%%%%%%%%%%%%%%%%%%%%%%%%%%%%%%%%%%%%%%%%%%%%%%%%%%%%%%%%%%%%%%%%%%%%%%%%%%
\begin{document}

% Here we set the space around the initial.
% Please report to http://home.gna.org/gregorio/gregoriotex/details for more details and options
\grechangedim{afterinitialshift}{2.2mm}{scalable}
\grechangedim{beforeinitialshift}{2.2mm}{scalable}
\grechangedim{interwordspacetext}{0.22 cm plus 0.15 cm minus 0.05 cm}{scalable}%
\grechangedim{annotationraise}{-0.2cm}{scalable}

% Here we set the initial font. Change 38 if you want a bigger initial.
% Emit the initials in red.
\grechangestyle{initial}{\color{red}\fontsize{38}{38}\selectfont}

\pagestyle{empty}

%%%% Titulni stranka
\begin{titulusOfficii}
\ifx\titulus\undefined
\nomenFesti{Feria II \hebdomada{}}
\else
\titulus
\fi
\end{titulusOfficii}

\vfill

\begin{center}
%Ad usum et secundum consuetudines chori \guillemotright{}Conventus Choralis\guillemotleft.

%Editio Sancti Wolfgangi \annusEditionis
\end{center}

\scriptura{}

\pars{}

\pagebreak

\renewcommand{\headrulewidth}{0pt} % no horiz. rule at the header
\fancyhf{}
\pagestyle{fancy}

\cantusSineNeumas

\ifx\oratio\undefined
\ifx\laudb\undefined
\else
\newcommand{\oratio}{\pars{Oratio.}

\noindent Dómine Deus omnípotens, qui ad princípium huius diéi nos perveníre fecísti, tua nos hódie salva virtúte, ut in hac die ad nullum declinémus peccátum, sed semper ad tuam iustítiam faciéndam nostra procédant elóquia, dirigántur cogitatiónes et ópera.

\noindent Per Dóminum nostrum Iesum Christum, Fílium tuum, qui tecum vivit et regnat in unitáte Spíritus Sancti, Deus, per ómnia sǽcula sæculórum.

\noindent \Rbardot{} Amen.}
\fi
\fi

\hora{Ad Matutinum.} %%%%%%%%%%%%%%%%%%%%%%%%%%%%%%%%%%%%%%%%%%%%%%%%%%%%%
%\sideThumbs{Matutinum}

\vspace{2mm}

\cuminitiali{}{temporalia/dominelabiamea.gtex}

\vfill
%\pagebreak

\vspace{2mm}

\ifx\invitatorium\undefined
\pars{Invitatorium.} \scriptura{Ps. 94, 1; Psalmus 94; \textbf{H451}}

\vspace{-6mm}

\antiphona{VI}{temporalia/inv-jubilemusdeo.gtex}\else
\invitatorium
\fi

\vfill
\pagebreak

\ifx\hymnusmatutinum\undefined
\ifx\matua\undefined
\else
\pars{Hymnus.}

{
\grechangedim{interwordspacetext}{0.10 cm plus 0.15 cm minus 0.05 cm}{scalable}%
\antiphona{II}{temporalia/hym-IpsumNunc.gtex}
\grechangedim{interwordspacetext}{0.22 cm plus 0.15 cm minus 0.05 cm}{scalable}%
}
\fi
\else
\hymnusmatutinum
\fi

\vspace{-3mm}

\vfill
\pagebreak

\ifx\matub\undefined
\else
% MAT B
\pars{Psalmus 1.} \scriptura{Ps. 30, 2; \textbf{H90}}

\vspace{-4mm}

\antiphona{VIII G}{temporalia/ant-intuaiustitia.gtex}

%\vspace{-2mm}

\scriptura{Ps. 30, 2-9}

%\vspace{-2mm}

\initiumpsalmi{temporalia/ps30i-initium-viii-G-auto.gtex}

\vspace{-1.5mm}

\input{temporalia/ps30i-viii-G.tex} \Abardot{}

\vfill
\pagebreak

\pars{Psalmus 2.} \scriptura{Ps. 66, 2}

\vspace{-4mm}

\antiphona{E}{temporalia/ant-illuminadomine.gtex}

%\vspace{-2mm}

\scriptura{Ps. 30, 10-17}

%\vspace{-2mm}

\initiumpsalmi{temporalia/ps30ii-initium-e-a-auto.gtex}

\input{temporalia/ps30ii-e-a.tex} \Abardot{}

\vfill
\pagebreak

\pars{Psalmus 3.} \scriptura{Ps. 30, 24}

\vspace{-4mm}

\antiphona{II D}{temporalia/ant-diligitedominum.gtex}

%\vspace{-5mm}

\scriptura{Ps. 30, 20-25}

%\vspace{-2mm}

\initiumpsalmi{temporalia/ps30iii-initium-ii-D-auto.gtex}

\input{temporalia/ps30iii-ii-D.tex} \Abardot{}

\vfill
\pagebreak
\fi

\pars{Versus.}

\ifx\matversus\undefined
\ifx\matub\undefined
\else
\noindent \Vbardot{} Dírige me, Dómine, in veritáte tua, et doce me.

\noindent \Rbardot{} Quia tu es Deus salútis meæ.
\fi
\else
\matversus
\fi

\vspace{5mm}

\sineinitiali{temporalia/oratiodominica-mat.gtex}

\vspace{5mm}

\pars{Absolutio.}

\cuminitiali{}{temporalia/absolutio-exaudi.gtex}

\vfill
\pagebreak

\cuminitiali{}{temporalia/benedictio-solemn-benedictione.gtex}

\vspace{7mm}

\lectioi

\noindent \Vbardot{} Tu autem, Dómine, miserére nobis.
\noindent \Rbardot{} Deo grátias.

\vfill
\pagebreak

\responsoriumi

\vfill
\pagebreak

\cuminitiali{}{temporalia/benedictio-solemn-unigenitus.gtex}

\vspace{7mm}

\lectioii

\noindent \Vbardot{} Tu autem, Dómine, miserére nobis.
\noindent \Rbardot{} Deo grátias.

\vfill
\pagebreak

\responsoriumii

\vfill
\pagebreak

\cuminitiali{}{temporalia/benedictio-solemn-spiritus.gtex}

\vspace{7mm}

\lectioiii

\noindent \Vbardot{} Tu autem, Dómine, miserére nobis.
\noindent \Rbardot{} Deo grátias.

\vfill
\pagebreak

\responsoriumiii

\vfill
\pagebreak

\rubrica{Reliqua omittuntur, nisi Laudes separandæ sint.}

\sineinitiali{temporalia/domineexaudi.gtex}

\vfill

\oratio

\vfill

\noindent \Vbardot{} Dómine, exáudi oratiónem meam.
\Rbardot{} Et clamor meus ad te véniat.

\vfill

\noindent \Vbardot{} Benedicámus Dómino.
\noindent \Rbardot{} Deo grátias.

\vfill

\noindent \Vbardot{} Fidélium ánimæ per misericórdiam Dei requiéscant in pace.
\Rbardot{} Amen.

\vfill
\pagebreak

\hora{Ad Laudes.} %%%%%%%%%%%%%%%%%%%%%%%%%%%%%%%%%%%%%%%%%%%%%%%%%%%%%
%\sideThumbs{Laudes}

\cantusSineNeumas

\vspace{0.5cm}
\grechangedim{interwordspacetext}{0.18 cm plus 0.15 cm minus 0.05 cm}{scalable}%
\cuminitiali{}{temporalia/deusinadiutorium-communis.gtex}
\grechangedim{interwordspacetext}{0.22 cm plus 0.15 cm minus 0.05 cm}{scalable}%

\vfill
\pagebreak

\ifx\hymnuslaudes\undefined
\ifx\laudbd\undefined
\else
\pars{Hymnus} \scriptura{Hilarius (\olddag{} 367)}

\grechangedim{interwordspacetext}{0.16 cm plus 0.15 cm minus 0.05 cm}{scalable}%
\cuminitiali{IV}{temporalia/hym-LucisLargitor.gtex}
\grechangedim{interwordspacetext}{0.22 cm plus 0.15 cm minus 0.05 cm}{scalable}%
\vspace{-3mm}
\fi
\else
\hymnuslaudes
\fi

\vfill
\pagebreak

\ifx\laudb\undefined
\else
\pars{Psalmus 1.} \scriptura{Ps. 41, 3; \textbf{H391}}

\vspace{-4mm}

\antiphona{II D}{temporalia/ant-sitivitanima.gtex}

%\vspace{-2mm}

\scriptura{Psalmus 41}

%\vspace{-2mm}

\initiumpsalmi{temporalia/ps41-initium-ii-D-auto.gtex}

%\vspace{-1.5mm}

\input{temporalia/ps41-ii-D.tex}

\vfill

\antiphona{}{temporalia/ant-sitivitanima.gtex}

\vfill
\pagebreak

\pars{Psalmus 2.}

\vspace{-4mm}

\antiphona{III a}{temporalia/ant-ostendenobisdomine.gtex}

%\vspace{-2mm}

\scriptura{Canticum Ecclesiastici, Sir. 36, 1-7.13-16}

%\vspace{-3mm}

\initiumpsalmi{temporalia/ecclesiastici-initium-iii-a-auto.gtex}

\input{temporalia/ecclesiastici-iii-a.tex} \Abardot{}

\vfill
\pagebreak

\pars{Psalmus 3.}

\vspace{-4mm}

\antiphona{II D}{temporalia/ant-operamanuumeius.gtex}

\scriptura{Psalmus 18, 1-7}

\initiumpsalmi{temporalia/ps18i-initium-ii-D-auto.gtex}

\input{temporalia/ps18i-ii-D.tex} \Abardot{}

\vfill
\pagebreak
\fi

\ifx\lectiobrevis\undefined
\ifx\laudb\undefined
\else
\pars{Lectio Brevis.} \scriptura{Ier. 15, 16}

\noindent Invénti sunt sermónes tui, et comédi eos, et factum est mihi verbum tuum in gáudium et in lætítiam cordis mei, quóniam invocátum est nomen tuum super me, Dómine Deus exercítuum.
\fi
\else
\lectiobrevis
\fi

\vfill

\ifx\responsoriumbreve\undefined
\ifx\laudbd\undefined
\else
\pars{Responsorium breve.} \scriptura{Ps. 32, 1.3}

\cuminitiali{VI}{temporalia/resp-exsultateiusti.gtex}
\fi
\else
\responsoriumbreve
\fi

\vfill
\pagebreak

\ifx\benedictus\undefined
\ifx\laudbd\undefined
\else
\pars{Canticum Zachariæ.} \scriptura{Lc. 1, 68; \textbf{H422}}

\vspace{-4mm}

{
\grechangedim{interwordspacetext}{0.18 cm plus 0.15 cm minus 0.05 cm}{scalable}%
\antiphona{IV E}{temporalia/ant-benedictusdominus.gtex}
\grechangedim{interwordspacetext}{0.22 cm plus 0.15 cm minus 0.05 cm}{scalable}%
}

%\vspace{-3mm}

\scriptura{Lc. 1, 68-79}

%\vspace{-2mm}

\cantusSineNeumas
\initiumpsalmi{temporalia/benedictus-initium-iv-E-auto.gtex}

%\vspace{-1.5mm}

\input{temporalia/benedictus-iv-E.tex} \Abardot{}
\fi
\else
\benedictus
\fi

\vspace{-1cm}

\vfill
\pagebreak

%\sideThumbs{{\scriptsize{}Fine horarum}}

\pars{Preces.}

\sineinitiali{}{temporalia/tonusprecum.gtex}

\ifx\preces\undefined
\ifx\laudb\undefined
\else
\noindent Salvátor noster fecit nos regnum et sacerdótium, ut hóstias Deo acceptábiles offerámus. \gredagger{} Grati ígitur eum invocémus:

\Rbardot{} Serva nos in tuo ministério, Dómine.

\noindent Christe, sacérdos ætérne, qui sanctum pópulo tuo sacerdótium concessísti, \gredagger{} concéde, ut spiritáles hóstias Deo acceptábiles iúgiter offerámus.

\Rbardot{} Serva nos in tuo ministério, Dómine.

\noindent Spíritus tui fructus nobis largíre propítius, \gredagger{} patiéntiam, benignitátem et mansuetúdinem.

\Rbardot{} Serva nos in tuo ministério, Dómine.

\noindent Da nobis te amáre, ut te, qui es cáritas, possideámus, \gredagger{} et bene ágere, ut per vitam étiam nostram te laudémus.

\Rbardot{} Serva nos in tuo ministério, Dómine.

\noindent Quæ frátribus nostris sunt utília, nos quǽrere concéde, \gredagger{} ut salútem facílius consequántur.

\Rbardot{} Serva nos in tuo ministério, Dómine.
\fi
\else
\preces
\fi

\vfill

\pars{Oratio Dominica.}

\cuminitiali{}{temporalia/oratiodominicaalt.gtex}

\vfill
\pagebreak

\rubrica{vel:}

\pars{Supplicatio Litaniæ.}

\cuminitiali{}{temporalia/supplicatiolitaniae.gtex}

\vfill

\pars{Oratio Dominica.}

\cuminitiali{}{temporalia/oratiodominica.gtex}

\vfill
\pagebreak

% Oratio. %%%
\oratio

\vspace{-1mm}

\vfill

\rubrica{Hebdomadarius dicit Dominus vobiscum, vel, absente sacerdote vel diacono, sic concluditur:}

\vspace{2mm}

\antiphona{C}{temporalia/dominusnosbenedicat.gtex}

\rubrica{Postea cantatur a cantore:}

\vspace{2mm}

\cuminitiali{IV}{temporalia/benedicamus-feria-laudes.gtex}

\vspace{1mm}

\vfill
\pagebreak

\end{document}

