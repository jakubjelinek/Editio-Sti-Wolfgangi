\newcommand{\titulus}{\nomenFesti{S. Agathæ, Virginis \& Martyris.}
\dies{Die 5. Februarii.}}
\newcommand{\oratio}{\pars{Oratio.}

\noindent Indulgéntiam nobis, quǽsumus, Dómine, beáta Agatha, virgo et martyr, implóret, quæ tibi grata semper éxstitit et virtúte martýrii et mérito castitátis.

\noindent Per Dóminum nostrum Iesum Christum, Fílium tuum, qui tecum vivit et regnat in unitáte Spíritus Sancti, Deus, per ómnia sǽcula sæculórum.

\noindent \Rbardot{} Amen.}
\newcommand{\invitatorium}{\pars{Invitatorium.}

\vspace{-4mm}

\antiphona{E}{temporalia/inv-regemvirginum.gtex}}
\newcommand{\hymnusmatutinum}{\pars{Hymnus.}

\vspace{-5mm}

\antiphona{I}{temporalia/hym-OChristeFlos.gtex}}
\newcommand{\matversus}{\noindent \Vbardot{} Non cessámus pro vobis orántes et postulántes.

\noindent \Rbardot{} Ut impleámini agnitióne voluntátis Dei.}
\newcommand{\lectioi}{\pars{Lectio I.} \scriptura{Gn. 9, 18-29}

\noindent De libro Génesis.

\noindent Erant ígitur fílii Noe, qui egréssi sunt de arca, Sem Cham et Iafeth: porro Cham ipse est pater Chánaan.

\noindent Tres isti sunt fílii Noe: et ab his disseminátum est omne hóminum genus super univérsam terram.

\noindent Cœpítque Noë vir agrícola exercére terram, et plantávit víneam.

\noindent Bibénsque vinum inebriátus est, et nudátus in tabernáculo suo.

\noindent Quod cum vidísset Cham, pater Chánaan, verénda scílicet patris sui esse nudáta, nuntiávit duóbus frátribus suis foras.

\noindent At vero Sem et Iapheth pállium imposuérunt húmeris suis, et incedéntes retrórsum, operuérunt verénda patris sui: faciésque eórum avérsæ erant, et patris virília non vidérunt.

\noindent Evígilans autem Noë ex vino, cum didicísset quæ fécerat ei fílius suus minor, ait: Maledíctus Chánaan, servus servórum erit frátribus suis.

\noindent Dixítque: Benedíctus Dóminus Deus Sem, sit Chánaan servus eius. Dilátet Deus Iapheth, et hábitet in tabernáculis Sem, sitque Chánaan servus eius.

\noindent Vixit autem Noë post dilúvium trecéntis quinquagínta annis. Et impléti sunt omnes dies eius nongentórum quinquagínta annórum: et mórtuus est.}
\newcommand{\responsoriumi}{\pars{Responsorium 1.} \scriptura{Cf. Gn. 6, 9.22}

\vspace{-5mm}

\responsorium{VIII}{temporalia/resp-noeviriustus.gtex}{}

\rubrica{vel ad libitum:}

\vspace{3mm}

\pars{Responsorium 1.}

\vspace{-5mm}

\responsorium{I}{temporalia/resp-egressonoedearcha-cumdox.gtex}{}}
\newcommand{\lectioii}{\pars{Lectio II.} \scriptura{Analecta Bollandiana 68, 76-78}

\noindent Ex Oratióne sancti Methódii Sículi epíscopi in sanctam Agatham.

\noindent Anniversária sanctæ mártyris commemorátio in unum hunc locum nos omnes collégit, ut nostis, o mei auditóres, antíquæ quidem mártyris et primáriæ propter insígne certámen, sed vero recéntis, quæ quasi nunc decertándo vincat propter divína mirácula, quibus cotídie coronátur, pulchréque decorátur.

\noindent Virgo quidem est, quia ex immortális Dei Verbo (quod étiam propter me in carne sua mortem gustávit) indivisóque Dei Fílio nata est, ut lóquitur theólogus Ioánnes: \emph{Quotquot autem recepérunt eum, dedit eis potestátem fílios Dei fíeri.}

\noindent Múlier ígitur virgo, quæ vos invitávit ad nostrum religiósum épulum, illa est múlier bene despónsa uni viro Christo, ut verbis utar apóstoli Pauli de thálami coniunctióne.

\noindent Virgo, quæ consciéntiæ a lúmine, coloréque sánguinis veri diviníque Agni et lábia sua et genas et linguam purpurissábat exornabátque, necnon étiam contínua mentis suæ cognitióne necem sui studiósi amatóris, quasi nuper sánguine madéntis, revolvébat intúsque circumferébat; ut hæc stola eius confessiónis non solum indelébilem ferret Christi notam purpúrei sánguinis alte iam imbíbiti, sed étiam virginális eloquéntiæ thesáuros adveniéntibus pósteris elegántes eiúsmodi colóres una cum non deficiénte verbórum scaturígine impartíret.}
\newcommand{\responsoriumii}{\pars{Responsorium 2.} \scriptura{\textbf{H123}}

\vspace{-5mm}

\responsorium{IV}{temporalia/resp-egoautemadjuvata-CROCHU.gtex}{}}
\newcommand{\lectioiii}{\pars{Lectio III.}

\noindent Hæc ítaque vere bona, quippe Dei pars exsístens, eídem Sponso suo et rursus nobis boni communióne, qua nóminis sui vim significationémque refert, Agatha, ab ipso bonitátis fonte Deo concéssa, donóque data.

\noindent Quid autem magis benéficum quam summum bonum? et quid magis celebrándum laudum præcóniis quíspiam invéniat quam Agatham?

\noindent Agatha, cuius bónitas ad nomen, ad ipsámque rem quadrat; Agatha, quæ rebus præcláre gestis bonum nomen præ se fert, et in ipso nómine præcláre abs se gesta demónstrat; Agatha, quæ vel suo nómine állicit, ut ad sese omnes maximópere occúrrant, quæ étiam suo exémplo docet, ut ad verum bonum, qui solus Deus est, omnes sine mora secum conténdant.}
\newcommand{\responsoriumiii}{\pars{Responsorium 3.} \scriptura{\textbf{H124}}

\vspace{-5mm}

\responsorium{I}{temporalia/resp-gaudeamusomnes-CROCHU-cumdox.gtex}{}}
\newcommand{\hymnuslaudes}{\pars{Hymnus.}

\cuminitiali{VI}{temporalia/hym-OCastitatisSignifer.gtex}}
\newcommand{\lectiobrevis}{\pars{Lectio Brevis.} \scriptura{2 Cor. 1, 3-5}

\noindent Benedíctus Deus et Pater Dómini nostri Iesu Christi, Pater misericordiárum et Deus totíus consolatiónis, qui consolátur nos in omni tribulatióne nostra, ut possímus et ipsi consolári eos, qui in omni pressúra sunt, per exhortatiónem, qua exhortámur et ipsi a Deo; quóniam, sicut abúndant passiónes Christi in nobis, ita per Christum abúndat et consolátio nostra.}
\newcommand{\responsoriumbreve}{\pars{Responsorium breve.} \scriptura{Ex. 15, 2}

\cuminitiali{VI}{temporalia/resp-fortitudomeaetlausmea.gtex}}
\newcommand{\preces}{\noindent Christo, vírginum sponso et corónæ,~\gredagger{} lætis vócibus iubilémus~\grestar{} et devóte eum exorémus, dicéntes:

\Rbardot{} Iesu, coróna vírginum, audi nos.

\noindent Christe, quem sanctæ vírgines ut únicum sponsum dilexérunt,~\grestar{} concéde ut nihil a tua caritáte nos séparet.

\Rbardot{} Iesu, coróna vírginum, audi nos.

\noindent Qui Maríam, matrem tuam, regínam vírginum coronásti,~\grestar{} da nos, eius intercessióne, cordis puritáte tibi iúgiter famulári.

\Rbardot{} Iesu, coróna vírginum, audi nos.

\noindent Per intercessiónem ancillárum tuárum,~\gredagger{} quæ íntegro et indivíso corde semper de te sollícitæ fuérunt, ut essent sanctæ córpore et spíritu,~\grestar{} præsta ut figúra huius mundi, quæ prǽterit, nos a te numquam ábstrahat.

\Rbardot{} Iesu, coróna vírginum, audi nos.

\noindent Dómine Iesu,~\gredagger{} sponse, quem vírgines sapiéntes adventúrum exspectábant,~\grestar{} concéde nobis, ut in spe vigilántes te præstolémur.

\Rbardot{} Iesu, coróna vírginum, audi nos.

\noindent Per intercessiónem sanctæ Agathæ,~\gredagger{} quæ virgo fuit sápiens et de número prudéntum,~\grestar{} præsta nobis innocéntiam vitæ et sapiéntiam.

\Rbardot{} Iesu, coróna vírginum, audi nos.}
\newcommand{\benedictus}{\pars{Canticum Zachariæ.} \scriptura{\textbf{H121}}

\vspace{-4mm}

\antiphona{VII c}{temporalia/ant-agathalaetissime.gtex}

\vspace{-2mm}

\scriptura{Lc. 1, 68-79}

%\vspace{-2mm}

\initiumpsalmi{temporalia/benedictus-initium-vii-c-auto.gtex}

%\vspace{-1.5mm}

\input{temporalia/benedictus-vii-c.tex}

\vfill

\antiphona{}{temporalia/ant-agathalaetissime.gtex}}
\newcommand{\benedicamuslaudes}{\cuminitiali{}{temporalia/benedicamus-memoria-laudes.gtex}}
\newcommand{\hebdomada}{infra Hebdom. V post Pentecosten.}
\newcommand{\oratioLaudes}{\cuminitiali{}{temporalia/oratio5.gtex}}

% LuaLaTeX

\documentclass[a4paper, twoside, 12pt]{article}
\usepackage[latin]{babel}
%\usepackage[landscape, left=3cm, right=1.5cm, top=2cm, bottom=1cm]{geometry} % okraje stranky
%\usepackage[landscape, a4paper, mag=1166, truedimen, left=2cm, right=1.5cm, top=1.6cm, bottom=0.95cm]{geometry} % okraje stranky
\usepackage[landscape, a4paper, mag=1400, truedimen, left=0.5cm, right=0.5cm, top=0.5cm, bottom=0.5cm]{geometry} % okraje stranky

\usepackage{fontspec}
\setmainfont[FeatureFile={junicode.fea}, Ligatures={Common, TeX}, RawFeature=+fixi]{Junicode}
%\setmainfont{Junicode}

% shortcut for Junicode without ligatures (for the Czech texts)
\newfontfamily\nlfont[FeatureFile={junicode.fea}, Ligatures={Common, TeX}, RawFeature=+fixi]{Junicode}

\usepackage{multicol}
\usepackage{color}
\usepackage{lettrine}
\usepackage{fancyhdr}

% usual packages loading:
\usepackage{luatextra}
\usepackage{graphicx} % support the \includegraphics command and options
\usepackage{gregoriotex} % for gregorio score inclusion
\usepackage{gregoriosyms}
\usepackage{wrapfig} % figures wrapped by the text
\usepackage{parcolumns}
\usepackage[contents={},opacity=1,scale=1,color=black]{background}
\usepackage{tikzpagenodes}
\usepackage{calc}
\usepackage{longtable}
\usetikzlibrary{calc}

\setlength{\headheight}{14.5pt}

\input{conventuscommune.tex} % Often used macros

\newcommand{\annusEditionis}{2021}

%%%% Vicekrat opakovane kousky

\newcommand{\anteOrationem}{
  \rubrica{Ante Orationem, cantatur a Superiore:}

  \pars{Supplicatio Litaniæ.}

  \cuminitiali{}{temporalia/supplicatiolitaniae.gtex}

  \pars{Oratio Dominica.}

  \cuminitiali{}{temporalia/oratiodominica.gtex}

  \rubrica{Deinde dicitur ab Hebdomadario:}

  \cuminitiali{}{temporalia/dominusvobiscum-solemnis.gtex}

  \rubrica{In choro monialium loco Dominus vobiscum dicitur:}

  \sineinitiali{temporalia/domineexaudi.gtex}
}

\setlength{\columnsep}{30pt} % prostor mezi sloupci

%%%%%%%%%%%%%%%%%%%%%%%%%%%%%%%%%%%%%%%%%%%%%%%%%%%%%%%%%%%%%%%%%%%%%%%%%%%%%%%%%%%%%%%%%%%%%%%%%%%%%%%%%%%%%
\begin{document}

% Here we set the space around the initial.
% Please report to http://home.gna.org/gregorio/gregoriotex/details for more details and options
\grechangedim{afterinitialshift}{2.2mm}{scalable}
\grechangedim{beforeinitialshift}{2.2mm}{scalable}
\grechangedim{interwordspacetext}{0.22 cm plus 0.15 cm minus 0.05 cm}{scalable}%
\grechangedim{annotationraise}{-0.2cm}{scalable}

% Here we set the initial font. Change 38 if you want a bigger initial.
% Emit the initials in red.
\grechangestyle{initial}{\color{red}\fontsize{38}{38}\selectfont}

\pagestyle{empty}

%%%% Titulni stranka
\begin{titulusOfficii}
\ifx\titulus\undefined
\nomenFesti{Feria II \hebdomada{}}
\else
\titulus
\fi
\end{titulusOfficii}

\vfill

\begin{center}
%Ad usum et secundum consuetudines chori \guillemotright{}Conventus Choralis\guillemotleft.

%Editio Sancti Wolfgangi \annusEditionis
\end{center}

\scriptura{}

\pars{}

\pagebreak

\renewcommand{\headrulewidth}{0pt} % no horiz. rule at the header
\fancyhf{}
\pagestyle{fancy}

\cantusSineNeumas

\ifx\oratio\undefined
\ifx\laudb\undefined
\else
\newcommand{\oratio}{\pars{Oratio.}

\noindent Dómine Deus omnípotens, qui ad princípium huius diéi nos perveníre fecísti, tua nos hódie salva virtúte, ut in hac die ad nullum declinémus peccátum, sed semper ad tuam iustítiam faciéndam nostra procédant elóquia, dirigántur cogitatiónes et ópera.

\noindent Per Dóminum nostrum Iesum Christum, Fílium tuum, qui tecum vivit et regnat in unitáte Spíritus Sancti, Deus, per ómnia sǽcula sæculórum.

\noindent \Rbardot{} Amen.}
\fi
\fi

\hora{Ad Matutinum.} %%%%%%%%%%%%%%%%%%%%%%%%%%%%%%%%%%%%%%%%%%%%%%%%%%%%%
%\sideThumbs{Matutinum}

\vspace{2mm}

\cuminitiali{}{temporalia/dominelabiamea.gtex}

\vfill
%\pagebreak

\vspace{2mm}

\ifx\invitatorium\undefined
\pars{Invitatorium.} \scriptura{Ps. 94, 1; Psalmus 94; \textbf{H451}}

\vspace{-6mm}

\antiphona{VI}{temporalia/inv-jubilemusdeo.gtex}\else
\invitatorium
\fi

\vfill
\pagebreak

\ifx\hymnusmatutinum\undefined
\ifx\matua\undefined
\else
\pars{Hymnus.}

{
\grechangedim{interwordspacetext}{0.10 cm plus 0.15 cm minus 0.05 cm}{scalable}%
\antiphona{II}{temporalia/hym-IpsumNunc.gtex}
\grechangedim{interwordspacetext}{0.22 cm plus 0.15 cm minus 0.05 cm}{scalable}%
}
\fi
\else
\hymnusmatutinum
\fi

\vspace{-3mm}

\vfill
\pagebreak

\ifx\matub\undefined
\else
% MAT B
\pars{Psalmus 1.} \scriptura{Ps. 30, 2; \textbf{H90}}

\vspace{-4mm}

\antiphona{VIII G}{temporalia/ant-intuaiustitia.gtex}

%\vspace{-2mm}

\scriptura{Ps. 30, 2-9}

%\vspace{-2mm}

\initiumpsalmi{temporalia/ps30i-initium-viii-G-auto.gtex}

\vspace{-1.5mm}

\input{temporalia/ps30i-viii-G.tex} \Abardot{}

\vfill
\pagebreak

\pars{Psalmus 2.} \scriptura{Ps. 66, 2}

\vspace{-4mm}

\antiphona{E}{temporalia/ant-illuminadomine.gtex}

%\vspace{-2mm}

\scriptura{Ps. 30, 10-17}

%\vspace{-2mm}

\initiumpsalmi{temporalia/ps30ii-initium-e-a-auto.gtex}

\input{temporalia/ps30ii-e-a.tex} \Abardot{}

\vfill
\pagebreak

\pars{Psalmus 3.} \scriptura{Ps. 30, 24}

\vspace{-4mm}

\antiphona{II D}{temporalia/ant-diligitedominum.gtex}

%\vspace{-5mm}

\scriptura{Ps. 30, 20-25}

%\vspace{-2mm}

\initiumpsalmi{temporalia/ps30iii-initium-ii-D-auto.gtex}

\input{temporalia/ps30iii-ii-D.tex} \Abardot{}

\vfill
\pagebreak
\fi

\pars{Versus.}

\ifx\matversus\undefined
\ifx\matub\undefined
\else
\noindent \Vbardot{} Dírige me, Dómine, in veritáte tua, et doce me.

\noindent \Rbardot{} Quia tu es Deus salútis meæ.
\fi
\else
\matversus
\fi

\vspace{5mm}

\sineinitiali{temporalia/oratiodominica-mat.gtex}

\vspace{5mm}

\pars{Absolutio.}

\cuminitiali{}{temporalia/absolutio-exaudi.gtex}

\vfill
\pagebreak

\cuminitiali{}{temporalia/benedictio-solemn-benedictione.gtex}

\vspace{7mm}

\lectioi

\noindent \Vbardot{} Tu autem, Dómine, miserére nobis.
\noindent \Rbardot{} Deo grátias.

\vfill
\pagebreak

\responsoriumi

\vfill
\pagebreak

\cuminitiali{}{temporalia/benedictio-solemn-unigenitus.gtex}

\vspace{7mm}

\lectioii

\noindent \Vbardot{} Tu autem, Dómine, miserére nobis.
\noindent \Rbardot{} Deo grátias.

\vfill
\pagebreak

\responsoriumii

\vfill
\pagebreak

\cuminitiali{}{temporalia/benedictio-solemn-spiritus.gtex}

\vspace{7mm}

\lectioiii

\noindent \Vbardot{} Tu autem, Dómine, miserére nobis.
\noindent \Rbardot{} Deo grátias.

\vfill
\pagebreak

\responsoriumiii

\vfill
\pagebreak

\rubrica{Reliqua omittuntur, nisi Laudes separandæ sint.}

\sineinitiali{temporalia/domineexaudi.gtex}

\vfill

\oratio

\vfill

\noindent \Vbardot{} Dómine, exáudi oratiónem meam.
\Rbardot{} Et clamor meus ad te véniat.

\vfill

\noindent \Vbardot{} Benedicámus Dómino.
\noindent \Rbardot{} Deo grátias.

\vfill

\noindent \Vbardot{} Fidélium ánimæ per misericórdiam Dei requiéscant in pace.
\Rbardot{} Amen.

\vfill
\pagebreak

\hora{Ad Laudes.} %%%%%%%%%%%%%%%%%%%%%%%%%%%%%%%%%%%%%%%%%%%%%%%%%%%%%
%\sideThumbs{Laudes}

\cantusSineNeumas

\vspace{0.5cm}
\grechangedim{interwordspacetext}{0.18 cm plus 0.15 cm minus 0.05 cm}{scalable}%
\cuminitiali{}{temporalia/deusinadiutorium-communis.gtex}
\grechangedim{interwordspacetext}{0.22 cm plus 0.15 cm minus 0.05 cm}{scalable}%

\vfill
\pagebreak

\ifx\hymnuslaudes\undefined
\ifx\laudbd\undefined
\else
\pars{Hymnus} \scriptura{Hilarius (\olddag{} 367)}

\grechangedim{interwordspacetext}{0.16 cm plus 0.15 cm minus 0.05 cm}{scalable}%
\cuminitiali{IV}{temporalia/hym-LucisLargitor.gtex}
\grechangedim{interwordspacetext}{0.22 cm plus 0.15 cm minus 0.05 cm}{scalable}%
\vspace{-3mm}
\fi
\else
\hymnuslaudes
\fi

\vfill
\pagebreak

\ifx\laudb\undefined
\else
\pars{Psalmus 1.} \scriptura{Ps. 41, 3; \textbf{H391}}

\vspace{-4mm}

\antiphona{II D}{temporalia/ant-sitivitanima.gtex}

%\vspace{-2mm}

\scriptura{Psalmus 41}

%\vspace{-2mm}

\initiumpsalmi{temporalia/ps41-initium-ii-D-auto.gtex}

%\vspace{-1.5mm}

\input{temporalia/ps41-ii-D.tex}

\vfill

\antiphona{}{temporalia/ant-sitivitanima.gtex}

\vfill
\pagebreak

\pars{Psalmus 2.}

\vspace{-4mm}

\antiphona{III a}{temporalia/ant-ostendenobisdomine.gtex}

%\vspace{-2mm}

\scriptura{Canticum Ecclesiastici, Sir. 36, 1-7.13-16}

%\vspace{-3mm}

\initiumpsalmi{temporalia/ecclesiastici-initium-iii-a-auto.gtex}

\input{temporalia/ecclesiastici-iii-a.tex} \Abardot{}

\vfill
\pagebreak

\pars{Psalmus 3.}

\vspace{-4mm}

\antiphona{II D}{temporalia/ant-operamanuumeius.gtex}

\scriptura{Psalmus 18, 1-7}

\initiumpsalmi{temporalia/ps18i-initium-ii-D-auto.gtex}

\input{temporalia/ps18i-ii-D.tex} \Abardot{}

\vfill
\pagebreak
\fi

\ifx\lectiobrevis\undefined
\ifx\laudb\undefined
\else
\pars{Lectio Brevis.} \scriptura{Ier. 15, 16}

\noindent Invénti sunt sermónes tui, et comédi eos, et factum est mihi verbum tuum in gáudium et in lætítiam cordis mei, quóniam invocátum est nomen tuum super me, Dómine Deus exercítuum.
\fi
\else
\lectiobrevis
\fi

\vfill

\ifx\responsoriumbreve\undefined
\ifx\laudbd\undefined
\else
\pars{Responsorium breve.} \scriptura{Ps. 32, 1.3}

\cuminitiali{VI}{temporalia/resp-exsultateiusti.gtex}
\fi
\else
\responsoriumbreve
\fi

\vfill
\pagebreak

\ifx\benedictus\undefined
\ifx\laudbd\undefined
\else
\pars{Canticum Zachariæ.} \scriptura{Lc. 1, 68; \textbf{H422}}

\vspace{-4mm}

{
\grechangedim{interwordspacetext}{0.18 cm plus 0.15 cm minus 0.05 cm}{scalable}%
\antiphona{IV E}{temporalia/ant-benedictusdominus.gtex}
\grechangedim{interwordspacetext}{0.22 cm plus 0.15 cm minus 0.05 cm}{scalable}%
}

%\vspace{-3mm}

\scriptura{Lc. 1, 68-79}

%\vspace{-2mm}

\cantusSineNeumas
\initiumpsalmi{temporalia/benedictus-initium-iv-E-auto.gtex}

%\vspace{-1.5mm}

\input{temporalia/benedictus-iv-E.tex} \Abardot{}
\fi
\else
\benedictus
\fi

\vspace{-1cm}

\vfill
\pagebreak

%\sideThumbs{{\scriptsize{}Fine horarum}}

\pars{Preces.}

\sineinitiali{}{temporalia/tonusprecum.gtex}

\ifx\preces\undefined
\ifx\laudb\undefined
\else
\noindent Salvátor noster fecit nos regnum et sacerdótium, ut hóstias Deo acceptábiles offerámus. \gredagger{} Grati ígitur eum invocémus:

\Rbardot{} Serva nos in tuo ministério, Dómine.

\noindent Christe, sacérdos ætérne, qui sanctum pópulo tuo sacerdótium concessísti, \gredagger{} concéde, ut spiritáles hóstias Deo acceptábiles iúgiter offerámus.

\Rbardot{} Serva nos in tuo ministério, Dómine.

\noindent Spíritus tui fructus nobis largíre propítius, \gredagger{} patiéntiam, benignitátem et mansuetúdinem.

\Rbardot{} Serva nos in tuo ministério, Dómine.

\noindent Da nobis te amáre, ut te, qui es cáritas, possideámus, \gredagger{} et bene ágere, ut per vitam étiam nostram te laudémus.

\Rbardot{} Serva nos in tuo ministério, Dómine.

\noindent Quæ frátribus nostris sunt utília, nos quǽrere concéde, \gredagger{} ut salútem facílius consequántur.

\Rbardot{} Serva nos in tuo ministério, Dómine.
\fi
\else
\preces
\fi

\vfill

\pars{Oratio Dominica.}

\cuminitiali{}{temporalia/oratiodominicaalt.gtex}

\vfill
\pagebreak

\rubrica{vel:}

\pars{Supplicatio Litaniæ.}

\cuminitiali{}{temporalia/supplicatiolitaniae.gtex}

\vfill

\pars{Oratio Dominica.}

\cuminitiali{}{temporalia/oratiodominica.gtex}

\vfill
\pagebreak

% Oratio. %%%
\oratio

\vspace{-1mm}

\vfill

\rubrica{Hebdomadarius dicit Dominus vobiscum, vel, absente sacerdote vel diacono, sic concluditur:}

\vspace{2mm}

\antiphona{C}{temporalia/dominusnosbenedicat.gtex}

\rubrica{Postea cantatur a cantore:}

\vspace{2mm}

\cuminitiali{IV}{temporalia/benedicamus-feria-laudes.gtex}

\vspace{1mm}

\vfill
\pagebreak

\end{document}

