\newcommand{\titulus}{\nomenFesti{S. Irenæi, Episcopi, Martyris.}
\dies{Die 28. Iunii.}}
\newcommand{\oratio}{\pars{Oratio.}

\noindent Deus, qui beáto Irenǽo, epíscopo, tribuísti, ut veritátem doctrínæ pacémque Ecclésiæ felíciter confirmáret, concéde, quǽsumus, eius intercessióne, ut nos, fide et caritáte renováti, ad unitátem concordiámque fovéndam semper simus inténti.

\pars{Pro pace in Ucraina.} \scriptura{Sir. 50, 25; 2 Esdr. 4, 20; \textbf{H416}}

\vspace{-4mm}

\antiphona{II D}{temporalia/ant-dapacemdomine.gtex}

\vfill

\noindent Deus, a quo sancta desidéria, recta consília et iusta sunt ópera: da servis tuis illam, quam mundus dare non potest, pacem; ut et corda nostra mandátis tuis dédita, et hóstium subláta formídine, témpora sint tua protectióne tranquílla.

\noindent Per Dóminum nostrum Iesum Christum, Fílium tuum, qui tecum vivit et regnat in unitáte Spíritus Sancti, Deus, per ómnia sǽcula sæculórum.

\noindent \Rbardot{} Amen.}
\newcommand{\invitatorium}{\pars{Invitatorium.}

\vspace{-4mm}

\antiphona{E}{temporalia/inv-regemmartyrumsimplex.gtex}}
\newcommand{\hymnusmatutinum}{\pars{Hymnus}

\cuminitiali{I}{temporalia/hym-BeateMartyr.gtex}}
\newcommand{\matversus}{\noindent \Vbardot{} Iustum dedúxit Dóminus per vias rectas.

\noindent \Rbardot{} Et osténdit illi regnum Dei.}
\newcommand{\lectioi}{\pars{Lectio I.} \scriptura{1 Sam. 19, 8-10; 20, 1-3.11-17}

\noindent De libro primo Samuélis.

\noindent In diébus illis: Motum est rursum bellum, et egréssus David pugnávit advérsum Philísthim percussítque eos plaga magna; et fugérunt a fácie eius. Et factus est spíritus Dómini malus in Saul; sedébat autem in domo sua et tenébat lánceam, porro David psallébat in manu sua. Nisúsque est Saul confígere láncea David in paríete; et declinávit David a fácie Saul, láncea autem, casso vúlnere, perláta est in paríetem. Et David fugit et salvátus est nocte illa.

\noindent Fugit autem David de Náioth, quæ est in Rama, veniénsque locútus est coram Iónathan: «Quid feci? Quæ est iníquitas mea et quod peccátum meum in patrem tuum, quia quærit ánimam meam?». Qui dixit ei: «Absit, non moriéris; neque enim fáciet pater meus quidquam grande vel parvum, nisi prius indicáverit mihi: hoc ergo celávit me pater meus tantúmmodo? Nequáquam erit istud». Et rursum respóndit David et ait: «Scit profécto pater tuus quia invéni grátiam in óculis tuis, et dixit: “Nésciat hoc Iónathan, ne forte tristétur”. Quinímmo vivit Dóminus et vivit ánima tua quia uno tantum gradu ego morsque divídimur».

\noindent {\color{gray} Et ait Iónathan ad David: «Quid desíderat ánima tua, ut fáciam tibi?». Dixit autem David ad Iónathan: «Ecce neoménia est crástino, et ego ex more sedére sóleo iuxta regem ad vescéndum; dimítte ergo me, ut abscóndar in agro usque ad vésperam diéi tértiæ. Si requisíerit me pater tuus, respondébis eis: “Rogávit me David, ut iret celériter in Béthlehem civitátem suam, quia víctimæ ánnuæ ibi sunt univérsis contribúlibus eius”. Si díxerit: “Bene”, pax erit servo tuo; si autem fúerit irátus, scito quia malum decrétum est ab eo. Fac ergo misericórdiam in servum tuum, quia fœdus Dómini me fámulum tuum tecum iníre fecísti; si autem est in me áliqua iníquitas, tu me intérfice et ad patrem tuum ne introdúcas me». Et ait Iónathan: «Absit hoc a te; neque enim fíeri potest ut, si certo cognóvero malum decrétum esse a patre meo contra te, non annúntiem tibi». Respondítque David ad Iónathan: «Quis nuntiábit mihi, si quid forte respónderit tibi pater tuus dure?».}

\noindent Et ait Iónathan ad David: «Veni, egrediámur foras in agrum». Cumque exíssent ambo in agrum, ait Iónathan ad David: «Vivit Dóminus, Deus Israel, investigábo senténtiam patris mei hoc fere témpore cras vel peréndie; et si áliquid boni fúerit super David, et non statim míserim ad te et notum tibi fécerim, hæc fáciat Dóminus in Iónathan et hæc áugeat! Si autem perseveráverit patris mei malítia advérsum te, hoc quoque notum fáciam tibi et dimíttam te, ut vadas in pace.

\noindent Ait Iónathan ad David: Sit Dóminus tecum, sicut fuit cum patre meo. Et, si víxero, fácies mihi misericórdiam Dómini; si vero mórtuus fúero, non áuferas misericórdiam tuam a domo mea usque in sempitérnum, quando eradicáverit Dóminus inimícos David unumquémque de terra». Pépigit ergo fœdus Iónathan cum domo David dicens: «Requírat Dóminus de manu inimicórum David!». Et áddidit Iónathan ut fáceret David iuráre per dilectiónem suam erga illum; sicut ánimam enim suam, ita diligébat eum.}
\newcommand{\responsoriumi}{\pars{Responsorium 1.} \scriptura{\Rbardot{} 1 Reg. 17, 37 \Vbardot{} Ps. 56, 4.5; \textbf{H395}}

\vspace{-5mm}

\responsorium{VIII}{temporalia/resp-dominusquieripuitme-sinedox.gtex}{}

\vfill

\rubrica{vel ad libitum:}

\vspace{3mm}

\pars{Responsorium 1.} \scriptura{\Rbardot{} 1 Reg. 7, 3 \Vbardot{} ibidem; \textbf{H397}}

\vspace{-5mm}

\responsorium{III}{temporalia/resp-praeparatecordavestra-sinedox.gtex}{}}
\newcommand{\lectioii}{\pars{Lectio II.} \scriptura{Lib. 4, 20, 5-7: SCh 100, 640-642. 644-648}

\noindent Ex Tractátu sancti Irenǽi epíscopi Advérsus hǽreses.

\noindent Vivíficat autem Dei cláritas: percípiunt ergo vitam qui vident Deum. Et propter hoc incapábilis et incomprehensíbilis et invisíbilis visíbilem se et comprehensíbilem et capácem homínibus præstat, ut vivíficet percipiéntes et vidéntes se. Quóniam vívere sine vita impossíbile est, subsisténtia autem vitæ de Dei participatióne évenit, participátio autem Dei est vidére Deum et frui benignitáte eius.

\noindent Hómines ígitur vidébunt Deum ut vivant, per visiónem immortáles facti et pertingéntes usque in Deum. Quod, sicut prædíxi, per prophétas figuráliter manifestabátur quóniam vidébitur Deus ab homínibus qui portant Spíritum eius et semper advéntum eius sústinent. Quemádmodum et in Deuteronómio Móyses ait: \emph{In die ista vidébimus, quóniam loquétur Deus ad hóminem, et vivet.}

\noindent Qui ómnia in ómnibus operátur qualis et quantus est, invisíbilis et inenarrábilis est ómnibus quæ ab eo facta sunt, incógnitus autem nequáquam: ómnia enim per Verbum eius discunt quia est unus Deus Pater, qui cóntinet ómnia et ómnibus esse præstat, quemádmodum in Evangélio scriptum est: \emph{Deum nemo vidit umquam, nisi unigénitus Fílius, qui est in sinu Patris, ipse enarrávit.}}
\newcommand{\responsoriumii}{\pars{Responsorium 2.} \scriptura{\textbf{H373}}

\vspace{-5mm}

\responsorium{III}{temporalia/resp-istecognovit-CROCHU.gtex}{}}
\newcommand{\lectioiii}{\pars{Lectio III.}

\noindent Enarrátor ergo ab inítio Fílius Patris, quippe qui ab inítio est cum Patre, qui et visiónes prophéticas et divisiónes charísmatum et ministéria sua et Patris glorificatiónem consequénter et compósite osténderit humáno géneri apto témpore ad utilitátem: ubi est enim consequéntia, illic et consonántia; et ubi consonántia, illic et pro témpore; et ubi pro témpore, illic et utílitas.

\noindent Et proptérea Verbum dispensátor patérnæ grátiæ factus est ad utilitátem hóminum, propter quos fecit tantas dispositiónes, homínibus quidem osténdens Deum, Deo autem éxhibens hóminem; et invisibilitátem quidem Patris custódiens, ne quando homo contémptor fíeret Dei et ut semper habéret ad quod profíceret, visíbilem autem rursus homínibus per multas dispositiónes osténdens Deum, ne in totum defíciens a Deo homo cessáret esse: glória enim Dei vivens homo, vita autem hóminis vísio Dei. Si enim quæ est per condiciónem osténsio Dei vitam præstat ómnibus in terra vivéntibus, multo magis ea quæ est per Verbum manifestátio Patris vitam præstat his qui vident Deum.}
\newcommand{\responsoriumiii}{\pars{Responsorium 3.} \scriptura{\Rbardot{} Ps. 8, 6-8 \Vbardot{} Ps. 20, 4; \textbf{H374}}

\vspace{-5mm}

\responsorium{VII}{temporalia/resp-gloriaethonore-CROCHU-cumdox.gtex}{}}
\newcommand{\hymnuslaudes}{\pars{Hymnus}

\cuminitiali{VI}{temporalia/hym-MartyrDei.gtex}}
\newcommand{\lectiobrevis}{\pars{Lectio Brevis.} \scriptura{2 Cor. 1, 3-5}

\noindent Benedíctus Deus et Pater Dómini nostri Iesu Christi, Pater misericordiárum et Deus totíus consolatiónis, qui consolátur nos in omni tribulatióne nostra, ut possímus et ipsi consolári eos, qui in omni pressúra sunt, per exhortatiónem, qua exhortámur et ipsi a Deo; quóniam, sicut abúndant passiónes Christi in nobis, ita per Christum abúndat et consolátio nostra.}
\newcommand{\responsoriumbreve}{\pars{Responsorium breve.} \scriptura{Ex. 15, 2}

\antiphona{VI}{temporalia/resp-fortitudomea.gtex}}
\newcommand{\preces}{\noindent Fratres, Salvatórem nostrum, testem fidélem, \gredagger{} per mártyres interféctos propter verbum Dei, \grestar{} celebrémus, clamántes:

\Rbardot{} Redemísti nos Deo in sánguine tuo.

\noindent Per mártyres tuos, qui líbere mortem in testimónium fídei sunt ampléxi, \grestar{} da nobis, Dómine, veram spíritus libertátem.

\Rbardot{} Redemísti nos Deo in sánguine tuo.

\noindent Per mártyres tuos, qui fidem usque ad sánguinem sunt conféssi, \grestar{} da nobis, Dómine, puritátem fideíque constántiam.

\Rbardot{} Redemísti nos Deo in sánguine tuo.

\noindent Per mártyres tuos, qui, sustinéntes crucem, \gredagger{} tua vestígia sunt secúti, \grestar{} da nobis, Dómine, ærúmnas vitæ fórtiter sustinére.

\Rbardot{} Redemísti nos Deo in sánguine tuo.

\noindent Per mártyres tuos, qui stolas suas lavérunt in sánguine Agni, \grestar{} da nobis, Dómine, omnes insídias carnis mundíque devíncere.

\Rbardot{} Redemísti nos Deo in sánguine tuo.}
\newcommand{\benedictus}{\pars{Canticum Zachariæ.} \scriptura{Lc. 12, 8; \textbf{H374}}

\vspace{-4mm}

\antiphona{I f}{temporalia/ant-quimeconfessusfuerit.gtex}

\vspace{-2mm}

\scriptura{Lc. 1, 68-79}

\vspace{-2mm}

\cantusSineNeumas
\initiumpsalmi{temporalia/benedictus-initium-i-f-auto.gtex}

%\vspace{-1.5mm}

\input{temporalia/benedictus-i-f.tex} \Abardot{}}
\newcommand{\benedicamuslaudes}{\cuminitiali{}{temporalia/benedicamus-memoria-laudes.gtex}}
\newcommand{\hebdomada}{infra Hebdom. XII post Pentecosten.}
\newcommand{\oratioLaudes}{\cuminitiali{}{temporalia/oratio12.gtex}}

% LuaLaTeX

\documentclass[a4paper, twoside, 12pt]{article}
\usepackage[latin]{babel} 
%\usepackage[landscape, left=3cm, right=1.5cm, top=2cm, bottom=1cm]{geometry} % okraje stranky
%\usepackage[landscape, a4paper, mag=1166, truedimen, left=2cm, right=1.5cm, top=1.6cm, bottom=0.95cm]{geometry} % okraje stranky
\usepackage[landscape, a4paper, mag=1400, truedimen, left=0.5cm, right=0.5cm, top=0.5cm, bottom=0.5cm]{geometry} % okraje stranky

\usepackage{fontspec}
\setmainfont[FeatureFile={junicode.fea}, Ligatures={Common, TeX}, RawFeature=+fixi]{Junicode}
%\setmainfont{Junicode}

% shortcut for Junicode without ligatures (for the Czech texts)
\newfontfamily\nlfont[FeatureFile={junicode.fea}, Ligatures={Common, TeX}, RawFeature=+fixi]{Junicode}

% Hebrew font:
% http://scripts.sil.org/cms/scripts/page.php?site_id=nrsi&id=SILHebrUnic2
\newfontfamily\hebfont[Scale=1]{Ezra SIL}

\usepackage{multicol}
\usepackage{color}
\usepackage{lettrine}
\usepackage{fancyhdr}

% usual packages loading:
\usepackage{luatextra}
\usepackage{graphicx} % support the \includegraphics command and options
\usepackage{gregoriotex} % for gregorio score inclusion
\usepackage{gregoriosyms}
\usepackage{wrapfig} % figures wrapped by the text
\usepackage{parcolumns}
\usepackage[contents={},opacity=1,scale=1,color=black]{background}
\usepackage{tikzpagenodes}
\usepackage{calc}
\usepackage{longtable}
\usetikzlibrary{calc}

\setlength{\headheight}{14.5pt}

\input{conventuscommune.tex} % Often used macros
%%%% Preklady jednotlivych zpevu (nektere se opakuji, a je dobre mit je
% vsechny na jedne hromade)

% HOURS ---

\newcommand{\trAntI}{\translatioCantus{Muž boží měl kožený toulec, pečlivě
zavázaný, jenž mu visel na šíji a~často se ho dotýkal.}}

\newcommand{\trAntII}{\translatioCantus{Klíč od~něho tak dobře střežil, že
dokud žil v~těle, nikdo z~jeho žáků nezvěděl, co je uvnitř.}}

\newcommand{\trAntIII}{\translatioCantus{Ale když se odebral z~tohoto
života, schránku otevřeli a~objevili v~ní žíněné roucho a~měděný řetěz
potřísněný krví.}}

\newcommand{\trAntIV}{\translatioCantus{A když prohlédli mistrovo tělo,
nalezli jeho tělo na čtyřech místech hluboce zbrázděno ranami od řetězu.}}

\newcommand{\trAntV}{\translatioCantus{Krev vytékající z~těch ran, místy
prostoupila i~žíněným rouchem.}}

\newcommand{\trCapituli}{\translatioCantus{
Miláčkovi Boha a~lidí,
Mojžíšovi požehnané paměti,~\gredagger{}
dopřál slávu rovnou slávě svatých~\grestar{}
učinil ho mocným na postrach nepřátelům
a~jeho slovy zastavil divy.}}

\newcommand{\trLectioBrevis}{\translatioCantus{
Pamatujte na své představené,
kteří vám hlásali Boží slovo.
Uvažte, jak oni skončili život, a~napodobujte jejich víru.
Ježíš Kristus je stejný včera i~dnes i~navěky.
Nenechte se svést věelijakými cizími naukami.}}

\newcommand{\trRespLaud}{\translatioCantus{Spravedlivého vodil Hospodin~\grestar{}
po přímých stezkách. \Vbardot{} A~ukázal mu Boží království.}}

\newcommand{\trRespLaudB}{\translatioCantus{Na tvých hradbách, Jeruzaléme,
ustanovil jsem strážné;~\grestar{}
budou bdít nad mým lidem. \Vbardot{} Ani ve dne, ani v~noci nesmějí nikdy
mlčet.}}

\newcommand{\trVersus}{\translatioCantus{\Vbardot{} Ústa spravedlivého šeptají moudrost, aleluja.
\Rbardot{} A~jeho jazyk ohlašuje právo, aleluja.}}

\newcommand{\trAntBenedictus}{\translatioCantus{Když na bujné oře vložili
nosítka a~sňali jim uzdu, vydali se přímo k~cele božího muže.}}

\newcommand{\trPreces}{\translatioCantus{
\noindent S vděčností chvalme Krista, dobrého Pastýře, \gredagger{} který dal život za své ovce, \grestar{} a~pokorně ho prosme: \Rbardot{} Pane, buď pastýřem svého lidu.

\noindent Kriste, ty dáváš církvi pastýře, a~jejich službou se ujímáš svého lidu, \grestar{} dej, ať v~lásce těch, kteří nás vedou, poznáváme, jak nás miluješ. \Rbardot{} Pane, buď pastýřem svého lidu.

\noindent Ty stále konáš skrze své zástupce službu pastýře a~učitele, \grestar{} nepřestávej nás nikdy vést prostřednictvím svých služebníků. \Rbardot{} Pane, buď pastýřem svého lidu.

\noindent Ty prokazuješ svému lidu skrze jeho pastýře službu lékaře duše i~těla, \grestar{} ochraňuj náš život a~veď nás ke svatosti. \Rbardot{} Pane, buď pastýřem svého lidu.

\noindent Ty posíláš své svaté, aby slovem i~příkladem vedli tvůj lid k~tobě, \grestar{} na jejich přímluvu nás posiluj, abychom vytrvali na cestě, která vede k~věčnému životu. \Rbardot{} Pane, buď pastýřem svého lidu.}}

\newcommand{\trOrationis}{\translatioCantus{Bože, jenž nám dopřáváš radovat
se z~výroční slavnosti svatého tvého vyznavače Havla, uděl dobrotivě,
abychom když slavíme jeho narození, též se řídili podobou jeho skutků.
Skrze…}}
 % Czech translations of the proper texts

\newcommand{\annusEditionis}{2020}

\def\hebinitial#1{%
\leavevmode{\newbox\hebbox\setbox\hebbox\hbox{\hebfont{#1}\hskip 1mm}\kern -\wd\hebbox\hbox{\hebfont{#1}\hskip 1mm}}%
}

%%%% Vicekrat opakovane kousky

\newcommand{\anteOrationem}{
  \rubrica{Ante Orationem, cantatur a Superiore:}

  \pars{Supplicatio Litaniæ.}

  \cuminitiali{}{temporalia/supplicatiolitaniae.gtex}

  \pars{Oratio Dominica.}

  \cuminitiali{}{temporalia/oratiodominica.gtex}

  \rubrica{Deinde dicitur ab Hebdomadario:}

  \cuminitiali{}{temporalia/dominusvobiscum-solemnis.gtex}

  \rubrica{In choro monialium loco Dominus vobiscum dicitur:}

  \sineinitiali{temporalia/domineexaudi.gtex}
}

\setlength{\columnsep}{30pt} % prostor mezi sloupci

%%%%%%%%%%%%%%%%%%%%%%%%%%%%%%%%%%%%%%%%%%%%%%%%%%%%%%%%%%%%%%%%%%%%%%%%%%%%%%%%%%%%%%%%%%%%%%%%%%%%%%%%%%%%%
\begin{document}

% Here we set the space around the initial.
% Please report to http://home.gna.org/gregorio/gregoriotex/details for more details and options
\grechangedim{afterinitialshift}{2.2mm}{scalable}
\grechangedim{beforeinitialshift}{2.2mm}{scalable}

\grechangedim{interwordspacetext}{0.32 cm plus 0.15 cm minus 0.05 cm}{scalable}%
\grechangedim{annotationraise}{-0.2cm}{scalable}

% Here we set the initial font. Change 38 if you want a bigger initial.
% Emit the initials in red.
\grechangestyle{initial}{\color{red}\fontsize{38}{38}\selectfont}

\pagestyle{empty}

%%%% Titulni stranka
\begin{titulusOfficii}
\nomenFesti{Feria IV \hebdomada{}}
\end{titulusOfficii}

\pagebreak

% graphic
\renewcommand{\headrulewidth}{0pt} % no horiz. rule at the header
\fancyhf{}
\pagestyle{fancy}

\cantusSineNeumas

\hora{Ad Matutinum.}

\vspace{2mm}

\cuminitiali{}{temporalia/dominelabiamea.gtex}

\vspace{2mm}

\pars{Invitatorium.} \scriptura{Lc. 24, 34; Psalmus 94; \textbf{H232}}

\vspace{-6mm}

\antiphona{VI}{temporalia/inv-surrexitdominusvere.gtex}

\vfill
\pagebreak

\pars{Hymnus.}

\vspace{-5mm}

\scriptura{\textbf{AR454}}

{
\grechangedim{interwordspacetext}{0.30 cm plus 0.15 cm minus 0.05 cm}{scalable}%
\antiphona{IV}{temporalia/hym-RexSempiterne.gtex}
\grechangedim{interwordspacetext}{0.32 cm plus 0.15 cm minus 0.05 cm}{scalable}%
}
%{
%\vspace{-5mm}
%\setlength{\columnsep}{0pt} % prostor mezi sloupci
%\input{hym-RexSempiterne-bohtext.tex}
%\setlength{\columnsep}{30pt} % prostor mezi sloupci
%}

\vfill
\pagebreak

\pars{Psalmus 1.}

%\vspace{-5mm}

\antiphona{I g}{temporalia/ant-alleluia-fiv-matutinum.gtex}

%\vspace{-5mm}

\scriptura{Ps. 44, 2-10}

%\vspace{-2mm}

\initiumpsalmi{temporalia/ps44i-initium-i-g-auto.gtex}

%\psalmusEtTranslatioT{temporalia/ps44i-III-comb.tex}{10cm}

\input{temporalia/ps44i-III.tex}

\vfill
\pagebreak

\pars{Psalmus 2.} \scriptura{Ps. 44, 11-18}

%\vspace{-2mm}

\initiumpsalmi{temporalia/ps44ii-initium-i-g-auto.gtex}

%\psalmusEtTranslatioT{temporalia/ps44i-III-comb.tex}{10cm}

\input{temporalia/ps44ii-III.tex}

\vfill
\pagebreak

\pars{Psalmus 3.} \scriptura{Ps. 45}

%\vspace{-2mm}

\initiumpsalmi{temporalia/ps45-initium-i-g-auto.gtex}

%\psalmusEtTranslatioT{temporalia/ps45-III-comb.tex}{10cm}

\input{temporalia/ps45-III.tex}

\vfill
\pagebreak

\pars{Psalmus 4.} \scriptura{Ps. 47}

%\vspace{-2mm}

\initiumpsalmi{temporalia/ps47-initium-i-g-auto.gtex}

%\psalmusEtTranslatioT{temporalia/ps47-III-comb.tex}{10cm}

\input{temporalia/ps47-III.tex}

\vfill
\pagebreak

\pars{Psalmus 5.} \scriptura{Ps. 48, 2-13}

%\vspace{-2mm}

\initiumpsalmi{temporalia/ps48i-initium-i-g-auto.gtex}

%\psalmusEtTranslatioT{temporalia/ps48i-III-comb.tex}{10cm}

\input{temporalia/ps48i-III.tex}

\vfill
\pagebreak

\pars{Psalmus 6.} \scriptura{Ps. 48, 14-21}

%\vspace{-2mm}

\initiumpsalmi{temporalia/ps48ii-initium-i-g-auto.gtex}

%\psalmusEtTranslatioT{temporalia/ps48ii-III-comb.tex}{10cm}

\input{temporalia/ps48ii-III.tex}

\vfill
\pagebreak

\pars{Psalmus 7.} \scriptura{Ps. 49, 1-15}

%\vspace{-2mm}

\initiumpsalmi{temporalia/ps49i-initium-i-g-auto.gtex}

%\psalmusEtTranslatioT{temporalia/ps49i-III-comb.tex}{10cm}

\input{temporalia/ps49i-III.tex}

\vfill
\pagebreak

\pars{Psalmus 8.} \scriptura{Ps. 49, 16-23}

%\vspace{-2mm}

\initiumpsalmi{temporalia/ps49ii-initium-i-g-auto.gtex}

%\psalmusEtTranslatioT{temporalia/ps49ii-III-comb.tex}{10cm}

\input{temporalia/ps49ii-III.tex}

\vfill
\pagebreak

\pars{Psalmus 9.} \scriptura{Ps. 50}

%\vspace{-2mm}

\initiumpsalmi{temporalia/ps50-initium-i-g-auto.gtex}

%\psalmusEtTranslatioT{temporalia/ps50-VI-comb.tex}{10cm}

\input{temporalia/ps50-VI.tex}

\vfill
%\pagebreak

\antiphona{}{temporalia/ant-alleluia-fiv-matutinum.gtex}

\vfill
\pagebreak

\noindent \Vbardot{} Gavísi sunt discípuli, allelúia.
\noindent \Rbardot{} Viso Dómino, allelúia.

\noindent Pater noster.

\pars{Absolutio.}

\cuminitiali{}{temporalia/absolutio-avinculis.gtex}

\vfill
\pagebreak

\ifx\magnificat\undefined
\cuminitiali{}{temporalia/benedictio-solemn-evangelica.gtex}
\else
\cuminitiali{}{temporalia/benedictio-solemn-ille.gtex}
\fi

\vspace{7mm}

\lectioi

\noindent \Vbardot{} Tu autem, Dómine, miserére nobis.
\noindent \Rbardot{} Deo grátias.

\vfill
\pagebreak

\responsoriumi

\vfill
\pagebreak

\cuminitiali{}{temporalia/benedictio-solemn-divinum.gtex}

\vspace{7mm}

\lectioii

\noindent \Vbardot{} Tu autem, Dómine, miserére nobis.
\noindent \Rbardot{} Deo grátias.

\vfill
\pagebreak

\responsoriumii

\vfill
\pagebreak

\ifx\magnificat\undefined
\cuminitiali{}{temporalia/benedictio-solemn-adsocietatem.gtex}
\else
\cuminitiali{}{temporalia/benedictio-solemn-ignem.gtex}
\fi

\vspace{7mm}

\lectioiii

\noindent \Vbardot{} Tu autem, Dómine, miserére nobis.
\noindent \Rbardot{} Deo grátias.

\vfill
\pagebreak

% Te Deum

%\pars{Hymnus Ambrosianus}

\vspace{-5mm}

{
\grechangedim{interwordspacetext}{0.22 cm plus 0.15 cm minus 0.05 cm}{scalable}%
\cuminitiali{III}{temporalia/tedeum-solemnis.gtex}
\grechangedim{interwordspacetext}{0.32 cm plus 0.15 cm minus 0.05 cm}{scalable}%
}

\vfill
\pagebreak

\rubrica{Reliqua omittuntur, nisi Laudes separandæ sint.}

\pars{Oratio}

\noindent \Vbardot{} Dómine, exáudi oratiónem meam.

\noindent \Rbardot{} Et clamor meus ad te véniat.

Orémus:

\oratioMatutinum

\noindent \Rbardot{} Amen.

\vspace{7mm}

\pars{Conclusio}

\noindent \Vbardot{} Dómine, exáudi oratiónem meam.

\noindent \Rbardot{} Et clamor meus ad te véniat.

\noindent \Vbardot{} Benedicámus Dómino, allelúia, allelúia.

\noindent \Rbardot{} Deo grátias, allelúia, allelúia.

\noindent \Vbardot{} Fidélium ánimæ per misericórdiam Dei requiéscant in pace.

\noindent \Rbardot{} Amen.

\vfill
\pagebreak

\hora{Ad Laudes.} %%%%%%%%%%%%%%%%%%%%%%%%%%%%%%%%%%%%%%%%%%%%%%%%%%%%%
%\sideThumbs{Laudes}

\cantusSineNeumas

\vspace{0.5cm}
\grechangedim{interwordspacetext}{0.18 cm plus 0.15 cm minus 0.05 cm}{scalable}%
\cuminitiali{}{temporalia/deusinadiutorium-communis.gtex}
\grechangedim{interwordspacetext}{0.32 cm plus 0.15 cm minus 0.05 cm}{scalable}%

\vfill
%\pagebreak

\pars{Psalmus 1.}

\vspace{-0.4cm}

\antiphona{VII a}{temporalia/ant-alleluia-fiv-laudes-1.gtex}

\scriptura{Psalmus 50.}

\initiumpsalmi{temporalia/ps50-initium-vii-a-auto.gtex}

%\psalmusEtTranslatioT{temporalia/ps50-III-comb.tex}{10cm}
\input{temporalia/ps50-III.tex}

\vspace{-1cm}

\vfill
\pagebreak

\pars{Psalmus 2.} \scriptura{Psalmus 63.}

\initiumpsalmi{temporalia/ps63-initium-vii-a-auto.gtex}

%\psalmusEtTranslatioT{temporalia/ps63-III-comb.tex}{10cm}
\input{temporalia/ps63-III.tex}

\vfill
\pagebreak

\pars{Psalmus 3.} \scriptura{Psalmus 64.}

\initiumpsalmi{temporalia/ps64-initium-vii-a-auto.gtex}

%\psalmusEtTranslatioT{temporalia/ps64-III-comb.tex}{10cm}
\input{temporalia/ps64-III.tex}

\vfill

\vspace{-6mm}

\antiphona{}{temporalia/ant-alleluia-fiv-laudes-1.gtex} % repeat the antiphon - new page

\vfill
\pagebreak

\pars{Psalmus 4.} \scriptura{1 Sam. 2, 10; \textbf{H96}}

\vspace{-7mm}

\antiphona{I g\textsuperscript{2}}{temporalia/ant-dominusjudicabit-tp.gtex}

%\vspace{-4mm}

\scriptura{Canticum Annæ, 1 Reg. 2, 1-10}

%\vspace{-3mm}

\initiumpsalmi{temporalia/anna-initium-i-g2-auto.gtex}

%\psalmusEtTranslatioT{temporalia/anna-comb.tex}{10cm}
\input{temporalia/anna.tex}

%\vfill

\antiphona{}{temporalia/ant-dominusjudicabit-tp.gtex}

\vfill
\pagebreak

\pars{Psalmus 5.}

\vspace{-0.4cm}

\antiphona{II D}{temporalia/ant-alleluia-fiv-laudes-2.gtex}

\scriptura{Psalmus 148.}

\initiumpsalmi{temporalia/ps148-initium-ii-D-auto.gtex}

%\psalmusEtTranslatioT{temporalia/ps148-III-comb.tex}{10cm}
\input{temporalia/ps148-III.tex}

\rubrica{Hic non dicitur Gloria Patri.}

\vfill
\pagebreak

%
\scriptura{Psalmus 149.}

\initiumpsalmi{temporalia/ps149-initium-ii-D-auto.gtex}

%\psalmusEtTranslatioT{temporalia/ps149-III-comb.tex}{10cm}
\input{temporalia/ps149-III.tex}

\rubrica{Hic non dicitur Gloria Patri.}

\vfill
\pagebreak

%
\scriptura{Psalmus 150.}

\initiumpsalmi{temporalia/ps150-initium-ii-D-auto.gtex}

%\psalmusEtTranslatioT{temporalia/ps150-III-comb.tex}{10cm}
\input{temporalia/ps150-III.tex}

\vfill

\vspace{-6mm}

\antiphona{}{temporalia/ant-alleluia-fiv-laudes-2.gtex} % repeat the antiphon - new page

\vfill
\pagebreak

\pars{Capitulum.} \scriptura{Rom. 6, 9-10}

\grechangedim{interwordspacetext}{0.12 cm plus 0.15 cm minus 0.05 cm}{scalable}%
\cuminitiali{}{temporalia/capitulum-ChristusResurgens.gtex}
\grechangedim{interwordspacetext}{0.32 cm plus 0.15 cm minus 0.05 cm}{scalable}%

% preklad Jeruz. bible
%\trCapituliI

\vfill

\pars{Responsorium breve.} \scriptura{Cf. Mt. 28, 6; Cf. Gal. 3, 13}

\cuminitiali{VI}{temporalia/respbr-laud.gtex}

%\trResp

\vfill
\pagebreak

\pars{Hymnus}

\cuminitiali{VIII}{temporalia/hym-AuroraLucis.gtex}
\vspace{-3mm}
%\input{hym-AuroraLucis-bohtext.tex}

\vfill
%\pagebreak

\pars{Versus.}

% Versus. %%%
\sineinitiali{temporalia/versus-inresurrectione.gtex}

%\noindent \trVersus

\vfill
\pagebreak

\benedictus

\vspace{-1cm}

\vfill
\pagebreak

%\sideThumbs{{\scriptsize{}Fine horarum}}

\anteOrationem

\pagebreak

% Oratio. %%%
\oratioLaudes

\vspace{-1mm}
%\trOrationisI

\vfill

\rubrica{Hebdomadarius dicit iterum Dominus vobiscum. Postea cantatur a cantore:}
\vspace{2mm}

\cuminitiali{VII}{temporalia/benedicamus-tempore-paschali.gtex}

\vspace{1mm}

\ifx\magnificat\undefined
\else
\vfill
\pagebreak

\hora{Ad Vesperas.} %%%%%%%%%%%%%%%%%%%%%%%%%%%%%%%%%%%%%%%%%%%%%%%%%%%%%
%\sideThumbs{Vesperæ}

\cantusSineNeumas

%\vspace{0.5cm}
\grechangedim{interwordspacetext}{0.18 cm plus 0.15 cm minus 0.05 cm}{scalable}%
\cuminitiali{}{temporalia/deusinadiutorium-communis.gtex}
\grechangedim{interwordspacetext}{0.32 cm plus 0.15 cm minus 0.05 cm}{scalable}%

\vfill
%\pagebreak

\vspace{4mm}

\pars{Psalmus 1.}

\vspace{-0.4cm}

\antiphona{III g}{temporalia/ant-alleluia-fiv-vesperas.gtex}

\vspace{-4mm}

\scriptura{Psalmus 134.}

\initiumpsalmi{temporalia/ps134-initium-iii-g-auto.gtex}

%\psalmusEtTranslatioT{temporalia/ps134-III-comb.tex}{10cm}
\input{temporalia/ps134-III.tex}

\vspace{-1cm}

\vfill
\pagebreak

\pars{Psalmus 2.} \scriptura{Psalmus 135.}

\initiumpsalmi{temporalia/ps135-initium-iii-g-auto.gtex}

%\psalmusEtTranslatioT{temporalia/ps135-III-comb.tex}{10cm}
\input{temporalia/ps135-III.tex}

\vfill
\pagebreak

\pars{Psalmus 3.} \scriptura{Psalmus 136.}

\initiumpsalmi{temporalia/ps136-initium-iii-g-auto.gtex}

%\psalmusEtTranslatioT{temporalia/ps136-III-comb.tex}{10cm}
\input{temporalia/ps136-III.tex}

\vfill
\pagebreak

\pars{Psalmus 4.} \scriptura{Psalmus 137.}

\initiumpsalmi{temporalia/ps137-initium-iii-g-auto.gtex}

%\psalmusEtTranslatioT{temporalia/ps137-III-comb.tex}{10cm}
\input{temporalia/ps137-III.tex}

\vfill

\vspace{-6mm}

\antiphona{}{temporalia/ant-alleluia-fiv-vesperas.gtex} % repeat the antiphon - new page

\vfill
\pagebreak

\pars{Capitulum.} \scriptura{Rom. 6, 9-10}

\grechangedim{interwordspacetext}{0.12 cm plus 0.15 cm minus 0.05 cm}{scalable}%
\cuminitiali{}{temporalia/capitulum-ChristusResurgens.gtex}
\grechangedim{interwordspacetext}{0.32 cm plus 0.15 cm minus 0.05 cm}{scalable}%

% preklad Jeruz. bible
%\trCapituliI

\vfill

\pars{Responsorium breve.} \scriptura{Lc. 24, 34}

\cuminitiali{VI}{temporalia/respbr-vesp.gtex}

%\trResp

\vfill
\pagebreak

\pars{Hymnus}

\cuminitiali{VIII}{temporalia/hym-AdCoenam.gtex}
\vspace{-3mm}
%\begin{translatioMulticol}{4}
U~Beránkovy hostiny\\
oděni rouchy bílými,\\
když Rudým mořem prošli jsme,\\
Vladaři Kristu zpívejme.\\
\\
Když jeho tělem posvátným,\\
na kříži obětovaným,\\
se sytíme a~pijeme\\
jeho krev, v~Bohu žijeme.\columnbreak

Chráněni tímto pokrmem\\
před smrtonosným andělem,\\
svrhli jsme z~beder kruté jho\\
tyrana bezohledného.\\
\\
Kristus je naší paschou teď,\\
on sám se vydal za oběť\\
a~místo přesnic našim rtům\\
své tělo dává za pokrm.\columnbreak

Tys, nejčistější Oběti,\\
zlomila vládu podsvětí.\\
Z~otroctví lid je vykoupen,\\
odměna žití kyne všem.\\
\\
Hle, Kristus, když vstal ze hrobu,\\
jde z~pekel v~slavném průvodu\\
a~brány nebes otevřev,\\
vládce tmy vleče v~okovech.\columnbreak

Buď věčně, Kriste, věrným svým\\
plesáním velikonočním.\\
Nás, milostí tvou vzkříšené,\\
vem k~oslavě své vítězné. \\
\\
Sláva tobě, Pane,\\
jenž jsi vstal z~mrtvých,\\
s~Otcem i~Svatým Duchem\\
na věčné věky.\\
Amen.
\end{translatioMulticol}


\vfill
\pagebreak

\pars{Versus.} \scriptura{Lc. 24, 29}

% Versus. %%%
\sineinitiali{temporalia/versus-mane.gtex}

%\noindent \trVersus

\vfill
\pagebreak

\magnificat

\vspace{-1cm}

\vfill
\pagebreak

%\sideThumbs{{\scriptsize{}Fine horarum}}

\anteOrationem

\pagebreak

% Oratio. %%%
\oratioLaudes

\vspace{-1mm}
%\trOrationisI

\vfill

\rubrica{Hebdomadarius dicit iterum Dominus vobiscum. Postea cantatur a cantore:}
\vspace{2mm}

\cuminitiali{VII}{temporalia/benedicamus-tempore-paschali.gtex}

\vspace{1mm}
\fi

\end{document}

