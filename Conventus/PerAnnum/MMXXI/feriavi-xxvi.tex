\newcommand{\titulus}{\nomenFesti{S. Teresiæ a Iesu Infante, Virginis \& Ecclesiæ Doctoris.}
\dies{Die 1. Octobris.}}
\newcommand{\oratio}{\pars{Oratio.}

\noindent Deus, qui regnum tuum humílibus parvulísque dispónis, fac nos beátæ Terésiæ trámitem prósequi confidénter, ut, eius intercessióne, glória tua nobis revelétur ætérna.

\noindent Per Dóminum nostrum Iesum Christum, Fílium tuum, qui tecum vivit et regnat in unitáte Spíritus Sancti, Deus, per ómnia sǽcula sæculórum.

\noindent \Rbardot{} Amen.}
\newcommand{\invitatorium}{\pars{Invitatorium.}

\vspace{-4mm}

\antiphona{E}{temporalia/inv-regemvirginum.gtex}}
\newcommand{\hymnusmatutinum}{\pars{Hymnus}

\cuminitiali{VI}{temporalia/hym-DulciDepromat.gtex}}
\newcommand{\lectioi}{\pars{Lectio I.} \scriptura{Esth. 5, 1-2l}

\noindent De libro Esther.

\noindent Et factum est die tértio, indúta Esther regálibus vestiméntis stetit in átrio domus régiæ, quod erat intérius contra basílicam regis. At ille sedébat super sólium suum in consistório palátii contra óstium domus. Et factum est, cum vidísset Esther regínam stantem, plácuit óculis eius, et exténdit contra eam virgam áuream, quam tenébat manu; quæ accédens tétigit summitátem virgæ eius. Cumque régio fulgéret hábitu et invocásset ómnium rectórem et salvatórem Deum, assúmpsit duas fámulas et super unam quidem innitebátur, quasi in delíciis; áltera autem sequebátur dóminam defluéntia in humum induménta susténtans. Ipsa autem róseo vultu colóre perfúsa et gratis ac niténtibus óculis tristem celábat ánimum et mortis timóre contráctum. Ingréssa ígitur cuncta per órdinem óstia, stetit in aula interióre contra regem, ubi ille residébat super sólium regni sui indútus véstibus régiis auróque fulgens et pretiósis lapídibus; erátque terríbilis aspéctu, et virga áurea in manu eius.

\noindent Cumque elevásset fáciem, vidit eam sicut taurus in ímpetu iræ suæ et cógitans eam pérdere clamávit ambíguus: “Quis ausus est introíre in aulam non vocátus?”. Et regína córruit et, in pallórem colóre mutáto, lassa se reclinávit super caput ancíllulæ, quæ antecedébat. Convertítque Iudæórum Deus et univérsæ creatúræ Dóminus spíritum regis in mansuetúdinem, et festínus ac métuens exilívit de sólio; et susténtans eam ulnis suis, donec redíret ad se, verbis pacíficis ei blandiebátur: “Quid habes, Esther regína, soror mea et consors regni? Ego sum frater tuus, noli metúere. Non moriéris; non enim pro te, sed pro ómnibus hæc lex constitúta est.  Accéde!”. Et élevans áuream virgam pósuit super collum eius et osculátus est eam et ait: “Lóquere mihi”.}
\newcommand{\responsoriumi}{\pars{Responsorium 1.} \scriptura{\Rbardot{} Cantor \Vbardot{} ibidem; \textbf{H412}}

\vspace{-5mm}

\responsorium{II}{temporalia/resp-speminalionumquamhabui-CROCHU.gtex}{}}
\newcommand{\lectioii}{\pars{Lectio II.} \scriptura{Esth. 5, 2m-14}

\noindent Esther respóndit: “Vidi te, dómine, quasi ángelum Dei, et conturbátum est cor meum præ timóre glóriæ tuæ; valde enim mirábilis es, dómine, et fácies tua plena est gratiárum”. Cumque loquerétur, rursus córruit et pæne exanimáta est. Rex autem turbabátur, et omnes minístri eius. Dixítque ad eam rex: “Quid vis, Esther regína? Quæ est petítio tua? Etiámsi dimídiam partem regni petíeris, dábitur tibi”. At illa respóndit: “Si regi placet, óbsecro, ut vénias ad me hódie et Aman tecum ad convívium, quod parávi”. Statímque rex: “Vocáte, inquit, cito Aman, ut fiat verbum Esther”. Venérunt ítaque rex et Aman ad convívium, quod eis regína paráverat.

\noindent Dixítque ei rex, postquam vinum bíberat: “Quid petis, ut detur tibi, et pro qua re póstulas? Etiámsi dimídiam partem regni mei petíeris, impetrábis”. Cui respóndit Esther: “Petítio mea et preces: Si invéni in conspéctu regis grátiam, et si regi placet, ut det mihi, quod póstulo, et meam ímpleat petitiónem, véniat rex et Aman ad convívium, quod parábo eis, et cras fáciam secúndum verbum regis” . Egréssus est ítaque illo die Aman lætus et álacer corde. Cumque vidísset Mardochǽum sedéntem in fóribus palátii, et non solum non assurrexísse sibi, sed nec motum quidem de loco sessiónis suæ, indignátus est valde. Et, dissimuláta ira, revérsus in domum suam convocávit ad se amícos suos et Zares uxórem suam et expósuit illis magnitúdinem divitiárum suárum filiorúmque turbam, et quanta eum glória super omnes príncipes et servos suos rex elevásset. Et post hæc ait: “Regína quoque Esther nullum álium vocávit ad convívium cum rege præter me; apud quam étiam cras cum rege pransúrus sum. Et, cum ómnia hæc hábeam, nihil me habére puto, quámdiu vídero Mardochǽum Iudǽum sedéntem in fóribus regis”. Responderúntque ei Zares uxor eius et céteri amíci: “Iube parári excélsam trabem habéntem altitúdinis quinquagínta cúbitos et dic mane regi, ut appendátur super eam Mardochǽus; et sic ibis cum rege lætus ad convívium”. Plácuit ei consílium et iussit excélsam parári trabem.}
\newcommand{\responsoriumii}{\pars{Responsorium 2.} \scriptura{\Rbardot{} Esth. 13, 17 \Vbardot{} Ps. 79, 2; \textbf{H412}}

\vspace{-5mm}

\responsorium{II}{temporalia/resp-danobisdominelocum-CROCHU.gtex}{}}
\newcommand{\lectioiii}{\pars{Lectio III.} \scriptura{Manuscrits autobiographiques, Lisieux 1957, 227-229}

\noindent Ex narratióne vitæ sanctæ Terésiæ a Iesu Infánte, vírginis, ab ipsa exaráta.

\noindent Cum imménsa mea desidéria mihi martýrio essent, epístolas adívi sancti Pauli, ut responsiónem tandem invenírem. Oculos casu coniéci in cápita duodécimum et décimum tértium epístolæ primæ ad Corínthios, in primóque legi omnes simul apóstolos, prophétas, doctóres esse non posse, Ecclesiámque váriis constáre membris, oculúmque simul manum esse non posse. Apérta quidem respónsio, non tamen ea quæ mea impléret desidéria mihíque pacem afférret.

\noindent In lectióne perseverávi neque ánimum dimísi, propositúmque invéni súblevans: Æmulámini charísmata melióra. Et adhuc excellentiórem viam vobis demónstro. Apóstolus enim commentátur charísmata melióra nihil sine caritáte esse eandémque caritátem excellentiórem viam esse tuto ad Deum deducéntem. Quiétem tandem invéneram.

\noindent Mýsticum Ecclésiæ corpus cum aspícerem, in nullo membrórum quæ sanctus Paulus descrípsit me ipsam agnóveram, seu pótius in univérsis me cérnere volébam. Cáritas vocatiónis meæ cárdinem mihi óbtulit. Intelléxi Ecclésiam quidem corpus habére váriis compáctum membris, at hoc in córpore membrum non deésse necessárium et nobílius; intelléxi Ecclésiam cor habére et huiúsmodi cor amóre incénsum exsístere. Intelléxi unum amórem membra Ecclésiæ ad agéndum compéllere eodémque amóre exstíncto, apóstolos Evangélium non ámplius annuntiatúros, mártyres sánguinem non ámplius effusúros. Perspéxi et agnóvi amórem omnes in se conclúdere vocatiónes, amórem ómnia esse, eundémque univérsa témpora et loca comprehéndere, uno verbo amórem ætérnum esse.

\noindent Tunc, summo delirántis ánimi gáudio, exclamávi: O Iesu, mi amor, vocatiónem meam tandem invéni: vocátio mea est amor. Utique, locum mihi próprium in Ecclésia invéni, locúmque istum tu mihi dedísti, mi Deus. In corde Ecclésiæ, matris meæ, ego amor ero, et ita ero ómnia, ac meum desidérium in rem deducétur.}
%\newcommand{\responsoriumiii}{\pars{Responsorium 3.} %\scriptura{\Rbardot{} Prv. 8, 23-25 \Vbardot{} ibid. 8, 26.30; \textbf{H398}}
%
%\vspace{-5mm}
%
%\responsorium{VIII}{temporalia/resp-ipsisumdesponsata-CROCHU-cumdox.gtex}{}}
\newcommand{\responsoriumiii}{\pars{Responsorium 3.} \scriptura{\Rbardot{} Sap. 7, 10-11 \Vbardot{} Prov. 7, 4; \textbf{H401}}

\vspace{-5mm}

\responsorium{III}{temporalia/resp-supersalutem-CROCHU-cumdox.gtex}{}}
\newcommand{\hymnuslaudes}{\pars{Hymnus}

\cuminitiali{VIII}{temporalia/hym-AptataVirgo.gtex}}
\newcommand{\lectiobrevis}{\pars{Lectio Brevis.} \scriptura{Ct. 8, 7}

\noindent Aquæ multæ non potuérunt exstínguere caritátem, nec flúmina óbruent illam; si déderit homo omnem substántiam domus suæ pro dilectióne, quasi nihil despícient eum.}
\newcommand{\responsoriumbreve}{\pars{Responsorium breve.} \scriptura{Ps. 45, 6}

\cuminitiali{VI}{temporalia/resp-adiuvabiteam.gtex}}
\newcommand{\preces}{\noindent Christo, vírginum sponso et corónæ, lætis vócibus iubilémus~\gredagger{} et devóte eum exorémus, dicéntes:

\Rbardot{} Iesu, coróna vírginum, audi nos.

\noindent Christe, quem sanctæ vírgines ut únicum sponsum dilexérunt,~\gredagger{} concéde ut nihil a tua caritáte nos séparet.

\Rbardot{} Iesu, coróna vírginum, audi nos.

\noindent Qui Maríam, matrem tuam, regínam vírginum coronásti,~\gredagger{} da nos, eius intercessióne, cordis puritáte tibi iúgiter famulári.

\Rbardot{} Iesu, coróna vírginum, audi nos.

\noindent Per intercessiónem ancillárum tuárum, quæ íntegro et indivíso corde semper de te sollícitæ fuérunt, ut essent sanctæ córpore et spíritu,~\gredagger{} præsta ut figúra huius mundi, quæ prǽterit, nos a te numquam ábstrahat.

\Rbardot{} Iesu, coróna vírginum, audi nos.

\noindent Dómine Iesu, sponse, quem vírgines sapiéntes adventúrum exspectábant,~\gredagger{} concéde nobis, ut in spe vigilántes te præstolémur.

\Rbardot{} Iesu, coróna vírginum, audi nos.

\noindent Per intercessiónem sanctæ Terésiæ, quæ virgo fuit sápiens et de número prudéntum,~\gredagger{} præsta nobis innocéntiam vitæ et sapiéntiam.

\Rbardot{} Iesu, coróna vírginum, audi nos.}
\newcommand{\benedictus}{\pars{Canticum Zachariæ.} \scriptura{Mt. 18, 3}

\vspace{-4mm}

\antiphona{VII c}{temporalia/ant-nisiefficiaminisicutparvuli.gtex}

\vspace{-2mm}

\scriptura{Lc. 1, 68-79}

\vspace{-2mm}

\cantusSineNeumas
\initiumpsalmi{temporalia/benedictus-initium-vii-c-auto.gtex}

%\vspace{-1.5mm}

\input{temporalia/benedictus-vii-c.tex} \Abardot{}}
\newcommand{\hebdomada}{infra Hebdom. XXVI per Annum.}
\newcommand{\matub}{Matutinum Hebdomadae B}
\newcommand{\matubd}{Matutinum Hebdomadae B vel D}
\newcommand{\laudb}{Laudes Hebdomadae B}
\newcommand{\laudbd}{Laudes Hebdomadae B vel D}

% LuaLaTeX

\documentclass[a4paper, twoside, 12pt]{article}
\usepackage[latin]{babel}
%\usepackage[landscape, left=3cm, right=1.5cm, top=2cm, bottom=1cm]{geometry} % okraje stranky
%\usepackage[landscape, a4paper, mag=1166, truedimen, left=2cm, right=1.5cm, top=1.6cm, bottom=0.95cm]{geometry} % okraje stranky
\usepackage[landscape, a4paper, mag=1400, truedimen, left=0.5cm, right=0.5cm, top=0.5cm, bottom=0.5cm]{geometry} % okraje stranky

\usepackage{fontspec}
\setmainfont[FeatureFile={junicode.fea}, Ligatures={Common, TeX}, RawFeature=+fixi]{Junicode}
%\setmainfont{Junicode}

% shortcut for Junicode without ligatures (for the Czech texts)
\newfontfamily\nlfont[FeatureFile={junicode.fea}, Ligatures={Common, TeX}, RawFeature=+fixi]{Junicode}

\usepackage{multicol}
\usepackage{color}
\usepackage{lettrine}
\usepackage{fancyhdr}

% usual packages loading:
\usepackage{luatextra}
\usepackage{graphicx} % support the \includegraphics command and options
\usepackage{gregoriotex} % for gregorio score inclusion
\usepackage{gregoriosyms}
\usepackage{wrapfig} % figures wrapped by the text
\usepackage{parcolumns}
\usepackage[contents={},opacity=1,scale=1,color=black]{background}
\usepackage{tikzpagenodes}
\usepackage{calc}
\usepackage{longtable}
\usetikzlibrary{calc}

\setlength{\headheight}{14.5pt}

\input{conventuscommune.tex} % Often used macros

\newcommand{\annusEditionis}{2021}

%%%% Vicekrat opakovane kousky

\newcommand{\anteOrationem}{
  \rubrica{Ante Orationem, cantatur a Superiore:}

  \pars{Supplicatio Litaniæ.}

  \cuminitiali{}{temporalia/supplicatiolitaniae.gtex}

  \pars{Oratio Dominica.}

  \cuminitiali{}{temporalia/oratiodominica.gtex}

  \rubrica{Deinde dicitur ab Hebdomadario:}

  \cuminitiali{}{temporalia/dominusvobiscum-solemnis.gtex}

  \rubrica{In choro monialium loco Dominus vobiscum dicitur:}

  \sineinitiali{temporalia/domineexaudi.gtex}
}

\setlength{\columnsep}{30pt} % prostor mezi sloupci

%%%%%%%%%%%%%%%%%%%%%%%%%%%%%%%%%%%%%%%%%%%%%%%%%%%%%%%%%%%%%%%%%%%%%%%%%%%%%%%%%%%%%%%%%%%%%%%%%%%%%%%%%%%%%
\begin{document}

% Here we set the space around the initial.
% Please report to http://home.gna.org/gregorio/gregoriotex/details for more details and options
\grechangedim{afterinitialshift}{2.2mm}{scalable}
\grechangedim{beforeinitialshift}{2.2mm}{scalable}
\grechangedim{interwordspacetext}{0.22 cm plus 0.15 cm minus 0.05 cm}{scalable}%
\grechangedim{annotationraise}{-0.2cm}{scalable}

% Here we set the initial font. Change 38 if you want a bigger initial.
% Emit the initials in red.
\grechangestyle{initial}{\color{red}\fontsize{38}{38}\selectfont}

\pagestyle{empty}

%%%% Titulni stranka
\begin{titulusOfficii}
\ifx\titulus\undefined
\nomenFesti{Feria VI \hebdomada{}}
\else
\titulus
\fi
\end{titulusOfficii}

\vfill

\begin{center}
%Ad usum et secundum consuetudines chori \guillemotright{}Conventus Choralis\guillemotleft.

%Editio Sancti Wolfgangi \annusEditionis
\end{center}

\scriptura{}

\pars{}

\pagebreak

\renewcommand{\headrulewidth}{0pt} % no horiz. rule at the header
\fancyhf{}
\pagestyle{fancy}

\cantusSineNeumas

\hora{Ad Matutinum.} %%%%%%%%%%%%%%%%%%%%%%%%%%%%%%%%%%%%%%%%%%%%%%%%%%%%%

\vspace{2mm}

\cuminitiali{}{temporalia/dominelabiamea.gtex}

\vfill
%\pagebreak

\vspace{2mm}

\ifx\invitatorium\undefined
\pars{Invitatorium.} \scriptura{Lc. 24, 34; Psalmus 94; \textbf{H232}}

\antiphona{VI}{temporalia/inv-surrexitdominusvere.gtex}
\else
\invitatorium
\fi

\vfill
\pagebreak

\ifx\hymnusmatutinum\undefined
\pars{Hymnus.}

\cuminitiali{VIII}{temporalia/hym-LaetareCaelum.gtex}
\else
\hymnusmatutinum
\fi

\vspace{-3mm}

\vfill
\pagebreak

\ifx\matutinum\undefined
\ifx\matua\undefined
\else
% MAT A
\pars{Psalmus 1.}

\vspace{-4mm}

\antiphona{I a\textsuperscript{3}}{temporalia/ant-alleluia-turco24.gtex}

%\vspace{-2mm}

\scriptura{Ps. 34, 1-10}

%\vspace{-2mm}

\initiumpsalmi{temporalia/ps34i-initium-i-a5-auto.gtex}

\input{temporalia/ps34i-i-a5.tex}

\vfill
\pagebreak

\pars{Psalmus 2.} \scriptura{Ps. 34, 11-17}

%\vspace{-2mm}

\initiumpsalmi{temporalia/ps34ii-initium-i-a5-auto.gtex}

\input{temporalia/ps34ii-i-a5.tex}

\vfill
\pagebreak

\pars{Psalmus 3.} \scriptura{Ps. 34, 18-28}

\vspace{-2mm}

\initiumpsalmi{temporalia/ps34iii-initium-i-a5-auto.gtex}

\input{temporalia/ps34iii-i-a5.tex}

\vfill

\antiphona{}{temporalia/ant-alleluia-turco24.gtex}

\vfill
\pagebreak
\fi
\ifx\matub\undefined
\else
% MAT B
\pars{Psalmus 1.}

\vspace{-4mm}

\antiphona{D}{temporalia/ant-alleluia-turco2.gtex}

%\vspace{-2mm}

\scriptura{Ps. 37, 2-5}

%\vspace{-2mm}

\initiumpsalmi{temporalia/ps37ii_v-initium-d-g-auto.gtex}

\input{temporalia/ps37ii_v-d-g.tex}

\vfill
\pagebreak

\pars{Psalmus 2.}

\scriptura{Ps. 37, 6-13}

%\vspace{-2mm}

\initiumpsalmi{temporalia/ps37vi_xiii-initium-d-g-auto.gtex}

\input{temporalia/ps37vi_xiii-d-g.tex}

\vfill
\pagebreak

\pars{Psalmus 3.}

\scriptura{Ps. 37, 14-23}

%\vspace{-2mm}

\initiumpsalmi{temporalia/ps37xiv_xxiii-initium-d-g-auto.gtex}

\input{temporalia/ps37xiv_xxiii-d-g.tex}

\vfill

\antiphona{}{temporalia/ant-alleluia-turco2.gtex}

\vfill
\pagebreak
\fi
\ifx\matuc\undefined
\else
% MAT C
\pars{Psalmus 1.}

\vspace{-4mm}

\antiphona{I d\textsuperscript{3}}{temporalia/ant-alleluia-auglx5.gtex}

%\vspace{-3mm}

\scriptura{Ps. 68, 2-13}

%\vspace{-2mm}

\initiumpsalmi{temporalia/ps68ii_xiii-initium-i-d-auto.gtex}

%\vspace{-1.5mm}

\input{temporalia/ps68ii_xiii-i-d.tex}

\vfill
\pagebreak

\pars{Psalmus 2.}

\scriptura{Ps. 68, 14-22}

%\vspace{-2mm}

\initiumpsalmi{temporalia/ps68xiv_xxii-initium-i-d-auto.gtex}

\input{temporalia/ps68xiv_xxii-i-d.tex}

\vfill
\pagebreak

\pars{Psalmus 3.}

\scriptura{Ps. 68, 30-37}

%\vspace{-2mm}

\initiumpsalmi{temporalia/ps68iii-initium-i-d-auto.gtex}

\input{temporalia/ps68iii-i-d.tex}

\vfill

\antiphona{}{temporalia/ant-alleluia-auglx5.gtex}

\vfill
\pagebreak
\fi
\ifx\matud\undefined
\else
% MAT D
\pars{Psalmus 1.}

\vspace{-4mm}

\antiphona{I a\textsuperscript{2}}{temporalia/ant-alleluia-turco24.gtex}

%\vspace{-3mm}

\scriptura{Ps. 77, 1-16}

%\vspace{-2mm}

\initiumpsalmi{temporalia/ps77i_xvi-initium-i-a4-auto.gtex}

\input{temporalia/ps77i_xvi-i-a4.tex}

\vfill
\pagebreak

\pars{Psalmus 2.} \scriptura{Ps. 77, 17-31}

%\vspace{-2mm}

\initiumpsalmi{temporalia/ps77iii-initium-i-a4-auto.gtex}

\input{temporalia/ps77iii-i-a4.tex}

\vfill
\pagebreak

\pars{Psalmus 3.} \scriptura{Ps. 77, 32-39}

%\vspace{-2mm}

\initiumpsalmi{temporalia/ps77xxxii_xxxix-initium-i-a4-auto.gtex}

\input{temporalia/ps77xxxii_xxxix-i-a4.tex}

\vfill

\antiphona{}{temporalia/ant-alleluia-turco24.gtex}

\vfill
\pagebreak
\fi
\else
\matutinum
\fi

\pars{Versus.}

\ifx\matversus\undefined
\noindent \Vbardot{} In resurrectióne tua, Christe, allelúia.

\noindent \Rbardot{} Cæli et terra læténtur, allelúia.
\else
\matversus
\fi

\vspace{5mm}

\sineinitiali{temporalia/oratiodominica-mat.gtex}

\vspace{5mm}

\pars{Absolutio.}

\cuminitiali{}{temporalia/absolutio-ipsius.gtex}

\vfill
\pagebreak

\cuminitiali{}{temporalia/benedictio-solemn-deus.gtex}

\vspace{7mm}

\lectioi

\noindent \Vbardot{} Tu autem, Dómine, miserére nobis.
\noindent \Rbardot{} Deo grátias.

\vfill
\pagebreak

\responsoriumi

\vfill
\pagebreak

\cuminitiali{}{temporalia/benedictio-solemn-christus.gtex}

\vspace{7mm}

\lectioii

\noindent \Vbardot{} Tu autem, Dómine, miserére nobis.
\noindent \Rbardot{} Deo grátias.

\vfill
\pagebreak

\responsoriumii

\vfill
\pagebreak

\cuminitiali{}{temporalia/benedictio-solemn-ignem.gtex}

\vspace{7mm}

\lectioiii

\noindent \Vbardot{} Tu autem, Dómine, miserére nobis.
\noindent \Rbardot{} Deo grátias.

\vfill
\pagebreak

\responsoriumiii

\vfill
\pagebreak

\rubrica{Reliqua omittuntur, nisi Laudes separandæ sint.}

\sineinitiali{temporalia/domineexaudi.gtex}

\vfill

\oratio

\vfill

\noindent \Vbardot{} Dómine, exáudi oratiónem meam.
\Rbardot{} Et clamor meus ad te véniat.

\vfill

\noindent \Vbardot{} Benedicámus Dómino.
\noindent \Rbardot{} Deo grátias.

\vfill

\noindent \Vbardot{} Fidélium ánimæ per misericórdiam Dei requiéscant in pace.
\Rbardot{} Amen.

\vfill
\pagebreak

\hora{Ad Laudes.} %%%%%%%%%%%%%%%%%%%%%%%%%%%%%%%%%%%%%%%%%%%%%%%%%%%%%

\cantusSineNeumas

\vspace{0.5cm}
\grechangedim{interwordspacetext}{0.18 cm plus 0.15 cm minus 0.05 cm}{scalable}%
\cuminitiali{}{temporalia/deusinadiutorium-communis.gtex}
\grechangedim{interwordspacetext}{0.22 cm plus 0.15 cm minus 0.05 cm}{scalable}%

\vfill
\pagebreak

\ifx\hymnuslaudes\undefined
\ifx\laudac\undefined
\else
\pars{Hymnus}

\cuminitiali{I}{temporalia/hym-ChorusNovae-praglia.gtex}
\vspace{-3mm}
\fi
\ifx\laudbd\undefined
\else
\pars{Hymnus}

\cuminitiali{I}{temporalia/hym-ChorusNovae.gtex}
\vspace{-3mm}
\fi
\else
\hymnuslaudes
\fi

\vfill
\pagebreak

\ifx\laudes\undefined
\ifx\lauda\undefined
\else
\pars{Psalmus 1.}

\vspace{-4mm}

\antiphona{VI F}{temporalia/ant-alleluia-turco6.gtex}

\scriptura{Psalmus 50.}

\initiumpsalmi{temporalia/ps50-initium-vi-F-auto.gtex}

\input{temporalia/ps50-vi-F.tex}

\vfill

\antiphona{}{temporalia/ant-alleluia-turco6.gtex}

\vfill
\pagebreak

\pars{Psalmus 2.} \scriptura{Is. 45, 25}

\vspace{-4mm}

\antiphona{V a}{temporalia/ant-indominoiustificabitur-tp.gtex}

\scriptura{Canticum Isaiæ, Is. 45, 15-30}

%\vspace{-2mm}

\initiumpsalmi{temporalia/isaiae2-initium-v-a-auto.gtex}

\input{temporalia/isaiae2-v-a.tex}

\vfill

\antiphona{}{temporalia/ant-indominoiustificabitur-tp.gtex}

\vfill
\pagebreak

\pars{Psalmus 3.}

\vspace{-4mm}

\antiphona{IV* e}{temporalia/ant-alleluia-turco9.gtex}

\scriptura{Psalmus 99.}

\initiumpsalmi{temporalia/ps99-initium-iv_-e-auto.gtex}

\input{temporalia/ps99-iv_-e.tex} \Abardot{}

\vfill
\pagebreak
\fi
\ifx\laudb\undefined
\else
\pars{Psalmus 1.}

\vspace{-4mm}

\antiphona{VII a}{temporalia/ant-alleluia-turco29.gtex}

\scriptura{Psalmus 50.}

\initiumpsalmi{temporalia/ps50-initium-vii-a-auto.gtex}

\input{temporalia/ps50-vii-a.tex}

\vfill

\antiphona{}{temporalia/ant-alleluia-turco29.gtex}

\vfill
\pagebreak

\pars{Psalmus 2.} \scriptura{Hab. 3, 2; \textbf{H99}}

\vspace{-6mm}

\antiphona{IV* e}{temporalia/ant-domineaudivi-tp.gtex}

\vspace{-2mm}

\scriptura{Canticum Habacuc, Hab. 3, 2-19}

%\vspace{-2mm}

%\initiumpsalmi{temporalia/habacuc-initium-iv_-e-auto.gtex}
\initiumpsalmi{temporalia/habacuc-initium-iv_-e.gtex}

\input{temporalia/habacuc-iv_-e.tex}

\vfill

\antiphona{}{temporalia/ant-domineaudivi-tp.gtex}

\vfill
\pagebreak

\pars{Psalmus 3.}

\vspace{-4mm}

\antiphona{E}{temporalia/ant-alleluia-turco4.gtex}

\vspace{-2mm}

\scriptura{Psalmus 147.}

%\vspace{-3mm}

%\initiumpsalmi{temporalia/ps147-initium-e-auto.gtex}
\initiumpsalmi{temporalia/ps147-initium-e.gtex}

\input{temporalia/ps147-e.tex} \Abardot{}

\vfill
\pagebreak
\fi
\ifx\laudc\undefined
\else
\pars{Psalmus 1.}

\vspace{-4mm}

\antiphona{VIII G\textsuperscript{2}}{temporalia/ant-alleluia-turco13.gtex}

\scriptura{Psalmus 50.}

\initiumpsalmi{temporalia/ps50-initium-viii-G5-auto.gtex}

\input{temporalia/ps50-viii-G5.tex}

\vfill

\antiphona{}{temporalia/ant-alleluia-turco13.gtex}

\vfill
\pagebreak

\pars{Psalmus 2.}

\vspace{-4mm}

\antiphona{VIII G}{temporalia/ant-nonnosderelinquas-tp.gtex}

%\vspace{-2mm}

\scriptura{Canticum Ieremiæ, Ier. 14, 17-31}

%\vspace{-2mm}

\initiumpsalmi{temporalia/jeremiae2-initium-viii-G.gtex}

\input{temporalia/jeremiae2-viii-G.tex} \Abardot{}

\vfill
\pagebreak

\pars{Psalmus 3.}

\vspace{-4mm}

\antiphona{E}{temporalia/ant-alleluia-praglia-e2.gtex}

\vspace{-2mm}

\scriptura{Psalmus 99.}

%\vspace{-2mm}

\initiumpsalmi{temporalia/ps99-initium-e-auto.gtex}

\input{temporalia/ps99-e.tex} \Abardot{}

\vfill
\pagebreak
\fi
\ifx\laudd\undefined
\else
\pars{Psalmus 1.}

\vspace{-4mm}

\antiphona{I f}{temporalia/ant-alleluia-turco20.gtex}

\scriptura{Psalmus 50.}

\initiumpsalmi{temporalia/ps50-initium-i-f-auto.gtex}

\input{temporalia/ps50-i-f.tex}

\vfill

\antiphona{}{temporalia/ant-alleluia-turco20.gtex}

\vfill
\pagebreak

\pars{Psalmus 2.} \scriptura{Ac. 22, 14}

\vspace{-4mm}

\antiphona{VIII G}{temporalia/ant-beatiquilavantstolas.gtex}

%\vspace{-2mm}

\scriptura{Canticum Tobiæ, Tob. 13, 10-18}

%\vspace{-2mm}

\initiumpsalmi{temporalia/tobiae2-initium-viii-G-auto.gtex}

\input{temporalia/tobiae2-viii-G.tex} \Abardot{}

\vfill
\pagebreak

\pars{Psalmus 3.}

\vspace{-4mm}

\antiphona{VI F}{temporalia/ant-alleluia-turco5.gtex}

\vspace{-2mm}

\scriptura{Psalmus 147.}

%\vspace{-2mm}

\initiumpsalmi{temporalia/ps147-initium-vi-F-auto.gtex}

\input{temporalia/ps147-vi-F.tex} \Abardot{}

\vfill
\pagebreak
\fi
\else
\laudes
\fi

\ifx\lectiobrevis\undefined
\pars{Lectio Brevis.} \scriptura{Ac. 5, 30-32}

\noindent Deus patrum nostrórum suscitávit Iesum, quem vos interemístis suspendéntes in ligno; hunc Deus Príncipem et Salvatórem exaltávit déxtera sua ad dandam pæniténtiam Israel et remissiónem peccatórum. Et nos sumus testes horum verbórum, et Spíritus Sanctus, quem dedit Deus obœdiéntibus sibi.
\else
\lectiobrevis
\fi

\vfill

\ifx\responsoriumbreve\undefined
\pars{Responsorium breve.} \scriptura{Cf. Mt. 28, 6; Cf. Gal. 3, 13}

\cuminitiali{VI}{temporalia/resp-surrexitdominusdesepulcro.gtex}
\else
\responsoriumbreve
\fi

\vfill
\pagebreak

\benedictus

\vspace{-1cm}

\vfill
\pagebreak

\pars{Preces.}

\sineinitiali{}{temporalia/tonusprecum.gtex}

\ifx\preces\undefined
\ifx\lauda\undefined
\else
\noindent Deum Patrem, qui vitam novam per Christi resurrectiónem cóntulit nobis,~\gredagger{} súpplices exorémus:

\Rbardot{} Clarífica nos claritáte Christi.

\noindent Deus, qui opéribus tuis antíquam dispensatiónem manifestásti, terram creásti et fidélis es in ómnibus generatiónibus,~\gredagger{} exáudi nos, clementíssime Pater.

\Rbardot{} Clarífica nos claritáte Christi.

\noindent Purífica nos puritáte veritátis tuæ, et gressus nostros dírige in cordis sanctitáte,~\gredagger{} ut quod iustum est tibíque plácitum agámus.

\Rbardot{} Clarífica nos claritáte Christi.

\noindent Illúmina vultum tuum super nos,~\gredagger{} ut a peccáto liberáti bonis domus tuæ repleámur.

\Rbardot{} Clarífica nos claritáte Christi.

\noindent Qui per Christum nos tibi reconciliásti,~\gredagger{} pacem nobis largíre omnibúsque in orbe terrárum degéntibus.

\Rbardot{} Clarífica nos claritáte Christi.
\fi
\ifx\laudb\undefined
\else
\noindent Deus Pater Christum per Spíritum suscitávit, et étiam mortália córpora nostra vivificábit.~\gredagger{} Quare clamémus:

\Rbardot{} Dómine, vivífica nos Spíritu Sancto tuo.

\noindent Pater sancte, qui accepísti holocáustum Fílii tui, resúscitans eum ex mórtuis,~\gredagger{} súscipe hodiérnam nostram oblatiónem et perduc nos in vitam ætérnam.

\Rbardot{} Dómine, vivífica nos Spíritu Sancto tuo.

\noindent Opera nostra hódie propítius intuére,~\gredagger{} ut fiant ad glóriam tuam et ad ómnium sanctificatiónem.

\Rbardot{} Dómine, vivífica nos Spíritu Sancto tuo.

\noindent Opus nostrum hódie non sit vanum, sed univérsis homínibus insérviat~\gredagger{} et sic operántes ad regnum tuum fac nos perveníre.

\Rbardot{} Dómine, vivífica nos Spíritu Sancto tuo.

\noindent Aperi hódie óculos nostros et cor nostrum ad fratres,~\gredagger{} ut nos ínvicem amémus nobísque serviámus.

\Rbardot{} Dómine, vivífica nos Spíritu Sancto tuo.
\fi
\ifx\laudc\undefined
\else
\noindent Deum Patrem, qui vitam novam per Christi resurrectiónem cóntulit nobis,~\gredagger{} súpplices exorémus:

\Rbardot{} Clarífica nos claritáte Christi.

\noindent Deus, qui opéribus tuis antíquam dispensatiónem manifestásti, terram creásti et fidélis es in ómnibus generatiónibus,~\gredagger{} exáudi nos, clementíssime Pater.

\Rbardot{} Clarífica nos claritáte Christi.

\noindent Purífica nos puritáte veritátis tuæ, et gressus nostros dírige in cordis sanctitáte,~\gredagger{} ut quod iustum est tibíque plácitum agámus.

\Rbardot{} Clarífica nos claritáte Christi.

\noindent Illúmina vultum tuum super nos,~\gredagger{} ut a peccáto liberáti bonis domus tuæ repleámur.

\Rbardot{} Clarífica nos claritáte Christi.

\noindent Qui per Christum nos tibi reconciliásti,~\gredagger{} pacem nobis largíre omnibúsque in orbe terrárum degéntibus.

\Rbardot{} Clarífica nos claritáte Christi.
\fi
\ifx\laudd\undefined
\else
\noindent Deus Pater Christum per Spíritum suscitávit, et étiam mortália córpora nostra vivificábit.~\gredagger{} Quare clamémus:

\Rbardot{} Dómine, vivífica nos Spíritu Sancto tuo.

\noindent Pater sancte, qui accepísti holocáustum Fílii tui, resúscitans eum ex mórtuis,~\gredagger{} súscipe hodiérnam nostram oblatiónem et perduc nos in vitam ætérnam.

\Rbardot{} Dómine, vivífica nos Spíritu Sancto tuo.

\noindent Opera nostra hódie propítius intuére,~\gredagger{} ut fiant ad glóriam tuam et ad ómnium sanctificatiónem.

\Rbardot{} Dómine, vivífica nos Spíritu Sancto tuo.

\noindent Opus nostrum hódie non sit vanum, sed univérsis homínibus insérviat~\gredagger{} et sic operántes ad regnum tuum fac nos perveníre.

\Rbardot{} Dómine, vivífica nos Spíritu Sancto tuo.

\noindent Aperi hódie óculos nostros et cor nostrum ad fratres,~\gredagger{} ut nos ínvicem amémus nobísque serviámus.

\Rbardot{} Dómine, vivífica nos Spíritu Sancto tuo.
\fi 
\else
\preces
\fi

\vfill

\pars{Oratio Dominica.}

\cuminitiali{}{temporalia/oratiodominicaalt.gtex}

\vfill
\pagebreak

\rubrica{vel:}

\pars{Supplicatio Litaniæ.}

\cuminitiali{}{temporalia/supplicatiolitaniae.gtex}

\vfill

\pars{Oratio Dominica.}

\cuminitiali{}{temporalia/oratiodominica.gtex}

\vfill
\pagebreak

% Oratio. %%%
\oratio

\vspace{-1mm}

\vfill

\rubrica{Hebdomadarius dicit Dominus vobiscum, vel, absente sacerdote vel diacono, sic concluditur:}

\vspace{2mm}

\antiphona{C}{temporalia/dominusnosbenedicat.gtex}

\rubrica{Postea cantatur a cantore:}

\vspace{2mm}

\cuminitiali{VII}{temporalia/benedicamus-tempore-paschali.gtex}

\vspace{1mm}

\vfill
\pagebreak

\end{document}

