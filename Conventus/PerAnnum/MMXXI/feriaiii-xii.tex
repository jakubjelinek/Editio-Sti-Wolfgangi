\newcommand{\titulus}{\nomenFesti{S. Paulini Nolani, Episcopi.}
\dies{Die 22. Iunii.}}
\newcommand{\oratio}{\pars{Oratio.}

\noindent Deus, qui beátum Paulínum, epíscopum, paupertátis amóre et pastoráli sollicitúdine claréscere voluísti, concéde propítius, ut, cuius mérita celebrámus, caritátis imitémur exémpla.

\noindent Per Dóminum nostrum Iesum Christum, Fílium tuum, qui tecum vivit et regnat in unitáte Spíritus Sancti, Deus, per ómnia sǽcula sæculórum.

\noindent \Rbardot{} Amen.}
\newcommand{\invitatorium}{\pars{Invitatorium.}

\vspace{-4mm}

\antiphona{IV}{temporalia/inv-mirabileminsanctis.gtex}}
\newcommand{\hymnusmatutinum}{\pars{Hymnus}

\cuminitiali{IV}{temporalia/hym-LaetiColentes.gtex}}
\newcommand{\matversus}{\noindent \Vbardot{} Miserére mei, Deus, quóniam conculcávit me homo, tota die impúgnans oppréssit me. Ego in te sperábo.

\noindent \Rbardot{} Quóniam eripuísti ánimam meam de morte et pedes meos de lapsu Ego.}
\newcommand{\lectioi}{\pars{Lectio I.} \scriptura{1 Sam. 17, 57-58; 18, 1-9}

\noindent De libro primo Samuélis.

\noindent In diébus illis: Cum regréssus esset David, percússo Philisthǽo, tulit eum Abner et introdúxit coram Saul caput Philisthǽi habéntem in manu. Et ait ad eum Saul: «De qua progénie es, o aduléscens?». Dixítque David: «Fílius servi tui Isai Bethlehemítæ ego sum».

\noindent Et factum est, cum complésset loqui ad Saul, ánima Iónathan colligáta est ánimæ David, et diléxit eum Iónathan quasi ánimam suam. Tulítque eum Saul in die illa et non concéssit ei, ut reverterétur in domum patris sui. Iniérunt autem Iónathan et David fœdus; diligébat enim eum quasi ánimam suam. Et exspoliávit se Iónathan túnicam, qua erat vestítus, et dedit eam David et réliqua vestiménta sua usque ad gládium et arcum suum et usque ad bálteum. Egrediebátur quoque David ad ómnia, quæcúmque misísset eum Saul, et próspere agébat; posuítque eum Saul super viros belli, et accéptus erat in óculis univérsi pópuli, étiam in conspéctu famulórum Saul.

\noindent Porro cum reverteréntur, cum redíret David, percússo Philisthǽo, egréssæ sunt mulíeres de univérsis úrbibus Israel cantántes chorósque ducéntes in occúrsum Saul regis in týmpanis et in cánticis lætítiæ et in sistris. Et præcinébant mulíeres ludéntes atque dicéntes: «Percússit Saul mília sua, et David decem mília sua». Irátus est autem Saul nimis, et displícuit in óculis eius iste sermo, dixítque: «Dedérunt David decem mília et mihi dedérunt mília; quid ei súperest nisi solum regnum?». Non rectis ergo óculis Saul aspiciébat David ex die illa et deínceps.}
\newcommand{\responsoriumi}{\pars{Responsorium 1.} \scriptura{\Rbardot{} 1 Sam. 18, 7 \Vbardot{} ibid.}

\vspace{-5mm}

\responsorium{VIII}{temporalia/resp-percussitsaulmille2.gtex}{}}
\newcommand{\lectioii}{\pars{Lectio II.} \scriptura{1 Sam. 18, 20-30}

\noindent Diléxit autem Michol fília Saul áltera David, et nuntiátum est Saul et plácuit ei; dixítque Saul: «Dabo eam illi, ut fiat ei in scándalum, et sit super eum manus Philisthinórum». Dixit ergo Saul ad David áltera vice: «Gener meus eris hódie». Et mandávit Saul servis suis: «Loquímini ad David secréto dicéntes: “Ecce places regi, et omnes servi eius díligunt te; nunc ergo esto gener regis”». Et locúti sunt servi Saul in áuribus David ómnia verba hæc, et ait David: «Num parum vobis vidétur génerum esse regis? Ego autem sum vir pauper et ténuis». Et renuntiavérunt servi Saul dicéntes: «Huiuscémodi verba locútus est David». Dixit autem Saul: «Sic loquímini ad David: “Non habet necésse rex sponsália, nisi tantum centum præpútia Philisthinórum, ut fiat últio de inimícis regis”». Porro Saul cogitábat trádere David in mánibus Philisthinórum.

\noindent Cumque renuntiássent servi eius David verba, quæ díxerat Saul, plácuit sermo in óculis David, ut fíeret gener regis. Et nondum erant dies impléti, cum David surgens ábiit cum viris, qui sub eo erant, et percússit ex Philísthim ducéntos viros; et áttulit præpútia eórum, et annumeravérunt ea regi, ut esset gener eius.

\noindent Dedit ítaque ei Saul Michol fíliam suam uxórem. Et vidit Saul et intelléxit quia Dóminus esset cum David; Michol autem fília Saul diligébat eum. Et Saul magis cœpit timére David; factúsque est Saul inimícus David cunctis diébus. Et egréssi sunt príncipes Philisthinórum; et, quotiescúmque egrediebántur, próspere agébat David magis quam omnes servi Saul, et célebre factum est nomen eius nimis.}
\newcommand{\responsoriumii}{\pars{Responsorium 2.} \scriptura{\Rbardot{} 2 Reg. 7, 8.9.12.13; \Vbardot{} ibid., 7, 9.1; \textbf{H395}}

\vspace{-5mm}

\responsorium{I}{temporalia/resp-egotetuli-CROCHU.gtex}{}}
\newcommand{\lectioiii}{\pars{Lectio III.} \scriptura{Epist. 3 ad Alypium, 1. 5. 6: CSEL 29, 13-14. 17-18}

\noindent Ex Epístolis sancti Paulíni Noláni epíscopi.

\noindent Hæc est vera cáritas, hæc perfécta diléctio, quam tibi circa humilitátem nostram inésse docuísti, dómine vere sancte et mérito beatíssime ac desiderábilis. Accépimus enim per hóminem nostrum Iuliánum de Carthágine reverténtem lítteras tantam nobis sanctitátis tuæ lucem afferéntes, ut nobis caritátem tuam non agnóscere, sed recognóscere viderémur. Quia vidélicet ex illo, qui nos ab orígine mundi prædestinávit sibi, cáritas ista manávit, in quo facti sumus ántequam nati, quia ipse fecit nos et non ipsi nos, qui fecit quæ futúra sunt. Huius ígitur præsciéntia et ópere formáti, in similitúdinem voluntátum et unitátem fídei vel unitátis fidem, præveniénte notítiam, caritáte conéxi sumus, ut nos ínvicem ante corporáles conspéctus revelánte spíritu noscerémus.

\noindent Gratulámur ítaque et gloriámur in Dómino, qui unus atque idem ubíque terrárum operátur in suis dilectiónem suam Spíritu Sancto, quem super omnem carnem effúdit, flúminis ímpetu lætíficans civitátem suam, in cuius te cívibus principálem cum princípibus pópuli sui Sede Apostólica mérito collocávit nosque étiam, quos eréxit elísos et de terra ínopes suscitávit, in vestra vóluit sorte numerári. Sed magis gratulámur in eo Dómini múnere, quo nos in péctoris tui habitatióne constítuit quoque ita viscéribus tuis insinuáre dignátus est, ut peculiárem nobis caritátis tuæ fidúciam vindicémus, his offíciis atque munéribus provocáti, ut nos diffidénter aut léviter te amáre non líceat.

\noindent De me ne quid ignóres, scias antiquíssimum peccatórem non ita olim de ténebris et umbra mortis edúctum spíritum auræ vitális hausísse nec ita olim posuísse in arátro manum et crucem Dómini sustulísse, quam ut in finem perférre valeámus, oratiónibus tuis adiuvémur. Accumulábitur hæc méritis tuis merces, si intervéntu tuo ónera nostra releváveris. Sanctus enim laborántem ádiuvans, quia fratrem non audémus dícere, exaltábitur sicut cívitas magna.

\noindent Panem unum sanctitáti tuæ unitátis grátia mísimus, in quo étiam Trinitátis solíditas continétur. Hunc panem eulogíam esse tu fácies dignatióne suméndi.}
\newcommand{\responsoriumiii}{\pars{Responsorium 3.} \scriptura{\Vbardot{} Ps. 45, 5; \textbf{H353}}

\vspace{-5mm}

\responsorium{VIII}{temporalia/resp-orantesanctoclemente-CROCHU-cumdox.gtex}{}}
\newcommand{\hymnuslaudes}{\pars{Hymnus}

\cuminitiali{VIII}{temporalia/hym-ORedemptorisPietas.gtex}}
\newcommand{\lectiobrevis}{\pars{Lectio Brevis.} \scriptura{Rom. 12, 1-2 }

\noindent Obsecro vos, fratres, per misericórdiam Dei, ut exhibeátis córpora vestra hóstiam vivéntem, sanctam, Deo placéntem, rationábile obséquium vestrum; et nolíte conformári huic sǽculo, sed transformámini renovatióne mentis, ut probétis quid sit volúntas Dei, quid bonum et bene placens et perféctum.}
\newcommand{\responsoriumbreve}{\pars{Responsorium breve.} \scriptura{Sir. 45, 9}

\antiphona{VI}{temporalia/resp-amaviteum.gtex}}
\newcommand{\preces}{\noindent Christum Deum sanctum, fratres, exaltémus, orántes ut serviámus illi in sanctitáte et iustítia coram ipso ómnibus diébus nostris,~\gredagger{} et acclamémus:

\Rbardot{} Tu solus sanctus, Dómine.

\noindent Qui tentári voluísti per ómnia pro similitúdine nostra absque peccáto,~\gredagger{} miserére nostri, Dómine Iesu.

\Rbardot{} Tu solus sanctus, Dómine.

\noindent Qui nos omnes ad perfectiónem caritátis vocásti,~\gredagger{} sanctífica nos, Dómine Iesu.

\Rbardot{} Tu solus sanctus, Dómine.

\noindent Qui nos iussísti esse salem terræ et lucem mundi,~\gredagger{} illúmina nos, Dómine Iesu.

\Rbardot{} Tu solus sanctus, Dómine.

\noindent Qui voluísti ministráre, non ministrári,~\gredagger{} fac nos tibi et frátribus humíliter servíre, Dómine Iesu.

\Rbardot{} Tu solus sanctus, Dómine.

\noindent Tu, splendor glóriæ Patris et figúra substántiæ eius,~\gredagger{} fac ut in glória vultum tuum respiciámus, Dómine Iesu.

\Rbardot{} Tu solus sanctus, Dómine.}
\newcommand{\benedictus}{\pars{Canticum Zachariæ.} \scriptura{Cf. Gregorius Magnus, Hom. ev. I, 15, 1; \textbf{H139}}

%\vspace{-6mm}

{
\grechangedim{interwordspacetext}{0.18 cm plus 0.15 cm minus 0.05 cm}{scalable}%
\antiphona{VII c}{temporalia/ant-siculmenveri.gtex}
\grechangedim{interwordspacetext}{0.22 cm plus 0.15 cm minus 0.05 cm}{scalable}%
}

%\vspace{-2mm}

\scriptura{Lc. 1, 68-79}

%\vspace{-2mm}

\cantusSineNeumas
\initiumpsalmi{temporalia/benedictus-initium-vii-c-auto.gtex}

%\vspace{-1.5mm}

\input{temporalia/benedictus-vii-c.tex} \Abardot{}}
\newcommand{\hebdomada}{infra Hebdom. XII post Pentecosten.}
\newcommand{\oratioLaudes}{\cuminitiali{}{temporalia/oratio12.gtex}}

% LuaLaTeX

\documentclass[a4paper, twoside, 12pt]{article}
\usepackage[latin]{babel}
%\usepackage[landscape, left=3cm, right=1.5cm, top=2cm, bottom=1cm]{geometry} % okraje stranky
%\usepackage[landscape, a4paper, mag=1166, truedimen, left=2cm, right=1.5cm, top=1.6cm, bottom=0.95cm]{geometry} % okraje stranky
\usepackage[landscape, a4paper, mag=1400, truedimen, left=0.5cm, right=0.5cm, top=0.5cm, bottom=0.5cm]{geometry} % okraje stranky

\usepackage{fontspec}
\setmainfont[FeatureFile={junicode.fea}, Ligatures={Common, TeX}, RawFeature=+fixi]{Junicode}
%\setmainfont{Junicode}

% shortcut for Junicode without ligatures (for the Czech texts)
\newfontfamily\nlfont[FeatureFile={junicode.fea}, Ligatures={Common, TeX}, RawFeature=+fixi]{Junicode}

\usepackage{multicol}
\usepackage{color}
\usepackage{lettrine}
\usepackage{fancyhdr}

% usual packages loading:
\usepackage{luatextra}
\usepackage{graphicx} % support the \includegraphics command and options
\usepackage{gregoriotex} % for gregorio score inclusion
\usepackage{gregoriosyms}
\usepackage{wrapfig} % figures wrapped by the text
\usepackage{parcolumns}
\usepackage[contents={},opacity=1,scale=1,color=black]{background}
\usepackage{tikzpagenodes}
\usepackage{calc}
\usepackage{longtable}
\usetikzlibrary{calc}

\setlength{\headheight}{14.5pt}

\input{conventuscommune.tex} % Often used macros

\newcommand{\annusEditionis}{2021}

%%%% Vicekrat opakovane kousky

\newcommand{\anteOrationem}{
  \rubrica{Ante Orationem, cantatur a Superiore:}

  \pars{Supplicatio Litaniæ.}

  \cuminitiali{}{temporalia/supplicatiolitaniae.gtex}

  \pars{Oratio Dominica.}

  \cuminitiali{}{temporalia/oratiodominica.gtex}

  \rubrica{Deinde dicitur ab Hebdomadario:}

  \cuminitiali{}{temporalia/dominusvobiscum-solemnis.gtex}

  \rubrica{In choro monialium loco Dominus vobiscum dicitur:}

  \sineinitiali{temporalia/domineexaudi.gtex}
}

\setlength{\columnsep}{30pt} % prostor mezi sloupci

%%%%%%%%%%%%%%%%%%%%%%%%%%%%%%%%%%%%%%%%%%%%%%%%%%%%%%%%%%%%%%%%%%%%%%%%%%%%%%%%%%%%%%%%%%%%%%%%%%%%%%%%%%%%%
\begin{document}

% Here we set the space around the initial.
% Please report to http://home.gna.org/gregorio/gregoriotex/details for more details and options
\grechangedim{afterinitialshift}{2.2mm}{scalable}
\grechangedim{beforeinitialshift}{2.2mm}{scalable}
\grechangedim{interwordspacetext}{0.22 cm plus 0.15 cm minus 0.05 cm}{scalable}%
\grechangedim{annotationraise}{-0.2cm}{scalable}

% Here we set the initial font. Change 38 if you want a bigger initial.
% Emit the initials in red.
\grechangestyle{initial}{\color{red}\fontsize{38}{38}\selectfont}

\pagestyle{empty}

%%%% Titulni stranka
\begin{titulusOfficii}
\ifx\titulus\undefined
\nomenFesti{Feria III \hebdomada{}}
\else
\titulus
\fi
\end{titulusOfficii}

\vfill

\begin{center}
%Ad usum et secundum consuetudines chori \guillemotright{}Conventus Choralis\guillemotleft.

%Editio Sancti Wolfgangi \annusEditionis
\end{center}

\scriptura{}

\pars{}

\pagebreak

\renewcommand{\headrulewidth}{0pt} % no horiz. rule at the header
\fancyhf{}
\pagestyle{fancy}

\cantusSineNeumas

\ifx\oratio\undefined
\ifx\laudb\undefined
\else
\newcommand{\oratio}{\pars{Oratio.}

\noindent Dómine Iesu Christe, lux vera, qui omnes hómines illúminas ad salútem, nobis, quǽsumus, concéde virtútem, ut ante te vias pacis et iustítiæ præparémus.

\noindent Qui vivis et regnas cum Deo Patre in unitáte Spíritus Sancti, Deus, per ómnia sǽcula sæculórum.

\noindent \Rbardot{} Amen.}
\fi
\fi

\hora{Ad Matutinum.} %%%%%%%%%%%%%%%%%%%%%%%%%%%%%%%%%%%%%%%%%%%%%%%%%%%%%

\vspace{2mm}

\cuminitiali{}{temporalia/dominelabiamea.gtex}

\vfill
%\pagebreak

\vspace{2mm}

\ifx\invitatorium\undefined
\ifx\matuac\undefined
\else
\pars{Invitatorium.} \scriptura{Ps. 94, 1; Psalmus 94; \textbf{H451}}

\vspace{-6mm}

\antiphona{VI}{temporalia/inv-jubilemusdeo.gtex}
\fi
\ifx\matubd\undefined
\else
\pars{Invitatorium.} \scriptura{Cantor; Psalmus 94; \textbf{H449}}

\vspace{-6mm}

\antiphona{E}{temporalia/inv-regemmagnum.gtex}
\fi
\else
\invitatorium
\fi

\vfill
\pagebreak

\ifx\hymnusmatutinum\undefined
\ifx\matuac\undefined
\else
\pars{Hymnus}

\cuminitiali{IV}{temporalia/hym-SomnoRefectis.gtex}
\fi
\ifx\matubd\undefined
\else
\pars{Hymnus.} \scriptura{Gregorius Magnus (\olddag{} 604)}

{
\grechangedim{interwordspacetext}{0.10 cm plus 0.15 cm minus 0.05 cm}{scalable}%
\antiphona{I}{temporalia/hym-NocteSurgentes.gtex}
\grechangedim{interwordspacetext}{0.22 cm plus 0.15 cm minus 0.05 cm}{scalable}%
}
\fi
\else
\hymnusmatutinum
\fi

\vspace{-3mm}

\vfill
\pagebreak

\ifx\matub\undefined
\else
% MAT B
\pars{Psalmus 1.} \scriptura{Ps. 36, 5; \textbf{H93}}

\vspace{-4mm}

\antiphona{VI F}{temporalia/ant-reveladomino.gtex}

%\vspace{-2mm}

\scriptura{Ps. 36, 1-11}

%\vspace{-2mm}

\initiumpsalmi{temporalia/ps36i_xi-initium-vi-F-auto.gtex}

\input{temporalia/ps36i_xi-vi-F.tex} \Abardot{}

\vfill
\pagebreak

\pars{Psalmus 2.}

\vspace{-4mm}

\antiphona{II D}{temporalia/ant-iuniorfui.gtex}

\vspace{-2mm}

\scriptura{Ps. 36, 12-29}

\vspace{-2mm}

\initiumpsalmi{temporalia/ps36xii_xxix-initium-ii-D-auto.gtex}

\input{temporalia/ps36xii_xxix-ii-D.tex}

\vfill

\antiphona{}{temporalia/ant-iuniorfui.gtex}

\vfill
\pagebreak

\pars{Psalmus 3.} \scriptura{Ps. 36, 3}

\vspace{-4mm}

\antiphona{VI F}{temporalia/ant-speraindomino.gtex}

%\vspace{-2mm}

\scriptura{Ps. 36, 30-40}

%\vspace{-2mm}

\initiumpsalmi{temporalia/ps36iii-initium-vi-F-auto.gtex}

\input{temporalia/ps36iii-vi-F.tex} \Abardot{}

\vfill
\pagebreak
\fi
\ifx\matuc\undefined
\else
% MAT C
\pars{Psalmus 1.} \scriptura{Ps. 67, 2}

\vspace{-4mm}

\antiphona{VII a}{temporalia/ant-exsurgatdeus.gtex}

%\vspace{-2mm}

\scriptura{Ps. 67, 2-11}

\initiumpsalmi{temporalia/ps67i-initium-vii-a-auto.gtex}

\input{temporalia/ps67i-vii-a.tex} \Abardot{}

\vfill
\pagebreak

\pars{Psalmus 2.}

\vspace{-4mm}

\antiphona{I f}{temporalia/ant-deusnosterdeussalvos.gtex}

%\vspace{-2mm}

\scriptura{Ps. 67, 12-24}

%\vspace{-2mm}

\initiumpsalmi{temporalia/ps67ii-initium-i-f-auto.gtex}

\input{temporalia/ps67ii-i-f.tex} \Abardot{}

\vfill
\pagebreak

\pars{Psalmus 3.} \scriptura{Ps. 67, 27; \textbf{H96}}

\vspace{-4mm}

\antiphona{D}{temporalia/ant-inecclesiis.gtex}

%\vspace{-2mm}

\scriptura{Ps. 67, 25-36}

\initiumpsalmi{temporalia/ps67iii-initium-d-g2-auto.gtex}

\input{temporalia/ps67iii-d-g2.tex} \Abardot{}

\vfill
\pagebreak
\fi

\pars{Versus.}

\ifx\matversus\undefined
\ifx\matub\undefined
\else
\noindent \Vbardot{} Bonitátem et prudéntiam et sciéntiam doce me.

\noindent \Rbardot{} Quia præcéptis tuis crédidi.
\fi
\ifx\matuc\undefined
\else
\noindent \Vbardot{} Audiam quid loquátur Dóminus Deus.

\noindent \Rbardot{} Loquétur pacem ad plebem suam.
\fi
\else
\matversus
\fi

\vspace{5mm}

\sineinitiali{temporalia/oratiodominica-mat.gtex}

\vspace{5mm}

\pars{Absolutio.}

\cuminitiali{}{temporalia/absolutio-ipsius.gtex}

\vfill
\pagebreak

\cuminitiali{}{temporalia/benedictio-solemn-deus.gtex}

\vspace{7mm}

\lectioi

\noindent \Vbardot{} Tu autem, Dómine, miserére nobis.
\noindent \Rbardot{} Deo grátias.

\vfill
\pagebreak

\responsoriumi

\vfill
\pagebreak

\cuminitiali{}{temporalia/benedictio-solemn-christus.gtex}

\vspace{7mm}

\lectioii

\noindent \Vbardot{} Tu autem, Dómine, miserére nobis.
\noindent \Rbardot{} Deo grátias.

\vfill
\pagebreak

\responsoriumii

\vfill
\pagebreak

\cuminitiali{}{temporalia/benedictio-solemn-ignem.gtex}

\vspace{7mm}

\lectioiii

\noindent \Vbardot{} Tu autem, Dómine, miserére nobis.
\noindent \Rbardot{} Deo grátias.

\vfill
\pagebreak

\responsoriumiii

\vfill
\pagebreak

\rubrica{Reliqua omittuntur, nisi Laudes separandæ sint.}

\sineinitiali{temporalia/domineexaudi.gtex}

\vfill

\oratio

\vfill

\noindent \Vbardot{} Dómine, exáudi oratiónem meam.
\Rbardot{} Et clamor meus ad te véniat.

\vfill

\noindent \Vbardot{} Benedicámus Dómino.
\noindent \Rbardot{} Deo grátias.

\vfill

\noindent \Vbardot{} Fidélium ánimæ per misericórdiam Dei requiéscant in pace.
\Rbardot{} Amen.

\vfill
\pagebreak

\hora{Ad Laudes.} %%%%%%%%%%%%%%%%%%%%%%%%%%%%%%%%%%%%%%%%%%%%%%%%%%%%%

\cantusSineNeumas

\vspace{0.5cm}
\grechangedim{interwordspacetext}{0.18 cm plus 0.15 cm minus 0.05 cm}{scalable}%
\cuminitiali{}{temporalia/deusinadiutorium-communis.gtex}
\grechangedim{interwordspacetext}{0.22 cm plus 0.15 cm minus 0.05 cm}{scalable}%

\vfill
\pagebreak

\ifx\hymnuslaudes\undefined
\ifx\laudac\undefined
\else
\pars{Hymnus} \scriptura{Ambrosius (\olddag{} 397)}

\cuminitiali{I}{temporalia/hym-SplendorPaternae-hiemalis.gtex}
\fi
\ifx\laudbd\undefined
\else
\pars{Hymnus}

\grechangedim{interwordspacetext}{0.16 cm plus 0.15 cm minus 0.05 cm}{scalable}%
\cuminitiali{IV}{temporalia/hym-AEterneLucis.gtex}
\grechangedim{interwordspacetext}{0.22 cm plus 0.15 cm minus 0.05 cm}{scalable}%
\vspace{-3mm}
\fi
\else
\hymnuslaudes
\fi

\vfill
\pagebreak

\ifx\laudb\undefined
\else
\pars{Psalmus 1.} \scriptura{Ps. 42, 5; \textbf{H95}}

\vspace{-4mm}

\antiphona{VI F}{temporalia/ant-salutarevultusmei.gtex}

\scriptura{Psalmus 42.}

\initiumpsalmi{temporalia/ps42-initium-vi-F-auto.gtex}

\input{temporalia/ps42-vi-F.tex} \Abardot{}

\vfill
\pagebreak

\pars{Psalmus 2.} \scriptura{Is. 38, 20; \textbf{H95}}

\vspace{-7mm}

\antiphona{E}{temporalia/ant-cunctisdiebus.gtex}

\vspace{-4mm}

\scriptura{Canticum Ezechiæ, Is. 38, 10-20}

\vspace{-3mm}

\initiumpsalmi{temporalia/ezechiae-initium-e-auto.gtex}

\input{temporalia/ezechiae-e.tex} \Abardot{}

\vfill
\pagebreak

\pars{Psalmus 3.} \scriptura{Ps. 64, 2; \textbf{H96}}

\vspace{-4mm}

\antiphona{VIII a}{temporalia/ant-tedecet.gtex}

\vspace{-2mm}

\scriptura{Psalmus 64.}

\vspace{-2mm}

\initiumpsalmi{temporalia/ps64-initium-viii-A-auto.gtex}

\input{temporalia/ps64-viii-A.tex} \Abardot{}

\vfill
\pagebreak
\fi
\ifx\laudc\undefined
\else
\pars{Psalmus 1.} \scriptura{Ps. 83, 5}

\vspace{-4mm}

\antiphona{VIII G}{temporalia/ant-beatiquihabitant.gtex}

\vspace{-2mm}

\scriptura{Psalmus 84.}

\vspace{-2mm}

\initiumpsalmi{temporalia/ps84-initium-viii-G-auto.gtex}

\input{temporalia/ps84-viii-G.tex} \Abardot{}

\vfill
\pagebreak

\pars{Psalmus 2.}

\vspace{-4mm}

\antiphona{VII d}{temporalia/ant-denoctespiritusmeus.gtex}

\vspace{-2mm}

\scriptura{Canticum Isaiæ, Is. 26, 1-12}

\vspace{-2mm}

\initiumpsalmi{temporalia/isaiae3-initium-vii-d.gtex}

\input{temporalia/isaiae3-vii-d.tex} \Abardot{}

\vfill
\pagebreak

\pars{Psalmus 3.} \scriptura{Ps. 66, 2}

\vspace{-4mm}

\antiphona{E}{temporalia/ant-illuminadomine.gtex}

%\vspace{-2mm}

\scriptura{Psalmus 66.}

%\vspace{-2mm}

\initiumpsalmi{temporalia/ps66-initium-e.gtex}

\input{temporalia/ps66-e.tex} \Abardot{}

\vfill
\pagebreak
\fi

\ifx\lectiobrevis\undefined
\ifx\laudb\undefined
\else
\pars{Lectio Brevis.} \scriptura{1 Th. 5, 4-5}

\noindent Vos, fratres, non estis in ténebris, ut vos dies ille tamquam fur comprehéndat; omnes enim vos fílii lucis estis et fílii diéi. Non sumus noctis neque tenebrárum.
\fi
\ifx\laudc\undefined
\else
\pars{Lectio Brevis.} \scriptura{1 Io. 4, 14-15}

\noindent Nos vídimus et testificámur quóniam Pater misit Fílium salvatórem mundi. Quisque conféssus fúerit: Iesus est Fílius Dei, Deus in ipso manet, et ipse in Deo.
\fi
\else
\lectiobrevis
\fi

\vfill

\ifx\responsoriumbreve\undefined
\ifx\laudac\undefined
\else
\pars{Responsorium breve.}

\cuminitiali{VI}{temporalia/resp-benedictusdominus.gtex}
\fi
\ifx\laudbd\undefined
\else
\pars{Responsorium breve.} \scriptura{Ps. 118, 149.147}

\cuminitiali{VI}{temporalia/resp-vocemmeamaudi.gtex}
\fi
\else
\responsoriumbreve
\fi

\vfill
\pagebreak

\ifx\benedictus\undefined
\ifx\laudbd\undefined
\else
\pars{Canticum Zachariæ.} \scriptura{Lc. 1, 71; \textbf{H423}}

\vspace{-5mm}

{
\grechangedim{interwordspacetext}{0.18 cm plus 0.15 cm minus 0.05 cm}{scalable}%
\antiphona{I g\textsuperscript{5}}{temporalia/ant-demanuomnium.gtex}
\grechangedim{interwordspacetext}{0.22 cm plus 0.15 cm minus 0.05 cm}{scalable}%
}

%\vspace{-3mm}

\scriptura{Lc. 1, 68-79}

%\vspace{-1mm}

\initiumpsalmi{temporalia/benedictus-initium-i-g5-auto.gtex}

\input{temporalia/benedictus-i-g5.tex} \Abardot{}
\fi
\else
\benedictus
\fi

\vspace{-1cm}

\vfill
\pagebreak

\pars{Preces.}

\sineinitiali{}{temporalia/tonusprecum.gtex}

\ifx\preces\undefined
\ifx\laudb\undefined
\else
\noindent Salvatóri nostro benedicámus, qui sua resurrectióne mundum clarificávit, \gredagger{} et humíliter invocémus eum dicéntes:

\Rbardot{} Salva nos, Dómine, in sémita tua.

\noindent Resurrectiónem tuam, Dómine Iesu, oratióne cólimus matutína, \gredagger{} spes glóriæ tuæ diem nostrum illúminet.

\Rbardot{} Salva nos, Dómine, in sémita tua.

\noindent Súscipe, Dómine, vota et propósita nostra, \gredagger{} tamquam diéi nostri primítias.

\Rbardot{} Salva nos, Dómine, in sémita tua.

\noindent Tríbue in dilectióne tua nos hódie profícere, \gredagger{} ut ómnia in nostrum omniúmque bonum cooperéntur.

\Rbardot{} Salva nos, Dómine, in sémita tua.

\noindent Da, Dómine, sic lucére lucem nostram coram homínibus, \gredagger{} ut vídeant ópera nostra bona et Patrem gloríficent.

\Rbardot{} Salva nos, Dómine, in sémita tua.
\fi
\else
\preces
\fi

\vfill

\pars{Oratio Dominica.}

\cuminitiali{}{temporalia/oratiodominicaalt.gtex}

\vfill
\pagebreak

\rubrica{vel:}

\pars{Supplicatio Litaniæ.}

\cuminitiali{}{temporalia/supplicatiolitaniae.gtex}

\vfill

\pars{Oratio Dominica.}

\cuminitiali{}{temporalia/oratiodominica.gtex}

\vfill
\pagebreak

% Oratio. %%%
\oratio

\vspace{-1mm}

\vfill

\rubrica{Hebdomadarius dicit Dominus vobiscum, vel, absente sacerdote vel diacono, sic concluditur:}

\vspace{2mm}

\antiphona{C}{temporalia/dominusnosbenedicat.gtex}

\rubrica{Postea cantatur a cantore:}

\vspace{2mm}

\cuminitiali{IV}{temporalia/benedicamus-feria-laudes.gtex}

\vspace{1mm}

\vfill
\pagebreak

\end{document}

