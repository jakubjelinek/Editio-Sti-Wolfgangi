\newcommand{\titulus}{\nomenFesti{S. Thomæ, Apostoli.}
\dies{Die 3. Iulii.}}
\newcommand{\sineobmv}{Sine officium B.M.V.}
\newcommand{\oratio}{\pars{Oratio.}

\noindent Da nobis, omnípotens Deus, beáti Thomæ apóstoli festivitáte gloriári, ut eius semper et patrocíniis sublevémur et vitam credéntes habeámus in nómine eius, quem ipse Dóminum agnóvit, Iesum Christum Fílium tuum.

\noindent Per Dóminum nostrum Iesum Christum, Fílium tuum, qui tecum vivit et regnat in unitáte Spíritus Sancti, Deus, per ómnia sǽcula sæculórum.

\noindent \Rbardot{} Amen.}
\newcommand{\invitatorium}{\pars{Invitatorium.} \scriptura{Cantor; \textbf{H443}}

\vspace{-6mm}

\antiphona{III}{temporalia/inv-regemapostolorum.gtex}}
\newcommand{\hymnusmatutinum}{\pars{Hymnus}

\cuminitiali{IV}{temporalia/hym-ExsultetCaelum.gtex}}
\newcommand{\matutinum}{\pars{Psalmus 1.} \scriptura{Ps. 18, 5; \textbf{H360}}

\vspace{-4mm}

\antiphona{II D}{temporalia/ant-inomnemterram-FKP.gtex}

%\vspace{-2mm}

\scriptura{Ps. 18}

%\vspace{-2mm}

\initiumpsalmi{temporalia/ps18-initium-ii-D-auto.gtex}

\input{temporalia/ps18-ii-D.tex}

\vfill

\antiphona{}{temporalia/ant-inomnemterram-FKP.gtex}

\vfill
\pagebreak

\pars{Psalmus 2.} \scriptura{Ps. 63, 10; \textbf{H361}}

\vspace{-4mm}

\antiphona{VIII G}{temporalia/ant-anuntiaverunt-FKP.gtex}

%\vspace{-2mm}

\scriptura{Ps. 63}

\initiumpsalmi{temporalia/ps63-initium-viii-G-auto.gtex}

\input{temporalia/ps63-viii-G.tex} \Abardot{}

\vfill
\pagebreak

\pars{Psalmus 3.} \scriptura{Ps. 96, 11; \textbf{H361}}

\vspace{-4mm}

\antiphona{VI F}{temporalia/ant-luxortaestiustis-FKP.gtex}

%\vspace{-2mm}

\scriptura{Ps. 96}

%\vspace{-2mm}

\initiumpsalmi{temporalia/ps96-initium-vi-F-auto.gtex}

%\vspace{-1mm}

\input{temporalia/ps96-vi-F.tex} \Abardot{}

\vfill
\pagebreak}
\newcommand{\matversus}{\noindent \Vbardot{} In omnem terram exívit sonus eórum.

\noindent \Rbardot{} Et in fines orbis terræ verba eórum.}
\newcommand{\lectioi}{\pars{Lectio I.} \scriptura{1 Cor. 4, 1-16}

\noindent De Epístola prima beáti Pauli apóstoli ad Corínthios.

\noindent Fratres: Sic nos exístimet homo ut minístros Christi et dispensatóres mysteriórum Dei. Hic iam quǽritur inter dispensatóres, ut fidélis quis inveniátur. Mihi autem pro mínimo est, ut a vobis iúdicer aut ab humáno die. Sed neque meípsum iúdico; nihil enim mihi cónscius sum, sed non in hoc iustificátus sum. Qui autem iúdicat me, Dóminus est! Itaque nolíte ante tempus quidquam iudicáre, quoadúsque véniat Dóminus, qui et illuminábit abscóndita tenebrárum et manifestábit consília córdium; et tunc laus erit unicuíque a Deo.

\noindent Hæc autem, fratres, transfigurávi in me et Apóllo propter vos, ut in nobis discátis illud: «Ne supra quæ scripta sunt», ne unus pro álio inflémini advérsus álterum. Quis enim te discérnit? Quid autem habes, quod non accepísti? Si autem accepísti, quid gloriáris, quasi non accéperis?

\noindent Iam saturáti estis, iam dívites facti estis. Sine nobis regnástis; et útinam regnarétis, ut et nos vobíscum regnarémus. Puto enim, Deus nos apóstolos novíssimos osténdit tamquam morti destinátos, quia spectáculum facti sumus mundo et ángelis et homínibus. Nos stulti propter Christum, vos autem prudéntes in Christo; nos infírmi, vos autem fortes; vos gloriósi, nos autem ignóbiles. Usque in hanc horam et esurímus et sitímus et nudi sumus et cólaphis cǽdimur et instábiles sumus et laborámus operántes mánibus nostris; maledícti benedícimus, persecutiónem passi sustinémus, blasphemáti obsecrámus; tamquam purgaménta mundi facti sumus, ómnium peripséma, usque adhuc.

\noindent Non ut confúndam vos, hæc scribo, sed ut quasi fílios meos caríssimos móneam; nam si decem mília pædagogórum habeátis in Christo, sed non multos patres, nam in Christo Iesu per evangélium ego vos génui. Rogo ergo vos: imitatóres mei estóte!}
\newcommand{\responsoriumi}{\pars{Responsorium 1.} \scriptura{\Rbardot{} Mt. 10, 10 \Vbardot{} Io. 12, 36; \textbf{H360}}

\vspace{-5mm}

\responsorium{VII}{temporalia/resp-ecceegomittovos.gtex}{}}
\newcommand{\lectioii}{\pars{Lectio II.} \scriptura{Hom. 26, 7-9: PL 76, 1201-1202}

\noindent Ex Homíliis sancti Gregórii Magni papæ in Evangélia.

\noindent Thomas autem unus ex duódecim, qui dícitur Dídymus, non erat cum eis quando venit Iesus. Iste unus discípulus défuit, revérsus quod gestum est audívit, audíta crédere rénuit. Venit íterum Dóminus, et non credénti discípulo latus palpándum prǽbuit, manus osténdit, et, osténsa suórum cicatríce vúlnerum, infidelitátis illíus vulnus sanávit. Quid, fratres caríssimi, quid inter hæc animadvértitis? Numquid casu gestum créditis, ut eléctus ille discípulus tunc deésset, post autem véniens audíret, áudiens dubitáret, dúbitans palpáret, palpans créderet?

\noindent Non hoc casu, sed divína dispensatióne gestum est. Egit namque miro modo supérna cleméntia ut discípulus ille dúbitans, dum in magístro suo vúlnera palpáret carnis, in nobis vúlnera sanáret infidelitátis. Plus enim nobis Thomæ infidélitas ad fidem quam fides credéntium discipulórum prófuit, quia dum ille ad fidem palpándo redúcitur, nostra mens, omni dubitatióne postpósita, in fide solidátur. Nam ita factus est discípulus dúbitans et palpans, testis veræ resurrectiónis.}
\newcommand{\responsoriumii}{\pars{Responsorium 2.} \scriptura{\Rbardot{} Mt. 11, 29-30; \textbf{H361}}

\vspace{-5mm}

\responsorium{VII}{temporalia/resp-tollite.gtex}}
\newcommand{\lectioiii}{\pars{Lectio III.}

\noindent Palpávit autem, et exclamávit: Dóminus meus et Deus meus. Dicit ei Iesus: Quia vidísti me, Thoma, credidísti. Cum Paulus apóstolus dicat: Est autem fides sperandárum substántia rerum, arguméntum non apparéntium, profécto liquet quia fides illárum rerum arguméntum est quæ apparére non possunt. Quæ étenim appárent iam fidem non habent, sed agnitiónem. Dum ergo vidit Thomas, dum palpávit, cur ei dícitur: Quia vidísti me, credidísti? Sed áliud vidit, áliud crédidit. A mortáli quippe hómine divínitas vidéri non pótuit. Hóminem ergo vidit, et Deum conféssus est, dicens: Dóminus meus et Deus meus. Vidéndo ergo crédidit, qui considerándo verum hóminem, hunc Deum, quem vidére non póterat, exclamávit.

\noindent Lætíficat valde quod séquitur: Beáti qui non vidérunt et credidérunt. In qua nimírum senténtia nos speciáliter signáti sumus, qui eum quem carne non vídimus, mente retinémus. Nos signáti sumus, sed si fidem nostram opéribus séquimur. Ille étenim vere credit, qui exércet operándo quod credit. Quo contra de his qui fidem nómine tenus rétinent Paulus dicit: Confiténtur se nosse Deum, factis autem negant. Hinc Iacóbus ait: Fides sine opéribus mórtua est.}
\newcommand{\responsoriumiii}{\pars{Responsorium 3.} \scriptura{Mt. 10, 18; \textbf{H361}}

\vspace{-5mm}

\responsorium{III}{temporalia/resp-dumsteteritis.gtex}{}

\vfill
\pagebreak

\pars{Hymnus Ambrosianus} \scriptura{Alio modo, iuxta morem Romanum}

\vspace{-2mm}

{
\grechangedim{interwordspacetext}{0.26 cm plus 0.15 cm minus 0.05 cm}{scalable}%
\cuminitiali{III}{temporalia/tedeum-romanum-gn.gtex}
\grechangedim{interwordspacetext}{0.22 cm plus 0.15 cm minus 0.05 cm}{scalable}%
}}
\newcommand{\hymnuslaudes}{\pars{Hymnus}

\cuminitiali{IV}{temporalia/hym-QuiLuce.gtex}}
\newcommand{\laudes}{\pars{Psalmus 1.} \scriptura{Lc. 24, 36; \textbf{H235}}

\vspace{-4mm}

\antiphona{VIII G\textsuperscript{5}}{temporalia/ant-stetitiesus.gtex}

%\vspace{-2mm}

\scriptura{Psalmus 62}

%\vspace{-2mm}

\initiumpsalmi{temporalia/ps62-initium-viii-G6-auto.gtex}

%\vspace{-1.5mm}

\input{temporalia/ps62-viii-G6.tex} \Abardot{}

\vfill
\pagebreak

\pars{Psalmus 2.} \scriptura{Io. 20, 24; \textbf{H239}}

\vspace{-4mm}

\antiphona{I f}{temporalia/ant-thomasquidicitur.gtex}

%\vspace{-2mm}

\scriptura{Canticum trium puerorum, Dan. 3, 57-88 et 56}

\initiumpsalmi{temporalia/dan3-initium-i-f-auto.gtex}

\input{temporalia/dan3-i-f-sinedox.tex}

\rubrica{Hic non dicitur Gloria Patri, neque Amen.}

\vfill

\antiphona{}{temporalia/ant-thomasquidicitur.gtex}

\vfill
\pagebreak

\pars{Psalmus 3.} \scriptura{Io. 20, 27; \textbf{H240}}

\vspace{-4mm}

\antiphona{VIII G\textsuperscript{2}}{temporalia/ant-mittemanumtuam.gtex}

%\vspace{-2mm}

\scriptura{Psalmus 149}

%\vspace{-2mm}

\initiumpsalmi{temporalia/ps149-initium-viii-G5-auto.gtex}

\input{temporalia/ps149-viii-G5.tex} \Abardot{}

\vfill
\pagebreak}
\newcommand{\lectiobrevis}{\pars{Lectio Brevis.} \scriptura{Eph. 2, 19-22}

\noindent Iam non estis extránei et ádvenæ, sed estis concíves sanctórum et doméstici Dei, superædificáti super fundaméntum apostolórum et prophetárum, ipso summo angulári lápide Christo Iesu, in quo omnis ædificátio compácta crescit in templum sanctum in Dómino, in quo et vos coædificámini in habitáculum Dei in Spíritu.}
\newcommand{\responsoriumbreve}{\pars{Responsorium breve.} \scriptura{Ps. 44, 17-18}

\cuminitiali{VI}{temporalia/resp-constitueseos.gtex}}
\newcommand{\preces}{\noindent Fratres caríssimi, hereditátem cæléstem ab Apóstolis habéntes, \gredagger{} grátias agámus Patri nostro pro ómnibus donis eius, clamántes:

\Rbardot{} Te laudat Apostolórum chorus, Dómine.

\noindent Laus tibi, Dómine, pro mensa Córporis et Sánguinis Christi, nobis ab Apóstolis trádita, \gredagger{} qua refícimur et vívimus:

\Rbardot{} Te laudat Apostolórum chorus, Dómine.

\noindent Pro mensa verbi tui, nobis ab Apóstolis paráta, \gredagger{} qua lumen et gáudium nobis dantur:

\Rbardot{} Te laudat Apostolórum chorus, Dómine.

\noindent Pro Ecclésia tua sancta, super Apóstolos ædificáta, \gredagger{} qua in unum concorporámur:

\Rbardot{} Te laudat Apostolórum chorus, Dómine.

\noindent Pro lavácro baptísmi et pæniténtiæ, Apóstolis concrédito, \gredagger{} quo ab ómnibus peccátis ablúimur:

\Rbardot{} Te laudat Apostolórum chorus, Dómine.}
\newcommand{\benedictus}{\pars{Canticum Zachariæ.} \scriptura{Io. 20, 29; \textbf{H240}}

\vspace{-4mm}

\antiphona{VIII G}{temporalia/ant-quiavidistime.gtex}

\vspace{-2mm}

\scriptura{Lc. 1, 68-79}

\vspace{-2mm}

\cantusSineNeumas
\initiumpsalmi{temporalia/benedictus-initium-viii-G-auto.gtex}

%\vspace{-1.5mm}

\input{temporalia/benedictus-viii-G.tex} \Abardot{}}
\newcommand{\hebdomada}{infra Hebdom. XIII per Annum.}
\newcommand{\matua}{Matutinum Hebdomadae A}
\newcommand{\matuac}{Matutinum Hebdomadae A vel C}
\newcommand{\lauda}{Laudes Hebdomadae A}
\newcommand{\laudac}{Laudes Hebdomadae A vel C}

% LuaLaTeX

\documentclass[a4paper, twoside, 12pt]{article}
\usepackage[latin]{babel}
%\usepackage[landscape, left=3cm, right=1.5cm, top=2cm, bottom=1cm]{geometry} % okraje stranky
%\usepackage[landscape, a4paper, mag=1166, truedimen, left=2cm, right=1.5cm, top=1.6cm, bottom=0.95cm]{geometry} % okraje stranky
\usepackage[landscape, a4paper, mag=1400, truedimen, left=0.5cm, right=0.5cm, top=0.5cm, bottom=0.5cm]{geometry} % okraje stranky

\usepackage{fontspec}
\setmainfont[FeatureFile={junicode.fea}, Ligatures={Common, TeX}, RawFeature=+fixi]{Junicode}
%\setmainfont{Junicode}

% shortcut for Junicode without ligatures (for the Czech texts)
\newfontfamily\nlfont[FeatureFile={junicode.fea}, Ligatures={Common, TeX}, RawFeature=+fixi]{Junicode}

% Hebrew font: http://scripts.sil.org/cms/scripts/page.php?site_id=nrsi&id=SILHebrUnic2
\newfontfamily\hebfont[Scale=1]{Ezra SIL}

\usepackage{multicol}
\usepackage{color}
\usepackage{lettrine}
\usepackage{fancyhdr}

% usual packages loading:
\usepackage{luatextra}
\usepackage{graphicx} % support the \includegraphics command and options
\usepackage{gregoriotex} % for gregorio score inclusion
\usepackage{gregoriosyms}
\usepackage{wrapfig} % figures wrapped by the text
\usepackage{parcolumns}
\usepackage[contents={},opacity=1,scale=1,color=black]{background}
\usepackage{tikzpagenodes}
\usepackage{calc}
\usepackage{longtable}
\usetikzlibrary{calc}

\setlength{\headheight}{14.5pt}

\input{conventuscommune.tex} % Often used macros

\newcommand{\annusEditionis}{2022}

\def\hebinitial#1{%
\leavevmode{\newbox\hebbox\setbox\hebbox\hbox{\hebfont{#1}\hskip 1mm}\kern -\wd\hebbox\hbox{\hebfont{#1}\hskip 1mm}}%
}

%%%% Vicekrat opakovane kousky

\newcommand{\anteOrationem}{
  \rubrica{Ante Orationem, cantatur a Superiore:}

  \pars{Supplicatio Litaniæ.}

  \cuminitiali{}{temporalia/supplicatiolitaniae.gtex}

  \pars{Oratio Dominica.}

  \cuminitiali{}{temporalia/oratiodominica.gtex}

  \rubrica{Deinde dicitur ab Hebdomadario:}

  \cuminitiali{}{temporalia/dominusvobiscum-solemnis.gtex}

  \rubrica{In choro monialium loco Dominus vobiscum dicitur:}

  \sineinitiali{temporalia/domineexaudi.gtex}
}

\setlength{\columnsep}{30pt} % prostor mezi sloupci

%%%%%%%%%%%%%%%%%%%%%%%%%%%%%%%%%%%%%%%%%%%%%%%%%%%%%%%%%%%%%%%%%%%%%%%%%%%%%%%%%%%%%%%%%%%%%%%%%%%%%%%%%%%%%
\begin{document}

% Here we set the space around the initial.
% Please report to http://home.gna.org/gregorio/gregoriotex/details for more details and options
\grechangedim{afterinitialshift}{2.2mm}{scalable}
\grechangedim{beforeinitialshift}{2.2mm}{scalable}
\grechangedim{interwordspacetext}{0.22 cm plus 0.15 cm minus 0.05 cm}{scalable}%
\grechangedim{annotationraise}{-0.2cm}{scalable}

% Here we set the initial font. Change 38 if you want a bigger initial.
% Emit the initials in red.
\grechangestyle{initial}{\color{red}\fontsize{38}{38}\selectfont}

\pagestyle{empty}

%%%% Titulni stranka
\begin{titulusOfficii}
\ifx\titulus\undefined
\nomenFesti{Sabbato \hebdomada{}}
\else
\titulus
\fi
\end{titulusOfficii}

\vfill

\begin{center}
%Ad usum et secundum consuetudines chori \guillemotright{}Conventus Choralis\guillemotleft.

%Editio Sancti Wolfgangi \annusEditionis
\end{center}

\scriptura{}

\pars{}

\pagebreak

\renewcommand{\headrulewidth}{0pt} % no horiz. rule at the header
\fancyhf{}
\pagestyle{fancy}

\cantusSineNeumas

\hora{Ad Matutinum.} %%%%%%%%%%%%%%%%%%%%%%%%%%%%%%%%%%%%%%%%%%%%%%%%%%%%%

\vspace{2mm}

\cuminitiali{}{temporalia/dominelabiamea.gtex}

\vfill
%\pagebreak

\vspace{2mm}

\ifx\invitatorium\undefined
\pars{Invitatorium.} \scriptura{Lc. 24, 34; Psalmus 94; \textbf{H232}}

\vspace{-4mm}

\antiphona{VI}{temporalia/inv-surrexitdominusvere.gtex}
\else
\invitatorium
\fi

\vfill
\pagebreak

\ifx\hymnusmatutinum\undefined
\pars{Hymnus.}

\cuminitiali{VIII}{temporalia/hym-LaetareCaelum.gtex}
\else
\hymnusmatutinum
\fi

\vspace{-3mm}

\vfill
\pagebreak

\ifx\matutinum\undefined
\ifx\matua\undefined
\else
% MAT A
\pars{Psalmus 1.}

\vspace{-4mm}

\antiphona{VIII G\textsuperscript{5}}{temporalia/ant-alleluia-turco15.gtex}

\vspace{-3mm}

\scriptura{Ps. 104, 1-15}

\vspace{-2mm}

\initiumpsalmi{temporalia/ps104i-initium-viii-g5.gtex}

\vspace{-1.5mm}

\input{temporalia/ps104i-viii-g.tex}

\vfill
\pagebreak

\pars{Psalmus 2.} \scriptura{Ps. 104, 16-27}

%\vspace{-2mm}

\initiumpsalmi{temporalia/ps104ii-initium-viii-g5.gtex}

\input{temporalia/ps104ii-viii-g.tex}

\vfill
\pagebreak

\pars{Psalmus 3.} \scriptura{Ps. 104, 28-45}

%\vspace{-2mm}

\initiumpsalmi{temporalia/ps104iii-initium-viii-g5.gtex}

\input{temporalia/ps104iii-viii-g.tex}

\vfill

\antiphona{}{temporalia/ant-alleluia-turco15.gtex}

\vfill
\pagebreak
\fi
\ifx\matub\undefined
\else
% MAT B
\pars{Psalmus 1.}

\vspace{-4mm}

\antiphona{t. pereg.}{temporalia/ant-alleluia-turco3.gtex}

%\vspace{-2mm}

\scriptura{Ps. 105, 1-15}

%\vspace{-2mm}

\initiumpsalmi{temporalia/ps105i-initium-per-auto.gtex}

\input{temporalia/ps105i-per.tex}

\vfill
\pagebreak

\pars{Psalmus 2.} \scriptura{Ps. 105, 16-31}

\vspace{-2.5mm}

\initiumpsalmi{temporalia/ps105ii-initium-per-auto.gtex}

\vspace{-1.5mm}

\input{temporalia/ps105ii-per.tex}

\vfill
\pagebreak

\pars{Psalmus 3.} \scriptura{Ps. 105, 32-48}

%\vspace{-2mm}

\initiumpsalmi{temporalia/ps105iii-initium-per-auto.gtex}

\input{temporalia/ps105iii-per.tex}

\vfill

\antiphona{}{temporalia/ant-alleluia-turco3.gtex}

\vfill
\pagebreak
\fi
\ifx\matuc\undefined
\else
% MAT C
\pars{Psalmus 1.} \scriptura{Ps. 106, 8}

\vspace{-4mm}

\antiphona{IV e}{temporalia/ant-alleluia-fo2.gtex}

%\vspace{-2mm}

\scriptura{Ps. 106, 1-14}

%\vspace{-2mm}

\initiumpsalmi{temporalia/ps106i-initium-iv-e2-auto.gtex}

\input{temporalia/ps106i-iv-e2.tex}

\vfill
\pagebreak

\pars{Psalmus 2.} \scriptura{Ps. 106, 15-30}

%\vspace{-2mm}

\initiumpsalmi{temporalia/ps106ii-initium-iv-e2-auto.gtex}

\input{temporalia/ps106ii-iv-e2.tex}

\vfill
\pagebreak

\pars{Psalmus 3.} \scriptura{Ps. 106, 31-43}

%\vspace{-2mm}

\initiumpsalmi{temporalia/ps106iii-initium-iv-e2-auto.gtex}

\input{temporalia/ps106iii-iv-e2.tex}

\vfill
\pagebreak

\antiphona{}{temporalia/ant-alleluia-fo2.gtex}

\vfill
\pagebreak
\fi
\ifx\matud\undefined
\else
% MAT D
\pars{Psalmus 1.}

\vspace{-4mm}

\antiphona{III g}{temporalia/ant-alleluia-turco26.gtex}

%\vspace{-2mm}

\scriptura{Ps. 77, 40-51}

%\vspace{-2mm}

\initiumpsalmi{temporalia/ps77xl_li-initium-iii-g-auto.gtex}

\input{temporalia/ps77xl_li-iii-g.tex}

\vfill
\pagebreak

\pars{Psalmus 2.} \scriptura{Ps. 77, 52-64}

\vspace{-2mm}

\initiumpsalmi{temporalia/ps77lii_lxiv-initium-iii-g-auto.gtex}

\input{temporalia/ps77lii_lxiv-iii-g.tex}

\vfill
\pagebreak

\pars{Psalmus 3.} \scriptura{Ps. 77, 65-72}

%\vspace{-2mm}

\initiumpsalmi{temporalia/ps77lxv_lxxii-initium-iii-g-auto.gtex}

\input{temporalia/ps77lxv_lxxii-iii-g.tex}

\vfill

\antiphona{}{temporalia/ant-alleluia-turco26.gtex}

\vfill
\pagebreak
\fi
\else
\matutinum
\fi

\pars{Versus.}

\ifx\matversus\undefined
\noindent \Vbardot{} Deus regenerávit nos in spem vivam, allelúia.

\noindent \Rbardot{} Per resurrectiónem Iesu Christi ex mórtuis, allelúia.
\else
\matversus
\fi

\vspace{5mm}

\sineinitiali{temporalia/oratiodominica-mat.gtex}

\vspace{5mm}

\pars{Absolutio.}

\cuminitiali{}{temporalia/absolutio-avinculis.gtex}

\vfill
\pagebreak

\cuminitiali{}{temporalia/benedictio-solemn-ille.gtex}

\vspace{7mm}

\lectioi

\noindent \Vbardot{} Tu autem, Dómine, miserére nobis.
\noindent \Rbardot{} Deo grátias.

\vfill
\pagebreak

\responsoriumi

\vfill
\pagebreak

\cuminitiali{}{temporalia/benedictio-solemn-divinum.gtex}

\vspace{7mm}

\lectioii

\noindent \Vbardot{} Tu autem, Dómine, miserére nobis.
\noindent \Rbardot{} Deo grátias.

\vfill
\pagebreak

\responsoriumii

\vfill
\pagebreak

\cuminitiali{}{temporalia/benedictio-solemn-adsocietatem.gtex}

\vspace{7mm}

\lectioiii

\noindent \Vbardot{} Tu autem, Dómine, miserére nobis.
\noindent \Rbardot{} Deo grátias.

\vfill
\pagebreak

\responsoriumiii

\vfill
\pagebreak

\rubrica{Reliqua omittuntur, nisi Laudes separandæ sint.}

\sineinitiali{temporalia/domineexaudi.gtex}

\vfill

\oratio

\vfill

\noindent \Vbardot{} Dómine, exáudi oratiónem meam.
\Rbardot{} Et clamor meus ad te véniat.

\vfill

\noindent \Vbardot{} Benedicámus Dómino.
\noindent \Rbardot{} Deo grátias.

\vfill

\noindent \Vbardot{} Fidélium ánimæ per misericórdiam Dei requiéscant in pace.
\Rbardot{} Amen.

\vfill
\pagebreak

\hora{Ad Laudes.} %%%%%%%%%%%%%%%%%%%%%%%%%%%%%%%%%%%%%%%%%%%%%%%%%%%%%

\cantusSineNeumas

\vspace{0.5cm}
\grechangedim{interwordspacetext}{0.18 cm plus 0.15 cm minus 0.05 cm}{scalable}%
\cuminitiali{}{temporalia/deusinadiutorium-communis.gtex}
\grechangedim{interwordspacetext}{0.22 cm plus 0.15 cm minus 0.05 cm}{scalable}%

\vfill
\pagebreak

\ifx\hymnuslaudes\undefined
\ifx\laudac\undefined
\else
\pars{Hymnus}

\cuminitiali{I}{temporalia/hym-ChorusNovae-praglia.gtex}
\vspace{-3mm}
\fi
\ifx\laudbd\undefined
\else
\pars{Hymnus}

\cuminitiali{I}{temporalia/hym-ChorusNovae.gtex}
\vspace{-3mm}
\fi
\else
\hymnuslaudes
\fi

\vfill
\pagebreak

\ifx\laudes\undefined
\ifx\lauda\undefined
\else
\pars{Psalmus 1.}

\vspace{-4mm}

\antiphona{VII a}{temporalia/ant-alleluia-turco29.gtex}

\scriptura{Psalmus 118, 145-152; \hspace{5mm} \hebinitial{ק}}

\initiumpsalmi{temporalia/ps118xix-initium-vii-a-auto.gtex}

\input{temporalia/ps118xix-vii-a.tex} \Abardot{}

\vfill
\pagebreak

\pars{Psalmus 2.} \scriptura{Ex. 15, 2}

\vspace{-4mm}

\antiphona{IV e}{temporalia/ant-fortitudomeaetlausmea.gtex}

\scriptura{Canticum Moysis, Ex. 15, 1-4a.7b-13.17-19}

\initiumpsalmi{temporalia/moysis1-initium-iv-e2-auto.gtex}

\input{temporalia/moysis1-iv-e2.tex}

\antiphona{}{temporalia/ant-fortitudomeaetlausmea.gtex}

\vfill
\pagebreak

\pars{Psalmus 3.}

\vspace{-4mm}

\antiphona{E}{temporalia/ant-alleluia-praglia-e2.gtex}

\scriptura{Psalmus 116.}

\initiumpsalmi{temporalia/ps116-initium-e-auto.gtex}

\input{temporalia/ps116-e.tex} \Abardot{}

\vfill
\pagebreak
\fi
\ifx\laudb\undefined
\else
\pars{Psalmus 1.}

\vspace{-4.5mm}

\antiphona{E}{temporalia/ant-alleluia-praglia-e2.gtex}

\vspace{-3mm}

\scriptura{Psalmus 91.}

\vspace{-2mm}

\initiumpsalmi{temporalia/ps91-initium-e-auto.gtex}

\vspace{-1.5mm}

\input{temporalia/ps91-e.tex} \Abardot{}

\vfill
\pagebreak

\pars{Psalmus 2.} \scriptura{Eccli. 39, 19}

\vspace{-4mm}

\antiphona{VII c\textsuperscript{2}}{temporalia/ant-effrondeteingratia.gtex}

\vspace{-2mm}

\scriptura{Canticum Moysi, Dt. 32, 1-32}

\vspace{-2mm}

\initiumpsalmi{temporalia/moysis2i_xii-initium-vii-c2-auto.gtex}

\input{temporalia/moysis2i_xii-vii-c2.tex}

\vfill

\antiphona{}{temporalia/ant-effrondeteingratia.gtex}

\vfill
\pagebreak

\pars{Psalmus 3.}

\vspace{-4mm}

\antiphona{I a\textsuperscript{2}}{temporalia/ant-alleluia-turco23.gtex}

%\vspace{-2mm}

\scriptura{Ps. 8}

%\vspace{-2mm}

\initiumpsalmi{temporalia/ps8-initium-i-a2-auto.gtex}

\input{temporalia/ps8-i-a2.tex} \Abardot{}

\vfill
\pagebreak
\fi
\ifx\laudc\undefined
\else
\pars{Psalmus 1.}

\vspace{-4mm}

\antiphona{E}{temporalia/ant-alleluia-praglia-e2.gtex}

%\vspace{-2mm}

\scriptura{Psalmus 118, 145-152.}

%\vspace{-2mm}

\initiumpsalmi{temporalia/ps118xix-initium-e-auto.gtex}

%\vspace{-1.5mm}

\input{temporalia/ps118xix-e.tex} \Abardot{}

\vfill
\pagebreak

\pars{Psalmus 2.}

\vspace{-4mm}

\antiphona{V a}{temporalia/ant-mecumsitdomine-tp.gtex}

%\vspace{-2mm}

\scriptura{Canticum Sapientiæ, Sap. 9, 1-6.9-11}

\initiumpsalmi{temporalia/sapientia-initium-v-a-auto.gtex}

\input{temporalia/sapientia-v-a.tex} \Abardot{}

\vfill
\pagebreak

\pars{Psalmus 3.}

\vspace{-4mm}

\antiphona{II* a}{temporalia/ant-alleluia-turco18.gtex}

%\vspace{-2mm}

\scriptura{Ps. 116}

%\vspace{-2mm}

\initiumpsalmi{temporalia/ps116-initium-ii_-a-auto.gtex}

\input{temporalia/ps116-ii_-a.tex} \Abardot{}

\vfill
\pagebreak
\fi
\ifx\laudd\undefined
\else
\pars{Psalmus 1.}

\vspace{-4.5mm}

\antiphona{VIII G\textsuperscript{2}}{temporalia/ant-alleluia-turco12.gtex}

\vspace{-3mm}

\scriptura{Psalmus 91.}

\vspace{-2mm}

\initiumpsalmi{temporalia/ps91-initium-viii-G5-auto.gtex}

\vspace{-1.5mm}

\input{temporalia/ps91-viii-G5.tex} \Abardot{}

\vfill
\pagebreak

\pars{Psalmus 2.} \scriptura{Heb. 13, 8}

\vspace{-4mm}

\antiphona{II D}{temporalia/ant-iesuschristusheriethodie.gtex}

%\vspace{-2mm}

\scriptura{Canticum Ezechiæ, Ez. 36, 24-28}

\initiumpsalmi{temporalia/ezechiae2-initium-ii-D-auto.gtex}

\input{temporalia/ezechiae2-ii-D.tex} \Abardot{}

\vfill
\pagebreak

\pars{Psalmus 3.}

\vspace{-4mm}

\antiphona{I a\textsuperscript{2}}{temporalia/ant-alleluia-turco23.gtex}

%\vspace{-2mm}

\scriptura{Ps. 8}

%\vspace{-2mm}

\initiumpsalmi{temporalia/ps8-initium-i-a4-auto.gtex}

\input{temporalia/ps8-i-a4.tex} \Abardot{}

\vfill
\pagebreak
\fi
\else
\laudes
\fi

\ifx\lectiobrevis\undefined
\pars{Lectio Brevis.} \scriptura{Rom. 14, 7-9}

\noindent Nemo nostrum sibi vivit et nemo sibi móritur; sive enim vívimus, Dómino vívimus, sive mórimur, Dómino mórimur. Sive ergo vívimus, sive mórimur, Dómini sumus. In hoc enim Christus et mórtuus est et vixit, ut et mortuórum et vivórum dominétur.
\else
\lectiobrevis
\fi

\vfill

\ifx\responsoriumbreve\undefined
\pars{Responsorium breve.} \scriptura{Cf. Mt. 28, 6; Cf. Gal. 3, 13}

\cuminitiali{VI}{temporalia/resp-surrexitdominusdesepulcro.gtex}
\else
\responsoriumbreve
\fi

\vfill
\pagebreak

\benedictus

\vspace{-1cm}

\vfill
\pagebreak

\ifx\precestotum\undefined
\pars{Preces.}

\sineinitiali{}{temporalia/tonusprecumnovum.gtex}

\ifx\preces\undefined
\ifx\lauda\undefined
\else
\noindent Christum, panem vitæ, \gredagger{} qui mensa verbi et córporis sui fruéntes suscitábit in novíssimo die, \grestar{} læti deprecémur:

\Rbardot{} Da nobis, Dómine, pacem et gáudium.

\noindent Fili Dei, qui, suscitátus a mórtuis, princeps es vitæ, \grestar{} nos omnésque fratres tuos bénedic et sanctífica.

\Rbardot{} Da nobis, Dómine, pacem et gáudium.

\noindent Tu, qui pacem et gáudium ómnibus in te credéntibus largíris, \grestar{} da nos sicut fílios lucis ambuláre et de victória tua lætári.

\Rbardot{} Da nobis, Dómine, pacem et gáudium.

\noindent Adáuge fidem Ecclésiæ peregrinántis in terra, \grestar{} ut resurrectiónis tuæ testimónium mundo perhíbeat.

\Rbardot{} Da nobis, Dómine, pacem et gáudium.

\noindent Tu qui, multa passus, \gredagger{} in glóriam Patris intrásti, \grestar{} luctum mæréntium convérte in gáudium.

\Rbardot{} Da nobis, Dómine, pacem et gáudium.
\fi
\ifx\laudb\undefined
\else
\noindent Christum, qui vitam ætérnam nobis manifestávit, \grestar{} devóta mente rogémus, clamántes:

\Rbardot{} Resurréctio tua locuplétet nos grátia, Dómine.

\noindent Pastor ætérne, \gredagger{} réspice gregem tuum e somno surgéntem \grestar{} et pasce nos verbi et panis tui ubérrimo alimónio.

\Rbardot{} Resurréctio tua locuplétet nos grátia, Dómine.

\noindent Ne permíttas nos a lupo rapi vel a mercenário perdi, \grestar{} sed fac, ut vocem tuam fidéliter audiámus.

\Rbardot{} Resurréctio tua locuplétet nos grátia, Dómine.

\noindent Tu, qui cum prædicatóribus ubíque cooperáris eorúmque sermónem confírmas, \grestar{} fac, ut hódie resurrectiónem tuam móribus et vita proclamémus.

\Rbardot{} Resurréctio tua locuplétet nos grátia, Dómine.

\noindent Esto ipse gáudium nostrum, \gredagger{} quod nemo tollat a nobis, \grestar{} ut, reiécta tristítia peccáti, vitam appetámus ætérnam.

\Rbardot{} Resurréctio tua locuplétet nos grátia, Dómine.
\fi
\ifx\laudc\undefined
\else
\noindent Christum, panem vitæ, \gredagger{} qui mensa verbi et córporis sui fruéntes suscitábit in novíssimo die, \grestar{} læti deprecémur:

\Rbardot{} Da nobis, Dómine, pacem et gáudium.

\noindent Fili Dei, qui, suscitátus a mórtuis, princeps es vitæ, \grestar{} nos omnésque fratres tuos bénedic et sanctífica.

\Rbardot{} Da nobis, Dómine, pacem et gáudium.

\noindent Tu, qui pacem et gáudium ómnibus in te credéntibus largíris, \grestar{} da nos sicut fílios lucis ambuláre et de victória tua lætári.

\Rbardot{} Da nobis, Dómine, pacem et gáudium.

\noindent Adáuge fidem Ecclésiæ peregrinántis in terra, \grestar{} ut resurrectiónis tuæ testimónium mundo perhíbeat.

\Rbardot{} Da nobis, Dómine, pacem et gáudium.

\noindent Tu qui, multa passus, \gredagger{} in glóriam Patris intrásti, \grestar{} luctum mæréntium convérte in gáudium.

\Rbardot{} Da nobis, Dómine, pacem et gáudium.
\fi
\ifx\laudd\undefined
\else
\noindent Christum, qui vitam ætérnam nobis manifestávit, \grestar{} devóta mente rogémus, clamántes:

\Rbardot{} Resurréctio tua locuplétet nos grátia, Dómine.

\noindent Pastor ætérne, \gredagger{} réspice gregem tuum e somno surgéntem \grestar{} et pasce nos verbi et panis tui ubérrimo alimónio.

\Rbardot{} Resurréctio tua locuplétet nos grátia, Dómine.

\noindent Ne permíttas nos a lupo rapi vel a mercenário perdi, \grestar{} sed fac, ut vocem tuam fidéliter audiámus.

\Rbardot{} Resurréctio tua locuplétet nos grátia, Dómine.

\noindent Tu, qui cum prædicatóribus ubíque cooperáris eorúmque sermónem confírmas, \grestar{} fac, ut hódie resurrectiónem tuam móribus et vita proclamémus.

\Rbardot{} Resurréctio tua locuplétet nos grátia, Dómine.

\noindent Esto ipse gáudium nostrum, \gredagger{} quod nemo tollat a nobis, \grestar{} ut, reiécta tristítia peccáti, vitam appetámus ætérnam.

\Rbardot{} Resurréctio tua locuplétet nos grátia, Dómine.
\fi
\else
\preces
\fi

\vfill

\pars{Oratio Dominica.}

\cuminitiali{}{temporalia/oratiodominicaalt.gtex}

\vfill
\pagebreak

\rubrica{vel:}

\pars{Deprecatio Gelasii}

\vspace{-5mm}

\grechangedim{interwordspacetext}{0.16 cm plus 0.15 cm minus 0.05 cm}{scalable}%
\antiphona{D\textsuperscript{1}}{temporalia/deprecatio4-propace.gtex}
\grechangedim{interwordspacetext}{0.22 cm plus 0.15 cm minus 0.05 cm}{scalable}%

\vfill

\pars{Oratio Dominica.}

\cuminitiali{D}{temporalia/oratiodominica-d.gtex}
\else
\precestotum
\fi

\vfill
\pagebreak

% Oratio. %%%
\oratio

\vspace{-1mm}

\vfill

\rubrica{Hebdomadarius dicit Dominus vobiscum, vel, absente sacerdote vel diacono, sic concluditur:}

\vspace{2mm}

\ifx\dominusnosbenedicat\undefined
\antiphona{C}{temporalia/dominusnosbenedicat.gtex}
\else
\dominusnosbenedicat
\fi

\rubrica{Postea cantatur a cantore:}

\vspace{2mm}

\cuminitiali{VII}{temporalia/benedicamus-tempore-paschali.gtex}

\vspace{1mm}

\vfill
\pagebreak

\end{document}

