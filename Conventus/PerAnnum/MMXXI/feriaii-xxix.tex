\newcommand{\titulus}{\nomenFesti{S. Lucæ, Evangelistæ.}
\dies{Die 18. Octobris.}}
\newcommand{\oratio}{\pars{Oratio.}

\noindent Dómine Deus, qui beátum Lucam elegísti, ut prædicatióne et scriptis mystérium tuæ in páuperes dilectiónis reveláret, concéde, ut, qui tuo iam nómine gloriántur, cor unum et ánima una esse persevérent et omnes gentes tuam mereántur vidére salútem.

\noindent Per Dóminum nostrum Iesum Christum, Fílium tuum, qui tecum vivit et regnat in unitáte Spíritus Sancti, Deus, per ómnia sǽcula sæculórum.

\noindent \Rbardot{} Amen.}
\newcommand{\invitatorium}{\pars{Invitatorium.}

\antiphona{IV*}{temporalia/inv-dominumloquenteminevangelio.gtex}}
\newcommand{\hymnusmatutinum}{\pars{Hymnus.}

\antiphona{III}{temporalia/hym-OVirBeate.gtex}}
\newcommand{\matutinum}{\pars{Psalmus 1.} \scriptura{Ps. 18, 5; \textbf{H360}}

\vspace{-4mm}

\antiphona{II D}{temporalia/ant-inomnemterram-FKP.gtex}

%\vspace{-2mm}

\scriptura{Ps. 18}

%\vspace{-2mm}

\initiumpsalmi{temporalia/ps18-initium-ii-D-auto.gtex}

\input{temporalia/ps18-ii-D.tex}

\antiphona{}{temporalia/ant-inomnemterram-FKP.gtex}

\vfill
\pagebreak

\pars{Psalmus 2.} \scriptura{Ps. 63, 10; \textbf{H361}}

\vspace{-4mm}

\antiphona{VIII G}{temporalia/ant-anuntiaverunt-FKP.gtex}

%\vspace{-2mm}

\scriptura{Ps. 63}

\initiumpsalmi{temporalia/ps63-initium-viii-G-auto.gtex}

\input{temporalia/ps63-viii-G.tex} \Abardot{}

\vfill
\pagebreak

\pars{Psalmus 3.} \scriptura{Ps. 96, 11; \textbf{H361}}

\vspace{-4mm}

\antiphona{VI F}{temporalia/ant-luxortaestiustis-FKP.gtex}

%\vspace{-2mm}

\scriptura{Ps. 96}

%\vspace{-2mm}

\initiumpsalmi{temporalia/ps96-initium-vi-F-auto.gtex}

\input{temporalia/ps96-vi-F.tex} \Abardot{}

\vfill
\pagebreak}
\newcommand{\matversus}{\noindent \Vbardot{} In omnem terram exívit sonus eórum.

\noindent \Rbardot{} Et in fines orbis terræ verba eórum.}
\newcommand{\lectioi}{\pars{Lectio I.} \scriptura{Ac. 9, 27-31; 11, 19-26}

\noindent De Actibus Apostolórum.

\noindent In diébus illis: Bárnabas apprehénsum Saulum duxit ad apóstolos et narrávit illis quómodo in via vidísset Dóminum et quia locútus est ei et quómodo in Damásco fiduciáliter égerit in nómine Iesu. Et erat cum illis intrans et éxiens in Ierúsalem, fiduciáliter agens in nómine Dómini. Loquebátur quoque et disputábat cum Græcis; illi autem quærébant occídere eum. Quod cum cognovíssent, fratres deduxérunt eum Cæsaréam et dimisérunt Tarsum.

\noindent Ecclésia quidem per totam Iudǽam et Galilǽam et Samaríam habébat pacem, ædificabátur et ambulábat in timóre Dómini et consolatióne Sancti Spíritus crescébat.

\noindent Illi quidem, qui dispérsi fúerant a tribulatióne, quæ facta fúerat sub Stéphano, perambulavérunt usque Phœnícen et Cyprum et Antiochíam, némini loquéntes verbum nisi solis Iudǽis. Erant autem quidam ex eis viri Cýprii et Cyrenǽi, qui, cum introíssent Antiochíam, loquebántur et ad Græcos evangelizántes Dóminum Iesum. Et erat manus Dómini cum eis; multúsque númerus credéntium convérsus est ad Dóminum.

\noindent Audítus est autem sermo in áuribus ecclésiæ, quæ erat in Ierúsalem, super istis et misérunt Bárnabam usque Antiochíam; qui cum pervenísset et vidísset grátiam Dei, gavísus est et hortabátur omnes propósito cordis permanére in Dómino, quia erat vir bonus et plenus Spíritu Sancto et fide. Et appósita est turba multa Dómino. Proféctus est autem Tarsum, ut quǽreret Saulum; quem cum invenísset, perdúxit Antiochíam. Factum est autem eis ut annum totum conversaréntur in ecclésia et docérent turbam multam et cognominaréntur primum Antiochíæ discípuli Christiáni.}
\newcommand{\responsoriumi}{\pars{Responsorium 1.} \scriptura{\textbf{H361}}

\vspace{-5mm}

\responsorium{IV}{temporalia/resp-vidiconjunctos.gtex}{}}
\newcommand{\lectioii}{\pars{Lectio II.} \scriptura{Hom. 17, 1-3: PL 76, 1139}

\noindent Ex Homíliis sancti Gregórii Magni papæ in Evangélia.

\noindent Dóminus et Salvátor noster, fratres caríssimi, aliquándo nos sermónibus, aliquándo vero opéribus ádmonet. Ipsa étenim facta eius præcépta sunt, quia dum áliquid tácitus facit, quid ágere debeámus innotéscit. Ecce enim binos in prædicatiónem discípulos mittit, quia duo sunt præcépta caritátis, Dei vidélicet amor et próximi.

\noindent Binos ad prædicándum discípulos Dóminus mittit, quátenus hoc nobis tácitus ínnuat, quia qui caritátem erga álterum non habet, prædicatiónis offícium suscípere nullátenus debet.

\noindent Bene autem dícitur quia \emph{misit eos ante fáciem suam in omnem civitátem et locum, quo erat ipse ventúrus.} Prædicatóres enim suos Dóminus séquitur, quia prædicátio prǽvenit, et tunc ad mentis nostræ habitáculum Dóminus venit, quando verba exhortatiónis præcúrrunt, atque per hæc véritas in mente suscípitur. Hinc namque eísdem prædicatóribus Isaías dicit: \emph{Paráte viam Dómini, rectas fácite sémitas Dei nostri.} Hinc illis Psalmísta ait: \emph{Iter fácite ei qui ascéndit super occásum.}}
\newcommand{\responsoriumii}{\pars{Responsorium 2.} \scriptura{\Rbardot{} Mt. 10, 10 \Vbardot{} Io. 12, 36; \textbf{H360}}

\vspace{-5mm}

\responsorium{VII}{temporalia/resp-ecceegomittovos.gtex}{}}
\newcommand{\lectioiii}{\pars{Lectio III.}

\noindent Super occásum namque Dóminus ascéndit, quia unde in passióne occúbuit, inde maiórem suam glóriam resurgéndo manifestávit. Super occásum vidélicet ascéndit, quia mortem quam pértulit, resurgéndo calcávit. Ei ergo qui ascéndit super occásum iter fácimus, cum nos eius glóriam vestris méntibus prædicámus, ut eas et ipse post véniens, per amóris sui præséntiam illústret.

\noindent Missis autem prædicatóribus, quid dicat audiámus: \emph{Messis quidem multa, operárii autem pauci. Rogáte ergo Dóminum messis, ut mittat operários in messem suam.} Ad messem multam operárii pauci sunt, quod sine gravi mæróre loqui non póssumus, quia etsi sunt qui bona áudiant, desunt qui dicant. Ecce mundus sacerdótibus plenus est, sed tamen in messe Dei rarus valde invenítur operátor, quia offícium quidem sacerdotále suscépimus, sed opus offícii non implémus.

\noindent Sed pensáte, fratres caríssimi, pensáte quod dícitur: \emph{Rogáte Dóminum messis, ut mittat operários in messem suam.} Vos pro nobis pétite, ut digna vobis operári valeámus, ne ab exhortatióne lingua tórpeat, ne postquam prædicatiónis locum suscépimus, apud iustum iúdicem nostra nos tacitúrnitas addícat.}
\newcommand{\responsoriumiii}{\pars{Responsorium 3.} \scriptura{\Vbardot{} Heb. 11, 33; \textbf{H362}}

\vspace{-5mm}

\responsorium{VIII}{temporalia/resp-istisuntvirisancti.gtex}{}

\vfill
\pagebreak

\pars{Hymnus Ambrosianus} \scriptura{Alio modo, iuxta morem Romanum}

\vspace{-2mm}

{
\grechangedim{interwordspacetext}{0.26 cm plus 0.15 cm minus 0.05 cm}{scalable}%
\cuminitiali{III}{temporalia/tedeum-romanum-gn.gtex}
\grechangedim{interwordspacetext}{0.22 cm plus 0.15 cm minus 0.05 cm}{scalable}%
}}
\newcommand{\deusinadiutorium}{\grechangedim{interwordspacetext}{0.18 cm plus 0.15 cm minus 0.05 cm}{scalable}%
\cuminitiali{}{temporalia/deusinadiutorium-alter.gtex}
\grechangedim{interwordspacetext}{0.22 cm plus 0.15 cm minus 0.05 cm}{scalable}}
\newcommand{\hymnuslaudes}{\pars{Hymnus}

\cuminitiali{I}{temporalia/hym-IamNunc-Luca.gtex}}
\newcommand{\laudes}{\pars{Psalmus 1.} \scriptura{Cf. Prv. 3, 20}

\vspace{-4mm}

\antiphona{IV d}{temporalia/ant-sapientiadominievangelii.gtex}

\vspace{-2mm}

\scriptura{Psalmus 62.}

\vspace{-1mm}

\initiumpsalmi{temporalia/ps62-initium-iv-d-auto.gtex}

%\vspace{-1.5mm}

\input{temporalia/ps62-iv-d.tex} \Abardot{}

\vfill
\pagebreak

\pars{Psalmus 2.} \scriptura{\textbf{H333}}

\vspace{-4mm}

\antiphona{VIII G\textsuperscript{2}}{temporalia/ant-istisuntvirisanctiet.gtex}

%\vspace{-2mm}

\scriptura{Canticum trium puerorum, Dan. 3, 57-88 et 56}

\initiumpsalmi{temporalia/dan3-initium-viii-G5-auto.gtex}

\input{temporalia/dan3-viii-G5-sinedox.tex}

\rubrica{Hic non dicitur Gloria Patri, neque Amen.}

\vfill

\antiphona{}{temporalia/ant-istisuntvirisanctiet.gtex}

\vfill
\pagebreak

\pars{Psalmus 3.} \scriptura{Eccli. 39, 12}

\vspace{-4mm}

\antiphona{III a}{temporalia/ant-collaudabuntmultisapientiam.gtex}

%\vspace{-2mm}

\scriptura{Psalmus 149}

%\vspace{-2mm}

\initiumpsalmi{temporalia/ps149-initium-iii-a-auto.gtex}

\input{temporalia/ps149-iii-a.tex} \Abardot{}

\vfill
\pagebreak}
\newcommand{\lectiobrevis}{\pars{Lectio Brevis.} \scriptura{1 Cor. 15,1-2.3-4}

\noindent Notum vobis fácio, fratres, evangélium, quod evangelizávi vobis, quod et accepístis, in quo et statis, per quod et salvámini. Trádidi enim vobis in primis, quod et accépi, quóniam Christus mórtuus est pro peccátis nostris secúndum Scriptúras et quia sepúltus est et quia suscitátus est tértia die secúndum Scriptúras.}
\newcommand{\responsoriumbreve}{\pars{Responsorium breve.} \scriptura{Cf. Ps. 77, 4}

\cuminitiali{VI}{temporalia/resp-narraveruntlaudes.gtex}}
\newcommand{\preces}{\noindent Salvatórem nostrum, qui, mortem déstruens, illuminávit vitam et incorruptiónem per Evangélium, dignis láudibus celebrémus,~\gredagger{} humíliter deprecántes:

\Rbardot{} Ecclésiam tuam in fide et caritáte confírma.

\noindent Qui Ecclésiam tuam per sanctos et exímios doctóres mirabíliter illustrásti,~\gredagger{} fac, ut christiáni eódem semper lætificéntur splendóre.

\Rbardot{} Ecclésiam tuam in fide et caritáte confírma.

\noindent Qui, cum sancti te pastóres sicut Móyses orárent, pópuli peccáta dimisísti,~\gredagger{} per intercessiónem eórum Ecclésiam tuam contínua purificatióne sanctífica.

\Rbardot{} Ecclésiam tuam in fide et caritáte confírma.

\noindent Qui sanctos tuos unxísti in médio fratrum et Spíritum tuum in illos direxísti,~\gredagger{} reple Spíritu Sancto omnes pópuli tui rectóres.

\Rbardot{} Ecclésiam tuam in fide et caritáte confírma.

\noindent Qui pastórum sanctórum ipse posséssio exstitísti,~\gredagger{} tríbue nullum ex iis, quos sánguine acquisísti, sine te manére.

\Rbardot{} Ecclésiam tuam in fide et caritáte confírma.}
\newcommand{\benedictus}{\pars{Canticum Zachariæ.} \scriptura{Is. 40, 9}

\vspace{-4mm}

\antiphona{VIII G}{temporalia/ant-supermontemexcelsum.gtex}


\vspace{-3mm}

\scriptura{Lc. 1, 68-79}

\vspace{-2mm}

\cantusSineNeumas
\initiumpsalmi{temporalia/benedictus-initium-viiisoll-G-auto.gtex}

\vspace{-1.5mm}

\input{temporalia/benedictus-viiisoll-G.tex} \Abardot{}}
\newcommand{\benedicamuslaudes}{\cuminitiali{II}{temporalia/benedicamus-solemnism-laud.gtex}}
\include{hebdomadaxxix}
% LuaLaTeX

\documentclass[a4paper, twoside, 12pt]{article}
\usepackage[latin]{babel}
%\usepackage[landscape, left=3cm, right=1.5cm, top=2cm, bottom=1cm]{geometry} % okraje stranky
%\usepackage[landscape, a4paper, mag=1166, truedimen, left=2cm, right=1.5cm, top=1.6cm, bottom=0.95cm]{geometry} % okraje stranky
\usepackage[landscape, a4paper, mag=1400, truedimen, left=0.5cm, right=0.5cm, top=0.5cm, bottom=0.5cm]{geometry} % okraje stranky

\usepackage{fontspec}
\setmainfont[FeatureFile={junicode.fea}, Ligatures={Common, TeX}, RawFeature=+fixi]{Junicode}
%\setmainfont{Junicode}

% shortcut for Junicode without ligatures (for the Czech texts)
\newfontfamily\nlfont[FeatureFile={junicode.fea}, Ligatures={Common, TeX}, RawFeature=+fixi]{Junicode}

\usepackage{multicol}
\usepackage{color}
\usepackage{lettrine}
\usepackage{fancyhdr}

% usual packages loading:
\usepackage{luatextra}
\usepackage{graphicx} % support the \includegraphics command and options
\usepackage{gregoriotex} % for gregorio score inclusion
\usepackage{gregoriosyms}
\usepackage{wrapfig} % figures wrapped by the text
\usepackage{parcolumns}
\usepackage[contents={},opacity=1,scale=1,color=black]{background}
\usepackage{tikzpagenodes}
\usepackage{calc}
\usepackage{longtable}
\usetikzlibrary{calc}

\setlength{\headheight}{14.5pt}

\input{conventuscommune.tex} % Often used macros

\newcommand{\annusEditionis}{2021}

%%%% Vicekrat opakovane kousky

\newcommand{\anteOrationem}{
  \rubrica{Ante Orationem, cantatur a Superiore:}

  \pars{Supplicatio Litaniæ.}

  \cuminitiali{}{temporalia/supplicatiolitaniae.gtex}

  \pars{Oratio Dominica.}

  \cuminitiali{}{temporalia/oratiodominica.gtex}

  \rubrica{Deinde dicitur ab Hebdomadario:}

  \cuminitiali{}{temporalia/dominusvobiscum-solemnis.gtex}

  \rubrica{In choro monialium loco Dominus vobiscum dicitur:}

  \sineinitiali{temporalia/domineexaudi.gtex}
}

\setlength{\columnsep}{30pt} % prostor mezi sloupci

%%%%%%%%%%%%%%%%%%%%%%%%%%%%%%%%%%%%%%%%%%%%%%%%%%%%%%%%%%%%%%%%%%%%%%%%%%%%%%%%%%%%%%%%%%%%%%%%%%%%%%%%%%%%%
\begin{document}

% Here we set the space around the initial.
% Please report to http://home.gna.org/gregorio/gregoriotex/details for more details and options
\grechangedim{afterinitialshift}{2.2mm}{scalable}
\grechangedim{beforeinitialshift}{2.2mm}{scalable}
\grechangedim{interwordspacetext}{0.22 cm plus 0.15 cm minus 0.05 cm}{scalable}%
\grechangedim{annotationraise}{-0.2cm}{scalable}

% Here we set the initial font. Change 38 if you want a bigger initial.
% Emit the initials in red.
\grechangestyle{initial}{\color{red}\fontsize{38}{38}\selectfont}

\pagestyle{empty}

%%%% Titulni stranka
\begin{titulusOfficii}
\ifx\titulus\undefined
\nomenFesti{Feria II \hebdomada{}}
\else
\titulus
\fi
\end{titulusOfficii}

\vfill

\begin{center}
%Ad usum et secundum consuetudines chori \guillemotright{}Conventus Choralis\guillemotleft.

%Editio Sancti Wolfgangi \annusEditionis
\end{center}

\scriptura{}

\pars{}

\pagebreak

\renewcommand{\headrulewidth}{0pt} % no horiz. rule at the header
\fancyhf{}
\pagestyle{fancy}

\cantusSineNeumas

\ifx\oratio\undefined
\ifx\laudb\undefined
\else
\newcommand{\oratio}{\pars{Oratio.}

\noindent Dómine Deus omnípotens, qui ad princípium huius diéi nos perveníre fecísti, tua nos hódie salva virtúte, ut in hac die ad nullum declinémus peccátum, sed semper ad tuam iustítiam faciéndam nostra procédant elóquia, dirigántur cogitatiónes et ópera.

\noindent Per Dóminum nostrum Iesum Christum, Fílium tuum, qui tecum vivit et regnat in unitáte Spíritus Sancti, Deus, per ómnia sǽcula sæculórum.

\noindent \Rbardot{} Amen.}
\fi
\fi

\hora{Ad Matutinum.} %%%%%%%%%%%%%%%%%%%%%%%%%%%%%%%%%%%%%%%%%%%%%%%%%%%%%
%\sideThumbs{Matutinum}

\vspace{2mm}

\cuminitiali{}{temporalia/dominelabiamea.gtex}

\vfill
%\pagebreak

\vspace{2mm}

\ifx\invitatorium\undefined
\pars{Invitatorium.} \scriptura{Ps. 94, 1; Psalmus 94; \textbf{H451}}

\vspace{-6mm}

\antiphona{VI}{temporalia/inv-jubilemusdeo.gtex}\else
\invitatorium
\fi

\vfill
\pagebreak

\ifx\hymnusmatutinum\undefined
\ifx\matua\undefined
\else
\pars{Hymnus.}

{
\grechangedim{interwordspacetext}{0.10 cm plus 0.15 cm minus 0.05 cm}{scalable}%
\antiphona{II}{temporalia/hym-IpsumNunc.gtex}
\grechangedim{interwordspacetext}{0.22 cm plus 0.15 cm minus 0.05 cm}{scalable}%
}
\fi
\else
\hymnusmatutinum
\fi

\vspace{-3mm}

\vfill
\pagebreak

\ifx\matub\undefined
\else
% MAT B
\pars{Psalmus 1.} \scriptura{Ps. 30, 2; \textbf{H90}}

\vspace{-4mm}

\antiphona{VIII G}{temporalia/ant-intuaiustitia.gtex}

%\vspace{-2mm}

\scriptura{Ps. 30, 2-9}

%\vspace{-2mm}

\initiumpsalmi{temporalia/ps30i-initium-viii-G-auto.gtex}

\vspace{-1.5mm}

\input{temporalia/ps30i-viii-G.tex} \Abardot{}

\vfill
\pagebreak

\pars{Psalmus 2.} \scriptura{Ps. 66, 2}

\vspace{-4mm}

\antiphona{E}{temporalia/ant-illuminadomine.gtex}

%\vspace{-2mm}

\scriptura{Ps. 30, 10-17}

%\vspace{-2mm}

\initiumpsalmi{temporalia/ps30ii-initium-e-a-auto.gtex}

\input{temporalia/ps30ii-e-a.tex} \Abardot{}

\vfill
\pagebreak

\pars{Psalmus 3.} \scriptura{Ps. 30, 24}

\vspace{-4mm}

\antiphona{II D}{temporalia/ant-diligitedominum.gtex}

%\vspace{-5mm}

\scriptura{Ps. 30, 20-25}

%\vspace{-2mm}

\initiumpsalmi{temporalia/ps30iii-initium-ii-D-auto.gtex}

\input{temporalia/ps30iii-ii-D.tex} \Abardot{}

\vfill
\pagebreak
\fi

\pars{Versus.}

\ifx\matversus\undefined
\ifx\matub\undefined
\else
\noindent \Vbardot{} Dírige me, Dómine, in veritáte tua, et doce me.

\noindent \Rbardot{} Quia tu es Deus salútis meæ.
\fi
\else
\matversus
\fi

\vspace{5mm}

\sineinitiali{temporalia/oratiodominica-mat.gtex}

\vspace{5mm}

\pars{Absolutio.}

\cuminitiali{}{temporalia/absolutio-exaudi.gtex}

\vfill
\pagebreak

\cuminitiali{}{temporalia/benedictio-solemn-benedictione.gtex}

\vspace{7mm}

\lectioi

\noindent \Vbardot{} Tu autem, Dómine, miserére nobis.
\noindent \Rbardot{} Deo grátias.

\vfill
\pagebreak

\responsoriumi

\vfill
\pagebreak

\cuminitiali{}{temporalia/benedictio-solemn-unigenitus.gtex}

\vspace{7mm}

\lectioii

\noindent \Vbardot{} Tu autem, Dómine, miserére nobis.
\noindent \Rbardot{} Deo grátias.

\vfill
\pagebreak

\responsoriumii

\vfill
\pagebreak

\cuminitiali{}{temporalia/benedictio-solemn-spiritus.gtex}

\vspace{7mm}

\lectioiii

\noindent \Vbardot{} Tu autem, Dómine, miserére nobis.
\noindent \Rbardot{} Deo grátias.

\vfill
\pagebreak

\responsoriumiii

\vfill
\pagebreak

\rubrica{Reliqua omittuntur, nisi Laudes separandæ sint.}

\sineinitiali{temporalia/domineexaudi.gtex}

\vfill

\oratio

\vfill

\noindent \Vbardot{} Dómine, exáudi oratiónem meam.
\Rbardot{} Et clamor meus ad te véniat.

\vfill

\noindent \Vbardot{} Benedicámus Dómino.
\noindent \Rbardot{} Deo grátias.

\vfill

\noindent \Vbardot{} Fidélium ánimæ per misericórdiam Dei requiéscant in pace.
\Rbardot{} Amen.

\vfill
\pagebreak

\hora{Ad Laudes.} %%%%%%%%%%%%%%%%%%%%%%%%%%%%%%%%%%%%%%%%%%%%%%%%%%%%%
%\sideThumbs{Laudes}

\cantusSineNeumas

\vspace{0.5cm}
\grechangedim{interwordspacetext}{0.18 cm plus 0.15 cm minus 0.05 cm}{scalable}%
\cuminitiali{}{temporalia/deusinadiutorium-communis.gtex}
\grechangedim{interwordspacetext}{0.22 cm plus 0.15 cm minus 0.05 cm}{scalable}%

\vfill
\pagebreak

\ifx\hymnuslaudes\undefined
\ifx\laudbd\undefined
\else
\pars{Hymnus} \scriptura{Hilarius (\olddag{} 367)}

\grechangedim{interwordspacetext}{0.16 cm plus 0.15 cm minus 0.05 cm}{scalable}%
\cuminitiali{IV}{temporalia/hym-LucisLargitor.gtex}
\grechangedim{interwordspacetext}{0.22 cm plus 0.15 cm minus 0.05 cm}{scalable}%
\vspace{-3mm}
\fi
\else
\hymnuslaudes
\fi

\vfill
\pagebreak

\ifx\laudb\undefined
\else
\pars{Psalmus 1.} \scriptura{Ps. 41, 3; \textbf{H391}}

\vspace{-4mm}

\antiphona{II D}{temporalia/ant-sitivitanima.gtex}

%\vspace{-2mm}

\scriptura{Psalmus 41}

%\vspace{-2mm}

\initiumpsalmi{temporalia/ps41-initium-ii-D-auto.gtex}

%\vspace{-1.5mm}

\input{temporalia/ps41-ii-D.tex}

\vfill

\antiphona{}{temporalia/ant-sitivitanima.gtex}

\vfill
\pagebreak

\pars{Psalmus 2.}

\vspace{-4mm}

\antiphona{III a}{temporalia/ant-ostendenobisdomine.gtex}

%\vspace{-2mm}

\scriptura{Canticum Ecclesiastici, Sir. 36, 1-7.13-16}

%\vspace{-3mm}

\initiumpsalmi{temporalia/ecclesiastici-initium-iii-a-auto.gtex}

\input{temporalia/ecclesiastici-iii-a.tex} \Abardot{}

\vfill
\pagebreak

\pars{Psalmus 3.}

\vspace{-4mm}

\antiphona{II D}{temporalia/ant-operamanuumeius.gtex}

\scriptura{Psalmus 18, 1-7}

\initiumpsalmi{temporalia/ps18i-initium-ii-D-auto.gtex}

\input{temporalia/ps18i-ii-D.tex} \Abardot{}

\vfill
\pagebreak
\fi

\ifx\lectiobrevis\undefined
\ifx\laudb\undefined
\else
\pars{Lectio Brevis.} \scriptura{Ier. 15, 16}

\noindent Invénti sunt sermónes tui, et comédi eos, et factum est mihi verbum tuum in gáudium et in lætítiam cordis mei, quóniam invocátum est nomen tuum super me, Dómine Deus exercítuum.
\fi
\else
\lectiobrevis
\fi

\vfill

\ifx\responsoriumbreve\undefined
\ifx\laudbd\undefined
\else
\pars{Responsorium breve.} \scriptura{Ps. 32, 1.3}

\cuminitiali{VI}{temporalia/resp-exsultateiusti.gtex}
\fi
\else
\responsoriumbreve
\fi

\vfill
\pagebreak

\ifx\benedictus\undefined
\ifx\laudbd\undefined
\else
\pars{Canticum Zachariæ.} \scriptura{Lc. 1, 68; \textbf{H422}}

\vspace{-4mm}

{
\grechangedim{interwordspacetext}{0.18 cm plus 0.15 cm minus 0.05 cm}{scalable}%
\antiphona{IV E}{temporalia/ant-benedictusdominus.gtex}
\grechangedim{interwordspacetext}{0.22 cm plus 0.15 cm minus 0.05 cm}{scalable}%
}

%\vspace{-3mm}

\scriptura{Lc. 1, 68-79}

%\vspace{-2mm}

\cantusSineNeumas
\initiumpsalmi{temporalia/benedictus-initium-iv-E-auto.gtex}

%\vspace{-1.5mm}

\input{temporalia/benedictus-iv-E.tex} \Abardot{}
\fi
\else
\benedictus
\fi

\vspace{-1cm}

\vfill
\pagebreak

%\sideThumbs{{\scriptsize{}Fine horarum}}

\pars{Preces.}

\sineinitiali{}{temporalia/tonusprecum.gtex}

\ifx\preces\undefined
\ifx\laudb\undefined
\else
\noindent Salvátor noster fecit nos regnum et sacerdótium, ut hóstias Deo acceptábiles offerámus. \gredagger{} Grati ígitur eum invocémus:

\Rbardot{} Serva nos in tuo ministério, Dómine.

\noindent Christe, sacérdos ætérne, qui sanctum pópulo tuo sacerdótium concessísti, \gredagger{} concéde, ut spiritáles hóstias Deo acceptábiles iúgiter offerámus.

\Rbardot{} Serva nos in tuo ministério, Dómine.

\noindent Spíritus tui fructus nobis largíre propítius, \gredagger{} patiéntiam, benignitátem et mansuetúdinem.

\Rbardot{} Serva nos in tuo ministério, Dómine.

\noindent Da nobis te amáre, ut te, qui es cáritas, possideámus, \gredagger{} et bene ágere, ut per vitam étiam nostram te laudémus.

\Rbardot{} Serva nos in tuo ministério, Dómine.

\noindent Quæ frátribus nostris sunt utília, nos quǽrere concéde, \gredagger{} ut salútem facílius consequántur.

\Rbardot{} Serva nos in tuo ministério, Dómine.
\fi
\else
\preces
\fi

\vfill

\pars{Oratio Dominica.}

\cuminitiali{}{temporalia/oratiodominicaalt.gtex}

\vfill
\pagebreak

\rubrica{vel:}

\pars{Supplicatio Litaniæ.}

\cuminitiali{}{temporalia/supplicatiolitaniae.gtex}

\vfill

\pars{Oratio Dominica.}

\cuminitiali{}{temporalia/oratiodominica.gtex}

\vfill
\pagebreak

% Oratio. %%%
\oratio

\vspace{-1mm}

\vfill

\rubrica{Hebdomadarius dicit Dominus vobiscum, vel, absente sacerdote vel diacono, sic concluditur:}

\vspace{2mm}

\antiphona{C}{temporalia/dominusnosbenedicat.gtex}

\rubrica{Postea cantatur a cantore:}

\vspace{2mm}

\cuminitiali{IV}{temporalia/benedicamus-feria-laudes.gtex}

\vspace{1mm}

\vfill
\pagebreak

\end{document}

