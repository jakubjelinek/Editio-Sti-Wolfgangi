\newcommand{\titulus}{\nomenFesti{S. Nominis Beatæ Mariæ Virginis.}
\dies{Die 12. Septembris.}}
\newcommand{\oratio}{\pars{Oratio.}

\noindent Concéde, quǽsumus, omnípotens Deus, ut cunctis gloriósum Beátæ Maríæ Vírginis nomen celebrántibus, misericórdiæ tuæ benefícia ipsa procúret.

\pars{Pro pace in Ucraina.} \scriptura{Sir. 50, 25; 2 Esdr. 4, 20; \textbf{H416}}

\vspace{-4mm}

\antiphona{II D}{temporalia/ant-dapacemdomine.gtex}

\vfill

\noindent Deus, a quo sancta desidéria, recta consília et iusta sunt ópera: da servis tuis illam, quam mundus dare non potest, pacem; ut et corda nostra mandátis tuis dédita, et hóstium subláta formídine, témpora sint tua protectióne tranquílla.

\noindent Per Dóminum nostrum Iesum Christum, Fílium tuum, qui tecum vivit et regnat in unitáte Spíritus Sancti, Deus, per ómnia sǽcula sæculórum.

\noindent \Rbardot{} Amen.}
\newcommand{\invitatorium}{\pars{Invitatorium.}

\vspace{-4mm}

\antiphona{V}{temporalia/inv-christummariaefilium.gtex}}
\newcommand{\hymnusmatutinum}{\pars{Hymnus}

\cuminitiali{VIII}{temporalia/hym-QuemTerra-simplex.gtex}}
\newcommand{\matversus}{\noindent \Vbardot{} María conservábat ómnia verba hæc.

\noindent \Rbardot{} Cónferens in corde suo.}
\newcommand{\absolutio}{\cuminitiali{}{temporalia/absolutio-precibus.gtex}}
\newcommand{\benedictioi}{\cuminitiali{}{temporalia/benedictio-solemn-noscum.gtex}}
\newcommand{\benedictioii}{\cuminitiali{}{temporalia/benedictio-solemn-ipsavirgo.gtex}}
\newcommand{\benedictioiii}{\cuminitiali{}{temporalia/benedictio-solemn-pervirginem.gtex}}
\newcommand{\lectioi}{\pars{Lectio I.} \scriptura{Iob 31, 1-8.13-15}

\noindent De libro Iob.

\noindent Dixit Iob: «Pépigi fœdus cum óculis meis, ut ne cogitárem quidem de vírgine.

\noindent Quæ enim pars mea apud Deum désuper, et quæ heréditas apud Omnipoténtem in excélsis?

\noindent Numquid non perdítio est iníquo, et alienátio operántibus iniustítiam?

\noindent Nonne ipse consíderat vias meas et cunctos gressus meos dinúmerat?

\noindent Si ambulávi in vanitáte, et festinávit in dolo pes meus, appéndat me in statéra iusta et sciat Deus integritátem meam.

\noindent Si declinávit gressus meus de via, et si secútum est óculos meos cor meum, et si mánibus meis adhǽsit mácula, seram, et álius cómedat, et progénies mea eradicétur.

\noindent Si contémpsi subíre iudícium cum servo meo et ancílla mea, cum disceptárent advérsum me, quid enim fáciam, cum surréxerit ad iudicándum Deus, et, cum quæsíerit, quid respondébo illi?

\noindent Numquid non in ventre fecit me, qui et illum operátus est, et formávit me in viscéribus unus?}
\newcommand{\responsoriumi}{\pars{Responsorium 1.} \scriptura{\Rbardot{} Iob 31, 4-6 \Vbardot{} ibid. 6, 8; \textbf{H405}}

\vspace{-5mm}

\responsorium{IV}{temporalia/resp-nonnecognoscit-CROCHU-sinedox.gtex}{}}
\newcommand{\lectioii}{\pars{Lectio II.} \scriptura{Iob 31, 16-23.35-37}

\noindent\noindent Si negávi, quod volébant, paupéribus et óculos víduæ languéscere feci;

\noindent si comédi buccéllam meam solus, et non comédit pupíllus ex ea, quia ab infántia mea educávi eum ut pater et de ventre matris meæ diréxi eam;

\noindent si despéxi pereúntem, eo quod non habúerit induméntum, et absque operiménto páuperem;

\noindent si non benedixérunt mihi látera eius, et de velléribus óvium meárum calefáctus est;

\noindent si levávi super pupíllum manum meam, cum vidérem in porta adiutórium mihi, úmerus meus a iunctúra sua cadat, et bráchium meum cum óssibus lacertórum confringátur, quia timor super me calámitas a Deo, et contra maiestátem eius nihil valérem!

\noindent Quis mihi tríbuat auditórem?

\noindent Ecce signum meum! Omnípotens respóndeat mihi!

\noindent Ecce liber, quem scripsit vir litis meæ, ut in úmero meo portem illum et álligem illum quasi corónam mihi. Númerum gráduum meórum pronuntiábo illi et quasi príncipem adíbo eum».}
\newcommand{\responsoriumii}{\pars{Responsorium 2.} \scriptura{\Rbardot{} Iob 6, 2 \Vbardot{} ibid. 6, 3; \textbf{H403}}

\vspace{-5mm}

\responsorium{IV}{temporalia/resp-utinamappenderentur-CROCHU.gtex}{}

\rubrica{vel ad libitum:}

\vspace{3mm}

\pars{Responsorium 2.} \scriptura{\Rbardot{} Iob 31, 25; 23, 7 \Vbardot{} ibid. 23, 6-7; \textbf{H406}}

\vspace{-5mm}

\responsorium{VII}{temporalia/resp-desideriummeum-CROCHU.gtex}{}}
\newcommand{\lectioiii}{\pars{Lectio III.} \scriptura{Hom. 2, 17, 1-33: SCh 390, 1993, 168-170}

\noindent Ex Homíliis in Láudibus Vírginis Matris sancti Bernárdi abbátis.

\noindent «Et nomen — inquit Evangelísta — Vírginis María». Loquámur pauca et super hoc nómine, quod interpretátum «maris stella» dícitur, et Matri Vírgini valde conveniénter aptátur. Ipsa namque aptíssime síderi comparátur, quia, sicut sine sui corruptióne sidus suum emíttit rádium, sic absque sui læsióne Virgo párturit Fílium. Nec síderi rádius suam mínuit claritátem, nec Vírgini Fílius suam integritátem. Ipsa ergo est nóbilis illa stella ex Iacob orta, cuius rádius univérsum orbem illúminat, cuius splendor et præfúlget in supérnis et ínferos pénetrat, terras étiam perlústrans et calefáciens magis mentes quam córpora fovet virtútem, excóquit vítia. Ipsa, inquam, est præclára et exímia stella super hoc mare magnum et spatiósum necessário subleváta, micans méritis, illústrans exémplis.

\noindent O quisquis te intélligis in huius sǽculi proflúvio magis inter procéllas et tempestátes fluctuáre quam per terram ambuláre, ne avértas óculos a fulgóre huius síderis, si non vis óbrui procéllis! Si insúrgant venti tentatiónum, si incúrras scópulos tribulatiónum, réspice stellam, voca Maríam. Si iactáris supérbiæ undis, si ambitiónis, si detractiónis, si æmulatiónis, réspice stellam, voca Maríam. Si iracúndia aut avarítia aut carnis illécebra navículam concússerit mentis, réspice ad Maríam. Si críminum immanitáte turbátus, consciéntiæ fœditáte confúsus, iudícii horróre pertérritus, báratro incípias absorbéri tristítiæ, desperatiónis abýsso, cógita Maríam.

\noindent In perículis, in angústiis, in rebus dúbiis Maríam cógita, Maríam invóca. Non recédat ab ore, non recédat a corde et, ut ímpetres eius oratiónis suffrágium, non déseras conversatiónis exémplum. Ipsam sequens non dévias, ipsam rogans non despéras, ipsam cógitans non erras; ipsa tenénte non córruis, ipsa protegénte non métuis, ipsa duce non fatigáris, ipsa propítia pérvenis et sic in temetípso experíris quam mérito dictum sit: «Et nomen Vírginis María».}
\newcommand{\responsoriumiii}{\pars{Responsorium 3.} \scriptura{\Rbardot{} Is. 61, 10; Ps. 44, 12; Ct. 1, 2 \Vbardot{} Ps. 44, 10; \textbf{H297}}

\vspace{-5mm}

\responsorium{VIII}{temporalia/resp-ornataminmonilibus-cumdox.gtex}{}}
\newcommand{\hymnuslaudes}{\pars{Hymnus}

\cuminitiali{I}{temporalia/hym-AveMarisStella.gtex}}
\newcommand{\lectiobrevis}{\pars{Lectio Brevis.} \scriptura{Cf. Is 61, 10}

\noindent Gaudens gaudébo in Dómino, et exsultábit ánima mea in Deo meo, quia índuit me vestiméntis salútis et induménto iustítiæ circúmdedit me, quasi sponsam ornátam monílibus suis.}
\newcommand{\responsoriumbreve}{\pars{Responsorium breve.}

\antiphona{VI}{temporalia/resp-elegiteamdeus.gtex}}
\newcommand{\benedictus}{\pars{Canticum Zachariæ.} \scriptura{Ct. 1, 2}

\vspace{-4mm}

\antiphona{VII a}{temporalia/ant-unguentumeffusum.gtex}

%\vspace{-2mm}

\scriptura{Lc. 1, 68-79}

%\vspace{-2mm}

\cantusSineNeumas
\initiumpsalmi{temporalia/benedictus-initium-vii-a-auto.gtex}

%\vspace{-1.5mm}

\input{temporalia/benedictus-vii-a.tex} \Abardot{}}
\newcommand{\preces}{\noindent Salvatórem nostrum celebrántes,~\gredagger{} qui ex María Vírgine nasci dignátus est,~\grestar{} exorémus dicéntes:

\Rbardot{} Intercédat pro nobis mater tua, Dómine.

\noindent O sol iustítiæ,~\gredagger{} quem Immaculáta Virgo ut lucens auróra præcéssit,~\grestar{} tríbue ut in lúmine visitatiónis tuæ semper ambulémus.

\Rbardot{} Intercédat pro nobis mater tua, Dómine.

\noindent Verbum ætérnum,~\gredagger{} quod Maríam habitatiónis tuæ arcam incorruptíbilem elegísti,~\grestar{} líbera nos a corruptióne peccáti.

\Rbardot{} Intercédat pro nobis mater tua, Dómine.

\noindent Salvátor noster,~\gredagger{} qui iuxta crucem matrem tuam habuísti,~\grestar{} præsta ut, ipsa intercedénte, communicántes tuis passiónibus gaudeámus.

\Rbardot{} Intercédat pro nobis mater tua, Dómine.

\noindent Benigníssime Iesu,~\gredagger{} qui pendens in cruce, Maríam Ioánni matrem dedísti,~\grestar{} da nobis ita vívere ut eius fílii agnoscámur.

\Rbardot{} Intercédat pro nobis mater tua, Dómine.}
\newcommand{\benedicamuslaudes}{\cuminitiali{I}{temporalia/benedicamus-festis-bmv.gtex}}
\newcommand{\hebdomada}{infra Hebdom. XXIII post Pentecosten.}
\newcommand{\oratioLaudes}{\cuminitiali{}{temporalia/oratio23.gtex}}
\newcommand{\hiemalis}{Hiemalis.}

% LuaLaTeX

\documentclass[a4paper, twoside, 12pt]{article}
\usepackage[latin]{babel}
%\usepackage[landscape, left=3cm, right=1.5cm, top=2cm, bottom=1cm]{geometry} % okraje stranky
%\usepackage[landscape, a4paper, mag=1166, truedimen, left=2cm, right=1.5cm, top=1.6cm, bottom=0.95cm]{geometry} % okraje stranky
\usepackage[landscape, a4paper, mag=1400, truedimen, left=0.5cm, right=0.5cm, top=0.5cm, bottom=0.5cm]{geometry} % okraje stranky

\usepackage{fontspec}
\setmainfont[FeatureFile={junicode.fea}, Ligatures={Common, TeX}, RawFeature=+fixi]{Junicode}
%\setmainfont{Junicode}

% shortcut for Junicode without ligatures (for the Czech texts)
\newfontfamily\nlfont[FeatureFile={junicode.fea}, Ligatures={Common, TeX}, RawFeature=+fixi]{Junicode}

\usepackage{multicol}
\usepackage{color}
\usepackage{lettrine}
\usepackage{fancyhdr}

% usual packages loading:
\usepackage{luatextra}
\usepackage{graphicx} % support the \includegraphics command and options
\usepackage{gregoriotex} % for gregorio score inclusion
\usepackage{gregoriosyms}
\usepackage{wrapfig} % figures wrapped by the text
\usepackage{parcolumns}
\usepackage[contents={},opacity=1,scale=1,color=black]{background}
\usepackage{tikzpagenodes}
\usepackage{calc}
\usepackage{longtable}
\usetikzlibrary{calc}

\setlength{\headheight}{14.5pt}

\input{conventuscommune.tex} % Often used macros

\newcommand{\annusEditionis}{2021}

%%%% Vicekrat opakovane kousky

\newcommand{\anteOrationem}{
  \rubrica{Ante Orationem, cantatur a Superiore:}

  \pars{Supplicatio Litaniæ.}

  \cuminitiali{}{temporalia/supplicatiolitaniae.gtex}

  \pars{Oratio Dominica.}

  \cuminitiali{}{temporalia/oratiodominica.gtex}

  \rubrica{Deinde dicitur ab Hebdomadario:}

  \cuminitiali{}{temporalia/dominusvobiscum-solemnis.gtex}

  \rubrica{In choro monialium loco Dominus vobiscum dicitur:}

  \sineinitiali{temporalia/domineexaudi.gtex}
}

\setlength{\columnsep}{30pt} % prostor mezi sloupci

%%%%%%%%%%%%%%%%%%%%%%%%%%%%%%%%%%%%%%%%%%%%%%%%%%%%%%%%%%%%%%%%%%%%%%%%%%%%%%%%%%%%%%%%%%%%%%%%%%%%%%%%%%%%%
\begin{document}

% Here we set the space around the initial.
% Please report to http://home.gna.org/gregorio/gregoriotex/details for more details and options
\grechangedim{afterinitialshift}{2.2mm}{scalable}
\grechangedim{beforeinitialshift}{2.2mm}{scalable}
\grechangedim{interwordspacetext}{0.22 cm plus 0.15 cm minus 0.05 cm}{scalable}%
\grechangedim{annotationraise}{-0.2cm}{scalable}

% Here we set the initial font. Change 38 if you want a bigger initial.
% Emit the initials in red.
\grechangestyle{initial}{\color{red}\fontsize{38}{38}\selectfont}

\pagestyle{empty}

%%%% Titulni stranka
\begin{titulusOfficii}
\ifx\titulus\undefined
\nomenFesti{Feria III \hebdomada{}}
\else
\titulus
\fi
\end{titulusOfficii}

\vfill

\begin{center}
%Ad usum et secundum consuetudines chori \guillemotright{}Conventus Choralis\guillemotleft.

%Editio Sancti Wolfgangi \annusEditionis
\end{center}

\scriptura{}

\pars{}

\pagebreak

\renewcommand{\headrulewidth}{0pt} % no horiz. rule at the header
\fancyhf{}
\pagestyle{fancy}

\cantusSineNeumas

\ifx\oratio\undefined
\ifx\laudb\undefined
\else
\newcommand{\oratio}{\pars{Oratio.}

\noindent Dómine Iesu Christe, lux vera, qui omnes hómines illúminas ad salútem, nobis, quǽsumus, concéde virtútem, ut ante te vias pacis et iustítiæ præparémus.

\noindent Qui vivis et regnas cum Deo Patre in unitáte Spíritus Sancti, Deus, per ómnia sǽcula sæculórum.

\noindent \Rbardot{} Amen.}
\fi
\fi

\hora{Ad Matutinum.} %%%%%%%%%%%%%%%%%%%%%%%%%%%%%%%%%%%%%%%%%%%%%%%%%%%%%

\vspace{2mm}

\cuminitiali{}{temporalia/dominelabiamea.gtex}

\vfill
%\pagebreak

\vspace{2mm}

\ifx\invitatorium\undefined
\ifx\matuac\undefined
\else
\pars{Invitatorium.} \scriptura{Ps. 94, 1; Psalmus 94; \textbf{H451}}

\vspace{-6mm}

\antiphona{VI}{temporalia/inv-jubilemusdeo.gtex}
\fi
\ifx\matubd\undefined
\else
\pars{Invitatorium.} \scriptura{Cantor; Psalmus 94; \textbf{H449}}

\vspace{-6mm}

\antiphona{E}{temporalia/inv-regemmagnum.gtex}
\fi
\else
\invitatorium
\fi

\vfill
\pagebreak

\ifx\hymnusmatutinum\undefined
\ifx\matuac\undefined
\else
\pars{Hymnus}

\cuminitiali{IV}{temporalia/hym-SomnoRefectis.gtex}
\fi
\ifx\matubd\undefined
\else
\pars{Hymnus.} \scriptura{Gregorius Magnus (\olddag{} 604)}

{
\grechangedim{interwordspacetext}{0.10 cm plus 0.15 cm minus 0.05 cm}{scalable}%
\antiphona{I}{temporalia/hym-NocteSurgentes.gtex}
\grechangedim{interwordspacetext}{0.22 cm plus 0.15 cm minus 0.05 cm}{scalable}%
}
\fi
\else
\hymnusmatutinum
\fi

\vspace{-3mm}

\vfill
\pagebreak

\ifx\matub\undefined
\else
% MAT B
\pars{Psalmus 1.} \scriptura{Ps. 36, 5; \textbf{H93}}

\vspace{-4mm}

\antiphona{VI F}{temporalia/ant-reveladomino.gtex}

%\vspace{-2mm}

\scriptura{Ps. 36, 1-11}

%\vspace{-2mm}

\initiumpsalmi{temporalia/ps36i_xi-initium-vi-F-auto.gtex}

\input{temporalia/ps36i_xi-vi-F.tex} \Abardot{}

\vfill
\pagebreak

\pars{Psalmus 2.}

\vspace{-4mm}

\antiphona{II D}{temporalia/ant-iuniorfui.gtex}

\vspace{-2mm}

\scriptura{Ps. 36, 12-29}

\vspace{-2mm}

\initiumpsalmi{temporalia/ps36xii_xxix-initium-ii-D-auto.gtex}

\input{temporalia/ps36xii_xxix-ii-D.tex}

\vfill

\antiphona{}{temporalia/ant-iuniorfui.gtex}

\vfill
\pagebreak

\pars{Psalmus 3.} \scriptura{Ps. 36, 3}

\vspace{-4mm}

\antiphona{VI F}{temporalia/ant-speraindomino.gtex}

%\vspace{-2mm}

\scriptura{Ps. 36, 30-40}

%\vspace{-2mm}

\initiumpsalmi{temporalia/ps36iii-initium-vi-F-auto.gtex}

\input{temporalia/ps36iii-vi-F.tex} \Abardot{}

\vfill
\pagebreak
\fi
\ifx\matuc\undefined
\else
% MAT C
\pars{Psalmus 1.} \scriptura{Ps. 67, 2}

\vspace{-4mm}

\antiphona{VII a}{temporalia/ant-exsurgatdeus.gtex}

%\vspace{-2mm}

\scriptura{Ps. 67, 2-11}

\initiumpsalmi{temporalia/ps67i-initium-vii-a-auto.gtex}

\input{temporalia/ps67i-vii-a.tex} \Abardot{}

\vfill
\pagebreak

\pars{Psalmus 2.}

\vspace{-4mm}

\antiphona{I f}{temporalia/ant-deusnosterdeussalvos.gtex}

%\vspace{-2mm}

\scriptura{Ps. 67, 12-24}

%\vspace{-2mm}

\initiumpsalmi{temporalia/ps67ii-initium-i-f-auto.gtex}

\input{temporalia/ps67ii-i-f.tex} \Abardot{}

\vfill
\pagebreak

\pars{Psalmus 3.} \scriptura{Ps. 67, 27; \textbf{H96}}

\vspace{-4mm}

\antiphona{D}{temporalia/ant-inecclesiis.gtex}

%\vspace{-2mm}

\scriptura{Ps. 67, 25-36}

\initiumpsalmi{temporalia/ps67iii-initium-d-g2-auto.gtex}

\input{temporalia/ps67iii-d-g2.tex} \Abardot{}

\vfill
\pagebreak
\fi

\pars{Versus.}

\ifx\matversus\undefined
\ifx\matub\undefined
\else
\noindent \Vbardot{} Bonitátem et prudéntiam et sciéntiam doce me.

\noindent \Rbardot{} Quia præcéptis tuis crédidi.
\fi
\ifx\matuc\undefined
\else
\noindent \Vbardot{} Audiam quid loquátur Dóminus Deus.

\noindent \Rbardot{} Loquétur pacem ad plebem suam.
\fi
\else
\matversus
\fi

\vspace{5mm}

\sineinitiali{temporalia/oratiodominica-mat.gtex}

\vspace{5mm}

\pars{Absolutio.}

\cuminitiali{}{temporalia/absolutio-ipsius.gtex}

\vfill
\pagebreak

\cuminitiali{}{temporalia/benedictio-solemn-deus.gtex}

\vspace{7mm}

\lectioi

\noindent \Vbardot{} Tu autem, Dómine, miserére nobis.
\noindent \Rbardot{} Deo grátias.

\vfill
\pagebreak

\responsoriumi

\vfill
\pagebreak

\cuminitiali{}{temporalia/benedictio-solemn-christus.gtex}

\vspace{7mm}

\lectioii

\noindent \Vbardot{} Tu autem, Dómine, miserére nobis.
\noindent \Rbardot{} Deo grátias.

\vfill
\pagebreak

\responsoriumii

\vfill
\pagebreak

\cuminitiali{}{temporalia/benedictio-solemn-ignem.gtex}

\vspace{7mm}

\lectioiii

\noindent \Vbardot{} Tu autem, Dómine, miserére nobis.
\noindent \Rbardot{} Deo grátias.

\vfill
\pagebreak

\responsoriumiii

\vfill
\pagebreak

\rubrica{Reliqua omittuntur, nisi Laudes separandæ sint.}

\sineinitiali{temporalia/domineexaudi.gtex}

\vfill

\oratio

\vfill

\noindent \Vbardot{} Dómine, exáudi oratiónem meam.
\Rbardot{} Et clamor meus ad te véniat.

\vfill

\noindent \Vbardot{} Benedicámus Dómino.
\noindent \Rbardot{} Deo grátias.

\vfill

\noindent \Vbardot{} Fidélium ánimæ per misericórdiam Dei requiéscant in pace.
\Rbardot{} Amen.

\vfill
\pagebreak

\hora{Ad Laudes.} %%%%%%%%%%%%%%%%%%%%%%%%%%%%%%%%%%%%%%%%%%%%%%%%%%%%%

\cantusSineNeumas

\vspace{0.5cm}
\grechangedim{interwordspacetext}{0.18 cm plus 0.15 cm minus 0.05 cm}{scalable}%
\cuminitiali{}{temporalia/deusinadiutorium-communis.gtex}
\grechangedim{interwordspacetext}{0.22 cm plus 0.15 cm minus 0.05 cm}{scalable}%

\vfill
\pagebreak

\ifx\hymnuslaudes\undefined
\ifx\laudac\undefined
\else
\pars{Hymnus} \scriptura{Ambrosius (\olddag{} 397)}

\cuminitiali{I}{temporalia/hym-SplendorPaternae-hiemalis.gtex}
\fi
\ifx\laudbd\undefined
\else
\pars{Hymnus}

\grechangedim{interwordspacetext}{0.16 cm plus 0.15 cm minus 0.05 cm}{scalable}%
\cuminitiali{IV}{temporalia/hym-AEterneLucis.gtex}
\grechangedim{interwordspacetext}{0.22 cm plus 0.15 cm minus 0.05 cm}{scalable}%
\vspace{-3mm}
\fi
\else
\hymnuslaudes
\fi

\vfill
\pagebreak

\ifx\laudb\undefined
\else
\pars{Psalmus 1.} \scriptura{Ps. 42, 5; \textbf{H95}}

\vspace{-4mm}

\antiphona{VI F}{temporalia/ant-salutarevultusmei.gtex}

\scriptura{Psalmus 42.}

\initiumpsalmi{temporalia/ps42-initium-vi-F-auto.gtex}

\input{temporalia/ps42-vi-F.tex} \Abardot{}

\vfill
\pagebreak

\pars{Psalmus 2.} \scriptura{Is. 38, 20; \textbf{H95}}

\vspace{-7mm}

\antiphona{E}{temporalia/ant-cunctisdiebus.gtex}

\vspace{-4mm}

\scriptura{Canticum Ezechiæ, Is. 38, 10-20}

\vspace{-3mm}

\initiumpsalmi{temporalia/ezechiae-initium-e-auto.gtex}

\input{temporalia/ezechiae-e.tex} \Abardot{}

\vfill
\pagebreak

\pars{Psalmus 3.} \scriptura{Ps. 64, 2; \textbf{H96}}

\vspace{-4mm}

\antiphona{VIII a}{temporalia/ant-tedecet.gtex}

\vspace{-2mm}

\scriptura{Psalmus 64.}

\vspace{-2mm}

\initiumpsalmi{temporalia/ps64-initium-viii-A-auto.gtex}

\input{temporalia/ps64-viii-A.tex} \Abardot{}

\vfill
\pagebreak
\fi
\ifx\laudc\undefined
\else
\pars{Psalmus 1.} \scriptura{Ps. 83, 5}

\vspace{-4mm}

\antiphona{VIII G}{temporalia/ant-beatiquihabitant.gtex}

\vspace{-2mm}

\scriptura{Psalmus 84.}

\vspace{-2mm}

\initiumpsalmi{temporalia/ps84-initium-viii-G-auto.gtex}

\input{temporalia/ps84-viii-G.tex} \Abardot{}

\vfill
\pagebreak

\pars{Psalmus 2.}

\vspace{-4mm}

\antiphona{VII d}{temporalia/ant-denoctespiritusmeus.gtex}

\vspace{-2mm}

\scriptura{Canticum Isaiæ, Is. 26, 1-12}

\vspace{-2mm}

\initiumpsalmi{temporalia/isaiae3-initium-vii-d.gtex}

\input{temporalia/isaiae3-vii-d.tex} \Abardot{}

\vfill
\pagebreak

\pars{Psalmus 3.} \scriptura{Ps. 66, 2}

\vspace{-4mm}

\antiphona{E}{temporalia/ant-illuminadomine.gtex}

%\vspace{-2mm}

\scriptura{Psalmus 66.}

%\vspace{-2mm}

\initiumpsalmi{temporalia/ps66-initium-e.gtex}

\input{temporalia/ps66-e.tex} \Abardot{}

\vfill
\pagebreak
\fi

\ifx\lectiobrevis\undefined
\ifx\laudb\undefined
\else
\pars{Lectio Brevis.} \scriptura{1 Th. 5, 4-5}

\noindent Vos, fratres, non estis in ténebris, ut vos dies ille tamquam fur comprehéndat; omnes enim vos fílii lucis estis et fílii diéi. Non sumus noctis neque tenebrárum.
\fi
\ifx\laudc\undefined
\else
\pars{Lectio Brevis.} \scriptura{1 Io. 4, 14-15}

\noindent Nos vídimus et testificámur quóniam Pater misit Fílium salvatórem mundi. Quisque conféssus fúerit: Iesus est Fílius Dei, Deus in ipso manet, et ipse in Deo.
\fi
\else
\lectiobrevis
\fi

\vfill

\ifx\responsoriumbreve\undefined
\ifx\laudac\undefined
\else
\pars{Responsorium breve.}

\cuminitiali{VI}{temporalia/resp-benedictusdominus.gtex}
\fi
\ifx\laudbd\undefined
\else
\pars{Responsorium breve.} \scriptura{Ps. 118, 149.147}

\cuminitiali{VI}{temporalia/resp-vocemmeamaudi.gtex}
\fi
\else
\responsoriumbreve
\fi

\vfill
\pagebreak

\ifx\benedictus\undefined
\ifx\laudbd\undefined
\else
\pars{Canticum Zachariæ.} \scriptura{Lc. 1, 71; \textbf{H423}}

\vspace{-5mm}

{
\grechangedim{interwordspacetext}{0.18 cm plus 0.15 cm minus 0.05 cm}{scalable}%
\antiphona{I g\textsuperscript{5}}{temporalia/ant-demanuomnium.gtex}
\grechangedim{interwordspacetext}{0.22 cm plus 0.15 cm minus 0.05 cm}{scalable}%
}

%\vspace{-3mm}

\scriptura{Lc. 1, 68-79}

%\vspace{-1mm}

\initiumpsalmi{temporalia/benedictus-initium-i-g5-auto.gtex}

\input{temporalia/benedictus-i-g5.tex} \Abardot{}
\fi
\else
\benedictus
\fi

\vspace{-1cm}

\vfill
\pagebreak

\pars{Preces.}

\sineinitiali{}{temporalia/tonusprecum.gtex}

\ifx\preces\undefined
\ifx\laudb\undefined
\else
\noindent Salvatóri nostro benedicámus, qui sua resurrectióne mundum clarificávit, \gredagger{} et humíliter invocémus eum dicéntes:

\Rbardot{} Salva nos, Dómine, in sémita tua.

\noindent Resurrectiónem tuam, Dómine Iesu, oratióne cólimus matutína, \gredagger{} spes glóriæ tuæ diem nostrum illúminet.

\Rbardot{} Salva nos, Dómine, in sémita tua.

\noindent Súscipe, Dómine, vota et propósita nostra, \gredagger{} tamquam diéi nostri primítias.

\Rbardot{} Salva nos, Dómine, in sémita tua.

\noindent Tríbue in dilectióne tua nos hódie profícere, \gredagger{} ut ómnia in nostrum omniúmque bonum cooperéntur.

\Rbardot{} Salva nos, Dómine, in sémita tua.

\noindent Da, Dómine, sic lucére lucem nostram coram homínibus, \gredagger{} ut vídeant ópera nostra bona et Patrem gloríficent.

\Rbardot{} Salva nos, Dómine, in sémita tua.
\fi
\else
\preces
\fi

\vfill

\pars{Oratio Dominica.}

\cuminitiali{}{temporalia/oratiodominicaalt.gtex}

\vfill
\pagebreak

\rubrica{vel:}

\pars{Supplicatio Litaniæ.}

\cuminitiali{}{temporalia/supplicatiolitaniae.gtex}

\vfill

\pars{Oratio Dominica.}

\cuminitiali{}{temporalia/oratiodominica.gtex}

\vfill
\pagebreak

% Oratio. %%%
\oratio

\vspace{-1mm}

\vfill

\rubrica{Hebdomadarius dicit Dominus vobiscum, vel, absente sacerdote vel diacono, sic concluditur:}

\vspace{2mm}

\antiphona{C}{temporalia/dominusnosbenedicat.gtex}

\rubrica{Postea cantatur a cantore:}

\vspace{2mm}

\cuminitiali{IV}{temporalia/benedicamus-feria-laudes.gtex}

\vspace{1mm}

\vfill
\pagebreak

\end{document}

