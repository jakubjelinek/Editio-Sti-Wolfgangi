\newcommand{\titulus}{\nomenFesti{Officium Defunctorum pro Maria Klein.}
\dies{Die 18. Octobris A.D. MMXXIV.}}
\newcommand{\sineobmv}{Sine officium B.M.V.}
\newcommand{\sineabsolutio}{Sine absolutio et benedictio.}
\newcommand{\oratio}{\pars{Oratio.}

\noindent Deus, glória fidélium et vita iustórum, cuius Fílii morte et resurrectióne redémpti sumus, propitiáre ancíllæ tuæ Maríæ, ut, qui resurrectiónis nostræ mystérium agnóvit, ætérnæ beatitúdinis gáudia percípere mereátur.

\noindent Per Dóminum nostrum Iesum Christum, Fílium tuum, qui tecum vivit et regnat in unitáte Spíritus Sancti, Deus, per ómnia sǽcula sæculórum.

\noindent \Rbardot{} Amen.}
\newcommand{\dominelabiamea}{\cuminitiali{}{temporalia/dominelabiamea-praglia.gtex}}
\newcommand{\invitatorium}{\pars{Invitatorium.}

\vspace{-4mm}

\antiphona{VI}{temporalia/inv-regemcuiomnia.gtex}}
\newcommand{\hymnusmatutinum}{\pars{Hymnus.}

\antiphona{II}{temporalia/hym-QuiVivis.gtex}}
\newcommand{\matutinum}{\pars{Psalmus 1.} \scriptura{Ps. 138, 15}

\vspace{-4mm}

\antiphona{VII c\textsuperscript{2}}{temporalia/ant-deterraplasmastime.gtex}

\vspace{-2mm}

\scriptura{Ps. 39, 2-9}

%\vspace{-2mm}

\initiumpsalmi{temporalia/ps39i-initium-vii-c2-auto.gtex}

\input{temporalia/ps39i-vii-c2.tex} \Abardot{}

\vfill
\pagebreak

\pars{Psalmus 2.} \scriptura{Ps. 39, 14; \textbf{H391}}

\vspace{-5mm}

\antiphona{II D}{temporalia/ant-complaceattibi.gtex}

\vspace{-2mm}

\scriptura{Ps. 39, 10-18}

\vspace{-2mm}

\initiumpsalmi{temporalia/ps39ii-initium-ii-D-auto.gtex}

\input{temporalia/ps39ii-ii-D.tex} \Abardot{}

\vfill
\pagebreak

\pars{Psalmus 3.} \scriptura{Ps. 41, 3; \textbf{H391}}

\vspace{-4mm}

\antiphona{II D}{temporalia/ant-sitivitanima.gtex}

\vspace{-2mm}

\scriptura{Ps. 41}

%\vspace{-2mm}

\initiumpsalmi{temporalia/ps41-initium-ii-D-auto.gtex}

%\vspace{-1mm}

\input{temporalia/ps41-ii-D.tex}

\antiphona{}{temporalia/ant-sitivitanima.gtex}

\vfill
\pagebreak}
\newcommand{\matversus}{\scriptura{Ps. 118, 156}

\sineinitiali{temporalia/versus-misericordiaetuae.gtex}}
\newcommand{\lectioi}{\rubrica{Lectiones leguntur sine Absolutione, Benedictione et Titulo in tono Prophetiæ.}

\cuminitiali{}{temporalia/tonus-lectionis-prophetiae.gtex}

\pars{Lectio I.} \scriptura{2 Cor. 4, 16-18; 5, 1-10}

\noindent De Epístola secúnda beáti Pauli apóstoli ad Corínthios.

\noindent Fratres: Licet is, qui foris est, noster homo corrúmpitur, tamen is, qui intus est, noster renovátur de die in diem. Id enim, quod in præsénti est, leve tribulatiónis nostræ supra modum in sublimitátem ætérnum glóriæ pondus operátur nobis, non contemplántibus nobis, quæ vidéntur, sed quæ non vidéntur; quæ enim vidéntur, temporália sunt, quæ autem non vidéntur, ætérna sunt.

\noindent Scimus enim quóniam si terréstris domus nostra huius tabernáculi dissolvátur, ædificatiónem ex Deo habémus domum non manufáctam, ætérnam in cælis. Nam et in hoc ingemíscimus habitatiónem nostram, quæ de cælo est, superíndui cupiéntes, si tamen et exspoliáti, non nudi inveniámur. Nam et, qui sumus in tabernáculo, ingemíscimus graváti, eo quod nólumus exspoliári sed supervestíri, ut absorbeátur, quod mortále est, a vita. Qui autem effécit nos in hoc ipsum, Deus, qui dedit nobis arrabónem Spíritus.

\noindent Audéntes ígitur semper et sciéntes quóniam, dum præséntes sumus in córpore, peregrinámur a Dómino; per fidem enim ambulámus et non per spéciem. Audémus autem et bonam voluntátem habémus magis peregrinári a córpore et præséntes esse ad Dóminum. Et ídeo conténdimus sive præséntes sive abséntes placére illi. Omnes enim nos manifestári opórtet ante tribúnal Christi, ut réferat unusquísque pro eis, quæ per corpus gessit, sive bonum sive malum.}
\newcommand{\responsoriumi}{\pars{Responsorium 1.} \scriptura{\Rbardot{} Iob 19, 25.26; \Vbardot{} ibid. 19, 27; \textbf{H390}}

\vspace{-5mm}

\responsorium{VIII}{temporalia/resp-credoquod-GN.gtex}{}}
\newcommand{\lectioii}{\pars{Lectio II.} \scriptura{Epist. 19: PL 80, 665-666}

\noindent Ex Epístolis sancti Brauliónis Cæsaraugustáni epíscopi.

\noindent Spes ómnium credéntium Christus excedéntes a mundo dormiéntes vocat, non mórtuos, dicens; \emph{Lázarus amícus noster dormit.}

\noindent Sed et sanctus Apóstolus non vult nos contristári de dormiéntibus, ac per hoc si fides nostra hoc habet, quia omnes credéntes in Christo secúndum vocem evangélicam non moriéntur in ætérnum, fide scimus quia nec ille mórtuus est, nec nos moriémur.

\noindent Quóniam ipse Dóminus in iussu et in voce archángeli et in tuba Dei descéndet de cælo, et mórtui qui in eo sunt resúrgent.

\noindent Spes ergo nos resurrectiónis ánimet, quóniam quos hic amíttimus illic revidébimus; tantum est ut in eo bene credámus, præcéptis scílicet eius paréntes, apud quem est summa virtútis suscitáre facílius mórtuos quam nobis somno déditos. Ecce ista dícimus, et tamen afféctu néscio quo in lácrimis retráhimur, et crédulam mentem desidérii frangit afféctus. Heu! miserábilis humána condício, et sine Christo vanum omne quod vívimus.}
\newcommand{\responsoriumii}{\pars{Responsorium 2.} \scriptura{\Rbardot{} Cantor; \Vbardot{} ibidem; \textbf{H390}}

\vspace{-5mm}

\responsorium{IV}{temporalia/resp-quilazarum-GN.gtex}{}

\rubrica{vel ad libitum:}

\vspace{3mm}

\pars{Responsorium 2.} \scriptura{\Rbardot{} Iob 7, 7 \Vbardot{} Ps. 129, 1; \textbf{H403}}

\vspace{-5mm}

\responsorium{II}{temporalia/resp-mementomeideusquiaventus-CROCHU.gtex}{}}
\newcommand{\lectioiii}{\pars{Lectio III.}

\noindent O mors, quas coniúnctos dívidis, et amicítia sociátos dura et crudélis dissócias! iam iam confráctæ sunt vires tuæ. Iam contrítum est ímpium iugum tuum ab illo, qui tibi per Oséæ  rugítus minabátur \emph{O mors, ero mors tua. Unde per Apóstolum insultámus: Ubi est, mors, victória tua? Ubi est, mors acúleus tuus?} Ipse qui te vicit nos redémit, qui ánimam suam diléctam trádidit in manus impiórum, ut ex ímpiis fáceret sibi diléctos. Longum quidem est et multum quæ ad consolatiónem commúnem de divínis Scriptúris débeant replicári. Sed suffíciat nobis spes resurrectiónis et oculórum nostrórum diréctio ad glóriam nostri Redemptóris, in quo nos fide iam resurrexísse putámus, dicénte Apóstolo: \emph{Si enim mórtui sumus cum Christo, crédimus quia simul étiam vívimus cum ipso.}

\noindent Non enim sumus nostri, sed eius qui nos redémit, ex cuius voluntáte volúntas semper nostra péndere debet; ob hoc et in oratióne dícimus: \emph{Fiat volúntas tua.} Quamóbrem necésse est ut cum lob in fúnere dicámus: \emph{Dóminus dedit, Dóminus ábstulit, sit nomen Dómini benedíctum.} Dicámus hoc cum Iob hic, ne dissímiles in causa præsénti ab eo inveniámur illic.}
\newcommand{\responsoriumiii}{\pars{Responsorium 3.} \scriptura{\Rbardot{} Ioel 3, 16; \Vbardot{} Cantor; \textbf{H392}}

\vspace{-5mm}

\responsorium{I}{temporalia/resp-liberamedominedemorte-CROCHU.gtex}{}}
\newcommand{\hymnuslaudes}{\pars{Hymnus.}

\cuminitiali{I}{temporalia/hym-ChristeRex.gtex}}
\newcommand{\laudes}{\pars{Psalmus 1.} \scriptura{Ps. 50, 10; \textbf{H393}}

\vspace{-4mm}

\antiphona{I g}{temporalia/ant-exsultabuntdomino.gtex}

%\vspace{-2mm}

\scriptura{Psalmus 50.}

%\vspace{-1mm}

\initiumpsalmi{temporalia/ps50-initium-i-g-auto.gtex}

%\vspace{-1.5mm}

\input{temporalia/ps50-i-g.tex}

\antiphona{}{temporalia/ant-exsultabuntdomino.gtex}

\vfill
\pagebreak

\pars{Psalmus 2.} \scriptura{Cf. Is. 38, 10.17; \textbf{H225}}

\vspace{-4mm}

\antiphona{II D}{temporalia/ant-aportainferi.gtex}

\vspace{-2mm}

\scriptura{Canticum Ezechiæ, Is. 38, 10-20}

\initiumpsalmi{temporalia/ezechiae-initium-ii-D-auto.gtex}

\input{temporalia/ezechiae-ii-D.tex}

\antiphona{}{temporalia/ant-aportainferi.gtex}

\vfill
\pagebreak

\pars{Psalmus 3.} \scriptura{Ps. 150, 6; \textbf{H393}}

\vspace{-4mm}

\antiphona{VII a}{temporalia/ant-omnisspiritus.gtex}

\scriptura{Ps. 150}

\vspace{-2mm}

\initiumpsalmi{temporalia/ps150-initium-vii-a-auto.gtex}

\input{temporalia/ps150-vii-a.tex} \Abardot{}

\vfill
\pagebreak}
\newcommand{\lectiobrevis}{\pars{Lectio Brevis.} \scriptura{Eph. 2, 13-16}

\noindent Si crédimus quod Iesus mórtuus est et resurréxit, ita et Deus eos, qui dormiérunt, per Iesum addúcet cum eo.}
\newcommand{\responsoriumbreve}{\pars{Responsorium breve.} \scriptura{Ps. 29, 2}

\vspace{-5mm}

\responsorium{VI}{temporalia/resp-exaltabotedomine.gtex}{}}
\newcommand{\preces}{\noindent Deum Patrem omnipoténtem, \gredagger{} qui suscitávit Iesum Christum a mórtuis et vivificábit mortália córpora nostra,\grestar{} deprecémur, clamántes:

\Rbardot{} Dómine, vivífica nos in Christo.

\noindent Pater sancte, \gredagger{} nos consepúltos per baptísmum in mortem cum Fílio tuo et conresuscitátos in resurrectiónem eius sic tríbue in novitáte vitæ ambuláre, \grestar{} ut étiam, cum mórtui fuérimus, cum Christo semper vivámus.

\Rbardot{} Dómine, vivífica nos in Christo.

\noindent Próvidens Pater, \gredagger{} qui dedísti nobis panem vivum, qui de cælo descéndit, sancte semper manducándum, \grestar{} fac, ut vitam ætérnam habeámus et resuscitémur in novíssimo die.

\Rbardot{} Dómine, vivífica nos in Christo.

\noindent Dómine, qui Fílium tuum in agonía factum ab ángelo confortári dedísti, \grestar{} in éxitu nostro spe nos suávi consolári dignéris.

\Rbardot{} Dómine, vivífica nos in Christo.

\noindent Qui tres púeros de médio ignis eruísti, \grestar{} líbera ánimas defunctórum a pœnis, quas pro peccátis patiúntur.

\Rbardot{} Dómine, vivífica nos in Christo.

\noindent Deus vivórum et mortuórum, \gredagger{} qui Iesum a mórtuis suscitásti, \grestar{} súscita defúnctos et nos cum ipsis in glória ætérna constítue.

\Rbardot{} Dómine, vivífica nos in Christo.}
\newcommand{\benedictus}{\pars{Canticum Zachariæ.} \scriptura{Io. 11, 25-26; \textbf{H393}}

\vspace{-6mm}

\antiphona{II D}{temporalia/ant-egosumresurrectio.gtex}

\vspace{-3mm}

\scriptura{Lc. 1, 68-79}

\vspace{-2mm}

\initiumpsalmi{temporalia/benedictus-initium-ii-D-auto.gtex}

\vspace{-1.5mm}

\input{temporalia/benedictus-ii-D.tex} \Abardot{}}
%\newcommand{\benedicamuslaudes}{\cuminitiali{IV}{temporalia/benedicamus-feria-advequad.gtex}}
\include{hebdomadaxxviii}
% LuaLaTeX

\documentclass[a4paper, twoside, 12pt]{article}
\usepackage[latin]{babel}
%\usepackage[landscape, left=3cm, right=1.5cm, top=2cm, bottom=1cm]{geometry} % okraje stranky
%\usepackage[landscape, a4paper, mag=1166, truedimen, left=2cm, right=1.5cm, top=1.6cm, bottom=0.95cm]{geometry} % okraje stranky
\usepackage[landscape, a4paper, mag=1400, truedimen, left=0.5cm, right=0.5cm, top=0.5cm, bottom=0.5cm]{geometry} % okraje stranky

\usepackage{fontspec}
\setmainfont[FeatureFile={junicode.fea}, Ligatures={Common, TeX}, RawFeature=+fixi]{Junicode}
%\setmainfont{Junicode}

% shortcut for Junicode without ligatures (for the Czech texts)
\newfontfamily\nlfont[FeatureFile={junicode.fea}, Ligatures={Common, TeX}, RawFeature=+fixi]{Junicode}

\usepackage{multicol}
\usepackage{color}
\usepackage{lettrine}
\usepackage{fancyhdr}

% usual packages loading:
\usepackage{luatextra}
\usepackage{graphicx} % support the \includegraphics command and options
\usepackage{gregoriotex} % for gregorio score inclusion
\usepackage{gregoriosyms}
\usepackage{wrapfig} % figures wrapped by the text
\usepackage{parcolumns}
\usepackage[contents={},opacity=1,scale=1,color=black]{background}
\usepackage{tikzpagenodes}
\usepackage{calc}
\usepackage{longtable}
\usetikzlibrary{calc}

\setlength{\headheight}{14.5pt}

\input{conventuscommune.tex} % Often used macros

\newcommand{\annusEditionis}{2021}

%%%% Vicekrat opakovane kousky

\newcommand{\anteOrationem}{
  \rubrica{Ante Orationem, cantatur a Superiore:}

  \pars{Supplicatio Litaniæ.}

  \cuminitiali{}{temporalia/supplicatiolitaniae.gtex}

  \pars{Oratio Dominica.}

  \cuminitiali{}{temporalia/oratiodominica.gtex}

  \rubrica{Deinde dicitur ab Hebdomadario:}

  \cuminitiali{}{temporalia/dominusvobiscum-solemnis.gtex}

  \rubrica{In choro monialium loco Dominus vobiscum dicitur:}

  \sineinitiali{temporalia/domineexaudi.gtex}
}

\setlength{\columnsep}{30pt} % prostor mezi sloupci

%%%%%%%%%%%%%%%%%%%%%%%%%%%%%%%%%%%%%%%%%%%%%%%%%%%%%%%%%%%%%%%%%%%%%%%%%%%%%%%%%%%%%%%%%%%%%%%%%%%%%%%%%%%%%
\begin{document}

% Here we set the space around the initial.
% Please report to http://home.gna.org/gregorio/gregoriotex/details for more details and options
\grechangedim{afterinitialshift}{2.2mm}{scalable}
\grechangedim{beforeinitialshift}{2.2mm}{scalable}
\grechangedim{interwordspacetext}{0.22 cm plus 0.15 cm minus 0.05 cm}{scalable}%
\grechangedim{annotationraise}{-0.2cm}{scalable}

% Here we set the initial font. Change 38 if you want a bigger initial.
% Emit the initials in red.
\grechangestyle{initial}{\color{red}\fontsize{38}{38}\selectfont}

\pagestyle{empty}

%%%% Titulni stranka
\begin{titulusOfficii}
\ifx\titulus\undefined
\nomenFesti{Feria VI \hebdomada{}}
\else
\titulus
\fi
\end{titulusOfficii}

\vfill

\begin{center}
%Ad usum et secundum consuetudines chori \guillemotright{}Conventus Choralis\guillemotleft.

%Editio Sancti Wolfgangi \annusEditionis
\end{center}

\scriptura{}

\pars{}

\pagebreak

\renewcommand{\headrulewidth}{0pt} % no horiz. rule at the header
\fancyhf{}
\pagestyle{fancy}

\cantusSineNeumas

\hora{Ad Matutinum.} %%%%%%%%%%%%%%%%%%%%%%%%%%%%%%%%%%%%%%%%%%%%%%%%%%%%%

\vspace{2mm}

\cuminitiali{}{temporalia/dominelabiamea.gtex}

\vfill
%\pagebreak

\vspace{2mm}

\ifx\invitatorium\undefined
\pars{Invitatorium.} \scriptura{Lc. 24, 34; Psalmus 94; \textbf{H232}}

\antiphona{VI}{temporalia/inv-surrexitdominusvere.gtex}
\else
\invitatorium
\fi

\vfill
\pagebreak

\ifx\hymnusmatutinum\undefined
\pars{Hymnus.}

\cuminitiali{VIII}{temporalia/hym-LaetareCaelum.gtex}
\else
\hymnusmatutinum
\fi

\vspace{-3mm}

\vfill
\pagebreak

\ifx\matutinum\undefined
\ifx\matua\undefined
\else
% MAT A
\pars{Psalmus 1.}

\vspace{-4mm}

\antiphona{I a\textsuperscript{3}}{temporalia/ant-alleluia-turco24.gtex}

%\vspace{-2mm}

\scriptura{Ps. 34, 1-10}

%\vspace{-2mm}

\initiumpsalmi{temporalia/ps34i-initium-i-a5-auto.gtex}

\input{temporalia/ps34i-i-a5.tex}

\vfill
\pagebreak

\pars{Psalmus 2.} \scriptura{Ps. 34, 11-17}

%\vspace{-2mm}

\initiumpsalmi{temporalia/ps34ii-initium-i-a5-auto.gtex}

\input{temporalia/ps34ii-i-a5.tex}

\vfill
\pagebreak

\pars{Psalmus 3.} \scriptura{Ps. 34, 18-28}

\vspace{-2mm}

\initiumpsalmi{temporalia/ps34iii-initium-i-a5-auto.gtex}

\input{temporalia/ps34iii-i-a5.tex}

\vfill

\antiphona{}{temporalia/ant-alleluia-turco24.gtex}

\vfill
\pagebreak
\fi
\ifx\matub\undefined
\else
% MAT B
\pars{Psalmus 1.}

\vspace{-4mm}

\antiphona{D}{temporalia/ant-alleluia-turco2.gtex}

%\vspace{-2mm}

\scriptura{Ps. 37, 2-5}

%\vspace{-2mm}

\initiumpsalmi{temporalia/ps37ii_v-initium-d-g-auto.gtex}

\input{temporalia/ps37ii_v-d-g.tex}

\vfill
\pagebreak

\pars{Psalmus 2.}

\scriptura{Ps. 37, 6-13}

%\vspace{-2mm}

\initiumpsalmi{temporalia/ps37vi_xiii-initium-d-g-auto.gtex}

\input{temporalia/ps37vi_xiii-d-g.tex}

\vfill
\pagebreak

\pars{Psalmus 3.}

\scriptura{Ps. 37, 14-23}

%\vspace{-2mm}

\initiumpsalmi{temporalia/ps37xiv_xxiii-initium-d-g-auto.gtex}

\input{temporalia/ps37xiv_xxiii-d-g.tex}

\vfill

\antiphona{}{temporalia/ant-alleluia-turco2.gtex}

\vfill
\pagebreak
\fi
\ifx\matuc\undefined
\else
% MAT C
\pars{Psalmus 1.}

\vspace{-4mm}

\antiphona{I d\textsuperscript{3}}{temporalia/ant-alleluia-auglx5.gtex}

%\vspace{-3mm}

\scriptura{Ps. 68, 2-13}

%\vspace{-2mm}

\initiumpsalmi{temporalia/ps68ii_xiii-initium-i-d-auto.gtex}

%\vspace{-1.5mm}

\input{temporalia/ps68ii_xiii-i-d.tex}

\vfill
\pagebreak

\pars{Psalmus 2.}

\scriptura{Ps. 68, 14-22}

%\vspace{-2mm}

\initiumpsalmi{temporalia/ps68xiv_xxii-initium-i-d-auto.gtex}

\input{temporalia/ps68xiv_xxii-i-d.tex}

\vfill
\pagebreak

\pars{Psalmus 3.}

\scriptura{Ps. 68, 30-37}

%\vspace{-2mm}

\initiumpsalmi{temporalia/ps68iii-initium-i-d-auto.gtex}

\input{temporalia/ps68iii-i-d.tex}

\vfill

\antiphona{}{temporalia/ant-alleluia-auglx5.gtex}

\vfill
\pagebreak
\fi
\ifx\matud\undefined
\else
% MAT D
\pars{Psalmus 1.}

\vspace{-4mm}

\antiphona{I a\textsuperscript{2}}{temporalia/ant-alleluia-turco24.gtex}

%\vspace{-3mm}

\scriptura{Ps. 77, 1-16}

%\vspace{-2mm}

\initiumpsalmi{temporalia/ps77i_xvi-initium-i-a4-auto.gtex}

\input{temporalia/ps77i_xvi-i-a4.tex}

\vfill
\pagebreak

\pars{Psalmus 2.} \scriptura{Ps. 77, 17-31}

%\vspace{-2mm}

\initiumpsalmi{temporalia/ps77iii-initium-i-a4-auto.gtex}

\input{temporalia/ps77iii-i-a4.tex}

\vfill
\pagebreak

\pars{Psalmus 3.} \scriptura{Ps. 77, 32-39}

%\vspace{-2mm}

\initiumpsalmi{temporalia/ps77xxxii_xxxix-initium-i-a4-auto.gtex}

\input{temporalia/ps77xxxii_xxxix-i-a4.tex}

\vfill

\antiphona{}{temporalia/ant-alleluia-turco24.gtex}

\vfill
\pagebreak
\fi
\else
\matutinum
\fi

\pars{Versus.}

\ifx\matversus\undefined
\noindent \Vbardot{} In resurrectióne tua, Christe, allelúia.

\noindent \Rbardot{} Cæli et terra læténtur, allelúia.
\else
\matversus
\fi

\vspace{5mm}

\sineinitiali{temporalia/oratiodominica-mat.gtex}

\vspace{5mm}

\pars{Absolutio.}

\cuminitiali{}{temporalia/absolutio-ipsius.gtex}

\vfill
\pagebreak

\cuminitiali{}{temporalia/benedictio-solemn-deus.gtex}

\vspace{7mm}

\lectioi

\noindent \Vbardot{} Tu autem, Dómine, miserére nobis.
\noindent \Rbardot{} Deo grátias.

\vfill
\pagebreak

\responsoriumi

\vfill
\pagebreak

\cuminitiali{}{temporalia/benedictio-solemn-christus.gtex}

\vspace{7mm}

\lectioii

\noindent \Vbardot{} Tu autem, Dómine, miserére nobis.
\noindent \Rbardot{} Deo grátias.

\vfill
\pagebreak

\responsoriumii

\vfill
\pagebreak

\cuminitiali{}{temporalia/benedictio-solemn-ignem.gtex}

\vspace{7mm}

\lectioiii

\noindent \Vbardot{} Tu autem, Dómine, miserére nobis.
\noindent \Rbardot{} Deo grátias.

\vfill
\pagebreak

\responsoriumiii

\vfill
\pagebreak

\rubrica{Reliqua omittuntur, nisi Laudes separandæ sint.}

\sineinitiali{temporalia/domineexaudi.gtex}

\vfill

\oratio

\vfill

\noindent \Vbardot{} Dómine, exáudi oratiónem meam.
\Rbardot{} Et clamor meus ad te véniat.

\vfill

\noindent \Vbardot{} Benedicámus Dómino.
\noindent \Rbardot{} Deo grátias.

\vfill

\noindent \Vbardot{} Fidélium ánimæ per misericórdiam Dei requiéscant in pace.
\Rbardot{} Amen.

\vfill
\pagebreak

\hora{Ad Laudes.} %%%%%%%%%%%%%%%%%%%%%%%%%%%%%%%%%%%%%%%%%%%%%%%%%%%%%

\cantusSineNeumas

\vspace{0.5cm}
\grechangedim{interwordspacetext}{0.18 cm plus 0.15 cm minus 0.05 cm}{scalable}%
\cuminitiali{}{temporalia/deusinadiutorium-communis.gtex}
\grechangedim{interwordspacetext}{0.22 cm plus 0.15 cm minus 0.05 cm}{scalable}%

\vfill
\pagebreak

\ifx\hymnuslaudes\undefined
\ifx\laudac\undefined
\else
\pars{Hymnus}

\cuminitiali{I}{temporalia/hym-ChorusNovae-praglia.gtex}
\vspace{-3mm}
\fi
\ifx\laudbd\undefined
\else
\pars{Hymnus}

\cuminitiali{I}{temporalia/hym-ChorusNovae.gtex}
\vspace{-3mm}
\fi
\else
\hymnuslaudes
\fi

\vfill
\pagebreak

\ifx\laudes\undefined
\ifx\lauda\undefined
\else
\pars{Psalmus 1.}

\vspace{-4mm}

\antiphona{VI F}{temporalia/ant-alleluia-turco6.gtex}

\scriptura{Psalmus 50.}

\initiumpsalmi{temporalia/ps50-initium-vi-F-auto.gtex}

\input{temporalia/ps50-vi-F.tex}

\vfill

\antiphona{}{temporalia/ant-alleluia-turco6.gtex}

\vfill
\pagebreak

\pars{Psalmus 2.} \scriptura{Is. 45, 25}

\vspace{-4mm}

\antiphona{V a}{temporalia/ant-indominoiustificabitur-tp.gtex}

\scriptura{Canticum Isaiæ, Is. 45, 15-30}

%\vspace{-2mm}

\initiumpsalmi{temporalia/isaiae2-initium-v-a-auto.gtex}

\input{temporalia/isaiae2-v-a.tex}

\vfill

\antiphona{}{temporalia/ant-indominoiustificabitur-tp.gtex}

\vfill
\pagebreak

\pars{Psalmus 3.}

\vspace{-4mm}

\antiphona{IV* e}{temporalia/ant-alleluia-turco9.gtex}

\scriptura{Psalmus 99.}

\initiumpsalmi{temporalia/ps99-initium-iv_-e-auto.gtex}

\input{temporalia/ps99-iv_-e.tex} \Abardot{}

\vfill
\pagebreak
\fi
\ifx\laudb\undefined
\else
\pars{Psalmus 1.}

\vspace{-4mm}

\antiphona{VII a}{temporalia/ant-alleluia-turco29.gtex}

\scriptura{Psalmus 50.}

\initiumpsalmi{temporalia/ps50-initium-vii-a-auto.gtex}

\input{temporalia/ps50-vii-a.tex}

\vfill

\antiphona{}{temporalia/ant-alleluia-turco29.gtex}

\vfill
\pagebreak

\pars{Psalmus 2.} \scriptura{Hab. 3, 2; \textbf{H99}}

\vspace{-6mm}

\antiphona{IV* e}{temporalia/ant-domineaudivi-tp.gtex}

\vspace{-2mm}

\scriptura{Canticum Habacuc, Hab. 3, 2-19}

%\vspace{-2mm}

%\initiumpsalmi{temporalia/habacuc-initium-iv_-e-auto.gtex}
\initiumpsalmi{temporalia/habacuc-initium-iv_-e.gtex}

\input{temporalia/habacuc-iv_-e.tex}

\vfill

\antiphona{}{temporalia/ant-domineaudivi-tp.gtex}

\vfill
\pagebreak

\pars{Psalmus 3.}

\vspace{-4mm}

\antiphona{E}{temporalia/ant-alleluia-turco4.gtex}

\vspace{-2mm}

\scriptura{Psalmus 147.}

%\vspace{-3mm}

%\initiumpsalmi{temporalia/ps147-initium-e-auto.gtex}
\initiumpsalmi{temporalia/ps147-initium-e.gtex}

\input{temporalia/ps147-e.tex} \Abardot{}

\vfill
\pagebreak
\fi
\ifx\laudc\undefined
\else
\pars{Psalmus 1.}

\vspace{-4mm}

\antiphona{VIII G\textsuperscript{2}}{temporalia/ant-alleluia-turco13.gtex}

\scriptura{Psalmus 50.}

\initiumpsalmi{temporalia/ps50-initium-viii-G5-auto.gtex}

\input{temporalia/ps50-viii-G5.tex}

\vfill

\antiphona{}{temporalia/ant-alleluia-turco13.gtex}

\vfill
\pagebreak

\pars{Psalmus 2.}

\vspace{-4mm}

\antiphona{VIII G}{temporalia/ant-nonnosderelinquas-tp.gtex}

%\vspace{-2mm}

\scriptura{Canticum Ieremiæ, Ier. 14, 17-31}

%\vspace{-2mm}

\initiumpsalmi{temporalia/jeremiae2-initium-viii-G.gtex}

\input{temporalia/jeremiae2-viii-G.tex} \Abardot{}

\vfill
\pagebreak

\pars{Psalmus 3.}

\vspace{-4mm}

\antiphona{E}{temporalia/ant-alleluia-praglia-e2.gtex}

\vspace{-2mm}

\scriptura{Psalmus 99.}

%\vspace{-2mm}

\initiumpsalmi{temporalia/ps99-initium-e-auto.gtex}

\input{temporalia/ps99-e.tex} \Abardot{}

\vfill
\pagebreak
\fi
\ifx\laudd\undefined
\else
\pars{Psalmus 1.}

\vspace{-4mm}

\antiphona{I f}{temporalia/ant-alleluia-turco20.gtex}

\scriptura{Psalmus 50.}

\initiumpsalmi{temporalia/ps50-initium-i-f-auto.gtex}

\input{temporalia/ps50-i-f.tex}

\vfill

\antiphona{}{temporalia/ant-alleluia-turco20.gtex}

\vfill
\pagebreak

\pars{Psalmus 2.} \scriptura{Ac. 22, 14}

\vspace{-4mm}

\antiphona{VIII G}{temporalia/ant-beatiquilavantstolas.gtex}

%\vspace{-2mm}

\scriptura{Canticum Tobiæ, Tob. 13, 10-18}

%\vspace{-2mm}

\initiumpsalmi{temporalia/tobiae2-initium-viii-G-auto.gtex}

\input{temporalia/tobiae2-viii-G.tex} \Abardot{}

\vfill
\pagebreak

\pars{Psalmus 3.}

\vspace{-4mm}

\antiphona{VI F}{temporalia/ant-alleluia-turco5.gtex}

\vspace{-2mm}

\scriptura{Psalmus 147.}

%\vspace{-2mm}

\initiumpsalmi{temporalia/ps147-initium-vi-F-auto.gtex}

\input{temporalia/ps147-vi-F.tex} \Abardot{}

\vfill
\pagebreak
\fi
\else
\laudes
\fi

\ifx\lectiobrevis\undefined
\pars{Lectio Brevis.} \scriptura{Ac. 5, 30-32}

\noindent Deus patrum nostrórum suscitávit Iesum, quem vos interemístis suspendéntes in ligno; hunc Deus Príncipem et Salvatórem exaltávit déxtera sua ad dandam pæniténtiam Israel et remissiónem peccatórum. Et nos sumus testes horum verbórum, et Spíritus Sanctus, quem dedit Deus obœdiéntibus sibi.
\else
\lectiobrevis
\fi

\vfill

\ifx\responsoriumbreve\undefined
\pars{Responsorium breve.} \scriptura{Cf. Mt. 28, 6; Cf. Gal. 3, 13}

\cuminitiali{VI}{temporalia/resp-surrexitdominusdesepulcro.gtex}
\else
\responsoriumbreve
\fi

\vfill
\pagebreak

\benedictus

\vspace{-1cm}

\vfill
\pagebreak

\pars{Preces.}

\sineinitiali{}{temporalia/tonusprecum.gtex}

\ifx\preces\undefined
\ifx\lauda\undefined
\else
\noindent Deum Patrem, qui vitam novam per Christi resurrectiónem cóntulit nobis,~\gredagger{} súpplices exorémus:

\Rbardot{} Clarífica nos claritáte Christi.

\noindent Deus, qui opéribus tuis antíquam dispensatiónem manifestásti, terram creásti et fidélis es in ómnibus generatiónibus,~\gredagger{} exáudi nos, clementíssime Pater.

\Rbardot{} Clarífica nos claritáte Christi.

\noindent Purífica nos puritáte veritátis tuæ, et gressus nostros dírige in cordis sanctitáte,~\gredagger{} ut quod iustum est tibíque plácitum agámus.

\Rbardot{} Clarífica nos claritáte Christi.

\noindent Illúmina vultum tuum super nos,~\gredagger{} ut a peccáto liberáti bonis domus tuæ repleámur.

\Rbardot{} Clarífica nos claritáte Christi.

\noindent Qui per Christum nos tibi reconciliásti,~\gredagger{} pacem nobis largíre omnibúsque in orbe terrárum degéntibus.

\Rbardot{} Clarífica nos claritáte Christi.
\fi
\ifx\laudb\undefined
\else
\noindent Deus Pater Christum per Spíritum suscitávit, et étiam mortália córpora nostra vivificábit.~\gredagger{} Quare clamémus:

\Rbardot{} Dómine, vivífica nos Spíritu Sancto tuo.

\noindent Pater sancte, qui accepísti holocáustum Fílii tui, resúscitans eum ex mórtuis,~\gredagger{} súscipe hodiérnam nostram oblatiónem et perduc nos in vitam ætérnam.

\Rbardot{} Dómine, vivífica nos Spíritu Sancto tuo.

\noindent Opera nostra hódie propítius intuére,~\gredagger{} ut fiant ad glóriam tuam et ad ómnium sanctificatiónem.

\Rbardot{} Dómine, vivífica nos Spíritu Sancto tuo.

\noindent Opus nostrum hódie non sit vanum, sed univérsis homínibus insérviat~\gredagger{} et sic operántes ad regnum tuum fac nos perveníre.

\Rbardot{} Dómine, vivífica nos Spíritu Sancto tuo.

\noindent Aperi hódie óculos nostros et cor nostrum ad fratres,~\gredagger{} ut nos ínvicem amémus nobísque serviámus.

\Rbardot{} Dómine, vivífica nos Spíritu Sancto tuo.
\fi
\ifx\laudc\undefined
\else
\noindent Deum Patrem, qui vitam novam per Christi resurrectiónem cóntulit nobis,~\gredagger{} súpplices exorémus:

\Rbardot{} Clarífica nos claritáte Christi.

\noindent Deus, qui opéribus tuis antíquam dispensatiónem manifestásti, terram creásti et fidélis es in ómnibus generatiónibus,~\gredagger{} exáudi nos, clementíssime Pater.

\Rbardot{} Clarífica nos claritáte Christi.

\noindent Purífica nos puritáte veritátis tuæ, et gressus nostros dírige in cordis sanctitáte,~\gredagger{} ut quod iustum est tibíque plácitum agámus.

\Rbardot{} Clarífica nos claritáte Christi.

\noindent Illúmina vultum tuum super nos,~\gredagger{} ut a peccáto liberáti bonis domus tuæ repleámur.

\Rbardot{} Clarífica nos claritáte Christi.

\noindent Qui per Christum nos tibi reconciliásti,~\gredagger{} pacem nobis largíre omnibúsque in orbe terrárum degéntibus.

\Rbardot{} Clarífica nos claritáte Christi.
\fi
\ifx\laudd\undefined
\else
\noindent Deus Pater Christum per Spíritum suscitávit, et étiam mortália córpora nostra vivificábit.~\gredagger{} Quare clamémus:

\Rbardot{} Dómine, vivífica nos Spíritu Sancto tuo.

\noindent Pater sancte, qui accepísti holocáustum Fílii tui, resúscitans eum ex mórtuis,~\gredagger{} súscipe hodiérnam nostram oblatiónem et perduc nos in vitam ætérnam.

\Rbardot{} Dómine, vivífica nos Spíritu Sancto tuo.

\noindent Opera nostra hódie propítius intuére,~\gredagger{} ut fiant ad glóriam tuam et ad ómnium sanctificatiónem.

\Rbardot{} Dómine, vivífica nos Spíritu Sancto tuo.

\noindent Opus nostrum hódie non sit vanum, sed univérsis homínibus insérviat~\gredagger{} et sic operántes ad regnum tuum fac nos perveníre.

\Rbardot{} Dómine, vivífica nos Spíritu Sancto tuo.

\noindent Aperi hódie óculos nostros et cor nostrum ad fratres,~\gredagger{} ut nos ínvicem amémus nobísque serviámus.

\Rbardot{} Dómine, vivífica nos Spíritu Sancto tuo.
\fi 
\else
\preces
\fi

\vfill

\pars{Oratio Dominica.}

\cuminitiali{}{temporalia/oratiodominicaalt.gtex}

\vfill
\pagebreak

\rubrica{vel:}

\pars{Supplicatio Litaniæ.}

\cuminitiali{}{temporalia/supplicatiolitaniae.gtex}

\vfill

\pars{Oratio Dominica.}

\cuminitiali{}{temporalia/oratiodominica.gtex}

\vfill
\pagebreak

% Oratio. %%%
\oratio

\vspace{-1mm}

\vfill

\rubrica{Hebdomadarius dicit Dominus vobiscum, vel, absente sacerdote vel diacono, sic concluditur:}

\vspace{2mm}

\antiphona{C}{temporalia/dominusnosbenedicat.gtex}

\rubrica{Postea cantatur a cantore:}

\vspace{2mm}

\cuminitiali{VII}{temporalia/benedicamus-tempore-paschali.gtex}

\vspace{1mm}

\vfill
\pagebreak

\end{document}

