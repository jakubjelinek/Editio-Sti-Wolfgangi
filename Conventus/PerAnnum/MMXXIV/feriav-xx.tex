\newcommand{\titulus}{\nomenFesti{Beatæ Mariæ Virginis Reginæ.}
\dies{Die 22. Augusti.}}
\newcommand{\oratio}{\pars{Oratio.}

\noindent Deus, qui Fílii tui Genetrícem nostram constituísti matrem atque regínam, concéde propítius, ut, ipsíus intercessióne suffúlti, tuórum in regno cælésti consequámur glóriam filiórum.

\pars{Pro pace in universo mundo.} \scriptura{Sir. 50, 25; 2 Esdr. 4, 20; \textbf{H416}}

\vspace{-4mm}

\antiphona{II D}{temporalia/ant-dapacemdomine.gtex}

\vfill

\noindent Deus, a quo sancta desidéria, recta consília et iusta sunt ópera: da servis tuis illam, quam mundus dare non potest, pacem; ut et corda nostra mandátis tuis dédita, et hóstium subláta formídine, témpora sint tua protectióne tranquílla.

\noindent Per Dóminum nostrum Iesum Christum, Fílium tuum, qui tecum vivit et regnat in unitáte Spíritus Sancti, Deus, per ómnia sǽcula sæculórum.

\noindent \Rbardot{} Amen.}
\newcommand{\invitatorium}{\pars{Invitatorium.}

\vspace{-4mm}

\antiphona{VII}{temporalia/inv-christumregemquisuam.gtex}}
\newcommand{\hymnusmatutinum}{\pars{Hymnus.} \scriptura{Victorius Genovesi (\olddag{} 1967)}

\cuminitiali{VIII}{temporalia/hym-RerumSupremo.gtex}}
\newcommand{\matversus}{\noindent \Vbardot{} Da mihi intelléctum et servábo legem tuam.

\noindent \Rbardot{} Et custódiam illam in toto corde meo.}
\newcommand{\absolutio}{\cuminitiali{}{temporalia/absolutio-precibus.gtex}}
\newcommand{\benedictioi}{\cuminitiali{}{temporalia/benedictio-solemn-noscum.gtex}}
\newcommand{\benedictioii}{\cuminitiali{}{temporalia/benedictio-solemn-ipsavirgo.gtex}}
\newcommand{\benedictioiii}{\cuminitiali{}{temporalia/benedictio-solemn-pervirginem.gtex}}
\newcommand{\lectioi}{\pars{Lectio I.} \scriptura{Is. 11, 1-16}

\noindent De libro Isaíæ prophétæ.

\noindent Hæc dicit Dóminus Deus:

\noindent Egrediétur virga de stirpe Iesse, et flos de radíce eius ascéndet; et requiéscet super eum spíritus Dómini: spíritus sapiéntiæ et intelléctus, spíritus consílii et fortitúdinis, spíritus sciéntiæ et timóris Dómini; et delíciæ eius in timóre Dómini.

\noindent Non secúndum visiónem oculórum iudicábit neque secúndum audítum áurium decérnet; sed iudicábit in iustítia páuperes et decérnet in æquitáte pro mansuétis terræ; et percútiet terram virga oris sui et spíritu labiórum suórum interfíciet ímpium. Et erit iustítia cíngulum lumbórum eius, et fides cinctórium renum eius.

\noindent Habitábit lupus cum agno, et pardus cum hædo accubábit; vítulus et leo simul saginabúntur, et puer párvulus minábit eos. Vítula et ursus pascéntur, simul accubábunt cátuli eórum; et leo sicut bos cómedet páleas. Et ludet infans ab úbere super forámine áspidis; et in cavérnam réguli, qui ablactátus fúerit, manum suam mittet. Non nocébunt et non occídent in univérso monte sancto meo, quia plena erit terra sciéntia Dómini, sicut aquæ mare opériunt.

\noindent In die illa radix Iesse stat in signum populórum; ipsam gentes requírent, et erit sedes eius gloriósa.

\noindent Et erit in die illa: rursus exténdet Dóminus manum suam ad possidéndum resíduum pópuli sui, {\color{gray} quod relíctum erit ab Assýria et ab Ægýpto et a Phátros et ab Æthiópia et ab Elam et a Sénnaar et ab Emath et ab ínsulis maris; } et levábit signum in natiónes et congregábit prófugos Israel et dispérsos Iudæ cólliget a quáttuor plagis terræ.

\noindent {\color{gray} Et auferétur zelus Ephraim, et hostes Iudæ abscindéntur; Ephraim non æmulábitur Iudam, et Iudas non pugnábit contra Ephraim. Et volábunt in úmeros Philísthim ad mare, simul prædabúntur fílios oriéntis; in Edom et Moab exténdent manus suas, et fílii Ammon obœ́dient eis. Et exsiccábit Dóminus linguam maris Ægýpti et levábit manum suam super flumen in fortitúdine spíritus sui et percútiet illud in septem rivos, ita ut transíre fáciat eos calceátos.}

\noindent Et erit via resíduo pópulo meo, qui relinquétur ab Assýria, sicut fuit Israéli in die illa, qua ascéndit de terra Ægýpti».}
\newcommand{\responsoriumi}{\pars{Responsorium 1.} \scriptura{\Rbardot{} Is. 7, 14 \Vbardot{} Lc. 1, 32; \textbf{H17}}

\vspace{-5mm}

\responsorium{III}{temporalia/resp-eccevirgoconcipiet-CROCHU.gtex}{}}
\newcommand{\lectioii}{\pars{Lectio II.} \scriptura{Hom. 7: SCh 72, 188. 190. 192. 200}

\noindent Ex Homíliis sancti Amedéi Lausannénsis epíscopi.

\noindent Vide quam recto órdine citra assumptiónem effúlsit admirábile nomen Maríæ in univérsa terra, et fama celebérrima eius ubíque diffúsa est, priúsquam elevarétur magnificéntia eius super cælos. Decébat enim Matrem vírginem, et ob Nati sui honórem, primo terris regnáre et ita demum cælos cum glória suscípere; dilatári in ínfimis, ut supérna in plenitúdine sancta penetráret, transláta sicut de virtúte in virtútem, sic a claritáte in claritátem a Dómini Spíritu.

\noindent Igitur in carne præsens, futúri regni primítias prælibábat, et nunc excédens Deo ineffábili sublimitáte, nunc próximis condescéndens inenarrábili caritáte. Inde angélicis frequentabátur offíciis, hinc humáno famulátu venerabátur. Assistébat ei Gábriel cum ángelis; cui Ioánnes, gaudens sibi vírgini matrem vírginem in cruce commendátam, cum Apóstolis ministrábat. Illi regínam, isti vidére dóminam lætabántur, et utríque pio ei devotiónis afféctu obsequebántur.}
\newcommand{\responsoriumii}{\pars{Responsorium 2.} \scriptura{\Vbardot{} Lc. 1, 28; \textbf{H118}}

\vspace{-5mm}

\responsorium{IV}{temporalia/resp-benedictaetvenerabilis-CROCHU.gtex}{}}
\newcommand{\lectioiii}{\pars{Lectio III.}

\noindent At illa résidens in arce sublimíssima virtútum, et pélago divinórum áffluens charísmatum, abýssum gratiárum, qua cunctos excésserat, credénti et sitiénti pópulo largíssima emanatióne profundébat. Salútem namque corpóribus et ánimis medélam conferébat, potens suscitáre a morte carnis et ánimæ. Quis umquam ab ea æger vel tristis aut ignárus cæléstium mysteriórum ábiit? Quis non lætus et gaudens rédiit ad própria, impetráto a matre Dómini María quod vóluit?

\noindent Tantis bonis exúberans sponsa, sponsi mater únici, suávis et caríssima in delíciis, ut fons hortórum rationabílium, et púteus aquárum vivéntium et vivificántium, quæ fluunt ímpetu de divíno Líbano, a monte Sion usque ad circumfúsas quasque extérnas natiónes, pacis flúmina et gratiárum emanatiónes cǽlica infusióne derivábat. Igitur cum Virgo vírginum a Deo et Fílio suo rege regum, exsultántibus ángelis, cum lætántibus archángelis, et cælo láudibus acclamánte, deducerétur, impléta est prophetía Psalmístæ dicéntis ad Dóminum: \emph{Astitit regína a dextris tuis, in vestítu deauráto, circúmdata varietáte.}}
\newcommand{\responsoriumiii}{\pars{Responsorium 3.} \scriptura{\Rbardot{} Is. 61, 10; Ps. 44, 12; Ct. 1, 2 \Vbardot{} Ps. 44, 10; \textbf{H297}}

\vspace{-5mm}

\responsorium{VIII}{temporalia/resp-ornataminmonilibus-cumdox.gtex}{}}
\newcommand{\hymnuslaudes}{\pars{Hymnus.}

\cuminitiali{II}{temporalia/hym-OQuamGlorifica.gtex}}
\newcommand{\lectiobrevis}{\pars{Lectio Brevis.} \scriptura{Cf. Is. 61, 10}

\noindent Gaudens gaudébo in Dómino, et exsultábit ánima mea in Deo meo, quia índuit me vestiméntis salútis et induménto iustítiæ circúmdedit me, quasi sponsam ornátam monílibus suis.}
\newcommand{\responsoriumbreve}{\pars{Responsorium breve.} \scriptura{Lc. 1, 28}

\cuminitiali{VI}{temporalia/resp-avemaria-alt.gtex}}
\newcommand{\benedictus}{\pars{Canticum Zachariæ.} \scriptura{\textbf{H300}}

\vspace{-4mm}

\antiphona{II D}{temporalia/ant-beatamateretinnupta.gtex}

\vspace{-2mm}

\scriptura{Lc. 1, 68-79}

\vspace{-2mm}

\cantusSineNeumas
\initiumpsalmi{temporalia/benedictus-initium-ii-D-auto.gtex}

%\vspace{-1.5mm}

\input{temporalia/benedictus-ii-D.tex} \Abardot{}}
\newcommand{\preces}{\noindent Salvatórem nostrum celebrántes,~\gredagger{} qui ex María Vírgine nasci dignátus est,~\grestar{} exorémus dicéntes:

\Rbardot{} Intercédat pro nobis mater tua, Dómine.

\noindent Salvátor mundi,~\gredagger{} qui redemptiónis tuæ virtúte ab omni peccáti labe matrem tuam præservásti,~\grestar{} serva nos mundos a peccáto.

\Rbardot{} Intercédat pro nobis mater tua, Dómine.

\noindent Redémptor noster,~\gredagger{} qui Vírginem Maríam thálamum puríssimum habitatiónis tuæ et Spíritus Sancti fecísti sacrárium,~\grestar{} nos templum tui Spíritus fac perénne.

\Rbardot{} Intercédat pro nobis mater tua, Dómine.

\noindent Verbum ætérnum, quod matrem tuam docuísti óptimam sibi partem elígere,~\grestar{} tríbue nobis eam imitári, cibum quæréntes, qui permáneat in vitam ætérnam.

\Rbardot{} Intercédat pro nobis mater tua, Dómine.

\noindent Rex regum, qui matrem tuam córpore et ánima tecum voluísti in cælum assúmptam,~\grestar{} fac ut quæ sursum sunt semper cogitémus.

\Rbardot{} Intercédat pro nobis mater tua, Dómine.

\noindent Dómine cæli et terræ, qui Maríam regínam a dextris tuis astáre fecísti,~\grestar{} tríbue nos eiúsdem glóriæ meréri consórtium.

\Rbardot{} Intercédat pro nobis mater tua, Dómine.}
\newcommand{\benedicamuslaudes}{\cuminitiali{I}{temporalia/benedicamus-festis-bmv.gtex}}
\newcommand{\hebdomada}{infra Hebdom. XX post Pentecosten.}
\newcommand{\oratioLaudes}{\cuminitiali{}{temporalia/oratio20.gtex}}
\newcommand{\hiemalis}{Hiemalis.}

% LuaLaTeX

\documentclass[a4paper, twoside, 12pt]{article}
\usepackage[latin]{babel}
%\usepackage[landscape, left=3cm, right=1.5cm, top=2cm, bottom=1cm]{geometry} % okraje stranky
%\usepackage[landscape, a4paper, mag=1166, truedimen, left=2cm, right=1.5cm, top=1.6cm, bottom=0.95cm]{geometry} % okraje stranky
\usepackage[landscape, a4paper, mag=1400, truedimen, left=0.5cm, right=0.5cm, top=0.5cm, bottom=0.5cm]{geometry} % okraje stranky

\usepackage{fontspec}
\setmainfont[FeatureFile={junicode.fea}, Ligatures={Common, TeX}, RawFeature=+fixi]{Junicode}
%\setmainfont{Junicode}

% shortcut for Junicode without ligatures (for the Czech texts)
\newfontfamily\nlfont[FeatureFile={junicode.fea}, Ligatures={Common, TeX}, RawFeature=+fixi]{Junicode}

\usepackage{multicol}
\usepackage{color}
\usepackage{lettrine}
\usepackage{fancyhdr}

% usual packages loading:
\usepackage{luatextra}
\usepackage{graphicx} % support the \includegraphics command and options
\usepackage{gregoriotex} % for gregorio score inclusion
\usepackage{gregoriosyms}
\usepackage{wrapfig} % figures wrapped by the text
\usepackage{parcolumns}
\usepackage[contents={},opacity=1,scale=1,color=black]{background}
\usepackage{tikzpagenodes}
\usepackage{calc}
\usepackage{longtable}
\usetikzlibrary{calc}

\setlength{\headheight}{14.5pt}

\input{conventuscommune.tex} % Often used macros

\newcommand{\annusEditionis}{2021}

%%%% Vicekrat opakovane kousky

\newcommand{\anteOrationem}{
  \rubrica{Ante Orationem, cantatur a Superiore:}

  \pars{Supplicatio Litaniæ.}

  \cuminitiali{}{temporalia/supplicatiolitaniae.gtex}

  \pars{Oratio Dominica.}

  \cuminitiali{}{temporalia/oratiodominica.gtex}

  \rubrica{Deinde dicitur ab Hebdomadario:}

  \cuminitiali{}{temporalia/dominusvobiscum-solemnis.gtex}

  \rubrica{In choro monialium loco Dominus vobiscum dicitur:}

  \sineinitiali{temporalia/domineexaudi.gtex}
}

\setlength{\columnsep}{30pt} % prostor mezi sloupci

%%%%%%%%%%%%%%%%%%%%%%%%%%%%%%%%%%%%%%%%%%%%%%%%%%%%%%%%%%%%%%%%%%%%%%%%%%%%%%%%%%%%%%%%%%%%%%%%%%%%%%%%%%%%%
\begin{document}

% Here we set the space around the initial.
% Please report to http://home.gna.org/gregorio/gregoriotex/details for more details and options
\grechangedim{afterinitialshift}{2.2mm}{scalable}
\grechangedim{beforeinitialshift}{2.2mm}{scalable}
\grechangedim{interwordspacetext}{0.22 cm plus 0.15 cm minus 0.05 cm}{scalable}%
\grechangedim{annotationraise}{-0.2cm}{scalable}

% Here we set the initial font. Change 38 if you want a bigger initial.
% Emit the initials in red.
\grechangestyle{initial}{\color{red}\fontsize{38}{38}\selectfont}

\pagestyle{empty}

%%%% Titulni stranka
\begin{titulusOfficii}
\ifx\titulus\undefined
\nomenFesti{Feria V \hebdomada{}}
\else
\titulus
\fi
\end{titulusOfficii}

\vfill

\begin{center}
%Ad usum et secundum consuetudines chori \guillemotright{}Conventus Choralis\guillemotleft.

%Editio Sancti Wolfgangi \annusEditionis
\end{center}

\scriptura{}

\pars{}

\pagebreak

\renewcommand{\headrulewidth}{0pt} % no horiz. rule at the header
\fancyhf{}
\pagestyle{fancy}

\cantusSineNeumas

\ifx\oratio\undefined
\ifx\lauda\undefined
\else
\newcommand{\oratio}{\pars{Oratio.}

\noindent Omnípotens sempitérne Deus, véspere, mane et merídie maiestátem tuam supplíciter deprecámur, ut, expúlsis de córdibus nostris peccatórum ténebris, ad veram lucem, quæ Christus est, nos fácias perveníre.

\noindent Qui tecum vivit et regnat in unitáte Spíritus Sancti, Deus, per ómnia sǽcula sæculórum.

\noindent \Rbardot{} Amen.}
\fi
\fi

\hora{Ad Matutinum.} %%%%%%%%%%%%%%%%%%%%%%%%%%%%%%%%%%%%%%%%%%%%%%%%%%%%%
%\sideThumbs{Matutinum}

\vspace{2mm}

\cuminitiali{}{temporalia/dominelabiamea.gtex}

\vfill
%\pagebreak

\vspace{2mm}

\ifx\invitatorium\undefined
\pars{Invitatorium.} \scriptura{Ps. 94, 6; Psalmus 94; \textbf{H136}}

\vspace{-6mm}

\antiphona{E}{temporalia/inv-adoremusdominum.gtex}
\else
\invitatorium
\fi

\vfill
\pagebreak

\ifx\hymnusmatutinum\undefined
\ifx\matuac\undefined
\else
\pars{Hymnus.} \scriptura{Gregorius Magnus (+604)}

{
\grechangedim{interwordspacetext}{0.10 cm plus 0.15 cm minus 0.05 cm}{scalable}%
\antiphona{IV}{temporalia/hym-NoxAtra.gtex}
\grechangedim{interwordspacetext}{0.22 cm plus 0.15 cm minus 0.05 cm}{scalable}%
}
\fi
\else
\hymnusmatutinum
\fi

\vspace{-3mm}

\vfill
\pagebreak

\ifx\matua\undefined
\else
% MAT A
\pars{Psalmus 1.} \scriptura{Ps. 17, 3; \textbf{H99}}

\vspace{-4mm}

\antiphona{VIII G}{temporalia/ant-dominusfirmamentum.gtex}

%\vspace{-2mm}

\scriptura{Ps. 17, 31-35}

%\vspace{-2mm}

\initiumpsalmi{temporalia/ps17xxxi_xxxv-initium-viii-G-auto.gtex}

\input{temporalia/ps17xxxi_xxxv-viii-G.tex} \Abardot{}

\vfill
\pagebreak

\pars{Psalmus 2.} \scriptura{Ps. 62, 9; \textbf{H393}}

\vspace{-4mm}

\antiphona{VII c trans.}{temporalia/ant-mesuscepit.gtex}

%\vspace{-2mm}

\scriptura{Ps. 17, 36-46}

%\vspace{-2mm}

\initiumpsalmi{temporalia/ps17xxxvi_xlvi-initium-vii-c-trans.gtex}

\input{temporalia/ps17xxxvi_xlvi-vii-c.tex} \Abardot{}

\vfill
\pagebreak

\pars{Psalmus 3.} \scriptura{Ps. 17, 47; \textbf{H100}}

\vspace{-4mm}

\antiphona{VII c\textsuperscript{2}}{temporalia/ant-vivitdominus.gtex}

%\vspace{-2mm}

\scriptura{Ps. 17, 47-51}

%\vspace{-2mm}

\initiumpsalmi{temporalia/ps17xlvii_li-initium-vii-c2-auto.gtex}

\input{temporalia/ps17xlvii_li-vii-c2.tex} \Abardot{}

\vfill
\pagebreak
\fi
\ifx\matuc\undefined
\else
% MAT C
\pars{Psalmus 1.} \scriptura{Lam. 1, 21; \textbf{H177}}

\vspace{-4mm}

\antiphona{VII a}{temporalia/ant-omnesinimici.gtex}

%\vspace{-2mm}

\scriptura{Ps. 88, 39-46}

%\vspace{-2mm}

\initiumpsalmi{temporalia/ps88xxxix_xlvi-initium-vii-a-auto.gtex}

\input{temporalia/ps88xxxix_xlvi-vii-a.tex} \Abardot{}

\vfill
\pagebreak

\pars{Psalmus 2.} \scriptura{Ps. 88, 53; \textbf{H98}}

\vspace{-4mm}

\antiphona{VI F}{temporalia/ant-benedictusdominusinaeternum.gtex}

%\vspace{-2mm}

\scriptura{Ps. 88, 47-53}

%\vspace{-2mm}

\initiumpsalmi{temporalia/ps88xlvii_liii-initium-vi-F-auto.gtex}

\input{temporalia/ps88xlvii_liii-vi-F.tex} \Abardot{}

\vfill
\pagebreak

\pars{Psalmus 3.} \scriptura{Ps. 89, 13}

\vspace{-4mm}

\antiphona{I g}{temporalia/ant-converteredomine.gtex}

%\vspace{-2mm}

\scriptura{Ps. 89}

%\vspace{-2mm}

\initiumpsalmi{temporalia/ps89-initium-i-g-auto.gtex}

\input{temporalia/ps89-i-g.tex}

\vfill

\antiphona{}{temporalia/ant-converteredomine.gtex}

\vfill
\pagebreak
\fi

\pars{Versus.}

\ifx\matversus\undefined
\ifx\matua\undefined
\else
\noindent \Vbardot{} Révela, Dómine, óculos meos.

\noindent \Rbardot{} Et considerábo mirabília de lege tua.
\fi
\ifx\matuc\undefined
\else
\noindent \Vbardot{} Audies de ore meo verbum.

\noindent \Rbardot{} Et annuntiábis eis ex me.
\fi
\else
\matversus
\fi

\vspace{5mm}

\sineinitiali{temporalia/oratiodominica-mat.gtex}

\vspace{5mm}

\pars{Absolutio.}

\cuminitiali{}{temporalia/absolutio-exaudi.gtex}

\vfill
\pagebreak

\cuminitiali{}{temporalia/benedictio-solemn-benedictione.gtex}

\vspace{7mm}

\lectioi

\noindent \Vbardot{} Tu autem, Dómine, miserére nobis.
\noindent \Rbardot{} Deo grátias.

\vfill
\pagebreak

\responsoriumi

\vfill
\pagebreak

\cuminitiali{}{temporalia/benedictio-solemn-unigenitus.gtex}

\vspace{7mm}

\lectioii

\noindent \Vbardot{} Tu autem, Dómine, miserére nobis.
\noindent \Rbardot{} Deo grátias.

\vfill
\pagebreak

\responsoriumii

\vfill
\pagebreak

\cuminitiali{}{temporalia/benedictio-solemn-spiritus.gtex}

\vspace{7mm}

\lectioiii

\noindent \Vbardot{} Tu autem, Dómine, miserére nobis.
\noindent \Rbardot{} Deo grátias.

\vfill
\pagebreak

\responsoriumiii

\vfill
\pagebreak

\rubrica{Reliqua omittuntur, nisi Laudes separandæ sint.}

\sineinitiali{temporalia/domineexaudi.gtex}

\vfill

\oratio

\vfill

\noindent \Vbardot{} Dómine, exáudi oratiónem meam.
\Rbardot{} Et clamor meus ad te véniat.

\vfill

\noindent \Vbardot{} Benedicámus Dómino.
\noindent \Rbardot{} Deo grátias.

\vfill

\noindent \Vbardot{} Fidélium ánimæ per misericórdiam Dei requiéscant in pace.
\Rbardot{} Amen.

\vfill
\pagebreak

\hora{Ad Laudes.} %%%%%%%%%%%%%%%%%%%%%%%%%%%%%%%%%%%%%%%%%%%%%%%%%%%%%
%\sideThumbs{Laudes}

\cantusSineNeumas

\vspace{0.5cm}
\grechangedim{interwordspacetext}{0.18 cm plus 0.15 cm minus 0.05 cm}{scalable}%
\cuminitiali{}{temporalia/deusinadiutorium-communis.gtex}
\grechangedim{interwordspacetext}{0.22 cm plus 0.15 cm minus 0.05 cm}{scalable}%

\vfill
\pagebreak

\ifx\hymnuslaudes\undefined
\ifx\laudac\undefined
\else
\pars{Hymnus}

\grechangedim{interwordspacetext}{0.16 cm plus 0.15 cm minus 0.05 cm}{scalable}%
\cuminitiali{I}{temporalia/hym-SolEcce.gtex}
\grechangedim{interwordspacetext}{0.22 cm plus 0.15 cm minus 0.05 cm}{scalable}%
\vspace{-3mm}
\fi
\else
\hymnuslaudes
\fi

\vfill
\pagebreak

\ifx\lauda\undefined
\else
\pars{Psalmus 1.}

\vspace{-4mm}

\antiphona{VIII G}{temporalia/ant-exsurgamdiluculo.gtex}

%\vspace{-2mm}

\scriptura{Psalmus 56}

%\vspace{-2mm}

\initiumpsalmi{temporalia/ps56-initium-viii-g-auto.gtex}

%\vspace{-1.5mm}

\input{temporalia/ps56-viii-g.tex} \Abardot{}

\vfill
\pagebreak

\pars{Psalmus 2.} \scriptura{Ier. 31, 14}

\vspace{-4mm}

\antiphona{IV* e}{temporalia/ant-populusmeusait.gtex}

%\vspace{-2mm}

\scriptura{Canticum Ieremiæ, 1 Ier. 31, 10-14}

%\vspace{-3mm}

\initiumpsalmi{temporalia/jeremiae3-initium-iv_-e-auto.gtex}

\input{temporalia/jeremiae3-iv_-e.tex} \Abardot{}

\vfill
\pagebreak

\pars{Psalmus 3.} \scriptura{Ps. 95, 4; \textbf{H94}}

\vspace{-4mm}

\antiphona{IV a}{temporalia/ant-magnusdominus.gtex}

\scriptura{Psalmus 47}

\initiumpsalmi{temporalia/ps47-initium-iv-a-auto.gtex}

\input{temporalia/ps47-iv-a.tex} \Abardot{}

\vfill
\pagebreak
\fi
\ifx\laudc\undefined
\else
\pars{Psalmus 1.} \scriptura{Ps. 86, 1; \textbf{H98}}

\vspace{-4mm}

\antiphona{I g}{temporalia/ant-fundamentaeius.gtex}

%\vspace{-2mm}

\scriptura{Psalmus 86}

%\vspace{-2mm}

\initiumpsalmi{temporalia/ps86-initium-i-g-auto.gtex}

%\vspace{-1.5mm}

\input{temporalia/ps86-i-g.tex} \Abardot{}

\vfill
\pagebreak

\pars{Psalmus 2.}

\vspace{-4mm}

\antiphona{II D}{temporalia/ant-eccedominusnosterbrachio.gtex}

%\vspace{-2mm}

\scriptura{Canticum Isaiæ, Is. 40, 10-17}

%\vspace{-3mm}

\initiumpsalmi{temporalia/isaiae9-initium-ii-D-auto.gtex}

\input{temporalia/isaiae9-ii-D.tex} \Abardot{}

\vfill
\pagebreak

\pars{Psalmus 3.} \scriptura{Ps. 144, 17}

\vspace{-4mm}

\antiphona{E}{temporalia/ant-iustusetsanctus.gtex}

\scriptura{Psalmus 98}

\initiumpsalmi{temporalia/ps98-initium-e.gtex}

\input{temporalia/ps98-e.tex} \Abardot{}

\vfill
\pagebreak
\fi

\ifx\lectiobrevis\undefined
\ifx\lauda\undefined
\else
\pars{Lectio Brevis.} \scriptura{Is. 66, 1-2}

\noindent Hæc dicit Dóminus: Cælum thronus meus, terra autem scabéllum pedum meórum. Quæ ista domus, quam ædificábitis mihi, et quis iste locus quiétis meæ? Omnia hæc manus mea fecit et mea sunt univérsa ista, dicit Dóminus. Ad hunc autem respíciam, ad paupérculum et contrítum spíritu et treméntem sermónes meos.
\fi
\else
\lectiobrevis
\fi

\vfill

\ifx\responsoriumbreve\undefined
\ifx\laudac\undefined
\else
\pars{Responsorium breve.} \scriptura{Ps. 118, 145}

\cuminitiali{VI}{temporalia/resp-clamaviintotocorde.gtex}
\fi
\else
\responsoriumbreve
\fi

\vfill
\pagebreak

\ifx\benedictus\undefined
\ifx\laudac\undefined
\else
\pars{Canticum Zachariæ.} \scriptura{Lc. 1, 74.75; \textbf{H423}}

%\vspace{-4mm}

{
\grechangedim{interwordspacetext}{0.18 cm plus 0.15 cm minus 0.05 cm}{scalable}%
\antiphona{VII a}{temporalia/ant-insanctitate.gtex}
\grechangedim{interwordspacetext}{0.22 cm plus 0.15 cm minus 0.05 cm}{scalable}%
}

%\vspace{-3mm}

\scriptura{Lc. 1, 68-79}

%\vspace{-2mm}

\cantusSineNeumas
\initiumpsalmi{temporalia/benedictus-initium-vii-a-auto.gtex}

%\vspace{-1.5mm}

\input{temporalia/benedictus-vii-a.tex} \Abardot{}
\fi
\else
\benedictus
\fi

\vspace{-1cm}

\vfill
\pagebreak

%\sideThumbs{{\scriptsize{}Fine horarum}}

\pars{Preces.}

\sineinitiali{}{temporalia/tonusprecum.gtex}

\ifx\preces\undefined
\ifx\lauda\undefined
\else
\noindent Grátias agámus Christo, qui lumen huius diéi nobis concédit, \gredagger{} et ad eum clamémus:

\Rbardot{} Bénedic et sanctífica nos, Dómine.

\noindent Qui te pro peccátis nostris hóstiam obtulísti, \gredagger{} incépta et propósita suscípias hodiérna.

\Rbardot{} Bénedic et sanctífica nos, Dómine.

\noindent Qui óculos nostros lucis dono lætíficas novæ, \gredagger{} lúcifer oriáris in córdibus nostris.

\Rbardot{} Bénedic et sanctífica nos, Dómine.

\noindent Tríbue hódie nos esse ómnibus longánimes, \gredagger{} ut imitatóres tui fíeri possímus.

\Rbardot{} Bénedic et sanctífica nos, Dómine.

\noindent Audítam, Dómine, fac nobis mane misericórdiam tuam. \gredagger{} Sit hódie gáudium tuum fortitúdo nostra.

\Rbardot{} Bénedic et sanctífica nos, Dómine.
\fi
\ifx\laudc\undefined
\else
\noindent Christo, bono pastóri, qui pro suis óvibus ánimam pósuit, \gredagger{} laudes grati exsolvámus et supplicémus, dicéntes:

\Rbardot{} Pasce pópulum tuum, Dómine.

\noindent Christe, qui in sanctis pastóribus misericórdiam et dilectiónem tuam dignátus es osténdere, \gredagger{} numquam désinas per eos nobíscum misericórditer ágere.

\Rbardot{} Pasce pópulum tuum, Dómine.

\noindent Qui múnere pastóris animárum fungi per tuos vicários pergis, \gredagger{} ne destíteris nos ipse per rectóres nostros dirígere.

\Rbardot{} Pasce pópulum tuum, Dómine.

\noindent Qui in sanctis tuis, populórum dúcibus, córporum animarúmque médicus exstitísti, \gredagger{} numquam cesses ministérium in nos vitæ et sanctitátis perágere.

\Rbardot{} Pasce pópulum tuum, Dómine.

\noindent Qui, prudéntia et caritáte sanctórum, tuum gregem erudísti, \gredagger{} nos in sanctitáte iúgiter per pastóres nostros ædífica.

\Rbardot{} Pasce pópulum tuum, Dómine.
\fi
\else
\preces
\fi

\vfill

\pars{Oratio Dominica.}

\cuminitiali{}{temporalia/oratiodominicaalt.gtex}

\vfill
\pagebreak

\rubrica{vel:}

\pars{Supplicatio Litaniæ.}

\cuminitiali{}{temporalia/supplicatiolitaniae.gtex}

\vfill

\pars{Oratio Dominica.}

\cuminitiali{}{temporalia/oratiodominica.gtex}

\vfill
\pagebreak

% Oratio. %%%
\oratio

\vspace{-1mm}

\vfill

\rubrica{Hebdomadarius dicit Dominus vobiscum, vel, absente sacerdote vel diacono, sic concluditur:}

\vspace{2mm}

\antiphona{C}{temporalia/dominusnosbenedicat.gtex}

\rubrica{Postea cantatur a cantore:}

\vspace{2mm}

\cuminitiali{IV}{temporalia/benedicamus-feria-laudes.gtex}

\vspace{1mm}

\vfill
\pagebreak

\end{document}

