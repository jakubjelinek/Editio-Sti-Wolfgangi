\newcommand{\titulus}{\nomenFesti{Ss. Cornelii, Papæ, \& Cypriani, Episcopi, Martyrum.}
\dies{Die 16. Septembris.}}
\newcommand{\oratio}{\pars{Oratio.}

\noindent Deus, qui pópulo tuo beátos Cornélium et Cypriánum sédulos pastóres et invíctos mártyres præstitísti, concéde, ut, eórum intercessióne, fide et constántia roborémur et pro Ecclésiæ unitáte óperam tribuámus impénse.

\pars{Pro pace in universo mundo.} \scriptura{Sir. 50, 25; 2 Esdr. 4, 20; \textbf{H416}}

\vspace{-4mm}

\antiphona{II D}{temporalia/ant-dapacemdomine.gtex}

\vfill

\noindent Deus, a quo sancta desidéria, recta consília et iusta sunt ópera: da servis tuis illam, quam mundus dare non potest, pacem; ut et corda nostra mandátis tuis dédita, et hóstium subláta formídine, témpora sint tua protectióne tranquílla.

\noindent Per Dóminum nostrum Iesum Christum, Fílium tuum, qui tecum vivit et regnat in unitáte Spíritus Sancti, Deus, per ómnia sǽcula sæculórum.

\noindent \Rbardot{} Amen.}
\newcommand{\invitatorium}{\pars{Invitatorium.}

\vspace{-4mm}

\antiphona{E}{temporalia/inv-regemmartyrumsimplex.gtex}}
\newcommand{\hymnusmatutinum}{\pars{Hymnus}

\cuminitiali{IV}{temporalia/hym-RexGloriose.gtex}}
\newcommand{\lectioi}{\pars{Lectio I.} \scriptura{Tob. 2, 1-5.7-10a}

\noindent De libro Thobis.

\noindent Et sub Asarháddon rege descéndi in domum meam, et réddita est mihi uxor mea Anna et fílius meus Thobías. In Pentecóste, die festo nostro, qui est sanctus a Septimánis, factum est mihi prándium bonum, et discúbui, ut prandérem. Et appósita est mihi mensa, et vidi pulmentária complúra.

\noindent Et dixi Thobíae fílio meo: “Vade et, quemcúmque páuperem invéneris ex frátribus nostris, qui sunt captívi in Ninéve, qui in mente habet Dóminum in toto corde suo, hunc adduc, et manducábit páriter mecum; ecce sustíneo te, fili, donec vénias”.

\noindent Et ábiit Thobías quǽrere áliquem páuperem ex frátribus nostris et revérsus dixit mihi: “Pater!”. Et ego dixi illi: “Ecce ego, fili”.

\noindent Et respóndens ait: “Ecce unus ex natióne nostra occísus est et proiéctus est in foro et nunc íbidem láqueo suffocátus est”.

\noindent Et exsíliens relíqui prándium, ántequam ex illo gustárem, et sústuli eum de platéa in unam domum, donec sol cáderet, et illum sepelírem.

\noindent Et revérsus lavi et manducávi panem meum cum luctu. Et lacrimátus sum. Et, postquam sol occídit, ábii et fódiens sepelívi illum.

\noindent Et próximi mei deridébant me dicéntes: “Non timet adhuc hic homo; iam enim inquisítus est huius rei causa, ut occiderétur, et fugit et ecce íterum sépelit mórtuos”.

\noindent Et lavi ea nocte, postquam illum sepelívi, et introívi in átrium meum et obdormívi circa paríetem átrii, et fácies mea nuda erat propter æstum.

\noindent Et ignorábam quóniam pásseres in paríete super me erant, quorum stércora insedérunt in óculos meos cálida et induxérunt albúgines.

\noindent Et ibam ad médicos, ut curárer, et, quanto inunxérunt me medicaméntis, tanto magis óculi mei excæcabántur máculis, donec perexcaecátus sum.

\noindent Et eram inútilis meis óculis annis quáttuor, et omnes fratres mei dolébant pro me.}
\newcommand{\responsoriumi}{\pars{Responsorium 1.} \scriptura{\Rbardot{} Tob. 3, 15 \Vbardot{} ibid. 3, 2; \textbf{H407}}

\vspace{-5mm}

\responsorium{I}{temporalia/resp-petodomine-CROCHU.gtex}{}

\vfill

\rubrica{vel ad libitum:}

\vspace{3mm}

\pars{Responsorium 1.} \scriptura{\Rbardot{} Tob. 4, 20 \Vbardot{} ibid. 14, 10.11; \textbf{H407}}

\vspace{-5mm}

\responsorium{VIII}{temporalia/resp-omnitempore-CROCHU.gtex}{}}
\newcommand{\lectioii}{\pars{Lectio II.} \scriptura{Acta, 3-6: CSEL 3, 112-114}

\noindent Ex Actis proconsuláribus de martýrio sancti Cypriáni epíscopi.

\noindent Die octáva décima calendárum octóbrium mane multa turba convénit ad Sexti, secúndum præcéptum Galérii Máximi procónsulis. Et ita idem Galérius Máximus procónsul eádem die Cypriánum sibi offérri præcépit in átrio Saucíolo sedénti. Cumque oblátus fuísset, Galérius Máximus procónsul Cypriáno epíscopo dixit: «Tu es Tháscius Cypriánus?». Cypriánus epíscopus respóndit: «Ego sum».

\noindent Galérius Máximus procónsul dixit: «Tu papam te sacrílegæ mentis homínibus præbuísti?». Cypriánus epíscopus respóndit: «Ego».

\noindent Galérius Máximus procónsul dixit: «Iussérunt te sacratíssimi imperatóres cæremoniári». Cypriánus epíscopus dixit: «Non fácio».

\noindent Galérius Máximus ait: «Cónsule tibi». Cypriánus epíscopus respóndit: «Fac quod tibi præcéptum est; in re tam iusta, nulla est consultátio».

\noindent Galérius Máximus collocútus cum concílio senténtiam vix et ægre dixit verbis huiúsmodi: «Diu sacrílega mente vixísti et plúrimos nefáriæ tibi conspiratiónis hómines aggregásti et inimícum te diis Románis et religiónibus sacris constituísti, nec te pii et sacratíssimi príncipes Valeriánus et Galliénus Augústi et Valeriánus nobilíssimus Cæsar ad sectam cæremoniárum suárum revocáre potuérunt. Et ídeo cum sis nequissimórum críminum auctor et sígnifer deprehénsus, eris ipse documénto his quos scélere tuo tecum aggregásti: sánguine tuo sanciétur disciplína». Et his dictis decrétum ex tabélla recitávit: «Tháscium Cypriánum gládio animadvérti placet». Cypriánus epíscopus dixit: «Deo grátias».}
\newcommand{\responsoriumii}{\pars{Responsorium 2.} \scriptura{Ap. 7, 14; \textbf{H367}}

\vspace{-5mm}

\responsorium{I}{temporalia/resp-tradideruntcorpora-CROCHU.gtex}{}}
\newcommand{\lectioiii}{\pars{Lectio III.}

\noindent Post hanc vero senténtiam turba fratrum dicébat: «Et nos cum ipso decollémur». Propter hoc tumúltus fratrum exórtus est et multa turba eum prosecúta est. Et ita idem Cypriánus in agrum Sexti prodúctus est et ibi se lacérna byrro exspoliávit et genu in terra flexit et in oratiónem se Dómino prostrávit. Et cum se dalmática exspoliásset et diacónibus tradidísset, in línea stetit et cœpit spiculatórem sustinére.

\noindent Cum venísset autem spiculátor, iussit suis ut eídem spiculatóri vigínti quinque áureos darent. Linteámina vero et manuália a frátribus ante eum mittebántur. Póstea vero beátus Cypriánus manu sua óculos sibi texit. Qui cum lacínias manuáles ligáre sibi non potuísset, Iuliánus présbyter et Iuliánus subdiáconus ei ligavérunt.

\noindent Ita beátus Cypriánus passus est, eiúsque corpus propter gentílium curiositátem in próximo pósitum est. Inde per noctem sublátum cum céreis et scolácibus ad áreas Macróbii Candidiáni procuratóris, quæ sunt in via Mappaliénsi iuxta piscínas, cum voto et triúmpho magno dedúctum est. Post paucos autem dies Galérius Máximus procónsul decéssit.

\noindent Passus est autem beatíssimus Cypriánus martyr die octáva décima calendárum octóbrium sub Valeriáno et Galliéno imperatóribus, regnánte vero Dómino nostro Iesu Christo, cui est honor et glória in sǽcula sæculórum. Amen.}
\newcommand{\responsoriumiii}{\pars{Responsorium 3.} \scriptura{\Rbardot{} Mt. 5, 10.9 \Vbardot{} ibid., 8; \textbf{H333}}

\vspace{-5mm}

\responsorium{VII}{temporalia/resp-beatiquipersecutionem-CROCHU-cumdox.gtex}{}}
\newcommand{\hymnuslaudes}{\pars{Hymnus}

\cuminitiali{VIII}{temporalia/hym-AEternaChristi.gtex}}
\newcommand{\lectiobrevis}{\pars{Lectio Brevis.} \scriptura{2 Cor. 1, 3-5}

\noindent Benedíctus Deus et Pater Dómini nostri Iesu Christi, Pater misericordiárum et Deus totíus consolatiónis, qui consolátur nos in omni tribulatióne nostra, ut possímus et ipsi consolári eos, qui in omni pressúra sunt, per exhortatiónem, qua exhortámur et ipsi a Deo; quóniam, sicut abúndant passiónes Christi in nobis, ita per Christum abúndat et consolátio nostra.}
\newcommand{\responsoriumbreve}{\pars{Responsorium breve.} \scriptura{Sap. 5, 16}

\antiphona{VI}{temporalia/resp-iustiautem.gtex}}
\newcommand{\preces}{\noindent Fratres, Salvatórem nostrum, testem fidélem,~\gredagger{} per mártyres interféctos propter verbum Dei, celebrémus,~\grestar{} clamántes:

\Rbardot{} Redemísti nos Deo in sánguine tuo.

\noindent Per mártyres tuos, qui líbere mortem in testimónium fídei sunt ampléxi,~\grestar{} da nobis, Dómine, veram spíritus libertátem.

\Rbardot{} Redemísti nos Deo in sánguine tuo.

\noindent Per mártyres tuos, qui fidem usque ad sánguinem sunt conféssi,~\grestar{} da nobis, Dómine, puritátem fideíque constántiam.

\Rbardot{} Redemísti nos Deo in sánguine tuo.

\noindent Per mártyres tuos,~\gredagger{} qui, sustinéntes crucem, tua vestígia sunt secúti,~\grestar{} da nobis, Dómine, ærúmnas vitæ fórtiter sustinére.

\Rbardot{} Redemísti nos Deo in sánguine tuo.

\noindent Per mártyres tuos, qui stolas suas lavérunt in sánguine Agni,~\grestar{} da nobis, Dómine, omnes insídias carnis mundíque devíncere.

\Rbardot{} Redemísti nos Deo in sánguine tuo.}
\newcommand{\benedictus}{\pars{Canticum Zachariæ.} \scriptura{\textbf{H279}}

\vspace{-5mm}

\antiphona{I a\textsuperscript{2}}{temporalia/ant-astiteruntiusti.gtex}

\vspace{-3mm}

\scriptura{Lc. 1, 68-79}

\vspace{-2mm}

\cantusSineNeumas
\initiumpsalmi{temporalia/benedictus-initium-i-a4-auto.gtex}

\vspace{-1.5mm}

\input{temporalia/benedictus-i-a4.tex} \Abardot{}}
\newcommand{\benedicamuslaudes}{\cuminitiali{}{temporalia/benedicamus-memoria-laudes.gtex}}
\newcommand{\hebdomada}{infra Hebdom. XXIV et ultima post Pentecosten.}
\newcommand{\oratioLaudes}{\cuminitiali{}{temporalia/oratio24.gtex}}
\newcommand{\hiemalis}{Hiemalis.}

% LuaLaTeX

\documentclass[a4paper, twoside, 12pt]{article}
\usepackage[latin]{babel}
%\usepackage[landscape, left=3cm, right=1.5cm, top=2cm, bottom=1cm]{geometry} % okraje stranky
%\usepackage[landscape, a4paper, mag=1166, truedimen, left=2cm, right=1.5cm, top=1.6cm, bottom=0.95cm]{geometry} % okraje stranky
\usepackage[landscape, a4paper, mag=1400, truedimen, left=0.5cm, right=0.5cm, top=0.5cm, bottom=0.5cm]{geometry} % okraje stranky

\usepackage{fontspec}
\setmainfont[FeatureFile={junicode.fea}, Ligatures={Common, TeX}, RawFeature=+fixi]{Junicode}
%\setmainfont{Junicode}

% shortcut for Junicode without ligatures (for the Czech texts)
\newfontfamily\nlfont[FeatureFile={junicode.fea}, Ligatures={Common, TeX}, RawFeature=+fixi]{Junicode}

\usepackage{multicol}
\usepackage{color}
\usepackage{lettrine}
\usepackage{fancyhdr}

% usual packages loading:
\usepackage{luatextra}
\usepackage{graphicx} % support the \includegraphics command and options
\usepackage{gregoriotex} % for gregorio score inclusion
\usepackage{gregoriosyms}
\usepackage{wrapfig} % figures wrapped by the text
\usepackage{parcolumns}
\usepackage[contents={},opacity=1,scale=1,color=black]{background}
\usepackage{tikzpagenodes}
\usepackage{calc}
\usepackage{longtable}
\usetikzlibrary{calc}

\setlength{\headheight}{14.5pt}

\input{conventuscommune.tex} % Often used macros

\newcommand{\annusEditionis}{2021}

%%%% Vicekrat opakovane kousky

\newcommand{\anteOrationem}{
  \rubrica{Ante Orationem, cantatur a Superiore:}

  \pars{Supplicatio Litaniæ.}

  \cuminitiali{}{temporalia/supplicatiolitaniae.gtex}

  \pars{Oratio Dominica.}

  \cuminitiali{}{temporalia/oratiodominica.gtex}

  \rubrica{Deinde dicitur ab Hebdomadario:}

  \cuminitiali{}{temporalia/dominusvobiscum-solemnis.gtex}

  \rubrica{In choro monialium loco Dominus vobiscum dicitur:}

  \sineinitiali{temporalia/domineexaudi.gtex}
}

\setlength{\columnsep}{30pt} % prostor mezi sloupci

%%%%%%%%%%%%%%%%%%%%%%%%%%%%%%%%%%%%%%%%%%%%%%%%%%%%%%%%%%%%%%%%%%%%%%%%%%%%%%%%%%%%%%%%%%%%%%%%%%%%%%%%%%%%%
\begin{document}

% Here we set the space around the initial.
% Please report to http://home.gna.org/gregorio/gregoriotex/details for more details and options
\grechangedim{afterinitialshift}{2.2mm}{scalable}
\grechangedim{beforeinitialshift}{2.2mm}{scalable}
\grechangedim{interwordspacetext}{0.22 cm plus 0.15 cm minus 0.05 cm}{scalable}%
\grechangedim{annotationraise}{-0.2cm}{scalable}

% Here we set the initial font. Change 38 if you want a bigger initial.
% Emit the initials in red.
\grechangestyle{initial}{\color{red}\fontsize{38}{38}\selectfont}

\pagestyle{empty}

%%%% Titulni stranka
\begin{titulusOfficii}
\ifx\titulus\undefined
\nomenFesti{Feria II \hebdomada{}}
\else
\titulus
\fi
\end{titulusOfficii}

\vfill

\begin{center}
%Ad usum et secundum consuetudines chori \guillemotright{}Conventus Choralis\guillemotleft.

%Editio Sancti Wolfgangi \annusEditionis
\end{center}

\scriptura{}

\pars{}

\pagebreak

\renewcommand{\headrulewidth}{0pt} % no horiz. rule at the header
\fancyhf{}
\pagestyle{fancy}

\cantusSineNeumas

\ifx\oratio\undefined
\ifx\laudb\undefined
\else
\newcommand{\oratio}{\pars{Oratio.}

\noindent Dómine Deus omnípotens, qui ad princípium huius diéi nos perveníre fecísti, tua nos hódie salva virtúte, ut in hac die ad nullum declinémus peccátum, sed semper ad tuam iustítiam faciéndam nostra procédant elóquia, dirigántur cogitatiónes et ópera.

\noindent Per Dóminum nostrum Iesum Christum, Fílium tuum, qui tecum vivit et regnat in unitáte Spíritus Sancti, Deus, per ómnia sǽcula sæculórum.

\noindent \Rbardot{} Amen.}
\fi
\fi

\hora{Ad Matutinum.} %%%%%%%%%%%%%%%%%%%%%%%%%%%%%%%%%%%%%%%%%%%%%%%%%%%%%
%\sideThumbs{Matutinum}

\vspace{2mm}

\cuminitiali{}{temporalia/dominelabiamea.gtex}

\vfill
%\pagebreak

\vspace{2mm}

\ifx\invitatorium\undefined
\pars{Invitatorium.} \scriptura{Ps. 94, 1; Psalmus 94; \textbf{H451}}

\vspace{-6mm}

\antiphona{VI}{temporalia/inv-jubilemusdeo.gtex}\else
\invitatorium
\fi

\vfill
\pagebreak

\ifx\hymnusmatutinum\undefined
\ifx\matua\undefined
\else
\pars{Hymnus.}

{
\grechangedim{interwordspacetext}{0.10 cm plus 0.15 cm minus 0.05 cm}{scalable}%
\antiphona{II}{temporalia/hym-IpsumNunc.gtex}
\grechangedim{interwordspacetext}{0.22 cm plus 0.15 cm minus 0.05 cm}{scalable}%
}
\fi
\else
\hymnusmatutinum
\fi

\vspace{-3mm}

\vfill
\pagebreak

\ifx\matub\undefined
\else
% MAT B
\pars{Psalmus 1.} \scriptura{Ps. 30, 2; \textbf{H90}}

\vspace{-4mm}

\antiphona{VIII G}{temporalia/ant-intuaiustitia.gtex}

%\vspace{-2mm}

\scriptura{Ps. 30, 2-9}

%\vspace{-2mm}

\initiumpsalmi{temporalia/ps30i-initium-viii-G-auto.gtex}

\vspace{-1.5mm}

\input{temporalia/ps30i-viii-G.tex} \Abardot{}

\vfill
\pagebreak

\pars{Psalmus 2.} \scriptura{Ps. 66, 2}

\vspace{-4mm}

\antiphona{E}{temporalia/ant-illuminadomine.gtex}

%\vspace{-2mm}

\scriptura{Ps. 30, 10-17}

%\vspace{-2mm}

\initiumpsalmi{temporalia/ps30ii-initium-e-a-auto.gtex}

\input{temporalia/ps30ii-e-a.tex} \Abardot{}

\vfill
\pagebreak

\pars{Psalmus 3.} \scriptura{Ps. 30, 24}

\vspace{-4mm}

\antiphona{II D}{temporalia/ant-diligitedominum.gtex}

%\vspace{-5mm}

\scriptura{Ps. 30, 20-25}

%\vspace{-2mm}

\initiumpsalmi{temporalia/ps30iii-initium-ii-D-auto.gtex}

\input{temporalia/ps30iii-ii-D.tex} \Abardot{}

\vfill
\pagebreak
\fi

\pars{Versus.}

\ifx\matversus\undefined
\ifx\matub\undefined
\else
\noindent \Vbardot{} Dírige me, Dómine, in veritáte tua, et doce me.

\noindent \Rbardot{} Quia tu es Deus salútis meæ.
\fi
\else
\matversus
\fi

\vspace{5mm}

\sineinitiali{temporalia/oratiodominica-mat.gtex}

\vspace{5mm}

\pars{Absolutio.}

\cuminitiali{}{temporalia/absolutio-exaudi.gtex}

\vfill
\pagebreak

\cuminitiali{}{temporalia/benedictio-solemn-benedictione.gtex}

\vspace{7mm}

\lectioi

\noindent \Vbardot{} Tu autem, Dómine, miserére nobis.
\noindent \Rbardot{} Deo grátias.

\vfill
\pagebreak

\responsoriumi

\vfill
\pagebreak

\cuminitiali{}{temporalia/benedictio-solemn-unigenitus.gtex}

\vspace{7mm}

\lectioii

\noindent \Vbardot{} Tu autem, Dómine, miserére nobis.
\noindent \Rbardot{} Deo grátias.

\vfill
\pagebreak

\responsoriumii

\vfill
\pagebreak

\cuminitiali{}{temporalia/benedictio-solemn-spiritus.gtex}

\vspace{7mm}

\lectioiii

\noindent \Vbardot{} Tu autem, Dómine, miserére nobis.
\noindent \Rbardot{} Deo grátias.

\vfill
\pagebreak

\responsoriumiii

\vfill
\pagebreak

\rubrica{Reliqua omittuntur, nisi Laudes separandæ sint.}

\sineinitiali{temporalia/domineexaudi.gtex}

\vfill

\oratio

\vfill

\noindent \Vbardot{} Dómine, exáudi oratiónem meam.
\Rbardot{} Et clamor meus ad te véniat.

\vfill

\noindent \Vbardot{} Benedicámus Dómino.
\noindent \Rbardot{} Deo grátias.

\vfill

\noindent \Vbardot{} Fidélium ánimæ per misericórdiam Dei requiéscant in pace.
\Rbardot{} Amen.

\vfill
\pagebreak

\hora{Ad Laudes.} %%%%%%%%%%%%%%%%%%%%%%%%%%%%%%%%%%%%%%%%%%%%%%%%%%%%%
%\sideThumbs{Laudes}

\cantusSineNeumas

\vspace{0.5cm}
\grechangedim{interwordspacetext}{0.18 cm plus 0.15 cm minus 0.05 cm}{scalable}%
\cuminitiali{}{temporalia/deusinadiutorium-communis.gtex}
\grechangedim{interwordspacetext}{0.22 cm plus 0.15 cm minus 0.05 cm}{scalable}%

\vfill
\pagebreak

\ifx\hymnuslaudes\undefined
\ifx\laudbd\undefined
\else
\pars{Hymnus} \scriptura{Hilarius (\olddag{} 367)}

\grechangedim{interwordspacetext}{0.16 cm plus 0.15 cm minus 0.05 cm}{scalable}%
\cuminitiali{IV}{temporalia/hym-LucisLargitor.gtex}
\grechangedim{interwordspacetext}{0.22 cm plus 0.15 cm minus 0.05 cm}{scalable}%
\vspace{-3mm}
\fi
\else
\hymnuslaudes
\fi

\vfill
\pagebreak

\ifx\laudb\undefined
\else
\pars{Psalmus 1.} \scriptura{Ps. 41, 3; \textbf{H391}}

\vspace{-4mm}

\antiphona{II D}{temporalia/ant-sitivitanima.gtex}

%\vspace{-2mm}

\scriptura{Psalmus 41}

%\vspace{-2mm}

\initiumpsalmi{temporalia/ps41-initium-ii-D-auto.gtex}

%\vspace{-1.5mm}

\input{temporalia/ps41-ii-D.tex}

\vfill

\antiphona{}{temporalia/ant-sitivitanima.gtex}

\vfill
\pagebreak

\pars{Psalmus 2.}

\vspace{-4mm}

\antiphona{III a}{temporalia/ant-ostendenobisdomine.gtex}

%\vspace{-2mm}

\scriptura{Canticum Ecclesiastici, Sir. 36, 1-7.13-16}

%\vspace{-3mm}

\initiumpsalmi{temporalia/ecclesiastici-initium-iii-a-auto.gtex}

\input{temporalia/ecclesiastici-iii-a.tex} \Abardot{}

\vfill
\pagebreak

\pars{Psalmus 3.}

\vspace{-4mm}

\antiphona{II D}{temporalia/ant-operamanuumeius.gtex}

\scriptura{Psalmus 18, 1-7}

\initiumpsalmi{temporalia/ps18i-initium-ii-D-auto.gtex}

\input{temporalia/ps18i-ii-D.tex} \Abardot{}

\vfill
\pagebreak
\fi

\ifx\lectiobrevis\undefined
\ifx\laudb\undefined
\else
\pars{Lectio Brevis.} \scriptura{Ier. 15, 16}

\noindent Invénti sunt sermónes tui, et comédi eos, et factum est mihi verbum tuum in gáudium et in lætítiam cordis mei, quóniam invocátum est nomen tuum super me, Dómine Deus exercítuum.
\fi
\else
\lectiobrevis
\fi

\vfill

\ifx\responsoriumbreve\undefined
\ifx\laudbd\undefined
\else
\pars{Responsorium breve.} \scriptura{Ps. 32, 1.3}

\cuminitiali{VI}{temporalia/resp-exsultateiusti.gtex}
\fi
\else
\responsoriumbreve
\fi

\vfill
\pagebreak

\ifx\benedictus\undefined
\ifx\laudbd\undefined
\else
\pars{Canticum Zachariæ.} \scriptura{Lc. 1, 68; \textbf{H422}}

\vspace{-4mm}

{
\grechangedim{interwordspacetext}{0.18 cm plus 0.15 cm minus 0.05 cm}{scalable}%
\antiphona{IV E}{temporalia/ant-benedictusdominus.gtex}
\grechangedim{interwordspacetext}{0.22 cm plus 0.15 cm minus 0.05 cm}{scalable}%
}

%\vspace{-3mm}

\scriptura{Lc. 1, 68-79}

%\vspace{-2mm}

\cantusSineNeumas
\initiumpsalmi{temporalia/benedictus-initium-iv-E-auto.gtex}

%\vspace{-1.5mm}

\input{temporalia/benedictus-iv-E.tex} \Abardot{}
\fi
\else
\benedictus
\fi

\vspace{-1cm}

\vfill
\pagebreak

%\sideThumbs{{\scriptsize{}Fine horarum}}

\pars{Preces.}

\sineinitiali{}{temporalia/tonusprecum.gtex}

\ifx\preces\undefined
\ifx\laudb\undefined
\else
\noindent Salvátor noster fecit nos regnum et sacerdótium, ut hóstias Deo acceptábiles offerámus. \gredagger{} Grati ígitur eum invocémus:

\Rbardot{} Serva nos in tuo ministério, Dómine.

\noindent Christe, sacérdos ætérne, qui sanctum pópulo tuo sacerdótium concessísti, \gredagger{} concéde, ut spiritáles hóstias Deo acceptábiles iúgiter offerámus.

\Rbardot{} Serva nos in tuo ministério, Dómine.

\noindent Spíritus tui fructus nobis largíre propítius, \gredagger{} patiéntiam, benignitátem et mansuetúdinem.

\Rbardot{} Serva nos in tuo ministério, Dómine.

\noindent Da nobis te amáre, ut te, qui es cáritas, possideámus, \gredagger{} et bene ágere, ut per vitam étiam nostram te laudémus.

\Rbardot{} Serva nos in tuo ministério, Dómine.

\noindent Quæ frátribus nostris sunt utília, nos quǽrere concéde, \gredagger{} ut salútem facílius consequántur.

\Rbardot{} Serva nos in tuo ministério, Dómine.
\fi
\else
\preces
\fi

\vfill

\pars{Oratio Dominica.}

\cuminitiali{}{temporalia/oratiodominicaalt.gtex}

\vfill
\pagebreak

\rubrica{vel:}

\pars{Supplicatio Litaniæ.}

\cuminitiali{}{temporalia/supplicatiolitaniae.gtex}

\vfill

\pars{Oratio Dominica.}

\cuminitiali{}{temporalia/oratiodominica.gtex}

\vfill
\pagebreak

% Oratio. %%%
\oratio

\vspace{-1mm}

\vfill

\rubrica{Hebdomadarius dicit Dominus vobiscum, vel, absente sacerdote vel diacono, sic concluditur:}

\vspace{2mm}

\antiphona{C}{temporalia/dominusnosbenedicat.gtex}

\rubrica{Postea cantatur a cantore:}

\vspace{2mm}

\cuminitiali{IV}{temporalia/benedicamus-feria-laudes.gtex}

\vspace{1mm}

\vfill
\pagebreak

\end{document}

