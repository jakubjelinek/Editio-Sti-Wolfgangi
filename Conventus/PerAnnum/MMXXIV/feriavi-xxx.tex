\newcommand{\titulus}{\nomenFesti{Omnium Sanctorum.}
\dies{Die 1. Novembris.}}
\newcommand{\oratio}{\pars{Oratio.}

\noindent Omnípotens sempitérne Deus, qui nos ómnium sanctórum tuórum mérita sub una tribuísti celebritáte venerári, quǽsumus, ut desiderátam nobis tuæ propitiatiónis abundántiam, multiplicátis intercessóribus, largiáris.

\pars{Pro pace in universo mundo.} \scriptura{Sir. 50, 25; 2 Esdr. 4, 20; \textbf{H416}}

\vspace{-4mm}

\antiphona{II D}{temporalia/ant-dapacemdomine.gtex}

\vfill

\noindent Deus, a quo sancta desidéria, recta consília et iusta sunt ópera: da servis tuis illam, quam mundus dare non potest, pacem; ut et corda nostra mandátis tuis dédita, et hóstium subláta formídine, témpora sint tua protectióne tranquílla.

\noindent Per Dóminum nostrum Iesum Christum, Fílium tuum, qui tecum vivit et regnat in unitáte Spíritus Sancti, Deus, per ómnia sǽcula sæculórum.

\noindent \Rbardot{} Amen.}
\newcommand{\invitatorium}{\pars{Invitatorium.}

\vspace{-4mm}

\antiphona{VII}{temporalia/inv-deumquiglorificatur.gtex}}
\newcommand{\hymnusmatutinum}{\pars{Hymnus.}

\antiphona{I}{temporalia/hym-ChristeCaelorum.gtex}}
\newcommand{\matutinum}{\pars{Psalmus 1.} \scriptura{Ps. 8, 2; \textbf{H331}}

\vspace{-4mm}

\antiphona{I g}{temporalia/ant-admirabileestnomentuum.gtex}

\scriptura{Psalmus 8.}

\initiumpsalmi{temporalia/ps8-initium-i-g-auto.gtex}

\input{temporalia/ps8-i-g.tex} \Abardot{}

\vfill
\pagebreak

\pars{Psalmus 2.} \scriptura{Ps. 14, 1.2; \textbf{H331}}

\vspace{-4mm}

\antiphona{VI F}{temporalia/ant-dominequioperatisunt.gtex}

\scriptura{Psalmus 14.}

\initiumpsalmi{temporalia/ps14-initium-vi-F-auto.gtex}

\input{temporalia/ps14-vi-F.tex} \Abardot{}

\vfill
\pagebreak

\pars{Psalmus 3.} \scriptura{Ps. 15, 3; \textbf{H66}}

\vspace{-4mm}

\antiphona{IV* e}{temporalia/ant-sanctisquiinterrasunteius.gtex}

\scriptura{Psalmus 15.}

\initiumpsalmi{temporalia/ps15-initium-iv_-e-auto.gtex}

\input{temporalia/ps15-iv_-e.tex} \Abardot{}

\vfill
\pagebreak}
\newcommand{\matversus}{\noindent \Vbardot{} Respícite ad Dóminum et illuminámini.

\noindent \Rbardot{} Et fácies vestræ non confundéntur.}
\newcommand{\lectioi}{\pars{Lectio I.} \scriptura{Ap. 5, 1-14}

\noindent De libro Apocalýpsis beáti Ioánnis apóstoli.

\noindent Ego Ioánnes vidi in déxtera sedéntis super thronum librum scriptum intus et foris, signátum sigíllis septem. 

\noindent Et vidi ángelum fortem prædicántem voce magna: «Quis est dignus aperíre librum et sólvere signácula eius?». 

\noindent Et nemo póterat in cælo neque in terra neque subtus terram aperíre librum neque respícere illum. 

\noindent Et ego flebam multum, quóniam nemo dignus invéntus est aperíre librum nec respícere eum. 

\noindent Et unus de senióribus dicit mihi: «Ne fléveris; ecce vicit leo de tribu Iudæ, radix David, aperíre librum et septem signácula eius».

\noindent Et vidi in médio throni et quáttuor animálium et in médio seniórum Agnum stantem tamquam occísum, habéntem córnua septem et óculos septem, qui sunt septem spíritus Dei missi in omnem terram. 

\noindent Et venit et accépit de déxtera sedéntis in throno. 

\noindent Et cum accepísset librum, quáttuor animália et vigínti quáttuor senióres cecidérunt coram Agno, habéntes sínguli cítharas et phíalas áureas plenas incensórum, quæ sunt oratiónes sanctórum. 

\noindent Et cantant novum cánticum dicéntes: «Dignus es accípere librum et aperíre signácula eius, quóniam occísus es et redemísti Deo in sánguine tuo ex omni tribu et lingua et pópulo et natióne et fecísti eos Deo nostro regnum et sacerdótes et regnábunt super terram». 

\noindent Et vidi et audívi vocem angelórum multórum in circúitu throni et animálium et seniórum, et erat númerus eórum myríades myríadum et mília mílium dicéntium voce magna: «Dignus est Agnus, qui occísus est, accípere virtútem et divítias et sapiéntiam et fortitúdinem et honórem et glóriam et benedictiónem». 

\noindent Et omnem creatúram, quæ in cælo est et super terram et sub terra et super mare et quæ in eis ómnia, audívi dicéntes: «Sedénti super thronum et Agno benedíctio et honor et glória et potéstas in sǽcula sæculórum». 

\noindent Et quáttuor animália dicébant: «Amen»; et senióres cecidérunt et adoravérunt.}
\newcommand{\responsoriumi}{\pars{Responsorium 1.} \scriptura{\Vbardot{} Mt. 25, 34; \textbf{H331}}

\vspace{-5mm}

\responsorium{VIII}{temporalia/resp-sanctimei-CROCHU-sinedox.gtex}{}

\vfill

\rubrica{vel ad libitum:}

\vspace{3mm}

\pars{Responsorium 1.} \scriptura{\Vbardot{} Sap. 5, 6; \textbf{H369}}

\vspace{-5mm}

\responsorium{VII}{temporalia/resp-fulgebuntiusti-CROCHU.gtex}{}}
\newcommand{\lectioii}{\pars{Lectio II.} \scriptura{Sermo 2: Opera omnia, Edit. Cisterc. 5 [1968], 364-368}

\noindent Ex Sermónibus sancti Bernárdi abbátis.

\noindent Ad quid ergo sanctis laus nostra, ad quid glorificátio nostra, ad quid nostra hæc ipsa sollémnitas? 

\noindent Quo eis terrénos honóres, quos iuxta verácem Fílii promissiónem honoríficat Pater cæléstis? 

\noindent Quo eis præcónia nostra? 

\noindent Honórum nostrórum sancti non egent, nec quidquam eis nostra devotióne præstátur. 

\noindent Plane quod eórum memóriam venerámur, nostra ínterest, non ipsórum. 

\noindent Ego in me, fáteor, ex hac recordatióne séntio desidérium véhemens inflammári.

\noindent Hoc enim primum desidérium, quod in nobis sanctórum memória vel éxcitat vel íncitat magis, ut eórum tam optábili societáte fruámur et mereámur concíves et contubernáles esse spirítuum beatórum, miscéri cœ́tui patriarchárum, cúneis prophetárum, senátui Apostolórum, mártyrum exercítibus numerósis, confessórum collégiis, vírginum choris, in ómnium dénique cólligi et collætári communióne sanctórum. 

\noindent Præstolátur nos Ecclésia illa primitivórum et neglégimus; desíderant nos sancti et parvi péndimus; exspéctant nos iusti, et dissimulámus.

\noindent Excitémur aliquándo, fratres; resurgámus cum Christo, quærámus quæ sursum sunt, quæ sursum sunt sapiámus. 

\noindent Desiderémus desiderántes nos, properémus ad præstolántes nos, exspectántes nos votis præoccupémus animórum. 

\noindent Non modo tantum socíetas, sed étiam felícitas nobis est optánda sanctórum, ut quorum desiderámus præséntiam, glóriam quoque ferventíssimis stúdiis ambiámus. 

\noindent Neque enim perniciósa ambítio hæc, aut illíus affectátio glóriæ ullátenus periculósa est.}
\newcommand{\responsoriumii}{\pars{Responsorium 2.} \scriptura{\Rbardot{} Mt. 5, 10.9 \Vbardot{} ibid., 8; \textbf{H333}}

\vspace{-5mm}

\responsorium{VII}{temporalia/resp-beatiquipersecutionem-CROCHU.gtex}{}}
\newcommand{\lectioiii}{\pars{Lectio III.}

\noindent Hoc ergo secúndum desidérium, quod ex sanctórum commemoratióne flagrat in nobis, ut sicut illis sic étiam nobis Christus appáreat, vita nostra, et nos quoque cum ipso appareámus in glória. 

\noindent Interim nempe non sicut est, sed sicut pro nobis factum est, caput nostrum nobis repræsentátur, non coronátum glória, sed peccatórum nostrórum circúmdatum spinis. 

\noindent Púdeat sub spináto cápite membrum fíeri delicátum, quod omnis ei ínterim púrpura non tam honóris sit quam irrisiónis. 

\noindent Erit cum vénerit Christus, nec mors eius ultra annuntiábitur, ut sciámus quóniam ipsi quoque mórtui sumus, et cum eo abscóndita est vita nostra. 

\noindent Apparébit caput gloriósum et cum eo membra glorificáta fulgébunt, cum vidélicet reformábit corpus humilitátis nostræ configurátum glóriæ cápitis, quod est ipse.

\noindent Hanc ergo glóriam tota et tuta ambitióne concupiscámus. 

\noindent Sane ut eam nobis speráre líceat et ad tantam beatitúdinem aspiráre, summópere nobis desideránda sunt suffrágia quoque sanctórum, ut quod possibílitas nostra non óbtinet, eórum nobis intercessióne donétur.}
\newcommand{\responsoriumiii}{\pars{Responsorium 3.} \scriptura{\Rbardot{} Mt. 5, 3.5-6 \Vbardot{} ibid., 7; \textbf{H333}}

\vspace{-5mm}

\responsorium{VII}{temporalia/resp-beatipauperesspiritu-CROCHU.gtex}{}

\vfill
\pagebreak

\pars{Cantica.} \scriptura{Tob. 13, 10}

\vspace{-4mm}

\antiphona{VIII G}{temporalia/ant-benedicitedominumomneselecti.gtex}

\scriptura{Canticum Tobiæ, Tob. 13, 1-10}

\initiumpsalmi{temporalia/tobiae-initium-viii-g-auto.gtex}

\input{temporalia/tobiae-viii-g.tex}

\vfill
\pagebreak

\scriptura{Canticum Tobiæ, Tob. 13, 11-15}

\initiumpsalmi{temporalia/tobiae_xi_xv-initium-viii-g-auto.gtex}

\input{temporalia/tobiae_xi_xv-viii-g.tex}

\vfill
\pagebreak

\scriptura{Canticum Tobiæ, Tob. 13, 17-23}

\initiumpsalmi{temporalia/tobiae_xvii_xxiii-initium-viii-g-auto.gtex}

\input{temporalia/tobiae_xvii_xxiii-viii-g.tex}

\antiphona{}{temporalia/ant-benedicitedominumomneselecti.gtex}

\vfill
\pagebreak

\pars{Versus.}

\noindent \Vbardot{} Lætámini in Dómino, et exsultáte iusti.

\noindent \Rbardot{} Et gloriámini omnes recti corde.

\vspace{5mm}

\sineinitiali{temporalia/oratiodominica-mat.gtex}

\vspace{5mm}

\pars{Absolutio.}

\cuminitiali{}{temporalia/absolutio-avinculis.gtex}

\vfill
\pagebreak

\cuminitiali{}{temporalia/benedictio-solemn-evangelica.gtex}

\vspace{7mm}

\pars{Evangelium} \scriptura{Mt. 5, 1-12a}

\noindent Léctio sancti Evangélii secúndum Matthǽum.

\noindent In illo témpore: Videns Iesus turbas, ascéndit in montem, et cum sedísset, accessérunt ad eum discípuli eius, et apériens os suum docébat eos dicens:

\noindent Beáti páuperes spíritu: quóniam ipsórum est regnum cælórum.

\noindent Beáti mites: quóniam ipsi possidébunt terram.

\noindent Beáti qui lugent: quóniam ipsi consolabúntur.

\noindent Beáti qui esúriunt et sítiunt iustítiam: quóniam ipsi saturabúntur.

\noindent Beáti misericórdes: quóniam ipsi misericórdiam consequéntur.

\noindent Beáti mundo corde: quóniam ipsi Deum vidébunt.

\noindent Beáti pacífici: quóniam fílii Dei vocabúntur.

\noindent Beáti qui persecutiónem patiúntur propter iustítiam: quóniam ipsórum est regnum cælórum.

\noindent Beáti estis cum maledíxerint vobis, et persecúti vos fúerint, et díxerint omne malum advérsum vos mentiéntes, propter me: gaudéte, et exsultáte, quóniam merces vestra copiósa est in cælis.

\scriptura{Lib. 1, 3 : CCL 35, 7}

\noindent Ex Libris sancti Augustíni epíscopi \emph{De sermóne Dómini in monte}.

\noindent Sunt istæ omnes octo senténtiæ. Iste sententiárum númerus diligénter considerándus est.

\noindent Incipit enim beatitúdo ab humilitáte: \emph{Beáti páuperes spíritu,} id est non infláti, dum se divínæ auctoritáti subdit ánima timens post hanc vitam ne pergat ad pœnas, etiámsi forte in hac vita sibi beáta esse videátur.

\noindent Inde venit ad divinárum Scripturárum cognitiónem, ubi opórtet eam se mitem præbére pietáte, ne id quod imperítis vidétur absúrdum vituperáre áudeat, et pervicácibus concertatiónibus efficiátur indócilis.

\noindent  Inde iam íncipit scire, quibus nodis sǽculi huius per carnálem consuetúdinem ac peccáta teneátur. Itaque in hoc tértio gradu, in quo sciéntia est, lugétur amíssio summi boni, quia inhærétur extrémis.

\noindent In quarto autem gradu labor est, ubi veheménter incúmbitur, ut sese ánimus avéllat ab eis quibus pestífera dulcédine innéxus est. Hic ergo esurítur et sitítur iustítia, et multum necessária fortitúdo, quia non relínquitur sine dolóre quod cum delectatióne retinétur.

\noindent Quinto autem gradu perseverántibus in labóre datur evadéndi consílium, quin nisi quisque adiuvétur a superióre, nullo modo sibi est idóneus, ut sese tantis miseriárum implicaméntis expédiat. Est autem iustum consílium, ut qui se a potentióre adiuvári vult, ádiuvet infirmiórem in quo est ipso poténtior. Itaque: \emph{Beáti misericórdes, quia ipsórum miserábitur,}

\noindent Sexto gradu est cordis mundítia de bona consciéntia bonórum óperum valens ad contemplándum illud summum bonum, quod solo puro et seréno intelléctu cerni potest.

\noindent Postréma est séptima ipsa sapiéntia, id est contemplátio veritátis, pacíficans totum hóminem et suscípiens similitúdinem Dei, quæ ita conclúditur: \emph{Beáti pacífici, quóniam ipsi fílii Dei vocabúntur.}

\vfill
\pagebreak

\pars{Responsorium 4.} \scriptura{\Vbardot{} Ps. 47, 2; \textbf{H368}}

\vspace{-5mm}

\responsorium{V}{temporalia/resp-incircuitutuodomine-CROCHU-cumdox.gtex}{}

\vfill
\pagebreak

\pars{Hymnus Ambrosianus} \scriptura{Tonus Solemnis}

\vspace{-2mm}

\grechangedim{interwordspacetext}{0.26 cm plus 0.15 cm minus 0.05 cm}{scalable}%
\cuminitiali{III}{temporalia/tedeum-solemnis-gn.gtex}
\grechangedim{interwordspacetext}{0.22 cm plus 0.15 cm minus 0.05 cm}{scalable}%

\grechangedim{interwordspacetext}{0.22 cm plus 0.15 cm minus 0.05 cm}{scalable}}
\newcommand{\deusinadiutorium}{\grechangedim{interwordspacetext}{0.18 cm plus 0.15 cm minus 0.05 cm}{scalable}%
\cuminitiali{}{temporalia/deusinadiutorium-alter.gtex}}
\newcommand{\hymnuslaudes}{\pars{Hymnus.}

\cuminitiali{VIII}{temporalia/hym-IesuSalvator.gtex}}
\newcommand{\laudes}{\pars{Psalmus 1.} \scriptura{Ps. 14, 1; \textbf{H254}}

\vspace{-4mm}

\antiphona{VII a}{temporalia/ant-incaelestibusregnis.gtex}

\scriptura{Psalmus 62.}

\initiumpsalmi{temporalia/ps62-initium-vii-a-auto.gtex}

\input{temporalia/ps62-vii-a.tex} \Abardot{}

\vfill
\pagebreak

\pars{Psalmus 2.} \scriptura{Dan. 3, 87}

\vspace{-4mm}

\antiphona{per.}{temporalia/ant-sanctidomini.gtex}

\scriptura{Canticum trium puerorum, Dan. 3, 57-88 et 56}

\vspace{-2mm}

\initiumpsalmi{temporalia/dan3-initium-per-auto.gtex}

\input{temporalia/dan3-per-sinedox.tex}

\rubrica{Hic non dicitur Gloria Patri, neque Amen.}

\vfill

\antiphona{}{temporalia/ant-sanctidomini.gtex}

\vfill
\pagebreak

\pars{Psalmus 3.} \scriptura{Ps. 148, 14; Ps. 149, 9}

\vspace{-4mm}

\antiphona{VIII G}{temporalia/ant-hymnusomnibus.gtex}

%\vspace{-2mm}

\scriptura{Psalmus 149}

%\vspace{-2mm}

\initiumpsalmi{temporalia/ps149-initium-viii-G-auto.gtex}

\input{temporalia/ps149-viii-G.tex} \Abardot{}

\vfill
\pagebreak}
\newcommand{\lectiobrevis}{\pars{Lectio Brevis.} \scriptura{Eph. 1, 17-18}

\noindent Deus Dómini nostri Iesu Christi, Pater glóriæ det vobis Spíritum sapiéntiæ et revelatiónis in agnitióne eius, illuminátos óculos cordis vestri, ut sciátis quæ sit spes vocatiónis eius, quæ divítiæ glóriæ hereditátis eius in sanctis.}
\newcommand{\responsoriumbreve}{\pars{Responsorium breve.} \pars{Responsorium breve.} \scriptura{Ps. 31, 11}

\vspace{-5mm}

\responsorium{VI}{temporalia/resp-laetaminiindomino.gtex}{}}
\newcommand{\preces}{\noindent Deum, corónam sanctórum ómnium,~\grestar{} gaudénter deprecémur:

\Rbardot{} Per intercessiónem sanctórum salva nos, Dómine.

\noindent Deus, fons sanctitátis,~\gredagger{} qui multifórmis grátiæ tuæ mirabília in sanctis fulgére fecísti,~\grestar{} concéde nobis in illis magnitúdinem tuam celebráre.

\Rbardot{} Per intercessiónem sanctórum salva nos, Dómine.

\noindent Providentíssime ætérne Deus,~\gredagger{} qui perfectióres Fílii tui imágines in sanctis nobis ostendísti,~\grestar{} præsta, ut ad uniónem cum Christo per illos efficácius moveámur.

\Rbardot{} Per intercessiónem sanctórum salva nos, Dómine.

\noindent Rex cælórum, qui per fidéles Christi sectatóres ad futúram civitátem nos íncitas,~\grestar{} fac, ut ab illis de tutióre via illuc perveniéndi edoceámur.

\Rbardot{} Per intercessiónem sanctórum salva nos, Dómine.

\noindent Deus, qui in sacrifício córporis Fílii tui árctius cæléstibus íncolis nos coniúngis,~\grestar{} auge devotiónem nostram, ut cúltui eórum perféctius conformémur.

\Rbardot{} Per intercessiónem sanctórum salva nos, Dómine.}
\newcommand{\benedictus}{\pars{Canticum Zachariæ.}

\vspace{-4mm}

\antiphona{VII a}{temporalia/ant-tegloriosusapostolorum.gtex}

\vspace{-2mm}

\scriptura{Lc. 1, 68-79}

\vspace{-2mm}

\initiumpsalmi{temporalia/benedictus-initium-viisoll-a-auto.gtex}

%\vspace{-1.5mm}

\input{temporalia/benedictus-viisoll-a.tex} \Abardot{}}
\newcommand{\benedicamuslaudes}{\cuminitiali{II}{temporalia/benedicamus-solemnism-laud.gtex}}
\include{hebdomadaxxx}
% LuaLaTeX

\documentclass[a4paper, twoside, 12pt]{article}
\usepackage[latin]{babel}
%\usepackage[landscape, left=3cm, right=1.5cm, top=2cm, bottom=1cm]{geometry} % okraje stranky
%\usepackage[landscape, a4paper, mag=1166, truedimen, left=2cm, right=1.5cm, top=1.6cm, bottom=0.95cm]{geometry} % okraje stranky
\usepackage[landscape, a4paper, mag=1400, truedimen, left=0.5cm, right=0.5cm, top=0.5cm, bottom=0.5cm]{geometry} % okraje stranky

\usepackage{fontspec}
\setmainfont[FeatureFile={junicode.fea}, Ligatures={Common, TeX}, RawFeature=+fixi]{Junicode}
%\setmainfont{Junicode}

% shortcut for Junicode without ligatures (for the Czech texts)
\newfontfamily\nlfont[FeatureFile={junicode.fea}, Ligatures={Common, TeX}, RawFeature=+fixi]{Junicode}

\usepackage{multicol}
\usepackage{color}
\usepackage{lettrine}
\usepackage{fancyhdr}

% usual packages loading:
\usepackage{luatextra}
\usepackage{graphicx} % support the \includegraphics command and options
\usepackage{gregoriotex} % for gregorio score inclusion
\usepackage{gregoriosyms}
\usepackage{wrapfig} % figures wrapped by the text
\usepackage{parcolumns}
\usepackage[contents={},opacity=1,scale=1,color=black]{background}
\usepackage{tikzpagenodes}
\usepackage{calc}
\usepackage{longtable}
\usetikzlibrary{calc}

\setlength{\headheight}{14.5pt}

\input{conventuscommune.tex} % Often used macros

\newcommand{\annusEditionis}{2021}

%%%% Vicekrat opakovane kousky

\newcommand{\anteOrationem}{
  \rubrica{Ante Orationem, cantatur a Superiore:}

  \pars{Supplicatio Litaniæ.}

  \cuminitiali{}{temporalia/supplicatiolitaniae.gtex}

  \pars{Oratio Dominica.}

  \cuminitiali{}{temporalia/oratiodominica.gtex}

  \rubrica{Deinde dicitur ab Hebdomadario:}

  \cuminitiali{}{temporalia/dominusvobiscum-solemnis.gtex}

  \rubrica{In choro monialium loco Dominus vobiscum dicitur:}

  \sineinitiali{temporalia/domineexaudi.gtex}
}

\setlength{\columnsep}{30pt} % prostor mezi sloupci

%%%%%%%%%%%%%%%%%%%%%%%%%%%%%%%%%%%%%%%%%%%%%%%%%%%%%%%%%%%%%%%%%%%%%%%%%%%%%%%%%%%%%%%%%%%%%%%%%%%%%%%%%%%%%
\begin{document}

% Here we set the space around the initial.
% Please report to http://home.gna.org/gregorio/gregoriotex/details for more details and options
\grechangedim{afterinitialshift}{2.2mm}{scalable}
\grechangedim{beforeinitialshift}{2.2mm}{scalable}
\grechangedim{interwordspacetext}{0.22 cm plus 0.15 cm minus 0.05 cm}{scalable}%
\grechangedim{annotationraise}{-0.2cm}{scalable}

% Here we set the initial font. Change 38 if you want a bigger initial.
% Emit the initials in red.
\grechangestyle{initial}{\color{red}\fontsize{38}{38}\selectfont}

\pagestyle{empty}

%%%% Titulni stranka
\begin{titulusOfficii}
\ifx\titulus\undefined
\nomenFesti{Feria VI \hebdomada{}}
\else
\titulus
\fi
\end{titulusOfficii}

\vfill

\begin{center}
%Ad usum et secundum consuetudines chori \guillemotright{}Conventus Choralis\guillemotleft.

%Editio Sancti Wolfgangi \annusEditionis
\end{center}

\scriptura{}

\pars{}

\pagebreak

\renewcommand{\headrulewidth}{0pt} % no horiz. rule at the header
\fancyhf{}
\pagestyle{fancy}

\cantusSineNeumas

\hora{Ad Matutinum.} %%%%%%%%%%%%%%%%%%%%%%%%%%%%%%%%%%%%%%%%%%%%%%%%%%%%%

\vspace{2mm}

\cuminitiali{}{temporalia/dominelabiamea.gtex}

\vfill
%\pagebreak

\vspace{2mm}

\ifx\invitatorium\undefined
\pars{Invitatorium.} \scriptura{Lc. 24, 34; Psalmus 94; \textbf{H232}}

\antiphona{VI}{temporalia/inv-surrexitdominusvere.gtex}
\else
\invitatorium
\fi

\vfill
\pagebreak

\ifx\hymnusmatutinum\undefined
\pars{Hymnus.}

\cuminitiali{VIII}{temporalia/hym-LaetareCaelum.gtex}
\else
\hymnusmatutinum
\fi

\vspace{-3mm}

\vfill
\pagebreak

\ifx\matutinum\undefined
\ifx\matua\undefined
\else
% MAT A
\pars{Psalmus 1.}

\vspace{-4mm}

\antiphona{I a\textsuperscript{3}}{temporalia/ant-alleluia-turco24.gtex}

%\vspace{-2mm}

\scriptura{Ps. 34, 1-10}

%\vspace{-2mm}

\initiumpsalmi{temporalia/ps34i-initium-i-a5-auto.gtex}

\input{temporalia/ps34i-i-a5.tex}

\vfill
\pagebreak

\pars{Psalmus 2.} \scriptura{Ps. 34, 11-17}

%\vspace{-2mm}

\initiumpsalmi{temporalia/ps34ii-initium-i-a5-auto.gtex}

\input{temporalia/ps34ii-i-a5.tex}

\vfill
\pagebreak

\pars{Psalmus 3.} \scriptura{Ps. 34, 18-28}

\vspace{-2mm}

\initiumpsalmi{temporalia/ps34iii-initium-i-a5-auto.gtex}

\input{temporalia/ps34iii-i-a5.tex}

\vfill

\antiphona{}{temporalia/ant-alleluia-turco24.gtex}

\vfill
\pagebreak
\fi
\ifx\matub\undefined
\else
% MAT B
\pars{Psalmus 1.}

\vspace{-4mm}

\antiphona{D}{temporalia/ant-alleluia-turco2.gtex}

%\vspace{-2mm}

\scriptura{Ps. 37, 2-5}

%\vspace{-2mm}

\initiumpsalmi{temporalia/ps37ii_v-initium-d-g-auto.gtex}

\input{temporalia/ps37ii_v-d-g.tex}

\vfill
\pagebreak

\pars{Psalmus 2.}

\scriptura{Ps. 37, 6-13}

%\vspace{-2mm}

\initiumpsalmi{temporalia/ps37vi_xiii-initium-d-g-auto.gtex}

\input{temporalia/ps37vi_xiii-d-g.tex}

\vfill
\pagebreak

\pars{Psalmus 3.}

\scriptura{Ps. 37, 14-23}

%\vspace{-2mm}

\initiumpsalmi{temporalia/ps37xiv_xxiii-initium-d-g-auto.gtex}

\input{temporalia/ps37xiv_xxiii-d-g.tex}

\vfill

\antiphona{}{temporalia/ant-alleluia-turco2.gtex}

\vfill
\pagebreak
\fi
\ifx\matuc\undefined
\else
% MAT C
\pars{Psalmus 1.}

\vspace{-4mm}

\antiphona{I d\textsuperscript{3}}{temporalia/ant-alleluia-auglx5.gtex}

%\vspace{-3mm}

\scriptura{Ps. 68, 2-13}

%\vspace{-2mm}

\initiumpsalmi{temporalia/ps68ii_xiii-initium-i-d-auto.gtex}

%\vspace{-1.5mm}

\input{temporalia/ps68ii_xiii-i-d.tex}

\vfill
\pagebreak

\pars{Psalmus 2.}

\scriptura{Ps. 68, 14-22}

%\vspace{-2mm}

\initiumpsalmi{temporalia/ps68xiv_xxii-initium-i-d-auto.gtex}

\input{temporalia/ps68xiv_xxii-i-d.tex}

\vfill
\pagebreak

\pars{Psalmus 3.}

\scriptura{Ps. 68, 30-37}

%\vspace{-2mm}

\initiumpsalmi{temporalia/ps68iii-initium-i-d-auto.gtex}

\input{temporalia/ps68iii-i-d.tex}

\vfill

\antiphona{}{temporalia/ant-alleluia-auglx5.gtex}

\vfill
\pagebreak
\fi
\ifx\matud\undefined
\else
% MAT D
\pars{Psalmus 1.}

\vspace{-4mm}

\antiphona{I a\textsuperscript{2}}{temporalia/ant-alleluia-turco24.gtex}

%\vspace{-3mm}

\scriptura{Ps. 77, 1-16}

%\vspace{-2mm}

\initiumpsalmi{temporalia/ps77i_xvi-initium-i-a4-auto.gtex}

\input{temporalia/ps77i_xvi-i-a4.tex}

\vfill
\pagebreak

\pars{Psalmus 2.} \scriptura{Ps. 77, 17-31}

%\vspace{-2mm}

\initiumpsalmi{temporalia/ps77iii-initium-i-a4-auto.gtex}

\input{temporalia/ps77iii-i-a4.tex}

\vfill
\pagebreak

\pars{Psalmus 3.} \scriptura{Ps. 77, 32-39}

%\vspace{-2mm}

\initiumpsalmi{temporalia/ps77xxxii_xxxix-initium-i-a4-auto.gtex}

\input{temporalia/ps77xxxii_xxxix-i-a4.tex}

\vfill

\antiphona{}{temporalia/ant-alleluia-turco24.gtex}

\vfill
\pagebreak
\fi
\else
\matutinum
\fi

\pars{Versus.}

\ifx\matversus\undefined
\noindent \Vbardot{} In resurrectióne tua, Christe, allelúia.

\noindent \Rbardot{} Cæli et terra læténtur, allelúia.
\else
\matversus
\fi

\vspace{5mm}

\sineinitiali{temporalia/oratiodominica-mat.gtex}

\vspace{5mm}

\pars{Absolutio.}

\cuminitiali{}{temporalia/absolutio-ipsius.gtex}

\vfill
\pagebreak

\cuminitiali{}{temporalia/benedictio-solemn-deus.gtex}

\vspace{7mm}

\lectioi

\noindent \Vbardot{} Tu autem, Dómine, miserére nobis.
\noindent \Rbardot{} Deo grátias.

\vfill
\pagebreak

\responsoriumi

\vfill
\pagebreak

\cuminitiali{}{temporalia/benedictio-solemn-christus.gtex}

\vspace{7mm}

\lectioii

\noindent \Vbardot{} Tu autem, Dómine, miserére nobis.
\noindent \Rbardot{} Deo grátias.

\vfill
\pagebreak

\responsoriumii

\vfill
\pagebreak

\cuminitiali{}{temporalia/benedictio-solemn-ignem.gtex}

\vspace{7mm}

\lectioiii

\noindent \Vbardot{} Tu autem, Dómine, miserére nobis.
\noindent \Rbardot{} Deo grátias.

\vfill
\pagebreak

\responsoriumiii

\vfill
\pagebreak

\rubrica{Reliqua omittuntur, nisi Laudes separandæ sint.}

\sineinitiali{temporalia/domineexaudi.gtex}

\vfill

\oratio

\vfill

\noindent \Vbardot{} Dómine, exáudi oratiónem meam.
\Rbardot{} Et clamor meus ad te véniat.

\vfill

\noindent \Vbardot{} Benedicámus Dómino.
\noindent \Rbardot{} Deo grátias.

\vfill

\noindent \Vbardot{} Fidélium ánimæ per misericórdiam Dei requiéscant in pace.
\Rbardot{} Amen.

\vfill
\pagebreak

\hora{Ad Laudes.} %%%%%%%%%%%%%%%%%%%%%%%%%%%%%%%%%%%%%%%%%%%%%%%%%%%%%

\cantusSineNeumas

\vspace{0.5cm}
\grechangedim{interwordspacetext}{0.18 cm plus 0.15 cm minus 0.05 cm}{scalable}%
\cuminitiali{}{temporalia/deusinadiutorium-communis.gtex}
\grechangedim{interwordspacetext}{0.22 cm plus 0.15 cm minus 0.05 cm}{scalable}%

\vfill
\pagebreak

\ifx\hymnuslaudes\undefined
\ifx\laudac\undefined
\else
\pars{Hymnus}

\cuminitiali{I}{temporalia/hym-ChorusNovae-praglia.gtex}
\vspace{-3mm}
\fi
\ifx\laudbd\undefined
\else
\pars{Hymnus}

\cuminitiali{I}{temporalia/hym-ChorusNovae.gtex}
\vspace{-3mm}
\fi
\else
\hymnuslaudes
\fi

\vfill
\pagebreak

\ifx\laudes\undefined
\ifx\lauda\undefined
\else
\pars{Psalmus 1.}

\vspace{-4mm}

\antiphona{VI F}{temporalia/ant-alleluia-turco6.gtex}

\scriptura{Psalmus 50.}

\initiumpsalmi{temporalia/ps50-initium-vi-F-auto.gtex}

\input{temporalia/ps50-vi-F.tex}

\vfill

\antiphona{}{temporalia/ant-alleluia-turco6.gtex}

\vfill
\pagebreak

\pars{Psalmus 2.} \scriptura{Is. 45, 25}

\vspace{-4mm}

\antiphona{V a}{temporalia/ant-indominoiustificabitur-tp.gtex}

\scriptura{Canticum Isaiæ, Is. 45, 15-30}

%\vspace{-2mm}

\initiumpsalmi{temporalia/isaiae2-initium-v-a-auto.gtex}

\input{temporalia/isaiae2-v-a.tex}

\vfill

\antiphona{}{temporalia/ant-indominoiustificabitur-tp.gtex}

\vfill
\pagebreak

\pars{Psalmus 3.}

\vspace{-4mm}

\antiphona{IV* e}{temporalia/ant-alleluia-turco9.gtex}

\scriptura{Psalmus 99.}

\initiumpsalmi{temporalia/ps99-initium-iv_-e-auto.gtex}

\input{temporalia/ps99-iv_-e.tex} \Abardot{}

\vfill
\pagebreak
\fi
\ifx\laudb\undefined
\else
\pars{Psalmus 1.}

\vspace{-4mm}

\antiphona{VII a}{temporalia/ant-alleluia-turco29.gtex}

\scriptura{Psalmus 50.}

\initiumpsalmi{temporalia/ps50-initium-vii-a-auto.gtex}

\input{temporalia/ps50-vii-a.tex}

\vfill

\antiphona{}{temporalia/ant-alleluia-turco29.gtex}

\vfill
\pagebreak

\pars{Psalmus 2.} \scriptura{Hab. 3, 2; \textbf{H99}}

\vspace{-6mm}

\antiphona{IV* e}{temporalia/ant-domineaudivi-tp.gtex}

\vspace{-2mm}

\scriptura{Canticum Habacuc, Hab. 3, 2-19}

%\vspace{-2mm}

%\initiumpsalmi{temporalia/habacuc-initium-iv_-e-auto.gtex}
\initiumpsalmi{temporalia/habacuc-initium-iv_-e.gtex}

\input{temporalia/habacuc-iv_-e.tex}

\vfill

\antiphona{}{temporalia/ant-domineaudivi-tp.gtex}

\vfill
\pagebreak

\pars{Psalmus 3.}

\vspace{-4mm}

\antiphona{E}{temporalia/ant-alleluia-turco4.gtex}

\vspace{-2mm}

\scriptura{Psalmus 147.}

%\vspace{-3mm}

%\initiumpsalmi{temporalia/ps147-initium-e-auto.gtex}
\initiumpsalmi{temporalia/ps147-initium-e.gtex}

\input{temporalia/ps147-e.tex} \Abardot{}

\vfill
\pagebreak
\fi
\ifx\laudc\undefined
\else
\pars{Psalmus 1.}

\vspace{-4mm}

\antiphona{VIII G\textsuperscript{2}}{temporalia/ant-alleluia-turco13.gtex}

\scriptura{Psalmus 50.}

\initiumpsalmi{temporalia/ps50-initium-viii-G5-auto.gtex}

\input{temporalia/ps50-viii-G5.tex}

\vfill

\antiphona{}{temporalia/ant-alleluia-turco13.gtex}

\vfill
\pagebreak

\pars{Psalmus 2.}

\vspace{-4mm}

\antiphona{VIII G}{temporalia/ant-nonnosderelinquas-tp.gtex}

%\vspace{-2mm}

\scriptura{Canticum Ieremiæ, Ier. 14, 17-31}

%\vspace{-2mm}

\initiumpsalmi{temporalia/jeremiae2-initium-viii-G.gtex}

\input{temporalia/jeremiae2-viii-G.tex} \Abardot{}

\vfill
\pagebreak

\pars{Psalmus 3.}

\vspace{-4mm}

\antiphona{E}{temporalia/ant-alleluia-praglia-e2.gtex}

\vspace{-2mm}

\scriptura{Psalmus 99.}

%\vspace{-2mm}

\initiumpsalmi{temporalia/ps99-initium-e-auto.gtex}

\input{temporalia/ps99-e.tex} \Abardot{}

\vfill
\pagebreak
\fi
\ifx\laudd\undefined
\else
\pars{Psalmus 1.}

\vspace{-4mm}

\antiphona{I f}{temporalia/ant-alleluia-turco20.gtex}

\scriptura{Psalmus 50.}

\initiumpsalmi{temporalia/ps50-initium-i-f-auto.gtex}

\input{temporalia/ps50-i-f.tex}

\vfill

\antiphona{}{temporalia/ant-alleluia-turco20.gtex}

\vfill
\pagebreak

\pars{Psalmus 2.} \scriptura{Ac. 22, 14}

\vspace{-4mm}

\antiphona{VIII G}{temporalia/ant-beatiquilavantstolas.gtex}

%\vspace{-2mm}

\scriptura{Canticum Tobiæ, Tob. 13, 10-18}

%\vspace{-2mm}

\initiumpsalmi{temporalia/tobiae2-initium-viii-G-auto.gtex}

\input{temporalia/tobiae2-viii-G.tex} \Abardot{}

\vfill
\pagebreak

\pars{Psalmus 3.}

\vspace{-4mm}

\antiphona{VI F}{temporalia/ant-alleluia-turco5.gtex}

\vspace{-2mm}

\scriptura{Psalmus 147.}

%\vspace{-2mm}

\initiumpsalmi{temporalia/ps147-initium-vi-F-auto.gtex}

\input{temporalia/ps147-vi-F.tex} \Abardot{}

\vfill
\pagebreak
\fi
\else
\laudes
\fi

\ifx\lectiobrevis\undefined
\pars{Lectio Brevis.} \scriptura{Ac. 5, 30-32}

\noindent Deus patrum nostrórum suscitávit Iesum, quem vos interemístis suspendéntes in ligno; hunc Deus Príncipem et Salvatórem exaltávit déxtera sua ad dandam pæniténtiam Israel et remissiónem peccatórum. Et nos sumus testes horum verbórum, et Spíritus Sanctus, quem dedit Deus obœdiéntibus sibi.
\else
\lectiobrevis
\fi

\vfill

\ifx\responsoriumbreve\undefined
\pars{Responsorium breve.} \scriptura{Cf. Mt. 28, 6; Cf. Gal. 3, 13}

\cuminitiali{VI}{temporalia/resp-surrexitdominusdesepulcro.gtex}
\else
\responsoriumbreve
\fi

\vfill
\pagebreak

\benedictus

\vspace{-1cm}

\vfill
\pagebreak

\pars{Preces.}

\sineinitiali{}{temporalia/tonusprecum.gtex}

\ifx\preces\undefined
\ifx\lauda\undefined
\else
\noindent Deum Patrem, qui vitam novam per Christi resurrectiónem cóntulit nobis,~\gredagger{} súpplices exorémus:

\Rbardot{} Clarífica nos claritáte Christi.

\noindent Deus, qui opéribus tuis antíquam dispensatiónem manifestásti, terram creásti et fidélis es in ómnibus generatiónibus,~\gredagger{} exáudi nos, clementíssime Pater.

\Rbardot{} Clarífica nos claritáte Christi.

\noindent Purífica nos puritáte veritátis tuæ, et gressus nostros dírige in cordis sanctitáte,~\gredagger{} ut quod iustum est tibíque plácitum agámus.

\Rbardot{} Clarífica nos claritáte Christi.

\noindent Illúmina vultum tuum super nos,~\gredagger{} ut a peccáto liberáti bonis domus tuæ repleámur.

\Rbardot{} Clarífica nos claritáte Christi.

\noindent Qui per Christum nos tibi reconciliásti,~\gredagger{} pacem nobis largíre omnibúsque in orbe terrárum degéntibus.

\Rbardot{} Clarífica nos claritáte Christi.
\fi
\ifx\laudb\undefined
\else
\noindent Deus Pater Christum per Spíritum suscitávit, et étiam mortália córpora nostra vivificábit.~\gredagger{} Quare clamémus:

\Rbardot{} Dómine, vivífica nos Spíritu Sancto tuo.

\noindent Pater sancte, qui accepísti holocáustum Fílii tui, resúscitans eum ex mórtuis,~\gredagger{} súscipe hodiérnam nostram oblatiónem et perduc nos in vitam ætérnam.

\Rbardot{} Dómine, vivífica nos Spíritu Sancto tuo.

\noindent Opera nostra hódie propítius intuére,~\gredagger{} ut fiant ad glóriam tuam et ad ómnium sanctificatiónem.

\Rbardot{} Dómine, vivífica nos Spíritu Sancto tuo.

\noindent Opus nostrum hódie non sit vanum, sed univérsis homínibus insérviat~\gredagger{} et sic operántes ad regnum tuum fac nos perveníre.

\Rbardot{} Dómine, vivífica nos Spíritu Sancto tuo.

\noindent Aperi hódie óculos nostros et cor nostrum ad fratres,~\gredagger{} ut nos ínvicem amémus nobísque serviámus.

\Rbardot{} Dómine, vivífica nos Spíritu Sancto tuo.
\fi
\ifx\laudc\undefined
\else
\noindent Deum Patrem, qui vitam novam per Christi resurrectiónem cóntulit nobis,~\gredagger{} súpplices exorémus:

\Rbardot{} Clarífica nos claritáte Christi.

\noindent Deus, qui opéribus tuis antíquam dispensatiónem manifestásti, terram creásti et fidélis es in ómnibus generatiónibus,~\gredagger{} exáudi nos, clementíssime Pater.

\Rbardot{} Clarífica nos claritáte Christi.

\noindent Purífica nos puritáte veritátis tuæ, et gressus nostros dírige in cordis sanctitáte,~\gredagger{} ut quod iustum est tibíque plácitum agámus.

\Rbardot{} Clarífica nos claritáte Christi.

\noindent Illúmina vultum tuum super nos,~\gredagger{} ut a peccáto liberáti bonis domus tuæ repleámur.

\Rbardot{} Clarífica nos claritáte Christi.

\noindent Qui per Christum nos tibi reconciliásti,~\gredagger{} pacem nobis largíre omnibúsque in orbe terrárum degéntibus.

\Rbardot{} Clarífica nos claritáte Christi.
\fi
\ifx\laudd\undefined
\else
\noindent Deus Pater Christum per Spíritum suscitávit, et étiam mortália córpora nostra vivificábit.~\gredagger{} Quare clamémus:

\Rbardot{} Dómine, vivífica nos Spíritu Sancto tuo.

\noindent Pater sancte, qui accepísti holocáustum Fílii tui, resúscitans eum ex mórtuis,~\gredagger{} súscipe hodiérnam nostram oblatiónem et perduc nos in vitam ætérnam.

\Rbardot{} Dómine, vivífica nos Spíritu Sancto tuo.

\noindent Opera nostra hódie propítius intuére,~\gredagger{} ut fiant ad glóriam tuam et ad ómnium sanctificatiónem.

\Rbardot{} Dómine, vivífica nos Spíritu Sancto tuo.

\noindent Opus nostrum hódie non sit vanum, sed univérsis homínibus insérviat~\gredagger{} et sic operántes ad regnum tuum fac nos perveníre.

\Rbardot{} Dómine, vivífica nos Spíritu Sancto tuo.

\noindent Aperi hódie óculos nostros et cor nostrum ad fratres,~\gredagger{} ut nos ínvicem amémus nobísque serviámus.

\Rbardot{} Dómine, vivífica nos Spíritu Sancto tuo.
\fi 
\else
\preces
\fi

\vfill

\pars{Oratio Dominica.}

\cuminitiali{}{temporalia/oratiodominicaalt.gtex}

\vfill
\pagebreak

\rubrica{vel:}

\pars{Supplicatio Litaniæ.}

\cuminitiali{}{temporalia/supplicatiolitaniae.gtex}

\vfill

\pars{Oratio Dominica.}

\cuminitiali{}{temporalia/oratiodominica.gtex}

\vfill
\pagebreak

% Oratio. %%%
\oratio

\vspace{-1mm}

\vfill

\rubrica{Hebdomadarius dicit Dominus vobiscum, vel, absente sacerdote vel diacono, sic concluditur:}

\vspace{2mm}

\antiphona{C}{temporalia/dominusnosbenedicat.gtex}

\rubrica{Postea cantatur a cantore:}

\vspace{2mm}

\cuminitiali{VII}{temporalia/benedicamus-tempore-paschali.gtex}

\vspace{1mm}

\vfill
\pagebreak

\end{document}

