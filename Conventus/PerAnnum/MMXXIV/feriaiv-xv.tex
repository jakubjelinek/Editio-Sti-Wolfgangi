\newcommand{\hymnusmatutinum}{\pars{Hymnus}

\cuminitiali{IV}{temporalia/hym-ScientiarumDomino-kempten.gtex}}
\newcommand{\lectioi}{\pars{Lectio I.} \scriptura{1 Reg. 21, 1-14}

\noindent De libro primo Regum.

\noindent Témpore illo vínea erat Naboth Iezrahelítæ, quæ erat in Iézrahel iuxta palátium Achab regis Samaríæ. Locútus est ergo Achab ad Naboth dicens: «Da mihi víneam tuam, ut fáciam mihi hortum hólerum, quia vicína est et prope domum meam. Dabóque tibi pro ea víneam meliórem aut, si tibi commódius putas, argénti prétium quanto digna est». Cui respóndit Naboth: «Propítius mihi sit Dóminus, ne dem hereditátem patrum meórum tibi».

\noindent Venit ergo Achab in domum suam tristis et indígnans super verbo, quod locútus fúerat ad eum Naboth Iezrahelítes dicens: «Non dabo tibi hereditátem patrum meórum». Et proíciens se in léctulum suum avértit fáciem ad paríetem et non comédit panem.

\noindent Ingréssa est autem ad eum Iézabel uxor sua dixítque ei: «Quid est hoc, unde ánima tua contristáta est? Et quare non cómedis panem?».

\noindent Qui respóndit ei: «Quia locútus sum Naboth Iezrahelítæ et dixi ei: Da mihi víneam tuam, accépta pecúnia; aut, si tibi placet, dabo tibi víneam pro ea. Et ille ait: “Non dabo tibi víneam meam”».

\noindent Dixit ergo ad eum Iézabel uxor eius: «Grandis auctoritátis es et bene regis regnum Israel! Surge et cómede panem et æquo esto ánimo: ego dabo tibi víneam Naboth Iezrahelítæ».

\noindent Scripsit ítaque lítteras ex nómine Achab et signávit eas ánulo eius et misit ad maióres natu et ad optimátes, qui erant in civitáte eius et habitábant cum Naboth. Litterárum autem hæc erat senténtia: «Prædicáte ieiúnium et sedére fácite Naboth in cápite pópuli et submíttite duos viros fílios Bélial contra eum, et testimónium dicant: “Maledixísti Deum et regem”; et edúcite eum et lapidáte, sicque moriátur».

\noindent Fecérunt ergo cives eius maióres natu et optimátes, qui habitábant cum eo in urbe, sicut præcéperat eis Iézabel et sicut scriptum erat in lítteris, quas míserat ad eos. {\color{gray} Prædicavérunt ieiúnium et sedére fecérunt Naboth in cápite pópuli; et ingréssi duo viri fílii Bélial sedérunt contra eum et illi, ut viri diabólici, dixérunt contra eum testimónium coram multitúdine: «Maledíxit Naboth Deum et regem». Quam ob rem eduxérunt eum extra civitátem et lapídibus interfecérunt;} miserúntque ad Iézabel dicéntes: «Lapidátus est Naboth et mórtuus est».}
\newcommand{\responsoriumi}{\pars{Responsorium 1.} \scriptura{\Rbardot{} Ps. 37, 7-8.12 \Vbardot{} ibid., 13; \textbf{H165}}

\vspace{-5mm}

\responsorium{IV}{temporalia/resp-totadiecontristatus-CROCHU.gtex}{}}
\newcommand{\lectioii}{\pars{Lectio II.} \scriptura{1 Reg. 21, 15-21.27-29}

\noindent Factum est autem, cum audísset Iézabel lapidátum Naboth et mórtuum, locúta est ad Achab: «Surge, pósside víneam Naboth Iezrahelítæ, qui nóluit tibi acquiéscere et dare eam, accépta pecúnia; non enim vivit Naboth, sed mórtuus est». Quod cum audísset Achab, mórtuum vidélicet Naboth, surréxit et descendébat in víneam Naboth Iezrahelítæ, ut possidéret eam.

\noindent Factus est ígitur sermo Dómini ad Elíam Thesbíten dicens: «Surge et descénde in occúrsum Achab regis Israel, qui est in Samaría; ecce est in vínea Naboth, ad quam descéndit, ut possídeat eam. Et loquéris ad eum dicens: Hæc dicit Dóminus: Occidísti, ínsuper et possedísti! Et post hæc addes: Hæc dicit Dóminus: In loco, in quo linxérunt canes sánguinem Naboth, lambent tuum quoque sánguinem». Et ait Achab ad Elíam: «Num invenísti me, inimíce mi?». Qui dixit: «Invéni, eo quod venúmdatus sis, ut fáceres malum in conspéctu Dómini. Ecce ego indúcam super te malum et démetam posterióra tua et interfíciam de Achab quidquid masculíni sexus sive impúberem sive púberem in Israel».

\noindent Itaque cum audísset Achab sermónes istos, scidit vestem suam et opéruit cilício carnem suam ieiunavítque et dormívit in sacco et ambulábat demísso cápite. Factus est autem sermo Dómini ad Elíam Thesbíten dicens: «Nonne vidísti humiliátum Achab coram me? Quia ígitur humiliátus est mei causa, non indúcam malum in diébus eius, sed in diébus fílii sui ínferam malum dómui eius».}

\newcommand{\responsoriumii}{\pars{Responsorium 2.} \scriptura{\Rbardot{} Ps. 79, 14.16 \Vbardot{} ibid., 15; \textbf{H88}}

\vspace{-5mm}

\responsorium{III}{temporalia/resp-devastavitvineam-CROCHU.gtex}{}

\vfill

\rubrica{vel ad libitum:}

\vspace{3mm}

\pars{Responsorium 2.} \scriptura{\Rbardot{} Ier. 2, 21 \Vbardot{} ibid.; \textbf{H217}}

\vspace{-5mm}

\responsorium{VIII}{temporalia/resp-vineamea-CROCHU.gtex}{}}
\newcommand{\lectioiii}{\pars{Lectio III.} \scriptura{Nn. 19-21. 24. 26-28: SCh 25 bis, 164-170}

\noindent Ex Tractátu sancti Ambrósii epíscopi De mystériis

\noindent Ideo tibi ante prædíctum est ut non hoc solum créderes quod vidébas, ne forte et tu díceres: Hoc est illud magnum mystérium \emph{quod óculus non vidit nec auris audívit nec in cor hóminis ascéndit?} Aquas vídeo quas vidébam cotídie, istæ me habent mundáre in quas sæpe descéndi et numquam mundátus sum? Hinc cognósce quod aqua non mundat sine Spíritu.

\noindent Ideóque legísti quod \emph{tres} testes in baptísmate \emph{unum sunt:} aqua, sanguis et Spíritus, quia si unum horum détrahas, non stat baptísmatis sacraméntum. Quid est enim aqua sine cruce Christi, nisi eleméntum commúne sine ullo sacraménti proféctu? Nec íterum sine aqua regeneratiónis mystérium est: \emph{nisi} enim \emph{quis renátus fúerit ex aqua et Spíritu, non potest introíre in regnum Dei.} Credit autem étiam catechúmenus in crucem Dómini Iesu qua et ipse signátur, sed nisi baptizátus fúerit in nómine Patris et Fílii et Spíritus Sancti, remissiónem non potest accípere peccatórum nec spiritális grátiæ munus hauríre.

\noindent Ergo ille Syrus sépties mersit in lege, tu autem baptizátus es in nómine Trinitátis. Conféssus es Patrem —recordáre quid féceris— conféssus es Fílium, conféssus es Spíritum. Tene órdinem rerum. In hac fide, mundo mórtuus es, Deo resurrexísti, et quasi in illo mundi consepúltus eleménto, peccáto mórtuus ad vitam es resuscitátus ætérnam. Crede ergo quia non sunt vácuæ aquæ.

\noindent Dénique paralýticus ille (Probáticæ piscínæ) exspectábat hóminem. Quem illum nisi Dóminum Iesum natum ex vírgine, cuius advéntu iam non umbra sanáret síngulos, sed véritas univérsos? Iste est ergo qui exspectabátur ut descénderet, de quo dixit Deus Pater ad Ioánnem Baptístam: \emph{Super quem víderis Spíritum descendéntem de cælo et manéntem super eum, hic est qui baptízat in Spíritu Sancto.} De quo testificátus est Ioánnes, dicens: \emph{Quia vidi Spíritum descendéntem de cælo quasi colúmbam et manéntem super eum.} Et hic quare Spíritus sicut colúmba descéndit, nisi ut tu vidéres, nisi ut tu cognósceres étiam illam colúmbam, quam Noe iustus emísit ex arca, istíus colúmbæ spéciem fuísse, ut typum agnósceres sacraménti?

\noindent Est adhuc quod dubitáre débeas? cum evidénter tibi clamet in Evangélio Pater, qui ait: \emph{Hic est Fílius meus in quo complácui;} clamet Fílius, super quem sicut colúmba se demonstrávit Spíritus Sanctus; clamet et Spíritus Sanctus, qui sicut colúmba descéndit; clamet David: \emph{Vox Dómini super aquas, Deus maiestátis intónuit, Dóminus super aquas multas,} cum tibi Scriptúra testétur quod ad Hieróboal preces ignis descéndit de cælo et rursus, precánte Elía, ignis est missus qui sacrifícium consecrávit.

\noindent {\color{gray} Non mérita personárum consíderes, sed offícia sacerdótum. Et si mérita spectes, sicut Elíam consíderes, Petri quoque mérita spectáto vel Pauli, qui accéptum a Dómino Iesu hoc nobis mystérium tradidérunt. Ignis illis visíbilis mittebátur ut créderent, nobis invisíbilis operátur qui crédimus, illis in figúram, nobis ad commonitiónem. Credo ergo adésse Dóminum Iesum invocátum précibus sacerdótum qui ait: \emph{Ubicúmque fúerit duo vel tres, ibi et ego sum.} Quanto magis ubi Ecclésia est, ubi mystéria sunt, ibi dignátur suam impertíre præséntiam.

\noindent Descendísti ígitur. Recordáre quid respónderis, quod credas in Patrem, credas in Fílium, credas in Spíritum Sanctum. Non habes illic: Credo in maiórem et minórem et últimum, sed eádem vocis tuæ cautióne constríngeris, ut simíliter credas in Fílium sicut in Patrem credis, simíliter in Spíritum credas sicut credis in Fílium, hoc solo excépto quod in crucem solíus Dómini Iesu fatéris tibi esse credéndum.}}

\newcommand{\responsoriumiii}{\pars{Responsorium 3.} \scriptura{\Rbar{} Mt. 3, 16-17 \Vbar{} ibidem; \textbf{H75}}

\vspace{-5mm}

\responsorium{II}{temporalia/resp-incolumbaespecie-CROCHU-cumdox.gtex}{}}
\newcommand{\hymnuslaudes}{\pars{Hymnus} \scriptura{Prudentius (\olddag{} 413)}

\cuminitiali{VIII}{temporalia/hym-NoxEtTenebrae-einsiedeln.gtex}}
\newcommand{\hebdomada}{infra Hebdom. XV per Annum.}
\newcommand{\matuc}{Matutinum Hebdomadae C}
\newcommand{\matuac}{Matutinum Hebdomadae A vel C}
\newcommand{\laudc}{Laudes Hebdomadae C}
\newcommand{\laudac}{Laudes Hebdomadae A vel C}

% LuaLaTeX

\documentclass[a4paper, twoside, 12pt]{article}
\usepackage[latin]{babel} 
%\usepackage[landscape, left=3cm, right=1.5cm, top=2cm, bottom=1cm]{geometry} % okraje stranky
%\usepackage[landscape, a4paper, mag=1166, truedimen, left=2cm, right=1.5cm, top=1.6cm, bottom=0.95cm]{geometry} % okraje stranky
\usepackage[landscape, a4paper, mag=1400, truedimen, left=0.5cm, right=0.5cm, top=0.5cm, bottom=0.5cm]{geometry} % okraje stranky

\usepackage{fontspec}
\setmainfont[FeatureFile={junicode.fea}, Ligatures={Common, TeX}, RawFeature=+fixi]{Junicode}
%\setmainfont{Junicode}

% shortcut for Junicode without ligatures (for the Czech texts)
\newfontfamily\nlfont[FeatureFile={junicode.fea}, Ligatures={Common, TeX}, RawFeature=+fixi]{Junicode}

% Hebrew font:
% http://scripts.sil.org/cms/scripts/page.php?site_id=nrsi&id=SILHebrUnic2
\newfontfamily\hebfont[Scale=1]{Ezra SIL}

\usepackage{multicol}
\usepackage{color}
\usepackage{lettrine}
\usepackage{fancyhdr}

% usual packages loading:
\usepackage{luatextra}
\usepackage{graphicx} % support the \includegraphics command and options
\usepackage{gregoriotex} % for gregorio score inclusion
\usepackage{gregoriosyms}
\usepackage{wrapfig} % figures wrapped by the text
\usepackage{parcolumns}
\usepackage[contents={},opacity=1,scale=1,color=black]{background}
\usepackage{tikzpagenodes}
\usepackage{calc}
\usepackage{longtable}
\usetikzlibrary{calc}

\setlength{\headheight}{14.5pt}

\input{conventuscommune.tex} % Often used macros
%%%% Preklady jednotlivych zpevu (nektere se opakuji, a je dobre mit je
% vsechny na jedne hromade)

% HOURS ---

\newcommand{\trAntI}{\translatioCantus{Muž boží měl kožený toulec, pečlivě
zavázaný, jenž mu visel na šíji a~často se ho dotýkal.}}

\newcommand{\trAntII}{\translatioCantus{Klíč od~něho tak dobře střežil, že
dokud žil v~těle, nikdo z~jeho žáků nezvěděl, co je uvnitř.}}

\newcommand{\trAntIII}{\translatioCantus{Ale když se odebral z~tohoto
života, schránku otevřeli a~objevili v~ní žíněné roucho a~měděný řetěz
potřísněný krví.}}

\newcommand{\trAntIV}{\translatioCantus{A když prohlédli mistrovo tělo,
nalezli jeho tělo na čtyřech místech hluboce zbrázděno ranami od řetězu.}}

\newcommand{\trAntV}{\translatioCantus{Krev vytékající z~těch ran, místy
prostoupila i~žíněným rouchem.}}

\newcommand{\trCapituli}{\translatioCantus{
Miláčkovi Boha a~lidí,
Mojžíšovi požehnané paměti,~\gredagger{}
dopřál slávu rovnou slávě svatých~\grestar{}
učinil ho mocným na postrach nepřátelům
a~jeho slovy zastavil divy.}}

\newcommand{\trLectioBrevis}{\translatioCantus{
Pamatujte na své představené,
kteří vám hlásali Boží slovo.
Uvažte, jak oni skončili život, a~napodobujte jejich víru.
Ježíš Kristus je stejný včera i~dnes i~navěky.
Nenechte se svést věelijakými cizími naukami.}}

\newcommand{\trRespLaud}{\translatioCantus{Spravedlivého vodil Hospodin~\grestar{}
po přímých stezkách. \Vbardot{} A~ukázal mu Boží království.}}

\newcommand{\trRespLaudB}{\translatioCantus{Na tvých hradbách, Jeruzaléme,
ustanovil jsem strážné;~\grestar{}
budou bdít nad mým lidem. \Vbardot{} Ani ve dne, ani v~noci nesmějí nikdy
mlčet.}}

\newcommand{\trVersus}{\translatioCantus{\Vbardot{} Ústa spravedlivého šeptají moudrost, aleluja.
\Rbardot{} A~jeho jazyk ohlašuje právo, aleluja.}}

\newcommand{\trAntBenedictus}{\translatioCantus{Když na bujné oře vložili
nosítka a~sňali jim uzdu, vydali se přímo k~cele božího muže.}}

\newcommand{\trPreces}{\translatioCantus{
\noindent S vděčností chvalme Krista, dobrého Pastýře, \gredagger{} který dal život za své ovce, \grestar{} a~pokorně ho prosme: \Rbardot{} Pane, buď pastýřem svého lidu.

\noindent Kriste, ty dáváš církvi pastýře, a~jejich službou se ujímáš svého lidu, \grestar{} dej, ať v~lásce těch, kteří nás vedou, poznáváme, jak nás miluješ. \Rbardot{} Pane, buď pastýřem svého lidu.

\noindent Ty stále konáš skrze své zástupce službu pastýře a~učitele, \grestar{} nepřestávej nás nikdy vést prostřednictvím svých služebníků. \Rbardot{} Pane, buď pastýřem svého lidu.

\noindent Ty prokazuješ svému lidu skrze jeho pastýře službu lékaře duše i~těla, \grestar{} ochraňuj náš život a~veď nás ke svatosti. \Rbardot{} Pane, buď pastýřem svého lidu.

\noindent Ty posíláš své svaté, aby slovem i~příkladem vedli tvůj lid k~tobě, \grestar{} na jejich přímluvu nás posiluj, abychom vytrvali na cestě, která vede k~věčnému životu. \Rbardot{} Pane, buď pastýřem svého lidu.}}

\newcommand{\trOrationis}{\translatioCantus{Bože, jenž nám dopřáváš radovat
se z~výroční slavnosti svatého tvého vyznavače Havla, uděl dobrotivě,
abychom když slavíme jeho narození, též se řídili podobou jeho skutků.
Skrze…}}
 % Czech translations of the proper texts

\newcommand{\annusEditionis}{2020}

\def\hebinitial#1{%
\leavevmode{\newbox\hebbox\setbox\hebbox\hbox{\hebfont{#1}\hskip 1mm}\kern -\wd\hebbox\hbox{\hebfont{#1}\hskip 1mm}}%
}

%%%% Vicekrat opakovane kousky

\newcommand{\anteOrationem}{
  \rubrica{Ante Orationem, cantatur a Superiore:}

  \pars{Supplicatio Litaniæ.}

  \cuminitiali{}{temporalia/supplicatiolitaniae.gtex}

  \pars{Oratio Dominica.}

  \cuminitiali{}{temporalia/oratiodominica.gtex}

  \rubrica{Deinde dicitur ab Hebdomadario:}

  \cuminitiali{}{temporalia/dominusvobiscum-solemnis.gtex}

  \rubrica{In choro monialium loco Dominus vobiscum dicitur:}

  \sineinitiali{temporalia/domineexaudi.gtex}
}

\setlength{\columnsep}{30pt} % prostor mezi sloupci

%%%%%%%%%%%%%%%%%%%%%%%%%%%%%%%%%%%%%%%%%%%%%%%%%%%%%%%%%%%%%%%%%%%%%%%%%%%%%%%%%%%%%%%%%%%%%%%%%%%%%%%%%%%%%
\begin{document}

% Here we set the space around the initial.
% Please report to http://home.gna.org/gregorio/gregoriotex/details for more details and options
\grechangedim{afterinitialshift}{2.2mm}{scalable}
\grechangedim{beforeinitialshift}{2.2mm}{scalable}

\grechangedim{interwordspacetext}{0.32 cm plus 0.15 cm minus 0.05 cm}{scalable}%
\grechangedim{annotationraise}{-0.2cm}{scalable}

% Here we set the initial font. Change 38 if you want a bigger initial.
% Emit the initials in red.
\grechangestyle{initial}{\color{red}\fontsize{38}{38}\selectfont}

\pagestyle{empty}

%%%% Titulni stranka
\begin{titulusOfficii}
\nomenFesti{Feria IV \hebdomada{}}
\end{titulusOfficii}

\pagebreak

% graphic
\renewcommand{\headrulewidth}{0pt} % no horiz. rule at the header
\fancyhf{}
\pagestyle{fancy}

\cantusSineNeumas

\hora{Ad Matutinum.}

\vspace{2mm}

\cuminitiali{}{temporalia/dominelabiamea.gtex}

\vspace{2mm}

\pars{Invitatorium.} \scriptura{Lc. 24, 34; Psalmus 94; \textbf{H232}}

\vspace{-6mm}

\antiphona{VI}{temporalia/inv-surrexitdominusvere.gtex}

\vfill
\pagebreak

\pars{Hymnus.}

\vspace{-5mm}

\scriptura{\textbf{AR454}}

{
\grechangedim{interwordspacetext}{0.30 cm plus 0.15 cm minus 0.05 cm}{scalable}%
\antiphona{IV}{temporalia/hym-RexSempiterne.gtex}
\grechangedim{interwordspacetext}{0.32 cm plus 0.15 cm minus 0.05 cm}{scalable}%
}
%{
%\vspace{-5mm}
%\setlength{\columnsep}{0pt} % prostor mezi sloupci
%\input{hym-RexSempiterne-bohtext.tex}
%\setlength{\columnsep}{30pt} % prostor mezi sloupci
%}

\vfill
\pagebreak

\pars{Psalmus 1.}

%\vspace{-5mm}

\antiphona{I g}{temporalia/ant-alleluia-fiv-matutinum.gtex}

%\vspace{-5mm}

\scriptura{Ps. 44, 2-10}

%\vspace{-2mm}

\initiumpsalmi{temporalia/ps44i-initium-i-g-auto.gtex}

%\psalmusEtTranslatioT{temporalia/ps44i-III-comb.tex}{10cm}

\input{temporalia/ps44i-III.tex}

\vfill
\pagebreak

\pars{Psalmus 2.} \scriptura{Ps. 44, 11-18}

%\vspace{-2mm}

\initiumpsalmi{temporalia/ps44ii-initium-i-g-auto.gtex}

%\psalmusEtTranslatioT{temporalia/ps44i-III-comb.tex}{10cm}

\input{temporalia/ps44ii-III.tex}

\vfill
\pagebreak

\pars{Psalmus 3.} \scriptura{Ps. 45}

%\vspace{-2mm}

\initiumpsalmi{temporalia/ps45-initium-i-g-auto.gtex}

%\psalmusEtTranslatioT{temporalia/ps45-III-comb.tex}{10cm}

\input{temporalia/ps45-III.tex}

\vfill
\pagebreak

\pars{Psalmus 4.} \scriptura{Ps. 47}

%\vspace{-2mm}

\initiumpsalmi{temporalia/ps47-initium-i-g-auto.gtex}

%\psalmusEtTranslatioT{temporalia/ps47-III-comb.tex}{10cm}

\input{temporalia/ps47-III.tex}

\vfill
\pagebreak

\pars{Psalmus 5.} \scriptura{Ps. 48, 2-13}

%\vspace{-2mm}

\initiumpsalmi{temporalia/ps48i-initium-i-g-auto.gtex}

%\psalmusEtTranslatioT{temporalia/ps48i-III-comb.tex}{10cm}

\input{temporalia/ps48i-III.tex}

\vfill
\pagebreak

\pars{Psalmus 6.} \scriptura{Ps. 48, 14-21}

%\vspace{-2mm}

\initiumpsalmi{temporalia/ps48ii-initium-i-g-auto.gtex}

%\psalmusEtTranslatioT{temporalia/ps48ii-III-comb.tex}{10cm}

\input{temporalia/ps48ii-III.tex}

\vfill
\pagebreak

\pars{Psalmus 7.} \scriptura{Ps. 49, 1-15}

%\vspace{-2mm}

\initiumpsalmi{temporalia/ps49i-initium-i-g-auto.gtex}

%\psalmusEtTranslatioT{temporalia/ps49i-III-comb.tex}{10cm}

\input{temporalia/ps49i-III.tex}

\vfill
\pagebreak

\pars{Psalmus 8.} \scriptura{Ps. 49, 16-23}

%\vspace{-2mm}

\initiumpsalmi{temporalia/ps49ii-initium-i-g-auto.gtex}

%\psalmusEtTranslatioT{temporalia/ps49ii-III-comb.tex}{10cm}

\input{temporalia/ps49ii-III.tex}

\vfill
\pagebreak

\pars{Psalmus 9.} \scriptura{Ps. 50}

%\vspace{-2mm}

\initiumpsalmi{temporalia/ps50-initium-i-g-auto.gtex}

%\psalmusEtTranslatioT{temporalia/ps50-VI-comb.tex}{10cm}

\input{temporalia/ps50-VI.tex}

\vfill
%\pagebreak

\antiphona{}{temporalia/ant-alleluia-fiv-matutinum.gtex}

\vfill
\pagebreak

\noindent \Vbardot{} Gavísi sunt discípuli, allelúia.
\noindent \Rbardot{} Viso Dómino, allelúia.

\noindent Pater noster.

\pars{Absolutio.}

\cuminitiali{}{temporalia/absolutio-avinculis.gtex}

\vfill
\pagebreak

\ifx\magnificat\undefined
\cuminitiali{}{temporalia/benedictio-solemn-evangelica.gtex}
\else
\cuminitiali{}{temporalia/benedictio-solemn-ille.gtex}
\fi

\vspace{7mm}

\lectioi

\noindent \Vbardot{} Tu autem, Dómine, miserére nobis.
\noindent \Rbardot{} Deo grátias.

\vfill
\pagebreak

\responsoriumi

\vfill
\pagebreak

\cuminitiali{}{temporalia/benedictio-solemn-divinum.gtex}

\vspace{7mm}

\lectioii

\noindent \Vbardot{} Tu autem, Dómine, miserére nobis.
\noindent \Rbardot{} Deo grátias.

\vfill
\pagebreak

\responsoriumii

\vfill
\pagebreak

\ifx\magnificat\undefined
\cuminitiali{}{temporalia/benedictio-solemn-adsocietatem.gtex}
\else
\cuminitiali{}{temporalia/benedictio-solemn-ignem.gtex}
\fi

\vspace{7mm}

\lectioiii

\noindent \Vbardot{} Tu autem, Dómine, miserére nobis.
\noindent \Rbardot{} Deo grátias.

\vfill
\pagebreak

% Te Deum

%\pars{Hymnus Ambrosianus}

\vspace{-5mm}

{
\grechangedim{interwordspacetext}{0.22 cm plus 0.15 cm minus 0.05 cm}{scalable}%
\cuminitiali{III}{temporalia/tedeum-solemnis.gtex}
\grechangedim{interwordspacetext}{0.32 cm plus 0.15 cm minus 0.05 cm}{scalable}%
}

\vfill
\pagebreak

\rubrica{Reliqua omittuntur, nisi Laudes separandæ sint.}

\pars{Oratio}

\noindent \Vbardot{} Dómine, exáudi oratiónem meam.

\noindent \Rbardot{} Et clamor meus ad te véniat.

Orémus:

\oratioMatutinum

\noindent \Rbardot{} Amen.

\vspace{7mm}

\pars{Conclusio}

\noindent \Vbardot{} Dómine, exáudi oratiónem meam.

\noindent \Rbardot{} Et clamor meus ad te véniat.

\noindent \Vbardot{} Benedicámus Dómino, allelúia, allelúia.

\noindent \Rbardot{} Deo grátias, allelúia, allelúia.

\noindent \Vbardot{} Fidélium ánimæ per misericórdiam Dei requiéscant in pace.

\noindent \Rbardot{} Amen.

\vfill
\pagebreak

\hora{Ad Laudes.} %%%%%%%%%%%%%%%%%%%%%%%%%%%%%%%%%%%%%%%%%%%%%%%%%%%%%
%\sideThumbs{Laudes}

\cantusSineNeumas

\vspace{0.5cm}
\grechangedim{interwordspacetext}{0.18 cm plus 0.15 cm minus 0.05 cm}{scalable}%
\cuminitiali{}{temporalia/deusinadiutorium-communis.gtex}
\grechangedim{interwordspacetext}{0.32 cm plus 0.15 cm minus 0.05 cm}{scalable}%

\vfill
%\pagebreak

\pars{Psalmus 1.}

\vspace{-0.4cm}

\antiphona{VII a}{temporalia/ant-alleluia-fiv-laudes-1.gtex}

\scriptura{Psalmus 50.}

\initiumpsalmi{temporalia/ps50-initium-vii-a-auto.gtex}

%\psalmusEtTranslatioT{temporalia/ps50-III-comb.tex}{10cm}
\input{temporalia/ps50-III.tex}

\vspace{-1cm}

\vfill
\pagebreak

\pars{Psalmus 2.} \scriptura{Psalmus 63.}

\initiumpsalmi{temporalia/ps63-initium-vii-a-auto.gtex}

%\psalmusEtTranslatioT{temporalia/ps63-III-comb.tex}{10cm}
\input{temporalia/ps63-III.tex}

\vfill
\pagebreak

\pars{Psalmus 3.} \scriptura{Psalmus 64.}

\initiumpsalmi{temporalia/ps64-initium-vii-a-auto.gtex}

%\psalmusEtTranslatioT{temporalia/ps64-III-comb.tex}{10cm}
\input{temporalia/ps64-III.tex}

\vfill

\vspace{-6mm}

\antiphona{}{temporalia/ant-alleluia-fiv-laudes-1.gtex} % repeat the antiphon - new page

\vfill
\pagebreak

\pars{Psalmus 4.} \scriptura{1 Sam. 2, 10; \textbf{H96}}

\vspace{-7mm}

\antiphona{I g\textsuperscript{2}}{temporalia/ant-dominusjudicabit-tp.gtex}

%\vspace{-4mm}

\scriptura{Canticum Annæ, 1 Reg. 2, 1-10}

%\vspace{-3mm}

\initiumpsalmi{temporalia/anna-initium-i-g2-auto.gtex}

%\psalmusEtTranslatioT{temporalia/anna-comb.tex}{10cm}
\input{temporalia/anna.tex}

%\vfill

\antiphona{}{temporalia/ant-dominusjudicabit-tp.gtex}

\vfill
\pagebreak

\pars{Psalmus 5.}

\vspace{-0.4cm}

\antiphona{II D}{temporalia/ant-alleluia-fiv-laudes-2.gtex}

\scriptura{Psalmus 148.}

\initiumpsalmi{temporalia/ps148-initium-ii-D-auto.gtex}

%\psalmusEtTranslatioT{temporalia/ps148-III-comb.tex}{10cm}
\input{temporalia/ps148-III.tex}

\rubrica{Hic non dicitur Gloria Patri.}

\vfill
\pagebreak

%
\scriptura{Psalmus 149.}

\initiumpsalmi{temporalia/ps149-initium-ii-D-auto.gtex}

%\psalmusEtTranslatioT{temporalia/ps149-III-comb.tex}{10cm}
\input{temporalia/ps149-III.tex}

\rubrica{Hic non dicitur Gloria Patri.}

\vfill
\pagebreak

%
\scriptura{Psalmus 150.}

\initiumpsalmi{temporalia/ps150-initium-ii-D-auto.gtex}

%\psalmusEtTranslatioT{temporalia/ps150-III-comb.tex}{10cm}
\input{temporalia/ps150-III.tex}

\vfill

\vspace{-6mm}

\antiphona{}{temporalia/ant-alleluia-fiv-laudes-2.gtex} % repeat the antiphon - new page

\vfill
\pagebreak

\pars{Capitulum.} \scriptura{Rom. 6, 9-10}

\grechangedim{interwordspacetext}{0.12 cm plus 0.15 cm minus 0.05 cm}{scalable}%
\cuminitiali{}{temporalia/capitulum-ChristusResurgens.gtex}
\grechangedim{interwordspacetext}{0.32 cm plus 0.15 cm minus 0.05 cm}{scalable}%

% preklad Jeruz. bible
%\trCapituliI

\vfill

\pars{Responsorium breve.} \scriptura{Cf. Mt. 28, 6; Cf. Gal. 3, 13}

\cuminitiali{VI}{temporalia/respbr-laud.gtex}

%\trResp

\vfill
\pagebreak

\pars{Hymnus}

\cuminitiali{VIII}{temporalia/hym-AuroraLucis.gtex}
\vspace{-3mm}
%\input{hym-AuroraLucis-bohtext.tex}

\vfill
%\pagebreak

\pars{Versus.}

% Versus. %%%
\sineinitiali{temporalia/versus-inresurrectione.gtex}

%\noindent \trVersus

\vfill
\pagebreak

\benedictus

\vspace{-1cm}

\vfill
\pagebreak

%\sideThumbs{{\scriptsize{}Fine horarum}}

\anteOrationem

\pagebreak

% Oratio. %%%
\oratioLaudes

\vspace{-1mm}
%\trOrationisI

\vfill

\rubrica{Hebdomadarius dicit iterum Dominus vobiscum. Postea cantatur a cantore:}
\vspace{2mm}

\cuminitiali{VII}{temporalia/benedicamus-tempore-paschali.gtex}

\vspace{1mm}

\ifx\magnificat\undefined
\else
\vfill
\pagebreak

\hora{Ad Vesperas.} %%%%%%%%%%%%%%%%%%%%%%%%%%%%%%%%%%%%%%%%%%%%%%%%%%%%%
%\sideThumbs{Vesperæ}

\cantusSineNeumas

%\vspace{0.5cm}
\grechangedim{interwordspacetext}{0.18 cm plus 0.15 cm minus 0.05 cm}{scalable}%
\cuminitiali{}{temporalia/deusinadiutorium-communis.gtex}
\grechangedim{interwordspacetext}{0.32 cm plus 0.15 cm minus 0.05 cm}{scalable}%

\vfill
%\pagebreak

\vspace{4mm}

\pars{Psalmus 1.}

\vspace{-0.4cm}

\antiphona{III g}{temporalia/ant-alleluia-fiv-vesperas.gtex}

\vspace{-4mm}

\scriptura{Psalmus 134.}

\initiumpsalmi{temporalia/ps134-initium-iii-g-auto.gtex}

%\psalmusEtTranslatioT{temporalia/ps134-III-comb.tex}{10cm}
\input{temporalia/ps134-III.tex}

\vspace{-1cm}

\vfill
\pagebreak

\pars{Psalmus 2.} \scriptura{Psalmus 135.}

\initiumpsalmi{temporalia/ps135-initium-iii-g-auto.gtex}

%\psalmusEtTranslatioT{temporalia/ps135-III-comb.tex}{10cm}
\input{temporalia/ps135-III.tex}

\vfill
\pagebreak

\pars{Psalmus 3.} \scriptura{Psalmus 136.}

\initiumpsalmi{temporalia/ps136-initium-iii-g-auto.gtex}

%\psalmusEtTranslatioT{temporalia/ps136-III-comb.tex}{10cm}
\input{temporalia/ps136-III.tex}

\vfill
\pagebreak

\pars{Psalmus 4.} \scriptura{Psalmus 137.}

\initiumpsalmi{temporalia/ps137-initium-iii-g-auto.gtex}

%\psalmusEtTranslatioT{temporalia/ps137-III-comb.tex}{10cm}
\input{temporalia/ps137-III.tex}

\vfill

\vspace{-6mm}

\antiphona{}{temporalia/ant-alleluia-fiv-vesperas.gtex} % repeat the antiphon - new page

\vfill
\pagebreak

\pars{Capitulum.} \scriptura{Rom. 6, 9-10}

\grechangedim{interwordspacetext}{0.12 cm plus 0.15 cm minus 0.05 cm}{scalable}%
\cuminitiali{}{temporalia/capitulum-ChristusResurgens.gtex}
\grechangedim{interwordspacetext}{0.32 cm plus 0.15 cm minus 0.05 cm}{scalable}%

% preklad Jeruz. bible
%\trCapituliI

\vfill

\pars{Responsorium breve.} \scriptura{Lc. 24, 34}

\cuminitiali{VI}{temporalia/respbr-vesp.gtex}

%\trResp

\vfill
\pagebreak

\pars{Hymnus}

\cuminitiali{VIII}{temporalia/hym-AdCoenam.gtex}
\vspace{-3mm}
%\begin{translatioMulticol}{4}
U~Beránkovy hostiny\\
oděni rouchy bílými,\\
když Rudým mořem prošli jsme,\\
Vladaři Kristu zpívejme.\\
\\
Když jeho tělem posvátným,\\
na kříži obětovaným,\\
se sytíme a~pijeme\\
jeho krev, v~Bohu žijeme.\columnbreak

Chráněni tímto pokrmem\\
před smrtonosným andělem,\\
svrhli jsme z~beder kruté jho\\
tyrana bezohledného.\\
\\
Kristus je naší paschou teď,\\
on sám se vydal za oběť\\
a~místo přesnic našim rtům\\
své tělo dává za pokrm.\columnbreak

Tys, nejčistější Oběti,\\
zlomila vládu podsvětí.\\
Z~otroctví lid je vykoupen,\\
odměna žití kyne všem.\\
\\
Hle, Kristus, když vstal ze hrobu,\\
jde z~pekel v~slavném průvodu\\
a~brány nebes otevřev,\\
vládce tmy vleče v~okovech.\columnbreak

Buď věčně, Kriste, věrným svým\\
plesáním velikonočním.\\
Nás, milostí tvou vzkříšené,\\
vem k~oslavě své vítězné. \\
\\
Sláva tobě, Pane,\\
jenž jsi vstal z~mrtvých,\\
s~Otcem i~Svatým Duchem\\
na věčné věky.\\
Amen.
\end{translatioMulticol}


\vfill
\pagebreak

\pars{Versus.} \scriptura{Lc. 24, 29}

% Versus. %%%
\sineinitiali{temporalia/versus-mane.gtex}

%\noindent \trVersus

\vfill
\pagebreak

\magnificat

\vspace{-1cm}

\vfill
\pagebreak

%\sideThumbs{{\scriptsize{}Fine horarum}}

\anteOrationem

\pagebreak

% Oratio. %%%
\oratioLaudes

\vspace{-1mm}
%\trOrationisI

\vfill

\rubrica{Hebdomadarius dicit iterum Dominus vobiscum. Postea cantatur a cantore:}
\vspace{2mm}

\cuminitiali{VII}{temporalia/benedicamus-tempore-paschali.gtex}

\vspace{1mm}
\fi

\end{document}

