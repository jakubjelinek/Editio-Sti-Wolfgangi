\newcommand{\titulus}{\nomenFesti{S. Antonii Patavini, Presbyteri \& Ecclesiæ Doctoris..}
\dies{Die 13. Iunii.}}
\newcommand{\oratio}{\pars{Oratio.}

\noindent Omnípotens sempitérne Deus, qui pópulo tuo beátum Antónium prædicatórem insígnem dedísti, eúmque in necessitátibus intercessórem, concéde, ut, eius auxílio, christiánæ vitæ documénta sectántes, in ómnibus adversitátibus te subveniéntem sentiámus.

\pars{Pro pace in Ucraina.} \scriptura{Sir. 50, 25; 2 Esdr. 4, 20; \textbf{H416}}

\vspace{-4mm}

\antiphona{II D}{temporalia/ant-dapacemdomine.gtex}

\vfill

\noindent Deus, a quo sancta desidéria, recta consília et iusta sunt ópera: da servis tuis illam, quam mundus dare non potest, pacem; ut et corda nostra mandátis tuis dédita, et hóstium subláta formídine, témpora sint tua protectióne tranquílla.

\noindent Per Dóminum nostrum Iesum Christum, Fílium tuum, qui tecum vivit et regnat in unitáte Spíritus Sancti, Deus, per ómnia sǽcula sæculórum.

\noindent \Rbardot{} Amen.}
\newcommand{\invitatorium}{\pars{Invitatorium.}

\vspace{-4mm}

\antiphona{IV}{temporalia/inv-mirabileminsanctis.gtex}}
\newcommand{\hymnusmatutinum}{\pars{Hymnus}

\cuminitiali{IV}{temporalia/hym-LaetiColentes.gtex}}
\newcommand{\matversus}{\noindent \Vbardot{} Iustum dedúxit Dóminus per vias rectas.

\noindent \Rbardot{} Et osténdit illi regnum Dei.}
\newcommand{\lectioi}{\pars{Lectio I.} \scriptura{Ios. 2, 1-12}

\noindent De libro Iósue.

\noindent In diébus illis: Misit ergo Iósue fílius Nun de Settim duos viros exploratóres in abscóndito et dixit eis: «Ite et consideráte terram urbémque Iéricho». Qui pergéntes ingréssi sunt domum mulíeris meretrícis nómine Rahab et quievérunt ibi. Nuntiatúmque est regi Iéricho et dictum: «Ecce viri ingréssi sunt huc per noctem de fíliis Israel, ut explorárent terram». Misítque rex Iéricho ad Rahab dicens: «Educ viros, qui venérunt ad te et ingréssi sunt domum tuam; exploratóres quippe sunt et omnem terram consideráre venérunt». Tollénsque múlier viros abscóndit et ait: «Fáteor, venérunt ad me, sed nesciébam unde essent; cumque porta clauderétur in ténebris, et illi páriter exiérunt, néscio quo abiérunt. Persequímini cito et comprehendétis eos». Ipsa autem fecit ascéndere viros in solárium domus suæ operuítque eos lini stípula, quæ ibi erat. Hi autem, qui missi fúerant, secúti sunt eos per viam, quæ ducit ad vadum Iordánis; illísque egréssis, statim porta clausa est. 

\noindent Necdum obdormíerant qui latébant, et ecce múlier ascéndit ad eos et ait: «Novi quod tradíderit Dóminus vobis terram, et írruit in nos terror vester, et elanguérunt omnes habitatóres terræ coram vobis. Audívimus enim quod siccáverit Dóminus aquas maris Rubri ad vestrum intróitum, quando egréssi estis ex Ægýpto, et quæ fecéritis duóbus Amorræórum régibus, qui erant trans Iordánem, Sehon et Og, quos interfecístis. Et hæc audiéntes pertimúimus, et elánguit cor nostrum, nec remánsit in nobis spíritus ad intróitum vestrum; Dóminus enim Deus vester ipse est Deus in cælo sursum et in terra deórsum.}
\newcommand{\responsoriumi}{\pars{Responsorium 1.} \scriptura{\Rbardot{} Ps. 58, 18 \Vbardot{} ibid., 2; \textbf{H87}}

\vspace{-5mm}

\responsorium{VII}{temporalia/resp-adiutormeustibipsallam-CROCHU.gtex}{}}
\newcommand{\lectioii}{\pars{Lectio II.} \scriptura{Ios. 2, 12-21}

\noindent Nunc ergo iuráte mihi per Dóminum, ut, quómodo ego feci vobíscum misericórdiam, ita et vos faciátis cum domo patris mei detísque mihi signum verum, ut salvétis patrem meum et matrem, fratres ac soróres meas et ómnia, quæ eórum sunt et eruátis ánimas nostras de morte». Qui respondérunt ei: «Anima nostra sit pro vobis in mortem, si tamen non prodíderis; cumque tradíderit nobis Dóminus terram, faciémus in te misericórdiam et veritátem». Demísit ergo eos per funem de fenéstra; domus enim eius hærébat muro, et in muro habitábat. Dixítque ad eos: «Ad montána pérgite, ne forte occúrrant vobis prosecutóres, ibíque latéte diébus tribus, donec rédeant; et póstea íbitis per viam vestram». 

\noindent Qui dixérunt ad eam: «Innóxii érimus a iuraménto hoc, quo adiurásti nos, si, ingrediéntibus nobis terram, signum fúerit funículus iste coccíneus, et ligáveris eum in fenéstra, per quam nos demisísti et patrem tuum ac matrem fratrésque et omnem cognatiónem tuam congregáveris in domum tuam. Qui óstium domus tuæ egréssus fúerit, sanguis ipsíus erit in caput eius, et nos érimus innóxii; cunctórum autem sanguis, qui tecum fúerint in domo, redundábit in caput nostrum, si eos áliquis tetígerit. Quod si prodíderis hoc verbum, érimus mundi ab hoc iuraménto, quo adiurásti nos». Et illa respóndit: «Sicut locúti estis, ita fiat». Dimitténsque eos, ut pérgerent, appéndit funículum coccíneum in fenéstra.}
\newcommand{\responsoriumii}{\pars{Responsorium 2.} \scriptura{\Rbardot{} Ps. 33, 2 \Vbardot{} ibid., 3; \textbf{H85}}

\vspace{-5mm}

\responsorium{V}{temporalia/resp-benedicamdominum-CROCHU.gtex}{}}
\newcommand{\lectioiii}{\pars{Lectio III.} \scriptura{I, 226}

\noindent Ex Sermónibus sancti Antónii presbýteri.

\noindent Qui Spíritu Sancto replétur, váriis linguis lóquitur. Váriæ linguæ sunt vária de Christo testimónia, útpote humílitas, paupértas, patiéntia et obœdiéntia, quibus tunc lóquimur, cum eas in nobis ipsis áliis osténdimus. Viva est loquéla cum loquúntur ópera. Cessent, óbsecro, verba, loquántur ópera. Verbis sumus pleni, sed opéribus vácui, et ídeo a Dómino maledícti, quia ipse maledíxit ficúlneæ in qua non invénit fructum sed fólia tantum. «Lex, inquit Gregórius, prædicatóri pósita est, ut ea quæ prǽdicat operétur». Frustra iactat legis notítiam, qui opéribus déstruit doctrínam.

\noindent Sed Apóstoli \emph{loquebántur prout Spíritus Sanctus dabat éloqui illis.} Beátus qui lóquitur prout dat Spíritus Sanctus, non prout suus ánimus! Sunt enim quidam qui ex suo spíritu loquúntur, verba aliórum furántur et ea ut sua propónunt et sibi attríbuunt. De tálibus et iis simílibus dicit Dóminus in Ieremía: \emph{Ecce ego ad prophétas, qui furántur verba mea unusquísque a próximo suo. Ecce ego ad prophétas, dicit Dóminus, qui assúmunt linguas suas et aiunt: Dicit Dóminus; ecce ego ad prophétas somniántes mendácium, ait Dóminus, qui narravérunt ea et seduxérunt pópulum meum in mendácio suo et in miráculis suis; cum ego non misíssem eos nec mandássem eis, qui nihil profuérunt pópulo huic, dicit Dóminus.}

\noindent Loquámur ergo prout Spíritus Sanctus déderit éloqui nobis, ipsum humíliter et devóte postulántes, quátenus grátiam suam infúndat, ut diem Pentecóstes in quinque sénsuum perfectióne et decálogi observatióne compleámus, et spíritu veheménti contritiónis repleámur, et linguis ígneis confessiónis accendámur, ut accénsi et illumináti in splendóribus sanctórum, Deum trinum et unum vidére mereámur.}
\newcommand{\responsoriumiii}{\pars{Responsorium 3.} \scriptura{\Rbardot{} Ps. 8, 6-8 \Vbardot{} Ps. 20, 4; \textbf{H374}}

\vspace{-5mm}

\responsorium{VII}{temporalia/resp-gloriaethonore-CROCHU-cumdox.gtex}{}}
\newcommand{\hymnuslaudes}{\pars{Hymnus}

\cuminitiali{VIII}{temporalia/hym-ORedemptorisPietas.gtex}}
\newcommand{\lectiobrevis}{\pars{Lectio Brevis.} \scriptura{Sap. 7, 13-14}

\noindent Sapiéntiam sine fictióne dídici et sine invídia commúnico; divítias illíus non abscóndo. Infinítus enim thesáurus est homínibus; quem qui acquisiérunt, ad amicítiam in Deum se paravérunt propter disciplínæ dona commendáti.}
\newcommand{\responsoriumbreve}{\pars{Responsorium breve.} \scriptura{Sir. 45, 9}

\antiphona{VI}{temporalia/resp-amaviteum.gtex}}
\newcommand{\benedictus}{\pars{Canticum Zachariæ.} \scriptura{Lc. 14, 33}

\vspace{-4mm}

\antiphona{VII c\textsuperscript{2}}{temporalia/ant-christusnobisdicit.gtex}

\vspace{-2mm}

\scriptura{Lc. 1, 68-79}

\vspace{-1mm}

\cantusSineNeumas
\initiumpsalmi{temporalia/benedictus-initium-vii-c2-auto.gtex}

%\vspace{-1.5mm}

\input{temporalia/benedictus-vii-c2.tex} \Abardot{}}
\newcommand{\preces}{\noindent Christo, bono pastóri, \gredagger{} qui pro suis óvibus ánimam pósuit, \grestar{} laudes grati exsolvámus et supplicémus, dicéntes:

\Rbardot{} Pasce pópulum tuum, Dómine.

\noindent Christe, qui in sanctis pastóribus misericórdiam et dilectiónem tuam dignátus es osténdere, \grestar{} numquam désinas per eos nobíscum misericórditer ágere.

\Rbardot{} Pasce pópulum tuum, Dómine.

\noindent Qui múnere pastóris animárum fungi per tuos vicários pergis, \grestar{} ne destíteris nos ipse per rectóres nostros dirígere.

\Rbardot{} Pasce pópulum tuum, Dómine.

\noindent Qui in sanctis tuis, populórum dúcibus, córporum animarúmque médicus exstitísti, \grestar{} numquam cesses ministérium in nos vitæ et sanctitátis perágere.

\Rbardot{} Pasce pópulum tuum, Dómine.

\noindent Qui, prudéntia et caritáte sanctórum, tuum gregem erudísti, \grestar{} nos in sanctitáte iúgiter per pastóres nostros ædífica.

\Rbardot{} Pasce pópulum tuum, Dómine.}
\newcommand{\benedicamuslaudes}{\cuminitiali{}{temporalia/benedicamus-memoria-laudes.gtex}}
\newcommand{\hebdomada}{infra Hebdom. X post Pentecosten.}
\newcommand{\oratioLaudes}{\cuminitiali{}{temporalia/oratio10.gtex}}

% LuaLaTeX

\documentclass[a4paper, twoside, 12pt]{article}
\usepackage[latin]{babel}
%\usepackage[landscape, left=3cm, right=1.5cm, top=2cm, bottom=1cm]{geometry} % okraje stranky
%\usepackage[landscape, a4paper, mag=1166, truedimen, left=2cm, right=1.5cm, top=1.6cm, bottom=0.95cm]{geometry} % okraje stranky
\usepackage[landscape, a4paper, mag=1400, truedimen, left=0.5cm, right=0.5cm, top=0.5cm, bottom=0.5cm]{geometry} % okraje stranky

\usepackage{fontspec}
\setmainfont[FeatureFile={junicode.fea}, Ligatures={Common, TeX}, RawFeature=+fixi]{Junicode}
%\setmainfont{Junicode}

% shortcut for Junicode without ligatures (for the Czech texts)
\newfontfamily\nlfont[FeatureFile={junicode.fea}, Ligatures={Common, TeX}, RawFeature=+fixi]{Junicode}

\usepackage{multicol}
\usepackage{color}
\usepackage{lettrine}
\usepackage{fancyhdr}

% usual packages loading:
\usepackage{luatextra}
\usepackage{graphicx} % support the \includegraphics command and options
\usepackage{gregoriotex} % for gregorio score inclusion
\usepackage{gregoriosyms}
\usepackage{wrapfig} % figures wrapped by the text
\usepackage{parcolumns}
\usepackage[contents={},opacity=1,scale=1,color=black]{background}
\usepackage{tikzpagenodes}
\usepackage{calc}
\usepackage{longtable}
\usetikzlibrary{calc}

\setlength{\headheight}{14.5pt}

\input{conventuscommune.tex} % Often used macros

\newcommand{\annusEditionis}{2021}

%%%% Vicekrat opakovane kousky

\newcommand{\anteOrationem}{
  \rubrica{Ante Orationem, cantatur a Superiore:}

  \pars{Supplicatio Litaniæ.}

  \cuminitiali{}{temporalia/supplicatiolitaniae.gtex}

  \pars{Oratio Dominica.}

  \cuminitiali{}{temporalia/oratiodominica.gtex}

  \rubrica{Deinde dicitur ab Hebdomadario:}

  \cuminitiali{}{temporalia/dominusvobiscum-solemnis.gtex}

  \rubrica{In choro monialium loco Dominus vobiscum dicitur:}

  \sineinitiali{temporalia/domineexaudi.gtex}
}

\setlength{\columnsep}{30pt} % prostor mezi sloupci

%%%%%%%%%%%%%%%%%%%%%%%%%%%%%%%%%%%%%%%%%%%%%%%%%%%%%%%%%%%%%%%%%%%%%%%%%%%%%%%%%%%%%%%%%%%%%%%%%%%%%%%%%%%%%
\begin{document}

% Here we set the space around the initial.
% Please report to http://home.gna.org/gregorio/gregoriotex/details for more details and options
\grechangedim{afterinitialshift}{2.2mm}{scalable}
\grechangedim{beforeinitialshift}{2.2mm}{scalable}
\grechangedim{interwordspacetext}{0.22 cm plus 0.15 cm minus 0.05 cm}{scalable}%
\grechangedim{annotationraise}{-0.2cm}{scalable}

% Here we set the initial font. Change 38 if you want a bigger initial.
% Emit the initials in red.
\grechangestyle{initial}{\color{red}\fontsize{38}{38}\selectfont}

\pagestyle{empty}

%%%% Titulni stranka
\begin{titulusOfficii}
\ifx\titulus\undefined
\nomenFesti{Feria III \hebdomada{}}
\else
\titulus
\fi
\end{titulusOfficii}

\vfill

\begin{center}
%Ad usum et secundum consuetudines chori \guillemotright{}Conventus Choralis\guillemotleft.

%Editio Sancti Wolfgangi \annusEditionis
\end{center}

\scriptura{}

\pars{}

\pagebreak

\renewcommand{\headrulewidth}{0pt} % no horiz. rule at the header
\fancyhf{}
\pagestyle{fancy}

\cantusSineNeumas

\ifx\oratio\undefined
\ifx\laudb\undefined
\else
\newcommand{\oratio}{\pars{Oratio.}

\noindent Dómine Iesu Christe, lux vera, qui omnes hómines illúminas ad salútem, nobis, quǽsumus, concéde virtútem, ut ante te vias pacis et iustítiæ præparémus.

\noindent Qui vivis et regnas cum Deo Patre in unitáte Spíritus Sancti, Deus, per ómnia sǽcula sæculórum.

\noindent \Rbardot{} Amen.}
\fi
\fi

\hora{Ad Matutinum.} %%%%%%%%%%%%%%%%%%%%%%%%%%%%%%%%%%%%%%%%%%%%%%%%%%%%%

\vspace{2mm}

\cuminitiali{}{temporalia/dominelabiamea.gtex}

\vfill
%\pagebreak

\vspace{2mm}

\ifx\invitatorium\undefined
\ifx\matuac\undefined
\else
\pars{Invitatorium.} \scriptura{Ps. 94, 1; Psalmus 94; \textbf{H451}}

\vspace{-6mm}

\antiphona{VI}{temporalia/inv-jubilemusdeo.gtex}
\fi
\ifx\matubd\undefined
\else
\pars{Invitatorium.} \scriptura{Cantor; Psalmus 94; \textbf{H449}}

\vspace{-6mm}

\antiphona{E}{temporalia/inv-regemmagnum.gtex}
\fi
\else
\invitatorium
\fi

\vfill
\pagebreak

\ifx\hymnusmatutinum\undefined
\ifx\matuac\undefined
\else
\pars{Hymnus}

\cuminitiali{IV}{temporalia/hym-SomnoRefectis.gtex}
\fi
\ifx\matubd\undefined
\else
\pars{Hymnus.} \scriptura{Gregorius Magnus (\olddag{} 604)}

{
\grechangedim{interwordspacetext}{0.10 cm plus 0.15 cm minus 0.05 cm}{scalable}%
\antiphona{I}{temporalia/hym-NocteSurgentes.gtex}
\grechangedim{interwordspacetext}{0.22 cm plus 0.15 cm minus 0.05 cm}{scalable}%
}
\fi
\else
\hymnusmatutinum
\fi

\vspace{-3mm}

\vfill
\pagebreak

\ifx\matub\undefined
\else
% MAT B
\pars{Psalmus 1.} \scriptura{Ps. 36, 5; \textbf{H93}}

\vspace{-4mm}

\antiphona{VI F}{temporalia/ant-reveladomino.gtex}

%\vspace{-2mm}

\scriptura{Ps. 36, 1-11}

%\vspace{-2mm}

\initiumpsalmi{temporalia/ps36i_xi-initium-vi-F-auto.gtex}

\input{temporalia/ps36i_xi-vi-F.tex} \Abardot{}

\vfill
\pagebreak

\pars{Psalmus 2.}

\vspace{-4mm}

\antiphona{II D}{temporalia/ant-iuniorfui.gtex}

\vspace{-2mm}

\scriptura{Ps. 36, 12-29}

\vspace{-2mm}

\initiumpsalmi{temporalia/ps36xii_xxix-initium-ii-D-auto.gtex}

\input{temporalia/ps36xii_xxix-ii-D.tex}

\vfill

\antiphona{}{temporalia/ant-iuniorfui.gtex}

\vfill
\pagebreak

\pars{Psalmus 3.} \scriptura{Ps. 36, 3}

\vspace{-4mm}

\antiphona{VI F}{temporalia/ant-speraindomino.gtex}

%\vspace{-2mm}

\scriptura{Ps. 36, 30-40}

%\vspace{-2mm}

\initiumpsalmi{temporalia/ps36iii-initium-vi-F-auto.gtex}

\input{temporalia/ps36iii-vi-F.tex} \Abardot{}

\vfill
\pagebreak
\fi
\ifx\matuc\undefined
\else
% MAT C
\pars{Psalmus 1.} \scriptura{Ps. 67, 2}

\vspace{-4mm}

\antiphona{VII a}{temporalia/ant-exsurgatdeus.gtex}

%\vspace{-2mm}

\scriptura{Ps. 67, 2-11}

\initiumpsalmi{temporalia/ps67i-initium-vii-a-auto.gtex}

\input{temporalia/ps67i-vii-a.tex} \Abardot{}

\vfill
\pagebreak

\pars{Psalmus 2.}

\vspace{-4mm}

\antiphona{I f}{temporalia/ant-deusnosterdeussalvos.gtex}

%\vspace{-2mm}

\scriptura{Ps. 67, 12-24}

%\vspace{-2mm}

\initiumpsalmi{temporalia/ps67ii-initium-i-f-auto.gtex}

\input{temporalia/ps67ii-i-f.tex} \Abardot{}

\vfill
\pagebreak

\pars{Psalmus 3.} \scriptura{Ps. 67, 27; \textbf{H96}}

\vspace{-4mm}

\antiphona{D}{temporalia/ant-inecclesiis.gtex}

%\vspace{-2mm}

\scriptura{Ps. 67, 25-36}

\initiumpsalmi{temporalia/ps67iii-initium-d-g2-auto.gtex}

\input{temporalia/ps67iii-d-g2.tex} \Abardot{}

\vfill
\pagebreak
\fi

\pars{Versus.}

\ifx\matversus\undefined
\ifx\matub\undefined
\else
\noindent \Vbardot{} Bonitátem et prudéntiam et sciéntiam doce me.

\noindent \Rbardot{} Quia præcéptis tuis crédidi.
\fi
\ifx\matuc\undefined
\else
\noindent \Vbardot{} Audiam quid loquátur Dóminus Deus.

\noindent \Rbardot{} Loquétur pacem ad plebem suam.
\fi
\else
\matversus
\fi

\vspace{5mm}

\sineinitiali{temporalia/oratiodominica-mat.gtex}

\vspace{5mm}

\pars{Absolutio.}

\cuminitiali{}{temporalia/absolutio-ipsius.gtex}

\vfill
\pagebreak

\cuminitiali{}{temporalia/benedictio-solemn-deus.gtex}

\vspace{7mm}

\lectioi

\noindent \Vbardot{} Tu autem, Dómine, miserére nobis.
\noindent \Rbardot{} Deo grátias.

\vfill
\pagebreak

\responsoriumi

\vfill
\pagebreak

\cuminitiali{}{temporalia/benedictio-solemn-christus.gtex}

\vspace{7mm}

\lectioii

\noindent \Vbardot{} Tu autem, Dómine, miserére nobis.
\noindent \Rbardot{} Deo grátias.

\vfill
\pagebreak

\responsoriumii

\vfill
\pagebreak

\cuminitiali{}{temporalia/benedictio-solemn-ignem.gtex}

\vspace{7mm}

\lectioiii

\noindent \Vbardot{} Tu autem, Dómine, miserére nobis.
\noindent \Rbardot{} Deo grátias.

\vfill
\pagebreak

\responsoriumiii

\vfill
\pagebreak

\rubrica{Reliqua omittuntur, nisi Laudes separandæ sint.}

\sineinitiali{temporalia/domineexaudi.gtex}

\vfill

\oratio

\vfill

\noindent \Vbardot{} Dómine, exáudi oratiónem meam.
\Rbardot{} Et clamor meus ad te véniat.

\vfill

\noindent \Vbardot{} Benedicámus Dómino.
\noindent \Rbardot{} Deo grátias.

\vfill

\noindent \Vbardot{} Fidélium ánimæ per misericórdiam Dei requiéscant in pace.
\Rbardot{} Amen.

\vfill
\pagebreak

\hora{Ad Laudes.} %%%%%%%%%%%%%%%%%%%%%%%%%%%%%%%%%%%%%%%%%%%%%%%%%%%%%

\cantusSineNeumas

\vspace{0.5cm}
\grechangedim{interwordspacetext}{0.18 cm plus 0.15 cm minus 0.05 cm}{scalable}%
\cuminitiali{}{temporalia/deusinadiutorium-communis.gtex}
\grechangedim{interwordspacetext}{0.22 cm plus 0.15 cm minus 0.05 cm}{scalable}%

\vfill
\pagebreak

\ifx\hymnuslaudes\undefined
\ifx\laudac\undefined
\else
\pars{Hymnus} \scriptura{Ambrosius (\olddag{} 397)}

\cuminitiali{I}{temporalia/hym-SplendorPaternae-hiemalis.gtex}
\fi
\ifx\laudbd\undefined
\else
\pars{Hymnus}

\grechangedim{interwordspacetext}{0.16 cm plus 0.15 cm minus 0.05 cm}{scalable}%
\cuminitiali{IV}{temporalia/hym-AEterneLucis.gtex}
\grechangedim{interwordspacetext}{0.22 cm plus 0.15 cm minus 0.05 cm}{scalable}%
\vspace{-3mm}
\fi
\else
\hymnuslaudes
\fi

\vfill
\pagebreak

\ifx\laudb\undefined
\else
\pars{Psalmus 1.} \scriptura{Ps. 42, 5; \textbf{H95}}

\vspace{-4mm}

\antiphona{VI F}{temporalia/ant-salutarevultusmei.gtex}

\scriptura{Psalmus 42.}

\initiumpsalmi{temporalia/ps42-initium-vi-F-auto.gtex}

\input{temporalia/ps42-vi-F.tex} \Abardot{}

\vfill
\pagebreak

\pars{Psalmus 2.} \scriptura{Is. 38, 20; \textbf{H95}}

\vspace{-7mm}

\antiphona{E}{temporalia/ant-cunctisdiebus.gtex}

\vspace{-4mm}

\scriptura{Canticum Ezechiæ, Is. 38, 10-20}

\vspace{-3mm}

\initiumpsalmi{temporalia/ezechiae-initium-e-auto.gtex}

\input{temporalia/ezechiae-e.tex} \Abardot{}

\vfill
\pagebreak

\pars{Psalmus 3.} \scriptura{Ps. 64, 2; \textbf{H96}}

\vspace{-4mm}

\antiphona{VIII a}{temporalia/ant-tedecet.gtex}

\vspace{-2mm}

\scriptura{Psalmus 64.}

\vspace{-2mm}

\initiumpsalmi{temporalia/ps64-initium-viii-A-auto.gtex}

\input{temporalia/ps64-viii-A.tex} \Abardot{}

\vfill
\pagebreak
\fi
\ifx\laudc\undefined
\else
\pars{Psalmus 1.} \scriptura{Ps. 83, 5}

\vspace{-4mm}

\antiphona{VIII G}{temporalia/ant-beatiquihabitant.gtex}

\vspace{-2mm}

\scriptura{Psalmus 84.}

\vspace{-2mm}

\initiumpsalmi{temporalia/ps84-initium-viii-G-auto.gtex}

\input{temporalia/ps84-viii-G.tex} \Abardot{}

\vfill
\pagebreak

\pars{Psalmus 2.}

\vspace{-4mm}

\antiphona{VII d}{temporalia/ant-denoctespiritusmeus.gtex}

\vspace{-2mm}

\scriptura{Canticum Isaiæ, Is. 26, 1-12}

\vspace{-2mm}

\initiumpsalmi{temporalia/isaiae3-initium-vii-d.gtex}

\input{temporalia/isaiae3-vii-d.tex} \Abardot{}

\vfill
\pagebreak

\pars{Psalmus 3.} \scriptura{Ps. 66, 2}

\vspace{-4mm}

\antiphona{E}{temporalia/ant-illuminadomine.gtex}

%\vspace{-2mm}

\scriptura{Psalmus 66.}

%\vspace{-2mm}

\initiumpsalmi{temporalia/ps66-initium-e.gtex}

\input{temporalia/ps66-e.tex} \Abardot{}

\vfill
\pagebreak
\fi

\ifx\lectiobrevis\undefined
\ifx\laudb\undefined
\else
\pars{Lectio Brevis.} \scriptura{1 Th. 5, 4-5}

\noindent Vos, fratres, non estis in ténebris, ut vos dies ille tamquam fur comprehéndat; omnes enim vos fílii lucis estis et fílii diéi. Non sumus noctis neque tenebrárum.
\fi
\ifx\laudc\undefined
\else
\pars{Lectio Brevis.} \scriptura{1 Io. 4, 14-15}

\noindent Nos vídimus et testificámur quóniam Pater misit Fílium salvatórem mundi. Quisque conféssus fúerit: Iesus est Fílius Dei, Deus in ipso manet, et ipse in Deo.
\fi
\else
\lectiobrevis
\fi

\vfill

\ifx\responsoriumbreve\undefined
\ifx\laudac\undefined
\else
\pars{Responsorium breve.}

\cuminitiali{VI}{temporalia/resp-benedictusdominus.gtex}
\fi
\ifx\laudbd\undefined
\else
\pars{Responsorium breve.} \scriptura{Ps. 118, 149.147}

\cuminitiali{VI}{temporalia/resp-vocemmeamaudi.gtex}
\fi
\else
\responsoriumbreve
\fi

\vfill
\pagebreak

\ifx\benedictus\undefined
\ifx\laudbd\undefined
\else
\pars{Canticum Zachariæ.} \scriptura{Lc. 1, 71; \textbf{H423}}

\vspace{-5mm}

{
\grechangedim{interwordspacetext}{0.18 cm plus 0.15 cm minus 0.05 cm}{scalable}%
\antiphona{I g\textsuperscript{5}}{temporalia/ant-demanuomnium.gtex}
\grechangedim{interwordspacetext}{0.22 cm plus 0.15 cm minus 0.05 cm}{scalable}%
}

%\vspace{-3mm}

\scriptura{Lc. 1, 68-79}

%\vspace{-1mm}

\initiumpsalmi{temporalia/benedictus-initium-i-g5-auto.gtex}

\input{temporalia/benedictus-i-g5.tex} \Abardot{}
\fi
\else
\benedictus
\fi

\vspace{-1cm}

\vfill
\pagebreak

\pars{Preces.}

\sineinitiali{}{temporalia/tonusprecum.gtex}

\ifx\preces\undefined
\ifx\laudb\undefined
\else
\noindent Salvatóri nostro benedicámus, qui sua resurrectióne mundum clarificávit, \gredagger{} et humíliter invocémus eum dicéntes:

\Rbardot{} Salva nos, Dómine, in sémita tua.

\noindent Resurrectiónem tuam, Dómine Iesu, oratióne cólimus matutína, \gredagger{} spes glóriæ tuæ diem nostrum illúminet.

\Rbardot{} Salva nos, Dómine, in sémita tua.

\noindent Súscipe, Dómine, vota et propósita nostra, \gredagger{} tamquam diéi nostri primítias.

\Rbardot{} Salva nos, Dómine, in sémita tua.

\noindent Tríbue in dilectióne tua nos hódie profícere, \gredagger{} ut ómnia in nostrum omniúmque bonum cooperéntur.

\Rbardot{} Salva nos, Dómine, in sémita tua.

\noindent Da, Dómine, sic lucére lucem nostram coram homínibus, \gredagger{} ut vídeant ópera nostra bona et Patrem gloríficent.

\Rbardot{} Salva nos, Dómine, in sémita tua.
\fi
\else
\preces
\fi

\vfill

\pars{Oratio Dominica.}

\cuminitiali{}{temporalia/oratiodominicaalt.gtex}

\vfill
\pagebreak

\rubrica{vel:}

\pars{Supplicatio Litaniæ.}

\cuminitiali{}{temporalia/supplicatiolitaniae.gtex}

\vfill

\pars{Oratio Dominica.}

\cuminitiali{}{temporalia/oratiodominica.gtex}

\vfill
\pagebreak

% Oratio. %%%
\oratio

\vspace{-1mm}

\vfill

\rubrica{Hebdomadarius dicit Dominus vobiscum, vel, absente sacerdote vel diacono, sic concluditur:}

\vspace{2mm}

\antiphona{C}{temporalia/dominusnosbenedicat.gtex}

\rubrica{Postea cantatur a cantore:}

\vspace{2mm}

\cuminitiali{IV}{temporalia/benedicamus-feria-laudes.gtex}

\vspace{1mm}

\vfill
\pagebreak

\end{document}

