\newcommand{\titulus}{\nomenFesti{S. Galli.}
\dies{Die 16. Octobris.}}
\newcommand{\oratio}{\pars{Oratio.}

\noindent Deus, qui nos beáti Galli Confessóris tui ánnua memória lætíficas: concéde propítus; ut cujus natalítia cólimus, étiam actiónes imitémur.

\pars{Pro pace in universo mundo..} \scriptura{Sir. 50, 25; 2 Esdr. 4, 20; \textbf{H416}}

\vspace{-4mm}

\antiphona{II D}{temporalia/ant-dapacemdomine.gtex}

\vfill

\noindent Deus, a quo sancta desidéria, recta consília et iusta sunt ópera: da servis tuis illam, quam mundus dare non potest, pacem; ut et corda nostra mandátis tuis dédita, et hóstium subláta formídine, témpora sint tua protectióne tranquílla.

\noindent Per Dóminum nostrum Iesum Christum, Fílium tuum, qui tecum vivit et regnat in unitáte Spíritus Sancti, Deus, per ómnia sǽcula sæculórum.

\noindent \Rbardot{} Amen.}
\newcommand{\invitatorium}{\pars{Invitatorium.} \scriptura{\textbf{H320}; Ps. 94 (Textus antiquus latinus); \textbf{H443}}

\vspace{-6mm}

\antiphona{III}{temporalia/inv-regemconfessorum.gtex}}
\newcommand{\hymnusmatutinum}{\pars{Sequentia.} \scriptura{beatus Notker Balbulus (\olddag{} 912); \textbf{Sg. 381, pag. 455}}

{
\grechangedim{interwordspacetext}{0.10 cm plus 0.15 cm minus 0.05 cm}{scalable}%
\antiphona{}{temporalia/seq-DilecteDeo.gtex}
\grechangedim{interwordspacetext}{0.22 cm plus 0.15 cm minus 0.05 cm}{scalable}%
}}
\newcommand{\matutinum}{\pars{Psalmus 1.} \scriptura{Vita S. Galli I, 1; \textbf{H321}}

\vspace{-5mm}

\antiphona{I D\textsuperscript{*}}{temporalia/ant-parentesverobeatigalli.gtex}

%\vspace{-2mm}

\scriptura{Ps. 20}

%\vspace{-2mm}

\initiumpsalmi{temporalia/ps20-initium-i-D_-auto.gtex}

\input{temporalia/ps20-i-D_.tex}

\vfill

\antiphona{}{temporalia/ant-parentesverobeatigalli.gtex}

\vfill
\pagebreak

\pars{Psalmus 2.} \scriptura{Vita S. Galli I, 1; \textbf{H321}}

\vspace{-4mm}

\antiphona{II D}{temporalia/ant-cumquebonaeindolis.gtex}

%\vspace{-2mm}

\scriptura{Ps. 91, 2-9}

%\vspace{-2mm}

\initiumpsalmi{temporalia/ps91i-initium-ii-D-auto.gtex}

\input{temporalia/ps91i-ii-D.tex} \Abardot{}

\vfill
\pagebreak

\pars{Psalmus 3.} \scriptura{Vita S. Galli IX, 1; \textbf{H321}}

\vspace{-5mm}

\antiphona{VII a}{temporalia/ant-cumproficiscenditempus.gtex}

%\vspace{-2mm}

\scriptura{Ps. 91, 10-16}

%\vspace{-2mm}

\initiumpsalmi{temporalia/ps91ii-initium-vii-a-auto.gtex}

\input{temporalia/ps91ii-vii-a.tex} \Abardot{}

\vfill
\pagebreak}
\newcommand{\matversus}{\noindent \Vbardot{} Iustum dedúxit Dóminus per vias rectas.

\noindent \Rbardot{} Et osténdit illi regnum Dei.}
\newcommand{\lectioi}{\pars{Lectio I.} \scriptura{Phil. 3, 7 — 4, 1.4-9}

\noindent De Epístola beáti Pauli apóstoli ad Philippénses.

\noindent Fratres: Quæ míhi erant lucra, hæc arbitrátus sum propter Christum detriméntum. Verúmtamen exístimo ómnia detriméntum esse propter eminéntiam sciéntiæ Christi Iesu Dómini mei, propter quem ómnia detriméntum feci et árbitror ut stércora, ut Christum lucri fáciam et invéniar in illo non habens meam iustítiam, quæ ex lege est, sed illam, quæ per fidem est Christi, quæ ex Deo est iustítia in fide, ad cognoscéndum illum et virtútem resurrectiónis eius et communiónem passiónum illíus, confórmans me morti eius, si quo modo occúrram ad resurrectiónem, quæ est ex mórtuis. Non quod iam accéperim aut iam perféctus sim, pérsequor autem si umquam comprehéndam, sicut et comprehénsus sum a Christo Iesu.

\noindent Fratres, ego me non árbitror comprehendísse; unum autem, quæ quidem retro sunt, oblivíscens, ad ea vero, quæ ante sunt, exténdens me ad destinátum pérsequor, ad bravíum supérnæ vocatiónis Dei in Christo Iesu.

\noindent {\color{gray} Quicúmque ergo perfécti, hoc sentiámus; et si quid áliter sápitis, et hoc vobis Deus revelábit; verúmtamen, ad quod pervénimus, in eódem ambulémus.

\noindent Coimitatóres mei estóte, fratres, et observáte eos, qui ita ámbulant, sicut habétis formam nos. Multi enim ámbulant, quos sæpe dicébam vobis, nunc autem et flens dico, inimícos crucis Christi, quorum finis intéritus, quorum deus venter et glória in confusióne ipsórum, qui terréna sápiunt. Noster enim municipátus in cælis est, unde étiam salvatórem exspectámus Dóminum Iesum Christum, qui transfigurábit corpus humilitátis nostræ confórme fíeri córpori glóriæ suæ secúndum operatiónem, qua possit étiam subícere sibi ómnia.

\noindent Itaque, fratres mei caríssimi et desideratíssimi, gáudium et coróna mea, sic state in Dómino, caríssimi!}

\noindent Gaudéte in Dómino semper. Iterum dico: Gaudéte! Modéstia vestra nota sit ómnibus homínibus. Dóminus prope. Nihil sollíciti sitis, sed in ómnibus oratióne et obsecratióne cum gratiárum actióne petitiónes vestræ innotéscant apud Deum. Et pax Dei, quæ exsúperat omnem sensum, custódiet corda vestra et intellegéntias vestras in Christo Iesu.

\noindent De cétero, fratres, quæcúmque sunt vera, quæcúmque pudíca, quæcúmque iusta, quæcúmque casta, quæcúmque amabília, quæcúmque bonæ famæ, si qua virtus et si qua laus, hæc cogitáte; quæ et didicístis et accepístis et audístis et vidístis in me, hæc ágite; et Deus pacis erit vobíscum.}
\newcommand{\responsoriumi}{\pars{Responsorium 1.} \scriptura{Vita S. Galli I, 1; \textbf{H321}}

\vspace{-5mm}

\responsorium{II}{temporalia/resp-parentesverobeatigalli.gtex}{}}
\newcommand{\lectioii}{\pars{Lectio II.} \scriptura{Vita S. Galli XXIX, 1-2}

\noindent Cum iam bonórum ómnium Auctor et Propagátor, athlétam suum de mundi agóne sublátum, præmiórum láureis vellet perénnibus adornáre, Willimárus présbyter véniens ad cellam viri sancti, rogávit eum ut secum egrederétur ad castrum; et ut obtinéret quod vóluit, huiúsmodi voce flébili quærimóniam submíssus explícuit: Cur, ínquiens, o Pater, me, qui tuórum mónitis doctórum innítor, quasi despéctum dereliquísti, et doctrínæ tuæ salutáribus institútis auditórem fraudásti benévolum? Cui hanc obiectiónem ascríbere possim, nisi peccatórum meórum fetóribus? Nisi enim vita mea tuo displicéret iudício, amábili me ædificatiónis tuæ non priváres solátio. Nunc ergo noli nos pro peccátis nostris abícere, sed Dómini provocátus mandátis, viam veritátis desiderántibus áperi, et sólitæ nobis benignitátis munus impénde.}
\newcommand{\responsoriumii}{\pars{Responsorium 2.} \scriptura{Vita S. Galli IV, 3; \textbf{H322}}

\vspace{-5mm}

\responsorium{VII}{temporalia/resp-beatusgalluszelopietatis.gtex}{}}
\newcommand{\lectioiii}{\pars{Lectio III.} \scriptura{Vita S. Galli XXIX, 3-4}

\noindent Motus ígitur hoc supplicántis allóquio pietátis amátor, descéndit cum illo, et venérunt ad castrum. Vocáta autem multitúdine in die solémni, vir sanctus prædicatiónis dulcédine avidórum corda refécit, et tanta quæ díxerat sapiéntiæ luce vestívit, ut summa ómnium gratulatióne audítus, et plena cunctórum veneratióne sit honorátus. Bíduo ítaque íbidem ducto, tértia die febre corréptus, tantum in brevi eius violéntia depréssus est, ut nec ad cellam redíre nec cibi sustentáculum potuísset percípere. Cumque hac infirmitáte per dies quatuórdecim laborásset, die sexto décimo mensis octóbris, id est XVII (ante diem séptimum décimum) Kaléndas novémbris, explétis nonagínta quinque annis suæ ætátis, in senectúte bona, huius vitæ liberátus ergástulo, ánimam méritis plenam felícibus réddidit bonis inhæsúram perénnibus.}
\newcommand{\responsoriumiii}{\pars{Responsorium 3.} \scriptura{Vita S. Galli VI, 3; \Vbar{} ibid. VI, 4; \textbf{H322}}

\vspace{-5mm}

{
\grechangedim{interwordspacetext}{0.18 cm plus 0.15 cm minus 0.05 cm}{scalable}%
\responsorium{VIII}{temporalia/resp-columbanusitaquebeatogallo-cumdox.gtex}{}
\grechangedim{interwordspacetext}{0.22 cm plus 0.15 cm minus 0.05 cm}{scalable}%
}}
\newcommand{\hymnuslaudes}{\pars{Hymnus.}

\vspace{-5mm}

\antiphona{I}{temporalia/hym-IsteConfessor-alt.gtex}}
\newcommand{\laudes}{\pars{Psalmus 1.} \scriptura{Vita S. Galli XXXII, 1.2; \textbf{H325}}

\vspace{-5mm}

\antiphona{VI F}{temporalia/ant-habuitvirdei.gtex}

%\vspace{-2mm}

\scriptura{Psalmus 62}

%\vspace{-2mm}

\initiumpsalmi{temporalia/ps62-initium-vi-F-auto.gtex}

%\vspace{-1.5mm}

\input{temporalia/ps62-vi-F.tex} \Abardot{}

\vfill
\pagebreak

\pars{Psalmus 2.} \scriptura{Vita S. Galli XXXII, 1; \textbf{H325}}

\vspace{-5mm}

\antiphona{VII a}{temporalia/ant-huiusipseclavem.gtex}

%\vspace{-2mm}

\scriptura{Canticum trium puerorum, Dan. 3, 57-88 et 56}

\initiumpsalmi{temporalia/dan3-initium-vii-a-auto.gtex}

\input{temporalia/dan3-vii-a-sinedox.tex}

\rubrica{Hic non dicitur Gloria Patri, neque Amen.}

\vfill

\antiphona{}{temporalia/ant-huiusipseclavem.gtex}

\vfill
\pagebreak

\pars{Psalmus 3.} \scriptura{Vita S. Galli XXXII, 2; \textbf{H326}}

\vspace{-5mm}

\antiphona{I g}{temporalia/ant-dehacverovita.gtex}

%\vspace{-2mm}

\scriptura{Psalmus 149}

%\vspace{-2mm}

\initiumpsalmi{temporalia/ps149-initium-i-g-auto.gtex}

\input{temporalia/ps149-i-g.tex} \Abardot{}

\vfill
\pagebreak}
\newcommand{\lectiobrevis}{\pars{Lectio Brevis.} \scriptura{Zach. 3, 7}

\noindent Hæc dicit Dóminus: si in viis meis ambuláveris, et custódiam meam custodíeris, tu quoque iudicábis domum meam, et custódies átria mea.}
\newcommand{\responsoriumbreve}{\pars{Responsorium breve.} \scriptura{Sap. 10, 10}

\cuminitiali{VI}{temporalia/resp-justumdeduxit.gtex}}
\newcommand{\preces}{\noindent Christo, bono pastóri,~\gredagger{} qui pro suis óvibus ánimam pósuit,~\grestar{} laudes grati exsolvámus et supplicémus, dicéntes:

\Rbardot{} Pasce pópulum tuum, Dómine.

\noindent Christe, qui in sanctis pastóribus misericórdiam et dilectiónem tuam dignátus es osténdere,~\grestar{} numquam désinas per eos nobíscum misericórditer ágere.

\Rbardot{} Pasce pópulum tuum, Dómine.

\noindent Qui múnere pastóris animárum fungi per tuos vicários pergis,~\grestar{} ne destíteris nos ipse per rectóres nostros dirígere.

\Rbardot{} Pasce pópulum tuum, Dómine.

\noindent Qui in sanctis tuis, populórum dúcibus, córporum animarúmque médicus exstitísti,~\grestar{} numquam cesses ministérium in nos vitæ et sanctitátis perágere.

\Rbardot{} Pasce pópulum tuum, Dómine.

\noindent Qui, prudéntia et caritáte sanctórum, tuum gregem erudísti,~\grestar{} nos in sanctitáte iúgiter per pastóres nostros ædífica.

\Rbardot{} Pasce pópulum tuum, Dómine.}
\newcommand{\benedictus}{\pars{Canticum Zachariæ.} \scriptura{Cf. Vita S. Galli XXX, 6; ibid. XXXIII, 1; ibid. XVII, 5; \textbf{H326}}

\vspace{-4mm}

\antiphona{VIII G}{temporalia/ant-superpositoequis.gtex}

%\vspace{-3mm}

\scriptura{Lc. 1, 68-79}

\vspace{-2mm}

\cantusSineNeumas
\initiumpsalmi{temporalia/benedictus-initium-viii-G-auto.gtex}

%\vspace{-1.5mm}

\input{temporalia/benedictus-viii-G.tex}

\antiphona{}{temporalia/ant-superpositoequis.gtex}}
\newcommand{\benedicamuslaudes}{\cuminitiali{}{temporalia/benedicamus-memoria-laudes.gtex}}
\include{hebdomadaxxviii}
% LuaLaTeX

\documentclass[a4paper, twoside, 12pt]{article}
\usepackage[latin]{babel}
%\usepackage[landscape, left=3cm, right=1.5cm, top=2cm, bottom=1cm]{geometry} % okraje stranky
%\usepackage[landscape, a4paper, mag=1166, truedimen, left=2cm, right=1.5cm, top=1.6cm, bottom=0.95cm]{geometry} % okraje stranky
\usepackage[landscape, a4paper, mag=1400, truedimen, left=0.5cm, right=0.5cm, top=0.5cm, bottom=0.5cm]{geometry} % okraje stranky

\usepackage{fontspec}
\setmainfont[FeatureFile={junicode.fea}, Ligatures={Common, TeX}, RawFeature=+fixi]{Junicode}
%\setmainfont{Junicode}

% shortcut for Junicode without ligatures (for the Czech texts)
\newfontfamily\nlfont[FeatureFile={junicode.fea}, Ligatures={Common, TeX}, RawFeature=+fixi]{Junicode}

\usepackage{multicol}
\usepackage{color}
\usepackage{lettrine}
\usepackage{fancyhdr}

% usual packages loading:
\usepackage{luatextra}
\usepackage{graphicx} % support the \includegraphics command and options
\usepackage{gregoriotex} % for gregorio score inclusion
\usepackage{gregoriosyms}
\usepackage{wrapfig} % figures wrapped by the text
\usepackage{parcolumns}
\usepackage[contents={},opacity=1,scale=1,color=black]{background}
\usepackage{tikzpagenodes}
\usepackage{calc}
\usepackage{longtable}
\usetikzlibrary{calc}

\setlength{\headheight}{14.5pt}

\input{conventuscommune.tex} % Often used macros

\newcommand{\annusEditionis}{2021}

%%%% Vicekrat opakovane kousky

\newcommand{\anteOrationem}{
  \rubrica{Ante Orationem, cantatur a Superiore:}

  \pars{Supplicatio Litaniæ.}

  \cuminitiali{}{temporalia/supplicatiolitaniae.gtex}

  \pars{Oratio Dominica.}

  \cuminitiali{}{temporalia/oratiodominica.gtex}

  \rubrica{Deinde dicitur ab Hebdomadario:}

  \cuminitiali{}{temporalia/dominusvobiscum-solemnis.gtex}

  \rubrica{In choro monialium loco Dominus vobiscum dicitur:}

  \sineinitiali{temporalia/domineexaudi.gtex}
}

\setlength{\columnsep}{30pt} % prostor mezi sloupci

%%%%%%%%%%%%%%%%%%%%%%%%%%%%%%%%%%%%%%%%%%%%%%%%%%%%%%%%%%%%%%%%%%%%%%%%%%%%%%%%%%%%%%%%%%%%%%%%%%%%%%%%%%%%%
\begin{document}

% Here we set the space around the initial.
% Please report to http://home.gna.org/gregorio/gregoriotex/details for more details and options
\grechangedim{afterinitialshift}{2.2mm}{scalable}
\grechangedim{beforeinitialshift}{2.2mm}{scalable}
\grechangedim{interwordspacetext}{0.22 cm plus 0.15 cm minus 0.05 cm}{scalable}%
\grechangedim{annotationraise}{-0.2cm}{scalable}

% Here we set the initial font. Change 38 if you want a bigger initial.
% Emit the initials in red.
\grechangestyle{initial}{\color{red}\fontsize{38}{38}\selectfont}

\pagestyle{empty}

%%%% Titulni stranka
\begin{titulusOfficii}
\ifx\titulus\undefined
\nomenFesti{Feria V \hebdomada{}}
\else
\titulus
\fi
\end{titulusOfficii}

\vfill

\begin{center}
%Ad usum et secundum consuetudines chori \guillemotright{}Conventus Choralis\guillemotleft.

%Editio Sancti Wolfgangi \annusEditionis
\end{center}

\scriptura{}

\pars{}

\pagebreak

\renewcommand{\headrulewidth}{0pt} % no horiz. rule at the header
\fancyhf{}
\pagestyle{fancy}

\cantusSineNeumas

\ifx\oratio\undefined
\ifx\lauda\undefined
\else
\newcommand{\oratio}{\pars{Oratio.}

\noindent Omnípotens sempitérne Deus, véspere, mane et merídie maiestátem tuam supplíciter deprecámur, ut, expúlsis de córdibus nostris peccatórum ténebris, ad veram lucem, quæ Christus est, nos fácias perveníre.

\noindent Qui tecum vivit et regnat in unitáte Spíritus Sancti, Deus, per ómnia sǽcula sæculórum.

\noindent \Rbardot{} Amen.}
\fi
\fi

\hora{Ad Matutinum.} %%%%%%%%%%%%%%%%%%%%%%%%%%%%%%%%%%%%%%%%%%%%%%%%%%%%%
%\sideThumbs{Matutinum}

\vspace{2mm}

\cuminitiali{}{temporalia/dominelabiamea.gtex}

\vfill
%\pagebreak

\vspace{2mm}

\ifx\invitatorium\undefined
\pars{Invitatorium.} \scriptura{Ps. 94, 6; Psalmus 94; \textbf{H136}}

\vspace{-6mm}

\antiphona{E}{temporalia/inv-adoremusdominum.gtex}
\else
\invitatorium
\fi

\vfill
\pagebreak

\ifx\hymnusmatutinum\undefined
\ifx\matuac\undefined
\else
\pars{Hymnus.} \scriptura{Gregorius Magnus (+604)}

{
\grechangedim{interwordspacetext}{0.10 cm plus 0.15 cm minus 0.05 cm}{scalable}%
\antiphona{IV}{temporalia/hym-NoxAtra.gtex}
\grechangedim{interwordspacetext}{0.22 cm plus 0.15 cm minus 0.05 cm}{scalable}%
}
\fi
\else
\hymnusmatutinum
\fi

\vspace{-3mm}

\vfill
\pagebreak

\ifx\matua\undefined
\else
% MAT A
\pars{Psalmus 1.} \scriptura{Ps. 17, 3; \textbf{H99}}

\vspace{-4mm}

\antiphona{VIII G}{temporalia/ant-dominusfirmamentum.gtex}

%\vspace{-2mm}

\scriptura{Ps. 17, 31-35}

%\vspace{-2mm}

\initiumpsalmi{temporalia/ps17xxxi_xxxv-initium-viii-G-auto.gtex}

\input{temporalia/ps17xxxi_xxxv-viii-G.tex} \Abardot{}

\vfill
\pagebreak

\pars{Psalmus 2.} \scriptura{Ps. 62, 9; \textbf{H393}}

\vspace{-4mm}

\antiphona{VII c trans.}{temporalia/ant-mesuscepit.gtex}

%\vspace{-2mm}

\scriptura{Ps. 17, 36-46}

%\vspace{-2mm}

\initiumpsalmi{temporalia/ps17xxxvi_xlvi-initium-vii-c-trans.gtex}

\input{temporalia/ps17xxxvi_xlvi-vii-c.tex} \Abardot{}

\vfill
\pagebreak

\pars{Psalmus 3.} \scriptura{Ps. 17, 47; \textbf{H100}}

\vspace{-4mm}

\antiphona{VII c\textsuperscript{2}}{temporalia/ant-vivitdominus.gtex}

%\vspace{-2mm}

\scriptura{Ps. 17, 47-51}

%\vspace{-2mm}

\initiumpsalmi{temporalia/ps17xlvii_li-initium-vii-c2-auto.gtex}

\input{temporalia/ps17xlvii_li-vii-c2.tex} \Abardot{}

\vfill
\pagebreak
\fi
\ifx\matuc\undefined
\else
% MAT C
\pars{Psalmus 1.} \scriptura{Lam. 1, 21; \textbf{H177}}

\vspace{-4mm}

\antiphona{VII a}{temporalia/ant-omnesinimici.gtex}

%\vspace{-2mm}

\scriptura{Ps. 88, 39-46}

%\vspace{-2mm}

\initiumpsalmi{temporalia/ps88xxxix_xlvi-initium-vii-a-auto.gtex}

\input{temporalia/ps88xxxix_xlvi-vii-a.tex} \Abardot{}

\vfill
\pagebreak

\pars{Psalmus 2.} \scriptura{Ps. 88, 53; \textbf{H98}}

\vspace{-4mm}

\antiphona{VI F}{temporalia/ant-benedictusdominusinaeternum.gtex}

%\vspace{-2mm}

\scriptura{Ps. 88, 47-53}

%\vspace{-2mm}

\initiumpsalmi{temporalia/ps88xlvii_liii-initium-vi-F-auto.gtex}

\input{temporalia/ps88xlvii_liii-vi-F.tex} \Abardot{}

\vfill
\pagebreak

\pars{Psalmus 3.} \scriptura{Ps. 89, 13}

\vspace{-4mm}

\antiphona{I g}{temporalia/ant-converteredomine.gtex}

%\vspace{-2mm}

\scriptura{Ps. 89}

%\vspace{-2mm}

\initiumpsalmi{temporalia/ps89-initium-i-g-auto.gtex}

\input{temporalia/ps89-i-g.tex}

\vfill

\antiphona{}{temporalia/ant-converteredomine.gtex}

\vfill
\pagebreak
\fi

\pars{Versus.}

\ifx\matversus\undefined
\ifx\matua\undefined
\else
\noindent \Vbardot{} Révela, Dómine, óculos meos.

\noindent \Rbardot{} Et considerábo mirabília de lege tua.
\fi
\ifx\matuc\undefined
\else
\noindent \Vbardot{} Audies de ore meo verbum.

\noindent \Rbardot{} Et annuntiábis eis ex me.
\fi
\else
\matversus
\fi

\vspace{5mm}

\sineinitiali{temporalia/oratiodominica-mat.gtex}

\vspace{5mm}

\pars{Absolutio.}

\cuminitiali{}{temporalia/absolutio-exaudi.gtex}

\vfill
\pagebreak

\cuminitiali{}{temporalia/benedictio-solemn-benedictione.gtex}

\vspace{7mm}

\lectioi

\noindent \Vbardot{} Tu autem, Dómine, miserére nobis.
\noindent \Rbardot{} Deo grátias.

\vfill
\pagebreak

\responsoriumi

\vfill
\pagebreak

\cuminitiali{}{temporalia/benedictio-solemn-unigenitus.gtex}

\vspace{7mm}

\lectioii

\noindent \Vbardot{} Tu autem, Dómine, miserére nobis.
\noindent \Rbardot{} Deo grátias.

\vfill
\pagebreak

\responsoriumii

\vfill
\pagebreak

\cuminitiali{}{temporalia/benedictio-solemn-spiritus.gtex}

\vspace{7mm}

\lectioiii

\noindent \Vbardot{} Tu autem, Dómine, miserére nobis.
\noindent \Rbardot{} Deo grátias.

\vfill
\pagebreak

\responsoriumiii

\vfill
\pagebreak

\rubrica{Reliqua omittuntur, nisi Laudes separandæ sint.}

\sineinitiali{temporalia/domineexaudi.gtex}

\vfill

\oratio

\vfill

\noindent \Vbardot{} Dómine, exáudi oratiónem meam.
\Rbardot{} Et clamor meus ad te véniat.

\vfill

\noindent \Vbardot{} Benedicámus Dómino.
\noindent \Rbardot{} Deo grátias.

\vfill

\noindent \Vbardot{} Fidélium ánimæ per misericórdiam Dei requiéscant in pace.
\Rbardot{} Amen.

\vfill
\pagebreak

\hora{Ad Laudes.} %%%%%%%%%%%%%%%%%%%%%%%%%%%%%%%%%%%%%%%%%%%%%%%%%%%%%
%\sideThumbs{Laudes}

\cantusSineNeumas

\vspace{0.5cm}
\grechangedim{interwordspacetext}{0.18 cm plus 0.15 cm minus 0.05 cm}{scalable}%
\cuminitiali{}{temporalia/deusinadiutorium-communis.gtex}
\grechangedim{interwordspacetext}{0.22 cm plus 0.15 cm minus 0.05 cm}{scalable}%

\vfill
\pagebreak

\ifx\hymnuslaudes\undefined
\ifx\laudac\undefined
\else
\pars{Hymnus}

\grechangedim{interwordspacetext}{0.16 cm plus 0.15 cm minus 0.05 cm}{scalable}%
\cuminitiali{I}{temporalia/hym-SolEcce.gtex}
\grechangedim{interwordspacetext}{0.22 cm plus 0.15 cm minus 0.05 cm}{scalable}%
\vspace{-3mm}
\fi
\else
\hymnuslaudes
\fi

\vfill
\pagebreak

\ifx\lauda\undefined
\else
\pars{Psalmus 1.}

\vspace{-4mm}

\antiphona{VIII G}{temporalia/ant-exsurgamdiluculo.gtex}

%\vspace{-2mm}

\scriptura{Psalmus 56}

%\vspace{-2mm}

\initiumpsalmi{temporalia/ps56-initium-viii-g-auto.gtex}

%\vspace{-1.5mm}

\input{temporalia/ps56-viii-g.tex} \Abardot{}

\vfill
\pagebreak

\pars{Psalmus 2.} \scriptura{Ier. 31, 14}

\vspace{-4mm}

\antiphona{IV* e}{temporalia/ant-populusmeusait.gtex}

%\vspace{-2mm}

\scriptura{Canticum Ieremiæ, 1 Ier. 31, 10-14}

%\vspace{-3mm}

\initiumpsalmi{temporalia/jeremiae3-initium-iv_-e-auto.gtex}

\input{temporalia/jeremiae3-iv_-e.tex} \Abardot{}

\vfill
\pagebreak

\pars{Psalmus 3.} \scriptura{Ps. 95, 4; \textbf{H94}}

\vspace{-4mm}

\antiphona{IV a}{temporalia/ant-magnusdominus.gtex}

\scriptura{Psalmus 47}

\initiumpsalmi{temporalia/ps47-initium-iv-a-auto.gtex}

\input{temporalia/ps47-iv-a.tex} \Abardot{}

\vfill
\pagebreak
\fi
\ifx\laudc\undefined
\else
\pars{Psalmus 1.} \scriptura{Ps. 86, 1; \textbf{H98}}

\vspace{-4mm}

\antiphona{I g}{temporalia/ant-fundamentaeius.gtex}

%\vspace{-2mm}

\scriptura{Psalmus 86}

%\vspace{-2mm}

\initiumpsalmi{temporalia/ps86-initium-i-g-auto.gtex}

%\vspace{-1.5mm}

\input{temporalia/ps86-i-g.tex} \Abardot{}

\vfill
\pagebreak

\pars{Psalmus 2.}

\vspace{-4mm}

\antiphona{II D}{temporalia/ant-eccedominusnosterbrachio.gtex}

%\vspace{-2mm}

\scriptura{Canticum Isaiæ, Is. 40, 10-17}

%\vspace{-3mm}

\initiumpsalmi{temporalia/isaiae9-initium-ii-D-auto.gtex}

\input{temporalia/isaiae9-ii-D.tex} \Abardot{}

\vfill
\pagebreak

\pars{Psalmus 3.} \scriptura{Ps. 144, 17}

\vspace{-4mm}

\antiphona{E}{temporalia/ant-iustusetsanctus.gtex}

\scriptura{Psalmus 98}

\initiumpsalmi{temporalia/ps98-initium-e.gtex}

\input{temporalia/ps98-e.tex} \Abardot{}

\vfill
\pagebreak
\fi

\ifx\lectiobrevis\undefined
\ifx\lauda\undefined
\else
\pars{Lectio Brevis.} \scriptura{Is. 66, 1-2}

\noindent Hæc dicit Dóminus: Cælum thronus meus, terra autem scabéllum pedum meórum. Quæ ista domus, quam ædificábitis mihi, et quis iste locus quiétis meæ? Omnia hæc manus mea fecit et mea sunt univérsa ista, dicit Dóminus. Ad hunc autem respíciam, ad paupérculum et contrítum spíritu et treméntem sermónes meos.
\fi
\else
\lectiobrevis
\fi

\vfill

\ifx\responsoriumbreve\undefined
\ifx\laudac\undefined
\else
\pars{Responsorium breve.} \scriptura{Ps. 118, 145}

\cuminitiali{VI}{temporalia/resp-clamaviintotocorde.gtex}
\fi
\else
\responsoriumbreve
\fi

\vfill
\pagebreak

\ifx\benedictus\undefined
\ifx\laudac\undefined
\else
\pars{Canticum Zachariæ.} \scriptura{Lc. 1, 74.75; \textbf{H423}}

%\vspace{-4mm}

{
\grechangedim{interwordspacetext}{0.18 cm plus 0.15 cm minus 0.05 cm}{scalable}%
\antiphona{VII a}{temporalia/ant-insanctitate.gtex}
\grechangedim{interwordspacetext}{0.22 cm plus 0.15 cm minus 0.05 cm}{scalable}%
}

%\vspace{-3mm}

\scriptura{Lc. 1, 68-79}

%\vspace{-2mm}

\cantusSineNeumas
\initiumpsalmi{temporalia/benedictus-initium-vii-a-auto.gtex}

%\vspace{-1.5mm}

\input{temporalia/benedictus-vii-a.tex} \Abardot{}
\fi
\else
\benedictus
\fi

\vspace{-1cm}

\vfill
\pagebreak

%\sideThumbs{{\scriptsize{}Fine horarum}}

\pars{Preces.}

\sineinitiali{}{temporalia/tonusprecum.gtex}

\ifx\preces\undefined
\ifx\lauda\undefined
\else
\noindent Grátias agámus Christo, qui lumen huius diéi nobis concédit, \gredagger{} et ad eum clamémus:

\Rbardot{} Bénedic et sanctífica nos, Dómine.

\noindent Qui te pro peccátis nostris hóstiam obtulísti, \gredagger{} incépta et propósita suscípias hodiérna.

\Rbardot{} Bénedic et sanctífica nos, Dómine.

\noindent Qui óculos nostros lucis dono lætíficas novæ, \gredagger{} lúcifer oriáris in córdibus nostris.

\Rbardot{} Bénedic et sanctífica nos, Dómine.

\noindent Tríbue hódie nos esse ómnibus longánimes, \gredagger{} ut imitatóres tui fíeri possímus.

\Rbardot{} Bénedic et sanctífica nos, Dómine.

\noindent Audítam, Dómine, fac nobis mane misericórdiam tuam. \gredagger{} Sit hódie gáudium tuum fortitúdo nostra.

\Rbardot{} Bénedic et sanctífica nos, Dómine.
\fi
\ifx\laudc\undefined
\else
\noindent Christo, bono pastóri, qui pro suis óvibus ánimam pósuit, \gredagger{} laudes grati exsolvámus et supplicémus, dicéntes:

\Rbardot{} Pasce pópulum tuum, Dómine.

\noindent Christe, qui in sanctis pastóribus misericórdiam et dilectiónem tuam dignátus es osténdere, \gredagger{} numquam désinas per eos nobíscum misericórditer ágere.

\Rbardot{} Pasce pópulum tuum, Dómine.

\noindent Qui múnere pastóris animárum fungi per tuos vicários pergis, \gredagger{} ne destíteris nos ipse per rectóres nostros dirígere.

\Rbardot{} Pasce pópulum tuum, Dómine.

\noindent Qui in sanctis tuis, populórum dúcibus, córporum animarúmque médicus exstitísti, \gredagger{} numquam cesses ministérium in nos vitæ et sanctitátis perágere.

\Rbardot{} Pasce pópulum tuum, Dómine.

\noindent Qui, prudéntia et caritáte sanctórum, tuum gregem erudísti, \gredagger{} nos in sanctitáte iúgiter per pastóres nostros ædífica.

\Rbardot{} Pasce pópulum tuum, Dómine.
\fi
\else
\preces
\fi

\vfill

\pars{Oratio Dominica.}

\cuminitiali{}{temporalia/oratiodominicaalt.gtex}

\vfill
\pagebreak

\rubrica{vel:}

\pars{Supplicatio Litaniæ.}

\cuminitiali{}{temporalia/supplicatiolitaniae.gtex}

\vfill

\pars{Oratio Dominica.}

\cuminitiali{}{temporalia/oratiodominica.gtex}

\vfill
\pagebreak

% Oratio. %%%
\oratio

\vspace{-1mm}

\vfill

\rubrica{Hebdomadarius dicit Dominus vobiscum, vel, absente sacerdote vel diacono, sic concluditur:}

\vspace{2mm}

\antiphona{C}{temporalia/dominusnosbenedicat.gtex}

\rubrica{Postea cantatur a cantore:}

\vspace{2mm}

\cuminitiali{IV}{temporalia/benedicamus-feria-laudes.gtex}

\vspace{1mm}

\vfill
\pagebreak

\end{document}

