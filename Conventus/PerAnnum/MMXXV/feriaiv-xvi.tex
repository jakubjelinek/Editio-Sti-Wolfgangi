\newcommand{\titulus}{\nomenFesti{S. Brigittæ Sueciæ, Religiosæ.}
\dies{Die 23. Iulii.}}
\newcommand{\oratio}{\pars{Oratio.}

\noindent Dómine Deus noster, qui beátæ Birgíttæ, Fílii tui passiónem meditánti, secréta cæléstia revelásti, da nobis fámulis tuis in revelatióne glóriæ tuæ gaudére lætántes.

\pars{Pro pace in universo mundo.} \scriptura{Sir. 50, 25; 2 Esdr. 4, 20; \textbf{H416}}

\vspace{-4mm}

\antiphona{II D}{temporalia/ant-dapacemdomine.gtex}

\vfill

\noindent Deus, a quo sancta desidéria, recta consília et iusta sunt ópera: da servis tuis illam, quam mundus dare non potest, pacem; ut et corda nostra mandátis tuis dédita, et hóstium subláta formídine, témpora sint tua protectióne tranquílla.

\noindent Per Dóminum nostrum Iesum Christum, Fílium tuum, qui tecum vivit et regnat in unitáte Spíritus Sancti, Deus, per ómnia sǽcula sæculórum.

\noindent \Rbardot{} Amen.}
\newcommand{\invitatorium}{\pars{Invitatorium.}

\vspace{-4mm}

\antiphona{IV}{temporalia/inv-mirabileminsanctis.gtex}}
\newcommand{\hymnusmatutinum}{\pars{Hymnus}

\cuminitiali{IV}{temporalia/hym-LaetiColentes.gtex}}
\newcommand{\matutinum}{\pars{Psalmus 1.} \scriptura{Ct. 1, 11}

\vspace{-4mm}

\antiphona{III a}{temporalia/ant-dumessetrex.gtex}

%\vspace{-2mm}

\scriptura{Ps. 18, 1-7}

%\vspace{-2mm}

\initiumpsalmi{temporalia/ps18i-initium-iii-a-auto.gtex}

\input{temporalia/ps18i-iii-a.tex} \Abardot{}

\vfill
\pagebreak

\pars{Psalmus 2.} \scriptura{Ps. 44, 8}

\vspace{-0.4cm}

\antiphona{IV e}{temporalia/ant-dilexistiiustitiam.gtex}

%\vspace{-2mm}

\scriptura{Ps. 44, 2-10}

%\vspace{-2mm}

\initiumpsalmi{temporalia/ps44i-initium-iv-e2-auto.gtex}

\input{temporalia/ps44i-iv-e2.tex} \Abardot{}

\vfill
\pagebreak

\pars{Psalmus 3.} \scriptura{Ct. 2, 6}

\vspace{-0.4cm}

\antiphona{II* b}{temporalia/ant-laevaeius.gtex}

%\vspace{-2mm}

\scriptura{Ps. 44, 11-18}

%\vspace{-2mm}

\initiumpsalmi{temporalia/ps44ii-initium-ii_-B-auto.gtex}

\input{temporalia/ps44ii-ii_-B.tex} \Abardot{}

\vfill
\pagebreak}
\newcommand{\matversus}{\noindent \Vbardot{} Dóminus virtútum nobíscum.

\noindent \Rbardot{} Suscéptor noster, Deus Iacob.}
\newcommand{\lectioi}{\pars{Lectio I.} \scriptura{Phil. 3, 7-16}

\noindent De Epístola beáti Pauli apóstoli ad Philippénses.

\noindent Fratres: Quæ mihi erant lucra, hæc arbitrátus sum propter Christum detriméntum. Verúmtamen exístimo ómnia detriméntum esse propter eminéntiam sciéntiæ Christi Iesu Dómini mei, propter quem ómnia detriméntum feci et árbitror ut stércora, ut Christum lucri fáciam et invéniar in illo non habens meam iustítiam, quæ ex lege est, sed illam, quæ per fidem est Christi, quæ ex Deo est iustítia in fide, ad cognoscéndum illum et virtútem resurrectiónis eius et communiónem passiónum illíus, confórmans me morti eius, si quo modo occúrram ad resurrectiónem, quæ est ex mórtuis. Non quod iam accéperim aut iam perféctus sim, pérsequor autem si umquam comprehéndam, sicut et comprehénsus sum a Christo Iesu.

\noindent Fratres, ego me non árbitror comprehendísse; unum autem, quæ quidem retro sunt, oblivíscens, ad ea vero, quæ ante sunt, exténdens me ad destinátum pérsequor, ad bravíum supérnæ vocatiónis Dei in Christo Iesu.

\noindent Quicúmque ergo perfécti, hoc sentiámus; et si quid áliter sápitis, et hoc vobis Deus revelábit; verúmtamen, ad quod pervénimus, in eódem ambulémus.}
\newcommand{\responsoriumi}{\pars{Responsorium 1.} \scriptura{\Rbardot{} Ps. 44, 12 \Vbardot{} ibid., 5; \textbf{H298}}

\vspace{-5mm}

\responsorium{II}{temporalia/resp-venielectamea-CROCHU.gtex}{}}
\newcommand{\lectioii}{\pars{Lectio II.} \scriptura{Phil. 3, 16-21; 4, 1.4-9}

\noindent Coimitatóres mei estóte, fratres, et observáte eos, qui ita ámbulant, sicut habétis formam nos. Multi enim ámbulant, quos sæpe dicébam vobis, nunc autem et flens dico, inimícos crucis Christi, quorum finis intéritus, quorum deus venter et glória in confusióne ipsórum, qui terréna sápiunt. Noster enim municipátus in cælis est, unde étiam salvatórem exspectámus Dóminum Iesum Christum, qui transfigurábit corpus humilitátis nostræ confórme fíeri córpori glóriæ suæ secúndum operatiónem, qua possit étiam subícere sibi ómnia.

\noindent Itaque, fratres mei caríssimi et desideratíssimi, gáudium et coróna mea, sic state in Dómino, caríssimi!

\noindent Gaudéte in Dómino semper. Iterum dico: Gaudéte! Modéstia vestra nota sit ómnibus homínibus. Dóminus prope. Nihil sollíciti sitis, sed in ómnibus oratióne et obsecratióne cum gratiárum actióne petitiónes vestræ innotéscant apud Deum. Et pax Dei, quæ exsúperat omnem sensum, custódiet corda vestra et intellegéntias vestras in Christo Iesu.}
\newcommand{\responsoriumii}{\pars{Responsorium 2.} \scriptura{\Rbardot{} Ps. 32, 1 \Vbardot{} ibid., 2; \textbf{H252} \& \textbf{E234}}

\vspace{-5mm}

\responsorium{I}{temporalia/resp-gaudeteiusti-CROCHU.gtex}{}}
\newcommand{\lectioiii}{\pars{Lectio III.} \scriptura{Oratio 2: Revelationum S. Birgittæ libri, II, Romæ 1628, pp. 409-410}

\noindent Ex Oratiónibus sanctæ Birgíttæ attribútis.

\noindent Benedíctio ætérna sit tibi, mi Dómine Iesu Christe, qui exsístens in mortis agonía ómnibus peccatóribus spem de vénia tribuísti, quando latróni ad te convérso paradísi glóriam misericórditer promisísti.

\noindent Benedíctus sis tu, Dómine mi Iesu Christe, qui tuo pretióso sánguine et sacratíssima morte ánimas redemísti, et eas ad vitam ætérnam de exsílio misericórditer reduxísti.

\noindent Benedíctus sis tu, Dómine mi Iesu Christe, qui pro salúte nostra tuum latus et cor perforári láncea permisísti, et pretiósum sánguinem tuum affluénter et aquam de eódem látere, ut nos redímeres, emisísti.

\noindent Glória sit tibi, mi Dómine Iesu Christe, eo quod corpus tuum benedíctum ab amícis tuis de cruce depóni et in mánibus tuæ mæstíssimæ matris reclinári voluísti, et ab ea pannis invólvi ac in monuménto sepelíri, a milítibus quoque ibídem custodíri permisísti.

\noindent Honor sempitérnus sit tibi, mi Dómine Iesu Christe, qui die tértia a mórtuis resurrexísti et, quibus tibi plácuit, te vivéntem manifestásti, post quadragínta quoque dies ad cælos multis vidéntibus ascendísti, ibíque amícos tuos, quos a tártaris liberáveras honorífice collocásti.

\noindent Iubilátio et laus ætérna sint tibi, Dómine Iesu Christe, qui Sanctum Spíritum in discipulórum córdibus transmisísti et imménsum amórem divínum in eórum spirítibus augmentásti.

\noindent Benedíctus sis tu et laudábilis et gloriósus in sǽcula, mi Dómine Iesu, qui sedes super thronum in tuo regno cælórum in glória tuæ divinitátis, vivens corporáliter cum ómnibus membris tuis sanctíssimis, quæ de carne Vírginis assumpsísti. Et sic ventúrus es in die iudícii ad iudicándum ánimas ómnium vivórum et mortuórum: qui vivis et regnas cum Patre et Spíritu Sancto in sǽcula sæculórum. Amen.}
\newcommand{\responsoriumiii}{\pars{Responsorium 3.} \scriptura{\Rbardot{} Ps. 44, 3.9-10 \Vbardot{} ibid. 5; \textbf{H297}}

\vspace{-5mm}

\responsorium{IV}{temporalia/resp-diffusaestgratia-CROCHU-cumdox.gtex}{}

\vfill
\pagebreak

\pars{Hymnus Ambrosianus} \scriptura{Alio modo, iuxta morem Romanum}

\vspace{-2mm}

{
\grechangedim{interwordspacetext}{0.26 cm plus 0.15 cm minus 0.05 cm}{scalable}%
\cuminitiali{III}{temporalia/tedeum-romanum-gn.gtex}
\grechangedim{interwordspacetext}{0.22 cm plus 0.15 cm minus 0.05 cm}{scalable}%
}}
\newcommand{\deusinadiutorium}{\grechangedim{interwordspacetext}{0.18 cm plus 0.15 cm minus 0.05 cm}{scalable}%
\cuminitiali{}{temporalia/deusinadiutorium-alter.gtex}
\grechangedim{interwordspacetext}{0.22 cm plus 0.15 cm minus 0.05 cm}{scalable}}
\newcommand{\hymnuslaudes}{\pars{Hymnus}

\cuminitiali{II}{temporalia/hym-IubiloCordis.gtex}}
\newcommand{\laudes}{\pars{Psalmus 1.} \scriptura{Mc. 16, 1}

\vspace{-4mm}

\antiphona{VIII G\textsuperscript{5}}{temporalia/ant-trahemepostte.gtex}

%\vspace{-2mm}

\scriptura{Psalmus 62}

%\vspace{-2mm}

\initiumpsalmi{temporalia/ps62-initium-viii-g6-auto.gtex}

%\vspace{-1.5mm}

\input{temporalia/ps62-viii-g6.tex} \Abardot{}

\vfill
\pagebreak

\pars{Psalmus 2.} \scriptura{\textbf{H125}}

\vspace{-0.4cm}

\antiphona{VII c}{temporalia/ant-egochristumconfiteor.gtex}

%\vspace{-2mm}

\scriptura{Canticum trium puerorum, Dan. 3, 57-88 et 56}

\initiumpsalmi{temporalia/dan3-initium-vii-c-auto.gtex}

\input{temporalia/dan3-vii-c-sinedox.tex}

\rubrica{Hic non dicitur Gloria Patri, neque Amen.}

\vfill

\antiphona{}{temporalia/ant-egochristumconfiteor.gtex}

\vfill
\pagebreak

\pars{Psalmus 3.} \scriptura{Ct. 2, 11.13}

\vspace{-4mm}

\antiphona{VIII G\textsuperscript{2}}{temporalia/ant-iamhiemstransiit.gtex}

%\vspace{-2mm}

\scriptura{Psalmus 149}

%\vspace{-2mm}

\initiumpsalmi{temporalia/ps149-initium-viii-G5-auto.gtex}

\input{temporalia/ps149-viii-G5.tex} \Abardot{}

\vfill
\pagebreak}
\newcommand{\lectiobrevis}{\pars{Lectio Brevis.} \scriptura{Rom. 12, 1-2}

\noindent Obsecro vos, fratres, per misericórdiam Dei, ut exhibeátis córpora vestra hóstiam vivéntem, sanctam, Deo placéntem, rationábile obséquium vestrum; et nolíte conformári huic sǽculo, sed transformámini renovatióne mentis, ut probétis quid sit volúntas Dei, quid bonum et bene placens et perféctum.}
\newcommand{\responsoriumbreve}{\pars{Responsorium breve.} \scriptura{Ps. 45, 6}

\cuminitiali{VI}{temporalia/resp-adiuvabiteam.gtex}}
\newcommand{\preces}{\noindent Cum ómnibus muliéribus sanctis, fratres, Salvatórem nostrum confiteámur,~\grestar{} simúlque invocémus:

\Rbardot{} Veni, Dómine Iesu.

\noindent Dómine Iesu, qui peccatríci multa dimisísti, quóniam diléxerat multum,~\grestar{} dimítte nobis, quia multum peccávimus.

\Rbardot{} Veni, Dómine Iesu.

\noindent Dómine Iesu, cui mulíeres sanctæ in itínere ministrábant,~\grestar{} concéde nobis ut vestígia tua sectémur.

\Rbardot{} Veni, Dómine Iesu.

\noindent Dómine Iesu, magíster, quem María audiébat, cum Martha tibi serviébat,~\grestar{} concéde nobis, ut in fide et caritáte serviámus tibi.

\Rbardot{} Veni, Dómine Iesu.

\noindent Dómine Iesu, qui fratrem, sorórem et matrem appellásti omnes tuam voluntátem faciéntes,~\grestar{} concéde nobis ut tibi semper verbis complaceámus et actis.

\Rbardot{} Veni, Dómine Iesu.}
\newcommand{\benedictus}{\pars{Canticum Zachariæ.} \scriptura{Mt. 13, 46; \textbf{H384}}

\vspace{-4mm}

\antiphona{VIII G}{temporalia/ant-inventabonamargarita.gtex}

\vspace{-2mm}

\scriptura{Lc. 1, 68-79}

\vspace{-2mm}

\cantusSineNeumas
\initiumpsalmi{temporalia/benedictus-initium-viii-G-auto.gtex}

%\vspace{-1.5mm}

\input{temporalia/benedictus-viii-G.tex} \Abardot{}}
\newcommand{\benedicamuslaudes}{\cuminitiali{VIII}{temporalia/benedicamus-adlaudes-festis.gtex}}
\newcommand{\hebdomada}{infra Hebdom. XVI post Pentecosten.}
\newcommand{\oratioLaudes}{\cuminitiali{}{temporalia/oratio16.gtex}}

% LuaLaTeX

\documentclass[a4paper, twoside, 12pt]{article}
\usepackage[latin]{babel} 
%\usepackage[landscape, left=3cm, right=1.5cm, top=2cm, bottom=1cm]{geometry} % okraje stranky
%\usepackage[landscape, a4paper, mag=1166, truedimen, left=2cm, right=1.5cm, top=1.6cm, bottom=0.95cm]{geometry} % okraje stranky
\usepackage[landscape, a4paper, mag=1400, truedimen, left=0.5cm, right=0.5cm, top=0.5cm, bottom=0.5cm]{geometry} % okraje stranky

\usepackage{fontspec}
\setmainfont[FeatureFile={junicode.fea}, Ligatures={Common, TeX}, RawFeature=+fixi]{Junicode}
%\setmainfont{Junicode}

% shortcut for Junicode without ligatures (for the Czech texts)
\newfontfamily\nlfont[FeatureFile={junicode.fea}, Ligatures={Common, TeX}, RawFeature=+fixi]{Junicode}

% Hebrew font:
% http://scripts.sil.org/cms/scripts/page.php?site_id=nrsi&id=SILHebrUnic2
\newfontfamily\hebfont[Scale=1]{Ezra SIL}

\usepackage{multicol}
\usepackage{color}
\usepackage{lettrine}
\usepackage{fancyhdr}

% usual packages loading:
\usepackage{luatextra}
\usepackage{graphicx} % support the \includegraphics command and options
\usepackage{gregoriotex} % for gregorio score inclusion
\usepackage{gregoriosyms}
\usepackage{wrapfig} % figures wrapped by the text
\usepackage{parcolumns}
\usepackage[contents={},opacity=1,scale=1,color=black]{background}
\usepackage{tikzpagenodes}
\usepackage{calc}
\usepackage{longtable}
\usetikzlibrary{calc}

\setlength{\headheight}{14.5pt}

\input{conventuscommune.tex} % Often used macros
%%%% Preklady jednotlivych zpevu (nektere se opakuji, a je dobre mit je
% vsechny na jedne hromade)

% HOURS ---

\newcommand{\trAntI}{\translatioCantus{Muž boží měl kožený toulec, pečlivě
zavázaný, jenž mu visel na šíji a~často se ho dotýkal.}}

\newcommand{\trAntII}{\translatioCantus{Klíč od~něho tak dobře střežil, že
dokud žil v~těle, nikdo z~jeho žáků nezvěděl, co je uvnitř.}}

\newcommand{\trAntIII}{\translatioCantus{Ale když se odebral z~tohoto
života, schránku otevřeli a~objevili v~ní žíněné roucho a~měděný řetěz
potřísněný krví.}}

\newcommand{\trAntIV}{\translatioCantus{A když prohlédli mistrovo tělo,
nalezli jeho tělo na čtyřech místech hluboce zbrázděno ranami od řetězu.}}

\newcommand{\trAntV}{\translatioCantus{Krev vytékající z~těch ran, místy
prostoupila i~žíněným rouchem.}}

\newcommand{\trCapituli}{\translatioCantus{
Miláčkovi Boha a~lidí,
Mojžíšovi požehnané paměti,~\gredagger{}
dopřál slávu rovnou slávě svatých~\grestar{}
učinil ho mocným na postrach nepřátelům
a~jeho slovy zastavil divy.}}

\newcommand{\trLectioBrevis}{\translatioCantus{
Pamatujte na své představené,
kteří vám hlásali Boží slovo.
Uvažte, jak oni skončili život, a~napodobujte jejich víru.
Ježíš Kristus je stejný včera i~dnes i~navěky.
Nenechte se svést věelijakými cizími naukami.}}

\newcommand{\trRespLaud}{\translatioCantus{Spravedlivého vodil Hospodin~\grestar{}
po přímých stezkách. \Vbardot{} A~ukázal mu Boží království.}}

\newcommand{\trRespLaudB}{\translatioCantus{Na tvých hradbách, Jeruzaléme,
ustanovil jsem strážné;~\grestar{}
budou bdít nad mým lidem. \Vbardot{} Ani ve dne, ani v~noci nesmějí nikdy
mlčet.}}

\newcommand{\trVersus}{\translatioCantus{\Vbardot{} Ústa spravedlivého šeptají moudrost, aleluja.
\Rbardot{} A~jeho jazyk ohlašuje právo, aleluja.}}

\newcommand{\trAntBenedictus}{\translatioCantus{Když na bujné oře vložili
nosítka a~sňali jim uzdu, vydali se přímo k~cele božího muže.}}

\newcommand{\trPreces}{\translatioCantus{
\noindent S vděčností chvalme Krista, dobrého Pastýře, \gredagger{} který dal život za své ovce, \grestar{} a~pokorně ho prosme: \Rbardot{} Pane, buď pastýřem svého lidu.

\noindent Kriste, ty dáváš církvi pastýře, a~jejich službou se ujímáš svého lidu, \grestar{} dej, ať v~lásce těch, kteří nás vedou, poznáváme, jak nás miluješ. \Rbardot{} Pane, buď pastýřem svého lidu.

\noindent Ty stále konáš skrze své zástupce službu pastýře a~učitele, \grestar{} nepřestávej nás nikdy vést prostřednictvím svých služebníků. \Rbardot{} Pane, buď pastýřem svého lidu.

\noindent Ty prokazuješ svému lidu skrze jeho pastýře službu lékaře duše i~těla, \grestar{} ochraňuj náš život a~veď nás ke svatosti. \Rbardot{} Pane, buď pastýřem svého lidu.

\noindent Ty posíláš své svaté, aby slovem i~příkladem vedli tvůj lid k~tobě, \grestar{} na jejich přímluvu nás posiluj, abychom vytrvali na cestě, která vede k~věčnému životu. \Rbardot{} Pane, buď pastýřem svého lidu.}}

\newcommand{\trOrationis}{\translatioCantus{Bože, jenž nám dopřáváš radovat
se z~výroční slavnosti svatého tvého vyznavače Havla, uděl dobrotivě,
abychom když slavíme jeho narození, též se řídili podobou jeho skutků.
Skrze…}}
 % Czech translations of the proper texts

\newcommand{\annusEditionis}{2020}

\def\hebinitial#1{%
\leavevmode{\newbox\hebbox\setbox\hebbox\hbox{\hebfont{#1}\hskip 1mm}\kern -\wd\hebbox\hbox{\hebfont{#1}\hskip 1mm}}%
}

%%%% Vicekrat opakovane kousky

\newcommand{\anteOrationem}{
  \rubrica{Ante Orationem, cantatur a Superiore:}

  \pars{Supplicatio Litaniæ.}

  \cuminitiali{}{temporalia/supplicatiolitaniae.gtex}

  \pars{Oratio Dominica.}

  \cuminitiali{}{temporalia/oratiodominica.gtex}

  \rubrica{Deinde dicitur ab Hebdomadario:}

  \cuminitiali{}{temporalia/dominusvobiscum-solemnis.gtex}

  \rubrica{In choro monialium loco Dominus vobiscum dicitur:}

  \sineinitiali{temporalia/domineexaudi.gtex}
}

\setlength{\columnsep}{30pt} % prostor mezi sloupci

%%%%%%%%%%%%%%%%%%%%%%%%%%%%%%%%%%%%%%%%%%%%%%%%%%%%%%%%%%%%%%%%%%%%%%%%%%%%%%%%%%%%%%%%%%%%%%%%%%%%%%%%%%%%%
\begin{document}

% Here we set the space around the initial.
% Please report to http://home.gna.org/gregorio/gregoriotex/details for more details and options
\grechangedim{afterinitialshift}{2.2mm}{scalable}
\grechangedim{beforeinitialshift}{2.2mm}{scalable}

\grechangedim{interwordspacetext}{0.32 cm plus 0.15 cm minus 0.05 cm}{scalable}%
\grechangedim{annotationraise}{-0.2cm}{scalable}

% Here we set the initial font. Change 38 if you want a bigger initial.
% Emit the initials in red.
\grechangestyle{initial}{\color{red}\fontsize{38}{38}\selectfont}

\pagestyle{empty}

%%%% Titulni stranka
\begin{titulusOfficii}
\nomenFesti{Feria IV \hebdomada{}}
\end{titulusOfficii}

\pagebreak

% graphic
\renewcommand{\headrulewidth}{0pt} % no horiz. rule at the header
\fancyhf{}
\pagestyle{fancy}

\cantusSineNeumas

\hora{Ad Matutinum.}

\vspace{2mm}

\cuminitiali{}{temporalia/dominelabiamea.gtex}

\vspace{2mm}

\pars{Invitatorium.} \scriptura{Lc. 24, 34; Psalmus 94; \textbf{H232}}

\vspace{-6mm}

\antiphona{VI}{temporalia/inv-surrexitdominusvere.gtex}

\vfill
\pagebreak

\pars{Hymnus.}

\vspace{-5mm}

\scriptura{\textbf{AR454}}

{
\grechangedim{interwordspacetext}{0.30 cm plus 0.15 cm minus 0.05 cm}{scalable}%
\antiphona{IV}{temporalia/hym-RexSempiterne.gtex}
\grechangedim{interwordspacetext}{0.32 cm plus 0.15 cm minus 0.05 cm}{scalable}%
}
%{
%\vspace{-5mm}
%\setlength{\columnsep}{0pt} % prostor mezi sloupci
%\input{hym-RexSempiterne-bohtext.tex}
%\setlength{\columnsep}{30pt} % prostor mezi sloupci
%}

\vfill
\pagebreak

\pars{Psalmus 1.}

%\vspace{-5mm}

\antiphona{I g}{temporalia/ant-alleluia-fiv-matutinum.gtex}

%\vspace{-5mm}

\scriptura{Ps. 44, 2-10}

%\vspace{-2mm}

\initiumpsalmi{temporalia/ps44i-initium-i-g-auto.gtex}

%\psalmusEtTranslatioT{temporalia/ps44i-III-comb.tex}{10cm}

\input{temporalia/ps44i-III.tex}

\vfill
\pagebreak

\pars{Psalmus 2.} \scriptura{Ps. 44, 11-18}

%\vspace{-2mm}

\initiumpsalmi{temporalia/ps44ii-initium-i-g-auto.gtex}

%\psalmusEtTranslatioT{temporalia/ps44i-III-comb.tex}{10cm}

\input{temporalia/ps44ii-III.tex}

\vfill
\pagebreak

\pars{Psalmus 3.} \scriptura{Ps. 45}

%\vspace{-2mm}

\initiumpsalmi{temporalia/ps45-initium-i-g-auto.gtex}

%\psalmusEtTranslatioT{temporalia/ps45-III-comb.tex}{10cm}

\input{temporalia/ps45-III.tex}

\vfill
\pagebreak

\pars{Psalmus 4.} \scriptura{Ps. 47}

%\vspace{-2mm}

\initiumpsalmi{temporalia/ps47-initium-i-g-auto.gtex}

%\psalmusEtTranslatioT{temporalia/ps47-III-comb.tex}{10cm}

\input{temporalia/ps47-III.tex}

\vfill
\pagebreak

\pars{Psalmus 5.} \scriptura{Ps. 48, 2-13}

%\vspace{-2mm}

\initiumpsalmi{temporalia/ps48i-initium-i-g-auto.gtex}

%\psalmusEtTranslatioT{temporalia/ps48i-III-comb.tex}{10cm}

\input{temporalia/ps48i-III.tex}

\vfill
\pagebreak

\pars{Psalmus 6.} \scriptura{Ps. 48, 14-21}

%\vspace{-2mm}

\initiumpsalmi{temporalia/ps48ii-initium-i-g-auto.gtex}

%\psalmusEtTranslatioT{temporalia/ps48ii-III-comb.tex}{10cm}

\input{temporalia/ps48ii-III.tex}

\vfill
\pagebreak

\pars{Psalmus 7.} \scriptura{Ps. 49, 1-15}

%\vspace{-2mm}

\initiumpsalmi{temporalia/ps49i-initium-i-g-auto.gtex}

%\psalmusEtTranslatioT{temporalia/ps49i-III-comb.tex}{10cm}

\input{temporalia/ps49i-III.tex}

\vfill
\pagebreak

\pars{Psalmus 8.} \scriptura{Ps. 49, 16-23}

%\vspace{-2mm}

\initiumpsalmi{temporalia/ps49ii-initium-i-g-auto.gtex}

%\psalmusEtTranslatioT{temporalia/ps49ii-III-comb.tex}{10cm}

\input{temporalia/ps49ii-III.tex}

\vfill
\pagebreak

\pars{Psalmus 9.} \scriptura{Ps. 50}

%\vspace{-2mm}

\initiumpsalmi{temporalia/ps50-initium-i-g-auto.gtex}

%\psalmusEtTranslatioT{temporalia/ps50-VI-comb.tex}{10cm}

\input{temporalia/ps50-VI.tex}

\vfill
%\pagebreak

\antiphona{}{temporalia/ant-alleluia-fiv-matutinum.gtex}

\vfill
\pagebreak

\noindent \Vbardot{} Gavísi sunt discípuli, allelúia.
\noindent \Rbardot{} Viso Dómino, allelúia.

\noindent Pater noster.

\pars{Absolutio.}

\cuminitiali{}{temporalia/absolutio-avinculis.gtex}

\vfill
\pagebreak

\ifx\magnificat\undefined
\cuminitiali{}{temporalia/benedictio-solemn-evangelica.gtex}
\else
\cuminitiali{}{temporalia/benedictio-solemn-ille.gtex}
\fi

\vspace{7mm}

\lectioi

\noindent \Vbardot{} Tu autem, Dómine, miserére nobis.
\noindent \Rbardot{} Deo grátias.

\vfill
\pagebreak

\responsoriumi

\vfill
\pagebreak

\cuminitiali{}{temporalia/benedictio-solemn-divinum.gtex}

\vspace{7mm}

\lectioii

\noindent \Vbardot{} Tu autem, Dómine, miserére nobis.
\noindent \Rbardot{} Deo grátias.

\vfill
\pagebreak

\responsoriumii

\vfill
\pagebreak

\ifx\magnificat\undefined
\cuminitiali{}{temporalia/benedictio-solemn-adsocietatem.gtex}
\else
\cuminitiali{}{temporalia/benedictio-solemn-ignem.gtex}
\fi

\vspace{7mm}

\lectioiii

\noindent \Vbardot{} Tu autem, Dómine, miserére nobis.
\noindent \Rbardot{} Deo grátias.

\vfill
\pagebreak

% Te Deum

%\pars{Hymnus Ambrosianus}

\vspace{-5mm}

{
\grechangedim{interwordspacetext}{0.22 cm plus 0.15 cm minus 0.05 cm}{scalable}%
\cuminitiali{III}{temporalia/tedeum-solemnis.gtex}
\grechangedim{interwordspacetext}{0.32 cm plus 0.15 cm minus 0.05 cm}{scalable}%
}

\vfill
\pagebreak

\rubrica{Reliqua omittuntur, nisi Laudes separandæ sint.}

\pars{Oratio}

\noindent \Vbardot{} Dómine, exáudi oratiónem meam.

\noindent \Rbardot{} Et clamor meus ad te véniat.

Orémus:

\oratioMatutinum

\noindent \Rbardot{} Amen.

\vspace{7mm}

\pars{Conclusio}

\noindent \Vbardot{} Dómine, exáudi oratiónem meam.

\noindent \Rbardot{} Et clamor meus ad te véniat.

\noindent \Vbardot{} Benedicámus Dómino, allelúia, allelúia.

\noindent \Rbardot{} Deo grátias, allelúia, allelúia.

\noindent \Vbardot{} Fidélium ánimæ per misericórdiam Dei requiéscant in pace.

\noindent \Rbardot{} Amen.

\vfill
\pagebreak

\hora{Ad Laudes.} %%%%%%%%%%%%%%%%%%%%%%%%%%%%%%%%%%%%%%%%%%%%%%%%%%%%%
%\sideThumbs{Laudes}

\cantusSineNeumas

\vspace{0.5cm}
\grechangedim{interwordspacetext}{0.18 cm plus 0.15 cm minus 0.05 cm}{scalable}%
\cuminitiali{}{temporalia/deusinadiutorium-communis.gtex}
\grechangedim{interwordspacetext}{0.32 cm plus 0.15 cm minus 0.05 cm}{scalable}%

\vfill
%\pagebreak

\pars{Psalmus 1.}

\vspace{-0.4cm}

\antiphona{VII a}{temporalia/ant-alleluia-fiv-laudes-1.gtex}

\scriptura{Psalmus 50.}

\initiumpsalmi{temporalia/ps50-initium-vii-a-auto.gtex}

%\psalmusEtTranslatioT{temporalia/ps50-III-comb.tex}{10cm}
\input{temporalia/ps50-III.tex}

\vspace{-1cm}

\vfill
\pagebreak

\pars{Psalmus 2.} \scriptura{Psalmus 63.}

\initiumpsalmi{temporalia/ps63-initium-vii-a-auto.gtex}

%\psalmusEtTranslatioT{temporalia/ps63-III-comb.tex}{10cm}
\input{temporalia/ps63-III.tex}

\vfill
\pagebreak

\pars{Psalmus 3.} \scriptura{Psalmus 64.}

\initiumpsalmi{temporalia/ps64-initium-vii-a-auto.gtex}

%\psalmusEtTranslatioT{temporalia/ps64-III-comb.tex}{10cm}
\input{temporalia/ps64-III.tex}

\vfill

\vspace{-6mm}

\antiphona{}{temporalia/ant-alleluia-fiv-laudes-1.gtex} % repeat the antiphon - new page

\vfill
\pagebreak

\pars{Psalmus 4.} \scriptura{1 Sam. 2, 10; \textbf{H96}}

\vspace{-7mm}

\antiphona{I g\textsuperscript{2}}{temporalia/ant-dominusjudicabit-tp.gtex}

%\vspace{-4mm}

\scriptura{Canticum Annæ, 1 Reg. 2, 1-10}

%\vspace{-3mm}

\initiumpsalmi{temporalia/anna-initium-i-g2-auto.gtex}

%\psalmusEtTranslatioT{temporalia/anna-comb.tex}{10cm}
\input{temporalia/anna.tex}

%\vfill

\antiphona{}{temporalia/ant-dominusjudicabit-tp.gtex}

\vfill
\pagebreak

\pars{Psalmus 5.}

\vspace{-0.4cm}

\antiphona{II D}{temporalia/ant-alleluia-fiv-laudes-2.gtex}

\scriptura{Psalmus 148.}

\initiumpsalmi{temporalia/ps148-initium-ii-D-auto.gtex}

%\psalmusEtTranslatioT{temporalia/ps148-III-comb.tex}{10cm}
\input{temporalia/ps148-III.tex}

\rubrica{Hic non dicitur Gloria Patri.}

\vfill
\pagebreak

%
\scriptura{Psalmus 149.}

\initiumpsalmi{temporalia/ps149-initium-ii-D-auto.gtex}

%\psalmusEtTranslatioT{temporalia/ps149-III-comb.tex}{10cm}
\input{temporalia/ps149-III.tex}

\rubrica{Hic non dicitur Gloria Patri.}

\vfill
\pagebreak

%
\scriptura{Psalmus 150.}

\initiumpsalmi{temporalia/ps150-initium-ii-D-auto.gtex}

%\psalmusEtTranslatioT{temporalia/ps150-III-comb.tex}{10cm}
\input{temporalia/ps150-III.tex}

\vfill

\vspace{-6mm}

\antiphona{}{temporalia/ant-alleluia-fiv-laudes-2.gtex} % repeat the antiphon - new page

\vfill
\pagebreak

\pars{Capitulum.} \scriptura{Rom. 6, 9-10}

\grechangedim{interwordspacetext}{0.12 cm plus 0.15 cm minus 0.05 cm}{scalable}%
\cuminitiali{}{temporalia/capitulum-ChristusResurgens.gtex}
\grechangedim{interwordspacetext}{0.32 cm plus 0.15 cm minus 0.05 cm}{scalable}%

% preklad Jeruz. bible
%\trCapituliI

\vfill

\pars{Responsorium breve.} \scriptura{Cf. Mt. 28, 6; Cf. Gal. 3, 13}

\cuminitiali{VI}{temporalia/respbr-laud.gtex}

%\trResp

\vfill
\pagebreak

\pars{Hymnus}

\cuminitiali{VIII}{temporalia/hym-AuroraLucis.gtex}
\vspace{-3mm}
%\input{hym-AuroraLucis-bohtext.tex}

\vfill
%\pagebreak

\pars{Versus.}

% Versus. %%%
\sineinitiali{temporalia/versus-inresurrectione.gtex}

%\noindent \trVersus

\vfill
\pagebreak

\benedictus

\vspace{-1cm}

\vfill
\pagebreak

%\sideThumbs{{\scriptsize{}Fine horarum}}

\anteOrationem

\pagebreak

% Oratio. %%%
\oratioLaudes

\vspace{-1mm}
%\trOrationisI

\vfill

\rubrica{Hebdomadarius dicit iterum Dominus vobiscum. Postea cantatur a cantore:}
\vspace{2mm}

\cuminitiali{VII}{temporalia/benedicamus-tempore-paschali.gtex}

\vspace{1mm}

\ifx\magnificat\undefined
\else
\vfill
\pagebreak

\hora{Ad Vesperas.} %%%%%%%%%%%%%%%%%%%%%%%%%%%%%%%%%%%%%%%%%%%%%%%%%%%%%
%\sideThumbs{Vesperæ}

\cantusSineNeumas

%\vspace{0.5cm}
\grechangedim{interwordspacetext}{0.18 cm plus 0.15 cm minus 0.05 cm}{scalable}%
\cuminitiali{}{temporalia/deusinadiutorium-communis.gtex}
\grechangedim{interwordspacetext}{0.32 cm plus 0.15 cm minus 0.05 cm}{scalable}%

\vfill
%\pagebreak

\vspace{4mm}

\pars{Psalmus 1.}

\vspace{-0.4cm}

\antiphona{III g}{temporalia/ant-alleluia-fiv-vesperas.gtex}

\vspace{-4mm}

\scriptura{Psalmus 134.}

\initiumpsalmi{temporalia/ps134-initium-iii-g-auto.gtex}

%\psalmusEtTranslatioT{temporalia/ps134-III-comb.tex}{10cm}
\input{temporalia/ps134-III.tex}

\vspace{-1cm}

\vfill
\pagebreak

\pars{Psalmus 2.} \scriptura{Psalmus 135.}

\initiumpsalmi{temporalia/ps135-initium-iii-g-auto.gtex}

%\psalmusEtTranslatioT{temporalia/ps135-III-comb.tex}{10cm}
\input{temporalia/ps135-III.tex}

\vfill
\pagebreak

\pars{Psalmus 3.} \scriptura{Psalmus 136.}

\initiumpsalmi{temporalia/ps136-initium-iii-g-auto.gtex}

%\psalmusEtTranslatioT{temporalia/ps136-III-comb.tex}{10cm}
\input{temporalia/ps136-III.tex}

\vfill
\pagebreak

\pars{Psalmus 4.} \scriptura{Psalmus 137.}

\initiumpsalmi{temporalia/ps137-initium-iii-g-auto.gtex}

%\psalmusEtTranslatioT{temporalia/ps137-III-comb.tex}{10cm}
\input{temporalia/ps137-III.tex}

\vfill

\vspace{-6mm}

\antiphona{}{temporalia/ant-alleluia-fiv-vesperas.gtex} % repeat the antiphon - new page

\vfill
\pagebreak

\pars{Capitulum.} \scriptura{Rom. 6, 9-10}

\grechangedim{interwordspacetext}{0.12 cm plus 0.15 cm minus 0.05 cm}{scalable}%
\cuminitiali{}{temporalia/capitulum-ChristusResurgens.gtex}
\grechangedim{interwordspacetext}{0.32 cm plus 0.15 cm minus 0.05 cm}{scalable}%

% preklad Jeruz. bible
%\trCapituliI

\vfill

\pars{Responsorium breve.} \scriptura{Lc. 24, 34}

\cuminitiali{VI}{temporalia/respbr-vesp.gtex}

%\trResp

\vfill
\pagebreak

\pars{Hymnus}

\cuminitiali{VIII}{temporalia/hym-AdCoenam.gtex}
\vspace{-3mm}
%\begin{translatioMulticol}{4}
U~Beránkovy hostiny\\
oděni rouchy bílými,\\
když Rudým mořem prošli jsme,\\
Vladaři Kristu zpívejme.\\
\\
Když jeho tělem posvátným,\\
na kříži obětovaným,\\
se sytíme a~pijeme\\
jeho krev, v~Bohu žijeme.\columnbreak

Chráněni tímto pokrmem\\
před smrtonosným andělem,\\
svrhli jsme z~beder kruté jho\\
tyrana bezohledného.\\
\\
Kristus je naší paschou teď,\\
on sám se vydal za oběť\\
a~místo přesnic našim rtům\\
své tělo dává za pokrm.\columnbreak

Tys, nejčistější Oběti,\\
zlomila vládu podsvětí.\\
Z~otroctví lid je vykoupen,\\
odměna žití kyne všem.\\
\\
Hle, Kristus, když vstal ze hrobu,\\
jde z~pekel v~slavném průvodu\\
a~brány nebes otevřev,\\
vládce tmy vleče v~okovech.\columnbreak

Buď věčně, Kriste, věrným svým\\
plesáním velikonočním.\\
Nás, milostí tvou vzkříšené,\\
vem k~oslavě své vítězné. \\
\\
Sláva tobě, Pane,\\
jenž jsi vstal z~mrtvých,\\
s~Otcem i~Svatým Duchem\\
na věčné věky.\\
Amen.
\end{translatioMulticol}


\vfill
\pagebreak

\pars{Versus.} \scriptura{Lc. 24, 29}

% Versus. %%%
\sineinitiali{temporalia/versus-mane.gtex}

%\noindent \trVersus

\vfill
\pagebreak

\magnificat

\vspace{-1cm}

\vfill
\pagebreak

%\sideThumbs{{\scriptsize{}Fine horarum}}

\anteOrationem

\pagebreak

% Oratio. %%%
\oratioLaudes

\vspace{-1mm}
%\trOrationisI

\vfill

\rubrica{Hebdomadarius dicit iterum Dominus vobiscum. Postea cantatur a cantore:}
\vspace{2mm}

\cuminitiali{VII}{temporalia/benedicamus-tempore-paschali.gtex}

\vspace{1mm}
\fi

\end{document}

