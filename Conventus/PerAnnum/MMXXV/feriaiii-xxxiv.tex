\newcommand{\titulus}{\nomenFesti{S. Catharinæ Alexandrinæ, Virginis \& Martyris.}
\dies{Die 25. Novembris.}}
\newcommand{\oratio}{\pars{Oratio.}

\noindent Omnípotens sempitérne Deus, qui pópulo tuo beátam Catharínam, vírginem, et invíctam mártyrem præstitísti, concéde, ut, eius intercessióne, fide et constántia roborémur et pro Ecclésiæ unitáte óperam tribuámus impénse.

\pars{Pro pace in universo mundo.} \scriptura{Sir. 50, 25; 2 Esdr. 4, 20; \textbf{H416}}

\vspace{-4mm}

\antiphona{II D}{temporalia/ant-dapacemdomine.gtex}

\vfill

\noindent Deus, a quo sancta desidéria, recta consília et iusta sunt ópera: da servis tuis illam, quam mundus dare non potest, pacem; ut et corda nostra mandátis tuis dédita, et hóstium subláta formídine, témpora sint tua protectióne tranquílla.

\noindent Per Dóminum nostrum Iesum Christum, Fílium tuum, qui tecum vivit et regnat in unitáte Spíritus Sancti, Deus, per ómnia sǽcula sæculórum.

\noindent \Rbardot{} Amen.}
\newcommand{\invitatorium}{\pars{Invitatorium.}

\vspace{-4mm}

\antiphona{E}{temporalia/inv-regemvirginum.gtex}}
\newcommand{\hymnusmatutinum}{\pars{Hymnus.}

\vspace{-5mm}

\antiphona{I}{temporalia/hym-OChristeFlos.gtex}}
\newcommand{\lectioi}{\pars{Lectio I.} \scriptura{2 Petr. 1, 12-21}

\noindent De Epístola secúnda beáti Petri apóstoli.

\noindent Caríssimi: Incípiam vos semper commonére de his, et quidem sciéntes et confirmátos in præsénti veritáte. Iustum autem árbitror, quámdiu sum in hoc tabernáculo, suscitáre vos in commonitióne, certus quod velox est deposítio tabernáculi mei, secúndum quod et Dóminus noster Iesus Christus significávit mihi; dabo autem óperam et frequénter habére vos post óbitum meum, ut horum memóriam faciátis.

\noindent Non enim captiósas fábulas secúti notam fécimus vobis Dómini nostri Iesu Christi virtútem et advéntum, sed speculatóres facti illíus magnitúdinis. Accípiens enim a Deo Patre honórem et glóriam, voce proláta ad eum huiuscémodi a magnífica glória: «Fílius meus, diléctus meus hic est, in quo ego mihi complácui»; et hanc vocem nos audívimus de cælo prolátam, cum essémus cum ipso in monte sancto. Et habémus firmiórem prophéticum sermónem, cui bene fácitis attendéntes quasi lucérnæ lucénti in caliginóso loco, donec dies illucéscat, et lúcifer oriátur in córdibus vestris, hoc primum intellegéntes quod omnis prophetía Scriptúræ própria interpretatióne non fit; non enim voluntáte humána proláta est prophetía aliquándo, sed a Spíritu Sancto ducti locúti sunt a Deo hómines.}
\newcommand{\responsoriumi}{\pars{Responsorium 1.} \scriptura{\Rbardot{} Ier. 31, 11.12 \Vbardot{} Ps. 4, 8; \textbf{H418}}

\vspace{-5mm}

\responsorium{III}{temporalia/resp-redemitdominus-CROCHU.gtex}{}}
\newcommand{\lectioii}{\pars{Lectio II.} \scriptura{Sermo 159, 1. 3-6: CCL 104, 650. 652-654}

\noindent Ex Sermónibus sancti Cæsárii Arelaténsis epíscopi.

\noindent Durum vidétur, fratres caríssimi, et quasi grave esse iudicátur illud, quod Dóminus in Evangélio imperávit, dicens: «Si quis vult post me veníre, ábneget semetípsum sibi». Sed non est durum, quod ille ímperat, qui ádiuvat ut fiat quod ímperat.

\noindent Abnéget se et tollat crucem suam et sequátur Christum. Et quo sequéndus est Christus, nisi quo ábiit? Nóvimus enim, quia surréxit, ascéndit in cælum: illo sequéndus est. Plane desperándum non est, quia ipse promísit, non quia homo áliquid potest. Longe a nobis erat cælum, ante quam caput nostrum iísset in cælum. Iam quare desperámus nos ibi futúros, si membra illíus cápitis simus? Ergo unde? Quia multis timóribus et dolóribus laborátur in terra: sequámur Christum, ubi summa est felícitas, summa pax, perpétua secúritas.

\noindent Sed qui Christum sequi desíderat, áudiat apóstolum dicéntem: «Si quis dicit se in Christo manére, debet quómodo ille ambulávit et ipse ambuláre». Sequi vis Christum? Esto húmilis, ubi ille húmilis fuit: noli humilitátem eius contémnere, si vis ad illíus altitúdinem perveníre.

\noindent Aspera quidem facta est via, quando homo peccávit; sed plana est, quando eam Christus resurgéndo calcávit, et de angustíssima sémita stratam regálem fecit. Per istam viam duóbus pédibus cúrritur, id est, humilitátis et caritátis. In hoc omnes deléctat celsitúdo: sed humílitas primus gradus est. Quid tendis pedem ultra te? Cádere vis, non ascéndere. A primo gradu, id est, ab humilitáte íncipe, et ascendísti.}
\newcommand{\responsoriumii}{\pars{Responsorium 2.} \scriptura{\Vbardot{} Ps. 44, 11; \textbf{H383}}

\vspace{-5mm}

\responsorium{III}{temporalia/resp-venisponsachristi-CROCHU.gtex}{}}
\newcommand{\lectioiii}{\pars{Lectio III.}

\noindent Et ídeo Dóminus et Salvátor noster non solum dixit ábneget seípsum sibi, sed áddidit: «tollat crucem suam, et sequátur me». Quid est, tollat crucem suam? Ferat quicquid moléstum est: sic me sequátur. Cum cœ́perit me móribus et præcéptis meis sequi, multos habébit contradictóres, multos habébit prohibitóres, multos habébit non solum derisóres, sed étiam persecutóres. Et hoc non solum de pagánis, qui extra ecclésiam sunt, sed étiam ex illis, qui intus vidéntur esse córpore, sed foris sunt óperum pravitáte, et, cum de solo nómine christiáno gloriéntur, bonos tamen christiános iúgiter persequúntur. Isti tales sic sunt in membris ecclésiæ, quómodo mali humóres in córpore. Tu ergo si Christum sequi desíderas, crucem eius portáre non dífferas: tólera malos, noli subcúmbere.

\noindent Unde, si vólumus implére illud, quod Dóminus dixit, si quis vult post me veníre, tollat crucem suam et sequátur me, quod ait apóstolus cum Dei adiutório studeámus implére: ut habéntes victum et vestítum, his conténti simus; ne forte, si plus quam opórtet terrénam substántiam quæréntes vólumus dívites fíeri, incidámus in temptatiónem et láqueum diáboli, et desidéria multa et inutília et nocíva, quæ mergunt hómines in intéritum et perditiónem. De qua temptatióne Dóminus nos sub sua protectióne liberáre dignétur.}
\newcommand{\responsoriumiii}{\pars{Responsorium 3.}

\vspace{-5mm}

\responsorium{VIII}{temporalia/resp-ipsisumdesponsata-CROCHU-cumdox.gtex}{}}
\newcommand{\hymnuslaudes}{\pars{Hymnus.}

\cuminitiali{VI}{temporalia/hym-OCastitatisSignifer.gtex}}
\newcommand{\lectiobrevis}{\pars{Lectio Brevis.} \scriptura{2 Cor. 1, 3-5}

\noindent Benedíctus Deus et Pater Dómini nostri Iesu Christi, Pater misericordiárum et Deus totíus consolatiónis, qui consolátur nos in omni tribulatióne nostra, ut possímus et ipsi consolári eos, qui in omni pressúra sunt, per exhortatiónem, qua exhortámur et ipsi a Deo; quóniam, sicut abúndant passiónes Christi in nobis, ita per Christum abúndat et consolátio nostra.}
\newcommand{\responsoriumbreve}{\pars{Responsorium breve.} \scriptura{Ex. 15, 2}

\cuminitiali{VI}{temporalia/resp-fortitudomeaetlausmea.gtex}}
\newcommand{\preces}{\noindent Christo, vírginum sponso et corónæ, lætis vócibus iubilémus~\grestar{} et devóte eum exorémus, dicéntes:

\Rbardot{} Iesu, coróna vírginum, audi nos.

\noindent Christe, quem sanctæ vírgines ut únicum sponsum dilexérunt,~\grestar{} concéde ut nihil a tua caritáte nos séparet.

\Rbardot{} Iesu, coróna vírginum, audi nos.

\noindent Qui Maríam, matrem tuam, regínam vírginum coronásti,~\grestar{} da nos, eius intercessióne, cordis puritáte tibi iúgiter famulári.

\Rbardot{} Iesu, coróna vírginum, audi nos.

\noindent Per intercessiónem ancillárum tuárum,~\gredagger{} quæ íntegro et indivíso corde semper de te sollícitæ fuérunt, ut essent sanctæ córpore et spíritu,~\grestar{} præsta ut figúra huius mundi, quæ prǽterit, nos a te numquam ábstrahat.

\Rbardot{} Iesu, coróna vírginum, audi nos.

\noindent Dómine Iesu,~\gredagger{} sponse, quem vírgines sapiéntes adventúrum exspectábant,~\grestar{} concéde nobis, ut in spe vigilántes te præstolémur.

\Rbardot{} Iesu, coróna vírginum, audi nos.

\noindent Per intercessiónem sanctæ Catharínæ,~\gredagger{} quæ virgo fuit sápiens et de número prudéntum,~\grestar{} præsta nobis innocéntiam vitæ et sapiéntiam.

\Rbardot{} Iesu, coróna vírginum, audi nos.}
\newcommand{\benedictus}{\pars{Canticum Zachariæ.}

\vspace{-4mm}

\antiphona{I a\textsuperscript{2}}{temporalia/ant-desponsatasumchristo.gtex}

\vspace{-2mm}

\scriptura{Lc. 1, 68-79}

\vspace{-2mm}

\initiumpsalmi{temporalia/benedictus-initium-i-a4-auto.gtex}

%\vspace{-1.5mm}

\input{temporalia/benedictus-i-a4.tex} \Abardot{}}
\newcommand{\benedicamuslaudes}{\cuminitiali{}{temporalia/benedicamus-memoria-laudes.gtex}}
\newcommand{\hebdomada}{infra Hebdom. XXIV per Annum.}
\newcommand{\hiemalis}{Hiemalis}
\newcommand{\matub}{Matutinum Hebdomadae B}
\newcommand{\matubd}{Matutinum Hebdomadae B vel D}
\newcommand{\laudb}{Laudes Hebdomadae B}
\newcommand{\laudbd}{Laudes Hebdomadae B vel D}

% LuaLaTeX

\documentclass[a4paper, twoside, 12pt]{article}
\usepackage[latin]{babel}
%\usepackage[landscape, left=3cm, right=1.5cm, top=2cm, bottom=1cm]{geometry} % okraje stranky
%\usepackage[landscape, a4paper, mag=1166, truedimen, left=2cm, right=1.5cm, top=1.6cm, bottom=0.95cm]{geometry} % okraje stranky
\usepackage[landscape, a4paper, mag=1400, truedimen, left=0.5cm, right=0.5cm, top=0.5cm, bottom=0.5cm]{geometry} % okraje stranky

\usepackage{fontspec}
\setmainfont[FeatureFile={junicode.fea}, Ligatures={Common, TeX}, RawFeature=+fixi]{Junicode}
%\setmainfont{Junicode}

% shortcut for Junicode without ligatures (for the Czech texts)
\newfontfamily\nlfont[FeatureFile={junicode.fea}, Ligatures={Common, TeX}, RawFeature=+fixi]{Junicode}

\usepackage{multicol}
\usepackage{color}
\usepackage{lettrine}
\usepackage{fancyhdr}

% usual packages loading:
\usepackage{luatextra}
\usepackage{graphicx} % support the \includegraphics command and options
\usepackage{gregoriotex} % for gregorio score inclusion
\usepackage{gregoriosyms}
\usepackage{wrapfig} % figures wrapped by the text
\usepackage{parcolumns}
\usepackage[contents={},opacity=1,scale=1,color=black]{background}
\usepackage{tikzpagenodes}
\usepackage{calc}
\usepackage{longtable}
\usetikzlibrary{calc}

\setlength{\headheight}{14.5pt}

\input{conventuscommune.tex} % Often used macros

\newcommand{\annusEditionis}{2021}

%%%% Vicekrat opakovane kousky

\newcommand{\anteOrationem}{
  \rubrica{Ante Orationem, cantatur a Superiore:}

  \pars{Supplicatio Litaniæ.}

  \cuminitiali{}{temporalia/supplicatiolitaniae.gtex}

  \pars{Oratio Dominica.}

  \cuminitiali{}{temporalia/oratiodominica.gtex}

  \rubrica{Deinde dicitur ab Hebdomadario:}

  \cuminitiali{}{temporalia/dominusvobiscum-solemnis.gtex}

  \rubrica{In choro monialium loco Dominus vobiscum dicitur:}

  \sineinitiali{temporalia/domineexaudi.gtex}
}

\setlength{\columnsep}{30pt} % prostor mezi sloupci

%%%%%%%%%%%%%%%%%%%%%%%%%%%%%%%%%%%%%%%%%%%%%%%%%%%%%%%%%%%%%%%%%%%%%%%%%%%%%%%%%%%%%%%%%%%%%%%%%%%%%%%%%%%%%
\begin{document}

% Here we set the space around the initial.
% Please report to http://home.gna.org/gregorio/gregoriotex/details for more details and options
\grechangedim{afterinitialshift}{2.2mm}{scalable}
\grechangedim{beforeinitialshift}{2.2mm}{scalable}
\grechangedim{interwordspacetext}{0.22 cm plus 0.15 cm minus 0.05 cm}{scalable}%
\grechangedim{annotationraise}{-0.2cm}{scalable}

% Here we set the initial font. Change 38 if you want a bigger initial.
% Emit the initials in red.
\grechangestyle{initial}{\color{red}\fontsize{38}{38}\selectfont}

\pagestyle{empty}

%%%% Titulni stranka
\begin{titulusOfficii}
\ifx\titulus\undefined
\nomenFesti{Feria III \hebdomada{}}
\else
\titulus
\fi
\end{titulusOfficii}

\vfill

\begin{center}
%Ad usum et secundum consuetudines chori \guillemotright{}Conventus Choralis\guillemotleft.

%Editio Sancti Wolfgangi \annusEditionis
\end{center}

\scriptura{}

\pars{}

\pagebreak

\renewcommand{\headrulewidth}{0pt} % no horiz. rule at the header
\fancyhf{}
\pagestyle{fancy}

\cantusSineNeumas

\ifx\oratio\undefined
\ifx\laudb\undefined
\else
\newcommand{\oratio}{\pars{Oratio.}

\noindent Dómine Iesu Christe, lux vera, qui omnes hómines illúminas ad salútem, nobis, quǽsumus, concéde virtútem, ut ante te vias pacis et iustítiæ præparémus.

\noindent Qui vivis et regnas cum Deo Patre in unitáte Spíritus Sancti, Deus, per ómnia sǽcula sæculórum.

\noindent \Rbardot{} Amen.}
\fi
\fi

\hora{Ad Matutinum.} %%%%%%%%%%%%%%%%%%%%%%%%%%%%%%%%%%%%%%%%%%%%%%%%%%%%%

\vspace{2mm}

\cuminitiali{}{temporalia/dominelabiamea.gtex}

\vfill
%\pagebreak

\vspace{2mm}

\ifx\invitatorium\undefined
\ifx\matuac\undefined
\else
\pars{Invitatorium.} \scriptura{Ps. 94, 1; Psalmus 94; \textbf{H451}}

\vspace{-6mm}

\antiphona{VI}{temporalia/inv-jubilemusdeo.gtex}
\fi
\ifx\matubd\undefined
\else
\pars{Invitatorium.} \scriptura{Cantor; Psalmus 94; \textbf{H449}}

\vspace{-6mm}

\antiphona{E}{temporalia/inv-regemmagnum.gtex}
\fi
\else
\invitatorium
\fi

\vfill
\pagebreak

\ifx\hymnusmatutinum\undefined
\ifx\matuac\undefined
\else
\pars{Hymnus}

\cuminitiali{IV}{temporalia/hym-SomnoRefectis.gtex}
\fi
\ifx\matubd\undefined
\else
\pars{Hymnus.} \scriptura{Gregorius Magnus (\olddag{} 604)}

{
\grechangedim{interwordspacetext}{0.10 cm plus 0.15 cm minus 0.05 cm}{scalable}%
\antiphona{I}{temporalia/hym-NocteSurgentes.gtex}
\grechangedim{interwordspacetext}{0.22 cm plus 0.15 cm minus 0.05 cm}{scalable}%
}
\fi
\else
\hymnusmatutinum
\fi

\vspace{-3mm}

\vfill
\pagebreak

\ifx\matub\undefined
\else
% MAT B
\pars{Psalmus 1.} \scriptura{Ps. 36, 5; \textbf{H93}}

\vspace{-4mm}

\antiphona{VI F}{temporalia/ant-reveladomino.gtex}

%\vspace{-2mm}

\scriptura{Ps. 36, 1-11}

%\vspace{-2mm}

\initiumpsalmi{temporalia/ps36i_xi-initium-vi-F-auto.gtex}

\input{temporalia/ps36i_xi-vi-F.tex} \Abardot{}

\vfill
\pagebreak

\pars{Psalmus 2.}

\vspace{-4mm}

\antiphona{II D}{temporalia/ant-iuniorfui.gtex}

\vspace{-2mm}

\scriptura{Ps. 36, 12-29}

\vspace{-2mm}

\initiumpsalmi{temporalia/ps36xii_xxix-initium-ii-D-auto.gtex}

\input{temporalia/ps36xii_xxix-ii-D.tex}

\vfill

\antiphona{}{temporalia/ant-iuniorfui.gtex}

\vfill
\pagebreak

\pars{Psalmus 3.} \scriptura{Ps. 36, 3}

\vspace{-4mm}

\antiphona{VI F}{temporalia/ant-speraindomino.gtex}

%\vspace{-2mm}

\scriptura{Ps. 36, 30-40}

%\vspace{-2mm}

\initiumpsalmi{temporalia/ps36iii-initium-vi-F-auto.gtex}

\input{temporalia/ps36iii-vi-F.tex} \Abardot{}

\vfill
\pagebreak
\fi
\ifx\matuc\undefined
\else
% MAT C
\pars{Psalmus 1.} \scriptura{Ps. 67, 2}

\vspace{-4mm}

\antiphona{VII a}{temporalia/ant-exsurgatdeus.gtex}

%\vspace{-2mm}

\scriptura{Ps. 67, 2-11}

\initiumpsalmi{temporalia/ps67i-initium-vii-a-auto.gtex}

\input{temporalia/ps67i-vii-a.tex} \Abardot{}

\vfill
\pagebreak

\pars{Psalmus 2.}

\vspace{-4mm}

\antiphona{I f}{temporalia/ant-deusnosterdeussalvos.gtex}

%\vspace{-2mm}

\scriptura{Ps. 67, 12-24}

%\vspace{-2mm}

\initiumpsalmi{temporalia/ps67ii-initium-i-f-auto.gtex}

\input{temporalia/ps67ii-i-f.tex} \Abardot{}

\vfill
\pagebreak

\pars{Psalmus 3.} \scriptura{Ps. 67, 27; \textbf{H96}}

\vspace{-4mm}

\antiphona{D}{temporalia/ant-inecclesiis.gtex}

%\vspace{-2mm}

\scriptura{Ps. 67, 25-36}

\initiumpsalmi{temporalia/ps67iii-initium-d-g2-auto.gtex}

\input{temporalia/ps67iii-d-g2.tex} \Abardot{}

\vfill
\pagebreak
\fi

\pars{Versus.}

\ifx\matversus\undefined
\ifx\matub\undefined
\else
\noindent \Vbardot{} Bonitátem et prudéntiam et sciéntiam doce me.

\noindent \Rbardot{} Quia præcéptis tuis crédidi.
\fi
\ifx\matuc\undefined
\else
\noindent \Vbardot{} Audiam quid loquátur Dóminus Deus.

\noindent \Rbardot{} Loquétur pacem ad plebem suam.
\fi
\else
\matversus
\fi

\vspace{5mm}

\sineinitiali{temporalia/oratiodominica-mat.gtex}

\vspace{5mm}

\pars{Absolutio.}

\cuminitiali{}{temporalia/absolutio-ipsius.gtex}

\vfill
\pagebreak

\cuminitiali{}{temporalia/benedictio-solemn-deus.gtex}

\vspace{7mm}

\lectioi

\noindent \Vbardot{} Tu autem, Dómine, miserére nobis.
\noindent \Rbardot{} Deo grátias.

\vfill
\pagebreak

\responsoriumi

\vfill
\pagebreak

\cuminitiali{}{temporalia/benedictio-solemn-christus.gtex}

\vspace{7mm}

\lectioii

\noindent \Vbardot{} Tu autem, Dómine, miserére nobis.
\noindent \Rbardot{} Deo grátias.

\vfill
\pagebreak

\responsoriumii

\vfill
\pagebreak

\cuminitiali{}{temporalia/benedictio-solemn-ignem.gtex}

\vspace{7mm}

\lectioiii

\noindent \Vbardot{} Tu autem, Dómine, miserére nobis.
\noindent \Rbardot{} Deo grátias.

\vfill
\pagebreak

\responsoriumiii

\vfill
\pagebreak

\rubrica{Reliqua omittuntur, nisi Laudes separandæ sint.}

\sineinitiali{temporalia/domineexaudi.gtex}

\vfill

\oratio

\vfill

\noindent \Vbardot{} Dómine, exáudi oratiónem meam.
\Rbardot{} Et clamor meus ad te véniat.

\vfill

\noindent \Vbardot{} Benedicámus Dómino.
\noindent \Rbardot{} Deo grátias.

\vfill

\noindent \Vbardot{} Fidélium ánimæ per misericórdiam Dei requiéscant in pace.
\Rbardot{} Amen.

\vfill
\pagebreak

\hora{Ad Laudes.} %%%%%%%%%%%%%%%%%%%%%%%%%%%%%%%%%%%%%%%%%%%%%%%%%%%%%

\cantusSineNeumas

\vspace{0.5cm}
\grechangedim{interwordspacetext}{0.18 cm plus 0.15 cm minus 0.05 cm}{scalable}%
\cuminitiali{}{temporalia/deusinadiutorium-communis.gtex}
\grechangedim{interwordspacetext}{0.22 cm plus 0.15 cm minus 0.05 cm}{scalable}%

\vfill
\pagebreak

\ifx\hymnuslaudes\undefined
\ifx\laudac\undefined
\else
\pars{Hymnus} \scriptura{Ambrosius (\olddag{} 397)}

\cuminitiali{I}{temporalia/hym-SplendorPaternae-hiemalis.gtex}
\fi
\ifx\laudbd\undefined
\else
\pars{Hymnus}

\grechangedim{interwordspacetext}{0.16 cm plus 0.15 cm minus 0.05 cm}{scalable}%
\cuminitiali{IV}{temporalia/hym-AEterneLucis.gtex}
\grechangedim{interwordspacetext}{0.22 cm plus 0.15 cm minus 0.05 cm}{scalable}%
\vspace{-3mm}
\fi
\else
\hymnuslaudes
\fi

\vfill
\pagebreak

\ifx\laudb\undefined
\else
\pars{Psalmus 1.} \scriptura{Ps. 42, 5; \textbf{H95}}

\vspace{-4mm}

\antiphona{VI F}{temporalia/ant-salutarevultusmei.gtex}

\scriptura{Psalmus 42.}

\initiumpsalmi{temporalia/ps42-initium-vi-F-auto.gtex}

\input{temporalia/ps42-vi-F.tex} \Abardot{}

\vfill
\pagebreak

\pars{Psalmus 2.} \scriptura{Is. 38, 20; \textbf{H95}}

\vspace{-7mm}

\antiphona{E}{temporalia/ant-cunctisdiebus.gtex}

\vspace{-4mm}

\scriptura{Canticum Ezechiæ, Is. 38, 10-20}

\vspace{-3mm}

\initiumpsalmi{temporalia/ezechiae-initium-e-auto.gtex}

\input{temporalia/ezechiae-e.tex} \Abardot{}

\vfill
\pagebreak

\pars{Psalmus 3.} \scriptura{Ps. 64, 2; \textbf{H96}}

\vspace{-4mm}

\antiphona{VIII a}{temporalia/ant-tedecet.gtex}

\vspace{-2mm}

\scriptura{Psalmus 64.}

\vspace{-2mm}

\initiumpsalmi{temporalia/ps64-initium-viii-A-auto.gtex}

\input{temporalia/ps64-viii-A.tex} \Abardot{}

\vfill
\pagebreak
\fi
\ifx\laudc\undefined
\else
\pars{Psalmus 1.} \scriptura{Ps. 83, 5}

\vspace{-4mm}

\antiphona{VIII G}{temporalia/ant-beatiquihabitant.gtex}

\vspace{-2mm}

\scriptura{Psalmus 84.}

\vspace{-2mm}

\initiumpsalmi{temporalia/ps84-initium-viii-G-auto.gtex}

\input{temporalia/ps84-viii-G.tex} \Abardot{}

\vfill
\pagebreak

\pars{Psalmus 2.}

\vspace{-4mm}

\antiphona{VII d}{temporalia/ant-denoctespiritusmeus.gtex}

\vspace{-2mm}

\scriptura{Canticum Isaiæ, Is. 26, 1-12}

\vspace{-2mm}

\initiumpsalmi{temporalia/isaiae3-initium-vii-d.gtex}

\input{temporalia/isaiae3-vii-d.tex} \Abardot{}

\vfill
\pagebreak

\pars{Psalmus 3.} \scriptura{Ps. 66, 2}

\vspace{-4mm}

\antiphona{E}{temporalia/ant-illuminadomine.gtex}

%\vspace{-2mm}

\scriptura{Psalmus 66.}

%\vspace{-2mm}

\initiumpsalmi{temporalia/ps66-initium-e.gtex}

\input{temporalia/ps66-e.tex} \Abardot{}

\vfill
\pagebreak
\fi

\ifx\lectiobrevis\undefined
\ifx\laudb\undefined
\else
\pars{Lectio Brevis.} \scriptura{1 Th. 5, 4-5}

\noindent Vos, fratres, non estis in ténebris, ut vos dies ille tamquam fur comprehéndat; omnes enim vos fílii lucis estis et fílii diéi. Non sumus noctis neque tenebrárum.
\fi
\ifx\laudc\undefined
\else
\pars{Lectio Brevis.} \scriptura{1 Io. 4, 14-15}

\noindent Nos vídimus et testificámur quóniam Pater misit Fílium salvatórem mundi. Quisque conféssus fúerit: Iesus est Fílius Dei, Deus in ipso manet, et ipse in Deo.
\fi
\else
\lectiobrevis
\fi

\vfill

\ifx\responsoriumbreve\undefined
\ifx\laudac\undefined
\else
\pars{Responsorium breve.}

\cuminitiali{VI}{temporalia/resp-benedictusdominus.gtex}
\fi
\ifx\laudbd\undefined
\else
\pars{Responsorium breve.} \scriptura{Ps. 118, 149.147}

\cuminitiali{VI}{temporalia/resp-vocemmeamaudi.gtex}
\fi
\else
\responsoriumbreve
\fi

\vfill
\pagebreak

\ifx\benedictus\undefined
\ifx\laudbd\undefined
\else
\pars{Canticum Zachariæ.} \scriptura{Lc. 1, 71; \textbf{H423}}

\vspace{-5mm}

{
\grechangedim{interwordspacetext}{0.18 cm plus 0.15 cm minus 0.05 cm}{scalable}%
\antiphona{I g\textsuperscript{5}}{temporalia/ant-demanuomnium.gtex}
\grechangedim{interwordspacetext}{0.22 cm plus 0.15 cm minus 0.05 cm}{scalable}%
}

%\vspace{-3mm}

\scriptura{Lc. 1, 68-79}

%\vspace{-1mm}

\initiumpsalmi{temporalia/benedictus-initium-i-g5-auto.gtex}

\input{temporalia/benedictus-i-g5.tex} \Abardot{}
\fi
\else
\benedictus
\fi

\vspace{-1cm}

\vfill
\pagebreak

\pars{Preces.}

\sineinitiali{}{temporalia/tonusprecum.gtex}

\ifx\preces\undefined
\ifx\laudb\undefined
\else
\noindent Salvatóri nostro benedicámus, qui sua resurrectióne mundum clarificávit, \gredagger{} et humíliter invocémus eum dicéntes:

\Rbardot{} Salva nos, Dómine, in sémita tua.

\noindent Resurrectiónem tuam, Dómine Iesu, oratióne cólimus matutína, \gredagger{} spes glóriæ tuæ diem nostrum illúminet.

\Rbardot{} Salva nos, Dómine, in sémita tua.

\noindent Súscipe, Dómine, vota et propósita nostra, \gredagger{} tamquam diéi nostri primítias.

\Rbardot{} Salva nos, Dómine, in sémita tua.

\noindent Tríbue in dilectióne tua nos hódie profícere, \gredagger{} ut ómnia in nostrum omniúmque bonum cooperéntur.

\Rbardot{} Salva nos, Dómine, in sémita tua.

\noindent Da, Dómine, sic lucére lucem nostram coram homínibus, \gredagger{} ut vídeant ópera nostra bona et Patrem gloríficent.

\Rbardot{} Salva nos, Dómine, in sémita tua.
\fi
\else
\preces
\fi

\vfill

\pars{Oratio Dominica.}

\cuminitiali{}{temporalia/oratiodominicaalt.gtex}

\vfill
\pagebreak

\rubrica{vel:}

\pars{Supplicatio Litaniæ.}

\cuminitiali{}{temporalia/supplicatiolitaniae.gtex}

\vfill

\pars{Oratio Dominica.}

\cuminitiali{}{temporalia/oratiodominica.gtex}

\vfill
\pagebreak

% Oratio. %%%
\oratio

\vspace{-1mm}

\vfill

\rubrica{Hebdomadarius dicit Dominus vobiscum, vel, absente sacerdote vel diacono, sic concluditur:}

\vspace{2mm}

\antiphona{C}{temporalia/dominusnosbenedicat.gtex}

\rubrica{Postea cantatur a cantore:}

\vspace{2mm}

\cuminitiali{IV}{temporalia/benedicamus-feria-laudes.gtex}

\vspace{1mm}

\vfill
\pagebreak

\end{document}

