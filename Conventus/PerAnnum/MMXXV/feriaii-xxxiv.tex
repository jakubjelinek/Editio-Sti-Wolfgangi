\newcommand{\titulus}{\nomenFesti{S. Andreæ Dũng Lac, Presbyteri \& Sociorum Martyrum.}
\dies{Die 24. Novembris.}}
\newcommand{\oratio}{\pars{Oratio.}

\noindent Deus, omnis paternitátis fons et orígo, qui beátos mártyres Andréam et sócios eius Cruci Fílii tui usque ad sánguinis effusiónem fidéles effecísti, eórum intercessióne concéde, ut amórem tuum inter fratres propagántes fílii tui nominári et esse valeámus.

\pars{Pro pace in universo mundo.} \scriptura{Sir. 50, 25; 2 Esdr. 4, 20; \textbf{H416}}

\vspace{-4mm}

\antiphona{II D}{temporalia/ant-dapacemdomine.gtex}

\vfill

\noindent Deus, a quo sancta desidéria, recta consília et iusta sunt ópera: da servis tuis illam, quam mundus dare non potest, pacem; ut et corda nostra mandátis tuis dédita, et hóstium subláta formídine, témpora sint tua protectióne tranquílla.

\noindent Per Dóminum nostrum Iesum Christum, Fílium tuum, qui tecum vivit et regnat in unitáte Spíritus Sancti, Deus, per ómnia sǽcula sæculórum.

\noindent \Rbardot{} Amen.}
\newcommand{\invitatorium}{\pars{Invitatorium.}

\vspace{-4mm}

\antiphona{E}{temporalia/inv-regemmartyrumsimplex.gtex}}
\newcommand{\hymnusmatutinum}{\pars{Hymnus}

\cuminitiali{IV}{temporalia/hym-RexGloriose.gtex}}
\newcommand{\lectioi}{\pars{Lectio I.} \scriptura{1 Petr. 1, 1-11}

\noindent Incipit Epístola secúnda beáti Petri apóstoli.

\noindent Simon Petrus servus et apóstolus Iesu Christi his, qui coæquálem nobis sortíti sunt fidem in iustítia Dei nostri et salvatóris Iesu Christi: grátia vobis et pax multiplicétur in cognitióne Dei et Iesu Dómini nostri.

\noindent Quómodo ómnia nobis divínæ virtútis suæ ad vitam et pietátem donátæ per cognitiónem eius, qui vocávit nos própria glória et virtúte, per quæ pretiósa et máxima nobis promíssa donáta sunt, ut per hæc efficiámini divínæ consórtes natúræ, fugiéntes eam, quæ in mundo est in concupiscéntia, corruptiónem; et propter hoc ipsum curam omnem subinferéntes ministráte in fide vestra virtútem, in virtúte autem sciéntiam, in sciéntia autem continéntiam, in continéntia autem patiéntiam, in patiéntia autem pietátem, in pietáte autem amórem fraternitátis, in amóre autem fraternitátis caritátem.

\noindent Hæc enim vobis, cum adsint et abúndent, non vácuos nec sine fructu vos constítuunt in Dómini nostri Iesu Christi cognitiónem; cui enim non præsto sunt hæc, cæcus est et nihil procul cernens, obliviónem accípiens purgatiónis véterum suórum delictórum.

\noindent Quaprópter, fratres, magis satágite, ut firmam vestram vocatiónem et electiónem faciátis. Hæc enim faciéntes non offendétis aliquándo; sic enim abundánter ministrábitur vobis intróitus in ætérnum regnum Dómini nostri et salvatóris Iesu Christi.}
\newcommand{\responsoriumi}{\pars{Responsorium 1.} \scriptura{\Rbardot{} Michææ 6, 8 \Vbardot{} Ps. 36, 6; \textbf{H418}}

\vspace{-5mm}

\responsorium{V}{temporalia/resp-indicabotibihomo-CROCHU.gtex}{}}
\newcommand{\lectioii}{\pars{Lectio II.} \scriptura{A. Launay, Le clergé tonkinois et ses prêtres martyrs, MEP, Paris 1925, pp. 80-83}

\noindent E lítteris sancti Pauli Le-Bao-Tinh {\color{gray} alúmnis Seminárii Ke-Vinh anno 1843 expedítis.}

\noindent Ego, Paulus, pro nómine Christi vinctus, tribulatiónes meas vobis reférre volo quibus cotídie immérsus sum, ita ut, amóre erga Deum accénsi laudes mecum Deo præbeátis, \emph{quóniam in ætérnum misericórdia eius.} Hic carcer vere imágo est inférni ætérni: ad supplícia crudélia omnis géneris, ut sunt cómpedes, caténæ férreæ et víncula, addúntur ódium, vindíctæ, calúmniæ, verba indecéntia, querélæ, actus mali, iuraménta iniústa, maledictiónes et tandem angústiæ et tristítia. Deus autem qui olim liberávit tres púeros de camíno ignis, mihi semper adest meque ab istis tribulatiónibus liberávit et eas in dulcédinem convértit, \emph{quóniam in ætérnum misericórdia eius.}

\noindent In médio autem horum tormentórum, quæ álios contérrere solent, grátia Dei, gáudio replétus sum et lætítia, quia non solus sed cum Christo sum.

\noindent Ipse magíster noster totum pondus crucis sústinet, mihi mínimam tantum et últimam partem impónens. Certáminis mei non solum spectátor, sed ipse est bellátor et victor totiúsque agónis consummátor. Proptérea super caput eius pósita est coróna victóriæ, cuius glóriam partícipant étiam eius membra.

\noindent Quómodo autem sustíneam spectáculum istud, videns cotídie imperatóres, mandarínos eorúmque satéllites blasphemántes nomen sanctum tuum, Dómine, \emph{qui sedes super Chérubim}et Séraphim? Ecce, crux tua a pédibus paganórum conculcáta est! Ubi est glória tua? Videns hæc ómnia, malo, amóre tui succénsus, abscíssis membris, mori in testimónium amóris tui.

\noindent Osténde, Dómine, poténtiam tuam, salva me et sústine me, ut \emph{virtus in infirmitáte} mea ostendátur et glorificétur coram géntibus, ne, cum vacíllem forte in via, inimíci tui in supérbia sua caput possint eleváre.}
\newcommand{\responsoriumii}{\pars{Responsorium 2.} \scriptura{\Rbardot{} Ps. 149, 2 \Vbardot{} Ps. 67, 4; \textbf{H369}}

\vspace{-5mm}

\responsorium{VII}{temporalia/resp-exsultabuntsancti-CROCHU.gtex}{}}
\newcommand{\lectioiii}{\pars{Lectio III.}

\noindent Fratres caríssimi, audiéntes hæc ómnia, grátias agátis immortáles in lætítia Deo, a quo bona cuncta procédunt, benedícite Dómino mecum, \emph{quóniam in ætérnum misericórdia eius!} Magníficet ánima mea Dóminum et exsúltet spíritus meus in Deo meo, quóniam respéxit humilitátem fámuli sui et ex hoc beátum me dicent omnes generatiónes futúræ: \emph{quóniam in ætérnum misericórdia eius.}

\noindent \emph{Laudáte Dóminum omnes gentes, collaudáte eum omnes pópuli,} quóniam quæ \emph{infírma sunt mundi, elégit Deus ut confúndat fórtia et ignobília mundi et contemptibília elégit Deus} ut nobília confúndat. Per os meum atque intelléctum meum confúdit philósophos qui discípuli sunt sapiéntum huius mundi, \emph{quóniam in ætérnum misericórdia eius.}

\noindent Scribo vobis hæc ómnia, ut uniántur fides vestra et mea. In médio huius tempestátis áncoram iácio usque ad thronum Dei; spem vivam, quæ est in corde meo.

\noindent Vos autem, fratres caríssimi, \emph{sic cúrrite ut corónam comprehendátis,} indúite \emph{lorícam fídei} et \emph{arma} Christi súmite \emph{a dextris et a sinístris,} sicut dócuit sanctus Paulus, patrónus meus. \emph{Bonum vobis est, unóculos vel débiles in vitam intráre,} quam ómnia membra habéntes foris mitti.

\noindent Succúrrite mihi précibus vestris, ut secúndum legem certáre, et quidem \emph{bonum certámen certáre} et usque in finem certáre váleam, cursum meum felíciter consummatúrus; si in hac vita non iam nos vidébimus, in futúro tamen sǽculo hæc felícitas nostra erit, quando astántes ad thronum Agni immaculáti, unánimes laudes eius cantábimus exsultántes in gáudio victóriæ in perpétuum. Amen.}
\newcommand{\responsoriumiii}{\pars{Responsorium 3.} \scriptura{Cf. Ap. 14, 4; \textbf{H68}}

\vspace{-5mm}

\responsorium{III}{temporalia/resp-coronaviteos-CROCHU-cumdox.gtex}{}}
\newcommand{\hymnuslaudes}{\pars{Hymnus}

\cuminitiali{VIII}{temporalia/hym-AEternaChristi.gtex}}
\newcommand{\lectiobrevis}{\pars{Lectio Brevis.} \scriptura{2 Cor. 1, 3-5}

\noindent Benedíctus Deus et Pater Dómini nostri Iesu Christi, Pater misericordiárum et Deus totíus consolatiónis, qui consolátur nos in omni tribulatióne nostra, ut possímus et ipsi consolári eos, qui in omni pressúra sunt, per exhortatiónem, qua exhortámur et ipsi a Deo; quóniam, sicut abúndant passiónes Christi in nobis, ita per Christum abúndat et consolátio nostra.}
\newcommand{\responsoriumbreve}{\pars{Responsorium breve.} \scriptura{Sap. 5, 16}

\antiphona{VI}{temporalia/resp-iustiautem.gtex}}
\newcommand{\benedictus}{\pars{Canticum Zachariæ.} \scriptura{Ap. 21, 4}

\vspace{-4mm}

\antiphona{I f}{temporalia/ant-abstergetdeus.gtex}

\vspace{-2mm}

\scriptura{Lc. 1, 68-79}

\vspace{-2mm}

\cantusSineNeumas
\initiumpsalmi{temporalia/benedictus-initium-i-f-auto.gtex}

%\vspace{-1.5mm}

\input{temporalia/benedictus-i-f.tex} \Abardot{}}
\newcommand{\preces}{\noindent Fratres, Salvatórem nostrum, testem fidélem, per mártyres interféctos propter verbum Dei, celebrémus,~\grestar{} clamántes:

\Rbardot{} Redemísti nos Deo in sánguine tuo.

\noindent Per mártyres tuos, qui líbere mortem in testimónium fídei sunt ampléxi,~\grestar{} da nobis, Dómine, veram spíritus libertátem.

\Rbardot{} Redemísti nos Deo in sánguine tuo.

\noindent Per mártyres tuos, qui fidem usque ad sánguinem sunt conféssi,~\grestar{} da nobis, Dómine, puritátem fideíque constántiam.

\Rbardot{} Redemísti nos Deo in sánguine tuo.

\noindent Per mártyres tuos, qui, sustinéntes crucem, tua vestígia sunt secúti,~\grestar{} da nobis, Dómine, ærúmnas vitæ fórtiter sustinére.

\Rbardot{} Redemísti nos Deo in sánguine tuo.

\noindent Per mártyres tuos, qui stolas suas lavérunt in sánguine Agni,~\grestar{} da nobis, Dómine, omnes insídias carnis mundíque devíncere.

\Rbardot{} Redemísti nos Deo in sánguine tuo.}
\newcommand{\benedicamuslaudes}{\cuminitiali{}{temporalia/benedicamus-memoria-laudes.gtex}}
\newcommand{\hebdomada}{infra Hebdom. XXIV per Annum.}
\newcommand{\hiemalis}{Hiemalis}
\newcommand{\matub}{Matutinum Hebdomadae B}
\newcommand{\matubd}{Matutinum Hebdomadae B vel D}
\newcommand{\laudb}{Laudes Hebdomadae B}
\newcommand{\laudbd}{Laudes Hebdomadae B vel D}

% LuaLaTeX

\documentclass[a4paper, twoside, 12pt]{article}
\usepackage[latin]{babel}
%\usepackage[landscape, left=3cm, right=1.5cm, top=2cm, bottom=1cm]{geometry} % okraje stranky
%\usepackage[landscape, a4paper, mag=1166, truedimen, left=2cm, right=1.5cm, top=1.6cm, bottom=0.95cm]{geometry} % okraje stranky
\usepackage[landscape, a4paper, mag=1400, truedimen, left=0.5cm, right=0.5cm, top=0.5cm, bottom=0.5cm]{geometry} % okraje stranky

\usepackage{fontspec}
\setmainfont[FeatureFile={junicode.fea}, Ligatures={Common, TeX}, RawFeature=+fixi]{Junicode}
%\setmainfont{Junicode}

% shortcut for Junicode without ligatures (for the Czech texts)
\newfontfamily\nlfont[FeatureFile={junicode.fea}, Ligatures={Common, TeX}, RawFeature=+fixi]{Junicode}

\usepackage{multicol}
\usepackage{color}
\usepackage{lettrine}
\usepackage{fancyhdr}

% usual packages loading:
\usepackage{luatextra}
\usepackage{graphicx} % support the \includegraphics command and options
\usepackage{gregoriotex} % for gregorio score inclusion
\usepackage{gregoriosyms}
\usepackage{wrapfig} % figures wrapped by the text
\usepackage{parcolumns}
\usepackage[contents={},opacity=1,scale=1,color=black]{background}
\usepackage{tikzpagenodes}
\usepackage{calc}
\usepackage{longtable}
\usetikzlibrary{calc}

\setlength{\headheight}{14.5pt}

\input{conventuscommune.tex} % Often used macros

\newcommand{\annusEditionis}{2021}

%%%% Vicekrat opakovane kousky

\newcommand{\anteOrationem}{
  \rubrica{Ante Orationem, cantatur a Superiore:}

  \pars{Supplicatio Litaniæ.}

  \cuminitiali{}{temporalia/supplicatiolitaniae.gtex}

  \pars{Oratio Dominica.}

  \cuminitiali{}{temporalia/oratiodominica.gtex}

  \rubrica{Deinde dicitur ab Hebdomadario:}

  \cuminitiali{}{temporalia/dominusvobiscum-solemnis.gtex}

  \rubrica{In choro monialium loco Dominus vobiscum dicitur:}

  \sineinitiali{temporalia/domineexaudi.gtex}
}

\setlength{\columnsep}{30pt} % prostor mezi sloupci

%%%%%%%%%%%%%%%%%%%%%%%%%%%%%%%%%%%%%%%%%%%%%%%%%%%%%%%%%%%%%%%%%%%%%%%%%%%%%%%%%%%%%%%%%%%%%%%%%%%%%%%%%%%%%
\begin{document}

% Here we set the space around the initial.
% Please report to http://home.gna.org/gregorio/gregoriotex/details for more details and options
\grechangedim{afterinitialshift}{2.2mm}{scalable}
\grechangedim{beforeinitialshift}{2.2mm}{scalable}
\grechangedim{interwordspacetext}{0.22 cm plus 0.15 cm minus 0.05 cm}{scalable}%
\grechangedim{annotationraise}{-0.2cm}{scalable}

% Here we set the initial font. Change 38 if you want a bigger initial.
% Emit the initials in red.
\grechangestyle{initial}{\color{red}\fontsize{38}{38}\selectfont}

\pagestyle{empty}

%%%% Titulni stranka
\begin{titulusOfficii}
\ifx\titulus\undefined
\nomenFesti{Feria II \hebdomada{}}
\else
\titulus
\fi
\end{titulusOfficii}

\vfill

\begin{center}
%Ad usum et secundum consuetudines chori \guillemotright{}Conventus Choralis\guillemotleft.

%Editio Sancti Wolfgangi \annusEditionis
\end{center}

\scriptura{}

\pars{}

\pagebreak

\renewcommand{\headrulewidth}{0pt} % no horiz. rule at the header
\fancyhf{}
\pagestyle{fancy}

\cantusSineNeumas

\ifx\oratio\undefined
\ifx\laudb\undefined
\else
\newcommand{\oratio}{\pars{Oratio.}

\noindent Dómine Deus omnípotens, qui ad princípium huius diéi nos perveníre fecísti, tua nos hódie salva virtúte, ut in hac die ad nullum declinémus peccátum, sed semper ad tuam iustítiam faciéndam nostra procédant elóquia, dirigántur cogitatiónes et ópera.

\noindent Per Dóminum nostrum Iesum Christum, Fílium tuum, qui tecum vivit et regnat in unitáte Spíritus Sancti, Deus, per ómnia sǽcula sæculórum.

\noindent \Rbardot{} Amen.}
\fi
\fi

\hora{Ad Matutinum.} %%%%%%%%%%%%%%%%%%%%%%%%%%%%%%%%%%%%%%%%%%%%%%%%%%%%%
%\sideThumbs{Matutinum}

\vspace{2mm}

\cuminitiali{}{temporalia/dominelabiamea.gtex}

\vfill
%\pagebreak

\vspace{2mm}

\ifx\invitatorium\undefined
\pars{Invitatorium.} \scriptura{Ps. 94, 1; Psalmus 94; \textbf{H451}}

\vspace{-6mm}

\antiphona{VI}{temporalia/inv-jubilemusdeo.gtex}\else
\invitatorium
\fi

\vfill
\pagebreak

\ifx\hymnusmatutinum\undefined
\ifx\matua\undefined
\else
\pars{Hymnus.}

{
\grechangedim{interwordspacetext}{0.10 cm plus 0.15 cm minus 0.05 cm}{scalable}%
\antiphona{II}{temporalia/hym-IpsumNunc.gtex}
\grechangedim{interwordspacetext}{0.22 cm plus 0.15 cm minus 0.05 cm}{scalable}%
}
\fi
\else
\hymnusmatutinum
\fi

\vspace{-3mm}

\vfill
\pagebreak

\ifx\matub\undefined
\else
% MAT B
\pars{Psalmus 1.} \scriptura{Ps. 30, 2; \textbf{H90}}

\vspace{-4mm}

\antiphona{VIII G}{temporalia/ant-intuaiustitia.gtex}

%\vspace{-2mm}

\scriptura{Ps. 30, 2-9}

%\vspace{-2mm}

\initiumpsalmi{temporalia/ps30i-initium-viii-G-auto.gtex}

\vspace{-1.5mm}

\input{temporalia/ps30i-viii-G.tex} \Abardot{}

\vfill
\pagebreak

\pars{Psalmus 2.} \scriptura{Ps. 66, 2}

\vspace{-4mm}

\antiphona{E}{temporalia/ant-illuminadomine.gtex}

%\vspace{-2mm}

\scriptura{Ps. 30, 10-17}

%\vspace{-2mm}

\initiumpsalmi{temporalia/ps30ii-initium-e-a-auto.gtex}

\input{temporalia/ps30ii-e-a.tex} \Abardot{}

\vfill
\pagebreak

\pars{Psalmus 3.} \scriptura{Ps. 30, 24}

\vspace{-4mm}

\antiphona{II D}{temporalia/ant-diligitedominum.gtex}

%\vspace{-5mm}

\scriptura{Ps. 30, 20-25}

%\vspace{-2mm}

\initiumpsalmi{temporalia/ps30iii-initium-ii-D-auto.gtex}

\input{temporalia/ps30iii-ii-D.tex} \Abardot{}

\vfill
\pagebreak
\fi

\pars{Versus.}

\ifx\matversus\undefined
\ifx\matub\undefined
\else
\noindent \Vbardot{} Dírige me, Dómine, in veritáte tua, et doce me.

\noindent \Rbardot{} Quia tu es Deus salútis meæ.
\fi
\else
\matversus
\fi

\vspace{5mm}

\sineinitiali{temporalia/oratiodominica-mat.gtex}

\vspace{5mm}

\pars{Absolutio.}

\cuminitiali{}{temporalia/absolutio-exaudi.gtex}

\vfill
\pagebreak

\cuminitiali{}{temporalia/benedictio-solemn-benedictione.gtex}

\vspace{7mm}

\lectioi

\noindent \Vbardot{} Tu autem, Dómine, miserére nobis.
\noindent \Rbardot{} Deo grátias.

\vfill
\pagebreak

\responsoriumi

\vfill
\pagebreak

\cuminitiali{}{temporalia/benedictio-solemn-unigenitus.gtex}

\vspace{7mm}

\lectioii

\noindent \Vbardot{} Tu autem, Dómine, miserére nobis.
\noindent \Rbardot{} Deo grátias.

\vfill
\pagebreak

\responsoriumii

\vfill
\pagebreak

\cuminitiali{}{temporalia/benedictio-solemn-spiritus.gtex}

\vspace{7mm}

\lectioiii

\noindent \Vbardot{} Tu autem, Dómine, miserére nobis.
\noindent \Rbardot{} Deo grátias.

\vfill
\pagebreak

\responsoriumiii

\vfill
\pagebreak

\rubrica{Reliqua omittuntur, nisi Laudes separandæ sint.}

\sineinitiali{temporalia/domineexaudi.gtex}

\vfill

\oratio

\vfill

\noindent \Vbardot{} Dómine, exáudi oratiónem meam.
\Rbardot{} Et clamor meus ad te véniat.

\vfill

\noindent \Vbardot{} Benedicámus Dómino.
\noindent \Rbardot{} Deo grátias.

\vfill

\noindent \Vbardot{} Fidélium ánimæ per misericórdiam Dei requiéscant in pace.
\Rbardot{} Amen.

\vfill
\pagebreak

\hora{Ad Laudes.} %%%%%%%%%%%%%%%%%%%%%%%%%%%%%%%%%%%%%%%%%%%%%%%%%%%%%
%\sideThumbs{Laudes}

\cantusSineNeumas

\vspace{0.5cm}
\grechangedim{interwordspacetext}{0.18 cm plus 0.15 cm minus 0.05 cm}{scalable}%
\cuminitiali{}{temporalia/deusinadiutorium-communis.gtex}
\grechangedim{interwordspacetext}{0.22 cm plus 0.15 cm minus 0.05 cm}{scalable}%

\vfill
\pagebreak

\ifx\hymnuslaudes\undefined
\ifx\laudbd\undefined
\else
\pars{Hymnus} \scriptura{Hilarius (\olddag{} 367)}

\grechangedim{interwordspacetext}{0.16 cm plus 0.15 cm minus 0.05 cm}{scalable}%
\cuminitiali{IV}{temporalia/hym-LucisLargitor.gtex}
\grechangedim{interwordspacetext}{0.22 cm plus 0.15 cm minus 0.05 cm}{scalable}%
\vspace{-3mm}
\fi
\else
\hymnuslaudes
\fi

\vfill
\pagebreak

\ifx\laudb\undefined
\else
\pars{Psalmus 1.} \scriptura{Ps. 41, 3; \textbf{H391}}

\vspace{-4mm}

\antiphona{II D}{temporalia/ant-sitivitanima.gtex}

%\vspace{-2mm}

\scriptura{Psalmus 41}

%\vspace{-2mm}

\initiumpsalmi{temporalia/ps41-initium-ii-D-auto.gtex}

%\vspace{-1.5mm}

\input{temporalia/ps41-ii-D.tex}

\vfill

\antiphona{}{temporalia/ant-sitivitanima.gtex}

\vfill
\pagebreak

\pars{Psalmus 2.}

\vspace{-4mm}

\antiphona{III a}{temporalia/ant-ostendenobisdomine.gtex}

%\vspace{-2mm}

\scriptura{Canticum Ecclesiastici, Sir. 36, 1-7.13-16}

%\vspace{-3mm}

\initiumpsalmi{temporalia/ecclesiastici-initium-iii-a-auto.gtex}

\input{temporalia/ecclesiastici-iii-a.tex} \Abardot{}

\vfill
\pagebreak

\pars{Psalmus 3.}

\vspace{-4mm}

\antiphona{II D}{temporalia/ant-operamanuumeius.gtex}

\scriptura{Psalmus 18, 1-7}

\initiumpsalmi{temporalia/ps18i-initium-ii-D-auto.gtex}

\input{temporalia/ps18i-ii-D.tex} \Abardot{}

\vfill
\pagebreak
\fi

\ifx\lectiobrevis\undefined
\ifx\laudb\undefined
\else
\pars{Lectio Brevis.} \scriptura{Ier. 15, 16}

\noindent Invénti sunt sermónes tui, et comédi eos, et factum est mihi verbum tuum in gáudium et in lætítiam cordis mei, quóniam invocátum est nomen tuum super me, Dómine Deus exercítuum.
\fi
\else
\lectiobrevis
\fi

\vfill

\ifx\responsoriumbreve\undefined
\ifx\laudbd\undefined
\else
\pars{Responsorium breve.} \scriptura{Ps. 32, 1.3}

\cuminitiali{VI}{temporalia/resp-exsultateiusti.gtex}
\fi
\else
\responsoriumbreve
\fi

\vfill
\pagebreak

\ifx\benedictus\undefined
\ifx\laudbd\undefined
\else
\pars{Canticum Zachariæ.} \scriptura{Lc. 1, 68; \textbf{H422}}

\vspace{-4mm}

{
\grechangedim{interwordspacetext}{0.18 cm plus 0.15 cm minus 0.05 cm}{scalable}%
\antiphona{IV E}{temporalia/ant-benedictusdominus.gtex}
\grechangedim{interwordspacetext}{0.22 cm plus 0.15 cm minus 0.05 cm}{scalable}%
}

%\vspace{-3mm}

\scriptura{Lc. 1, 68-79}

%\vspace{-2mm}

\cantusSineNeumas
\initiumpsalmi{temporalia/benedictus-initium-iv-E-auto.gtex}

%\vspace{-1.5mm}

\input{temporalia/benedictus-iv-E.tex} \Abardot{}
\fi
\else
\benedictus
\fi

\vspace{-1cm}

\vfill
\pagebreak

%\sideThumbs{{\scriptsize{}Fine horarum}}

\pars{Preces.}

\sineinitiali{}{temporalia/tonusprecum.gtex}

\ifx\preces\undefined
\ifx\laudb\undefined
\else
\noindent Salvátor noster fecit nos regnum et sacerdótium, ut hóstias Deo acceptábiles offerámus. \gredagger{} Grati ígitur eum invocémus:

\Rbardot{} Serva nos in tuo ministério, Dómine.

\noindent Christe, sacérdos ætérne, qui sanctum pópulo tuo sacerdótium concessísti, \gredagger{} concéde, ut spiritáles hóstias Deo acceptábiles iúgiter offerámus.

\Rbardot{} Serva nos in tuo ministério, Dómine.

\noindent Spíritus tui fructus nobis largíre propítius, \gredagger{} patiéntiam, benignitátem et mansuetúdinem.

\Rbardot{} Serva nos in tuo ministério, Dómine.

\noindent Da nobis te amáre, ut te, qui es cáritas, possideámus, \gredagger{} et bene ágere, ut per vitam étiam nostram te laudémus.

\Rbardot{} Serva nos in tuo ministério, Dómine.

\noindent Quæ frátribus nostris sunt utília, nos quǽrere concéde, \gredagger{} ut salútem facílius consequántur.

\Rbardot{} Serva nos in tuo ministério, Dómine.
\fi
\else
\preces
\fi

\vfill

\pars{Oratio Dominica.}

\cuminitiali{}{temporalia/oratiodominicaalt.gtex}

\vfill
\pagebreak

\rubrica{vel:}

\pars{Supplicatio Litaniæ.}

\cuminitiali{}{temporalia/supplicatiolitaniae.gtex}

\vfill

\pars{Oratio Dominica.}

\cuminitiali{}{temporalia/oratiodominica.gtex}

\vfill
\pagebreak

% Oratio. %%%
\oratio

\vspace{-1mm}

\vfill

\rubrica{Hebdomadarius dicit Dominus vobiscum, vel, absente sacerdote vel diacono, sic concluditur:}

\vspace{2mm}

\antiphona{C}{temporalia/dominusnosbenedicat.gtex}

\rubrica{Postea cantatur a cantore:}

\vspace{2mm}

\cuminitiali{IV}{temporalia/benedicamus-feria-laudes.gtex}

\vspace{1mm}

\vfill
\pagebreak

\end{document}

