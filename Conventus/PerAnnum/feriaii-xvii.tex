\newcommand{\titulus}{\nomenFesti{Ss. Ioachim \& Annæ, parentum B. M. V.}
\dies{Die 26. Iulii.}}
\newcommand{\oratio}{\pars{Oratio.}

\noindent Dómine, Deus patrum nostrórum, qui beátis Ióachim et Annæ hanc grátiam contulísti, ut ex eis incarnáti Fílii tui Mater nascerétur, utriúsque précibus concéde, ut salútem tuo promíssam pópulo consequámur.

\noindent Per Dóminum nostrum Iesum Christum, Fílium tuum, qui tecum vivit et regnat in unitáte Spíritus Sancti, Deus, per ómnia sǽcula sæculórum.

\noindent \Rbardot{} Amen.}
\newcommand{\invitatorium}{\pars{Invitatorium.}

\vspace{-4mm}

\antiphona{IV}{temporalia/inv-mirabileminsanctis.gtex}}
\newcommand{\hymnusmatutinum}{\pars{Hymnus}

\cuminitiali{IV}{temporalia/hym-DumTuas.gtex}}
\newcommand{\matversus}{\noindent \Vbardot{} Spécie tua et pulchritúdine tua.

\noindent \Rbardot{} Inténde, próspere, procéde, et regna.}
\newcommand{\lectioi}{\pars{Lectio I.} \scriptura{Gal. 1, 13-24}

\noindent De Epístola beáti Pauli apóstoli ad Gálatas.

\noindent Fratres: Audístis enim conversatiónem meam aliquándo in Iudaísmo quóniam supra modum persequébar ecclésiam Dei et expugnábam illam et proficiébam in Iudaísmo supra multos coætáneos in génere meo, abundántius æmulátor exsístens paternárum meárum traditiónum. Cum autem plácuit Deo, qui me segregávit de útero matris meæ et vocávit per grátiam suam, ut reveláret Fílium suum in me, ut evangelizárem illum in géntibus, contínuo non cóntuli cum carne et sánguine neque ascéndi Hierosólymam ad antecessóres meos apóstolos, sed ábii in Arábiam et íterum revérsus sum Damáscum.

\noindent Deínde post annos tres ascéndi Hierosólymam vidére Cepham et mansi apud eum diébus quíndecim; álium autem apostolórum non vidi nisi Iacóbum fratrem Dómini. Quæ autem scribo vobis, ecce coram Deo quia non méntior. Deínde veni in partes Sýriæ et Cilíciæ. Eram autem ignótus fácie ecclésiis Iudǽæ, quæ sunt in Christo, tantum autem audítum habébant: «Qui persequebátur nos aliquándo, nunc evangelízat fidem, quam aliquándo expugnábat», et in me glorificábant Deum.}
\newcommand{\responsoriumi}{\pars{Responsorium 1.} \scriptura{\Rbardot{} Ps. 33, 2 \Vbardot{} ibid., 3; \textbf{H85}}

\vspace{-5mm}

\responsorium{V}{temporalia/resp-benedicamdominum-CROCHU.gtex}{}}
\newcommand{\lectioii}{\pars{Lectio II.} \scriptura{Gal. 2, 1-10}

\noindent Deínde post annos quattuórdecim íterum ascéndi Hierosólymam cum Bárnaba, assúmpto et Tito; ascéndi autem secúndum revelatiónem; et cóntuli cum illis evangélium, quod prǽdico in géntibus, seórsum autem his, qui observabántur, ne forte in vácuum cúrrerem aut cucurríssem.Sed neque Titus, qui mecum erat, cum esset Græcus, compúlsus est circumcídi. Sed propter subintrodúctos falsos fratres, qui subintroiérunt exploráre libertátem nostram, quam habémus in Christo Iesu, ut nos in servitútem redígerent; quibus neque ad horam céssimus subiciéntes nos, ut véritas evangélii permáneat apud vos.

\noindent Ab his autem, qui videbántur esse áliquid —quales aliquándo fúerint, nihil mea ínterest; Deus persónam hóminis non áccipit— mihi enim, qui observabántur, nihil contulérunt, sed e contra, cum vidíssent quod créditum est mihi evangélium præpútii, sicut Petro circumcisiónis, —qui enim operátus est Petro in apostolátum circumcisiónis operátus est et mihi inter gentes— et cum cognovíssent grátiam, quæ data est mihi, Iacóbus et Cephas et Ioánnes, qui videbántur colúmnæ esse, déxteras dedérunt mihi et Bárnabæ communiónis, ut nos in gentes, ipsi autem in circumcisiónem; tantum ut páuperum mémores essémus, quod étiam sollícitus fui hoc ipsum fácere.}
\newcommand{\responsoriumii}{\pars{Responsorium 2.} \scriptura{\Rbardot{} Ps. 39, 3-4 \Vbardot{} ibid., 2; \textbf{H86}}

\vspace{-5mm}

\responsorium{I}{temporalia/resp-statuitdominussupra-CROCHU.gtex}{}}
\newcommand{\lectioiii}{\pars{Lectio III.} \scriptura{Orat. 6, in Nativitatem Beatæ Mariæ Virginis, 2. 4. 5. 6: PG 96, 663. 667. 670}

\noindent Ex Sermónibus sancti Ioánnis Damascéni presbýteri.

\noindent Quóniam futúrum erat ut Dei Génetrix Virgo ex Anna nascerétur, natúra grátiæ germen antevértere non ausa est; sed mansit fructus expers, dum grátia fructum éderet. Nasci síquidem primogénitam oportébat, ex qua nascitúrus esset omnis creatúræ primogénitus, in quo ómnia constant.

\noindent O par beátum Ióachim et Anna! Vobis omnis creatúra obstrícta est. Per vos enim donum ómnium donórum præstantíssimum Creatóri óbtulit, nempe castam matrem, quæ sola Creatóre digna erat.

\noindent Lætáre, Anna stérilis, quæ non paris: erúmpe et clama, quæ non párturis. Exsúlta, Ióachim, quóniam ex fília tua puer natus est nobis, et fílius datus est nobis, et vocábitur nomen eius magni consílii, salútis univérsi mundi, Angelus, Deus fortis. Puer iste Deus est.

\noindent O beátum par Ióachim etAnna, immaculatíssimum prorsus! Ex fructu ventris vestri cognoscímini, velut alícubi Dóminus ait: Ex frúctibus eórum cognoscétis eos. Uti Deo gratum erat, atque ea dignum quæ ex vobis orta est, vitæ vestræ ratiónes instituístis. Casta enim et sancta conversatióne vestra virginitátis moníle protulístis, eam, quæ ante partum virgo foret, atque in partu virgo nec non virgo post partum; illam, inquam, quæ sola semper, tum mente tum ánimo, tum étiam córpore virginitátem cultúra esset.

\noindent O castíssimum par Ióachim et Anna! Vos castitátem, quam natúræ lex præscríbit, conservántes, ea quæ natúram súperant, divínitus estis consecúti: mundo quippe Dei matrem viri nésciam peperístis. Vos pie et sancte in humána natúra vitam agéntes, fíliam ángelis superiórem nuncque angelórum dóminam edidístis. O speciosíssima dulcissimáque puélla! O fília Adámi et Dei mater! Beáti lumbi et venter, ex quibus prodiísti! Beátæ ulnæ, quæ te gestavérunt; lábia item, quibus castis ósculis frui concéssa es, paréntum nempe dumtáxat tuórum, ut in ómnibus semper virginitátem cóleres! Iubiláte Deo, omnis terra, cantáte, exsultáte et psállite. Exaltáte vocem vestram, exaltáte, nolíte timére.}
\newcommand{\responsoriumiii}{\pars{Responsorium 3.} \scriptura{\textbf{H306}}

\vspace{-5mm}

\responsorium{I}{temporalia/resp-nativitasgloriosae-CROCHU-cumdox.gtex}{}}
\newcommand{\hymnuslaudes}{\pars{Hymnus}

\cuminitiali{III}{temporalia/hym-NoctiSuccedit.gtex}}
\newcommand{\lectiobrevis}{\pars{Lectio Brevis.} \scriptura{Is. 55, 3}

\noindent Inclináte aurem vestram et veníte ad me; audíte, ut vivat ánima vestra, et fériam vobíscum pactum sempitérnum, misericórdias David fidéles.}
\newcommand{\responsoriumbreve}{\pars{Responsorium breve.} \scriptura{Ps. 31, 11}

\cuminitiali{VI}{temporalia/resp-laetaminidomino.gtex}}
\newcommand{\preces}{\noindent Christum magnificémus, plenum grátia et Spíritu Sancto, \gredagger{} et fidénter eum implorémus:

\Rbardot{} Spíritum tuum da nobis, Dómine.

\noindent Concéde nobis diem istum iucúndum, pacíficum et sine mácula, \gredagger{} ut, ad vésperam perdúcti, cum gáudio et mundo corde te collaudáre valeámus.

\Rbardot{} Spíritum tuum da nobis, Dómine.

\noindent Sit hódie splendor tuus super nos, \gredagger{} et opus mánuum nostrárum dírige.

\Rbardot{} Spíritum tuum da nobis, Dómine.

\noindent Osténde fáciem tuam super nos ad bonum in pace, \gredagger{} ut hódie manu tua válida contegámur.

\Rbardot{} Spíritum tuum da nobis, Dómine.

\noindent Réspice propítius omnes, qui oratiónibus nostris confídunt, \gredagger{} eos adímple bonis ánimæ et córporis univérsis.

\Rbardot{} Spíritum tuum da nobis, Dómine.}
\newcommand{\benedictus}{\pars{Canticum Zachariæ.} \scriptura{Lc. 1, 69.68; \textbf{H423}}

\vspace{-4mm}

\antiphona{VIII c}{temporalia/ant-indomodavidpuerisui.gtex}

\vspace{-2mm}

\scriptura{Lc. 1, 68-79}

\vspace{-2mm}

\cantusSineNeumas
\initiumpsalmi{temporalia/benedictus-initium-viii-C-auto.gtex}

%\vspace{-1.5mm}

\input{temporalia/benedictus-viii-C.tex} \Abardot{}}
\newcommand{\hebdomada}{infra Hebdom. XVII per Annum.}
\newcommand{\matua}{Matutinum Hebdomadae A}
\newcommand{\matuac}{Matutinum Hebdomadae A vel C}
\newcommand{\lauda}{Laudes Hebdomadae A}
\newcommand{\laudac}{Laudes Hebdomadae A vel C}

% LuaLaTeX

\documentclass[a4paper, twoside, 12pt]{article}
\usepackage[latin]{babel}
%\usepackage[landscape, left=3cm, right=1.5cm, top=2cm, bottom=1cm]{geometry} % okraje stranky
%\usepackage[landscape, a4paper, mag=1166, truedimen, left=2cm, right=1.5cm, top=1.6cm, bottom=0.95cm]{geometry} % okraje stranky
\usepackage[landscape, a4paper, mag=1400, truedimen, left=0.5cm, right=0.5cm, top=0.5cm, bottom=0.5cm]{geometry} % okraje stranky

\usepackage{fontspec}
\setmainfont[FeatureFile={junicode.fea}, Ligatures={Common, TeX}, RawFeature=+fixi]{Junicode}
%\setmainfont{Junicode}

% shortcut for Junicode without ligatures (for the Czech texts)
\newfontfamily\nlfont[FeatureFile={junicode.fea}, Ligatures={Common, TeX}, RawFeature=+fixi]{Junicode}

\usepackage{multicol}
\usepackage{color}
\usepackage{lettrine}
\usepackage{fancyhdr}

% usual packages loading:
\usepackage{luatextra}
\usepackage{graphicx} % support the \includegraphics command and options
\usepackage{gregoriotex} % for gregorio score inclusion
\usepackage{gregoriosyms}
\usepackage{wrapfig} % figures wrapped by the text
\usepackage{parcolumns}
\usepackage[contents={},opacity=1,scale=1,color=black]{background}
\usepackage{tikzpagenodes}
\usepackage{calc}
\usepackage{longtable}
\usetikzlibrary{calc}

\setlength{\headheight}{14.5pt}

\input{conventuscommune.tex} % Often used macros

\newcommand{\annusEditionis}{2021}

%%%% Vicekrat opakovane kousky

\newcommand{\anteOrationem}{
  \rubrica{Ante Orationem, cantatur a Superiore:}

  \pars{Supplicatio Litaniæ.}

  \cuminitiali{}{temporalia/supplicatiolitaniae.gtex}

  \pars{Oratio Dominica.}

  \cuminitiali{}{temporalia/oratiodominica.gtex}

  \rubrica{Deinde dicitur ab Hebdomadario:}

  \cuminitiali{}{temporalia/dominusvobiscum-solemnis.gtex}

  \rubrica{In choro monialium loco Dominus vobiscum dicitur:}

  \sineinitiali{temporalia/domineexaudi.gtex}
}

\setlength{\columnsep}{30pt} % prostor mezi sloupci

%%%%%%%%%%%%%%%%%%%%%%%%%%%%%%%%%%%%%%%%%%%%%%%%%%%%%%%%%%%%%%%%%%%%%%%%%%%%%%%%%%%%%%%%%%%%%%%%%%%%%%%%%%%%%
\begin{document}

% Here we set the space around the initial.
% Please report to http://home.gna.org/gregorio/gregoriotex/details for more details and options
\grechangedim{afterinitialshift}{2.2mm}{scalable}
\grechangedim{beforeinitialshift}{2.2mm}{scalable}
\grechangedim{interwordspacetext}{0.22 cm plus 0.15 cm minus 0.05 cm}{scalable}%
\grechangedim{annotationraise}{-0.2cm}{scalable}

% Here we set the initial font. Change 38 if you want a bigger initial.
% Emit the initials in red.
\grechangestyle{initial}{\color{red}\fontsize{38}{38}\selectfont}

\pagestyle{empty}

%%%% Titulni stranka
\begin{titulusOfficii}
\ifx\titulus\undefined
\nomenFesti{Feria II \hebdomada{}}
\else
\titulus
\fi
\end{titulusOfficii}

\vfill

\begin{center}
%Ad usum et secundum consuetudines chori \guillemotright{}Conventus Choralis\guillemotleft.

%Editio Sancti Wolfgangi \annusEditionis
\end{center}

\scriptura{}

\pars{}

\pagebreak

\renewcommand{\headrulewidth}{0pt} % no horiz. rule at the header
\fancyhf{}
\pagestyle{fancy}

\cantusSineNeumas

\ifx\oratio\undefined
\ifx\laudb\undefined
\else
\newcommand{\oratio}{\pars{Oratio.}

\noindent Dómine Deus omnípotens, qui ad princípium huius diéi nos perveníre fecísti, tua nos hódie salva virtúte, ut in hac die ad nullum declinémus peccátum, sed semper ad tuam iustítiam faciéndam nostra procédant elóquia, dirigántur cogitatiónes et ópera.

\noindent Per Dóminum nostrum Iesum Christum, Fílium tuum, qui tecum vivit et regnat in unitáte Spíritus Sancti, Deus, per ómnia sǽcula sæculórum.

\noindent \Rbardot{} Amen.}
\fi
\fi

\hora{Ad Matutinum.} %%%%%%%%%%%%%%%%%%%%%%%%%%%%%%%%%%%%%%%%%%%%%%%%%%%%%
%\sideThumbs{Matutinum}

\vspace{2mm}

\cuminitiali{}{temporalia/dominelabiamea.gtex}

\vfill
%\pagebreak

\vspace{2mm}

\ifx\invitatorium\undefined
\pars{Invitatorium.} \scriptura{Ps. 94, 1; Psalmus 94; \textbf{H451}}

\vspace{-6mm}

\antiphona{VI}{temporalia/inv-jubilemusdeo.gtex}\else
\invitatorium
\fi

\vfill
\pagebreak

\ifx\hymnusmatutinum\undefined
\ifx\matua\undefined
\else
\pars{Hymnus.}

{
\grechangedim{interwordspacetext}{0.10 cm plus 0.15 cm minus 0.05 cm}{scalable}%
\antiphona{II}{temporalia/hym-IpsumNunc.gtex}
\grechangedim{interwordspacetext}{0.22 cm plus 0.15 cm minus 0.05 cm}{scalable}%
}
\fi
\else
\hymnusmatutinum
\fi

\vspace{-3mm}

\vfill
\pagebreak

\ifx\matub\undefined
\else
% MAT B
\pars{Psalmus 1.} \scriptura{Ps. 30, 2; \textbf{H90}}

\vspace{-4mm}

\antiphona{VIII G}{temporalia/ant-intuaiustitia.gtex}

%\vspace{-2mm}

\scriptura{Ps. 30, 2-9}

%\vspace{-2mm}

\initiumpsalmi{temporalia/ps30i-initium-viii-G-auto.gtex}

\vspace{-1.5mm}

\input{temporalia/ps30i-viii-G.tex} \Abardot{}

\vfill
\pagebreak

\pars{Psalmus 2.} \scriptura{Ps. 66, 2}

\vspace{-4mm}

\antiphona{E}{temporalia/ant-illuminadomine.gtex}

%\vspace{-2mm}

\scriptura{Ps. 30, 10-17}

%\vspace{-2mm}

\initiumpsalmi{temporalia/ps30ii-initium-e-a-auto.gtex}

\input{temporalia/ps30ii-e-a.tex} \Abardot{}

\vfill
\pagebreak

\pars{Psalmus 3.} \scriptura{Ps. 30, 24}

\vspace{-4mm}

\antiphona{II D}{temporalia/ant-diligitedominum.gtex}

%\vspace{-5mm}

\scriptura{Ps. 30, 20-25}

%\vspace{-2mm}

\initiumpsalmi{temporalia/ps30iii-initium-ii-D-auto.gtex}

\input{temporalia/ps30iii-ii-D.tex} \Abardot{}

\vfill
\pagebreak
\fi

\pars{Versus.}

\ifx\matversus\undefined
\ifx\matub\undefined
\else
\noindent \Vbardot{} Dírige me, Dómine, in veritáte tua, et doce me.

\noindent \Rbardot{} Quia tu es Deus salútis meæ.
\fi
\else
\matversus
\fi

\vspace{5mm}

\sineinitiali{temporalia/oratiodominica-mat.gtex}

\vspace{5mm}

\pars{Absolutio.}

\cuminitiali{}{temporalia/absolutio-exaudi.gtex}

\vfill
\pagebreak

\cuminitiali{}{temporalia/benedictio-solemn-benedictione.gtex}

\vspace{7mm}

\lectioi

\noindent \Vbardot{} Tu autem, Dómine, miserére nobis.
\noindent \Rbardot{} Deo grátias.

\vfill
\pagebreak

\responsoriumi

\vfill
\pagebreak

\cuminitiali{}{temporalia/benedictio-solemn-unigenitus.gtex}

\vspace{7mm}

\lectioii

\noindent \Vbardot{} Tu autem, Dómine, miserére nobis.
\noindent \Rbardot{} Deo grátias.

\vfill
\pagebreak

\responsoriumii

\vfill
\pagebreak

\cuminitiali{}{temporalia/benedictio-solemn-spiritus.gtex}

\vspace{7mm}

\lectioiii

\noindent \Vbardot{} Tu autem, Dómine, miserére nobis.
\noindent \Rbardot{} Deo grátias.

\vfill
\pagebreak

\responsoriumiii

\vfill
\pagebreak

\rubrica{Reliqua omittuntur, nisi Laudes separandæ sint.}

\sineinitiali{temporalia/domineexaudi.gtex}

\vfill

\oratio

\vfill

\noindent \Vbardot{} Dómine, exáudi oratiónem meam.
\Rbardot{} Et clamor meus ad te véniat.

\vfill

\noindent \Vbardot{} Benedicámus Dómino.
\noindent \Rbardot{} Deo grátias.

\vfill

\noindent \Vbardot{} Fidélium ánimæ per misericórdiam Dei requiéscant in pace.
\Rbardot{} Amen.

\vfill
\pagebreak

\hora{Ad Laudes.} %%%%%%%%%%%%%%%%%%%%%%%%%%%%%%%%%%%%%%%%%%%%%%%%%%%%%
%\sideThumbs{Laudes}

\cantusSineNeumas

\vspace{0.5cm}
\grechangedim{interwordspacetext}{0.18 cm plus 0.15 cm minus 0.05 cm}{scalable}%
\cuminitiali{}{temporalia/deusinadiutorium-communis.gtex}
\grechangedim{interwordspacetext}{0.22 cm plus 0.15 cm minus 0.05 cm}{scalable}%

\vfill
\pagebreak

\ifx\hymnuslaudes\undefined
\ifx\laudbd\undefined
\else
\pars{Hymnus} \scriptura{Hilarius (\olddag{} 367)}

\grechangedim{interwordspacetext}{0.16 cm plus 0.15 cm minus 0.05 cm}{scalable}%
\cuminitiali{IV}{temporalia/hym-LucisLargitor.gtex}
\grechangedim{interwordspacetext}{0.22 cm plus 0.15 cm minus 0.05 cm}{scalable}%
\vspace{-3mm}
\fi
\else
\hymnuslaudes
\fi

\vfill
\pagebreak

\ifx\laudb\undefined
\else
\pars{Psalmus 1.} \scriptura{Ps. 41, 3; \textbf{H391}}

\vspace{-4mm}

\antiphona{II D}{temporalia/ant-sitivitanima.gtex}

%\vspace{-2mm}

\scriptura{Psalmus 41}

%\vspace{-2mm}

\initiumpsalmi{temporalia/ps41-initium-ii-D-auto.gtex}

%\vspace{-1.5mm}

\input{temporalia/ps41-ii-D.tex}

\vfill

\antiphona{}{temporalia/ant-sitivitanima.gtex}

\vfill
\pagebreak

\pars{Psalmus 2.}

\vspace{-4mm}

\antiphona{III a}{temporalia/ant-ostendenobisdomine.gtex}

%\vspace{-2mm}

\scriptura{Canticum Ecclesiastici, Sir. 36, 1-7.13-16}

%\vspace{-3mm}

\initiumpsalmi{temporalia/ecclesiastici-initium-iii-a-auto.gtex}

\input{temporalia/ecclesiastici-iii-a.tex} \Abardot{}

\vfill
\pagebreak

\pars{Psalmus 3.}

\vspace{-4mm}

\antiphona{II D}{temporalia/ant-operamanuumeius.gtex}

\scriptura{Psalmus 18, 1-7}

\initiumpsalmi{temporalia/ps18i-initium-ii-D-auto.gtex}

\input{temporalia/ps18i-ii-D.tex} \Abardot{}

\vfill
\pagebreak
\fi

\ifx\lectiobrevis\undefined
\ifx\laudb\undefined
\else
\pars{Lectio Brevis.} \scriptura{Ier. 15, 16}

\noindent Invénti sunt sermónes tui, et comédi eos, et factum est mihi verbum tuum in gáudium et in lætítiam cordis mei, quóniam invocátum est nomen tuum super me, Dómine Deus exercítuum.
\fi
\else
\lectiobrevis
\fi

\vfill

\ifx\responsoriumbreve\undefined
\ifx\laudbd\undefined
\else
\pars{Responsorium breve.} \scriptura{Ps. 32, 1.3}

\cuminitiali{VI}{temporalia/resp-exsultateiusti.gtex}
\fi
\else
\responsoriumbreve
\fi

\vfill
\pagebreak

\ifx\benedictus\undefined
\ifx\laudbd\undefined
\else
\pars{Canticum Zachariæ.} \scriptura{Lc. 1, 68; \textbf{H422}}

\vspace{-4mm}

{
\grechangedim{interwordspacetext}{0.18 cm plus 0.15 cm minus 0.05 cm}{scalable}%
\antiphona{IV E}{temporalia/ant-benedictusdominus.gtex}
\grechangedim{interwordspacetext}{0.22 cm plus 0.15 cm minus 0.05 cm}{scalable}%
}

%\vspace{-3mm}

\scriptura{Lc. 1, 68-79}

%\vspace{-2mm}

\cantusSineNeumas
\initiumpsalmi{temporalia/benedictus-initium-iv-E-auto.gtex}

%\vspace{-1.5mm}

\input{temporalia/benedictus-iv-E.tex} \Abardot{}
\fi
\else
\benedictus
\fi

\vspace{-1cm}

\vfill
\pagebreak

%\sideThumbs{{\scriptsize{}Fine horarum}}

\pars{Preces.}

\sineinitiali{}{temporalia/tonusprecum.gtex}

\ifx\preces\undefined
\ifx\laudb\undefined
\else
\noindent Salvátor noster fecit nos regnum et sacerdótium, ut hóstias Deo acceptábiles offerámus. \gredagger{} Grati ígitur eum invocémus:

\Rbardot{} Serva nos in tuo ministério, Dómine.

\noindent Christe, sacérdos ætérne, qui sanctum pópulo tuo sacerdótium concessísti, \gredagger{} concéde, ut spiritáles hóstias Deo acceptábiles iúgiter offerámus.

\Rbardot{} Serva nos in tuo ministério, Dómine.

\noindent Spíritus tui fructus nobis largíre propítius, \gredagger{} patiéntiam, benignitátem et mansuetúdinem.

\Rbardot{} Serva nos in tuo ministério, Dómine.

\noindent Da nobis te amáre, ut te, qui es cáritas, possideámus, \gredagger{} et bene ágere, ut per vitam étiam nostram te laudémus.

\Rbardot{} Serva nos in tuo ministério, Dómine.

\noindent Quæ frátribus nostris sunt utília, nos quǽrere concéde, \gredagger{} ut salútem facílius consequántur.

\Rbardot{} Serva nos in tuo ministério, Dómine.
\fi
\else
\preces
\fi

\vfill

\pars{Oratio Dominica.}

\cuminitiali{}{temporalia/oratiodominicaalt.gtex}

\vfill
\pagebreak

\rubrica{vel:}

\pars{Supplicatio Litaniæ.}

\cuminitiali{}{temporalia/supplicatiolitaniae.gtex}

\vfill

\pars{Oratio Dominica.}

\cuminitiali{}{temporalia/oratiodominica.gtex}

\vfill
\pagebreak

% Oratio. %%%
\oratio

\vspace{-1mm}

\vfill

\rubrica{Hebdomadarius dicit Dominus vobiscum, vel, absente sacerdote vel diacono, sic concluditur:}

\vspace{2mm}

\antiphona{C}{temporalia/dominusnosbenedicat.gtex}

\rubrica{Postea cantatur a cantore:}

\vspace{2mm}

\cuminitiali{IV}{temporalia/benedicamus-feria-laudes.gtex}

\vspace{1mm}

\vfill
\pagebreak

\end{document}

