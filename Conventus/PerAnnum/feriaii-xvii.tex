\newcommand{\titulus}{\nomenFesti{Ss. Marthæ, Mariæ \& Lazari, Hospitum Domini.}
\dies{Die 29. Iulii.}}
\newcommand{\sineobmv}{Sine officium B.M.V.}
\newcommand{\oratio}{\pars{Oratio.}

\noindent Deus, cuius Fílius de sepúlcro ad vitam Lázarum revocávit, et in domo Marthæ dignátus est hospitári, da nobis, quǽsumus, ut ipsi in frátribus nostris fidéliter ministrántes, cum María verbi eius meditatióne pasci mereámur.

%\noindent Qui tecum vivit et regnat in unitáte Spíritus Sancti, Deus, per ómnia sǽcula sæculórum.

\pars{Pro pace in universo mundo.} \scriptura{Sir. 50, 25; 2 Esdr. 4, 20; \textbf{H416}}

\vspace{-4mm}

\antiphona{II D}{temporalia/ant-dapacemdomine.gtex}

\vfill

\noindent Deus, a quo sancta desidéria, recta consília et iusta sunt ópera: da servis tuis illam, quam mundus dare non potest, pacem; ut et corda nostra mandátis tuis dédita, et hóstium subláta formídine, témpora sint tua protectióne tranquílla.

\noindent Per Dóminum nostrum Iesum Christum, Fílium tuum, qui tecum vivit et regnat in unitáte Spíritus Sancti, Deus, per ómnia sǽcula sæculórum.

\noindent \Rbardot{} Amen.}
\newcommand{\invitatorium}{\pars{Invitatorium.}

\vspace{-4mm}

\antiphona{IV}{temporalia/inv-mirabileminsanctis.gtex}}
\newcommand{\hymnusmatutinum}{\pars{Hymnus}

\cuminitiali{II}{temporalia/hym-QuasTibi.gtex}}
\newcommand{\matversus}{\noindent \Vbardot{} Non amóvit Dóminus oratiónem meam.

\noindent \Rbardot{} Et misericórdiam suam a me.}
\newcommand{\lectioi}{\pars{Lectio I.} \scriptura{2 Co 8, 1-24}

\noindent De Epístola secúnda beáti Pauli apóstoli ad Corínthios.

\noindent Notam fácimus vobis, fratres, grátiam Dei, quæ data est in ecclésiis Macedóniæ, quod in multo experiménto tribulatiónis abundántia gáudii ipsórum et altíssima paupértas eórum abundávit in divítias simplicitátis eórum; quia secúndum virtútem, testimónium reddo, et supra virtútem voluntárii fuérunt cum multa exhortatióne obsecrántes nos grátiam et communicatiónem ministérii, quod fit in sanctos. Et non sicut sperávimus, sed semetípsos dedérunt primum Dómino, deínde nobis per voluntátem Dei, ita ut rogarémus Titum, ut, quemádmodum cœpit, ita et perfíciat in vos étiam grátiam istam.

\noindent Sed sicut in ómnibus abundátis, fide et sermóne et sciéntia et omni sollicitúdine et caritáte ex nobis in vobis, ut et in hac grátia abundétis. Non quasi ímperans dico, sed per aliórum sollicitúdinem étiam vestræ caritátis ingénitum bonum cómprobans; scitis enim grátiam Dómini nostri Iesu Christi, quóniam propter vos egénus factus est, cum esset dives, ut illíus inópia vos dívites essétis. Et consílium in hoc do. Hoc enim vobis útile est, qui non solum fácere sed et velle cœpístis ab anno prióre; nunc vero et facto perfícite, ut, quemádmodum promptus est ánimus velle, ita sit et perfícere ex eo, quod habétis. Si enim volúntas prompta est, secúndum id quod habet, accépta est, non secúndum quod non habet. Non enim, ut áliis sit remíssio, vobis autem tribulátio; sed ex æqualitáte in præsénti témpore vestra abundántia illórum inópiam súppleat, ut et illórum abundántia vestram inópiam súppleat, ut fiat æquálitas, sicut scriptum est: « \emph{Qui multum, non abundávit; et, qui módicum, non minorávit».}

\noindent {\color{gray} Grátias autem Deo, qui dedit eándem sollicitúdinem, pro vobis in corde Titi, quóniam exhortatiónem quidem suscépit, sed, cum sollicítior esset, sua voluntáte proféctus est ad vos. Mísimus étiam cum illo fratrem, cuius laus est in evangélio per omnes ecclésias —non solum autem sed et ordinátus ab ecclésiis comes noster cum hac grátia, quæ ministrátur a nobis ad Dómini glóriam et destinátam voluntátem nostram— devitántes hoc, ne quis nos vitúperet in hac plenitúdine, quæ ministrátur a nobis; \emph{providémus} enim \emph{bona} non solum \emph{coram Dómino} sed \emph{étiam} coram \emph{homínibus.} Mísimus autem cum illis et fratrem nostrum, quem probávimus in multis sæpe sollícitum esse, nunc autem multo sollicitiórem, confidéntia multa in vos.

\noindent Sive pro Tito, est sócius meus et in vos adiútor; sive fratres nostri, apóstoli ecclesiárum, glória Christi. Ostensiónem ergo, quæ est caritátis vestræ et nostræ gloriatiónis pro vobis, in illos osténdite in fáciem ecclesiárum.}}
\newcommand{\responsoriumi}{\pars{Responsorium 1.} \scriptura{\Vbardot{} Ps. 132, 1; \textbf{H368}}

\vspace{-5mm}

\responsorium{I}{temporalia/resp-proptertestamentumdomini-CROCHU.gtex}{}}
\newcommand{\lectioii}{\pars{Lectio II.} \scriptura{Sermo 3 in Assumptione beatae Mariae Virginis, 4. 5: PL 183, 423. 424}

\noindent Ex Sermónibus sancti Bemárdi abbátis.

\noindent Considerémus, fratres, quemádmodum in hac domo nostra tria hæc distribúerit ordinátio caritátis: Marthæ administratiónem, Maríæ contemplatiónem, Lázari pæniténtiam. Habet hæc simul quæcúmque perfécta est ánima; magis tamen vidéntur ad síngulos síngula pertinére, ut álii vacens sanctæ contemplatióni, álii déditi sint fratérnæ administratióni, álii in amaritúdine ánimæ suæ recógitent annos suos, tamquam vulneráti dormiéntes in sepúlcris. Sic plane, sic opus est, ut María pie et sublímiter séntiat de Deo suo, Martha benígne et misericórditer de próximo, Lázarus mísere et humíliter de se ipso.

\noindent Gradum suum quisque consíderet. \emph{Si invénti fúerint in civitáte hac Noe, Dániel, Iob, ipsi iustítia sua liberabúntur, ait Dóminus: sed fílium aut fíliam non liberábunt.} Némini nos blándimur, útinam nec vestrum quíspiam se sedúcat! Quibus enim nulla crédita est dispensátio, administrátio nulla commíssa, his omníno sedéndum erit, aut secus pedes Iesu cum Maria, aut certe cum Lázaro intra septa sepúlcri. Quidni erga multa turbétur Martha, quæ sollícita est pro multis? Tibi vero cui necéssitas hæc non incúmbit, e duóbus unum est necessárium: aut non turbári pénitus, sed delectári magis in Dómino; aut, si id necdum potes, turbári non erga plúrima, sed, ut de se prophéta lóquitur, ad te ipsum.}
\newcommand{\responsoriumii}{\pars{Responsorium 2.} \scriptura{\Rbardot{} Sir. 15, 3 \Vbardot{} Ps. 131, 18; \textbf{H62}}

\vspace{-5mm}

\responsorium{VII}{temporalia/resp-cibavitillum-CROCHU.gtex}{}}
\newcommand{\lectioiii}{\pars{Lectio III.}

\noindent Sed et ipsam quoque Martham admónitam esse necésse est, id máxime quæri inter dispensatóres, ut fidélis quis inveniátur. Erit autem fidélis, si neque quæ Iesu Christi, ut sit inténtio pura; nec suam fáciat, sed Dómini voluntátem, ut sit áctio ordináta. Sunt enim quorum non simplex est óculus, et recípiunt mercédem suam. Sunt qui ferúntur própriis mótibus animórum, et contamináta sunt univérsa quæ ófferunt, quippe cum voluntátes eórum inveniántur in eis.

\noindent Veni nunc mecum ad nuptiále carmen, et considerémus quemádmodum sponsus, ubi sponsam vocat, nec ullum omíserit ex his tribus, nec his addíderit quidquam. \emph{Surge,} inquit, \emph{própera, amíca mea, formósa mea, colúmba mea, et veni.} An non amíca est, quas Domínicis lucris inténta, fidéliter ipsam quoque pro eo ponit ánimam suam? Quóties enim pro uno ex mínimis eius spirituále stúdium intermíttit, tóties pro eo spirituáliter ponit ánimam suam. An non formósa, quas reveláta fácie glóriam Dómini speculándo, in eándem imáginem transformátur de claritáte in claritátem, tamquam a Dómini Spíritu? An non colúmba, quæ plangit et gemit in foramínibus petræ, in cavérnis macériæ, tamquam sepúlta sub lápide?}
\newcommand{\responsoriumiii}{\pars{Responsorium 3.} \scriptura{\Rbardot{} Ct. 6, 9 \Vbardot{} ibid. 3, 6; \textbf{H297}}

\vspace{-5mm}

\responsorium{IV}{temporalia/resp-quaeestista-CROCHU-cumdox.gtex}{}}
\newcommand{\hymnuslaudes}{\pars{Hymnus}

\cuminitiali{VIII}{temporalia/hym-FlagransAmore.gtex}}
\newcommand{\lectiobrevis}{\pars{Lectio Brevis.} \scriptura{Phil. 3, 7-8}

\noindent Quæ mihi erant lucra, hæc arbitrátus sum propter Christum detriméntum. Verúmtamen exístimo ómnia detriméntum esse propter eminéntiam sciéntiæ Christi Iesu Dómini mei, propter quem ómnia detriméntum feci et árbitror ut stércora, ut Christum lucri fáciam.}
\newcommand{\responsoriumbreve}{\pars{Responsorium breve.} \scriptura{Ps. 67, 36}

\cuminitiali{VI}{temporalia/resp-mirabilisdeus.gtex}}
\newcommand{\preces}{\noindent Implorémus Patrem, fontem omnis sanctitátis,~\gredagger{} ut, per sanctórum exémpla et intercessiónem,~\grestar{} ad vitam sanctam nos perdúcat, et dicámus:

\Rbardot{} Sancti simus, quia tu sanctus es.

\noindent Pater sancte, qui voluísti ut fílii tui nominémur et simus,~\grestar{} fac ut te per orbem terrárum sancta confiteátur Ecclésia.

\Rbardot{} Sancti simus, quia tu sanctus es.

\noindent Pater sancte, qui voluísti ut ambulémus digne tibi per ómnia placéntes,~\grestar{} da, ut in ópere bono fructificémus.

\Rbardot{} Sancti simus, quia tu sanctus es.

\noindent Pater sancte, qui nos tibi reconciliásti per Christum,~\grestar{} serva nos in nómine tuo, ut omnes unum simus.

\Rbardot{} Sancti simus, quia tu sanctus es.

\noindent Pater sancte, qui nos ad cæléste vocásti convívium,~\grestar{} per panem, qui de cælo descéndit, præsta nobis, ut caritátem pleniórem cónsequi valeámus.

\Rbardot{} Sancti simus, quia tu sanctus es.

\noindent Pater sancte, ómnibus peccatóribus dimítte delícta,~\grestar{} et súscipe defúnctos ad lumen vultus tui.

\Rbardot{} Sancti simus, quia tu sanctus es.}
\newcommand{\benedictus}{\pars{Canticum Zachariæ.} \scriptura{Lc. 10, 38-39}

\vspace{-4mm}

\antiphona{VII d}{temporalia/ant-intrantemiesum.gtex}

\vspace{-2mm}

\scriptura{Lc. 1, 68-79}

\vspace{-2mm}

\cantusSineNeumas
\initiumpsalmi{temporalia/benedictus-initium-vii-d3-auto.gtex}

%\vspace{-1.5mm}

\input{temporalia/benedictus-vii-d3.tex} \Abardot{}}
\newcommand{\benedicamuslaudes}{\cuminitiali{}{temporalia/benedicamus-memoria-laudes.gtex}}
\newcommand{\hebdomada}{infra Hebdom. XVII per Annum.}
\newcommand{\matua}{Matutinum Hebdomadae A}
\newcommand{\matuac}{Matutinum Hebdomadae A vel C}
\newcommand{\lauda}{Laudes Hebdomadae A}
\newcommand{\laudac}{Laudes Hebdomadae A vel C}

% LuaLaTeX

\documentclass[a4paper, twoside, 12pt]{article}
\usepackage[latin]{babel}
%\usepackage[landscape, left=3cm, right=1.5cm, top=2cm, bottom=1cm]{geometry} % okraje stranky
%\usepackage[landscape, a4paper, mag=1166, truedimen, left=2cm, right=1.5cm, top=1.6cm, bottom=0.95cm]{geometry} % okraje stranky
\usepackage[landscape, a4paper, mag=1400, truedimen, left=0.5cm, right=0.5cm, top=0.5cm, bottom=0.5cm]{geometry} % okraje stranky

\usepackage{fontspec}
\setmainfont[FeatureFile={junicode.fea}, Ligatures={Common, TeX}, RawFeature=+fixi]{Junicode}
%\setmainfont{Junicode}

% shortcut for Junicode without ligatures (for the Czech texts)
\newfontfamily\nlfont[FeatureFile={junicode.fea}, Ligatures={Common, TeX}, RawFeature=+fixi]{Junicode}

\usepackage{multicol}
\usepackage{color}
\usepackage{lettrine}
\usepackage{fancyhdr}

% usual packages loading:
\usepackage{luatextra}
\usepackage{graphicx} % support the \includegraphics command and options
\usepackage{gregoriotex} % for gregorio score inclusion
\usepackage{gregoriosyms}
\usepackage{wrapfig} % figures wrapped by the text
\usepackage{parcolumns}
\usepackage[contents={},opacity=1,scale=1,color=black]{background}
\usepackage{tikzpagenodes}
\usepackage{calc}
\usepackage{longtable}
\usetikzlibrary{calc}

\setlength{\headheight}{14.5pt}

\input{conventuscommune.tex} % Often used macros

\newcommand{\annusEditionis}{2021}

%%%% Vicekrat opakovane kousky

\newcommand{\anteOrationem}{
  \rubrica{Ante Orationem, cantatur a Superiore:}

  \pars{Supplicatio Litaniæ.}

  \cuminitiali{}{temporalia/supplicatiolitaniae.gtex}

  \pars{Oratio Dominica.}

  \cuminitiali{}{temporalia/oratiodominica.gtex}

  \rubrica{Deinde dicitur ab Hebdomadario:}

  \cuminitiali{}{temporalia/dominusvobiscum-solemnis.gtex}

  \rubrica{In choro monialium loco Dominus vobiscum dicitur:}

  \sineinitiali{temporalia/domineexaudi.gtex}
}

\setlength{\columnsep}{30pt} % prostor mezi sloupci

%%%%%%%%%%%%%%%%%%%%%%%%%%%%%%%%%%%%%%%%%%%%%%%%%%%%%%%%%%%%%%%%%%%%%%%%%%%%%%%%%%%%%%%%%%%%%%%%%%%%%%%%%%%%%
\begin{document}

% Here we set the space around the initial.
% Please report to http://home.gna.org/gregorio/gregoriotex/details for more details and options
\grechangedim{afterinitialshift}{2.2mm}{scalable}
\grechangedim{beforeinitialshift}{2.2mm}{scalable}
\grechangedim{interwordspacetext}{0.22 cm plus 0.15 cm minus 0.05 cm}{scalable}%
\grechangedim{annotationraise}{-0.2cm}{scalable}

% Here we set the initial font. Change 38 if you want a bigger initial.
% Emit the initials in red.
\grechangestyle{initial}{\color{red}\fontsize{38}{38}\selectfont}

\pagestyle{empty}

%%%% Titulni stranka
\begin{titulusOfficii}
\ifx\titulus\undefined
\nomenFesti{Feria II \hebdomada{}}
\else
\titulus
\fi
\end{titulusOfficii}

\vfill

\begin{center}
%Ad usum et secundum consuetudines chori \guillemotright{}Conventus Choralis\guillemotleft.

%Editio Sancti Wolfgangi \annusEditionis
\end{center}

\scriptura{}

\pars{}

\pagebreak

\renewcommand{\headrulewidth}{0pt} % no horiz. rule at the header
\fancyhf{}
\pagestyle{fancy}

\cantusSineNeumas

\ifx\oratio\undefined
\ifx\laudb\undefined
\else
\newcommand{\oratio}{\pars{Oratio.}

\noindent Dómine Deus omnípotens, qui ad princípium huius diéi nos perveníre fecísti, tua nos hódie salva virtúte, ut in hac die ad nullum declinémus peccátum, sed semper ad tuam iustítiam faciéndam nostra procédant elóquia, dirigántur cogitatiónes et ópera.

\noindent Per Dóminum nostrum Iesum Christum, Fílium tuum, qui tecum vivit et regnat in unitáte Spíritus Sancti, Deus, per ómnia sǽcula sæculórum.

\noindent \Rbardot{} Amen.}
\fi
\fi

\hora{Ad Matutinum.} %%%%%%%%%%%%%%%%%%%%%%%%%%%%%%%%%%%%%%%%%%%%%%%%%%%%%
%\sideThumbs{Matutinum}

\vspace{2mm}

\cuminitiali{}{temporalia/dominelabiamea.gtex}

\vfill
%\pagebreak

\vspace{2mm}

\ifx\invitatorium\undefined
\pars{Invitatorium.} \scriptura{Ps. 94, 1; Psalmus 94; \textbf{H451}}

\vspace{-6mm}

\antiphona{VI}{temporalia/inv-jubilemusdeo.gtex}\else
\invitatorium
\fi

\vfill
\pagebreak

\ifx\hymnusmatutinum\undefined
\ifx\matua\undefined
\else
\pars{Hymnus.}

{
\grechangedim{interwordspacetext}{0.10 cm plus 0.15 cm minus 0.05 cm}{scalable}%
\antiphona{II}{temporalia/hym-IpsumNunc.gtex}
\grechangedim{interwordspacetext}{0.22 cm plus 0.15 cm minus 0.05 cm}{scalable}%
}
\fi
\else
\hymnusmatutinum
\fi

\vspace{-3mm}

\vfill
\pagebreak

\ifx\matub\undefined
\else
% MAT B
\pars{Psalmus 1.} \scriptura{Ps. 30, 2; \textbf{H90}}

\vspace{-4mm}

\antiphona{VIII G}{temporalia/ant-intuaiustitia.gtex}

%\vspace{-2mm}

\scriptura{Ps. 30, 2-9}

%\vspace{-2mm}

\initiumpsalmi{temporalia/ps30i-initium-viii-G-auto.gtex}

\vspace{-1.5mm}

\input{temporalia/ps30i-viii-G.tex} \Abardot{}

\vfill
\pagebreak

\pars{Psalmus 2.} \scriptura{Ps. 66, 2}

\vspace{-4mm}

\antiphona{E}{temporalia/ant-illuminadomine.gtex}

%\vspace{-2mm}

\scriptura{Ps. 30, 10-17}

%\vspace{-2mm}

\initiumpsalmi{temporalia/ps30ii-initium-e-a-auto.gtex}

\input{temporalia/ps30ii-e-a.tex} \Abardot{}

\vfill
\pagebreak

\pars{Psalmus 3.} \scriptura{Ps. 30, 24}

\vspace{-4mm}

\antiphona{II D}{temporalia/ant-diligitedominum.gtex}

%\vspace{-5mm}

\scriptura{Ps. 30, 20-25}

%\vspace{-2mm}

\initiumpsalmi{temporalia/ps30iii-initium-ii-D-auto.gtex}

\input{temporalia/ps30iii-ii-D.tex} \Abardot{}

\vfill
\pagebreak
\fi

\pars{Versus.}

\ifx\matversus\undefined
\ifx\matub\undefined
\else
\noindent \Vbardot{} Dírige me, Dómine, in veritáte tua, et doce me.

\noindent \Rbardot{} Quia tu es Deus salútis meæ.
\fi
\else
\matversus
\fi

\vspace{5mm}

\sineinitiali{temporalia/oratiodominica-mat.gtex}

\vspace{5mm}

\pars{Absolutio.}

\cuminitiali{}{temporalia/absolutio-exaudi.gtex}

\vfill
\pagebreak

\cuminitiali{}{temporalia/benedictio-solemn-benedictione.gtex}

\vspace{7mm}

\lectioi

\noindent \Vbardot{} Tu autem, Dómine, miserére nobis.
\noindent \Rbardot{} Deo grátias.

\vfill
\pagebreak

\responsoriumi

\vfill
\pagebreak

\cuminitiali{}{temporalia/benedictio-solemn-unigenitus.gtex}

\vspace{7mm}

\lectioii

\noindent \Vbardot{} Tu autem, Dómine, miserére nobis.
\noindent \Rbardot{} Deo grátias.

\vfill
\pagebreak

\responsoriumii

\vfill
\pagebreak

\cuminitiali{}{temporalia/benedictio-solemn-spiritus.gtex}

\vspace{7mm}

\lectioiii

\noindent \Vbardot{} Tu autem, Dómine, miserére nobis.
\noindent \Rbardot{} Deo grátias.

\vfill
\pagebreak

\responsoriumiii

\vfill
\pagebreak

\rubrica{Reliqua omittuntur, nisi Laudes separandæ sint.}

\sineinitiali{temporalia/domineexaudi.gtex}

\vfill

\oratio

\vfill

\noindent \Vbardot{} Dómine, exáudi oratiónem meam.
\Rbardot{} Et clamor meus ad te véniat.

\vfill

\noindent \Vbardot{} Benedicámus Dómino.
\noindent \Rbardot{} Deo grátias.

\vfill

\noindent \Vbardot{} Fidélium ánimæ per misericórdiam Dei requiéscant in pace.
\Rbardot{} Amen.

\vfill
\pagebreak

\hora{Ad Laudes.} %%%%%%%%%%%%%%%%%%%%%%%%%%%%%%%%%%%%%%%%%%%%%%%%%%%%%
%\sideThumbs{Laudes}

\cantusSineNeumas

\vspace{0.5cm}
\grechangedim{interwordspacetext}{0.18 cm plus 0.15 cm minus 0.05 cm}{scalable}%
\cuminitiali{}{temporalia/deusinadiutorium-communis.gtex}
\grechangedim{interwordspacetext}{0.22 cm plus 0.15 cm minus 0.05 cm}{scalable}%

\vfill
\pagebreak

\ifx\hymnuslaudes\undefined
\ifx\laudbd\undefined
\else
\pars{Hymnus} \scriptura{Hilarius (\olddag{} 367)}

\grechangedim{interwordspacetext}{0.16 cm plus 0.15 cm minus 0.05 cm}{scalable}%
\cuminitiali{IV}{temporalia/hym-LucisLargitor.gtex}
\grechangedim{interwordspacetext}{0.22 cm plus 0.15 cm minus 0.05 cm}{scalable}%
\vspace{-3mm}
\fi
\else
\hymnuslaudes
\fi

\vfill
\pagebreak

\ifx\laudb\undefined
\else
\pars{Psalmus 1.} \scriptura{Ps. 41, 3; \textbf{H391}}

\vspace{-4mm}

\antiphona{II D}{temporalia/ant-sitivitanima.gtex}

%\vspace{-2mm}

\scriptura{Psalmus 41}

%\vspace{-2mm}

\initiumpsalmi{temporalia/ps41-initium-ii-D-auto.gtex}

%\vspace{-1.5mm}

\input{temporalia/ps41-ii-D.tex}

\vfill

\antiphona{}{temporalia/ant-sitivitanima.gtex}

\vfill
\pagebreak

\pars{Psalmus 2.}

\vspace{-4mm}

\antiphona{III a}{temporalia/ant-ostendenobisdomine.gtex}

%\vspace{-2mm}

\scriptura{Canticum Ecclesiastici, Sir. 36, 1-7.13-16}

%\vspace{-3mm}

\initiumpsalmi{temporalia/ecclesiastici-initium-iii-a-auto.gtex}

\input{temporalia/ecclesiastici-iii-a.tex} \Abardot{}

\vfill
\pagebreak

\pars{Psalmus 3.}

\vspace{-4mm}

\antiphona{II D}{temporalia/ant-operamanuumeius.gtex}

\scriptura{Psalmus 18, 1-7}

\initiumpsalmi{temporalia/ps18i-initium-ii-D-auto.gtex}

\input{temporalia/ps18i-ii-D.tex} \Abardot{}

\vfill
\pagebreak
\fi

\ifx\lectiobrevis\undefined
\ifx\laudb\undefined
\else
\pars{Lectio Brevis.} \scriptura{Ier. 15, 16}

\noindent Invénti sunt sermónes tui, et comédi eos, et factum est mihi verbum tuum in gáudium et in lætítiam cordis mei, quóniam invocátum est nomen tuum super me, Dómine Deus exercítuum.
\fi
\else
\lectiobrevis
\fi

\vfill

\ifx\responsoriumbreve\undefined
\ifx\laudbd\undefined
\else
\pars{Responsorium breve.} \scriptura{Ps. 32, 1.3}

\cuminitiali{VI}{temporalia/resp-exsultateiusti.gtex}
\fi
\else
\responsoriumbreve
\fi

\vfill
\pagebreak

\ifx\benedictus\undefined
\ifx\laudbd\undefined
\else
\pars{Canticum Zachariæ.} \scriptura{Lc. 1, 68; \textbf{H422}}

\vspace{-4mm}

{
\grechangedim{interwordspacetext}{0.18 cm plus 0.15 cm minus 0.05 cm}{scalable}%
\antiphona{IV E}{temporalia/ant-benedictusdominus.gtex}
\grechangedim{interwordspacetext}{0.22 cm plus 0.15 cm minus 0.05 cm}{scalable}%
}

%\vspace{-3mm}

\scriptura{Lc. 1, 68-79}

%\vspace{-2mm}

\cantusSineNeumas
\initiumpsalmi{temporalia/benedictus-initium-iv-E-auto.gtex}

%\vspace{-1.5mm}

\input{temporalia/benedictus-iv-E.tex} \Abardot{}
\fi
\else
\benedictus
\fi

\vspace{-1cm}

\vfill
\pagebreak

%\sideThumbs{{\scriptsize{}Fine horarum}}

\pars{Preces.}

\sineinitiali{}{temporalia/tonusprecum.gtex}

\ifx\preces\undefined
\ifx\laudb\undefined
\else
\noindent Salvátor noster fecit nos regnum et sacerdótium, ut hóstias Deo acceptábiles offerámus. \gredagger{} Grati ígitur eum invocémus:

\Rbardot{} Serva nos in tuo ministério, Dómine.

\noindent Christe, sacérdos ætérne, qui sanctum pópulo tuo sacerdótium concessísti, \gredagger{} concéde, ut spiritáles hóstias Deo acceptábiles iúgiter offerámus.

\Rbardot{} Serva nos in tuo ministério, Dómine.

\noindent Spíritus tui fructus nobis largíre propítius, \gredagger{} patiéntiam, benignitátem et mansuetúdinem.

\Rbardot{} Serva nos in tuo ministério, Dómine.

\noindent Da nobis te amáre, ut te, qui es cáritas, possideámus, \gredagger{} et bene ágere, ut per vitam étiam nostram te laudémus.

\Rbardot{} Serva nos in tuo ministério, Dómine.

\noindent Quæ frátribus nostris sunt utília, nos quǽrere concéde, \gredagger{} ut salútem facílius consequántur.

\Rbardot{} Serva nos in tuo ministério, Dómine.
\fi
\else
\preces
\fi

\vfill

\pars{Oratio Dominica.}

\cuminitiali{}{temporalia/oratiodominicaalt.gtex}

\vfill
\pagebreak

\rubrica{vel:}

\pars{Supplicatio Litaniæ.}

\cuminitiali{}{temporalia/supplicatiolitaniae.gtex}

\vfill

\pars{Oratio Dominica.}

\cuminitiali{}{temporalia/oratiodominica.gtex}

\vfill
\pagebreak

% Oratio. %%%
\oratio

\vspace{-1mm}

\vfill

\rubrica{Hebdomadarius dicit Dominus vobiscum, vel, absente sacerdote vel diacono, sic concluditur:}

\vspace{2mm}

\antiphona{C}{temporalia/dominusnosbenedicat.gtex}

\rubrica{Postea cantatur a cantore:}

\vspace{2mm}

\cuminitiali{IV}{temporalia/benedicamus-feria-laudes.gtex}

\vspace{1mm}

\vfill
\pagebreak

\end{document}

