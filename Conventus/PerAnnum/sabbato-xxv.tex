\newcommand{\titulus}{\nomenFesti{S. Wenceslai, Ducis \& Martyris.}
\dies{Die 28. Septembris.}}
\newcommand{\sineobmv}{Sine officium B.M.V.}
\newcommand{\oratio}{\pars{Oratio.}

\noindent Deus, qui beátum mártyrem Vencesláum cælésti regno terrénum postpónere docuísti, eius précibus concéde, ut, nosmetípsos abnegántes, tibi toto corde adhærére valeámus.

\pars{Pro pace in universo mundo.} \scriptura{Sir. 50, 25; 2 Esdr. 4, 20; \textbf{H416}}

\vspace{-4mm}

\antiphona{II D}{temporalia/ant-dapacemdomine.gtex}

\vfill

\noindent Deus, a quo sancta desidéria, recta consília et iusta sunt ópera: da servis tuis illam, quam mundus dare non potest, pacem; ut et corda nostra mandátis tuis dédita, et hóstium subláta formídine, témpora sint tua protectióne tranquílla.

\noindent Per Dóminum nostrum Iesum Christum, Fílium tuum, qui tecum vivit et regnat in unitáte Spíritus Sancti, Deus, per ómnia sǽcula sæculórum.

\noindent \Rbardot{} Amen.}
\newcommand{\invitatorium}{\pars{Invitatorium.}

\vspace{-4mm}

\antiphona{E}{temporalia/inv-regemmartyrumsimplex.gtex}}
\newcommand{\hymnusmatutinum}{\pars{Hymnus.}

\antiphona{VIII}{temporalia/hym-DeusTuorum.gtex}}
\newcommand{\matversus}{\noindent \Vbardot{} Díriget Dóminus mansuétos in iudício.

\noindent \Rbardot{} Docébit mites vias suas.}
\newcommand{\lectioi}{\pars{Lectio I.} \scriptura{Idt. 13, 11-20; 15, 8a. 9b-10}

\noindent De libro Iudith.

\noindent Dixit Iudith a longe eis, qui custodiébant in portis Betúliæ: “Aperíte, aperíte portam! Nobíscum est Deus, Deus noster, ut fáciat adhuc virtútem in Israel et poténtiam advérsus inimícos nostros, sicut et hódie fecit”.

\noindent Et factum est, ut audiérunt viri civitátis vocem eius, festinavérunt descéndere ad portam civitátis suæ et convocavérunt senióres civitátis. Et concurrérunt omnes a mínimo usque ad magnum, quóniam mirum erat eis illam revérsam esse. Et aperuérunt portam et recepérunt eas et accendéntes ignem ad lumen congyravérunt eam.

\noindent Quæ dixit ad eos voce magna: “Laudáte Dóminum nostrum, laudáte, quia non ábstulit misericórdiam suam a domo Israel, sed contéruit inimícos nostros per manum meam in hac nocte”. Et próferens caput de pera sua osténdit et dixit eis: “Ecce caput Holoférnis príncipis milítiæ virtútis Assyriórum, et ecce conópeum, in quo recumbébat in ebrietáte sua. Et percússit eum Dóminus in manu féminæ. Et vivit Dóminus, qui custodívit me in via mea, qua profécta sum, quóniam sedúxit eum fácies mea in perditiónem eius, et non fecit peccátum mecum in coinquinatiónem et confusiónem”.

\noindent Et obstúpuit omnis pópulus valde et inclinántes se adoravérunt Deum et dixérunt unánimes: “Benedíctus es, Deus noster, qui ad níhilum redegísti inimícos pópuli tui in hodiérna die”.

\noindent Dixit ad Iudith Ozías: “Benedícta tu es, fília, a Deo excélso præ ómnibus muliéribus, quæ sunt super terram. Et benedíctus Dóminus Deus noster, qui creávit cælum et terram, qui diréxit te in vulnus cápitis príncipis inimicórum nostrórum. Quóniam non discédet laus tua a corde hóminum memorántium virtútis Dei usque in sempitérnum. Et fáciat tibi ea Deus in exaltatiónem ætérnam, ut vísitet te in bonis, pro eo quod non pepercísti ánimæ tuæ propter humilitátem géneris nostri, sed prosilísti in ruínam nostram in diréctum ámbulans in conspéctu Dei nostri”.

\noindent Et dixit omnis pópulus: “Fiat, fiat!”.

\noindent Et dixérunt ad Iudith Ióachim sacérdos magnus et senióres filiórum Israel, qui inhabitábant Ierúsalem: “Tu exaltátio Ierúsalem, tu glória magna Israel, tu laus magna géneris nostri. Fecísti ómnia hæc in manu tua, fecísti bona cum Israel, et complácuit in illis Deus. Benedícta esto tu, múlier, apud Deum omnipoténtem in ætérnum tempus”. Et dixit omnis pópulus: “Fiat, fiat!”.}
\newcommand{\responsoriumi}{\pars{Responsorium 1.} \scriptura{\Rbardot{} Idt. 13, 17 \Vbardot{} Ps. 116, 1 \Vbardot{} Ps. 117, 1; \textbf{H411}}

\vspace{-5mm}

\responsorium{II}{temporalia/resp-laudatedominumdeumnostrum-CROCHU.gtex}{}

\rubrica{vel ad libitum:}

\vspace{3mm}

\pars{Responsorium 1.} \scriptura{\Rbardot{} Idt. 13, 22.25 \Vbardot{} ibid. 13, 24.25; \textbf{H409}}

\vspace{-5mm}

\responsorium{I}{temporalia/resp-benedixittedominus-CROCHU.gtex}{}}
\newcommand{\lectioii}{\pars{Lectio II.} \scriptura{(Edit. M. Weingart in Svatováclavský sborník, I, Pragæ 1934, pag. 974-980)}

\noindent E Legénda prima palæosláva.

\noindent Cum pater eius Vratisláus mórtuus esset, constituérunt Bohémi Vencesláum príncipem. Et Dei grátia in fide perféctus erat. Omnibus enim paupéribus bene faciébat, nudos vestiébat, esuriéntes alébat, peregrínos excipiébat secúndum vocem evangélicam. Víduis non patiebátur iniúriam fíeri, hómines omnes, páuperes et dívites amábat, Deo serviéntibus ministrábat, ecclésias multas ornábat.

\noindent Supérbi vero facti sunt viri Bohémi et persuasérunt Bolesláo, fratri eius minóri, dicéntes: «Occisúrus te est frater Vencesláus conspírans cum matre et viris suis».}
\newcommand{\responsoriumii}{\pars{Responsorium 2.} \scriptura{\textbf{Codice Raygradensis 17, folio 201r}}

\vspace{-5mm}

\responsorium{I}{temporalia/resp-wencezlausduxgratiae.gtex}

\rubrica{vel ad libitum:}

\vspace{3mm}

\pars{Responsorium 2.} \scriptura{\Rbardot{} Os. 14, 6 \Vbardot{} Ps. 91, 13; \textbf{H373}}

\vspace{-5mm}

\responsorium{I}{temporalia/resp-justusgerminabit-CROCHU.gtex}{}}
\newcommand{\lectioiii}{\pars{Lectio III.}

\noindent Cum essent festivitátes dedicatiónum ecclesiárum in ómnibus civitátibus, Vencesláus visitábat omnes civitátes. Ingréssus est ígitur Bolesláviam civitátem die domínica, in festo Cosmæ et Damiáni. Audíta missa, vóluit Pragam revérti. Bolesláus autem retínuit eum scelésta mente dicens: «Cur abitúrus es, frater?». Mane facto campánam pulsavérunt ad offícium matutínum. Vencesláus vero, audíto campánæ sono, dixit: «Laus tibi, Dómine, qui dedísti mihi vívere ad hoc mane». Et surréxit et ivit ad offícium matutínum.

\noindent Statim assecútus est eum Bolesláus in iánua. Vencesláus respéxit ad eum et dixit: «Frater, bonus eras nobis fámulus heri». Bolesláo autem diábolus inclinávit se ad aures et pervértit cor eius et, evagináto gládio, respóndit ei dicens: «Nunc volo tibi mélior fíeri». His dictis, ferit caput eius gládio.

\noindent Vencesláus vero convérsus ad eum dixit: «Quid in ánimo habes, frater?». Et prehénsum humi eum prostrávit. Accúrrit autem quidam de consiliáriis Boleslái et percússit manum Venceslái. Hic manu vulnerátus, fratre dimísso, confúgit ad ecclésiam. Maléfici autem duo occidérunt eum in ecclésiæ iánua. Alius accúrrens, latus eius transfódit gládio. Vencesláus inde statim efflávit ánimam cum his verbis: \emph{In manus tuas, Dómine, comméndo spíritum meum.}}
\newcommand{\responsoriumiii}{\pars{Responsorium 3.} \scriptura{\textbf{Codice Raygradensis 17, folio 201v}}

\vspace{-5mm}

\responsorium{III}{temporalia/resp-noctesurgensagrum.gtex}{}

\rubrica{vel ad libitum:}

\vspace{3mm}

\pars{Responsorium 3.} \scriptura{\textbf{H373}}

\vspace{-5mm}

\responsorium{III}{temporalia/resp-istecognovit-CROCHU-cumdox.gtex}{}}
\newcommand{\hymnuslaudes}{\pars{Hymnus.}

\cuminitiali{VII}{temporalia/hym-DiesVenit.gtex}}
\newcommand{\laudes}{\pars{Psalmus 1.} \scriptura{\textbf{R17 205r}}

\vspace{-4mm}

\antiphona{I g}{temporalia/ant-laudemotus-cgp.gtex}

%\vspace{-2mm}

\scriptura{Psalmus 62.}

%\vspace{-1mm}

\initiumpsalmi{temporalia/ps62-initium-i-g.gtex}

%\vspace{-1.5mm}

\input{temporalia/ps62-i-g.tex} \Abardot{}

\vfill
\pagebreak

\pars{Psalmus 2.} \scriptura{\textbf{R17 206r}}

\vspace{-4mm}

\antiphona{IV E}{temporalia/ant-morbidisrefugium-cgp.gtex}

\vspace{-2mm}

%\trAntIV

\scriptura{Canticum trium puerorum, Dan. 3, 57-88 et 56}

\vspace{-2mm}

\initiumpsalmi{temporalia/dan3-initium-iv-E.gtex}

\input{temporalia/dan3-iv-E-sinedox.tex}

\rubrica{Hic non dicitur Gloria Patri, neque Amen.}

\vfill

\antiphona{}{temporalia/ant-morbidisrefugium-cgp.gtex}

\vfill
\pagebreak

\pars{Psalmus 3.} \scriptura{\textbf{R17 206r}}

\vspace{-4mm}

\antiphona{V a}{temporalia/ant-laudantdominum-cgp.gtex}

%\vspace{-2mm}

\scriptura{Psalmus 149}

%\vspace{-2mm}

\initiumpsalmi{temporalia/ps149-initium-v-a-auto.gtex}

\input{temporalia/ps149-v-a.tex} \Abardot{}

\vfill
\pagebreak}
\newcommand{\lectiobrevis}{\pars{Lectio Brevis.} \scriptura{Sir. 39, 6.8.12.13-14}

\noindent Cor suum tradet ad vigilándum dilúculo ad Dóminum, qui fecit illum; spíritu intellegéntiae replébitur. Collaudábunt multi sapiéntiam eius; non recédet memória eius, et nomen eius requirétur a generatióne in generatiónem. Sapiéntiam eius enarrábunt gentes, et laudem eius enuntiábit ecclésia.}
\newcommand{\responsoriumbreve}{\pars{Responsorium breve.} \scriptura{Ps. 20, 6}

\cuminitiali{VI}{temporalia/resp-magnaestgloria.gtex}}
\newcommand{\preces}{\noindent \noindent Fratres, Salvatórem nostrum, testem fidélem,~\gredagger{} per mártyres interféctos propter verbum Dei,~\grestar{} celebrémus, clamántes:

\Rbardot{} Redemísti nos Deo in sánguine tuo.

\noindent Per mártyres tuos, qui líbere mortem in testimónium fídei sunt ampléxi,~\grestar{} da nobis, Dómine, veram spíritus libertátem.

\Rbardot{} Redemísti nos Deo in sánguine tuo.

\noindent Per mártyres tuos, qui fidem usque ad sánguinem sunt conféssi,~\grestar{} da nobis, Dómine, puritátem fideíque constántiam.

\Rbardot{} Redemísti nos Deo in sánguine tuo.

\noindent Per mártyres tuos, qui, sustinéntes crucem, tua vestígia sunt secúti,~\grestar{} da nobis, Dómine, ærúmnas vitæ fórtiter sustinére.

\Rbardot{} Redemísti nos Deo in sánguine tuo.

\noindent Per mártyres tuos, qui stolas suas lavérunt in sánguine Agni,~\grestar{} da nobis, Dómine, omnes insídias carnis mundíque devíncere.

\Rbardot{} Redemísti nos Deo in sánguine tuo.}
\newcommand{\benedictus}{\pars{Canticum Zachariæ.}

\vspace{-6mm}

\antiphona{V a}{temporalia/ant-cordeetlingua-cgp.gtex}

\vspace{-3mm}

\scriptura{Lc. 1, 68-79}

\vspace{-2mm}

\initiumpsalmi{temporalia/benedictus-initium-v-a.gtex}

\vspace{-1.5mm}

\input{temporalia/benedictus-v-a.tex} \Abardot{}}
\newcommand{\benedicamuslaudes}{\cuminitiali{II}{temporalia/benedicamus-solemnism-laud.gtex}}
\newcommand{\hebdomada}{infra Hebdom. XXV per Annum.}
\newcommand{\matua}{Matutinum Hebdomadae A}
\newcommand{\matuac}{Matutinum Hebdomadae A vel C}
\newcommand{\lauda}{Laudes Hebdomadae A}
\newcommand{\laudac}{Laudes Hebdomadae A vel C}

% LuaLaTeX

\documentclass[a4paper, twoside, 12pt]{article}
\usepackage[latin]{babel}
%\usepackage[landscape, left=3cm, right=1.5cm, top=2cm, bottom=1cm]{geometry} % okraje stranky
%\usepackage[landscape, a4paper, mag=1166, truedimen, left=2cm, right=1.5cm, top=1.6cm, bottom=0.95cm]{geometry} % okraje stranky
\usepackage[landscape, a4paper, mag=1400, truedimen, left=0.5cm, right=0.5cm, top=0.5cm, bottom=0.5cm]{geometry} % okraje stranky

\usepackage{fontspec}
\setmainfont[FeatureFile={junicode.fea}, Ligatures={Common, TeX}, RawFeature=+fixi]{Junicode}
%\setmainfont{Junicode}

% shortcut for Junicode without ligatures (for the Czech texts)
\newfontfamily\nlfont[FeatureFile={junicode.fea}, Ligatures={Common, TeX}, RawFeature=+fixi]{Junicode}

% Hebrew font: http://scripts.sil.org/cms/scripts/page.php?site_id=nrsi&id=SILHebrUnic2
\newfontfamily\hebfont[Scale=1]{Ezra SIL}

\usepackage{multicol}
\usepackage{color}
\usepackage{lettrine}
\usepackage{fancyhdr}

% usual packages loading:
\usepackage{luatextra}
\usepackage{graphicx} % support the \includegraphics command and options
\usepackage{gregoriotex} % for gregorio score inclusion
\usepackage{gregoriosyms}
\usepackage{wrapfig} % figures wrapped by the text
\usepackage{parcolumns}
\usepackage[contents={},opacity=1,scale=1,color=black]{background}
\usepackage{tikzpagenodes}
\usepackage{calc}
\usepackage{longtable}
\usetikzlibrary{calc}

\setlength{\headheight}{14.5pt}

\input{conventuscommune.tex} % Often used macros

\newcommand{\annusEditionis}{2022}

\def\hebinitial#1{%
\leavevmode{\newbox\hebbox\setbox\hebbox\hbox{\hebfont{#1}\hskip 1mm}\kern -\wd\hebbox\hbox{\hebfont{#1}\hskip 1mm}}%
}

%%%% Vicekrat opakovane kousky

\newcommand{\anteOrationem}{
  \rubrica{Ante Orationem, cantatur a Superiore:}

  \pars{Supplicatio Litaniæ.}

  \cuminitiali{}{temporalia/supplicatiolitaniae.gtex}

  \pars{Oratio Dominica.}

  \cuminitiali{}{temporalia/oratiodominica.gtex}

  \rubrica{Deinde dicitur ab Hebdomadario:}

  \cuminitiali{}{temporalia/dominusvobiscum-solemnis.gtex}

  \rubrica{In choro monialium loco Dominus vobiscum dicitur:}

  \sineinitiali{temporalia/domineexaudi.gtex}
}

\setlength{\columnsep}{30pt} % prostor mezi sloupci

%%%%%%%%%%%%%%%%%%%%%%%%%%%%%%%%%%%%%%%%%%%%%%%%%%%%%%%%%%%%%%%%%%%%%%%%%%%%%%%%%%%%%%%%%%%%%%%%%%%%%%%%%%%%%
\begin{document}

% Here we set the space around the initial.
% Please report to http://home.gna.org/gregorio/gregoriotex/details for more details and options
\grechangedim{afterinitialshift}{2.2mm}{scalable}
\grechangedim{beforeinitialshift}{2.2mm}{scalable}
\grechangedim{interwordspacetext}{0.22 cm plus 0.15 cm minus 0.05 cm}{scalable}%
\grechangedim{annotationraise}{-0.2cm}{scalable}

% Here we set the initial font. Change 38 if you want a bigger initial.
% Emit the initials in red.
\grechangestyle{initial}{\color{red}\fontsize{38}{38}\selectfont}

\pagestyle{empty}

%%%% Titulni stranka
\begin{titulusOfficii}
\ifx\titulus\undefined
\nomenFesti{Sabbato \hebdomada{}}
\else
\titulus
\fi
\end{titulusOfficii}

\vfill

\begin{center}
%Ad usum et secundum consuetudines chori \guillemotright{}Conventus Choralis\guillemotleft.

%Editio Sancti Wolfgangi \annusEditionis
\end{center}

\scriptura{}

\pars{}

\pagebreak

\renewcommand{\headrulewidth}{0pt} % no horiz. rule at the header
\fancyhf{}
\pagestyle{fancy}

\cantusSineNeumas

\hora{Ad Matutinum.} %%%%%%%%%%%%%%%%%%%%%%%%%%%%%%%%%%%%%%%%%%%%%%%%%%%%%

\vspace{2mm}

\cuminitiali{}{temporalia/dominelabiamea.gtex}

\vfill
%\pagebreak

\vspace{2mm}

\ifx\invitatorium\undefined
\pars{Invitatorium.} \scriptura{Lc. 24, 34; Psalmus 94; \textbf{H232}}

\vspace{-4mm}

\antiphona{VI}{temporalia/inv-surrexitdominusvere.gtex}
\else
\invitatorium
\fi

\vfill
\pagebreak

\ifx\hymnusmatutinum\undefined
\pars{Hymnus.}

\cuminitiali{VIII}{temporalia/hym-LaetareCaelum.gtex}
\else
\hymnusmatutinum
\fi

\vspace{-3mm}

\vfill
\pagebreak

\ifx\matutinum\undefined
\ifx\matua\undefined
\else
% MAT A
\pars{Psalmus 1.}

\vspace{-4mm}

\antiphona{VIII G\textsuperscript{5}}{temporalia/ant-alleluia-turco15.gtex}

\vspace{-3mm}

\scriptura{Ps. 104, 1-15}

\vspace{-2mm}

\initiumpsalmi{temporalia/ps104i-initium-viii-g5.gtex}

\vspace{-1.5mm}

\input{temporalia/ps104i-viii-g.tex}

\vfill
\pagebreak

\pars{Psalmus 2.} \scriptura{Ps. 104, 16-27}

%\vspace{-2mm}

\initiumpsalmi{temporalia/ps104ii-initium-viii-g5.gtex}

\input{temporalia/ps104ii-viii-g.tex}

\vfill
\pagebreak

\pars{Psalmus 3.} \scriptura{Ps. 104, 28-45}

%\vspace{-2mm}

\initiumpsalmi{temporalia/ps104iii-initium-viii-g5.gtex}

\input{temporalia/ps104iii-viii-g.tex}

\vfill

\antiphona{}{temporalia/ant-alleluia-turco15.gtex}

\vfill
\pagebreak
\fi
\ifx\matub\undefined
\else
% MAT B
\pars{Psalmus 1.}

\vspace{-4mm}

\antiphona{t. pereg.}{temporalia/ant-alleluia-turco3.gtex}

%\vspace{-2mm}

\scriptura{Ps. 105, 1-15}

%\vspace{-2mm}

\initiumpsalmi{temporalia/ps105i-initium-per-auto.gtex}

\input{temporalia/ps105i-per.tex}

\vfill
\pagebreak

\pars{Psalmus 2.} \scriptura{Ps. 105, 16-31}

\vspace{-2.5mm}

\initiumpsalmi{temporalia/ps105ii-initium-per-auto.gtex}

\vspace{-1.5mm}

\input{temporalia/ps105ii-per.tex}

\vfill
\pagebreak

\pars{Psalmus 3.} \scriptura{Ps. 105, 32-48}

%\vspace{-2mm}

\initiumpsalmi{temporalia/ps105iii-initium-per-auto.gtex}

\input{temporalia/ps105iii-per.tex}

\vfill

\antiphona{}{temporalia/ant-alleluia-turco3.gtex}

\vfill
\pagebreak
\fi
\ifx\matuc\undefined
\else
% MAT C
\pars{Psalmus 1.} \scriptura{Ps. 106, 8}

\vspace{-4mm}

\antiphona{IV e}{temporalia/ant-alleluia-fo2.gtex}

%\vspace{-2mm}

\scriptura{Ps. 106, 1-14}

%\vspace{-2mm}

\initiumpsalmi{temporalia/ps106i-initium-iv-e2-auto.gtex}

\input{temporalia/ps106i-iv-e2.tex}

\vfill
\pagebreak

\pars{Psalmus 2.} \scriptura{Ps. 106, 15-30}

%\vspace{-2mm}

\initiumpsalmi{temporalia/ps106ii-initium-iv-e2-auto.gtex}

\input{temporalia/ps106ii-iv-e2.tex}

\vfill
\pagebreak

\pars{Psalmus 3.} \scriptura{Ps. 106, 31-43}

%\vspace{-2mm}

\initiumpsalmi{temporalia/ps106iii-initium-iv-e2-auto.gtex}

\input{temporalia/ps106iii-iv-e2.tex}

\vfill
\pagebreak

\antiphona{}{temporalia/ant-alleluia-fo2.gtex}

\vfill
\pagebreak
\fi
\ifx\matud\undefined
\else
% MAT D
\pars{Psalmus 1.}

\vspace{-4mm}

\antiphona{III g}{temporalia/ant-alleluia-turco26.gtex}

%\vspace{-2mm}

\scriptura{Ps. 77, 40-51}

%\vspace{-2mm}

\initiumpsalmi{temporalia/ps77xl_li-initium-iii-g-auto.gtex}

\input{temporalia/ps77xl_li-iii-g.tex}

\vfill
\pagebreak

\pars{Psalmus 2.} \scriptura{Ps. 77, 52-64}

\vspace{-2mm}

\initiumpsalmi{temporalia/ps77lii_lxiv-initium-iii-g-auto.gtex}

\input{temporalia/ps77lii_lxiv-iii-g.tex}

\vfill
\pagebreak

\pars{Psalmus 3.} \scriptura{Ps. 77, 65-72}

%\vspace{-2mm}

\initiumpsalmi{temporalia/ps77lxv_lxxii-initium-iii-g-auto.gtex}

\input{temporalia/ps77lxv_lxxii-iii-g.tex}

\vfill

\antiphona{}{temporalia/ant-alleluia-turco26.gtex}

\vfill
\pagebreak
\fi
\else
\matutinum
\fi

\pars{Versus.}

\ifx\matversus\undefined
\noindent \Vbardot{} Deus regenerávit nos in spem vivam, allelúia.

\noindent \Rbardot{} Per resurrectiónem Iesu Christi ex mórtuis, allelúia.
\else
\matversus
\fi

\vspace{5mm}

\sineinitiali{temporalia/oratiodominica-mat.gtex}

\vspace{5mm}

\pars{Absolutio.}

\cuminitiali{}{temporalia/absolutio-avinculis.gtex}

\vfill
\pagebreak

\cuminitiali{}{temporalia/benedictio-solemn-ille.gtex}

\vspace{7mm}

\lectioi

\noindent \Vbardot{} Tu autem, Dómine, miserére nobis.
\noindent \Rbardot{} Deo grátias.

\vfill
\pagebreak

\responsoriumi

\vfill
\pagebreak

\cuminitiali{}{temporalia/benedictio-solemn-divinum.gtex}

\vspace{7mm}

\lectioii

\noindent \Vbardot{} Tu autem, Dómine, miserére nobis.
\noindent \Rbardot{} Deo grátias.

\vfill
\pagebreak

\responsoriumii

\vfill
\pagebreak

\cuminitiali{}{temporalia/benedictio-solemn-adsocietatem.gtex}

\vspace{7mm}

\lectioiii

\noindent \Vbardot{} Tu autem, Dómine, miserére nobis.
\noindent \Rbardot{} Deo grátias.

\vfill
\pagebreak

\responsoriumiii

\vfill
\pagebreak

\rubrica{Reliqua omittuntur, nisi Laudes separandæ sint.}

\sineinitiali{temporalia/domineexaudi.gtex}

\vfill

\oratio

\vfill

\noindent \Vbardot{} Dómine, exáudi oratiónem meam.
\Rbardot{} Et clamor meus ad te véniat.

\vfill

\noindent \Vbardot{} Benedicámus Dómino.
\noindent \Rbardot{} Deo grátias.

\vfill

\noindent \Vbardot{} Fidélium ánimæ per misericórdiam Dei requiéscant in pace.
\Rbardot{} Amen.

\vfill
\pagebreak

\hora{Ad Laudes.} %%%%%%%%%%%%%%%%%%%%%%%%%%%%%%%%%%%%%%%%%%%%%%%%%%%%%

\cantusSineNeumas

\vspace{0.5cm}
\grechangedim{interwordspacetext}{0.18 cm plus 0.15 cm minus 0.05 cm}{scalable}%
\cuminitiali{}{temporalia/deusinadiutorium-communis.gtex}
\grechangedim{interwordspacetext}{0.22 cm plus 0.15 cm minus 0.05 cm}{scalable}%

\vfill
\pagebreak

\ifx\hymnuslaudes\undefined
\ifx\laudac\undefined
\else
\pars{Hymnus}

\cuminitiali{I}{temporalia/hym-ChorusNovae-praglia.gtex}
\vspace{-3mm}
\fi
\ifx\laudbd\undefined
\else
\pars{Hymnus}

\cuminitiali{I}{temporalia/hym-ChorusNovae.gtex}
\vspace{-3mm}
\fi
\else
\hymnuslaudes
\fi

\vfill
\pagebreak

\ifx\laudes\undefined
\ifx\lauda\undefined
\else
\pars{Psalmus 1.}

\vspace{-4mm}

\antiphona{VII a}{temporalia/ant-alleluia-turco29.gtex}

\scriptura{Psalmus 118, 145-152; \hspace{5mm} \hebinitial{ק}}

\initiumpsalmi{temporalia/ps118xix-initium-vii-a-auto.gtex}

\input{temporalia/ps118xix-vii-a.tex} \Abardot{}

\vfill
\pagebreak

\pars{Psalmus 2.} \scriptura{Ex. 15, 2}

\vspace{-4mm}

\antiphona{IV e}{temporalia/ant-fortitudomeaetlausmea.gtex}

\scriptura{Canticum Moysis, Ex. 15, 1-4a.7b-13.17-19}

\initiumpsalmi{temporalia/moysis1-initium-iv-e2-auto.gtex}

\input{temporalia/moysis1-iv-e2.tex}

\antiphona{}{temporalia/ant-fortitudomeaetlausmea.gtex}

\vfill
\pagebreak

\pars{Psalmus 3.}

\vspace{-4mm}

\antiphona{E}{temporalia/ant-alleluia-praglia-e2.gtex}

\scriptura{Psalmus 116.}

\initiumpsalmi{temporalia/ps116-initium-e-auto.gtex}

\input{temporalia/ps116-e.tex} \Abardot{}

\vfill
\pagebreak
\fi
\ifx\laudb\undefined
\else
\pars{Psalmus 1.}

\vspace{-4.5mm}

\antiphona{E}{temporalia/ant-alleluia-praglia-e2.gtex}

\vspace{-3mm}

\scriptura{Psalmus 91.}

\vspace{-2mm}

\initiumpsalmi{temporalia/ps91-initium-e-auto.gtex}

\vspace{-1.5mm}

\input{temporalia/ps91-e.tex} \Abardot{}

\vfill
\pagebreak

\pars{Psalmus 2.} \scriptura{Eccli. 39, 19}

\vspace{-4mm}

\antiphona{VII c\textsuperscript{2}}{temporalia/ant-effrondeteingratia.gtex}

\vspace{-2mm}

\scriptura{Canticum Moysi, Dt. 32, 1-32}

\vspace{-2mm}

\initiumpsalmi{temporalia/moysis2i_xii-initium-vii-c2-auto.gtex}

\input{temporalia/moysis2i_xii-vii-c2.tex}

\vfill

\antiphona{}{temporalia/ant-effrondeteingratia.gtex}

\vfill
\pagebreak

\pars{Psalmus 3.}

\vspace{-4mm}

\antiphona{I a\textsuperscript{2}}{temporalia/ant-alleluia-turco23.gtex}

%\vspace{-2mm}

\scriptura{Ps. 8}

%\vspace{-2mm}

\initiumpsalmi{temporalia/ps8-initium-i-a2-auto.gtex}

\input{temporalia/ps8-i-a2.tex} \Abardot{}

\vfill
\pagebreak
\fi
\ifx\laudc\undefined
\else
\pars{Psalmus 1.}

\vspace{-4mm}

\antiphona{E}{temporalia/ant-alleluia-praglia-e2.gtex}

%\vspace{-2mm}

\scriptura{Psalmus 118, 145-152.}

%\vspace{-2mm}

\initiumpsalmi{temporalia/ps118xix-initium-e-auto.gtex}

%\vspace{-1.5mm}

\input{temporalia/ps118xix-e.tex} \Abardot{}

\vfill
\pagebreak

\pars{Psalmus 2.}

\vspace{-4mm}

\antiphona{V a}{temporalia/ant-mecumsitdomine-tp.gtex}

%\vspace{-2mm}

\scriptura{Canticum Sapientiæ, Sap. 9, 1-6.9-11}

\initiumpsalmi{temporalia/sapientia-initium-v-a-auto.gtex}

\input{temporalia/sapientia-v-a.tex} \Abardot{}

\vfill
\pagebreak

\pars{Psalmus 3.}

\vspace{-4mm}

\antiphona{II* a}{temporalia/ant-alleluia-turco18.gtex}

%\vspace{-2mm}

\scriptura{Ps. 116}

%\vspace{-2mm}

\initiumpsalmi{temporalia/ps116-initium-ii_-a-auto.gtex}

\input{temporalia/ps116-ii_-a.tex} \Abardot{}

\vfill
\pagebreak
\fi
\ifx\laudd\undefined
\else
\pars{Psalmus 1.}

\vspace{-4.5mm}

\antiphona{VIII G\textsuperscript{2}}{temporalia/ant-alleluia-turco12.gtex}

\vspace{-3mm}

\scriptura{Psalmus 91.}

\vspace{-2mm}

\initiumpsalmi{temporalia/ps91-initium-viii-G5-auto.gtex}

\vspace{-1.5mm}

\input{temporalia/ps91-viii-G5.tex} \Abardot{}

\vfill
\pagebreak

\pars{Psalmus 2.} \scriptura{Heb. 13, 8}

\vspace{-4mm}

\antiphona{II D}{temporalia/ant-iesuschristusheriethodie.gtex}

%\vspace{-2mm}

\scriptura{Canticum Ezechiæ, Ez. 36, 24-28}

\initiumpsalmi{temporalia/ezechiae2-initium-ii-D-auto.gtex}

\input{temporalia/ezechiae2-ii-D.tex} \Abardot{}

\vfill
\pagebreak

\pars{Psalmus 3.}

\vspace{-4mm}

\antiphona{I a\textsuperscript{2}}{temporalia/ant-alleluia-turco23.gtex}

%\vspace{-2mm}

\scriptura{Ps. 8}

%\vspace{-2mm}

\initiumpsalmi{temporalia/ps8-initium-i-a4-auto.gtex}

\input{temporalia/ps8-i-a4.tex} \Abardot{}

\vfill
\pagebreak
\fi
\else
\laudes
\fi

\ifx\lectiobrevis\undefined
\pars{Lectio Brevis.} \scriptura{Rom. 14, 7-9}

\noindent Nemo nostrum sibi vivit et nemo sibi móritur; sive enim vívimus, Dómino vívimus, sive mórimur, Dómino mórimur. Sive ergo vívimus, sive mórimur, Dómini sumus. In hoc enim Christus et mórtuus est et vixit, ut et mortuórum et vivórum dominétur.
\else
\lectiobrevis
\fi

\vfill

\ifx\responsoriumbreve\undefined
\pars{Responsorium breve.} \scriptura{Cf. Mt. 28, 6; Cf. Gal. 3, 13}

\cuminitiali{VI}{temporalia/resp-surrexitdominusdesepulcro.gtex}
\else
\responsoriumbreve
\fi

\vfill
\pagebreak

\benedictus

\vspace{-1cm}

\vfill
\pagebreak

\ifx\precestotum\undefined
\pars{Preces.}

\sineinitiali{}{temporalia/tonusprecumnovum.gtex}

\ifx\preces\undefined
\ifx\lauda\undefined
\else
\noindent Christum, panem vitæ, \gredagger{} qui mensa verbi et córporis sui fruéntes suscitábit in novíssimo die, \grestar{} læti deprecémur:

\Rbardot{} Da nobis, Dómine, pacem et gáudium.

\noindent Fili Dei, qui, suscitátus a mórtuis, princeps es vitæ, \grestar{} nos omnésque fratres tuos bénedic et sanctífica.

\Rbardot{} Da nobis, Dómine, pacem et gáudium.

\noindent Tu, qui pacem et gáudium ómnibus in te credéntibus largíris, \grestar{} da nos sicut fílios lucis ambuláre et de victória tua lætári.

\Rbardot{} Da nobis, Dómine, pacem et gáudium.

\noindent Adáuge fidem Ecclésiæ peregrinántis in terra, \grestar{} ut resurrectiónis tuæ testimónium mundo perhíbeat.

\Rbardot{} Da nobis, Dómine, pacem et gáudium.

\noindent Tu qui, multa passus, \gredagger{} in glóriam Patris intrásti, \grestar{} luctum mæréntium convérte in gáudium.

\Rbardot{} Da nobis, Dómine, pacem et gáudium.
\fi
\ifx\laudb\undefined
\else
\noindent Christum, qui vitam ætérnam nobis manifestávit, \grestar{} devóta mente rogémus, clamántes:

\Rbardot{} Resurréctio tua locuplétet nos grátia, Dómine.

\noindent Pastor ætérne, \gredagger{} réspice gregem tuum e somno surgéntem \grestar{} et pasce nos verbi et panis tui ubérrimo alimónio.

\Rbardot{} Resurréctio tua locuplétet nos grátia, Dómine.

\noindent Ne permíttas nos a lupo rapi vel a mercenário perdi, \grestar{} sed fac, ut vocem tuam fidéliter audiámus.

\Rbardot{} Resurréctio tua locuplétet nos grátia, Dómine.

\noindent Tu, qui cum prædicatóribus ubíque cooperáris eorúmque sermónem confírmas, \grestar{} fac, ut hódie resurrectiónem tuam móribus et vita proclamémus.

\Rbardot{} Resurréctio tua locuplétet nos grátia, Dómine.

\noindent Esto ipse gáudium nostrum, \gredagger{} quod nemo tollat a nobis, \grestar{} ut, reiécta tristítia peccáti, vitam appetámus ætérnam.

\Rbardot{} Resurréctio tua locuplétet nos grátia, Dómine.
\fi
\ifx\laudc\undefined
\else
\noindent Christum, panem vitæ, \gredagger{} qui mensa verbi et córporis sui fruéntes suscitábit in novíssimo die, \grestar{} læti deprecémur:

\Rbardot{} Da nobis, Dómine, pacem et gáudium.

\noindent Fili Dei, qui, suscitátus a mórtuis, princeps es vitæ, \grestar{} nos omnésque fratres tuos bénedic et sanctífica.

\Rbardot{} Da nobis, Dómine, pacem et gáudium.

\noindent Tu, qui pacem et gáudium ómnibus in te credéntibus largíris, \grestar{} da nos sicut fílios lucis ambuláre et de victória tua lætári.

\Rbardot{} Da nobis, Dómine, pacem et gáudium.

\noindent Adáuge fidem Ecclésiæ peregrinántis in terra, \grestar{} ut resurrectiónis tuæ testimónium mundo perhíbeat.

\Rbardot{} Da nobis, Dómine, pacem et gáudium.

\noindent Tu qui, multa passus, \gredagger{} in glóriam Patris intrásti, \grestar{} luctum mæréntium convérte in gáudium.

\Rbardot{} Da nobis, Dómine, pacem et gáudium.
\fi
\ifx\laudd\undefined
\else
\noindent Christum, qui vitam ætérnam nobis manifestávit, \grestar{} devóta mente rogémus, clamántes:

\Rbardot{} Resurréctio tua locuplétet nos grátia, Dómine.

\noindent Pastor ætérne, \gredagger{} réspice gregem tuum e somno surgéntem \grestar{} et pasce nos verbi et panis tui ubérrimo alimónio.

\Rbardot{} Resurréctio tua locuplétet nos grátia, Dómine.

\noindent Ne permíttas nos a lupo rapi vel a mercenário perdi, \grestar{} sed fac, ut vocem tuam fidéliter audiámus.

\Rbardot{} Resurréctio tua locuplétet nos grátia, Dómine.

\noindent Tu, qui cum prædicatóribus ubíque cooperáris eorúmque sermónem confírmas, \grestar{} fac, ut hódie resurrectiónem tuam móribus et vita proclamémus.

\Rbardot{} Resurréctio tua locuplétet nos grátia, Dómine.

\noindent Esto ipse gáudium nostrum, \gredagger{} quod nemo tollat a nobis, \grestar{} ut, reiécta tristítia peccáti, vitam appetámus ætérnam.

\Rbardot{} Resurréctio tua locuplétet nos grátia, Dómine.
\fi
\else
\preces
\fi

\vfill

\pars{Oratio Dominica.}

\cuminitiali{}{temporalia/oratiodominicaalt.gtex}

\vfill
\pagebreak

\rubrica{vel:}

\pars{Deprecatio Gelasii}

\vspace{-5mm}

\grechangedim{interwordspacetext}{0.16 cm plus 0.15 cm minus 0.05 cm}{scalable}%
\antiphona{D\textsuperscript{1}}{temporalia/deprecatio4-propace.gtex}
\grechangedim{interwordspacetext}{0.22 cm plus 0.15 cm minus 0.05 cm}{scalable}%

\vfill

\pars{Oratio Dominica.}

\cuminitiali{D}{temporalia/oratiodominica-d.gtex}
\else
\precestotum
\fi

\vfill
\pagebreak

% Oratio. %%%
\oratio

\vspace{-1mm}

\vfill

\rubrica{Hebdomadarius dicit Dominus vobiscum, vel, absente sacerdote vel diacono, sic concluditur:}

\vspace{2mm}

\ifx\dominusnosbenedicat\undefined
\antiphona{C}{temporalia/dominusnosbenedicat.gtex}
\else
\dominusnosbenedicat
\fi

\rubrica{Postea cantatur a cantore:}

\vspace{2mm}

\cuminitiali{VII}{temporalia/benedicamus-tempore-paschali.gtex}

\vspace{1mm}

\vfill
\pagebreak

\end{document}

