% LuaLaTeX

\documentclass[a4paper, twoside, 12pt]{article}
\usepackage[latin]{babel}
%\usepackage[landscape, left=3cm, right=1.5cm, top=2cm, bottom=1cm]{geometry} % okraje stranky
%\usepackage[landscape, a4paper, mag=1166, truedimen, left=2cm, right=1.5cm, top=1.6cm, bottom=0.95cm]{geometry} % okraje stranky
\usepackage[landscape, a4paper, mag=1400, truedimen, left=0.5cm, right=0.5cm, top=0.5cm, bottom=0.5cm]{geometry} % okraje stranky

\usepackage{fontspec}
\setmainfont[FeatureFile={junicode.fea}, Ligatures={Common, TeX}, RawFeature=+fixi]{Junicode}
%\setmainfont{Junicode}

% shortcut for Junicode without ligatures (for the Czech texts)
\newfontfamily\nlfont[FeatureFile={junicode.fea}, Ligatures={Common, TeX}, RawFeature=+fixi]{Junicode}

% Hebrew font: http://scripts.sil.org/cms/scripts/page.php?site_id=nrsi&id=SILHebrUnic2
\newfontfamily\hebfont[Scale=1]{Ezra SIL}

\usepackage{multicol}
\usepackage{color}
\usepackage{lettrine}
\usepackage{fancyhdr}

% usual packages loading:
\usepackage{luatextra}
\usepackage{graphicx} % support the \includegraphics command and options
\usepackage{gregoriotex} % for gregorio score inclusion
\usepackage{gregoriosyms}
\usepackage{wrapfig} % figures wrapped by the text
\usepackage{parcolumns}
\usepackage[contents={},opacity=1,scale=1,color=black]{background}
\usepackage{tikzpagenodes}
\usepackage{calc}
\usepackage{longtable}
\usetikzlibrary{calc}

\setlength{\headheight}{14.5pt}

\input{conventuscommune.tex} % Often used macros
%%%% Preklady jednotlivych zpevu (nektere se opakuji, a je dobre mit je
% vsechny na jedne hromade)

% HOURS ---

\newcommand{\trAntI}{\translatioCantus{Muž boží měl kožený toulec, pečlivě
zavázaný, jenž mu visel na šíji a~často se ho dotýkal.}}

\newcommand{\trAntII}{\translatioCantus{Klíč od~něho tak dobře střežil, že
dokud žil v~těle, nikdo z~jeho žáků nezvěděl, co je uvnitř.}}

\newcommand{\trAntIII}{\translatioCantus{Ale když se odebral z~tohoto
života, schránku otevřeli a~objevili v~ní žíněné roucho a~měděný řetěz
potřísněný krví.}}

\newcommand{\trAntIV}{\translatioCantus{A když prohlédli mistrovo tělo,
nalezli jeho tělo na čtyřech místech hluboce zbrázděno ranami od řetězu.}}

\newcommand{\trAntV}{\translatioCantus{Krev vytékající z~těch ran, místy
prostoupila i~žíněným rouchem.}}

\newcommand{\trCapituli}{\translatioCantus{
Miláčkovi Boha a~lidí,
Mojžíšovi požehnané paměti,~\gredagger{}
dopřál slávu rovnou slávě svatých~\grestar{}
učinil ho mocným na postrach nepřátelům
a~jeho slovy zastavil divy.}}

\newcommand{\trLectioBrevis}{\translatioCantus{
Pamatujte na své představené,
kteří vám hlásali Boží slovo.
Uvažte, jak oni skončili život, a~napodobujte jejich víru.
Ježíš Kristus je stejný včera i~dnes i~navěky.
Nenechte se svést věelijakými cizími naukami.}}

\newcommand{\trRespLaud}{\translatioCantus{Spravedlivého vodil Hospodin~\grestar{}
po přímých stezkách. \Vbardot{} A~ukázal mu Boží království.}}

\newcommand{\trRespLaudB}{\translatioCantus{Na tvých hradbách, Jeruzaléme,
ustanovil jsem strážné;~\grestar{}
budou bdít nad mým lidem. \Vbardot{} Ani ve dne, ani v~noci nesmějí nikdy
mlčet.}}

\newcommand{\trVersus}{\translatioCantus{\Vbardot{} Ústa spravedlivého šeptají moudrost, aleluja.
\Rbardot{} A~jeho jazyk ohlašuje právo, aleluja.}}

\newcommand{\trAntBenedictus}{\translatioCantus{Když na bujné oře vložili
nosítka a~sňali jim uzdu, vydali se přímo k~cele božího muže.}}

\newcommand{\trPreces}{\translatioCantus{
\noindent S vděčností chvalme Krista, dobrého Pastýře, \gredagger{} který dal život za své ovce, \grestar{} a~pokorně ho prosme: \Rbardot{} Pane, buď pastýřem svého lidu.

\noindent Kriste, ty dáváš církvi pastýře, a~jejich službou se ujímáš svého lidu, \grestar{} dej, ať v~lásce těch, kteří nás vedou, poznáváme, jak nás miluješ. \Rbardot{} Pane, buď pastýřem svého lidu.

\noindent Ty stále konáš skrze své zástupce službu pastýře a~učitele, \grestar{} nepřestávej nás nikdy vést prostřednictvím svých služebníků. \Rbardot{} Pane, buď pastýřem svého lidu.

\noindent Ty prokazuješ svému lidu skrze jeho pastýře službu lékaře duše i~těla, \grestar{} ochraňuj náš život a~veď nás ke svatosti. \Rbardot{} Pane, buď pastýřem svého lidu.

\noindent Ty posíláš své svaté, aby slovem i~příkladem vedli tvůj lid k~tobě, \grestar{} na jejich přímluvu nás posiluj, abychom vytrvali na cestě, která vede k~věčnému životu. \Rbardot{} Pane, buď pastýřem svého lidu.}}

\newcommand{\trOrationis}{\translatioCantus{Bože, jenž nám dopřáváš radovat
se z~výroční slavnosti svatého tvého vyznavače Havla, uděl dobrotivě,
abychom když slavíme jeho narození, též se řídili podobou jeho skutků.
Skrze…}}
 % Czech translations of the proper texts

\newcommand{\annusEditionis}{2020}

%%%% Vicekrat opakovane kousky

\newcommand{\anteOrationem}{
  \rubrica{Ante Orationem, cantatur a Superiore:}

  \pars{Supplicatio Litaniæ.}

  \cuminitiali{}{temporalia/supplicatiolitaniae.gtex}

  \pars{Oratio Dominica.}

  \cuminitiali{}{temporalia/oratiodominica.gtex}

  \rubrica{Deinde dicitur ab Hebdomadario:}

  \cuminitiali{}{temporalia/dominusvobiscum-solemnis.gtex}

  \rubrica{In choro monialium loco Dominus vobiscum dicitur:}

  \sineinitiali{temporalia/domineexaudi.gtex}
}

\newcommand{\tuAutem}{
  \vfill

  \sineinitiali{temporalia/tuautem.gtex}
}

\setlength{\columnsep}{30pt} % prostor mezi sloupci

%%%%%%%%%%%%%%%%%%%%%%%%%%%%%%%%%%%%%%%%%%%%%%%%%%%%%%%%%%%%%%%%%%%%%%%%%%%%%%%%%%%%%%%%%%%%%%%%%%%%%%%%%%%%%
\begin{document}

% Here we set the space around the initial.
% Please report to http://home.gna.org/gregorio/gregoriotex/details for more details and options
\grechangedim{afterinitialshift}{2.2mm}{scalable}
\grechangedim{beforeinitialshift}{2.2mm}{scalable}
\grechangedim{interwordspacetext}{0.22 cm plus 0.15 cm minus 0.05 cm}{scalable}%
\grechangedim{annotationraise}{-2mm}{scalable}

% Here we set the initial font. Change 38 if you want a bigger initial.
% Emit the initials in red.
\grechangestyle{initial}{\color{red}\fontsize{38}{38}\selectfont}

\pagestyle{empty}

%%%% Titulni stranka
\begin{titulusOfficii}
\nomenFesti{In Festo Pentecostes.}
\celebratio{Duplex 1. classis.} % puvodne "cum Octava." Oktavy byly ale zruseny
\end{titulusOfficii}

\vfill
\pagebreak

\renewcommand{\headrulewidth}{0pt} % no horiz. rule at the header
\fancyhf{}
\pagestyle{fancy}

\pars{Oratio ante divinum Officium.}

\lettrine{{\color{red}A}}{peri,} Dómine, os meum ad benedicéndum nomen sanctum tuum:
munda quoque cor meum ab ómnibus vanis, pervérsis, et aliénis
cogitatiónibus:
intelléctum illúmina, afféctum inflámma,
ut digne, atténte ac devóte hoc Offícium recitáre váleam,
et exaudíri mérear ante conspéctum Divínæ Maiestátis tuæ.
Per Christum, Dóminum nostrum.
\Rbardot{} Amen.

Dómine, in unióne illíus divínæ intentiónis,
qua ipse in terris laudes Deo persolvísti,
has tibi Horas \rubricatum{(vel \textnormal{hanc tibi Horam})} persólvo.

\vfill

\pars{Oratio post divinum Officium.}

\rubrica{
  Orationem sequentem devote post Officium recitantibus
  Leo Papa X. defectus, et culpas in eo persolvendo ex humana
  fragilitate contractas, indulsit, et dicitur flexis genibus.
}

\lettrine{{\color{red}S}}{acrosánctæ} et indivíduæ Trinitáti,
crucifíxi Dómini nostri Iesu Christi humanitáti,
beatíssimæ et gloriosíssimæ sempérque Vírginis Maríæ
fecúndæ integritáti, 
et ómnium Sanctórum universitáti
sit sempitérna laus, honor, virtus et glória
ab omni creatúra,
nobísque remíssio ómnium peccatórum,
per infiníta sǽcula sæculórum.
\Rbardot{} Amen.

\noindent \Vbardot{} Beáta víscera Maríæ Virginis, quæ portavérunt
ætérni Patris Fílium.\\
\Rbardot{} Et beáta úbera, quæ lactavérunt Christum Dominum.

\rubrica{Et dicitur secreto \textnormal{Pater noster.} et \textnormal{Ave María.}}

\vfill

\hora{In Vesperis.} %%%%%%%%%%%%%%%%%%%%%%%%%%%%%%%%%%%%%%%%%%%%%%%%%%%%%
%\sideThumbs{Vesperæ}

\cantusSineNeumas

\cuminitiali{}{temporalia/deusinadiutorium-solemnis.gtex}

\vfill
\pagebreak

\cantusCumNeumis

\pars{Psalmus 1.} \scriptura{Act. 2, 1; \textbf{H270}}

\antiphona{III a\textsuperscript{2}}{temporalia/ant1.gtex}

\cantusSineNeumas

\scriptura{Ps. 109}

\initiumpsalmi{temporalia/ps109-initium-iii-a2-auto.gtex}

%\psalmusEtTranslatioT{temporalia/ps109-comb.tex}{10cm}
\input{temporalia/ps109.tex} \Abardot{}

\vfill
\pagebreak

\pars{Psalmus 2.} \scriptura{Sap. 1, 7; \textbf{H270}}

\vspace{-4mm}

\antiphona{VIII G}{temporalia/ant2.gtex}

\scriptura{Ps. 110}

\initiumpsalmi{temporalia/ps110-initium-viii-G-auto.gtex}

%\psalmusEtTranslatioT{temporalia/ps110-comb.tex}{10cm}
\input{temporalia/ps110.tex} \Abardot{}

\vfill
\pagebreak

\pars{Psalmus 3.} \scriptura{Act. 2, 4; \textbf{H270}}

\antiphona{VIII G}{temporalia/ant3.gtex}

\scriptura{Ps. 111}

\initiumpsalmi{temporalia/ps111-initium-viii-G-auto.gtex}

%\psalmusEtTranslatioT{temporalia/ps111-comb.tex}{10cm}
\input{temporalia/ps111.tex} \Abardot{}

\vfill
\pagebreak

\pars{Psalmus 4.} \scriptura{Dan. 3, 77.79; \textbf{H270}}

\antiphona{I a\textsuperscript{2}}{temporalia/ant4.gtex}

\scriptura{Ps. 112}

\initiumpsalmi{temporalia/ps112-initium-i-a2-auto.gtex}

%\psalmusEtTranslatioT{temporalia/ps112-comb.tex}{10cm}
\input{temporalia/ps112.tex} \Abardot{}

\vfill
\pagebreak

\rubrica{In I. Vesperis:}

\pars{Psalmus 5.} \scriptura{\textbf{H271}}

\antiphona{VII c\textsuperscript{2}}{temporalia/ant5.gtex}

\scriptura{Ps. 116}

\initiumpsalmi{temporalia/ps116-initium-vii-c2-auto.gtex}

%\psalmusEtTranslatioT{temporalia/ps116-comb.tex}{10cm}
\input{temporalia/ps116.tex} \Abardot{}

\vfill
\pagebreak

\rubrica{In II. Vesperis:}

\pars{Psalmus 5.} \scriptura{\textbf{H271}}

\antiphona{VII c\textsuperscript{2}}{temporalia/ant5.gtex}

\scriptura{Ps. 113}

\initiumpsalmi{temporalia/ps113-initium-vii-c2-auto.gtex}

%\psalmusEtTranslatioT{temporalia/ps113-comb.tex}{10cm}
\input{temporalia/ps113.tex} \Abardot{}

\antiphona{}{temporalia/ant5.gtex} % repeat the antiphon - new page

\vfill
\pagebreak

\raggedcolumns

% Capitulum. %%%
\cantusSineNeumas

\pars{Capitulum.} \scriptura{Act. 2, 1 - 2}

\cuminitiali{}{temporalia/capitulum-CumComplerentur.gtex}

\vfill
\pars{Responsorium breve.}\rubrica{ - in I. vesperis:} \scriptura{\textbf{H269}}

\cuminitiali{VI}{temporalia/resp1v.gtex}

\vfill
\pagebreak

\pars{Responsorium breve.}\rubrica{ - in II. vesperis:}

\cuminitiali{VI}{temporalia/resp2v.gtex}

\vfill
\pagebreak

% Hymnus. %%%
\pars{Hymnus.}

\cuminitiali{VIII}{temporalia/hym-VeniCreator.gtex}
%\input{hym-VeniCreator-bohtext.tex}

\vfill
\pagebreak

\pars{Versus.}\rubrica{ - in I. vesperis:}

% Versus. %%%
\sineinitiali{temporalia/versus-repleti.gtex}

\vfill

\pars{Versus.}\rubrica{ - in II. vesperis:}

% Versus. %%%
\sineinitiali{temporalia/versus-loquebantur.gtex}

\vfill
\pagebreak

\cantusCumNeumis

\pars{Canticum B. Mariæ V.}\rubrica{ - in I. vesperis:} \scriptura{Io. 14, 18.28; ibid. 16, 22; \textbf{H267}}

\antiphona{I D\textsuperscript{*}}{temporalia/ant-magn-vesp1.gtex}

\vfill

\scriptura{Lc. 1, 46-55}

\cantusSineNeumas
\initiumpsalmi{temporalia/magnificat-initium-isoll-D_.gtex}

%\psalmusEtTranslatioT{temporalia/magnificat-comb.tex}{10.3cm}
\input{temporalia/magnificat.tex} \Abardot{}

%\antiphona{}{temporalia/ant-magn-vesp1.gtex} % repeat the antiphon - new page

\vfill
\pagebreak

\pars{Canticum B. Mariæ V.}\rubrica{ - in II. vesperis:} \scriptura{\textbf{Sg. 388 p. 247}}

\cantusCumNeumis

\antiphona{I D\textsuperscript{*}}{temporalia/ant-magn-vesp2.gtex}

\vfill

\scriptura{Lc. 1, 46-55}

\cantusSineNeumas
\initiumpsalmi{temporalia/magnificat-initium-isoll-D_.gtex}

% maly svindl, ale prizvukova struktura VIIIsoll-G2 a Isoll-D* je stejna
%\psalmusEtTranslatioT{temporalia/magnificat-comb.tex}{10.3cm}
\input{temporalia/magnificat.tex} \Abardot{}

\antiphona{}{temporalia/ant-magn-vesp2.gtex} % repeat the antiphon - new page

\vfill
\pagebreak

\anteOrationem

\pagebreak

% Oratio. %%%
\pars{Oratio.}

\cuminitiali{}{temporalia/oratio.gtex}

\vspace{1cm}
\rubrica{Hebdomadarius dicit iterum Dominus vobiscum. Postea cantatur a cantore:}
\vspace{2mm}

\rubrica{In I. Vesperis:}

\cuminitiali{II}{temporalia/benedicamus-solemnis-1vesp.gtex}

\rubrica{In II. Vesperis:}

\cuminitiali{VI}{temporalia/benedicamus-solemnis-2vesp.gtex}

\vfill
\pagebreak

\hora{Ad Completorium.} %%%%%%%%%%%%%%%%%%%%%%%%%%%%%%%%%%%%%%%%%%%%%%%%%%%%%%%%%%
%\sideThumbs{{\scriptsize{}Completorium}}

\rubrica{Lector petit benedictionem, dicens:}

\cuminitiali{}{temporalia/jubedomnebenedicere.gtex}

\vfill

\pars{Benedictio.}

\cuminitiali{}{temporalia/benedictio-noctemquietam.gtex}

\vfill

\pars{Lectio brevis.} \scriptura{1Ptr. 5, 8-9}

\cuminitiali{}{temporalia/lectiobrevis-fratressobrii.gtex}

\vfill

\noindent \Vbardot{} Adiutórium nostrum in nómine Dómini. \Rbardot{} Qui fecit cælum, et terram.

\vfill

\noindent Pater noster \rubricatum{quod dicitur totum secreto.}

\vfill
\pagebreak

\pars{Confessio.}

\noindent Confíteor Deo omnipoténti, beátæ Maríæ semper Vírgini, beáto
Michaéli Archángelo, beáto Ioánni Baptístæ, sanctis Apóstolis Petro
et Paulo, ómnibus Sanctis, et vobis fratres: quia peccávi nimis cogitatióne,
verbo et ópere: mea culpa, mea culpa, mea máxima culpa.
Ídeo precor beátam Maríam semper Vírginem, beátum Michaélum
Archángelum, beátum Ioánnem Baptístam, sanctos Apóstolos Petrum
et Paulum, omnes Sanctos, et vos fratres, oráre pro me ad Dóminum
Deum nostrum.

\vfill

\noindent \Vbardot{} Misereátur nostri omnípotens Deus, et, dimíssis peccátis nostris, perdúcat
nos ad vitam ætérnam. \Rbardot{} Amen.

\vfill

\noindent \Vbardot{} Indulgéntiam, absolutiónem et remissiónem peccatórum nostrórum tríbuat nobis
omnípotens et miséricors Dóminus. \Rbardot{} Amen.

\vfill

\rubrica{Et facta absolutione dicitur:}

\sineinitiali{temporalia/convertenosdeus.gtex}

\vfill

\cuminitiali{}{temporalia/deusinadiutorium-communis.gtex}

\vfill
\pagebreak

\pars{Psalmus 1.}

\antiphona{VIII G}{temporalia/ant-alleluia.gtex}

\scriptura{Ps. 4}

\initiumpsalmi{temporalia/ps4-initium-viii-G-auto.gtex}

%\psalmusEtTranslatioT{temporalia/ps4-comb.tex}{10cm}
\input{temporalia/ps4.tex} \Abardot{}

\vfill
\pagebreak

\pars{Psalmus 2.} \scriptura{Ps. 90}

%\psalmusEtTranslatioT{temporalia/ps90-comb.tex}{10cm}
\input{temporalia/ps90.tex} \Abardot{}

\pagebreak

\pars{Psalmus 3.} \scriptura{Ps. 133}

%\psalmusEtTranslatioT{temporalia/ps133-comb.tex}{10cm}
\input{temporalia/ps133.tex} \Abardot{}

\vfill

\antiphona{VIII G}{temporalia/ant-alleluia.gtex}

\vfill

\pars{Hymnus.}

\antiphona{I}{temporalia/hym-TeLucis.gtex}
%\input{hym-TeLucis-bohtext.tex}

\pagebreak

\pars{Capitulum.} \scriptura{Ier. 14, 9}

\cuminitiali{}{temporalia/capitulum-tuautem.gtex}

\vfill

\pars{Responsorium breve.} \scriptura{Ps. 30, 6}

\cuminitiali{VI}{temporalia/resp-inmanus.gtex}

\vfill

\pars{Versus.} \scriptura{Ps. 16, 8}

\sineinitiali{temporalia/versus-custodi.gtex}

\vfill
\pagebreak

\cantusCumNeumis

\pars{Canticum Simeonis.}

\antiphona{III a}{temporalia/ant-salvanos-antiquo.gtex}

\scriptura{Lc. 2, 29-32}

\initiumpsalmi{temporalia/nuncdimittis-initium-iii-a-auto.gtex}

%\psalmusEtTranslatioT{temporalia/nuncdimittis-comb.tex}{10cm}
\input{temporalia/nuncdimittis.tex} \Abardot{}

\vfill
\pagebreak

\pars{Oratio.}

\cantusSineNeumas

\cuminitiali{}{temporalia/oratio-visita.gtex}

\vfill

\sineinitiali{temporalia/domineexaudi.gtex}

\vfill

\sineinitiali{temporalia/benedicamus-minor.gtex}

\vfill

\pars{Benedictio.}

\noindent Benedícat et custódiat nos omnípotens et miséricors Dóminus, \gredagger{}
Pater, et Fílius, et Spíritus Sanctus. \Rbardot{} Amen.

\vfill
\pagebreak

\pars{Antiphona finalis B. M. V.}

\antiphona{V}{temporalia/an_regina_caeli_simplex.gtex}

\sineinitiali{temporalia/versus-gaude.gtex}

\vfill
\pagebreak

\hora{Ad Matutinum.} %%%%%%%%%%%%%%%%%%%%%%%%%%%%%%%%%%%%%%%%%%%%%%%%%%%%%%%%%%
%\sideThumbs{Matutinum}

\vspace{2mm}

\cuminitiali{}{temporalia/dominelabiamea.gtex}

\vspace{2mm}

\pars{Invitatorium.} \scriptura{Sap. 1, 7; Psalmus 94, 6; Psalmus 94; \textbf{H268} \& \textbf{H447}}

\vspace{-6mm}

\antiphona{V}{temporalia/inv-alleluiaspiritus.gtex}

\vfill
\pagebreak

\pars{Hymnus.}

\vspace{-5mm}

\antiphona{VIII}{temporalia/hym-JamChristus.gtex}

\vfill
\pagebreak

\pars{Psalmus 1.} \scriptura{Act. 2, 2; \textbf{H268}}

\antiphona{VIII c}{temporalia/matant1.gtex}

\scriptura{Ps. 47}

\initiumpsalmi{temporalia/ps47-initium-viii-C-auto.gtex}

%\psalmusEtTranslatioT{temporalia/ps47-comb.tex}{10cm}
\input{temporalia/ps47.tex} \Abardot{}

\antiphona{}{temporalia/matant1.gtex} % repeat the antiphon - new page

\vfill
\pagebreak

\pars{Psalmus 2.} \scriptura{Ps. 67, 29.30; \textbf{H268}}

\antiphona{VIII c}{temporalia/matant2.gtex}

\scriptura{Ps. 67}

\initiumpsalmi{temporalia/ps67-initium-viii-C-auto.gtex}

%\psalmusEtTranslatioT{temporalia/ps67-comb.tex}{10cm}
\input{temporalia/ps67.tex} \Abardot{}

\antiphona{}{temporalia/matant2.gtex} % repeat the antiphon - new page

\vfill
\pagebreak

\pars{Psalmus 3.} \scriptura{Ps. 103, 30; \textbf{H268}}

\vspace{-3mm}

\antiphona{VIII c}{temporalia/matant3.gtex}

\scriptura{Ps. 103}

\initiumpsalmi{temporalia/ps103-initium-viii-C-auto.gtex}

%\psalmusEtTranslatioT{temporalia/ps103-comb.tex}{9.5cm}
\input{temporalia/ps103.tex} \Abardot{}

\vfill
\pagebreak

\pars{Versus.} \scriptura{Sap. 1, 7}

\sineinitiali{temporalia/versus-spiritus.gtex}

\vfill

\sineinitiali{temporalia/oratiodominica-mat.gtex}

\vfill

\pars{Absolutio.}

\cuminitiali{}{temporalia/absolutio-exaudi.gtex}

\vfill
\pagebreak

\cuminitiali{}{temporalia/benedictio-solemn-evangelica.gtex}

\vfill

\cuminitiali{}{temporalia/tonus-lectionis-solemnis.gtex}

\vfill

% Léctio sancti Evangélii secúndum Ioánnem.
\pars{Lectio I.} \scriptura{Io. 14, 23-31}

\noindent Léctio sancti Evangélii secúndum Ioánnem.

\noindent In illo témpore: Dixit Iesus discípulis suis: Si quis díligit me, sermónem meum servábit, et Pater meus díliget eum, et ad eum veniémus, et mansiónem apud eum faciémus. Et réliqua.

% Homilía sancti Gregórii Papæ.
\scriptura{Homilia 30. in Evang.}

\noindent Homilía sancti Gregórii Papæ.

\noindent Libet, fratres caríssimi, evangelicæ verba lectiónis sub brevitate transcúrrere, ut post diutius liceat in contemplatióne tantæ solemnitátis immorari. Hódie namque Spíritus Sanctus repentino sónitu super discípulos venit, mentesque carnalium in sui amórem permutávit, et foris apparéntibus linguis igneis, intus facta sunt corda flammántia; quia dum Deum in ignis visióne suscepérunt, per amórem suáviter arsérunt. Ipse namque Spíritus Sanctus amor est: unde et Ioánnes dicit Deus cáritas est. Qui ergo mente integra Deum desíderat, profecto iam habet, quem amat. Neque enim quisquam posset Deum dilígere, si eum quem díligit, non habéret.

\tuAutem

\vfill
\pagebreak

\pars{Responsorium 1.} \scriptura{\Rbar{} Act. 2, 1.2 \Vbar{} ibidem; \textbf{H269}}

\responsorium{III}{temporalia/matresp1.gtex}{}

\vfill
\pagebreak

\cuminitiali{}{temporalia/benedictio-solemn-divinum.gtex}

\vfill

\pars{Lectio II.} \scriptura{Homilia 30. in Evang.}

\noindent Sed ecce, si unusquísque vestrum requirátur an diligat Deum: tota fiducia et secura mente respondet, Diligo. In ipso autem lectiónis exordio audístis quid Véritas dicit: Si quis díligit me, sermónem meum servábit. Probátio ergo dilectiónis, exhibítio est operis. Hinc in epistola sua idem Ioánnes dicit: Qui dicit: Diligo Deum, et mandáta eius non custódit, mendax est. Vere étenim Deum dilígimus et mandáta eius custodímus, si nos a nostris voluptátibus coarctamus. Nam qui adhuc per illicita desidéria diffluit, profecto Deum non amat: quia ei in sua voluntáte contradicit.

\tuAutem

\vfill
\pagebreak

\pars{Responsorium 2.} \scriptura{\Rbar{} Act. 2, 4.6 \Vbar{} ibid. 2, 4.11; \textbf{H269}}

\responsorium{II}{temporalia/matresp2.gtex}{}

\vfill
\pagebreak

\cuminitiali{}{temporalia/benedictio-solemn-adsocietatem.gtex}

\vfill

\pars{Lectio III.} \scriptura{Homilia 30. in Evang.}

\noindent Et Pater meus diliget eum, et ad eum veniemus, et mansiónem apud eum faciémus. Pensate, fratres caríssimi, quanta sit ista dígnitas, habére in cordis hospítio advéntum Dei. Certe, si domum nostram quisquam dives aut præpotens amícus intraret, omni festinántia domus tota mundarétur, ne quid fortásse esset, quod óculos amíci intrántis offenderet. Tergat ergo sordes pravi operis, qui Deo præparat domum mentis. Sed vidéte quid Véritas dicat: Veniemus, et mansiónem apud eum faciémus. In quorúmdam étenim corda venit, et mansiónem non facit: quia, per compunctiónem quidem, Dei respectum percipiunt, sed tentatiónis témpore hoc ipsum quo compúncti fúerant, obliviscúntur; sicque ad perpetranda peccáta redeunt, ac si hæc minime planxíssent.

\tuAutem

\vfill
\pagebreak

% Te Deum

\pars{Hymnus Ambrosianus}

\cuminitiali{III}{temporalia/tedeum-solemnis.gtex}

\vfill
\pagebreak

\sineinitiali{temporalia/domineexaudi.gtex}

\vfill

\pars{Oratio.}

\cuminitiali{}{temporalia/oratio.gtex}

\vfill

\noindent \Vbardot{} Dómine, exáudi oratiónem meam.
\Rbardot{} Et clamor meus ad te véniat.

\vfill

% Nocturnale Romanum 2002, p. LXXVI Benedicamus Domino seems to match
% the one from Solemn Laudes.
\cuminitiali{V}{temporalia/benedicamus-solemnis-laud.gtex}

\vfill

\noindent \Vbardot{} Fidélium ánimæ per misericórdiam Dei requiéscant in pace.
\Rbardot{} Amen.

\vfill
\pagebreak

\hora{Ad Laudes.} %%%%%%%%%%%%%%%%%%%%%%%%%%%%%%%%%%%%%%%%%%%%%%%%%%%%%%%%%%
%\sideThumbs{Laudes}

% Psalmi festivi (AM33, pg. 721):
% 92, 99, 62, Dan3, 148+149+150

\vspace{1cm}
\cuminitiali{}{temporalia/deusinadiutorium-communis.gtex}
\vspace{1cm}

\vfill
\pagebreak

\cantusSineNeumas

\pars{Psalmus 1.} \scriptura{Act. 2, 1; \textbf{H270}}

\antiphona{III a\textsuperscript{2}}{temporalia/ant1.gtex}

\scriptura{Ps. 92}

\initiumpsalmi{temporalia/ps92-initium-iii-a2-auto.gtex}

%\psalmusEtTranslatioT{temporalia/ps92-comb.tex}{10cm}
\input{temporalia/ps92.tex} \Abardot{}

\vfill
\pagebreak

\pars{Psalmus 2.} \scriptura{Sap. 1, 7; \textbf{H270}}

\antiphona{VIII G}{temporalia/ant2.gtex}

\scriptura{Ps. 99}

\initiumpsalmi{temporalia/ps99-initium-viii-G-auto.gtex}

%\psalmusEtTranslatioT{temporalia/ps99-comb.tex}{10cm}
\input{temporalia/ps99.tex} \Abardot{}

\vfill
\pagebreak

\pars{Psalmus 3.} \scriptura{Act. 2, 4; \textbf{H270}}

\vspace{-4mm}

\antiphona{VIII G}{temporalia/ant3.gtex}

\scriptura{Ps. 62}

\initiumpsalmi{temporalia/ps62-initium-viii-G-auto.gtex}

%\psalmusEtTranslatioT{temporalia/ps62-comb.tex}{10cm}
\input{temporalia/ps62.tex} \Abardot{}

\vfill
\pagebreak

\pars{Psalmus 4.} \scriptura{Dan. 3, 77.79; \textbf{H270}}

\antiphona{I a\textsuperscript{2}}{temporalia/ant4.gtex}

\scriptura{Canticum trium puerorum, Dan. 3, 57-88 et 56}

\initiumpsalmi{temporalia/dan3-initium-i-a2-auto.gtex}

%\psalmusEtTranslatioT{temporalia/dan3-comb.tex}{10cm}
\input{temporalia/dan3.tex} \Abardot{}

\rubrica{Hic non dicitur Gloria Patri, neque Amen.}
\vspace{1cm}

\hicSuntNeumae
\antiphona{}{temporalia/ant4.gtex} % repeat the antiphon - new page

\vfill
\pagebreak

\pars{Psalmus 5.} \scriptura{\textbf{H271}}

\antiphona{VII c\textsuperscript{2}}{temporalia/ant5.gtex}

%
\scriptura{Ps. 148}

\initiumpsalmi{temporalia/ps148-initium-vii-c2-auto.gtex}

%\psalmusEtTranslatioT{temporalia/ps148-comb.tex}{10cm}
\input{temporalia/ps148.tex}

\rubrica{Hic non dicitur Gloria Patri.}

%
\scriptura{Ps. 149}

\initiumpsalmi{temporalia/ps149-initium-vii-c2-auto.gtex}

%\psalmusEtTranslatioT{temporalia/ps149-comb.tex}{10cm}
\input{temporalia/ps149.tex}

\rubrica{Hic non dicitur Gloria Patri.}

\vfill
\pagebreak

%
\scriptura{Ps. 150}

\initiumpsalmi{temporalia/ps150-initium-vii-c2-auto.gtex}

%\psalmusEtTranslatioT{temporalia/ps150-comb.tex}{10cm}
\input{temporalia/ps150.tex}

\antiphona{}{temporalia/ant5.gtex} % repeat the antiphon - new page

\vfill
\pagebreak

\cantusSineNeumas

\pars{Capitulum.} \scriptura{Act. 2, 1 - 2}

\cuminitiali{}{temporalia/capitulum-CumComplerentur.gtex}

\vspace{1cm}
\pars{Responsorium breve.} \scriptura{Sap. 1, 7}

\cuminitiali{VI}{temporalia/respl.gtex}

\vfill
\pagebreak

\pars{Hymnus.}

\cuminitiali{I}{temporalia/hym-BeataNobis.gtex}

\vfill
\pagebreak

\pars{Versus.}

% Versus. %%%
\sineinitiali{temporalia/versus-repleti.gtex}

\vfill
\vspace{1cm}

\cantusCumNeumis

\pars{Canticum Zachariæ.} \scriptura{Io. 20, 22-23; \textbf{H271}}

\antiphona{VII a}{temporalia/ant-ben-laud.gtex}

\scriptura{Lc. 1, 68-79}

\initiumpsalmi{temporalia/benedictus-initium-viisoll-a-auto.gtex}

%\psalmusEtTranslatioT{temporalia/benedictus-comb.tex}{10cm}
\input{temporalia/benedictus.tex} \Abardot{}

\antiphona{}{temporalia/ant-ben-laud.gtex} % repeat the antiphon - new page

\vfill
\pagebreak

\cantusSineNeumas

\anteOrationem

\pagebreak

% Oratio. %%%
\pars{Oratio.}

\cuminitiali{}{temporalia/oratio.gtex}

\vspace{1cm}
\rubrica{Hebdomadarius dicit iterum Dominus vobiscum. Postea cantatur a cantore:}
\vspace{2mm}

\cuminitiali{V}{temporalia/benedicamus-solemnis-laud.gtex}

\vfill

\end{document}
