% LuaLaTeX

\documentclass[a4paper, twoside, 12pt]{article}
\usepackage[latin]{babel}
%\usepackage[landscape, left=3cm, right=1.5cm, top=2cm, bottom=1cm]{geometry} % okraje stranky
\usepackage[landscape, a4paper, mag=1166, truedimen, left=2cm, right=1.5cm, top=1.6cm, bottom=0.95cm]{geometry} % okraje stranky

\usepackage{fontspec}
\setmainfont[FeatureFile={junicode.fea}, Ligatures={Common, TeX}, RawFeature=+fixi]{Junicode}
%\setmainfont{Junicode}

% shortcut for Junicode without ligatures (for the Czech texts)
\newfontfamily\nlfont[FeatureFile={junicode.fea}, Ligatures={Common, TeX}, RawFeature=+fixi]{Junicode}

% Hebrew font: http://scripts.sil.org/cms/scripts/page.php?site_id=nrsi&id=SILHebrUnic2
\newfontfamily\hebfont[Scale=1]{Ezra SIL}

\usepackage{multicol}
\usepackage{color}
\usepackage{lettrine}
\usepackage{fancyhdr}

% usual packages loading:
\usepackage{luatextra}
\usepackage{graphicx} % support the \includegraphics command and options
\usepackage{gregoriotex} % for gregorio score inclusion
\usepackage{gregoriosyms}
\usepackage{wrapfig} % figures wrapped by the text
\usepackage{parcolumns}
\usepackage[contents={},opacity=1,scale=1,color=black]{background}
\usepackage{tikzpagenodes}
\usepackage{calc}
\usepackage{longtable}
\usetikzlibrary{calc}

\setlength{\headheight}{14.5pt}

\input{conventuscommune.tex} % Often used macros
%%%% Preklady jednotlivych zpevu (nektere se opakuji, a je dobre mit je
% vsechny na jedne hromade)

% HOURS ---

\newcommand{\trAntI}{\translatioCantus{Muž boží měl kožený toulec, pečlivě
zavázaný, jenž mu visel na šíji a~často se ho dotýkal.}}

\newcommand{\trAntII}{\translatioCantus{Klíč od~něho tak dobře střežil, že
dokud žil v~těle, nikdo z~jeho žáků nezvěděl, co je uvnitř.}}

\newcommand{\trAntIII}{\translatioCantus{Ale když se odebral z~tohoto
života, schránku otevřeli a~objevili v~ní žíněné roucho a~měděný řetěz
potřísněný krví.}}

\newcommand{\trAntIV}{\translatioCantus{A když prohlédli mistrovo tělo,
nalezli jeho tělo na čtyřech místech hluboce zbrázděno ranami od řetězu.}}

\newcommand{\trAntV}{\translatioCantus{Krev vytékající z~těch ran, místy
prostoupila i~žíněným rouchem.}}

\newcommand{\trCapituli}{\translatioCantus{
Miláčkovi Boha a~lidí,
Mojžíšovi požehnané paměti,~\gredagger{}
dopřál slávu rovnou slávě svatých~\grestar{}
učinil ho mocným na postrach nepřátelům
a~jeho slovy zastavil divy.}}

\newcommand{\trLectioBrevis}{\translatioCantus{
Pamatujte na své představené,
kteří vám hlásali Boží slovo.
Uvažte, jak oni skončili život, a~napodobujte jejich víru.
Ježíš Kristus je stejný včera i~dnes i~navěky.
Nenechte se svést věelijakými cizími naukami.}}

\newcommand{\trRespLaud}{\translatioCantus{Spravedlivého vodil Hospodin~\grestar{}
po přímých stezkách. \Vbardot{} A~ukázal mu Boží království.}}

\newcommand{\trRespLaudB}{\translatioCantus{Na tvých hradbách, Jeruzaléme,
ustanovil jsem strážné;~\grestar{}
budou bdít nad mým lidem. \Vbardot{} Ani ve dne, ani v~noci nesmějí nikdy
mlčet.}}

\newcommand{\trVersus}{\translatioCantus{\Vbardot{} Ústa spravedlivého šeptají moudrost, aleluja.
\Rbardot{} A~jeho jazyk ohlašuje právo, aleluja.}}

\newcommand{\trAntBenedictus}{\translatioCantus{Když na bujné oře vložili
nosítka a~sňali jim uzdu, vydali se přímo k~cele božího muže.}}

\newcommand{\trPreces}{\translatioCantus{
\noindent S vděčností chvalme Krista, dobrého Pastýře, \gredagger{} který dal život za své ovce, \grestar{} a~pokorně ho prosme: \Rbardot{} Pane, buď pastýřem svého lidu.

\noindent Kriste, ty dáváš církvi pastýře, a~jejich službou se ujímáš svého lidu, \grestar{} dej, ať v~lásce těch, kteří nás vedou, poznáváme, jak nás miluješ. \Rbardot{} Pane, buď pastýřem svého lidu.

\noindent Ty stále konáš skrze své zástupce službu pastýře a~učitele, \grestar{} nepřestávej nás nikdy vést prostřednictvím svých služebníků. \Rbardot{} Pane, buď pastýřem svého lidu.

\noindent Ty prokazuješ svému lidu skrze jeho pastýře službu lékaře duše i~těla, \grestar{} ochraňuj náš život a~veď nás ke svatosti. \Rbardot{} Pane, buď pastýřem svého lidu.

\noindent Ty posíláš své svaté, aby slovem i~příkladem vedli tvůj lid k~tobě, \grestar{} na jejich přímluvu nás posiluj, abychom vytrvali na cestě, která vede k~věčnému životu. \Rbardot{} Pane, buď pastýřem svého lidu.}}

\newcommand{\trOrationis}{\translatioCantus{Bože, jenž nám dopřáváš radovat
se z~výroční slavnosti svatého tvého vyznavače Havla, uděl dobrotivě,
abychom když slavíme jeho narození, též se řídili podobou jeho skutků.
Skrze…}}
 % Czech translations of the proper texts

\newcommand{\annusEditionis}{2016}

\def\hebinitial#1{%
\leavevmode{\newbox\hebbox\setbox\hebbox\hbox{\hebfont{#1}\hskip 1mm}\kern -\wd\hebbox\hbox{\hebfont{#1}\hskip 1mm}}%
}

%%%% Vicekrat opakovane kousky

\newcommand{\anteOrationem}{
  \rubrica{Ante Orationem, cantatur a Superiore:}

  \pars{Supplicatio Litaniæ.}

  \gregorioscore{temporalia/supplicatiolitaniae.gtex}

  \pars{Oratio Dominica.}

  \gregorioscore{temporalia/oratiodominica.gtex}

  \rubrica{Deinde dicitur ab Hebdomadario:}

  \gregorioscore{temporalia/dominusvobiscum-solemnis.gtex}

  \rubrica{In choro monialium loco Dominus vobiscum dicitur:}

  \gregorioscore{temporalia/domineexaudi.gtex}
}

\newcommand{\tuAutem}{
  \vfill

  \gregorioscore{temporalia/tuautem.gtex}
}

\setlength{\columnsep}{30pt} % prostor mezi sloupci

%%%%%%%%%%%%%%%%%%%%%%%%%%%%%%%%%%%%%%%%%%%%%%%%%%%%%%%%%%%%%%%%%%%%%%%%%%%%%%%%%%%%%%%%%%%%%%%%%%%%%%%%%%%%%
\begin{document}

% Here we set the space around the initial.
% Please report to http://home.gna.org/gregorio/gregoriotex/details for more details and options
\grechangedim{afterinitialshift}{2.2mm}{scalable}
\grechangedim{beforeinitialshift}{2.2mm}{scalable}
\grechangedim{interwordspacetext}{0.32 cm plus 0.15 cm minus 0.05 cm}{scalable}%
\grechangedim{annotationraise}{-0.2cm}{scalable}

% Here we set the initial font. Change 38 if you want a bigger initial.
% Emit the initials in red.
\grechangestyle{initial}{\color{red}\fontsize{38}{38}\selectfont}

\pagestyle{empty}

\begin{titulusOfficii}
\nomenFesti{Sabbato Sancto.}
\celebratio{Duplex 1. classis.}
\end{titulusOfficii}

%\hora{Ad Matutinum.}
%%%%%%%%%%%%%%%%%%%%%%%%%%%%%%%%%%%%%%%%%%%%%%%%%%%%%%%%%%
\sideThumbs{Matutinum}

\vfill

% Incipit Lamentátio Ieremíæ Prophétæ.
\pars{Lectio I.} \scriptura{Lam. 3, 22-30}

\noindent De Lamentátio Ieremíæ Prophétæ.

\textusEtTranslatio{
\hebinitial{ח} Misericórdiæ Dómini quia non sumus consúmpti: quia non defecérunt miseratiónes eius. Novi dilúculo, multa est fides tua. Pars mea Dóminus, dixit ánima mea: proptérea exspectábo eum.
\hebinitial{ט} Bonus est Dóminus sperántibus in eum, ánimæ quærénti illum. Bonum est præstolári cum siléntio salutáre Dei. Bonum est viro, cum portáverit iugum ab adolescéntia sua.
\hebinitial{י} Sedébit solitárius, et tacébit: quia levávit super se. Ponet in púlvere os suum, si forte sit spes. Dabit percutiénti se maxíllam, saturábitur oppróbriis.
Ierúsalem, Ierúsalem, convértere ad Dóminum Deum tuum.
}{\trMatLecI}{10cm}

\vfill

\pars{Lectio II.} \scriptura{Lam. 4, 1-6}

\textusEtTranslatio{
\hebinitial{א} Quómodo obscurátum est aurum, mutátus est color óptimus, dispérsi sunt lápides sanctuárii in cápite ómnium plateárum?\\
\hebinitial{ב} Fílii Sion ínclyti, et amícti áuro prímo: quómodo reputáta sunt in vasa téstea, opus mánuum fíguli?\\
\hebinitial{ג} Sed et lámiæ nudavérunt mammam, lactavérunt cátulos suos: fília pópuli mei crudélis quasi strúthio in desérto.\\
\hebinitial{ד} Adhǽsit lingua lacténtis ad palátum eius in siti: párvuli petiérunt panem: et non erat qui frángeret eis.\\
\hebinitial{ה} Qui vescebántur voluptuóse, interiérunt in viis: qui nutriebántur in cróceis, amplexáti sunt stércora.\\
\hebinitial{ו} Et maior effécta est iníquitas fíliæ pópuli mei peccáto Sodomórum, quæ subvérsa est in moménto, et non cepérunt in ea manus.
Ierúsalem, Ierúsalem, convértere ad Dóminum Deum tuum.
}{\trMatLecII}{10cm}

\vfill

\pars{Lectio III.} \scriptura{Lam. 5, 1-11}

\noindent Incípit Orátio Ieremíæ Prophétæ.

\textusEtTranslatio{
Recordáre, Dómine, quid accíderit nobis: intuére et réspice oppróbrium nostrum.
Hæréditas nostra versa est ad aliénos: domus nostræ ad extráneos.
Pupílli facti sumus absque patre, matres nostræ quasi víduæ.
Aquam nostram pecúnia bíbimus: ligna nostra prétio comparávimus.
Cervícibus nostris minabámur, lassis non dabátur réquies.
Ægýpto dédimus manum, et Assýriis, ut saturarémur pane.
Patres nostri peccavérunt, et non sunt: et nos iniquitátes eórum portávimus.
Servi domináti sunt nostri: non fuit qui redímeret de manu eórum.
In animábus nostris afferebámus panem nobis, a fácie gládii in desérto.
Pellis nostra quasi clíbanus exústa est a facie tempestátum famis.
Mulíeres in Sion humiliavérunt, et vírgines in civitátibus Iuda.
Ierúsalem, Ierúsalem, convértere ad Dóminum Deum tuum.
}{\trMatLecIII}{10cm}

\vfill

\sineinitiali{temporalia/tonus-lectionis-solemnis.gtex}

\vspace{-0.4cm}

\vfill

\pars{Lectio IV.} \scriptura{In Psalmum 63. ad 7. versum}

\noindent Ex Tractátu sancti Augustíni Epíscopi su\textit{per} Psalmos.

\textusEtTranslatio{
Accédet homo ad cor altum, et exaltábitur Deus.
Illi dixérunt: Quis nos vidébit?
Defecérunt scrutántes scrutatiónes, consília mala.
Accéssit homo ad ipsa consília, passus est se tenéri ut homo.
Non enim tenerétur nisi homo, aut viderétur nisi homo,
aut cæderétur nisi homo, aut crucifigerétur, aut morerétur nisi homo.
Accéssit ergo homo ad illas omnes passiónes, quæ in illo nihil valérent, nisi esset homo.
Sed si ille non esset homo, non liberarétur homo.
Accéssit homo ad cor altum, id est, cor secrétum, obiíciens aspéctibus humánis hóminem, servans intus Deum:
celans formam Dei, in qua æquális est Patri, et ófferens formam servi, qua minor est Patre.
}{\trMatLecIV}{10cm}

\pars{Lectio V.}

\textusEtTranslatio{
Quo perduxérunt illas scrutatiónes suas, quas perscrutántes defecérunt,
ut étiam mórtuo Dómino et sepúlto, custódes pónerent ad sepúlcrum?
Dixérunt enim Piláto: Sedúctor ille: hoc appellabátur nómine Dóminus Iesus Christus,
ad solátium servórum suórum, quando dicúntur seductóres:
ergo illi Piláto: Sedúctor ille, ínquiunt, dixit adhuc vivens: Post tres dies resúrgam.
Iube ítaque custodíri sepúlcrum usque in diem tértium,
ne forte véniant discípuli eius, et furéntur eum, et dicant plebi:
Surréxit a mórtuis: et erit novíssimus error peior prióre.
Ait illis Pilátus: Habétis custódiam, ite, custodíte sicut scitis.
Illi autem abeúntes, muniérunt sepúlcrum, signántes lápidem cum custodibus.
}{\trMatLecV}{10cm}

\pars{Lectio VI.}

\textusEtTranslatio{
Posuérunt custódes mílites ad sepúlcrum.
Concússa terra Dóminus resurréxit: mirácula facta sunt tália circa sepúlcrum,
ut et ipsi mílites, qui custódes advénerant, testes fíerent, si vellent vera nuntiáre.
Sed avarítia illa, quæ captivávit discípulum cómitem Christi,
captivávit et mílitem custódem sepúlcri.
Damus, ínquiunt, vobis pecúniam: et dícite,
quia vobis dormiéntibus venérunt discípuli eius, et abstulérunt eum.
Vere defecérunt scrutántes scrutatiónes.
Quid est quod dixísti, o infélix a stútia?
Tantúmne déseris lucem consílii pietátis, et in profúnda versútiæ demérgeris, ut hoc dicas:
Dícite, quia vobis dormiéntibus venérunt discípuli eius, et abstulérunt eum?
Dormiéntes testes ádhibes: vere tu ipse obdormísti, qui scrutándo tália defecísti.
}{\trMatLecVI}{10cm}

\pars{Lectio VII.} \scriptura{Heb. 9, 11-14}

\noindent De Epístola beáti Pauli Apóstoli ad \textit{Heb}rǽos.

\textusEtTranslatio{
Christus assístens póntifex futurórum bonórum, per ámplius et perféctius tabernáculum, non manufáctum, id est non huius creatiónis:
Neque per sánguinem hircórum aut vitulórum, sed per próprium sánguinem introívit semel in Sancta, ætérna redemptióne invénta.
Si enim sanguis hircórum et taurórum, et cinis vítulæ aspérsus inquinátos sanctíficat ad emundatiónem carnis:
Quanto magis sanguis Christi, qui per Spíritum Sanctum semetípsum óbtulit immaculátum Deo,
emundábit consciéntiam nostram ab opéribus mórtuis ad serviéndum Deo vivénti?
}{\trMatLecVII}{10cm}

\pars{Lectio VIII.} \scriptura{Heb. 9, 15-18}

\textusEtTranslatio{
Et ídeo novi testaménti mediátor est: ut, morte intercedénte, in redemptiónem eárum prævaricatiónum,
quæ erant sub prióri testaménto, repromissiónem accípiant qui vocáti sunt ætérnæ hereditátis.
Ubi enim testaméntum est, mors necésse est intercédat testatóris.
Testaméntum enim in mórtuis confirmátum est:
alióquin nondum valet dum vivit qui testátus est.
Unde nec primum quidem sine sánguine dedicátum est.
}{\trMatLecVIII}{10cm}

\pars{Lectio IX.} \scriptura{Heb. 9, 19-22}

\textusEtTranslatio{
Lecto enim omni mandáto legis a Móyse univérso pópulo, accípiens sánguinem vitulórum et hircórum cum aqua et lana coccínea et hyssópo,
ipsum quoque librum et omnem pópulum aspérsit,
dicens: ,,Hic sanguis testaménti quod mandávit ad vos Deus``.
Etiam tabernáculum et ómnia vasa ministérii sánguine simíliter aspérsit;
et ómnia pene in sánguine secúndum legem mundántur, et sine sánguinis effusióne non fit remíssio.
}{\trMatLecIX}{10cm}

\end{document}
