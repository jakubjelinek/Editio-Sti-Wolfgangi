\newcommand{\oratio}{\pars{Oratio.}

\noindent Deus, qui ineffabílibus mundum rénovas sacraméntis, præsta, quǽsumus, ut Ecclésia tua et ætérnis profíciat institútis et temporálibus non destituátur auxíliis.

\pars{Pro pace in universo mundo.} \scriptura{Sir. 50, 25; 2 Esdr. 4, 20; \textbf{H416}}

\vspace{-4mm}

\antiphona{II D}{temporalia/ant-dapacemdomine.gtex}

\vfill

\noindent Deus, a quo sancta desidéria, recta consília et iusta sunt ópera: da servis tuis illam, quam mundus dare non potest, pacem; ut et corda nostra mandátis tuis dédita, et hóstium subláta formídine, témpora sint tua protectióne tranquílla.

\noindent Per Dóminum nostrum Iesum Christum, Fílium tuum, qui tecum vivit et regnat in unitáte Spíritus Sancti, Deus, per ómnia sǽcula sæculórum.

\noindent \Rbardot{} Amen.}
\newcommand{\invitatorium}{\pars{Invitatorium.}

\vspace{-4mm}

\antiphona{IV*}{temporalia/inv-christumdominum-cumdox.gtex}}
\newcommand{\hymnusmatutinum}{\pars{Hymnus}

\cuminitiali{I}{temporalia/hym-NuncTempus.gtex}}
\newcommand{\matversus}{\noindent \Vbardot{} Pænitémini et crédite Evangélio.

\noindent \Rbardot{} Appropinquávit enim regnum Dei.}
\newcommand{\lectioi}{\vspace{-4mm}

\pars{Lectio I.} \scriptura{Ex. 5, 1-22}

\noindent De libro Exodi.

\noindent In diébus illis, ingréssi sunt Móyses et Aaron et dixérunt pharaóni: «Hæc dicit Dóminus, Deus Israel: Dimítte pópulum meum, ut sacríficet mihi in desérto». At ille respóndit: «Quis est Dóminus, ut áudiam vocem eius et dimíttam Israel? Néscio Dóminum et Israel non dimíttam». Dixerúntque: «Deus Hebræórum occúrrit nobis; eámus, quæso, viam trium diérum in solitúdinem et sacrificémus Dómino Deo nostro, ne forte áccidat nobis pestis aut gládius». Ait ad eos rex Ægýpti: «Quare, Móyses et Aaron, sollicitátis pópulum ab opéribus suis? Ite ad ónera vestra». Dixítque phárao: «Multus nimis iam est pópulus terræ; vidétis quod turba succréverit; quanto magis si dedéritis eis réquiem ab opéribus?».

\noindent Præcépit ergo in die illo exactóribus pópuli et præféctis eius dicens: «Nequáquam ultra dábitis páleas pópulo ad conficiéndos láteres sicut prius, sed ipsi vadant et cólligant stípulas. Et mensúram láterum, quam prius faciébant, imponétis super eos; nec minuétis quidquam.

\noindent {\color{gray} Vacant enim et idcírco vociferántur dicéntes: “Eámus et sacrificémus Deo nostro”. Opprimántur opéribus et éxpleant ea, ut non acquiéscant verbis mendácibus».

\noindent Igitur egréssi exactóres pópuli et præfécti eius dixérunt ad pópulum: «Sic dicit phárao: “Non do vobis páleas. Ite et collígite, sícubi inveníre potéritis, nec minuétur quidquam de ópere vestro”». Dispersúsque est pópulus per omnem terram Ægýpti ad colligéndas páleas. Exactóres quoque instábant dicéntes: «Compléte opus vestrum quotídie, ut prius fácere solebátis, quando dabántur vobis páleæ». Flagellatíque sunt præfécti filiórum Israel, quos constitúerant super eos exactóres pharaónis dicéntes: «Quare non impléstis mensúram láterum sicut prius, nec heri nec hódie?».}

\noindent Venerúntque præfécti filiórum Israel et vociferáti sunt ad pharaónem dicéntes: «Cur ita agis contra servos tuos? Páleæ non dantur nobis, et láteres simíliter imperántur; en fámuli tui flagéllis cǽdimur, et pópulus tuus est in culpa». Qui ait: «Vacátis ótio et idcírco dícitis: “Eámus et sacrificémus Dómino”. Ite ergo et operámini; páleæ non dabúntur vobis, et reddétis consuétum númerum láterum».

\noindent Videbántque se præfécti filiórum Israel in malo, eo quod dicerétur eis: «Non minuétur quidquam de latéribus per síngulos dies»; occurrerúntque Móysi et Aaron, qui stabant ex advérso egrediéntibus a pharaóne, et dixérunt ad eos: «Vídeat Dóminus et iúdicet, quóniam fœtére fecístis odórem nostrum coram pharaóne et servis eius; et præbuístis ei gládium, ut occíderet nos». Reversúsque est Móyses ad Dóminum et ait: «Dómine, cur afflixísti pópulum istum? Quare misísti me? Ex eo enim quo ingréssus sum ad pharaónem, ut lóquerer in nómine tuo, afflíxit pópulum tuum; et non liberásti eos».}
\newcommand{\responsoriumi}{\pars{Responsorium 1.} \scriptura{\Vbardot{} Ex. 34, 29; \textbf{H160}}

\vspace{-5mm}

\responsorium{VIII}{temporalia/resp-splendidafactaestfacies-CROCHU.gtex}{}}
\newcommand{\lectioii}{\pars{Lectio II.} \scriptura{Ex. 6, 1-13}

\noindent Dixítque Dóminus ad Móysen: «Nunc vidébis quæ factúrus sim pharaóni; per manum enim fortem dimíttet eos et in manu robústa eíciet illos de terra sua».

\noindent Locútus est Dóminus ad Móysen dicens: «Ego Dóminus, qui appárui Abraham, Isaac et Iacob ut Deus omnípotens; et nomen meum Dóminum non indicávi eis. Pepigíque cum eis fœdus, ut darem illis terram Chánaan, terram peregrinatiónis eórum, in qua fuérunt ádvenæ. Ego audívi gémitum filiórum Israel, quia Ægýptii oppressérunt eos, et recordátus sum pacti mei. Ideo dic fíliis Israel: Ego Dóminus, qui edúcam vos de ergástulo Ægyptiórum; et éruam de servitúte ac rédimam in bráchio excélso et iudíciis magnis. Et assúmam vos mihi in pópulum et ero vester Deus; et sciétis quod ego sum Dóminus Deus vester, qui edúxerim vos de ergástulo Ægyptiórum et indúxerim in terram, super quam levávi manum meam, ut darem eam Abraham, Isaac et Iacob; dabóque illam vobis possidéndam, ego Dóminus».

\noindent Narrávit ergo Móyses ómnia fíliis Israel; qui non acquievérunt ei propter angústiam spíritus et opus duríssimum. Locutúsque Dóminus ad Móysen dicens: «Ingrédere et lóquere ad pharaónem regem Ægýpti, ut dimíttat fílios Israel de terra sua». Respóndit Móyses coram Dómino: «Ecce, fílii Israel non áudiunt me, et quómodo áudiet me phárao, præsértim cum incircumcísus sim lábiis?».

\noindent Locutúsque est Dóminus ad Móysen et Aaron et dedit mandátum ad fílios Israel et ad pharaónem regem Ægýpti, ut edúcerent fílios Israel de terra Ægýpti.}
\newcommand{\responsoriumii}{\pars{Responsorium 2.} \scriptura{\Rbardot{} Ps. 77, 1 \Vbardot{} ibid., 2; \textbf{H161}}

\vspace{-5mm}

\responsorium{VIII}{temporalia/resp-attenditepopulemeus-CROCHU.gtex}{}}
\newcommand{\lectioiii}{\pars{Lectio III.} \scriptura{Hom. 9, 5. 10: PG 12, 515. 523}

\noindent Ex Homíliis Orígenis presbýteri in Levíticum.

\noindent Semel in anno póntifex, pópulum derelínquens, ingréditur ad illum locum ubi est repropitiatórium, et super repropitiatórium chérubim, ubi est arca testaménti, et altáre incénsi quo nulli introíre fas est, nisi pontífici soli.

\noindent Si ergo consíderem verum pontíficem meum, Dóminum Iesum Christum, quómodo, in carne quidem pósitus, per totum annum erat cum pópulo, annum illum de quo ipse dicit: \emph{Evangelizáre paupéribus misit me et vocáre annum Dómini accéptum, et diem remissiónis,} advérte quómodo semel in anno isto, in die repropitiatiónis intrat in sancta sanctórum, hoc est, cum impléta dispensatióne pénetrat cælos, et intrat ad Patrem, ut eum propítium humáno géneri fáciat, et exóret pro ómnibus credéntibus in se.

\noindent {\color{gray} Hanc repropitiatiónem eius, qua homínibus repropítiat Patrem, sciens Ioánnes apóstolus, dicit: \emph{Hæc dico, filíoli, ut non peccémus. Quod et si peccavérimus, advocátum habémus apud Patrem, Iesum Christum, iustum et ipse est repropitiátio pro peccátis nostris.}

\noindent Sed et Paulus simíliter de hac repropitiatióne commémorat, cum dicit de Christo: \emph{Quem pósuit Deus propitiatiónem in sánguine ipsíus per fidem.} Igitur dies propitiatiónis manet nobis úsquequo finem mundus accípiat.

\noindent Ait elóquium divínum: \emph{Et impónet incénsum super ignem in conspéctu Dómini et opériet fumus incénsi propitiatórium quod est super testimónia et non moriétur et sumet de sánguine vítuli et respérget dígito suo super propitiatórium contra oriéntem.}}

\noindent Ritus quidem apud véteres propitiatiónis pro homínibus, qui fiébat ad Deum, quáliter celebrarétur edócuit: sed tu qui ad Christum venísti, pontíficem verum, qui sánguine suo Deum tibi propítium fecit et reconciliávit te Patri, non hǽreas in sánguine carnis; sed disce pótius sánguinem Verbi, et audi ipsum tibi dicéntem, quia \emph{hic sanguis meus est, qui pro vobis effundétur in remissiónem peccatórum.}

\noindent Quod autem contra oriéntem respérgit, non otióse accípias. Ab oriénte tibi propitiátio venit. Inde est enim vir, cui Oriens nomen est, qui mediátor Dei et hóminum factus est.

\noindent Invitáris ergo per hoc, ut ad oriéntem semper aspícias, unde tibi óritur sol iustítiæ, unde semper tibi lumen náscitur: ut numquam in ténebris ámbules, neque dies ille novíssimus te in ténebris comprehéndat: ne tibi ignorántiæ nox et calígo subrépat, sed ut semper in sciéntiæ luce verséris, semper hábeas diem fídei, semper lumen caritátis et pacis obtíneas.}
\newcommand{\responsoriumiii}{\pars{Responsorium 3.} \scriptura{\Rbardot{} Dt. 4, 1; 27, 3 \Vbardot{} Ps. 80, 9-10; \textbf{H161}}

\vspace{-5mm}

\responsorium{II}{temporalia/resp-audiisraelpraecepta-CROCHU-cumdox.gtex}{}}
\newcommand{\lectiobrevis}{\pars{Lectio Brevis.} \scriptura{Ex. 19, 4-6}

\noindent Vos ipsi vidístis quómodo portáverim vos super alas aquilárum et addúxerim ad me. Si ergo audiéritis vocem meam et custodiéritis pactum meum, éritis mihi in pecúlium de cunctis pópulis, mea est enim omnis terra. Et vos éritis mihi regnum sacerdótum et gens sancta.}
\newcommand{\responsoriumbreve}{\pars{Responsorium breve.} \scriptura{Ps. 90, 3}

\cuminitiali{IV}{temporalia/resp-ipseliberavitme.gtex}}
\newcommand{\hymnuslaudes}{\pars{Hymnus}

\cuminitiali{D}{temporalia/hym-IamChriste.gtex}}
\newcommand{\preces}{\noindent Benedicámus Deo Patri, qui nobis largítur ut hoc quadragesimáli die sacrifícium laudis ei offerámus.~\grestar{} Eum deprecémur, invocántes:

\Rbardot{} Cæléstibus, Dómine, nos ínstrue disciplínis.

\noindent Omnípotens et miséricors Deus, concéde nobis spíritum oratiónis et pæniténtiæ,~\grestar{} ut caritáte tui et hóminum ardeámus.

\Rbardot{} Cæléstibus, Dómine, nos ínstrue disciplínis.

\noindent Da nos tibi cooperári, ut ómnia instauréntur in Christo,~\grestar{} atque iustítia et pax in terris abúndent.

\Rbardot{} Cæléstibus, Dómine, nos ínstrue disciplínis.

\noindent Intimam totíus creatúræ natúram et prétium áperi nobis,~\grestar{} ut, te celebrántes, eam in cármine laudis nobis consociémus.

\Rbardot{} Cæléstibus, Dómine, nos ínstrue disciplínis.

\noindent Ignósce nobis, qui Christi tui præséntiam in paupéribus, míseris et moléstis ignorávimus,~\grestar{} nec vériti sumus Fílium tuum in his frátribus nostris.

\Rbardot{} Cæléstibus, Dómine, nos ínstrue disciplínis.}
\newcommand{\benedictus}{\pars{Canticum Zachariæ.} \scriptura{Io. 9, 2-3; \textbf{H163}}

\vspace{-4mm}

{
\grechangedim{interwordspacetext}{0.18 cm plus 0.15 cm minus 0.05 cm}{scalable}%
\antiphona{VIII G\textsuperscript{2}}{temporalia/ant-rabbiquidpeccavit.gtex}
\grechangedim{interwordspacetext}{0.22 cm plus 0.15 cm minus 0.05 cm}{scalable}%
}

\vspace{-2mm}

\scriptura{Lc. 1, 68-79}

%\vspace{-2mm}

\cantusSineNeumas
\initiumpsalmi{temporalia/benedictus-initium-viii-G5-auto.gtex}

%\vspace{-1.5mm}

\input{temporalia/benedictus-viii-G5.tex}

\vfill

{
\grechangedim{interwordspacetext}{0.18 cm plus 0.15 cm minus 0.05 cm}{scalable}%
\antiphona{}{temporalia/ant-rabbiquidpeccavit.gtex}
\grechangedim{interwordspacetext}{0.22 cm plus 0.15 cm minus 0.05 cm}{scalable}%
}}
\newcommand{\magnificat}{\pars{Canticum B. Mariæ V.} \scriptura{Io. 2, 19.21; \textbf{H163}}

\vspace{-4mm}

{
\grechangedim{interwordspacetext}{0.18 cm plus 0.15 cm minus 0.05 cm}{scalable}%
\antiphona{V a}{temporalia/ant-solvitetemplumhoc.gtex}
\grechangedim{interwordspacetext}{0.22 cm plus 0.15 cm minus 0.05 cm}{scalable}%
}

\vspace{-2mm}

\scriptura{Lc. 1, 46-55}

\vspace{-2mm}

\cantusSineNeumas
\initiumpsalmi{temporalia/magnificat-initium-v-a.gtex}

\vspace{-1.5mm}

\input{temporalia/magnificat-v-a.tex} \Abardot{}}
\newcommand{\oratiovesperas}{\pars{Oratio.}

\noindent Deprecatiónem nostram, quǽsumus Dómine, benígnus exáudi:~\grestar{} et quibus supplicándi præstas afféctum, tríbue defensiónis auxílium.

\noindent Per Dóminum nostrum Iesum Christum, Fílium tuum, qui tecum vivit et regnat in unitáte Spíritus Sancti, Deus, per ómnia sǽcula sæculórum.

\noindent \Rbardot{} Amen.}
\newcommand{\hebdomada}{infra Hebdom. IV post Pentecosten.}
\newcommand{\oratioLaudes}{\cuminitiali{}{temporalia/oratio4.gtex}}

% LuaLaTeX

\documentclass[a4paper, twoside, 12pt]{article}
\usepackage[latin]{babel}
%\usepackage[landscape, left=3cm, right=1.5cm, top=2cm, bottom=1cm]{geometry} % okraje stranky
%\usepackage[landscape, a4paper, mag=1166, truedimen, left=2cm, right=1.5cm, top=1.6cm, bottom=0.95cm]{geometry} % okraje stranky
\usepackage[landscape, a4paper, mag=1400, truedimen, left=0.5cm, right=0.5cm, top=0.5cm, bottom=0.5cm]{geometry} % okraje stranky

\usepackage{fontspec}
\setmainfont[FeatureFile={junicode.fea}, Ligatures={Common, TeX}, RawFeature=+fixi]{Junicode}
%\setmainfont{Junicode}

% shortcut for Junicode without ligatures (for the Czech texts)
\newfontfamily\nlfont[FeatureFile={junicode.fea}, Ligatures={Common, TeX}, RawFeature=+fixi]{Junicode}

\usepackage{multicol}
\usepackage{color}
\usepackage{lettrine}
\usepackage{fancyhdr}

% usual packages loading:
\usepackage{luatextra}
\usepackage{graphicx} % support the \includegraphics command and options
\usepackage{gregoriotex} % for gregorio score inclusion
\usepackage{gregoriosyms}
\usepackage{wrapfig} % figures wrapped by the text
\usepackage{parcolumns}
\usepackage[contents={},opacity=1,scale=1,color=black]{background}
\usepackage{tikzpagenodes}
\usepackage{calc}
\usepackage{longtable}
\usetikzlibrary{calc}

\setlength{\headheight}{14.5pt}

\input{conventuscommune.tex} % Often used macros

\newcommand{\annusEditionis}{2021}

%%%% Vicekrat opakovane kousky

\newcommand{\anteOrationem}{
  \rubrica{Ante Orationem, cantatur a Superiore:}

  \pars{Supplicatio Litaniæ.}

  \cuminitiali{}{temporalia/supplicatiolitaniae.gtex}

  \pars{Oratio Dominica.}

  \cuminitiali{}{temporalia/oratiodominica.gtex}

  \rubrica{Deinde dicitur ab Hebdomadario:}

  \cuminitiali{}{temporalia/dominusvobiscum-solemnis.gtex}

  \rubrica{In choro monialium loco Dominus vobiscum dicitur:}

  \sineinitiali{temporalia/domineexaudi.gtex}
}

\setlength{\columnsep}{30pt} % prostor mezi sloupci

%%%%%%%%%%%%%%%%%%%%%%%%%%%%%%%%%%%%%%%%%%%%%%%%%%%%%%%%%%%%%%%%%%%%%%%%%%%%%%%%%%%%%%%%%%%%%%%%%%%%%%%%%%%%%
\begin{document}

% Here we set the space around the initial.
% Please report to http://home.gna.org/gregorio/gregoriotex/details for more details and options
\grechangedim{afterinitialshift}{2.2mm}{scalable}
\grechangedim{beforeinitialshift}{2.2mm}{scalable}
\grechangedim{interwordspacetext}{0.22 cm plus 0.15 cm minus 0.05 cm}{scalable}%
\grechangedim{annotationraise}{-0.2cm}{scalable}

% Here we set the initial font. Change 38 if you want a bigger initial.
% Emit the initials in red.
\grechangestyle{initial}{\color{red}\fontsize{38}{38}\selectfont}

\pagestyle{empty}

%%%% Titulni stranka
\begin{titulusOfficii}
\ifx\titulus\undefined
\nomenFesti{Feria II \hebdomada{}}
\else
\titulus
\fi
\end{titulusOfficii}

\vfill

\begin{center}
%Ad usum et secundum consuetudines chori \guillemotright{}Conventus Choralis\guillemotleft.

%Editio Sancti Wolfgangi \annusEditionis
\end{center}

\scriptura{}

\pars{}

\pagebreak

\renewcommand{\headrulewidth}{0pt} % no horiz. rule at the header
\fancyhf{}
\pagestyle{fancy}

\cantusSineNeumas

\ifx\oratio\undefined
\ifx\laudb\undefined
\else
\newcommand{\oratio}{\pars{Oratio.}

\noindent Dómine Deus omnípotens, qui ad princípium huius diéi nos perveníre fecísti, tua nos hódie salva virtúte, ut in hac die ad nullum declinémus peccátum, sed semper ad tuam iustítiam faciéndam nostra procédant elóquia, dirigántur cogitatiónes et ópera.

\noindent Per Dóminum nostrum Iesum Christum, Fílium tuum, qui tecum vivit et regnat in unitáte Spíritus Sancti, Deus, per ómnia sǽcula sæculórum.

\noindent \Rbardot{} Amen.}
\fi
\fi

\hora{Ad Matutinum.} %%%%%%%%%%%%%%%%%%%%%%%%%%%%%%%%%%%%%%%%%%%%%%%%%%%%%
%\sideThumbs{Matutinum}

\vspace{2mm}

\cuminitiali{}{temporalia/dominelabiamea.gtex}

\vfill
%\pagebreak

\vspace{2mm}

\ifx\invitatorium\undefined
\pars{Invitatorium.} \scriptura{Ps. 94, 1; Psalmus 94; \textbf{H451}}

\vspace{-6mm}

\antiphona{VI}{temporalia/inv-jubilemusdeo.gtex}\else
\invitatorium
\fi

\vfill
\pagebreak

\ifx\hymnusmatutinum\undefined
\ifx\matua\undefined
\else
\pars{Hymnus.}

{
\grechangedim{interwordspacetext}{0.10 cm plus 0.15 cm minus 0.05 cm}{scalable}%
\antiphona{II}{temporalia/hym-IpsumNunc.gtex}
\grechangedim{interwordspacetext}{0.22 cm plus 0.15 cm minus 0.05 cm}{scalable}%
}
\fi
\else
\hymnusmatutinum
\fi

\vspace{-3mm}

\vfill
\pagebreak

\ifx\matub\undefined
\else
% MAT B
\pars{Psalmus 1.} \scriptura{Ps. 30, 2; \textbf{H90}}

\vspace{-4mm}

\antiphona{VIII G}{temporalia/ant-intuaiustitia.gtex}

%\vspace{-2mm}

\scriptura{Ps. 30, 2-9}

%\vspace{-2mm}

\initiumpsalmi{temporalia/ps30i-initium-viii-G-auto.gtex}

\vspace{-1.5mm}

\input{temporalia/ps30i-viii-G.tex} \Abardot{}

\vfill
\pagebreak

\pars{Psalmus 2.} \scriptura{Ps. 66, 2}

\vspace{-4mm}

\antiphona{E}{temporalia/ant-illuminadomine.gtex}

%\vspace{-2mm}

\scriptura{Ps. 30, 10-17}

%\vspace{-2mm}

\initiumpsalmi{temporalia/ps30ii-initium-e-a-auto.gtex}

\input{temporalia/ps30ii-e-a.tex} \Abardot{}

\vfill
\pagebreak

\pars{Psalmus 3.} \scriptura{Ps. 30, 24}

\vspace{-4mm}

\antiphona{II D}{temporalia/ant-diligitedominum.gtex}

%\vspace{-5mm}

\scriptura{Ps. 30, 20-25}

%\vspace{-2mm}

\initiumpsalmi{temporalia/ps30iii-initium-ii-D-auto.gtex}

\input{temporalia/ps30iii-ii-D.tex} \Abardot{}

\vfill
\pagebreak
\fi

\pars{Versus.}

\ifx\matversus\undefined
\ifx\matub\undefined
\else
\noindent \Vbardot{} Dírige me, Dómine, in veritáte tua, et doce me.

\noindent \Rbardot{} Quia tu es Deus salútis meæ.
\fi
\else
\matversus
\fi

\vspace{5mm}

\sineinitiali{temporalia/oratiodominica-mat.gtex}

\vspace{5mm}

\pars{Absolutio.}

\cuminitiali{}{temporalia/absolutio-exaudi.gtex}

\vfill
\pagebreak

\cuminitiali{}{temporalia/benedictio-solemn-benedictione.gtex}

\vspace{7mm}

\lectioi

\noindent \Vbardot{} Tu autem, Dómine, miserére nobis.
\noindent \Rbardot{} Deo grátias.

\vfill
\pagebreak

\responsoriumi

\vfill
\pagebreak

\cuminitiali{}{temporalia/benedictio-solemn-unigenitus.gtex}

\vspace{7mm}

\lectioii

\noindent \Vbardot{} Tu autem, Dómine, miserére nobis.
\noindent \Rbardot{} Deo grátias.

\vfill
\pagebreak

\responsoriumii

\vfill
\pagebreak

\cuminitiali{}{temporalia/benedictio-solemn-spiritus.gtex}

\vspace{7mm}

\lectioiii

\noindent \Vbardot{} Tu autem, Dómine, miserére nobis.
\noindent \Rbardot{} Deo grátias.

\vfill
\pagebreak

\responsoriumiii

\vfill
\pagebreak

\rubrica{Reliqua omittuntur, nisi Laudes separandæ sint.}

\sineinitiali{temporalia/domineexaudi.gtex}

\vfill

\oratio

\vfill

\noindent \Vbardot{} Dómine, exáudi oratiónem meam.
\Rbardot{} Et clamor meus ad te véniat.

\vfill

\noindent \Vbardot{} Benedicámus Dómino.
\noindent \Rbardot{} Deo grátias.

\vfill

\noindent \Vbardot{} Fidélium ánimæ per misericórdiam Dei requiéscant in pace.
\Rbardot{} Amen.

\vfill
\pagebreak

\hora{Ad Laudes.} %%%%%%%%%%%%%%%%%%%%%%%%%%%%%%%%%%%%%%%%%%%%%%%%%%%%%
%\sideThumbs{Laudes}

\cantusSineNeumas

\vspace{0.5cm}
\grechangedim{interwordspacetext}{0.18 cm plus 0.15 cm minus 0.05 cm}{scalable}%
\cuminitiali{}{temporalia/deusinadiutorium-communis.gtex}
\grechangedim{interwordspacetext}{0.22 cm plus 0.15 cm minus 0.05 cm}{scalable}%

\vfill
\pagebreak

\ifx\hymnuslaudes\undefined
\ifx\laudbd\undefined
\else
\pars{Hymnus} \scriptura{Hilarius (\olddag{} 367)}

\grechangedim{interwordspacetext}{0.16 cm plus 0.15 cm minus 0.05 cm}{scalable}%
\cuminitiali{IV}{temporalia/hym-LucisLargitor.gtex}
\grechangedim{interwordspacetext}{0.22 cm plus 0.15 cm minus 0.05 cm}{scalable}%
\vspace{-3mm}
\fi
\else
\hymnuslaudes
\fi

\vfill
\pagebreak

\ifx\laudb\undefined
\else
\pars{Psalmus 1.} \scriptura{Ps. 41, 3; \textbf{H391}}

\vspace{-4mm}

\antiphona{II D}{temporalia/ant-sitivitanima.gtex}

%\vspace{-2mm}

\scriptura{Psalmus 41}

%\vspace{-2mm}

\initiumpsalmi{temporalia/ps41-initium-ii-D-auto.gtex}

%\vspace{-1.5mm}

\input{temporalia/ps41-ii-D.tex}

\vfill

\antiphona{}{temporalia/ant-sitivitanima.gtex}

\vfill
\pagebreak

\pars{Psalmus 2.}

\vspace{-4mm}

\antiphona{III a}{temporalia/ant-ostendenobisdomine.gtex}

%\vspace{-2mm}

\scriptura{Canticum Ecclesiastici, Sir. 36, 1-7.13-16}

%\vspace{-3mm}

\initiumpsalmi{temporalia/ecclesiastici-initium-iii-a-auto.gtex}

\input{temporalia/ecclesiastici-iii-a.tex} \Abardot{}

\vfill
\pagebreak

\pars{Psalmus 3.}

\vspace{-4mm}

\antiphona{II D}{temporalia/ant-operamanuumeius.gtex}

\scriptura{Psalmus 18, 1-7}

\initiumpsalmi{temporalia/ps18i-initium-ii-D-auto.gtex}

\input{temporalia/ps18i-ii-D.tex} \Abardot{}

\vfill
\pagebreak
\fi

\ifx\lectiobrevis\undefined
\ifx\laudb\undefined
\else
\pars{Lectio Brevis.} \scriptura{Ier. 15, 16}

\noindent Invénti sunt sermónes tui, et comédi eos, et factum est mihi verbum tuum in gáudium et in lætítiam cordis mei, quóniam invocátum est nomen tuum super me, Dómine Deus exercítuum.
\fi
\else
\lectiobrevis
\fi

\vfill

\ifx\responsoriumbreve\undefined
\ifx\laudbd\undefined
\else
\pars{Responsorium breve.} \scriptura{Ps. 32, 1.3}

\cuminitiali{VI}{temporalia/resp-exsultateiusti.gtex}
\fi
\else
\responsoriumbreve
\fi

\vfill
\pagebreak

\ifx\benedictus\undefined
\ifx\laudbd\undefined
\else
\pars{Canticum Zachariæ.} \scriptura{Lc. 1, 68; \textbf{H422}}

\vspace{-4mm}

{
\grechangedim{interwordspacetext}{0.18 cm plus 0.15 cm minus 0.05 cm}{scalable}%
\antiphona{IV E}{temporalia/ant-benedictusdominus.gtex}
\grechangedim{interwordspacetext}{0.22 cm plus 0.15 cm minus 0.05 cm}{scalable}%
}

%\vspace{-3mm}

\scriptura{Lc. 1, 68-79}

%\vspace{-2mm}

\cantusSineNeumas
\initiumpsalmi{temporalia/benedictus-initium-iv-E-auto.gtex}

%\vspace{-1.5mm}

\input{temporalia/benedictus-iv-E.tex} \Abardot{}
\fi
\else
\benedictus
\fi

\vspace{-1cm}

\vfill
\pagebreak

%\sideThumbs{{\scriptsize{}Fine horarum}}

\pars{Preces.}

\sineinitiali{}{temporalia/tonusprecum.gtex}

\ifx\preces\undefined
\ifx\laudb\undefined
\else
\noindent Salvátor noster fecit nos regnum et sacerdótium, ut hóstias Deo acceptábiles offerámus. \gredagger{} Grati ígitur eum invocémus:

\Rbardot{} Serva nos in tuo ministério, Dómine.

\noindent Christe, sacérdos ætérne, qui sanctum pópulo tuo sacerdótium concessísti, \gredagger{} concéde, ut spiritáles hóstias Deo acceptábiles iúgiter offerámus.

\Rbardot{} Serva nos in tuo ministério, Dómine.

\noindent Spíritus tui fructus nobis largíre propítius, \gredagger{} patiéntiam, benignitátem et mansuetúdinem.

\Rbardot{} Serva nos in tuo ministério, Dómine.

\noindent Da nobis te amáre, ut te, qui es cáritas, possideámus, \gredagger{} et bene ágere, ut per vitam étiam nostram te laudémus.

\Rbardot{} Serva nos in tuo ministério, Dómine.

\noindent Quæ frátribus nostris sunt utília, nos quǽrere concéde, \gredagger{} ut salútem facílius consequántur.

\Rbardot{} Serva nos in tuo ministério, Dómine.
\fi
\else
\preces
\fi

\vfill

\pars{Oratio Dominica.}

\cuminitiali{}{temporalia/oratiodominicaalt.gtex}

\vfill
\pagebreak

\rubrica{vel:}

\pars{Supplicatio Litaniæ.}

\cuminitiali{}{temporalia/supplicatiolitaniae.gtex}

\vfill

\pars{Oratio Dominica.}

\cuminitiali{}{temporalia/oratiodominica.gtex}

\vfill
\pagebreak

% Oratio. %%%
\oratio

\vspace{-1mm}

\vfill

\rubrica{Hebdomadarius dicit Dominus vobiscum, vel, absente sacerdote vel diacono, sic concluditur:}

\vspace{2mm}

\antiphona{C}{temporalia/dominusnosbenedicat.gtex}

\rubrica{Postea cantatur a cantore:}

\vspace{2mm}

\cuminitiali{IV}{temporalia/benedicamus-feria-laudes.gtex}

\vspace{1mm}

\vfill
\pagebreak

\end{document}

