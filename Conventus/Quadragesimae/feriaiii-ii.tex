\newcommand{\oratio}{\pars{Oratio.}

\noindent Custódi, Dómine, quǽsumus, Ecclésiam tuam propitiatióne perpétua et, quia sine te lábitur humána mortálitas, tuis semper auxíliis et abstrahátur a nóxiis et ad salutária dirigátur.

\pars{Pro commemoratione Sancti Cyrilli Episcopi Hierosolymitani et Ecclesiæ Doctoris.} \scriptura{Dn. 12, 3}

\vspace{-4mm}

\antiphona{VII a trans.}{temporalia/ant-quidoctifuerint.gtex}

\vfill

\noindent Deus, qui Ecclésiam tuam per beátum Cyríllum, epíscopum, ad mystéria salútis profúndius attingénda mirabíliter adduxísti, da nobis, eius intercessióne, Fílium tuum ita agnóscere, ut vitam abundántius habeámus.

\pars{Pro pace in universo mundo.} \scriptura{Sir. 50, 25; 2 Esdr. 4, 20; \textbf{H416}}

\vspace{-4mm}

\antiphona{II D}{temporalia/ant-dapacemdomine.gtex}

\vfill

\noindent Deus, a quo sancta desidéria, recta consília et iusta sunt ópera: da servis tuis illam, quam mundus dare non potest, pacem; ut et corda nostra mandátis tuis dédita, et hóstium subláta formídine, témpora sint tua protectióne tranquílla.

\noindent Per Dóminum nostrum Iesum Christum, Fílium tuum, qui tecum vivit et regnat in unitáte Spíritus Sancti, Deus, per ómnia sǽcula sæculórum.

\noindent \Rbardot{} Amen.}
\newcommand{\invitatorium}{\pars{Invitatorium.}

\vspace{-4mm}

\antiphona{IV*}{temporalia/inv-christumdominum-cumdox.gtex}}
\newcommand{\hymnusmatutinum}{\pars{Hymnus}

\cuminitiali{I}{temporalia/hym-NuncTempus.gtex}}
\newcommand{\matversus}{\noindent \Vbardot{} Ecce nunc tempus acceptábile.

\noindent \Rbardot{} Ecce nunc dies salútis.}
\newcommand{\lectioi}{\vspace{-4mm}

\pars{Lectio I.} \scriptura{Gn. 28, 1-5. 10-15}

\noindent De libro Génesis.

\noindent Vocávit ítaque Isaac Iacob, et benedíxit eum, præcepítque ei dicens: "Noli accípere cóniugem de génere Chánaan: sed vade, et proficíscere in Mesopotámiam Sýriæ, ad domum Báthuel patris matris tuæ, et áccipe tibi inde uxórem de filiábus Laban avúnculi tui.

\noindent Deus autem omnípotens benedícat tibi, et créscere te fáciat, atque multíplicet, ut sis in turbas populórum. Et det tibi benedictiónes Abrahæ, et sémini tuo post te: ut possídeas terram peregrinatiónis tuæ, quam pollícitus est avo tuo."

\noindent Cumque dimisísset eum Isaac, proféctus venit in Mesopotámiam Sýriæ ad Laban fílium Báthuel Syri, fratrem Rebéccæ matris suæ. Igitur egréssus Iacob de Bersabée, pergébat Haran.

\noindent Cumque venísset ad quemdam locum, et vellet in eo requiéscere post solis occúbitum, tulit de lapídibus qui iacébant, et suppónens cápiti suo, dormívit in eódem loco.

\noindent Vidítque in somnis scalam stantem super terram, et cacúmen illíus tangens cælum; ángelos quoque Dei ascendéntes et descendéntes per eam, et Dóminum inníxum scalæ dicéntem sibi: "Ego sum Dóminus Deus Abraham patris tui, et Deus Isaac. Terram, in qua dormis, tibi dabo et sémini tuo. Erítque semen tuum quasi pulvis terræ: dilatáberis ad occidéntem, et oriéntem, et septentriónem, et merídiem. Et benedicéntur in te et in sémine tuo cunctæ tribus terræ."

\noindent Et ero custos tuus quocúmque perréxeris, et redúcam te in terram hanc: nec dimíttam nisi complévero univérsa quæ dixi.}
\newcommand{\responsoriumi}{\pars{Responsorium 1.} \scriptura{\Rbardot{} Gn. 28, 17 \Vbardot{} ibid., 16; \textbf{H150}}

\vspace{-5mm}

\responsorium{I}{temporalia/resp-dumexiretiacob-CROCHU.gtex}{}}
\newcommand{\lectioii}{\pars{Lectio II.} \scriptura{Gn. 28, 16-22}

\noindent Cumque evigilásset Iacob de somno, ait: "Vere Dóminus est in loco isto, et ego nesciébam."

\noindent Pavénsque, "Quam terríbilis est, inquit, locus iste! non est hic áliud nisi domus Dei, et porta cæli."

\noindent Surgens ergo Iacob mane, tulit lápidem quem supposúerat cápiti suo, et eréxit in títulum, fundens óleum désuper.

\noindent Appellavítque nomen urbis Bethel, quæ prius Luza vocabátur. Vovit étiam votum, dicens: "Si fúerit Deus mecum, et custodíerit me in via, per quam ego ámbulo, et déderit mihi panem ad vescéndum, et vestiméntum ad induéndum, reversúsque fúero próspere ad domum patris mei, erit mihi Dóminus in Deum, et lapis iste, quem eréxi in títulum, vocábitur Domus Dei; cunctorúmque quæ déderis mihi, décimas ófferam tibi."}
\newcommand{\responsoriumii}{\pars{Responsorium 2.} \scriptura{\Rbardot{} Gn. 28, 20-22 \Vbardot{} ibid., 16; \textbf{H150}}

\vspace{-5mm}

\responsorium{I}{temporalia/resp-sidominusdeusmeus-CROCHU.gtex}{}}
\newcommand{\lectioiii}{\pars{Lectio III.} \scriptura{Ps. 140, 4-6: CCL 40, 2028-2029}

\noindent Ex Enarratiónibus sancti Augustíni epíscopi in Psalmos.

\noindent \emph{Dómine, clamávi ad te, exáudi me.} Hoc omnes póssumus dícere. Hoc non dico, totus Christus dicit. Sed magis ex persóna córporis dictum est; quia et cum hic esset, carnem portans orávit, et ex persóna córporis orávit Patrem, et cum oráret globi sánguinis de toto córpore eius distillábant. Sic scriptum est in Evangélio: \emph{Orávit Iesus inténta oratióne, et sudávit sánguinem.} Quid est de toto córpore sánguinis efflúxio, nisi de tota Ecclésia mártyrum pássio?

\noindent \emph{Dómine, clamávi ad te, exáudi me; inténde voci deprecatiónis meæ, dum clamávero ad te.} Iam finítum negótium clamándi putábas, cum díceres: \emph{Clamávi ad te?} Clamásti, iam noli esse secúrus. Si finíta est tribulátio, finítus est clamor; si autem manet tribulátio Ecclésiæ et córporis Christi usque in finem sǽculi, non tantum dicat: \emph{Clamávi ad te, exáudi me,} sed: \emph{Inténde voci deprecatiónis meæ, dum clamávero ad te.}

\noindent \emph{Dirigátur orátio mea tamquam incénsum in conspéctu tuo; elevátio mánuum meárum sacrifícium vespertínum.}

\noindent Hoc de ipso cápite solére intéllegi, omnis christiánus agnóscit. Declinánte enim iam die in vésperum, Dóminus in cruce ánimam depósuit receptúrus, non amísit invítus. Sed tamen et ibi nos figuráti sumus. Quid enim illíus pepéndit in ligno, nisi quod de nobis accépit? Et unde fíeri potest ut aliquándo Deus Pater dimíttat et déserat únicum Fílium, qui útique cum illo unus Deus est? Et tamen nostram infirmitátem figens in cruce, ubi \emph{vetus homo noster,} sicut dicit Apóstolus, \emph{confíxus est cruci cum illo,} ex voce ipsíus hóminis nostri clamávit: \emph{Deus meus, Deus meus, ut quid me dereliquísti?}

\noindent Illud ergo est sacrifícium vespertínum, pássio Dómini, crux Dómini, oblátio víctimæ salutáris, holocáustum accéptum Deo. Illud sacrifícium vespertínum fecit in resurrectióne munus matutínum. Orátio ergo pure dirécta de corde fidéli, tamquam de ara sancta surgit incénsum. Nihil est delectabílius odóre Dómini: sic óleant omnes qui credunt.

\noindent \emph{Vetus} ergo \emph{homo noster,} Apóstoli verba sunt, \emph{confíxus est cruci cum illo; ut evacuarétur,} inquit, \emph{corpus peccáti, ut ultra non serviámus peccáto.}}
\newcommand{\responsoriumiii}{\pars{Responsorium 3.} \scriptura{\Rbardot{} Gn. 28, 21-22 \Vbardot{} ibid., 20; \textbf{H150}}

\vspace{-5mm}

\responsorium{I}{temporalia/resp-eritmihidominusindeum-CROCHU-cumdox.gtex}{}}
\newcommand{\lectiobrevis}{\pars{Lectio Brevis.} \scriptura{Ioel 2, 12-13}

\noindent Convertímini ad me in toto corde vestro, in ieiúnio et in fletu et in planctu; et scíndite corda vestra et non vestiménta vestra, et convertímini ad Dóminum Deum vestrum, quia benígnus et miséricors est, pátiens et multæ misericórdiæ, et placábilis super malítia.}
\newcommand{\responsoriumbreve}{\pars{Responsorium breve.} \scriptura{Ps. 90, 3}

\cuminitiali{IV}{temporalia/resp-ipseliberavitme.gtex}}
\newcommand{\hymnuslaudes}{\pars{Hymnus}

\cuminitiali{D}{temporalia/hym-IamChriste.gtex}}
\newcommand{\preces}{\noindent Grátias Deo Patri, qui Unigénitum suum, Verbum carnem factum, nobis dedit,\gredagger{} ut sit noster cibus et vita.\grestar{} Eum deprecémur:

\Rbardot{} Verbum Christi hábitet abundánter in nobis.

\noindent Frequentióre verbi tui auditióne nos dúcere tempus hoc quadragesimále concéde,\grestar{} ut Christum, Pascha nostrum, in magna sollemnitáte religiósius honorémus.

\Rbardot{} Verbum Christi hábitet abundánter in nobis.

\noindent Spíritus Sanctus tuus nos dóceat,\grestar{} ut ad verum et bonum sequéndum dúbios et errántes hortémur.

\Rbardot{} Verbum Christi hábitet abundánter in nobis.

\noindent Præsta nobis, ut Christi tui áltius percipiámus mystérium\grestar{} et vita nostra id clárius exprimámus.

\Rbardot{} Verbum Christi hábitet abundánter in nobis.

\noindent Purífica et rénova Ecclésiam tuam his diébus salútis,\grestar{} ut magis ac magis testimónium tibi éadem reddat.

\Rbardot{} Verbum Christi hábitet abundánter in nobis.}
\newcommand{\benedictus}{\pars{Canticum Zachariæ.} \scriptura{Mt. 23, 9.10; \textbf{H152}}

\vspace{-4mm}

{
\grechangedim{interwordspacetext}{0.18 cm plus 0.15 cm minus 0.05 cm}{scalable}%
\antiphona{I g}{temporalia/ant-unusestenim.gtex}
\grechangedim{interwordspacetext}{0.22 cm plus 0.15 cm minus 0.05 cm}{scalable}%
}

%\vspace{-2mm}

\scriptura{Lc. 1, 68-79}

%\vspace{-2mm}

\initiumpsalmi{temporalia/benedictus-initium-i-f-auto.gtex}

%\vspace{-1mm}

\input{temporalia/benedictus-i-f.tex} \Abardot{}}
\newcommand{\magnificat}{\pars{Canticum B. Mariæ V.} \scriptura{Mt. 23, 8-10}

\vspace{-4mm}

{
\grechangedim{interwordspacetext}{0.18 cm plus 0.15 cm minus 0.05 cm}{scalable}%
\antiphona{IV d}{temporalia/ant-omnesenimvos.gtex}
\grechangedim{interwordspacetext}{0.22 cm plus 0.15 cm minus 0.05 cm}{scalable}%
}

%\vspace{-2mm}

\scriptura{Lc. 1, 46-55}

%\vspace{-2mm}

\cantusSineNeumas
\initiumpsalmi{temporalia/magnificat-initium-iv-d.gtex}

%\vspace{-1.5mm}

\input{temporalia/magnificat-iv-d.tex} \Abardot{}}
\newcommand{\oratiovesperas}{\pars{Oratio.}

\noindent Propitiáre Dómine supplicatiónibus nostris, et animárum nostrárum medére languóribus:~\gredagger{} ut remissióne percépta, in tua semper benedictióne lætémur.

\noindent Per Dóminum nostrum Iesum Christum, Fílium tuum, qui tecum vivit et regnat in unitáte Spíritus Sancti, Deus, per ómnia sǽcula sæculórum.

\noindent \Rbardot{} Amen.}
\newcommand{\hebdomada}{infra Hebdom. II per Annum.}
\newcommand{\matub}{Matutinum Hebdomadae B}
\newcommand{\laudb}{Laudes Hebdomadae B}
\newcommand{\laudbd}{Laudes Hebdomadae B vel D}

% LuaLaTeX

\documentclass[a4paper, twoside, 12pt]{article}
\usepackage[latin]{babel}
%\usepackage[landscape, left=3cm, right=1.5cm, top=2cm, bottom=1cm]{geometry} % okraje stranky
%\usepackage[landscape, a4paper, mag=1166, truedimen, left=2cm, right=1.5cm, top=1.6cm, bottom=0.95cm]{geometry} % okraje stranky
\usepackage[landscape, a4paper, mag=1400, truedimen, left=0.5cm, right=0.5cm, top=0.5cm, bottom=0.5cm]{geometry} % okraje stranky

\usepackage{fontspec}
\setmainfont[FeatureFile={junicode.fea}, Ligatures={Common, TeX}, RawFeature=+fixi]{Junicode}
%\setmainfont{Junicode}

% shortcut for Junicode without ligatures (for the Czech texts)
\newfontfamily\nlfont[FeatureFile={junicode.fea}, Ligatures={Common, TeX}, RawFeature=+fixi]{Junicode}

\usepackage{multicol}
\usepackage{color}
\usepackage{lettrine}
\usepackage{fancyhdr}

% usual packages loading:
\usepackage{luatextra}
\usepackage{graphicx} % support the \includegraphics command and options
\usepackage{gregoriotex} % for gregorio score inclusion
\usepackage{gregoriosyms}
\usepackage{wrapfig} % figures wrapped by the text
\usepackage{parcolumns}
\usepackage[contents={},opacity=1,scale=1,color=black]{background}
\usepackage{tikzpagenodes}
\usepackage{calc}
\usepackage{longtable}
\usetikzlibrary{calc}

\setlength{\headheight}{14.5pt}

\input{conventuscommune.tex} % Often used macros

\newcommand{\annusEditionis}{2021}

%%%% Vicekrat opakovane kousky

\newcommand{\anteOrationem}{
  \rubrica{Ante Orationem, cantatur a Superiore:}

  \pars{Supplicatio Litaniæ.}

  \cuminitiali{}{temporalia/supplicatiolitaniae.gtex}

  \pars{Oratio Dominica.}

  \cuminitiali{}{temporalia/oratiodominica.gtex}

  \rubrica{Deinde dicitur ab Hebdomadario:}

  \cuminitiali{}{temporalia/dominusvobiscum-solemnis.gtex}

  \rubrica{In choro monialium loco Dominus vobiscum dicitur:}

  \sineinitiali{temporalia/domineexaudi.gtex}
}

\setlength{\columnsep}{30pt} % prostor mezi sloupci

%%%%%%%%%%%%%%%%%%%%%%%%%%%%%%%%%%%%%%%%%%%%%%%%%%%%%%%%%%%%%%%%%%%%%%%%%%%%%%%%%%%%%%%%%%%%%%%%%%%%%%%%%%%%%
\begin{document}

% Here we set the space around the initial.
% Please report to http://home.gna.org/gregorio/gregoriotex/details for more details and options
\grechangedim{afterinitialshift}{2.2mm}{scalable}
\grechangedim{beforeinitialshift}{2.2mm}{scalable}
\grechangedim{interwordspacetext}{0.22 cm plus 0.15 cm minus 0.05 cm}{scalable}%
\grechangedim{annotationraise}{-0.2cm}{scalable}

% Here we set the initial font. Change 38 if you want a bigger initial.
% Emit the initials in red.
\grechangestyle{initial}{\color{red}\fontsize{38}{38}\selectfont}

\pagestyle{empty}

%%%% Titulni stranka
\begin{titulusOfficii}
\ifx\titulus\undefined
\nomenFesti{Feria III \hebdomada{}}
\else
\titulus
\fi
\end{titulusOfficii}

\vfill

\begin{center}
%Ad usum et secundum consuetudines chori \guillemotright{}Conventus Choralis\guillemotleft.

%Editio Sancti Wolfgangi \annusEditionis
\end{center}

\scriptura{}

\pars{}

\pagebreak

\renewcommand{\headrulewidth}{0pt} % no horiz. rule at the header
\fancyhf{}
\pagestyle{fancy}

\cantusSineNeumas

\ifx\oratio\undefined
\ifx\laudb\undefined
\else
\newcommand{\oratio}{\pars{Oratio.}

\noindent Dómine Iesu Christe, lux vera, qui omnes hómines illúminas ad salútem, nobis, quǽsumus, concéde virtútem, ut ante te vias pacis et iustítiæ præparémus.

\noindent Qui vivis et regnas cum Deo Patre in unitáte Spíritus Sancti, Deus, per ómnia sǽcula sæculórum.

\noindent \Rbardot{} Amen.}
\fi
\fi

\hora{Ad Matutinum.} %%%%%%%%%%%%%%%%%%%%%%%%%%%%%%%%%%%%%%%%%%%%%%%%%%%%%

\vspace{2mm}

\cuminitiali{}{temporalia/dominelabiamea.gtex}

\vfill
%\pagebreak

\vspace{2mm}

\ifx\invitatorium\undefined
\ifx\matuac\undefined
\else
\pars{Invitatorium.} \scriptura{Ps. 94, 1; Psalmus 94; \textbf{H451}}

\vspace{-6mm}

\antiphona{VI}{temporalia/inv-jubilemusdeo.gtex}
\fi
\ifx\matubd\undefined
\else
\pars{Invitatorium.} \scriptura{Cantor; Psalmus 94; \textbf{H449}}

\vspace{-6mm}

\antiphona{E}{temporalia/inv-regemmagnum.gtex}
\fi
\else
\invitatorium
\fi

\vfill
\pagebreak

\ifx\hymnusmatutinum\undefined
\ifx\matuac\undefined
\else
\pars{Hymnus}

\cuminitiali{IV}{temporalia/hym-SomnoRefectis.gtex}
\fi
\ifx\matubd\undefined
\else
\pars{Hymnus.} \scriptura{Gregorius Magnus (\olddag{} 604)}

{
\grechangedim{interwordspacetext}{0.10 cm plus 0.15 cm minus 0.05 cm}{scalable}%
\antiphona{I}{temporalia/hym-NocteSurgentes.gtex}
\grechangedim{interwordspacetext}{0.22 cm plus 0.15 cm minus 0.05 cm}{scalable}%
}
\fi
\else
\hymnusmatutinum
\fi

\vspace{-3mm}

\vfill
\pagebreak

\ifx\matub\undefined
\else
% MAT B
\pars{Psalmus 1.} \scriptura{Ps. 36, 5; \textbf{H93}}

\vspace{-4mm}

\antiphona{VI F}{temporalia/ant-reveladomino.gtex}

%\vspace{-2mm}

\scriptura{Ps. 36, 1-11}

%\vspace{-2mm}

\initiumpsalmi{temporalia/ps36i_xi-initium-vi-F-auto.gtex}

\input{temporalia/ps36i_xi-vi-F.tex} \Abardot{}

\vfill
\pagebreak

\pars{Psalmus 2.}

\vspace{-4mm}

\antiphona{II D}{temporalia/ant-iuniorfui.gtex}

\vspace{-2mm}

\scriptura{Ps. 36, 12-29}

\vspace{-2mm}

\initiumpsalmi{temporalia/ps36xii_xxix-initium-ii-D-auto.gtex}

\input{temporalia/ps36xii_xxix-ii-D.tex}

\vfill

\antiphona{}{temporalia/ant-iuniorfui.gtex}

\vfill
\pagebreak

\pars{Psalmus 3.} \scriptura{Ps. 36, 3}

\vspace{-4mm}

\antiphona{VI F}{temporalia/ant-speraindomino.gtex}

%\vspace{-2mm}

\scriptura{Ps. 36, 30-40}

%\vspace{-2mm}

\initiumpsalmi{temporalia/ps36iii-initium-vi-F-auto.gtex}

\input{temporalia/ps36iii-vi-F.tex} \Abardot{}

\vfill
\pagebreak
\fi
\ifx\matuc\undefined
\else
% MAT C
\pars{Psalmus 1.} \scriptura{Ps. 67, 2}

\vspace{-4mm}

\antiphona{VII a}{temporalia/ant-exsurgatdeus.gtex}

%\vspace{-2mm}

\scriptura{Ps. 67, 2-11}

\initiumpsalmi{temporalia/ps67i-initium-vii-a-auto.gtex}

\input{temporalia/ps67i-vii-a.tex} \Abardot{}

\vfill
\pagebreak

\pars{Psalmus 2.}

\vspace{-4mm}

\antiphona{I f}{temporalia/ant-deusnosterdeussalvos.gtex}

%\vspace{-2mm}

\scriptura{Ps. 67, 12-24}

%\vspace{-2mm}

\initiumpsalmi{temporalia/ps67ii-initium-i-f-auto.gtex}

\input{temporalia/ps67ii-i-f.tex} \Abardot{}

\vfill
\pagebreak

\pars{Psalmus 3.} \scriptura{Ps. 67, 27; \textbf{H96}}

\vspace{-4mm}

\antiphona{D}{temporalia/ant-inecclesiis.gtex}

%\vspace{-2mm}

\scriptura{Ps. 67, 25-36}

\initiumpsalmi{temporalia/ps67iii-initium-d-g2-auto.gtex}

\input{temporalia/ps67iii-d-g2.tex} \Abardot{}

\vfill
\pagebreak
\fi

\pars{Versus.}

\ifx\matversus\undefined
\ifx\matub\undefined
\else
\noindent \Vbardot{} Bonitátem et prudéntiam et sciéntiam doce me.

\noindent \Rbardot{} Quia præcéptis tuis crédidi.
\fi
\ifx\matuc\undefined
\else
\noindent \Vbardot{} Audiam quid loquátur Dóminus Deus.

\noindent \Rbardot{} Loquétur pacem ad plebem suam.
\fi
\else
\matversus
\fi

\vspace{5mm}

\sineinitiali{temporalia/oratiodominica-mat.gtex}

\vspace{5mm}

\pars{Absolutio.}

\cuminitiali{}{temporalia/absolutio-ipsius.gtex}

\vfill
\pagebreak

\cuminitiali{}{temporalia/benedictio-solemn-deus.gtex}

\vspace{7mm}

\lectioi

\noindent \Vbardot{} Tu autem, Dómine, miserére nobis.
\noindent \Rbardot{} Deo grátias.

\vfill
\pagebreak

\responsoriumi

\vfill
\pagebreak

\cuminitiali{}{temporalia/benedictio-solemn-christus.gtex}

\vspace{7mm}

\lectioii

\noindent \Vbardot{} Tu autem, Dómine, miserére nobis.
\noindent \Rbardot{} Deo grátias.

\vfill
\pagebreak

\responsoriumii

\vfill
\pagebreak

\cuminitiali{}{temporalia/benedictio-solemn-ignem.gtex}

\vspace{7mm}

\lectioiii

\noindent \Vbardot{} Tu autem, Dómine, miserére nobis.
\noindent \Rbardot{} Deo grátias.

\vfill
\pagebreak

\responsoriumiii

\vfill
\pagebreak

\rubrica{Reliqua omittuntur, nisi Laudes separandæ sint.}

\sineinitiali{temporalia/domineexaudi.gtex}

\vfill

\oratio

\vfill

\noindent \Vbardot{} Dómine, exáudi oratiónem meam.
\Rbardot{} Et clamor meus ad te véniat.

\vfill

\noindent \Vbardot{} Benedicámus Dómino.
\noindent \Rbardot{} Deo grátias.

\vfill

\noindent \Vbardot{} Fidélium ánimæ per misericórdiam Dei requiéscant in pace.
\Rbardot{} Amen.

\vfill
\pagebreak

\hora{Ad Laudes.} %%%%%%%%%%%%%%%%%%%%%%%%%%%%%%%%%%%%%%%%%%%%%%%%%%%%%

\cantusSineNeumas

\vspace{0.5cm}
\grechangedim{interwordspacetext}{0.18 cm plus 0.15 cm minus 0.05 cm}{scalable}%
\cuminitiali{}{temporalia/deusinadiutorium-communis.gtex}
\grechangedim{interwordspacetext}{0.22 cm plus 0.15 cm minus 0.05 cm}{scalable}%

\vfill
\pagebreak

\ifx\hymnuslaudes\undefined
\ifx\laudac\undefined
\else
\pars{Hymnus} \scriptura{Ambrosius (\olddag{} 397)}

\cuminitiali{I}{temporalia/hym-SplendorPaternae-hiemalis.gtex}
\fi
\ifx\laudbd\undefined
\else
\pars{Hymnus}

\grechangedim{interwordspacetext}{0.16 cm plus 0.15 cm minus 0.05 cm}{scalable}%
\cuminitiali{IV}{temporalia/hym-AEterneLucis.gtex}
\grechangedim{interwordspacetext}{0.22 cm plus 0.15 cm minus 0.05 cm}{scalable}%
\vspace{-3mm}
\fi
\else
\hymnuslaudes
\fi

\vfill
\pagebreak

\ifx\laudb\undefined
\else
\pars{Psalmus 1.} \scriptura{Ps. 42, 5; \textbf{H95}}

\vspace{-4mm}

\antiphona{VI F}{temporalia/ant-salutarevultusmei.gtex}

\scriptura{Psalmus 42.}

\initiumpsalmi{temporalia/ps42-initium-vi-F-auto.gtex}

\input{temporalia/ps42-vi-F.tex} \Abardot{}

\vfill
\pagebreak

\pars{Psalmus 2.} \scriptura{Is. 38, 20; \textbf{H95}}

\vspace{-7mm}

\antiphona{E}{temporalia/ant-cunctisdiebus.gtex}

\vspace{-4mm}

\scriptura{Canticum Ezechiæ, Is. 38, 10-20}

\vspace{-3mm}

\initiumpsalmi{temporalia/ezechiae-initium-e-auto.gtex}

\input{temporalia/ezechiae-e.tex} \Abardot{}

\vfill
\pagebreak

\pars{Psalmus 3.} \scriptura{Ps. 64, 2; \textbf{H96}}

\vspace{-4mm}

\antiphona{VIII a}{temporalia/ant-tedecet.gtex}

\vspace{-2mm}

\scriptura{Psalmus 64.}

\vspace{-2mm}

\initiumpsalmi{temporalia/ps64-initium-viii-A-auto.gtex}

\input{temporalia/ps64-viii-A.tex} \Abardot{}

\vfill
\pagebreak
\fi
\ifx\laudc\undefined
\else
\pars{Psalmus 1.} \scriptura{Ps. 83, 5}

\vspace{-4mm}

\antiphona{VIII G}{temporalia/ant-beatiquihabitant.gtex}

\vspace{-2mm}

\scriptura{Psalmus 84.}

\vspace{-2mm}

\initiumpsalmi{temporalia/ps84-initium-viii-G-auto.gtex}

\input{temporalia/ps84-viii-G.tex} \Abardot{}

\vfill
\pagebreak

\pars{Psalmus 2.}

\vspace{-4mm}

\antiphona{VII d}{temporalia/ant-denoctespiritusmeus.gtex}

\vspace{-2mm}

\scriptura{Canticum Isaiæ, Is. 26, 1-12}

\vspace{-2mm}

\initiumpsalmi{temporalia/isaiae3-initium-vii-d.gtex}

\input{temporalia/isaiae3-vii-d.tex} \Abardot{}

\vfill
\pagebreak

\pars{Psalmus 3.} \scriptura{Ps. 66, 2}

\vspace{-4mm}

\antiphona{E}{temporalia/ant-illuminadomine.gtex}

%\vspace{-2mm}

\scriptura{Psalmus 66.}

%\vspace{-2mm}

\initiumpsalmi{temporalia/ps66-initium-e.gtex}

\input{temporalia/ps66-e.tex} \Abardot{}

\vfill
\pagebreak
\fi

\ifx\lectiobrevis\undefined
\ifx\laudb\undefined
\else
\pars{Lectio Brevis.} \scriptura{1 Th. 5, 4-5}

\noindent Vos, fratres, non estis in ténebris, ut vos dies ille tamquam fur comprehéndat; omnes enim vos fílii lucis estis et fílii diéi. Non sumus noctis neque tenebrárum.
\fi
\ifx\laudc\undefined
\else
\pars{Lectio Brevis.} \scriptura{1 Io. 4, 14-15}

\noindent Nos vídimus et testificámur quóniam Pater misit Fílium salvatórem mundi. Quisque conféssus fúerit: Iesus est Fílius Dei, Deus in ipso manet, et ipse in Deo.
\fi
\else
\lectiobrevis
\fi

\vfill

\ifx\responsoriumbreve\undefined
\ifx\laudac\undefined
\else
\pars{Responsorium breve.}

\cuminitiali{VI}{temporalia/resp-benedictusdominus.gtex}
\fi
\ifx\laudbd\undefined
\else
\pars{Responsorium breve.} \scriptura{Ps. 118, 149.147}

\cuminitiali{VI}{temporalia/resp-vocemmeamaudi.gtex}
\fi
\else
\responsoriumbreve
\fi

\vfill
\pagebreak

\ifx\benedictus\undefined
\ifx\laudbd\undefined
\else
\pars{Canticum Zachariæ.} \scriptura{Lc. 1, 71; \textbf{H423}}

\vspace{-5mm}

{
\grechangedim{interwordspacetext}{0.18 cm plus 0.15 cm minus 0.05 cm}{scalable}%
\antiphona{I g\textsuperscript{5}}{temporalia/ant-demanuomnium.gtex}
\grechangedim{interwordspacetext}{0.22 cm plus 0.15 cm minus 0.05 cm}{scalable}%
}

%\vspace{-3mm}

\scriptura{Lc. 1, 68-79}

%\vspace{-1mm}

\initiumpsalmi{temporalia/benedictus-initium-i-g5-auto.gtex}

\input{temporalia/benedictus-i-g5.tex} \Abardot{}
\fi
\else
\benedictus
\fi

\vspace{-1cm}

\vfill
\pagebreak

\pars{Preces.}

\sineinitiali{}{temporalia/tonusprecum.gtex}

\ifx\preces\undefined
\ifx\laudb\undefined
\else
\noindent Salvatóri nostro benedicámus, qui sua resurrectióne mundum clarificávit, \gredagger{} et humíliter invocémus eum dicéntes:

\Rbardot{} Salva nos, Dómine, in sémita tua.

\noindent Resurrectiónem tuam, Dómine Iesu, oratióne cólimus matutína, \gredagger{} spes glóriæ tuæ diem nostrum illúminet.

\Rbardot{} Salva nos, Dómine, in sémita tua.

\noindent Súscipe, Dómine, vota et propósita nostra, \gredagger{} tamquam diéi nostri primítias.

\Rbardot{} Salva nos, Dómine, in sémita tua.

\noindent Tríbue in dilectióne tua nos hódie profícere, \gredagger{} ut ómnia in nostrum omniúmque bonum cooperéntur.

\Rbardot{} Salva nos, Dómine, in sémita tua.

\noindent Da, Dómine, sic lucére lucem nostram coram homínibus, \gredagger{} ut vídeant ópera nostra bona et Patrem gloríficent.

\Rbardot{} Salva nos, Dómine, in sémita tua.
\fi
\else
\preces
\fi

\vfill

\pars{Oratio Dominica.}

\cuminitiali{}{temporalia/oratiodominicaalt.gtex}

\vfill
\pagebreak

\rubrica{vel:}

\pars{Supplicatio Litaniæ.}

\cuminitiali{}{temporalia/supplicatiolitaniae.gtex}

\vfill

\pars{Oratio Dominica.}

\cuminitiali{}{temporalia/oratiodominica.gtex}

\vfill
\pagebreak

% Oratio. %%%
\oratio

\vspace{-1mm}

\vfill

\rubrica{Hebdomadarius dicit Dominus vobiscum, vel, absente sacerdote vel diacono, sic concluditur:}

\vspace{2mm}

\antiphona{C}{temporalia/dominusnosbenedicat.gtex}

\rubrica{Postea cantatur a cantore:}

\vspace{2mm}

\cuminitiali{IV}{temporalia/benedicamus-feria-laudes.gtex}

\vspace{1mm}

\vfill
\pagebreak

\end{document}

