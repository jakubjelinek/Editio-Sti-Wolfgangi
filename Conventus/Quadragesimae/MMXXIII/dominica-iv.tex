\newcommand{\titulus}{\nomenFesti{Dominica IV in Quadragesima.}
%\celebratio{1. Classis. Semiduplex}
}
\newcommand{\tempquad}{Tempore Quadragesimae}
\newcommand{\oratio}{\pars{Oratio.}

\noindent Deus, qui per Verbum tuum humáni géneris reconciliatiónem mirabíliter operáris, præsta, quǽsumus, ut pópulus christiánus prompta devotióne et álacri fide ad ventúra sollémnia váleat festináre.

\pars{Pro pace in Ucraina.} \scriptura{Sir. 50, 25; 2 Esdr. 4, 20; \textbf{H416}}

\vspace{-4mm}

\antiphona{II D}{temporalia/ant-dapacemdomine.gtex}

\vfill

\noindent Deus, a quo sancta desidéria, recta consília et iusta sunt ópera: da servis tuis illam, quam mundus dare non potest, pacem; ut et corda nostra mandátis tuis dédita, et hóstium subláta formídine, témpora sint tua protectióne tranquílla.

\noindent Per Dóminum nostrum Iesum Christum, Fílium tuum, qui tecum vivit et regnat in unitáte Spíritus Sancti, Deus, per ómnia sǽcula sæculórum.

\noindent \Rbardot{} Amen.}
\newcommand{\capitulum}{\pars{Capitulum.} \scriptura{Gal. 4, 22-24}

\grechangedim{interwordspacetext}{0.12 cm plus 0.15 cm minus 0.05 cm}{scalable}%
\cuminitiali{}{temporalia/capitulum-FratresScriptum.gtex}
\grechangedim{interwordspacetext}{0.22 cm plus 0.15 cm minus 0.05 cm}{scalable}}
\newcommand{\magnificati}{\pars{Canticum B. Mariæ V.} \scriptura{Io. 8, 10-11; \textbf{H158}}

\vspace{-4mm}

{
\grechangedim{interwordspacetext}{0.18 cm plus 0.15 cm minus 0.05 cm}{scalable}%
\antiphona{III a}{temporalia/ant-nemotecondemnavit.gtex}
\grechangedim{interwordspacetext}{0.22 cm plus 0.15 cm minus 0.05 cm}{scalable}%
}

%\vspace{-3mm}

\scriptura{Lc. 1, 46-55}

%\vspace{-2mm}

\cantusSineNeumas
\initiumpsalmi{temporalia/magnificat-initium-iiisoll-a.gtex}

%\vspace{-1.5mm}

\input{temporalia/magnificat-iiisoll-a.tex} \Abardot{}}
\newcommand{\invitatorium}{\pars{Invitatorium.}

\vspace{-4mm}

\antiphona{IV*}{temporalia/inv-christumdominum-cumdox.gtex}}
\newcommand{\hymnusmatutinum}{\pars{Hymnus.} \scriptura{Gregorius Magnus (\olddag{} 604)}

\vspace{-5mm}

\antiphona{I}{temporalia/hym-ExMore.gtex}}
\newcommand{\nocturnoii}{\vspace{-4mm}

\pars{Psalmus 4.} \scriptura{Ps. 23, 1}

\vspace{-4mm}

\antiphona{VIII G}{temporalia/ant-dominiestterra.gtex}

%\vspace{-2mm}

\scriptura{Ps. 23}

%\vspace{-2mm}

\initiumpsalmi{temporalia/ps23-initium-viii-G-auto.gtex}

\input{temporalia/ps23-viii-G.tex} \Abardot{}

\vfill
\pagebreak

\pars{Psalmus 5.} \scriptura{Ps. 65, 8}

\vspace{-4mm}

\antiphona{VI F}{temporalia/ant-benedicitegentes.gtex}

%\vspace{-2mm}

\scriptura{Ps. 65, 1-12}

\initiumpsalmi{temporalia/ps65i-initium-vi-F-auto.gtex}

\input{temporalia/ps65i-vi-F.tex} \Abardot{}

\vfill
\pagebreak

\pars{Psalmus 6.} \scriptura{Ps. 5, 8; \textbf{H312}}

\vspace{-4mm}

\antiphona{II* a}{temporalia/ant-introiboindomumtuam.gtex}

%\vspace{-4mm}

\scriptura{Ps. 65, 13-20}

%\vspace{-2mm}

\initiumpsalmi{temporalia/ps65ii-initium-ii_-a-auto.gtex}

%\vspace{-1.5mm}

\input{temporalia/ps65ii-ii_-a.tex} \Abardot{}

\vfill
\pagebreak}
\newcommand{\nocturnoiii}{\pars{Cantica.} \scriptura{Ier. 14, 19.20}

\vspace{-4mm}

\antiphona{I d}{temporalia/ant-sustinuimuspacem.gtex}

%\vspace{-2mm}

\scriptura{Canticum Ieremiæ, Ier. 14, 17-21}

%\vspace{-2mm}

\initiumpsalmi{temporalia/jeremiae2-initium-i-d.gtex}

\input{temporalia/jeremiae2-i-d.tex} \hfill \rubrica{Hic non dicitur antiphona.}

\vfill
\pagebreak

\scriptura{Canticum Ezechiæ, Ez. 36, 24-28}

%\vspace{-2mm}

\initiumpsalmi{temporalia/ezechiae2-initium-i-d-auto.gtex}

\input{temporalia/ezechiae2-i-d.tex}

\vfill
\pagebreak

\scriptura{Canticum, Lam. 5, 1-7.15-17.19-21}

%\vspace{-2mm}

\initiumpsalmi{temporalia/lamentatio-initium-i-d-auto.gtex}

\input{temporalia/lamentatio-i-d.tex}

\vfill
\pagebreak

\antiphona{}{temporalia/ant-sustinuimuspacem.gtex}

\vfill
\pagebreak}
\newcommand{\matversusi}{\pars{Versus.}

\noindent \Vbardot{} Ipse liberávit me de láqueo venántium.

\noindent \Rbardot{} Et a verbo áspero.}
\newcommand{\matversusii}{\pars{Versus.}

\noindent \Vbardot{} Scápulis suis obumbrábit tibi.

\noindent \Rbardot{} Et sub pennis eius sperábis.}
\newcommand{\lectioi}{\pars{Lectio I.} \scriptura{Ex. 3, 1-6}

\noindent De libro Exodi.

\noindent In diébus illis: Móyses pascébat oves Iethro sóceri sui sacerdótis Mádian; cumque minásset gregem ultra desértum, venit ad montem Dei Horeb. Apparuítque ei ángelus Dómini in flamma ignis de médio rubi; et vidébat quod rubus ardéret et non comburerétur. Dixit ergo Móyses: «Vadam et vidébo visiónem hanc magnam, quare non comburátur rubus». Cernens autem Dóminus quod pérgeret ad vidéndum, vocávit eum Deus de médio rubi et ait: «Móyses, Móyses». Qui respóndit: «Adsum». At ille: «Ne apprópies, inquit, huc; solve calceaméntum de pédibus tuis; locus enim, in quo stas, terra sancta est». Et ait: «Ego sum Deus patris tui, Deus Abraham, Deus Isaac et Deus Iacob». Abscóndit Móyses fáciem suam; non enim audébat aspícere contra Deum.}
\newcommand{\responsoriumi}{\pars{Responsorium 1.} \scriptura{\Rbardot{} Ex. 7, 14 \Vbardot{} ibid., 3, 7; \textbf{H159}}

\vspace{-5mm}

\responsorium{IV}{temporalia/resp-locutusestdominusadmoysen-CROCHU.gtex}{}}
\newcommand{\lectioii}{\pars{Lectio II.} \scriptura{Ex. 3, 7-11}

\noindent Cui ait Dóminus: «Vidi afflictiónem pópuli mei in Ægýpto et clamórem eius audívi propter durítiam exactórum eórum. Et sciens dolórem eius descéndi, ut líberem eum de mánibus Ægyptiórum et edúcam de terra illa in terram bonam et spatiósam, in terram, quæ fluit lacte et melle, ad loca Chananǽi et Hetthǽi et Amorrǽi et Pherezǽi et Hevǽi et Iebusǽi. Clamor ergo filiórum Israel venit ad me, vidíque afflictiónem eórum, qua ab Ægýptiis opprimúntur; sed veni, mittam te ad pharaónem, ut edúcas pópulum meum, fílios Israel, de Ægýpto». Dixítque Móyses ad Deum: «Quis sum ego, ut vadam ad pharaónem et edúcam fílios Israel de Ægýpto?». Qui dixit ei: «Ego ero tecum; et hoc habébis signum quod míserim te: cum edúxeris pópulum de Ægýpto, serviétis Deo super montem istum».}
\newcommand{\responsoriumii}{\pars{Responsorium 2.} \scriptura{\Rbardot{} Ex. 5, 1; \textbf{H159}}

\vspace{-5mm}

\responsorium{VIII}{temporalia/resp-stetitmoysescoram-CROCHU.gtex}{}}
\newcommand{\lectioiii}{\pars{Lectio III.} \scriptura{Ex. 3, 12-20}

\noindent Ait Móyses ad Deum: «Ecce, ego vadam ad fílios Israel et dicam eis: Deus patrum vestrórum misit me ad vos. Si díxerint mihi: “Quod est nomen eius?”, quid dicam eis?». Dixit Deus ad Móysen: «Ego sum qui sum». Ait: «Sic dices fíliis Israel: Qui sum misit me ad vos». Dixítque íterum Deus ad Móysen: «Hæc dices fíliis Israel: Dóminus, Deus patrum vestrórum, Deus Abraham, Deus Isaac et Deus Iacob misit me ad vos; hoc nomen mihi est in ætérnum, et hoc memoriále meum in generatiónem et generatiónem. Vade et cóngrega senióres Israel et dices ad eos: Dóminus, Deus patrum vestrórum, appáruit mihi, Deus Abraham, Deus Isaac et Deus Iacob dicens: Vísitans visitávi vos et vidi ómnia, quæ accidérunt vobis in Ægýpto; et dixi: Edúcam vos de afflictióne Ægýpti in terram Chananǽi et Hetthǽi et Amorrǽi et Pherezǽi et Hevǽi et Iebusǽi, ad terram fluéntem lacte et melle. Et áudient vocem tuam, ingredierísque tu et senióres Israel ad regem Ægýpti, et dicétis ad eum: Dóminus, Deus Hebræórum, occúrrit nobis; et nunc eámus viam trium diérum in solitúdinem, ut immolémus Dómino Deo nostro. Sed ego scio quod non dimíttet vos rex Ægýpti, ut eátis, nisi per manum válidam. Exténdam enim manum meam et percútiam Ægýptum in cunctis mirabílibus meis, quæ factúrus sum in médio eius; post hæc dimíttet vos».}
\newcommand{\responsoriumiii}{\pars{Responsorium 3.} \scriptura{\Rbardot{} Ex. 15, 1-2 \Vbardot{} ibid., 4; \textbf{H159}}

\vspace{-5mm}

\responsorium{VIII}{temporalia/resp-cantemusdominogloriose-CROCHU-cumdox.gtex}{}}
\newcommand{\lectioiv}{\pars{Lectio IV.} \scriptura{Tract. 34, 8-9: CCL 36, 315-316}

\noindent Ex Tractátibus sancti Augustíni epíscopi in Ioánnem.

\noindent Quóniam Dóminus bréviter ait: \emph{Ego sum lux mundi; qui me séquitur, non ambulábit in ténebris, sed habébit lumen vitæ,} quibus verbis áliud est quod iussit, áliud quod promísit, faciámus quod iussit, ne impudénti fronte desiderémus quod promísit; ne dicat nobis in iudício suo: Fecísti enim quod iussi, ut éxpetas quod promísi? Quid ergo iussísti, Dómine Deus noster? Dicit tibi: Ut sequeréris me. Consílium vitæ petísti. Cuius vitæ, nisi de qua dictum est: \emph{Apud te fons vitæ?}

\noindent Ergo modo faciámus, sequámur Dóminum; solvámus cómpedes quibus impedímur sequi. Et quis idóneus sólvere tales nodos, nisi ille ádiuvet cui dictum est: \emph{Dirupísti víncula mea?} De quo álius psalmus dicit: \emph{Dóminus solvit compedítos, Dóminus érigit elísos.}}
\newcommand{\responsoriumiv}{\pars{Responsorium 4.} \scriptura{\Rbardot{} Ps. 76, 20 \Vbardot{} ibid., 19; \textbf{H159}}

\vspace{-5mm}

\responsorium{II}{temporalia/resp-inmariviatua-CROCHU.gtex}{}}
\newcommand{\lectiov}{\pars{Lectio V.}

\noindent Et quid sequúntur solúti et erécti, nisi lumen a quo áudiunt: \emph{Ego sum lumen mundi; qui me séquitur, non ambulábit in ténebris?} quia Dóminus illúminat cæcos. Illuminámur ergo modo, fratres, habéntes collýrium fídei. Præcéssit enim eius salíva cum terra, unde inungerétur qui cæcus est natus. Et nos de Adam cæci nati sumus, et illo illuminánte opus habémus. Míscuit salívam cum terra: \emph{Verbum caro factum est et habitávit in nobis.} Míscuit salívam cum terra; ídeo prædíctum est: \emph{Véritas de terra orta est}; ipse autem dixit: \emph{Ego sum via, véritas et vita.}

\noindent Veritáte perfruémur, cum vidérimus fácie ad fáciem, quia et hoc promíttitur nobis. Nam quis audéret speráre quod Deus non dignátus esset vel pollicéri vel dare?

\noindent Vidébimus fácie ad fáciem. Apóstolus dicit: \emph{Nunc scio ex parte, nunc in ænígmate per spéculum, tunc autem fácie ad fáciem.} Et Ioánnes apóstolus in epístola sua: \emph{Dilectíssimi, nunc fílii Dei sumus et nondum appáruit quid érimus; scimus quia, cum apparúerit, símiles ei érimus, quóniam vidébimus eum sícuti est.} Hæc est magna promíssio.}
\newcommand{\responsoriumv}{\pars{Responsorium 5.} \scriptura{\Rbardot{} Cf. Sap. 10, 19; \textbf{H160}}

\vspace{-5mm}

\responsorium{VII}{temporalia/resp-quipersequebanturpopulum-CROCHU.gtex}{}}
\newcommand{\lectiovi}{\pars{Lectio VI.}

\noindent Si amas, séquere. Amo, inquis, sed qua sequor? Si dixísset tibi Dóminus Deus tuus: Ego sum véritas et vita, desíderans veritátem, concupíscens vitam, viam qua ad hæc perveníre posses profécto quǽreres et díceres tibi: Magna res véritas, magna res vita, si esset quómodo illuc perveníret ánima mea!

\noindent Quæris qua? Audi eum dicéntem primo: \emph{Ego sum via.} Antequam díceret tibi quo, præmísit qua: \emph{Ego sum,} inquit, \emph{via.} Quo via? \emph{Et véritas et vita.} Primo dixit qua vénias, póstea dixit quo vénias. Ego sum via, ego sum véritas, ego vita. Manens apud Patrem, véritas et vita; índuens se carnem, factus est via.

\noindent Non tibi dícitur: Labóra quæréndo viam, ut pervénias ad veritátem et vitam; non hoc tibi dícitur. Piger, surge! via ipsa ad te venit, et te de somno dormiéntem excitávit, si tamen excitávit; surge, et ámbula.

\noindent Forte conáris ambuláre, et non potes, quia dolent pedes. Unde dolent pedes? an iubénte avarítia per áspera cucurrérunt? Sed Dei Verbum sanávit et claudos. Ecce, inquis, sanos hábeo pedes, sed ipsam viam non vídeo. Illuminávit et cæcos.}
\newcommand{\responsoriumvi}{\pars{Responsorium 6.} \scriptura{\Rbardot{} Ex. 24, 18; \textbf{H160}}

\vspace{-5mm}

\responsorium{VIII}{temporalia/resp-moysesfamulusdomini-CROCHU-cumdox.gtex}{}}
\newcommand{\evangelium}{\pars{Versus.}

\noindent \Vbardot{} Scuto circúmdabit te véritas eius.

\noindent \Rbardot{} Non timébis a timóre noctúrno.

\vspace{5mm}

\sineinitiali{temporalia/oratiodominica-mat.gtex}

\vspace{5mm}

\pars{Absolutio.}

\cuminitiali{}{temporalia/absolutio-avinculis.gtex}

\vfill
\pagebreak

\cuminitiali{}{temporalia/benedictio-solemn-evangelica.gtex}

\vspace{7mm}

\pars{Evangelium} \scriptura{Io 9, 1-41}

\noindent Léctio sancti Evangélii secúndum Ioánnem.

\noindent In illo témpore:

\noindent Prætériens Iesus vidit hóminem cæcum a nativitáte.

\noindent {\color{gray} Et interrogavérunt eum discípuli sui dicéntes: «Rabbi, quis peccávit, hic aut paréntes eius, ut cæcus nascerétur?».

\noindent Respóndit Iesus: «Neque hic peccávit neque paréntes eius, sed ut manifesténtur ópera Dei in illo. Nos opórtet operári ópera eius, qui misit me, donec dies est; venit nox, quando nemo potest operári. Quámdiu in mundo sum, lux sum mundi».

\noindent Hæc cum dixísset,} éxspuit in terram et fecit lutum ex sputo et linívit lutum super óculos eius et dixit ei: «Vade, lava in natatória Síloæ!»— quod interpretátur Missus—. Abiit ergo et lavit et venit videns.

\noindent Itaque vicíni et, qui vidébant eum prius quia méndicus erat, dicébant: «Nonne hic est, qui sedébat et mendicábat?»; álii dicébant: «Hic est!»; álii dicébant: «Nequáquam, sed símilis est eius!». Ille dicébat: «Ego sum!».

\noindent {\color{gray} Dicébant ergo ei: «Quómodo ígitur apérti sunt óculi tibi?».

\noindent Respóndit ille: «Homo, qui dícitur Iesus, lutum fecit et unxit óculos meos et dixit mihi: “Vade ad Síloam et lava!”. Abii ergo et lavi et vidi». Et dixérunt ei: «Ubi est ille?». Ait: «Néscio».}

\noindent Addúcunt eum ad pharisǽos, qui cæcus fúerat. Erat autem sábbatum, in qua die lutum fecit Iesus et apéruit óculos eius. Iterum ergo interrogábant et eum pharisǽi quómodo vidísset. Ille autem dixit eis: «Lutum pósuit super óculos meos, et lavi et vídeo».

\noindent Dicébant ergo ex pharisǽis quidam: «Non est hic homo a Deo, quia sábbatum non custódit!»; álii autem dicébant: «Quómodo potest homo peccátor hæc signa fácere?». Et schisma erat in eis. Dicunt ergo cæco íterum: «Tu quid dicis de eo quia apéruit óculos tuos?». Ille autem dixit: «Prophéta est!».

\noindent {\color{gray} Non credidérunt ergo Iudǽi de illo quia cæcus fuísset et vidísset, donec vocavérunt paréntes eius, qui víderat. Et interrogavérunt eos dicéntes: «Hic est fílius vester, quem vos dícitis quia cæcus natus est? Quómodo ergo nunc videt?». Respondérunt ergo paréntes eius et dixérunt: «Scimus quia hic est fílius noster et quia cæcus natus est. Quómodo autem nunc vídeat nescímus, aut quis eius apéruit óculos nos nescímus; ipsum interrogáte. Ætátem habet; ipse de se loquétur!». Hæc dixérunt paréntes eius, quia timébant Iudǽos; iam enim conspiráverant Iudǽi, ut, si quis eum confiterétur Christum, extra synagógam fíeret. Proptérea paréntes eius dixérunt: «Ætátem habet; ipsum interrogáte!».

\noindent Vocavérunt ergo rursum hóminem, qui fúerat cæcus, et dixérunt ei: «Da glóriam Deo! Nos scimus quia hic homo peccátor est». Respóndit ergo ille: «Si peccátor est néscio; unum scio quia, cæcus cum essem, modo vídeo». Dixérunt ergo illi: «Quid fecit tibi? Quómodo apéruit óculos tuos?». Respóndit eis: «Dixi vobis iam, et non audístis; quid íterum vultis audíre? Numquid et vos vultis discípuli eius fíeri?».

\noindent Et maledixérunt ei et dixérunt: «Tu discípulus illíus es, nos autem Móysis discípuli sumus. Nos scimus quia Móysi locútus est Deus; hunc autem nescímus unde sit».

\noindent Respóndit homo et dixit eis: «In hoc enim mirábile est, quia vos nescítis unde sit, et apéruit meos óculos! Scimus quia peccatóres Deus non audit; sed, si quis Dei cultor est et voluntátem eius facit, hunc exáudit. A sǽculo non est audítum quia apéruit quis óculos cæci nati; nisi esset hic a Deo, non póterat fácere quidquam».}

\noindent Respondérunt et dixérunt ei: «In peccátis tu natus es totus et tu doces nos?». Et eiecérunt eum foras.

\noindent Audívit Iesus quia eiecérunt eum foras et, cum invenísset eum, dixit ei: «Tu credis in Fílium hóminis?». Respóndit ille et dixit: «Et quis est, Dómine, ut credam in eum?». Dixit ei Iesus: «Et vidísti eum; et, qui lóquitur tecum, ipse est». At ille ait: «Credo, Dómine!»; et adorávit eum.

\noindent {\color{gray} Et dixit Iesus: «In iudícium ego in hunc mundum veni, ut, qui non vident, vídeant, et, qui vident, cæci fiant». Audiérunt hæc ex pharisǽis, qui cum ipso erant, et dixérunt ei: «Numquid et nos cæci sumus?». Dixit eis Iesus: «Si cæci essétis, non haberétis peccátum. Nunc vero dícitis: “Vidémus!”; peccátum vestrum manet».}

\vfill
\pagebreak

\pars{Responsorium 7.} \scriptura{\Rbardot{} Ex. 23, 20 \Vbardot{} Ps. 80, 9-10; \textbf{H160}}

\vspace{-5mm}

\responsorium{IV}{temporalia/resp-eccemittoangelummeum-CROCHU-cumdox.gtex}{}

\vfill
\pagebreak
}
\newcommand{\laudes}{\pars{Psalmus 1.} \scriptura{Ps. 117, 28; \textbf{H142}}

\vspace{-4mm}

\antiphona{VIII c}{temporalia/ant-deusmeusestu.gtex}

\scriptura{Psalmus 117.}

\initiumpsalmi{temporalia/ps117-initium-viii-C-auto.gtex}

\input{temporalia/ps117-viii-C.tex}

\vfill

\antiphona{}{temporalia/ant-deusmeusestu.gtex}

\vfill
\pagebreak

\pars{Psalmus 2.} \scriptura{Cf. Dan. 3, 17.88; \textbf{H162}}

\vspace{-4mm}

\antiphona{VIII G}{temporalia/ant-potensesdomine.gtex}

\scriptura{Canticum Danielis, Dan. 3, 52-57}

%\vspace{-3mm}

\initiumpsalmi{temporalia/dan33-initium-viii-G-auto.gtex}

\input{temporalia/dan33-viii-G.tex} \Abardot{}

\vfill
\pagebreak

\pars{Psalmus 3.} \scriptura{Ps. 150, 5; \textbf{H100}}

\vspace{-4mm}

\antiphona{E}{temporalia/ant-incymbalisbenesonantibus.gtex}

\scriptura{Psalmus 150.}

\initiumpsalmi{temporalia/ps150-initium-e-e-auto.gtex}

\input{temporalia/ps150-e-e.tex} \Abardot{}

\vfill
\pagebreak}
\newcommand{\lectiobrevis}{\pars{Lectio brevis.} \scriptura{Neh. 8, 9,10}

\noindent Dies iste sanctificátus est Dómino Deo nostro! Nolíte lugére et nolíte flere. Quia sanctus dies Dómini nostri est; et nolíte contristári, gáudium étenim Dómini est fortitúdo vestra.}
\newcommand{\responsoriumbreve}{\pars{Responsorium breve.}

\cuminitiali{IV}{temporalia/resp-christefilidei-tq.gtex}}
\newcommand{\hymnuslaudes}{\pars{Hymnus} \scriptura{Gregorius Magnus (\olddag{} 604)}

\cuminitiali{II}{temporalia/hym-PrecemurOmnes.gtex}}
\newcommand{\benedictus}{\pars{Canticum Zachariæ.} \scriptura{Io. 9, 2-3; \textbf{H163}}

\vspace{-4mm}

{
\grechangedim{interwordspacetext}{0.18 cm plus 0.15 cm minus 0.05 cm}{scalable}%
\antiphona{VIII G\textsuperscript{2}}{temporalia/ant-rabbiquidpeccavit.gtex}
\grechangedim{interwordspacetext}{0.22 cm plus 0.15 cm minus 0.05 cm}{scalable}%
}

\vspace{-2mm}

\scriptura{Lc. 1, 68-79}

%\vspace{-2mm}

\cantusSineNeumas
\initiumpsalmi{temporalia/benedictus-initium-viiisoll-G5-auto.gtex}

%\vspace{-1.5mm}

\input{temporalia/benedictus-viiisoll-G5.tex}

\vfill

{
\grechangedim{interwordspacetext}{0.18 cm plus 0.15 cm minus 0.05 cm}{scalable}%
\antiphona{}{temporalia/ant-rabbiquidpeccavit.gtex}
\grechangedim{interwordspacetext}{0.22 cm plus 0.15 cm minus 0.05 cm}{scalable}%
}}
\newcommand{\magnificatii}{\pars{Canticum B. Mariæ V.} \scriptura{Io. 6, 3.4}

\vspace{-4mm}

{
\grechangedim{interwordspacetext}{0.18 cm plus 0.15 cm minus 0.05 cm}{scalable}%
\antiphona{I g}{temporalia/ant-subiitergoinmontem.gtex}
\grechangedim{interwordspacetext}{0.22 cm plus 0.15 cm minus 0.05 cm}{scalable}%
}

\vspace{-2mm}

\scriptura{Lc. 1, 46-55}

\vspace{-2mm}

\cantusSineNeumas
\initiumpsalmi{temporalia/magnificat-initium-isoll-g.gtex}

\vspace{-1.5mm}

\input{temporalia/magnificat-isoll-g.tex} \Abardot{}}
\newcommand{\hebdomada}{infra Hebdom. IV post Pentecosten.}
\newcommand{\oratioLaudes}{\cuminitiali{}{temporalia/oratio4.gtex}}

% LuaLaTeX

\documentclass[a4paper, twoside, 12pt]{article}
\usepackage[latin]{babel}
%\usepackage[landscape, left=3cm, right=1.5cm, top=2cm, bottom=1cm]{geometry} % okraje stranky
%\usepackage[landscape, a4paper, mag=1166, truedimen, left=2cm, right=1.5cm, top=1.6cm, bottom=0.95cm]{geometry} % okraje stranky
\usepackage[landscape, a4paper, mag=1400, truedimen, left=0.5cm, right=0.5cm, top=0.5cm, bottom=0.5cm]{geometry} % okraje stranky

\usepackage{fontspec}
\setmainfont[FeatureFile={junicode.fea}, Ligatures={Common, TeX}, RawFeature=+fixi]{Junicode}
%\setmainfont{Junicode}

% shortcut for Junicode without ligatures (for the Czech texts)
\newfontfamily\nlfont[FeatureFile={junicode.fea}, Ligatures={Common, TeX}, RawFeature=+fixi]{Junicode}

\usepackage{multicol}
\usepackage{color}
\usepackage{lettrine}
\usepackage{fancyhdr}

% usual packages loading:
\usepackage{luatextra}
\usepackage{graphicx} % support the \includegraphics command and options
\usepackage{gregoriotex} % for gregorio score inclusion
\usepackage{gregoriosyms}
\usepackage{wrapfig} % figures wrapped by the text
\usepackage{parcolumns}
\usepackage[contents={},opacity=1,scale=1,color=black]{background}
\usepackage{tikzpagenodes}
\usepackage{calc}
\usepackage{longtable}
\usetikzlibrary{calc}

\setlength{\headheight}{14.5pt}

\input{conventuscommune.tex} % Often used macros
%%%% Preklady jednotlivych zpevu (nektere se opakuji, a je dobre mit je
% vsechny na jedne hromade)

% HOURS ---

\newcommand{\trAntI}{\translatioCantus{Muž boží měl kožený toulec, pečlivě
zavázaný, jenž mu visel na šíji a~často se ho dotýkal.}}

\newcommand{\trAntII}{\translatioCantus{Klíč od~něho tak dobře střežil, že
dokud žil v~těle, nikdo z~jeho žáků nezvěděl, co je uvnitř.}}

\newcommand{\trAntIII}{\translatioCantus{Ale když se odebral z~tohoto
života, schránku otevřeli a~objevili v~ní žíněné roucho a~měděný řetěz
potřísněný krví.}}

\newcommand{\trAntIV}{\translatioCantus{A když prohlédli mistrovo tělo,
nalezli jeho tělo na čtyřech místech hluboce zbrázděno ranami od řetězu.}}

\newcommand{\trAntV}{\translatioCantus{Krev vytékající z~těch ran, místy
prostoupila i~žíněným rouchem.}}

\newcommand{\trCapituli}{\translatioCantus{
Miláčkovi Boha a~lidí,
Mojžíšovi požehnané paměti,~\gredagger{}
dopřál slávu rovnou slávě svatých~\grestar{}
učinil ho mocným na postrach nepřátelům
a~jeho slovy zastavil divy.}}

\newcommand{\trLectioBrevis}{\translatioCantus{
Pamatujte na své představené,
kteří vám hlásali Boží slovo.
Uvažte, jak oni skončili život, a~napodobujte jejich víru.
Ježíš Kristus je stejný včera i~dnes i~navěky.
Nenechte se svést věelijakými cizími naukami.}}

\newcommand{\trRespLaud}{\translatioCantus{Spravedlivého vodil Hospodin~\grestar{}
po přímých stezkách. \Vbardot{} A~ukázal mu Boží království.}}

\newcommand{\trRespLaudB}{\translatioCantus{Na tvých hradbách, Jeruzaléme,
ustanovil jsem strážné;~\grestar{}
budou bdít nad mým lidem. \Vbardot{} Ani ve dne, ani v~noci nesmějí nikdy
mlčet.}}

\newcommand{\trVersus}{\translatioCantus{\Vbardot{} Ústa spravedlivého šeptají moudrost, aleluja.
\Rbardot{} A~jeho jazyk ohlašuje právo, aleluja.}}

\newcommand{\trAntBenedictus}{\translatioCantus{Když na bujné oře vložili
nosítka a~sňali jim uzdu, vydali se přímo k~cele božího muže.}}

\newcommand{\trPreces}{\translatioCantus{
\noindent S vděčností chvalme Krista, dobrého Pastýře, \gredagger{} který dal život za své ovce, \grestar{} a~pokorně ho prosme: \Rbardot{} Pane, buď pastýřem svého lidu.

\noindent Kriste, ty dáváš církvi pastýře, a~jejich službou se ujímáš svého lidu, \grestar{} dej, ať v~lásce těch, kteří nás vedou, poznáváme, jak nás miluješ. \Rbardot{} Pane, buď pastýřem svého lidu.

\noindent Ty stále konáš skrze své zástupce službu pastýře a~učitele, \grestar{} nepřestávej nás nikdy vést prostřednictvím svých služebníků. \Rbardot{} Pane, buď pastýřem svého lidu.

\noindent Ty prokazuješ svému lidu skrze jeho pastýře službu lékaře duše i~těla, \grestar{} ochraňuj náš život a~veď nás ke svatosti. \Rbardot{} Pane, buď pastýřem svého lidu.

\noindent Ty posíláš své svaté, aby slovem i~příkladem vedli tvůj lid k~tobě, \grestar{} na jejich přímluvu nás posiluj, abychom vytrvali na cestě, která vede k~věčnému životu. \Rbardot{} Pane, buď pastýřem svého lidu.}}

\newcommand{\trOrationis}{\translatioCantus{Bože, jenž nám dopřáváš radovat
se z~výroční slavnosti svatého tvého vyznavače Havla, uděl dobrotivě,
abychom když slavíme jeho narození, též se řídili podobou jeho skutků.
Skrze…}}
 % Czech translations of the proper texts

\newcommand{\annusEditionis}{2020}

%%%% Vicekrat opakovane kousky

\newcommand{\anteOrationem}{
  \rubrica{Ante Orationem, cantatur a Superiore:}

  \pars{Supplicatio Litaniæ.}

  \cuminitiali{}{temporalia/supplicatiolitaniae.gtex}

  \pars{Oratio Dominica.}

  \cuminitiali{}{temporalia/oratiodominica.gtex}

  \rubrica{Deinde dicitur ab Hebdomadario:}

  \cuminitiali{}{temporalia/dominusvobiscum-solemnis.gtex}

  \rubrica{In choro monialium loco Dominus vobiscum dicitur:}

  \sineinitiali{temporalia/domineexaudi.gtex}
}

\setlength{\columnsep}{30pt} % prostor mezi sloupci

%%%%%%%%%%%%%%%%%%%%%%%%%%%%%%%%%%%%%%%%%%%%%%%%%%%%%%%%%%%%%%%%%%%%%%%%%%%%%%%%%%%%%%%%%%%%%%%%%%%%%%%%%%%%%
\begin{document}

% Here we set the space around the initial.
% Please report to http://home.gna.org/gregorio/gregoriotex/details for more details and options
\grechangedim{afterinitialshift}{2.2mm}{scalable}
\grechangedim{beforeinitialshift}{2.2mm}{scalable}
\grechangedim{interwordspacetext}{0.22 cm plus 0.15 cm minus 0.05 cm}{scalable}%
\grechangedim{annotationraise}{-0.2cm}{scalable}

% Here we set the initial font. Change 38 if you want a bigger initial.
% Emit the initials in red.
\grechangestyle{initial}{\color{red}\fontsize{38}{38}\selectfont}

\pagestyle{empty}

%%%% Titulni stranka
\begin{titulusOfficii}
\titulus{}
\end{titulusOfficii}

% graphic
%\vspace{1.5cm}
%\begin{center}
%\includegraphics[width=8cm]{emmaus.jpg}
%\end{center}

\vfill

\begin{center}
%Ad usum et secundum consuetudines chori \guillemotright{}Conventus Choralis\guillemotleft.

%Editio Sancti Wolfgangi \annusEditionis
\end{center}

\pagebreak

\renewcommand{\headrulewidth}{0pt} % no horiz. rule at the header
\fancyhf{}
\pagestyle{fancy}

\pars{Oratio ante divinum Officium.}

\lettrine{{\color{red}A}}{peri,} Dómine, os meum ad benedicéndum nomen sanctum tuum:
munda quoque cor meum ab ómnibus vanis, pervérsis, et aliénis
cogitatiónibus:
intelléctum illúmina, afféctum inflámma,
ut digne, atténte ac devóte hoc Offícium recitáre váleam,
et exaudíri mérear ante conspéctum Divínæ Maiestátis tuæ.
Per Christum, Dóminum nostrum.
\Rbardot{} Amen.

Dómine, in unióne illíus divínæ intentiónis,
qua ipse in terris laudes Deo persolvísti,
has tibi Horas \rubricatum{(vel \textnormal{hanc tibi Horam})} persólvo.

%\trOratioAnteOfficium

\vfill

\pars{Oratio post divinum Officium.}

\rubrica{
  Orationem sequentem devote post Officium recitantibus
  Leo Papa X. defectus, et culpas in eo persolvendo ex humana
  fragilitate contractas, indulsit, et dicitur flexis genibus.
}

\lettrine{{\color{red}S}}{acrosánctæ} et indivíduæ Trinitáti,
crucifíxi Dómini nostri Iesu Christi humanitáti,
beatíssimæ et gloriosíssimæ sempérque Vírginis Maríæ
fecúndæ integritáti, 
et ómnium Sanctórum universitáti
sit sempitérna laus, honor, virtus et glória
ab omni creatúra,
nobísque remíssio ómnium peccatórum,
per infiníta sǽcula sæculórum.
\Rbardot{} Amen.

\noindent \Vbardot{} Beáta víscera Maríæ Virginis, quæ portavérunt
ætérni Patris Fílium.\\
\Rbardot{} Et beáta úbera, quæ lactavérunt Christum Dominum.

\rubrica{Et dicitur secreto \textnormal{Pater noster.} et \textnormal{Ave María.}}

%\trOratioPostOfficium

\vfill

\hora{Ad I. Vesperas.} %%%%%%%%%%%%%%%%%%%%%%%%%%%%%%%%%%%%%%%%%%%%%%%%%%%%%
%\sideThumbs{I. Vesperæ}

\cantusSineNeumas

\vspace{0.5cm}
\grechangedim{interwordspacetext}{0.18 cm plus 0.15 cm minus 0.05 cm}{scalable}%
\cuminitiali{}{temporalia/deusinadiutorium-solemnis.gtex}
\grechangedim{interwordspacetext}{0.22 cm plus 0.15 cm minus 0.05 cm}{scalable}%

\vfill
\pagebreak

\pars{Psalmus 1.} \scriptura{Ps. 144, 13; \textbf{H100}}

\vspace{-4mm}

\antiphona{VII c\textsuperscript{2}}{temporalia/ant-regnumtuum.gtex}

\scriptura{Psalmus 144, 10-21.}

\initiumpsalmi{temporalia/ps144ii-initium-vii-c2-auto.gtex}

%\psalmusEtTranslatioT{temporalia/ps144ii-VII-comb.tex}{10cm}
\input{temporalia/ps144ii-VII.tex} \Abardot{}

\vspace{-1cm}

\vfill
\pagebreak

\pars{Psalmus 2.} \scriptura{Ps. 145, 2; \textbf{H100}}

\vspace{-4mm}

\antiphona{IV E}{temporalia/ant-laudabodeum.gtex}

\scriptura{Psalmus 145.}

\initiumpsalmi{temporalia/ps145-initium-iv-E-auto.gtex}

%\psalmusEtTranslatioT{temporalia/ps145-VII-comb.tex}{10cm}
\input{temporalia/ps145-VII.tex} \Abardot{}

\vfill
\pagebreak

\pars{Psalmus 3.} \scriptura{Ps. 146, 1; \textbf{H101}}

\vspace{-4mm}

\antiphona{VIII a}{temporalia/ant-deonostro.gtex}

\scriptura{Psalmus 146.}

\initiumpsalmi{temporalia/ps146-initium-viii-A-auto.gtex}

%\psalmusEtTranslatioT{temporalia/ps146-VII-comb.tex}{10cm}
\input{temporalia/ps146-VII.tex} \Abardot{}

\vfill
\pagebreak

\pars{Psalmus 4.} \scriptura{Ps. 147, 1}

\vspace{-4mm}

\antiphona{E}{temporalia/ant-laudajerusalem.gtex}

\scriptura{Psalmus 147.}

\initiumpsalmi{temporalia/ps147-initium-e-auto.gtex}

%\psalmusEtTranslatioT{temporalia/ps147-VII-comb.tex}{10cm}
\input{temporalia/ps147-VII.tex} \Abardot{}

\vfill
\pagebreak

\pars{Capitulum.} \scriptura{Rom. 11, 33}

\grechangedim{interwordspacetext}{0.12 cm plus 0.15 cm minus 0.05 cm}{scalable}%
\cuminitiali{}{temporalia/capitulum-OAltitudo.gtex}
\grechangedim{interwordspacetext}{0.22 cm plus 0.15 cm minus 0.05 cm}{scalable}

% preklad Jeruz. bible
%\trCapituliI

\vfill

\pars{Responsorium breve.} \scriptura{Ps. 146, 5}

\cuminitiali{VI}{temporalia/resp-magnusdominusnoster.gtex}

%\trResp

\vfill
\pagebreak

\pars{Hymnus} \scriptura{Ambrosius (\olddag{} 397)}

\cuminitiali{I}{temporalia/hym-OLuxBeata-aestivalis.gtex}
\vspace{-3mm}
%\input{hym-OLuxBeata-bohtext.tex}

\vfill
%\pagebreak

\pars{Versus.}

% Versus. %%%
\sineinitiali{temporalia/versus-vespertina.gtex}

%\noindent \trVersus

\vfill
\pagebreak

\magnificati

\vfill
\pagebreak

%\sideThumbs{{\scriptsize{}Fine horarum}}

\anteOrationem

\pagebreak

% Oratio. %%%
\oratioLaudes

\vspace{-1mm}
%\trOrationisI

\vfill

\rubrica{Hebdomadarius dicit iterum Dominus vobiscum, vel cantor dicit:}

\vspace{2mm}

\sineinitiali{temporalia/domineexaudi.gtex}

\rubrica{Postea cantatur a cantore:}

\vspace{2mm}

\cuminitiali{I}{temporalia/benedicamus-dominica-perannum.gtex}

\vspace{1mm}

\vfill
\pagebreak

\hora{Ad Matutinum.} %%%%%%%%%%%%%%%%%%%%%%%%%%%%%%%%%%%%%%%%%%%%%%%%%%%%%
%\sideThumbs{Matutinum}

\vspace{2mm}

\cuminitiali{}{temporalia/dominelabiamea.gtex}

\vspace{2mm}

\pars{Invitatorium.} \scriptura{Ps. 94, 1; Psalmus 94}

\vspace{-6mm}

\antiphona{E}{temporalia/inv-veniteexsultemus.gtex}

\vfill
\pagebreak

\pars{Hymnus.} \scriptura{Adamus Sancti Victoris (\olddag 1146)}

\vspace{-5mm}

\antiphona{VII}{temporalia/hym-SalveDies.gtex}

\scriptura{Non dicitur \textnormal{Amen} in fine.}
%{
%\vspace{-5mm}
%\setlength{\columnsep}{0pt} % prostor mezi sloupci
%\input{hym-SalveDies-bohtext.tex}
%\setlength{\columnsep}{30pt} % prostor mezi sloupci
%}

\vfill
\pagebreak

\subhora{In I. Nocturno}

\pars{Psalmus 1.} \scriptura{Ps. 1, 1}

\vspace{-4mm}

\antiphona{VIII G}{temporalia/ant-beatusvir.gtex}

%\vspace{-5mm}

\scriptura{Ps. 1}

%\vspace{-2mm}

\initiumpsalmi{temporalia/ps1-initium-viii-G-auto.gtex}

%\psalmusEtTranslatioT{temporalia/ps1-I-comb.tex}{10cm}
\input{temporalia/ps1-I.tex} \Abardot{}

\vfill
\pagebreak

\pars{Psalmus 2.} \scriptura{Ps. 2, 11; \textbf{H93}}

\vspace{-4mm}

\antiphona{VII a}{temporalia/ant-servitedomino.gtex}

\vspace{-3mm}

\scriptura{Ps. 2}

\vspace{-2mm}

\initiumpsalmi{temporalia/ps2-initium-vii-a-auto.gtex}

%\psalmusEtTranslatioT{temporalia/ps2-I-comb.tex}{10cm}
\input{temporalia/ps2-I.tex} \Abardot{}

\vfill
\pagebreak

\pars{Psalmus 3.} \scriptura{Ps. 3, 7}

\vspace{-4mm}

\antiphona{VI F}{temporalia/ant-exsurgedominesalvum.gtex}

%\vspace{-5mm}

\scriptura{Ps. 3}

\initiumpsalmi{temporalia/ps3-initium-vi-F-auto.gtex}

%\psalmusEtTranslatioT{temporalia/ps3-I-comb.tex}{10cm}
\input{temporalia/ps3-I.tex} \Abardot{}

\vfill
\pagebreak

\pars{Versus.} \scriptura{Ps. 118, 55}

% Versus. %%%
\sineinitiali{temporalia/versus-memorfui.gtex}

\vspace{5mm}

\sineinitiali{temporalia/oratiodominica-mat.gtex}

\vspace{5mm}

\pars{Absolutio.}

\cuminitiali{}{temporalia/absolutio-exaudi.gtex}

\vfill
\pagebreak

\cuminitiali{}{temporalia/benedictio-solemn-benedictione.gtex}

\vspace{7mm}

\lectioi

\noindent \Vbardot{} Tu autem, Dómine, miserére nobis.
\noindent \Rbardot{} Deo grátias.

\vfill
\pagebreak

\responsoriumi

\vfill
\pagebreak

\cuminitiali{}{temporalia/benedictio-solemn-unigenitus.gtex}

\vspace{7mm}

\lectioii

\noindent \Vbardot{} Tu autem, Dómine, miserére nobis.
\noindent \Rbardot{} Deo grátias.

\vfill
\pagebreak

\responsoriumii

\vfill
\pagebreak

\cuminitiali{}{temporalia/benedictio-solemn-spiritus.gtex}

\vspace{7mm}

\lectioiii

\noindent \Vbardot{} Tu autem, Dómine, miserére nobis.
\noindent \Rbardot{} Deo grátias.

\vfill
\pagebreak

\responsoriumiii

\vfill
\pagebreak

\subhora{In II. Nocturno}

\pars{Psalmus 4.} \scriptura{Ps. 8, 2}

\vspace{-4mm}

\antiphona{I g}{temporalia/ant-quamadmirabileest.gtex}

%\vspace{-5mm}

\scriptura{Ps. 8}

%A\vspace{-2mm}

\initiumpsalmi{temporalia/ps8-initium-i-g-auto.gtex}

%\psalmusEtTranslatioT{temporalia/ps8-I-comb.tex}{10cm}
\input{temporalia/ps8-I.tex} \Abardot{}

\vfill
\pagebreak

\pars{Psalmus 5.} \scriptura{Ps. 9, 5}

\vspace{-4mm}

\antiphona{VIII G}{temporalia/ant-sedistisuperthronum.gtex}

%\vspace{-5mm}

\scriptura{Ps. 9, 2-11}

\initiumpsalmi{temporalia/ps9ii_xi-initium-viii-G-auto.gtex}

%\psalmusEtTranslatioT{temporalia/ps9ii_xi-I-comb.tex}{10cm}
\input{temporalia/ps9ii_xi-I.tex} \Abardot{}

\vfill
\pagebreak

\pars{Psalmus 6.} \scriptura{Ps. 9, 20}

\vspace{-4mm}

\antiphona{I g\textsuperscript{3}}{temporalia/ant-exsurgedominenon.gtex}

%\vspace{-5mm}

\scriptura{Ps. 9, 12-21}

\initiumpsalmi{temporalia/ps9xii_xxi-initium-i-g3-auto.gtex}

%\psalmusEtTranslatioT{temporalia/ps9xii_xxi-I-comb.tex}{10cm}
\input{temporalia/ps9xii_xxi-I.tex} \Abardot{}

\vfill
\pagebreak

\pars{Versus.} \scriptura{Ps. 118, 62}

% Versus. %%%
\sineinitiali{temporalia/versus-medianocte.gtex}

\vspace{5mm}

\sineinitiali{temporalia/oratiodominica-mat.gtex}

\vspace{5mm}

\pars{Absolutio.}

\cuminitiali{}{temporalia/absolutio-ipsius.gtex}

\vfill
\pagebreak

\cuminitiali{}{temporalia/benedictio-solemn-deus.gtex}

\vspace{7mm}

\lectioiv

\noindent \Vbardot{} Tu autem, Dómine, miserére nobis.
\noindent \Rbardot{} Deo grátias.

\vfill
\pagebreak

\responsoriumiv

\vfill
\pagebreak

\cuminitiali{}{temporalia/benedictio-solemn-christus.gtex}

\vspace{7mm}

\lectiov

\noindent \Vbardot{} Tu autem, Dómine, miserére nobis.
\noindent \Rbardot{} Deo grátias.

\vfill
\pagebreak

\responsoriumv

\vfill
\pagebreak

\cuminitiali{}{temporalia/benedictio-solemn-ignem.gtex}

\vspace{7mm}

\lectiovi

\noindent \Vbardot{} Tu autem, Dómine, miserére nobis.
\noindent \Rbardot{} Deo grátias.

\vfill
\pagebreak

\responsoriumvi

\vfill
\pagebreak

\subhora{In III. Nocturno}

\pars{Psalmus 7.} \scriptura{Ps. 9, 22}

\vspace{-4mm}

\antiphona{II D}{temporalia/ant-utquiddomine.gtex}

\vspace{-4mm}

\scriptura{Ps. 9, 22-32}

%\vspace{-2mm}

\initiumpsalmi{temporalia/ps9xxii_xxxii-initium-ii-D-auto.gtex}

%\psalmusEtTranslatioT{temporalia/ps9xxii_xxxii-I-comb.tex}{10cm}
\input{temporalia/ps9xxii_xxxii-I.tex} \Abardot{}

\vfill
\pagebreak

\pars{Psalmus 8.}\scriptura{Ex. 15, 18}

\vspace{-4mm}

\antiphona{IV* e}{temporalia/ant-inaeternum.gtex}

%\vspace{-4mm}

\scriptura{Ps. 9, 33-39}

\initiumpsalmi{temporalia/ps9xxxiii_xxxix-initium-iv_-e-auto.gtex}

%\psalmusEtTranslatioT{temporalia/ps9xxxiii_xxxix-I-comb.tex}{10cm}
\input{temporalia/ps9xxxiii_xxxix-I.tex} \Abardot{}

\vfill
\pagebreak

\pars{Psalmus 9.} \scriptura{Ps. 10, 8}

\vspace{-4mm}

\antiphona{II* f}{temporalia/ant-justusdominus.gtex}

%\vspace{-4mm}

\scriptura{Ps. 10}

%\initiumpsalmi{temporalia/ps10-initium-iv-c-auto.gtex}
\initiumpsalmi{temporalia/ps10-initium-ii_-f.gtex}

%\psalmusEtTranslatioT{temporalia/ps10-I-comb.tex}{10cm}
\input{temporalia/ps10-I.tex} \Abardot{}

\vfill
\pagebreak

\pars{Versus.} \scriptura{Ps. 118, 148}

% Versus. %%%
\sineinitiali{temporalia/versus-praevenerunt.gtex}

\vspace{5mm}

\sineinitiali{temporalia/oratiodominica-mat.gtex}

\vspace{5mm}

\pars{Absolutio.}

\cuminitiali{}{temporalia/absolutio-avinculis.gtex}

\vfill
\pagebreak

\cuminitiali{}{temporalia/benedictio-solemn-evangelica.gtex}

\vspace{7mm}

\lectiovii

\noindent \Vbardot{} Tu autem, Dómine, miserére nobis.
\noindent \Rbardot{} Deo grátias.

\vfill
\pagebreak

\responsoriumvii

\vfill
\pagebreak

\cuminitiali{}{temporalia/benedictio-solemn-divinum.gtex}

\vspace{7mm}

\lectioviii

\noindent \Vbardot{} Tu autem, Dómine, miserére nobis.
\noindent \Rbardot{} Deo grátias.

\vfill
\pagebreak

\responsoriumviii

\vfill
\pagebreak

\cuminitiali{}{temporalia/benedictio-solemn-adsocietatem.gtex}

\vspace{7mm}

\lectioix

\noindent \Vbardot{} Tu autem, Dómine, miserére nobis.
\noindent \Rbardot{} Deo grátias.

\vfill
\pagebreak

% Te Deum

{
\pars{Hymnus Ambrosianus} \scriptura{Tonus Solemnis}

\vspace{-2mm}

\grechangedim{interwordspacetext}{0.26 cm plus 0.15 cm minus 0.05 cm}{scalable}%
\cuminitiali{III}{temporalia/tedeum-solemnis-gn.gtex}
\grechangedim{interwordspacetext}{0.22 cm plus 0.15 cm minus 0.05 cm}{scalable}%
}

\vfill
\pagebreak

\rubrica{Reliqua omittuntur, nisi Laudes separandæ sint.}

\pars{Oratio}

\noindent \Vbardot{} Dómine, exáudi oratiónem meam.

\noindent \Rbardot{} Et clamor meus ad te véniat.

Orémus:

\oratioLaudes

\vspace{7mm}

\pars{Conclusio}

\noindent \Vbardot{} Dómine, exáudi oratiónem meam.

\noindent \Rbardot{} Et clamor meus ad te véniat.

\noindent \Vbardot{} Benedicámus Dómino, allelúia, allelúia.

\noindent \Rbardot{} Deo grátias, allelúia, allelúia.

\noindent \Vbardot{} Fidélium ánimæ per misericórdiam Dei requiéscant in pace.

\noindent \Rbardot{} Amen.

\vfill
\pagebreak

\hora{Ad Laudes.} %%%%%%%%%%%%%%%%%%%%%%%%%%%%%%%%%%%%%%%%%%%%%%%%%%%%%
%\sideThumbs{Laudes}

\cantusSineNeumas

\vspace{0.5cm}
\grechangedim{interwordspacetext}{0.18 cm plus 0.15 cm minus 0.05 cm}{scalable}%
\cuminitiali{}{temporalia/deusinadiutorium-alter.gtex}
\grechangedim{interwordspacetext}{0.22 cm plus 0.15 cm minus 0.05 cm}{scalable}%

\vfill
%\pagebreak

\pars{Psalmus 1.}

\vspace{-4mm}

\antiphona{VI F}{temporalia/ant-alleluia1.gtex}

\scriptura{Psalmus 50.}

\initiumpsalmi{temporalia/ps50-initium-vi-F-auto.gtex}

%\psalmusEtTranslatioT{temporalia/ps50-I-comb.tex}{10cm}
\input{temporalia/ps50-I.tex}

\vfill
\pagebreak

\pars{Psalmus 2.}

\scriptura{Psalmus 117.}

\initiumpsalmi{temporalia/ps117-initium-vi-F-auto.gtex}

%\psalmusEtTranslatioT{temporalia/ps117-I-comb.tex}{10cm}
\input{temporalia/ps117-I.tex}

\vfill
\pagebreak

\pars{Psalmus 3.}

\scriptura{Psalmus 62.}

\initiumpsalmi{temporalia/ps62-initium-vi-F-auto.gtex}

%\psalmusEtTranslatioT{temporalia/ps62-I-comb.tex}{10cm}
\input{temporalia/ps62-I.tex}

\vfill

\vspace{-6mm}

\antiphona{}{temporalia/ant-alleluia1.gtex} % repeat the antiphon - new page

\vfill
\pagebreak

\pars{Psalmus 4.} \scriptura{Dan. 3, 22-26; \textbf{H422}}

\vspace{-4mm}

\antiphona{VIII G}{temporalia/ant-trespueri.gtex}

\scriptura{Canticum trium puerorum, Dan. 3, 57-88 et 56}

\initiumpsalmi{temporalia/dan3-initium-viii-G-auto.gtex}

%\psalmusEtTranslatioT{temporalia/dan3-comb.tex}{10cm}
\input{temporalia/dan3.tex}

\rubrica{Hic non dicitur Gloria Patri, neque Amen.}

\vfill

\vspace{-6mm}

\antiphona{}{temporalia/ant-trespueri.gtex} % repeat the antiphon - new page

\vfill
\pagebreak

\pars{Psalmus 5.}

\vspace{-4mm}

\antiphona{VIII G}{temporalia/ant-alleluia2.gtex}

\scriptura{Psalmus 148.}

\initiumpsalmi{temporalia/ps148-initium-viii-G-auto.gtex}

%\psalmusEtTranslatioT{temporalia/ps148-I-comb.tex}{10cm}
\input{temporalia/ps148-I.tex}

\rubrica{Hic non dicitur Gloria Patri.}

\vfill
\pagebreak

%
\scriptura{Psalmus 149.}

\initiumpsalmi{temporalia/ps149-initium-viii-G-auto.gtex}

%\psalmusEtTranslatioT{temporalia/ps149-I-comb.tex}{10cm}
\input{temporalia/ps149-I.tex}

\rubrica{Hic non dicitur Gloria Patri.}

\vfill
\pagebreak

%
\scriptura{Psalmus 150.}

\initiumpsalmi{temporalia/ps150-initium-viii-G-auto.gtex}

%\psalmusEtTranslatioT{temporalia/ps150-I-comb.tex}{10cm}
\input{temporalia/ps150-I.tex}

\vfill

\vspace{-6mm}

\antiphona{}{temporalia/ant-alleluia2.gtex} % repeat the antiphon - new page

\vfill
\pagebreak

\pars{Capitulum.} \scriptura{Ac. 7, 12}

\grechangedim{interwordspacetext}{0.12 cm plus 0.15 cm minus 0.05 cm}{scalable}%
\cuminitiali{}{temporalia/capitulum-Benedictio.gtex}
\grechangedim{interwordspacetext}{0.22 cm plus 0.15 cm minus 0.05 cm}{scalable}

% preklad Jeruz. bible
%\trCapituliI

\vfill

\pars{Responsorium breve.} \scriptura{Ps. 118, 36-37}

\cuminitiali{IV}{temporalia/resp-inclinacormeum.gtex}

%\trResp

\vfill
\pagebreak

\pars{Hymnus} \scriptura{Gregorius Magnus (\olddag{} 604)}

\cuminitiali{IV}{temporalia/hym-EcceJamNoctis.gtex}
\vspace{-3mm}
%\input{hym-EcceJamNocis-bohtext.tex}

\vfill
%\pagebreak

\pars{Versus.} \scriptura{Ps. 92, 1}

% Versus. %%%
\sineinitiali{temporalia/versus-dominusregnavit.gtex}

%\noindent \trVersus

\vfill
\pagebreak

\benedictus

\vspace{-1cm}

\vfill
\pagebreak

%\sideThumbs{{\scriptsize{}Fine horarum}}

\anteOrationem

\pagebreak

% Oratio. %%%
\oratioLaudes

\vspace{-1mm}
%\trOrationisI

\vfill

\rubrica{Hebdomadarius dicit iterum Dominus vobiscum, vel cantor dicit:}

\vspace{2mm}

\sineinitiali{temporalia/domineexaudi.gtex}

\rubrica{Postea cantatur a cantore:}

\vspace{2mm}

\cuminitiali{I}{temporalia/benedicamus-dominica-perannum.gtex}

\vspace{1mm}

\vfill
\pagebreak

\hora{Ad II. Vesperas.} %%%%%%%%%%%%%%%%%%%%%%%%%%%%%%%%%%%%%%%%%%%%%%%%%%%%%
%\sideThumbs{II. Vesperæ}

\cantusSineNeumas

%\vspace{0.5cm}
\grechangedim{interwordspacetext}{0.18 cm plus 0.15 cm minus 0.05 cm}{scalable}%
\cuminitiali{}{temporalia/deusinadiutorium-solemnis.gtex}
\grechangedim{interwordspacetext}{0.22 cm plus 0.15 cm minus 0.05 cm}{scalable}%

\vfill
%\pagebreak

\vspace{-2mm}

\pars{Psalmus 1.} \scriptura{Ps. 109, 1; \textbf{H91}}

\vspace{-4mm}

\antiphona{VII c\textsuperscript{2}}{temporalia/ant-dixitdominus.gtex}

\vspace{-4mm}

\scriptura{Psalmus 109.}

\initiumpsalmi{temporalia/ps109-initium-vii-c2-auto.gtex}

%\psalmusEtTranslatioT{temporalia/ps109-I-comb.tex}{10cm}
\input{temporalia/ps109-I.tex} \Abardot{}

\vspace{-1cm}

\vfill
\pagebreak

\pars{Psalmus 2.} \scriptura{Ps. 110, 8; \textbf{H91}}

\vspace{-4mm}

\antiphona{IV g}{temporalia/ant-fideliaomnia.gtex}

\scriptura{Psalmus 110.}

\initiumpsalmi{temporalia/ps110-initium-iv-g-auto.gtex}

%\psalmusEtTranslatioT{temporalia/ps110-I-comb.tex}{10cm}
\input{temporalia/ps110-I.tex} \Abardot{}

\vfill
\pagebreak

\pars{Psalmus 3.} \scriptura{Ps. 111, 1; \textbf{H92}}

\vspace{-4mm}

\antiphona{IV a}{temporalia/ant-inmandatis.gtex}

\scriptura{Psalmus 111.}

\initiumpsalmi{temporalia/ps111-initium-iv-a-auto.gtex}

%\psalmusEtTranslatioT{temporalia/ps111-I-comb.tex}{10cm}
\input{temporalia/ps111-I.tex} \Abardot{}

\vfill
\pagebreak

\pars{Psalmus 4.} \scriptura{Ps. 112, 2; \textbf{H92}}

\vspace{-4mm}

\antiphona{VII c}{temporalia/ant-sitnomendomini.gtex}

\scriptura{Psalmus 112.}

\initiumpsalmi{temporalia/ps112-initium-vii-c-auto.gtex}

%\psalmusEtTranslatioT{temporalia/ps112-I-comb.tex}{10cm}
\input{temporalia/ps112-I.tex} \Abardot{}

\vfill
\pagebreak

\pars{Capitulum.} \scriptura{2 Cor. 1, 3-4}

\grechangedim{interwordspacetext}{0.12 cm plus 0.15 cm minus 0.05 cm}{scalable}%
\cuminitiali{}{temporalia/capitulum-BenedictusDeus.gtex}
\grechangedim{interwordspacetext}{0.22 cm plus 0.15 cm minus 0.05 cm}{scalable}

% preklad Jeruz. bible
%\trCapituliI

\vfill

\pars{Responsorium breve.} \scriptura{Ps. 103, 24}

\cuminitiali{VI}{temporalia/resp-quammagnificata.gtex}

%\trResp

\vfill
\pagebreak

\pars{Hymnus} \scriptura{Gregorius Magnus (\olddag{} 604)}

\cuminitiali{I}{temporalia/hym-LucisCreator-aestivalis.gtex}
\vspace{-3mm}
%\begin{translatioMulticol}{3}
Tvůrce světa předobrý,\\
tys ustanovil denní řád\\
a proudy světla rozhodil,\\
když světu základy jsi klad.\\
\\
A spojils ráno s večerem\\
a dnem tu dobu nazýváš;\\
hle padá temné noci stín -\\
slyš prosbu, vyslyš nářek náš.\columnbreak

Ach, nedej, by nás stihla smrt,\\
když svědomí nám tíží hřích,\\
když nemyslíme na věčnost\\
v té síti hříchů šalebných.\\
\\
Vzbuď naši touhu po nebi,\\
kde věčný život čeká nás,\\
a pomoz odložit vše zlé\\
a smýti z duše každý kaz.\columnbreak

To splň nám, dobrý Otče náš,\\
i ty, jenž rovné božství máš,\\
i Duchu, který těšíš nás\\
a vládneš, Bože, v každý čas.\\
Amen. 
\end{translatioMulticol}


\vfill
%\pagebreak

\pars{Versus.} \scriptura{Ps. 140, 2}

% Versus. %%%
\sineinitiali{temporalia/versus-dirigatur.gtex}

%\noindent \trVersus

\vfill
\pagebreak

\magnificatii

\vfill
\pagebreak

%\sideThumbs{{\scriptsize{}Fine horarum}}

\anteOrationem

\pagebreak

% Oratio. %%%
\oratioLaudes

\vspace{-1mm}
%\trOrationisI

\vfill

\rubrica{Hebdomadarius dicit iterum Dominus vobiscum, vel cantor dicit:}

\vspace{2mm}

\sineinitiali{temporalia/domineexaudi.gtex}

\rubrica{Postea cantatur a cantore:}

\vspace{2mm}

\cuminitiali{I}{temporalia/benedicamus-dominica-perannum.gtex}

\vspace{1mm}

\end{document}

