\newcommand{\titulus}{\nomenFesti{Dominica III in Quadragesima.}
\celebratio{}}
\newcommand{\tempquad}{Tempore Quadragesimae}
\newcommand{\oratio}{\pars{Oratio.}

\noindent Deus, ómnium misericordiárum et totíus bonitátis auctor, qui peccatórum remédia in ieiúniis, oratiónibus et eleemósynis demonstrásti, hanc humilitátis nostræ confessiónem propítius intuére, ut, qui inclinámur consciéntia nostra, tua semper misericórdia sublevémur.

\pars{Pro pace in Ucraina.} \scriptura{Sir. 50, 25; 2 Esdr. 4, 20; \textbf{H416}}

\vspace{-4mm}

\antiphona{II D}{temporalia/ant-dapacemdomine.gtex}

\vfill

\noindent Deus, a quo sancta desidéria, recta consília et iusta sunt ópera: da servis tuis illam, quam mundus dare non potest, pacem; ut et corda nostra mandátis tuis dédita, et hóstium subláta formídine, témpora sint tua protectióne tranquílla.

\noindent Per Dóminum nostrum Iesum Christum, Fílium tuum, qui tecum vivit et regnat in unitáte Spíritus Sancti, Deus, per ómnia sǽcula sæculórum.

\noindent \Rbardot{} Amen.}
\newcommand{\capitulum}{\pars{Capitulum.} \scriptura{Eph. 5, 1-2}

\grechangedim{interwordspacetext}{0.12 cm plus 0.15 cm minus 0.05 cm}{scalable}%
\cuminitiali{}{temporalia/capitulum-FratresEstote.gtex}
\grechangedim{interwordspacetext}{0.22 cm plus 0.15 cm minus 0.05 cm}{scalable}}
\newcommand{\magnificati}{\pars{Canticum B. Mariæ V.} \scriptura{Lc. 15, 22; \textbf{H153}}

\vspace{-6mm}

{
\grechangedim{interwordspacetext}{0.18 cm plus 0.15 cm minus 0.05 cm}{scalable}%
\antiphona{VIII G}{temporalia/ant-dixitautempateradservos.gtex}
\grechangedim{interwordspacetext}{0.22 cm plus 0.15 cm minus 0.05 cm}{scalable}%
}

\vspace{-3mm}

\scriptura{Lc. 1, 46-55}

\vspace{-2mm}

\cantusSineNeumas
\initiumpsalmi{temporalia/magnificat-initium-viiisoll-g.gtex}

\vspace{-1.5mm}

\input{temporalia/magnificat-viiisoll-g.tex} \Abardot{}}
\newcommand{\invitatorium}{\pars{Invitatorium.} \scriptura{Ps. 94, 8; Psalmus 94; \textbf{H143}}

\vspace{-4mm}

\antiphona{E}{temporalia/inv-hodiesivocem.gtex}}
\newcommand{\hymnusmatutinum}{\pars{Hymnus.} \scriptura{Gregorius Magnus (\olddag{} 604)}

\vspace{-5mm}

\antiphona{I}{temporalia/hym-ExMore.gtex}}
\newcommand{\nocturnoi}{\pars{Psalmus 1.} \scriptura{Ps. 1, 2; \textbf{H372}}

\vspace{-4mm}

\antiphona{I d}{temporalia/ant-inlegedominifuit.gtex}

%\vspace{-2mm}

\scriptura{Ps. 1}

%\vspace{-2mm}

\initiumpsalmi{temporalia/ps1-initium-i-d-auto.gtex}

\input{temporalia/ps1-i-d.tex} \Abardot{}

\vfill
\pagebreak

\pars{Psalmus 2.} \scriptura{Ps. 2, 6; \textbf{H372}}

\vspace{-4mm}

\antiphona{III a}{temporalia/ant-praedicanspraeceptumdomini.gtex}

\vspace{-1mm}

\scriptura{Ps. 2}

\vspace{-2mm}

\initiumpsalmi{temporalia/ps2-initium-iii-a-auto.gtex}

\input{temporalia/ps2-iii-a.tex} \Abardot{}

\vfill
\pagebreak

\pars{Psalmus 3.} \scriptura{Ps. 3, 5; \textbf{H372}}

\vspace{-4mm}

\antiphona{VII c trans.}{temporalia/ant-vocemeaaddominum.gtex}

%\vspace{-2mm}

\scriptura{Ps. 3}

\initiumpsalmi{temporalia/ps3-initium-vii-c-trans.gtex}

\input{temporalia/ps3-vii-c.tex} \Abardot{}

\vfill
\pagebreak
}
\newcommand{\nocturnoii}{\vspace{-4mm}

\pars{Psalmus 4.} \scriptura{Ex. 15, 2; \textbf{H39}}

\vspace{-4mm}

\antiphona{VIII G}{temporalia/ant-eccedeusmeus.gtex}

%\vspace{-2mm}

\scriptura{Ps. 144, 1-9}

%\vspace{-2mm}

\initiumpsalmi{temporalia/ps144i-initium-viii-G-auto.gtex}

\input{temporalia/ps144i-viii-G.tex} \Abardot{}

\vfill
\pagebreak

\pars{Psalmus 5.} \scriptura{Ps. 144, 13; \textbf{H100}}

\vspace{-4mm}

\antiphona{VII c}{temporalia/ant-regnumtuumdomine.gtex}

%\vspace{-2mm}

\scriptura{Ps. 144, 10-13}

\initiumpsalmi{temporalia/ps144x_xiii-initium-vii-c-auto.gtex}

\input{temporalia/ps144x_xiii-vii-c.tex} \Abardot{}

\vfill
\pagebreak

\pars{Psalmus 6.} \scriptura{\textbf{H99}}

\vspace{-4mm}

\antiphona{VIII a}{temporalia/ant-inaeternumet.gtex}

%\vspace{-4mm}

\scriptura{Ps. 144, 14-21}

%\vspace{-2mm}

\initiumpsalmi{temporalia/ps144xiv_xxi-initium-viii-a-auto.gtex}

%\vspace{-1.5mm}

\input{temporalia/ps144xiv_xxi-viii-a.tex} \Abardot{}

\vfill
\pagebreak}
\newcommand{\nocturnoiii}{\pars{Cantica.} \scriptura{Ier. 14, 19.20}

\vspace{-4mm}

\antiphona{I d}{temporalia/ant-sustinuimuspacem.gtex}

%\vspace{-2mm}

\scriptura{Canticum Ieremiæ, Ier. 14, 17-21}

%\vspace{-2mm}

\initiumpsalmi{temporalia/jeremiae2-initium-i-d.gtex}

\input{temporalia/jeremiae2-i-d.tex} \hfill \rubrica{Hic non dicitur antiphona.}

\vfill
\pagebreak

\scriptura{Canticum Ezechiæ, Ez. 36, 24-28}

%\vspace{-2mm}

\initiumpsalmi{temporalia/ezechiae2-initium-i-d-auto.gtex}

\input{temporalia/ezechiae2-i-d.tex}

\vfill
\pagebreak

\scriptura{Canticum, Lam. 5, 1-7.15-17.19-21}

%\vspace{-2mm}

\initiumpsalmi{temporalia/lamentatio-initium-i-d-auto.gtex}

\input{temporalia/lamentatio-i-d.tex}

\vfill
\pagebreak

\antiphona{}{temporalia/ant-sustinuimuspacem.gtex}

\vfill
\pagebreak}
\newcommand{\matversusi}{\pars{Versus.}

\noindent \Vbardot{} Ipse liberávit me de láqueo venántium.

\noindent \Rbardot{} Et a verbo áspero.}
\newcommand{\matversusii}{\pars{Versus.}

\noindent \Vbardot{} Scápulis suis obumbrábit tibi.

\noindent \Rbardot{} Et sub pennis eius sperábis.}
\newcommand{\lectioi}{\pars{Lectio I.} \scriptura{Gn. 37, 2-20}

\noindent De libro Génesis.

\noindent Ioseph, cum sédecim esset annórum, pascébat gregem cum frátribus suis adhuc puer: et erat cum fíliis Balæ et Zelphæ uxórum patris sui: accusavítque fratres suos apud patrem crímine péssimo. Israël autem diligébat Ioseph super omnes fílios suos, eo quod in senectúte genuísset eum: fecítque ei túnicam polýmitam. Vidéntes autem fratres eius quod a patre plus cunctis fíliis amarétur, óderant eum, nec póterant ei quidquam pacífice loqui.

\noindent {\color{gray} Accidit quoque ut visum sómnium reférret frátribus suis: quæ causa maióris ódii seminárium fuit. Dixítque ad eos: Audíte sómnium meum quod vidi: Putábam nos ligáre manípulos in agro: et quasi consúrgere manípulum meum, et stare, vestrósque manípulos circumstántes adoráre manípulum meum. Respondérunt fratres eius: Numquid rex noster eris? aut subiciémur ditióni tuæ? Hæc ergo causa somniórum atque sermónum, invídiæ et ódii fómitem ministrávit. Aliud quoque vidit sómnium, quod narrans frátribus, ait: Vidi per sómnium, quasi solem, et lunam, et stellas úndecim adoráre me. Quod cum patri suo, et frátribus retulísset, increpávit eum pater suus, et dixit: Quid sibi vult hoc sómnium quod vidísti? num ego et mater tua, et fratres tui adorábimus te super terram? Invidébant ei ígitur fratres sui: pater vero rem tácitus considerábat. Cumque fratres illíus in pascéndis grégibus patris moraréntur in Sichem.}

\noindent Dixit ad eum Israël: Fratres tui pascunt oves in Síchimis: veni, mittam te ad eos. Quo respondénte, Præsto sum, ait ei: Vade, et vide si cuncta próspera sint erga fratres tuos, et pécora: et renúntia mihi quid agátur. Missus de valle Hebron, venit in Sichem: Invenítque eum vir errántem in agro, et interrogávit quid quǽreret. At ille respóndit: Fratres meos quæro: índica mihi ubi pascant greges. Dixítque ei vir: Recessérunt de loco isto: audívi autem eos dicéntes: Eámus in Dóthain. Perréxit ergo Ioseph post fratres suos, et invénit eos in Dóthain. Qui cum vidíssent eum procul, ántequam accéderet ad eos, cogitavérunt illum occídere: Et mútuo loquebántur: Ecce somniátor venit: Veníte, occidámus eum, et mittámus in cistérnam véterem: dicemúsque: Fera péssima devorávit eum: et tunc apparébit quid illi prosint sómnia sua.}
\newcommand{\responsoriumi}{\pars{Responsorium 1.} \scriptura{\Rbardot{} Gn. 37, 19-20 \Vbardot{} ibid., 4; \textbf{H153}}

\vspace{-5mm}

\responsorium{VIII}{temporalia/resp-videntesiosephalonge-CROCHU.gtex}{}}
\newcommand{\lectioii}{\pars{Lectio II.} \scriptura{Gn. 37, 21-28}

\noindent Audiens autem hoc Ruben, nitebátur liberáre eum de mánibus eórum, et dicébat: Non interficiátis ánimam eius, nec effundátis sánguinem: sed proícite eum in cistérnam hanc, quæ est in solitúdine, manúsque vestras serváte innóxias: hoc autem dicébat, volens erípere eum de mánibus eórum, et réddere patri suo. Conféstim ígitur ut pervénit ad fratres suos, nudavérunt eum túnica talári et polýmita: Miserúntque eum in cistérnam véterem, quæ non habébat aquam. Et sedéntes ut coméderent panem, vidérunt Ismaëlítas viatóres veníre de Gálaad, et camélos eórum portántes arómata, et resínam, et stacten in Ægýptum. Dixit ergo Iudas frátribus suis: Quid nobis prodest si occidérimus fratrem nostrum, et celavérimus sánguinem ipsíus? Mélius est ut venundétur Ismaëlítis, et manus nostræ non polluántur: frater enim et caro nostra est. Acquievérunt fratres sermónibus illíus. Et prætereúntibus Madianítis negotiatóribus, extrahéntes eum de cistérna, vendidérunt eum Ismaëlítis, vigínti argénteis: qui duxérunt eum in Ægýptum.}
\newcommand{\responsoriumii}{\pars{Responsorium 2.} \scriptura{\Rbardot{} Gn. 37, 26-27 \Vbardot{} ibid., 29-30; \textbf{H153}}

\vspace{-5mm}

\responsorium{VII}{temporalia/resp-dixitiudasfratribussuis-CROCHU.gtex}{}}
\newcommand{\lectioiii}{\pars{Lectio III.} \scriptura{Gn. 37, 29-36}

\noindent Reversúsque Ruben ad cistérnam, non invénit púerum: et scissis véstibus pergens ad fratres suos, ait: Puer non compáret, et ego quo ibo? Tulérunt autem túnicam eius, et in sánguine hædi, quem occíderant, tinxérunt: mitténtes qui ferrent ad patrem, et dícerent: Hanc invenímus: vide utrum túnica fílii tui sit, an non. Quam cum agnovísset pater, ait: Túnica fílii mei est: fera péssima comédit eum, béstia devorávit Ioseph. Scissísque véstibus, indútus est cilício, lugens fílium suum multo témpore. Congregátis autem cunctis líberis eius ut lenírent dolórem patris, nóluit consolatiónem accípere, sed ait: Descéndam ad fílium meum lugens in inférnum. Et illo perseveránte in fletu, Madianítæ vendidérunt Ioseph in Ægýpto Putíphari eunúcho Pharaónis, magístro mílitum.}
\newcommand{\responsoriumiii}{\pars{Responsorium 3.} \scriptura{\Rbardot{} Gn. 37, 33 \Vbardot{} ibid., 32; \textbf{H154}}

\vspace{-5mm}

\responsorium{VI}{temporalia/resp-vidensiacobvestimenta-CROCHU-cumdox.gtex}{}}
\newcommand{\lectioiv}{\pars{Lectio IV.} \scriptura{Tract. 15, 10-12. 16-17: CCL 36, 154-156}

\noindent Ex Tractátibus sancti Augustíni epíscopi in Ioánnem.

\noindent \emph{Et venit múlier.} Forma Ecclésiæ, non iam iustificátæ, sed iam iustificándæ; nam hoc agit sermo. Venit ignára, invénit eum, et ágitur cum illa. Videámus quid, videámus quare \emph{venit múlier de Samaría hauríre aquam.} Samaritáni ad Iudæórum gentem non pertinébant; alienígenæ enim fuérunt. Pértinet ad imáginem rei, quod ab alienígenis venit ista múlier, quæ typum gerébat Ecclésiæ; ventúra enim erat Ecclésia de géntibus, alienígena a génere Iudæórum.

\noindent Audiámus ergo in illa nos, et in illa agnoscámus nos, et in illa grátias Deo agámus pro nobis. Illa enim figúra erat, non véritas; quia et ipsa præmísit figúram, et facta est véritas. Nam crédidit in eum, qui de illa figúram nobis prætendébat. \emph{}Venit ergo \emph{}hauríre aquam. Simplíciter vénerat hauríre aquam, sicut solent vel viri vel féminæ.}
\newcommand{\responsoriumiv}{\pars{Responsorium 4.} \scriptura{\Rbardot{} Ps. 80, 6 \Vbardot{} ibid., 7; \textbf{H154}}

\vspace{-5mm}

\responsorium{V}{temporalia/resp-iosephdumintraret-CROCHU.gtex}{}}
\newcommand{\lectiov}{\pars{Lectio V.}

\noindent \emph{Dicit ei Iesus: Da mihi bíbere. Discípuli enim eius abíerant in civitátem, ut cibos émerent. Dicit ergo ei múlier illa Samaritána: Quómodo tu, Iudǽus cum sis, bíbere a me poscis, quæ sum múlier Samaritána? Non enim coutúntur Iudǽi Samaritánis.}

\noindent Vidétis alienígenas: omníno vásculis eórum Iudǽi non utebántur. Et quia ferébat secum múlier vásculum unde aquam hauríret, eo miráta est, quia Iudǽus petébat ab ea bíbere, quod non solébant fácere Iudǽi. Ille autem, qui bíbere quærébat, fidem ipsíus mulíeris sitiébat.

\noindent Dénique audi quis petat bíbere. \emph{Respóndit Iesus et dixit ei: Si scires donum Dei et quis est qui dicit tibi “Da mihi bíbere”, tu fórsitan petísses ab eo et dedísset tibi aquam vivam.}}
\newcommand{\responsoriumv}{\pars{Responsorium 5.} \scriptura{\Rbardot{} Gn. 40, 14-15 \Vbardot{} ibid., 12-13; \textbf{H154}}

\vspace{-5mm}

\responsorium{VII}{temporalia/resp-mementomeidumbene-CROCHU.gtex}{}}
\newcommand{\lectiovi}{\pars{Lectio VI.}

\noindent Petit bíbere, et promíttit bíbere. Eget quasi acceptúrus, et áffluit tamquam satiatúrus. \emph{Si scires,} inquit, \emph{donum Dei.} Donum Dei est Spíritus Sanctus. Sed adhuc mulíeri tecte lóquitur, et paulátim intrat in cor. Fortássis iam docet. Quid enim ista hortatióne suávius et benígnius? \emph{Si scires donum Dei et scires quis est qui dicit tibi “Da mihi bíbere”, tu fórsitan péteres et daret tibi aquam vivam.}

\noindent De qua ergo aqua datúrus est, nisi de illa de qua dictum est: \emph{Apud te est fons vitæ?} Nam quómodo sítient \emph{qui inebriabúntur ab ubertáte domus tuæ?}

\noindent Promittébat ergo sagínam quamdam et satietátem Spíritus Sancti, et illa nondum intellegébat; et non intéllegens, quid respondébat? \emph{Dicit ad eum múlier: Dómine, da mihi hanc aquam, ut non sítiam, neque véniam huc hauríre.} Ad labórem indigéntia cogébat, et labórem infírmitas recusábat. Utinam audíret: \emph{Veníte ad me, omnes qui laborátis et oneráti estis, et ego vos refíciam!} Hoc enim ei dicébat Iesus, ut iam non laboráret; sed illa nondum intellegébat.}

\newcommand{\responsoriumvi}{\pars{Responsorium 6.} \scriptura{\Rbardot{} Gn. 43, 11.14 \Vbardot{} ibid., 11; \textbf{H154}}

\vspace{-5mm}

\responsorium{VII}{temporalia/resp-tollitehincvobiscum-CROCHU-cumdox.gtex}{}}
\newcommand{\evangelium}{\pars{Versus.}

\noindent \Vbardot{} Scuto circúmdabit te véritas eius.

\noindent \Rbardot{} Non timébis a timóre noctúrno.

\vspace{5mm}

\sineinitiali{temporalia/oratiodominica-mat.gtex}

\vspace{5mm}

\pars{Absolutio.}

\cuminitiali{}{temporalia/absolutio-avinculis.gtex}

\vfill
\pagebreak

\cuminitiali{}{temporalia/benedictio-solemn-evangelica.gtex}

\vspace{7mm}

\pars{Evangelium} \scriptura{Lc 4, 5-42}

\noindent Léctio sancti Evangélii secúndum Ioánnem.

\noindent In illo témpore:

\noindent Venit Iesus in civitátem Samaríæ, quæ dícitur Sichar, iuxta prǽdium, quod dedit Iacob Ioseph fílio suo; erat autem ibi fons Iacob. Iesus ergo fatigátus ex itínere sedébat sic super fontem; hora erat quasi sexta.

\noindent Venit múlier de Samaría hauríre aquam. Dicit ei Iesus: «Da mihi bíbere»; discípuli enim eius abíerant in civitátem, ut cibos émerent.

\noindent Dicit ergo ei múlier illa Samaritána: «Quómodo tu, Iudǽus cum sis, bíbere a me poscis, quæ sum múlier Samaritána?». Non enim coutúntur Iudǽi Samaritánis.

\noindent Respóndit Iesus et dixit ei: «Si scires donum Dei, et quis est, qui dicit tibi: “Da mihi bíbere”, tu fórsitan petísses ab eo, et dedísset tibi aquam vivam».

\noindent Dicit ei múlier: «Dómine, neque in quo háurias habes, et púteus altus est; unde ergo habes aquam vivam? Numquid tu maior es patre nostro Iacob, qui dedit nobis púteum, et ipse ex eo bibit et fílii eius et pécora eius?».

\noindent Respóndit Iesus et dixit ei: «Omnis, qui bibit ex aqua hac, sítiet íterum; qui autem bíberit ex aqua, quam ego dabo ei, non sítiet in ætérnum; sed aqua, quam dabo ei, fiet in eo fons aquæ saliéntis in vitam ætérnam».

\noindent Dicit ad eum múlier: «Dómine, da mihi hanc aquam, ut non sítiam neque véniam huc hauríre».

\noindent {\color{gray} Dicit ei: «Vade, voca virum tuum et veni huc».

\noindent Respóndit múlier et dixit ei: «Non hábeo virum».

\noindent Dicit ei Iesus: «Bene dixísti: “Non hábeo virum”; quinque enim viros habuísti, et nunc, quem habes, non est tuus vir. Hoc vere dixísti».

\noindent Dicit ei múlier:} «Dómine, vídeo quia prophéta es tu. Patres nostri in monte hoc adoravérunt, et vos dícitis quia in Hierosólymis est locus, ubi adoráre opórtet».

\noindent Dicit ei Iesus: «Crede mihi, múlier, quia venit hora, quando neque in monte hoc neque in Hierosólymis adorábitis Patrem. Vos adorátis, quod nescítis; nos adorámus, quod scimus, quia salus ex Iudǽis est. Sed venit hora, et nunc est, quando veri adoratóres adorábunt Patrem in Spíritu et veritáte; nam et Pater tales quærit, qui adórent eum. Spíritus est Deus, et eos, qui adórant eum, in Spíritu et veritáte opórtet adoráre».

\noindent Dicit ei múlier: «Scio quia Messías venit —qui dícitur Christus—; cum vénerit ille, nobis annuntiábit ómnia».

\noindent Dicit ei Iesus: «Ego sum, qui loquor tecum».

\noindent {\color{gray} Et contínuo venérunt discípuli eius et mirabántur quia cum mulíere loquebátur; nemo tamen dixit: «Quid quæris aut quid lóqueris cum ea?». Relíquit ergo hýdriam suam múlier et ábiit in civitátem et dicit illis homínibus: «Veníte, vidéte hóminem, qui dixit mihi ómnia, quæcúmque feci; numquid ipse est Christus?». Exiérunt de civitáte et veniébant ad eum.

\noindent Intérea rogábant eum discípuli dicéntes: «Rabbi, mandúca».

\noindent Ille autem dixit eis: «Ego cibum hábeo manducáre, quem vos nescítis».

\noindent Dicébant ergo discípuli ad ínvicem: «Numquid áliquis áttulit ei manducáre?».

\noindent Dicit eis Iesus: «Meus cibus est, ut fáciam voluntátem eius, qui misit me, et ut perfíciam opus eius. Nonne vos dícitis: “Adhuc quáttuor menses sunt, et messis venit”? Ecce dico vobis: Leváte óculos vestros et vidéte regiónes, quia albæ sunt ad messem! Iam qui metit, mercédem áccipit et cóngregat fructum in vitam ætérnam, ut et qui séminat, simul gáudeat et qui metit. In hoc enim est verbum verum: Alius est qui séminat, et álius est qui metit. Ego misi vos métere, quod vos non laborástis; álii laboravérunt, et vos in labórem eórum introístis».}

\noindent Ex civitáte autem illa multi credidérunt in eum Samaritanórum propter verbum mulíeris testimónium perhibéntis: «Dixit mihi ómnia, quæcúmque feci!». Cum veníssent ergo ad illum Samaritáni, rogavérunt eum, ut apud ipsos manéret; et mansit ibi duos dies. Et multo plures credidérunt propter sermónem eius; et mulíeri dicébant: «Iam non propter tuam loquélam crédimus; ipsi enim audívimus et scimus quia hic est vere Salvátor mundi!».

\vfill
\pagebreak

\pars{Responsorium 7.} \scriptura{\Rbardot{} Gn. 47, 25; \textbf{H156}}

\vspace{-5mm}

\responsorium{II}{temporalia/resp-salusnostrainmanutuaest-CROCHU-cumdox.gtex}{}

\vfill
\pagebreak}
\newcommand{\laudes}{\pars{Psalmus 1.} \scriptura{Ps. 92, 1}

\vspace{-4mm}

\antiphona{I g}{temporalia/ant-regnavitdominusdecorem.gtex}

%\vspace{-2mm}

\scriptura{Psalmus 92}

%\vspace{-2mm}

\initiumpsalmi{temporalia/ps92-initium-i-g-auto.gtex}

%\vspace{-1.5mm}

\input{temporalia/ps92-i-g.tex} \Abardot{}

\vfill
\pagebreak

\pars{Psalmus 2.} \scriptura{Cf. Sap. 16, 22.23; 19, 6; \textbf{H156}}

\vspace{-4mm}

\antiphona{I a\textsuperscript{2}}{temporalia/ant-vimvirtutissuae.gtex}

\scriptura{Canticum trium puerorum, Dan. 3, 57-88 et 56}

\initiumpsalmi{temporalia/dan3-initium-i-a4-auto.gtex}

\input{temporalia/dan3-i-a4-sinedox.tex}

\rubrica{Hic non dicitur Gloria Patri, neque Amen.}

\vfill

\vspace{-6mm}

\antiphona{}{temporalia/ant-vimvirtutissuae.gtex} % repeat the antiphon - new page

\vfill
\pagebreak

\pars{Psalmus 3.} \scriptura{Ps. 148, 11.12; \textbf{H162}}

\vspace{-4mm}

\antiphona{I a}{temporalia/ant-regesterrae.gtex}

\vspace{-2mm}

\scriptura{Psalmus 148.}

\vspace{-2mm}

\initiumpsalmi{temporalia/ps148-initium-i-a-auto.gtex}

\vspace{-1.5mm}

\input{temporalia/ps148-i-a.tex} \Abardot{}

\vfill
\pagebreak}
\newcommand{\lectiobrevis}{\pars{Lectio brevis.} \scriptura{Neh. 8, 9.10}

\noindent Dies iste sanctificátus est Dómino Deo nostro! Nolíte lugére et nolíte flere. Quia sanctus dies Dómini nostri est; et nolíte contristári, gáudium étenim Dómini est fortitúdo vestra.}
\newcommand{\responsoriumbreve}{\pars{Responsorium breve.}

\cuminitiali{IV}{temporalia/resp-christefilidei-tq.gtex}}
\newcommand{\hymnuslaudes}{\pars{Hymnus} \scriptura{Gregorius Magnus (\olddag{} 604)}

\cuminitiali{II}{temporalia/hym-PrecemurOmnes.gtex}}
\newcommand{\preces}{\noindent Redemptórem nostrum,~\gredagger{} qui hoc tempus salútis nobis benígne méruit, benedicámus,~\grestar{} eúmque súpplices exorémus:

\Rbardot{} Spíritum novum crea in nobis, Dómine.

\noindent Christe, vita nostra,~\gredagger{} qui per baptísmum nos mýstice tecum sepelíri donásti~\gredagger{} ac per eúndem tecum resuscitári voluísti,~\grestar{} tríbue nos hódie in novitáte vitæ ambuláre.

\Rbardot{} Spíritum novum crea in nobis, Dómine.

\noindent Dómine, qui ómnibus benefecísti,~\grestar{} fac nos étiam de commúni ómnium bono esse sollícitos.

\Rbardot{} Spíritum novum crea in nobis, Dómine.

\noindent Tríbue nobis ad terrénam civitátem ædificándam concórditer operári~\grestar{} et simul cæléstem inquírere.

\Rbardot{} Spíritum novum crea in nobis, Dómine.

\noindent Médice córporum et animárum, sana nostri vúlnera cordis,~\grestar{} ut contínua capiámus subsídia sanctitátis.

\Rbardot{} Spíritum novum crea in nobis, Dómine.}
\newcommand{\benedictus}{\pars{Canticum Zachariæ.} \scriptura{Cf. Io. 4, 13; \textbf{H158}}

\vspace{-4mm}

{
\grechangedim{interwordspacetext}{0.18 cm plus 0.15 cm minus 0.05 cm}{scalable}%
\antiphona{VIII c}{temporalia/ant-aquaquamego.gtex}
\grechangedim{interwordspacetext}{0.22 cm plus 0.15 cm minus 0.05 cm}{scalable}%
}

\vspace{-1mm}

\scriptura{Lc. 1, 68-79}

\vspace{-2mm}

\cantusSineNeumas
\initiumpsalmi{temporalia/benedictus-initium-viiisoll-c-auto.gtex}

%\vspace{-1.5mm}

\input{temporalia/benedictus-viiisoll-c.tex} \Abardot{}}
\newcommand{\magnificatii}{\pars{Canticum B. Mariæ V.} \scriptura{Lc. 11, 27-28; \textbf{H157}}

\vspace{-4mm}

{
\grechangedim{interwordspacetext}{0.18 cm plus 0.15 cm minus 0.05 cm}{scalable}%
\antiphona{VIII G}{temporalia/ant-extollensquaedam.gtex}
\grechangedim{interwordspacetext}{0.22 cm plus 0.15 cm minus 0.05 cm}{scalable}%
}

\vspace{-2mm}

\scriptura{Lc. 1, 46-55}

\vspace{-2mm}

\cantusSineNeumas
\initiumpsalmi{temporalia/magnificat-initium-viiisoll-g.gtex}

\vspace{-1.5mm}

\input{temporalia/magnificat-viiisoll-g.tex} \Abardot{}}
\newcommand{\hebdomada}{infra Hebdom. III Adventus.}
\newcommand{\oratioLaudes}{\cuminitiali{}{temporalia/oratio3vo.gtex}}
\newcommand{\responsoriumbreve}{\pars{Responsorium breve.} \scriptura{Is. 60, 2; \textbf{H20}}

\cuminitiali{IV}{temporalia/resp-superte.gtex}}

% LuaLaTeX

\documentclass[a4paper, twoside, 12pt]{article}
\usepackage[latin]{babel}
%\usepackage[landscape, left=3cm, right=1.5cm, top=2cm, bottom=1cm]{geometry} % okraje stranky
%\usepackage[landscape, a4paper, mag=1166, truedimen, left=2cm, right=1.5cm, top=1.6cm, bottom=0.95cm]{geometry} % okraje stranky
\usepackage[landscape, a4paper, mag=1400, truedimen, left=0.5cm, right=0.5cm, top=0.5cm, bottom=0.5cm]{geometry} % okraje stranky

\usepackage{fontspec}
\setmainfont[FeatureFile={junicode.fea}, Ligatures={Common, TeX}, RawFeature=+fixi]{Junicode}
%\setmainfont{Junicode}

% shortcut for Junicode without ligatures (for the Czech texts)
\newfontfamily\nlfont[FeatureFile={junicode.fea}, Ligatures={Common, TeX}, RawFeature=+fixi]{Junicode}

\usepackage{multicol}
\usepackage{color}
\usepackage{lettrine}
\usepackage{fancyhdr}

% usual packages loading:
\usepackage{luatextra}
\usepackage{graphicx} % support the \includegraphics command and options
\usepackage{gregoriotex} % for gregorio score inclusion
\usepackage{gregoriosyms}
\usepackage{wrapfig} % figures wrapped by the text
\usepackage{parcolumns}
\usepackage[contents={},opacity=1,scale=1,color=black]{background}
\usepackage{tikzpagenodes}
\usepackage{calc}
\usepackage{longtable}
\usetikzlibrary{calc}

\setlength{\headheight}{14.5pt}

\input{conventuscommune.tex} % Often used macros
%%%% Preklady jednotlivych zpevu (nektere se opakuji, a je dobre mit je
% vsechny na jedne hromade)

% HOURS ---

\newcommand{\trAntI}{\translatioCantus{Muž boží měl kožený toulec, pečlivě
zavázaný, jenž mu visel na šíji a~často se ho dotýkal.}}

\newcommand{\trAntII}{\translatioCantus{Klíč od~něho tak dobře střežil, že
dokud žil v~těle, nikdo z~jeho žáků nezvěděl, co je uvnitř.}}

\newcommand{\trAntIII}{\translatioCantus{Ale když se odebral z~tohoto
života, schránku otevřeli a~objevili v~ní žíněné roucho a~měděný řetěz
potřísněný krví.}}

\newcommand{\trAntIV}{\translatioCantus{A když prohlédli mistrovo tělo,
nalezli jeho tělo na čtyřech místech hluboce zbrázděno ranami od řetězu.}}

\newcommand{\trAntV}{\translatioCantus{Krev vytékající z~těch ran, místy
prostoupila i~žíněným rouchem.}}

\newcommand{\trCapituli}{\translatioCantus{
Miláčkovi Boha a~lidí,
Mojžíšovi požehnané paměti,~\gredagger{}
dopřál slávu rovnou slávě svatých~\grestar{}
učinil ho mocným na postrach nepřátelům
a~jeho slovy zastavil divy.}}

\newcommand{\trLectioBrevis}{\translatioCantus{
Pamatujte na své představené,
kteří vám hlásali Boží slovo.
Uvažte, jak oni skončili život, a~napodobujte jejich víru.
Ježíš Kristus je stejný včera i~dnes i~navěky.
Nenechte se svést věelijakými cizími naukami.}}

\newcommand{\trRespLaud}{\translatioCantus{Spravedlivého vodil Hospodin~\grestar{}
po přímých stezkách. \Vbardot{} A~ukázal mu Boží království.}}

\newcommand{\trRespLaudB}{\translatioCantus{Na tvých hradbách, Jeruzaléme,
ustanovil jsem strážné;~\grestar{}
budou bdít nad mým lidem. \Vbardot{} Ani ve dne, ani v~noci nesmějí nikdy
mlčet.}}

\newcommand{\trVersus}{\translatioCantus{\Vbardot{} Ústa spravedlivého šeptají moudrost, aleluja.
\Rbardot{} A~jeho jazyk ohlašuje právo, aleluja.}}

\newcommand{\trAntBenedictus}{\translatioCantus{Když na bujné oře vložili
nosítka a~sňali jim uzdu, vydali se přímo k~cele božího muže.}}

\newcommand{\trPreces}{\translatioCantus{
\noindent S vděčností chvalme Krista, dobrého Pastýře, \gredagger{} který dal život za své ovce, \grestar{} a~pokorně ho prosme: \Rbardot{} Pane, buď pastýřem svého lidu.

\noindent Kriste, ty dáváš církvi pastýře, a~jejich službou se ujímáš svého lidu, \grestar{} dej, ať v~lásce těch, kteří nás vedou, poznáváme, jak nás miluješ. \Rbardot{} Pane, buď pastýřem svého lidu.

\noindent Ty stále konáš skrze své zástupce službu pastýře a~učitele, \grestar{} nepřestávej nás nikdy vést prostřednictvím svých služebníků. \Rbardot{} Pane, buď pastýřem svého lidu.

\noindent Ty prokazuješ svému lidu skrze jeho pastýře službu lékaře duše i~těla, \grestar{} ochraňuj náš život a~veď nás ke svatosti. \Rbardot{} Pane, buď pastýřem svého lidu.

\noindent Ty posíláš své svaté, aby slovem i~příkladem vedli tvůj lid k~tobě, \grestar{} na jejich přímluvu nás posiluj, abychom vytrvali na cestě, která vede k~věčnému životu. \Rbardot{} Pane, buď pastýřem svého lidu.}}

\newcommand{\trOrationis}{\translatioCantus{Bože, jenž nám dopřáváš radovat
se z~výroční slavnosti svatého tvého vyznavače Havla, uděl dobrotivě,
abychom když slavíme jeho narození, též se řídili podobou jeho skutků.
Skrze…}}
 % Czech translations of the proper texts

\newcommand{\annusEditionis}{2020}

%%%% Vicekrat opakovane kousky

\newcommand{\anteOrationem}{
  \rubrica{Ante Orationem, cantatur a Superiore:}

  \pars{Supplicatio Litaniæ.}

  \cuminitiali{}{temporalia/supplicatiolitaniae.gtex}

  \pars{Oratio Dominica.}

  \cuminitiali{}{temporalia/oratiodominica.gtex}

  \rubrica{Deinde dicitur ab Hebdomadario:}

  \cuminitiali{}{temporalia/dominusvobiscum-solemnis.gtex}

  \rubrica{In choro monialium loco Dominus vobiscum dicitur:}

  \sineinitiali{temporalia/domineexaudi.gtex}
}

\setlength{\columnsep}{30pt} % prostor mezi sloupci

%%%%%%%%%%%%%%%%%%%%%%%%%%%%%%%%%%%%%%%%%%%%%%%%%%%%%%%%%%%%%%%%%%%%%%%%%%%%%%%%%%%%%%%%%%%%%%%%%%%%%%%%%%%%%
\begin{document}

% Here we set the space around the initial.
% Please report to http://home.gna.org/gregorio/gregoriotex/details for more details and options
\grechangedim{afterinitialshift}{2.2mm}{scalable}
\grechangedim{beforeinitialshift}{2.2mm}{scalable}
\grechangedim{interwordspacetext}{0.22 cm plus 0.15 cm minus 0.05 cm}{scalable}%
\grechangedim{annotationraise}{-0.2cm}{scalable}

% Here we set the initial font. Change 38 if you want a bigger initial.
% Emit the initials in red.
\grechangestyle{initial}{\color{red}\fontsize{38}{38}\selectfont}

\pagestyle{empty}

%%%% Titulni stranka
\begin{titulusOfficii}
\titulus{}
\end{titulusOfficii}

% graphic
%\vspace{1.5cm}
%\begin{center}
%\includegraphics[width=8cm]{emmaus.jpg}
%\end{center}

\vfill

\begin{center}
%Ad usum et secundum consuetudines chori \guillemotright{}Conventus Choralis\guillemotleft.

%Editio Sancti Wolfgangi \annusEditionis
\end{center}

\pagebreak

\renewcommand{\headrulewidth}{0pt} % no horiz. rule at the header
\fancyhf{}
\pagestyle{fancy}

\pars{Oratio ante divinum Officium.}

\lettrine{{\color{red}A}}{peri,} Dómine, os meum ad benedicéndum nomen sanctum tuum:
munda quoque cor meum ab ómnibus vanis, pervérsis, et aliénis
cogitatiónibus:
intelléctum illúmina, afféctum inflámma,
ut digne, atténte ac devóte hoc Offícium recitáre váleam,
et exaudíri mérear ante conspéctum Divínæ Maiestátis tuæ.
Per Christum, Dóminum nostrum.
\Rbardot{} Amen.

Dómine, in unióne illíus divínæ intentiónis,
qua ipse in terris laudes Deo persolvísti,
has tibi Horas \rubricatum{(vel \textnormal{hanc tibi Horam})} persólvo.

%\trOratioAnteOfficium

\vfill

\pars{Oratio post divinum Officium.}

\rubrica{
  Orationem sequentem devote post Officium recitantibus
  Leo Papa X. defectus, et culpas in eo persolvendo ex humana
  fragilitate contractas, indulsit, et dicitur flexis genibus.
}

\lettrine{{\color{red}S}}{acrosánctæ} et indivíduæ Trinitáti,
crucifíxi Dómini nostri Iesu Christi humanitáti,
beatíssimæ et gloriosíssimæ sempérque Vírginis Maríæ
fecúndæ integritáti, 
et ómnium Sanctórum universitáti
sit sempitérna laus, honor, virtus et glória
ab omni creatúra,
nobísque remíssio ómnium peccatórum,
per infiníta sǽcula sæculórum.
\Rbardot{} Amen.

\noindent \Vbardot{} Beáta víscera Maríæ Virginis, quæ portavérunt
ætérni Patris Fílium.\\
\Rbardot{} Et beáta úbera, quæ lactavérunt Christum Dominum.

\rubrica{Et dicitur secreto \textnormal{Pater noster.} et \textnormal{Ave María.}}

%\trOratioPostOfficium

\vfill

\hora{Ad I. Vesperas.} %%%%%%%%%%%%%%%%%%%%%%%%%%%%%%%%%%%%%%%%%%%%%%%%%%%%%
%\sideThumbs{I. Vesperæ}

\cantusSineNeumas

\vspace{0.5cm}
\grechangedim{interwordspacetext}{0.18 cm plus 0.15 cm minus 0.05 cm}{scalable}%
\cuminitiali{}{temporalia/deusinadiutorium-solemnis.gtex}
\grechangedim{interwordspacetext}{0.22 cm plus 0.15 cm minus 0.05 cm}{scalable}%

\vfill
\pagebreak

\pars{Psalmus 1.} \scriptura{Ps. 144, 13; \textbf{H100}}

\vspace{-4mm}

\antiphona{VII c\textsuperscript{2}}{temporalia/ant-regnumtuum.gtex}

\scriptura{Psalmus 144, 10-21.}

\initiumpsalmi{temporalia/ps144ii-initium-vii-c2-auto.gtex}

%\psalmusEtTranslatioT{temporalia/ps144ii-VII-comb.tex}{10cm}
\input{temporalia/ps144ii-VII.tex} \Abardot{}

\vspace{-1cm}

\vfill
\pagebreak

\pars{Psalmus 2.} \scriptura{Ps. 145, 2; \textbf{H100}}

\vspace{-4mm}

\antiphona{IV E}{temporalia/ant-laudabodeum.gtex}

\scriptura{Psalmus 145.}

\initiumpsalmi{temporalia/ps145-initium-iv-E-auto.gtex}

%\psalmusEtTranslatioT{temporalia/ps145-VII-comb.tex}{10cm}
\input{temporalia/ps145-VII.tex} \Abardot{}

\vfill
\pagebreak

\pars{Psalmus 3.} \scriptura{Ps. 146, 1; \textbf{H101}}

\vspace{-4mm}

\antiphona{VIII a}{temporalia/ant-deonostro.gtex}

\scriptura{Psalmus 146.}

\initiumpsalmi{temporalia/ps146-initium-viii-A-auto.gtex}

%\psalmusEtTranslatioT{temporalia/ps146-VII-comb.tex}{10cm}
\input{temporalia/ps146-VII.tex} \Abardot{}

\vfill
\pagebreak

\pars{Psalmus 4.} \scriptura{Ps. 147, 1}

\vspace{-4mm}

\antiphona{E}{temporalia/ant-laudajerusalem.gtex}

\scriptura{Psalmus 147.}

\initiumpsalmi{temporalia/ps147-initium-e-auto.gtex}

%\psalmusEtTranslatioT{temporalia/ps147-VII-comb.tex}{10cm}
\input{temporalia/ps147-VII.tex} \Abardot{}

\vfill
\pagebreak

\pars{Capitulum.} \scriptura{Rom. 11, 33}

\grechangedim{interwordspacetext}{0.12 cm plus 0.15 cm minus 0.05 cm}{scalable}%
\cuminitiali{}{temporalia/capitulum-OAltitudo.gtex}
\grechangedim{interwordspacetext}{0.22 cm plus 0.15 cm minus 0.05 cm}{scalable}

% preklad Jeruz. bible
%\trCapituliI

\vfill

\pars{Responsorium breve.} \scriptura{Ps. 146, 5}

\cuminitiali{VI}{temporalia/resp-magnusdominusnoster.gtex}

%\trResp

\vfill
\pagebreak

\pars{Hymnus} \scriptura{Ambrosius (\olddag{} 397)}

\cuminitiali{I}{temporalia/hym-OLuxBeata-aestivalis.gtex}
\vspace{-3mm}
%\input{hym-OLuxBeata-bohtext.tex}

\vfill
%\pagebreak

\pars{Versus.}

% Versus. %%%
\sineinitiali{temporalia/versus-vespertina.gtex}

%\noindent \trVersus

\vfill
\pagebreak

\magnificati

\vfill
\pagebreak

%\sideThumbs{{\scriptsize{}Fine horarum}}

\anteOrationem

\pagebreak

% Oratio. %%%
\oratioLaudes

\vspace{-1mm}
%\trOrationisI

\vfill

\rubrica{Hebdomadarius dicit iterum Dominus vobiscum, vel cantor dicit:}

\vspace{2mm}

\sineinitiali{temporalia/domineexaudi.gtex}

\rubrica{Postea cantatur a cantore:}

\vspace{2mm}

\cuminitiali{I}{temporalia/benedicamus-dominica-perannum.gtex}

\vspace{1mm}

\vfill
\pagebreak

\hora{Ad Matutinum.} %%%%%%%%%%%%%%%%%%%%%%%%%%%%%%%%%%%%%%%%%%%%%%%%%%%%%
%\sideThumbs{Matutinum}

\vspace{2mm}

\cuminitiali{}{temporalia/dominelabiamea.gtex}

\vspace{2mm}

\pars{Invitatorium.} \scriptura{Ps. 94, 1; Psalmus 94}

\vspace{-6mm}

\antiphona{E}{temporalia/inv-veniteexsultemus.gtex}

\vfill
\pagebreak

\pars{Hymnus.} \scriptura{Adamus Sancti Victoris (\olddag 1146)}

\vspace{-5mm}

\antiphona{VII}{temporalia/hym-SalveDies.gtex}

\scriptura{Non dicitur \textnormal{Amen} in fine.}
%{
%\vspace{-5mm}
%\setlength{\columnsep}{0pt} % prostor mezi sloupci
%\input{hym-SalveDies-bohtext.tex}
%\setlength{\columnsep}{30pt} % prostor mezi sloupci
%}

\vfill
\pagebreak

\subhora{In I. Nocturno}

\pars{Psalmus 1.} \scriptura{Ps. 1, 1}

\vspace{-4mm}

\antiphona{VIII G}{temporalia/ant-beatusvir.gtex}

%\vspace{-5mm}

\scriptura{Ps. 1}

%\vspace{-2mm}

\initiumpsalmi{temporalia/ps1-initium-viii-G-auto.gtex}

%\psalmusEtTranslatioT{temporalia/ps1-I-comb.tex}{10cm}
\input{temporalia/ps1-I.tex} \Abardot{}

\vfill
\pagebreak

\pars{Psalmus 2.} \scriptura{Ps. 2, 11; \textbf{H93}}

\vspace{-4mm}

\antiphona{VII a}{temporalia/ant-servitedomino.gtex}

\vspace{-3mm}

\scriptura{Ps. 2}

\vspace{-2mm}

\initiumpsalmi{temporalia/ps2-initium-vii-a-auto.gtex}

%\psalmusEtTranslatioT{temporalia/ps2-I-comb.tex}{10cm}
\input{temporalia/ps2-I.tex} \Abardot{}

\vfill
\pagebreak

\pars{Psalmus 3.} \scriptura{Ps. 3, 7}

\vspace{-4mm}

\antiphona{VI F}{temporalia/ant-exsurgedominesalvum.gtex}

%\vspace{-5mm}

\scriptura{Ps. 3}

\initiumpsalmi{temporalia/ps3-initium-vi-F-auto.gtex}

%\psalmusEtTranslatioT{temporalia/ps3-I-comb.tex}{10cm}
\input{temporalia/ps3-I.tex} \Abardot{}

\vfill
\pagebreak

\pars{Versus.} \scriptura{Ps. 118, 55}

% Versus. %%%
\sineinitiali{temporalia/versus-memorfui.gtex}

\vspace{5mm}

\sineinitiali{temporalia/oratiodominica-mat.gtex}

\vspace{5mm}

\pars{Absolutio.}

\cuminitiali{}{temporalia/absolutio-exaudi.gtex}

\vfill
\pagebreak

\cuminitiali{}{temporalia/benedictio-solemn-benedictione.gtex}

\vspace{7mm}

\lectioi

\noindent \Vbardot{} Tu autem, Dómine, miserére nobis.
\noindent \Rbardot{} Deo grátias.

\vfill
\pagebreak

\responsoriumi

\vfill
\pagebreak

\cuminitiali{}{temporalia/benedictio-solemn-unigenitus.gtex}

\vspace{7mm}

\lectioii

\noindent \Vbardot{} Tu autem, Dómine, miserére nobis.
\noindent \Rbardot{} Deo grátias.

\vfill
\pagebreak

\responsoriumii

\vfill
\pagebreak

\cuminitiali{}{temporalia/benedictio-solemn-spiritus.gtex}

\vspace{7mm}

\lectioiii

\noindent \Vbardot{} Tu autem, Dómine, miserére nobis.
\noindent \Rbardot{} Deo grátias.

\vfill
\pagebreak

\responsoriumiii

\vfill
\pagebreak

\subhora{In II. Nocturno}

\pars{Psalmus 4.} \scriptura{Ps. 8, 2}

\vspace{-4mm}

\antiphona{I g}{temporalia/ant-quamadmirabileest.gtex}

%\vspace{-5mm}

\scriptura{Ps. 8}

%A\vspace{-2mm}

\initiumpsalmi{temporalia/ps8-initium-i-g-auto.gtex}

%\psalmusEtTranslatioT{temporalia/ps8-I-comb.tex}{10cm}
\input{temporalia/ps8-I.tex} \Abardot{}

\vfill
\pagebreak

\pars{Psalmus 5.} \scriptura{Ps. 9, 5}

\vspace{-4mm}

\antiphona{VIII G}{temporalia/ant-sedistisuperthronum.gtex}

%\vspace{-5mm}

\scriptura{Ps. 9, 2-11}

\initiumpsalmi{temporalia/ps9ii_xi-initium-viii-G-auto.gtex}

%\psalmusEtTranslatioT{temporalia/ps9ii_xi-I-comb.tex}{10cm}
\input{temporalia/ps9ii_xi-I.tex} \Abardot{}

\vfill
\pagebreak

\pars{Psalmus 6.} \scriptura{Ps. 9, 20}

\vspace{-4mm}

\antiphona{I g\textsuperscript{3}}{temporalia/ant-exsurgedominenon.gtex}

%\vspace{-5mm}

\scriptura{Ps. 9, 12-21}

\initiumpsalmi{temporalia/ps9xii_xxi-initium-i-g3-auto.gtex}

%\psalmusEtTranslatioT{temporalia/ps9xii_xxi-I-comb.tex}{10cm}
\input{temporalia/ps9xii_xxi-I.tex} \Abardot{}

\vfill
\pagebreak

\pars{Versus.} \scriptura{Ps. 118, 62}

% Versus. %%%
\sineinitiali{temporalia/versus-medianocte.gtex}

\vspace{5mm}

\sineinitiali{temporalia/oratiodominica-mat.gtex}

\vspace{5mm}

\pars{Absolutio.}

\cuminitiali{}{temporalia/absolutio-ipsius.gtex}

\vfill
\pagebreak

\cuminitiali{}{temporalia/benedictio-solemn-deus.gtex}

\vspace{7mm}

\lectioiv

\noindent \Vbardot{} Tu autem, Dómine, miserére nobis.
\noindent \Rbardot{} Deo grátias.

\vfill
\pagebreak

\responsoriumiv

\vfill
\pagebreak

\cuminitiali{}{temporalia/benedictio-solemn-christus.gtex}

\vspace{7mm}

\lectiov

\noindent \Vbardot{} Tu autem, Dómine, miserére nobis.
\noindent \Rbardot{} Deo grátias.

\vfill
\pagebreak

\responsoriumv

\vfill
\pagebreak

\cuminitiali{}{temporalia/benedictio-solemn-ignem.gtex}

\vspace{7mm}

\lectiovi

\noindent \Vbardot{} Tu autem, Dómine, miserére nobis.
\noindent \Rbardot{} Deo grátias.

\vfill
\pagebreak

\responsoriumvi

\vfill
\pagebreak

\subhora{In III. Nocturno}

\pars{Psalmus 7.} \scriptura{Ps. 9, 22}

\vspace{-4mm}

\antiphona{II D}{temporalia/ant-utquiddomine.gtex}

\vspace{-4mm}

\scriptura{Ps. 9, 22-32}

%\vspace{-2mm}

\initiumpsalmi{temporalia/ps9xxii_xxxii-initium-ii-D-auto.gtex}

%\psalmusEtTranslatioT{temporalia/ps9xxii_xxxii-I-comb.tex}{10cm}
\input{temporalia/ps9xxii_xxxii-I.tex} \Abardot{}

\vfill
\pagebreak

\pars{Psalmus 8.}\scriptura{Ex. 15, 18}

\vspace{-4mm}

\antiphona{IV* e}{temporalia/ant-inaeternum.gtex}

%\vspace{-4mm}

\scriptura{Ps. 9, 33-39}

\initiumpsalmi{temporalia/ps9xxxiii_xxxix-initium-iv_-e-auto.gtex}

%\psalmusEtTranslatioT{temporalia/ps9xxxiii_xxxix-I-comb.tex}{10cm}
\input{temporalia/ps9xxxiii_xxxix-I.tex} \Abardot{}

\vfill
\pagebreak

\pars{Psalmus 9.} \scriptura{Ps. 10, 8}

\vspace{-4mm}

\antiphona{II* f}{temporalia/ant-justusdominus.gtex}

%\vspace{-4mm}

\scriptura{Ps. 10}

%\initiumpsalmi{temporalia/ps10-initium-iv-c-auto.gtex}
\initiumpsalmi{temporalia/ps10-initium-ii_-f.gtex}

%\psalmusEtTranslatioT{temporalia/ps10-I-comb.tex}{10cm}
\input{temporalia/ps10-I.tex} \Abardot{}

\vfill
\pagebreak

\pars{Versus.} \scriptura{Ps. 118, 148}

% Versus. %%%
\sineinitiali{temporalia/versus-praevenerunt.gtex}

\vspace{5mm}

\sineinitiali{temporalia/oratiodominica-mat.gtex}

\vspace{5mm}

\pars{Absolutio.}

\cuminitiali{}{temporalia/absolutio-avinculis.gtex}

\vfill
\pagebreak

\cuminitiali{}{temporalia/benedictio-solemn-evangelica.gtex}

\vspace{7mm}

\lectiovii

\noindent \Vbardot{} Tu autem, Dómine, miserére nobis.
\noindent \Rbardot{} Deo grátias.

\vfill
\pagebreak

\responsoriumvii

\vfill
\pagebreak

\cuminitiali{}{temporalia/benedictio-solemn-divinum.gtex}

\vspace{7mm}

\lectioviii

\noindent \Vbardot{} Tu autem, Dómine, miserére nobis.
\noindent \Rbardot{} Deo grátias.

\vfill
\pagebreak

\responsoriumviii

\vfill
\pagebreak

\cuminitiali{}{temporalia/benedictio-solemn-adsocietatem.gtex}

\vspace{7mm}

\lectioix

\noindent \Vbardot{} Tu autem, Dómine, miserére nobis.
\noindent \Rbardot{} Deo grátias.

\vfill
\pagebreak

% Te Deum

{
\pars{Hymnus Ambrosianus} \scriptura{Tonus Solemnis}

\vspace{-2mm}

\grechangedim{interwordspacetext}{0.26 cm plus 0.15 cm minus 0.05 cm}{scalable}%
\cuminitiali{III}{temporalia/tedeum-solemnis-gn.gtex}
\grechangedim{interwordspacetext}{0.22 cm plus 0.15 cm minus 0.05 cm}{scalable}%
}

\vfill
\pagebreak

\rubrica{Reliqua omittuntur, nisi Laudes separandæ sint.}

\pars{Oratio}

\noindent \Vbardot{} Dómine, exáudi oratiónem meam.

\noindent \Rbardot{} Et clamor meus ad te véniat.

Orémus:

\oratioLaudes

\vspace{7mm}

\pars{Conclusio}

\noindent \Vbardot{} Dómine, exáudi oratiónem meam.

\noindent \Rbardot{} Et clamor meus ad te véniat.

\noindent \Vbardot{} Benedicámus Dómino, allelúia, allelúia.

\noindent \Rbardot{} Deo grátias, allelúia, allelúia.

\noindent \Vbardot{} Fidélium ánimæ per misericórdiam Dei requiéscant in pace.

\noindent \Rbardot{} Amen.

\vfill
\pagebreak

\hora{Ad Laudes.} %%%%%%%%%%%%%%%%%%%%%%%%%%%%%%%%%%%%%%%%%%%%%%%%%%%%%
%\sideThumbs{Laudes}

\cantusSineNeumas

\vspace{0.5cm}
\grechangedim{interwordspacetext}{0.18 cm plus 0.15 cm minus 0.05 cm}{scalable}%
\cuminitiali{}{temporalia/deusinadiutorium-alter.gtex}
\grechangedim{interwordspacetext}{0.22 cm plus 0.15 cm minus 0.05 cm}{scalable}%

\vfill
%\pagebreak

\pars{Psalmus 1.}

\vspace{-4mm}

\antiphona{VI F}{temporalia/ant-alleluia1.gtex}

\scriptura{Psalmus 50.}

\initiumpsalmi{temporalia/ps50-initium-vi-F-auto.gtex}

%\psalmusEtTranslatioT{temporalia/ps50-I-comb.tex}{10cm}
\input{temporalia/ps50-I.tex}

\vfill
\pagebreak

\pars{Psalmus 2.}

\scriptura{Psalmus 117.}

\initiumpsalmi{temporalia/ps117-initium-vi-F-auto.gtex}

%\psalmusEtTranslatioT{temporalia/ps117-I-comb.tex}{10cm}
\input{temporalia/ps117-I.tex}

\vfill
\pagebreak

\pars{Psalmus 3.}

\scriptura{Psalmus 62.}

\initiumpsalmi{temporalia/ps62-initium-vi-F-auto.gtex}

%\psalmusEtTranslatioT{temporalia/ps62-I-comb.tex}{10cm}
\input{temporalia/ps62-I.tex}

\vfill

\vspace{-6mm}

\antiphona{}{temporalia/ant-alleluia1.gtex} % repeat the antiphon - new page

\vfill
\pagebreak

\pars{Psalmus 4.} \scriptura{Dan. 3, 22-26; \textbf{H422}}

\vspace{-4mm}

\antiphona{VIII G}{temporalia/ant-trespueri.gtex}

\scriptura{Canticum trium puerorum, Dan. 3, 57-88 et 56}

\initiumpsalmi{temporalia/dan3-initium-viii-G-auto.gtex}

%\psalmusEtTranslatioT{temporalia/dan3-comb.tex}{10cm}
\input{temporalia/dan3.tex}

\rubrica{Hic non dicitur Gloria Patri, neque Amen.}

\vfill

\vspace{-6mm}

\antiphona{}{temporalia/ant-trespueri.gtex} % repeat the antiphon - new page

\vfill
\pagebreak

\pars{Psalmus 5.}

\vspace{-4mm}

\antiphona{VIII G}{temporalia/ant-alleluia2.gtex}

\scriptura{Psalmus 148.}

\initiumpsalmi{temporalia/ps148-initium-viii-G-auto.gtex}

%\psalmusEtTranslatioT{temporalia/ps148-I-comb.tex}{10cm}
\input{temporalia/ps148-I.tex}

\rubrica{Hic non dicitur Gloria Patri.}

\vfill
\pagebreak

%
\scriptura{Psalmus 149.}

\initiumpsalmi{temporalia/ps149-initium-viii-G-auto.gtex}

%\psalmusEtTranslatioT{temporalia/ps149-I-comb.tex}{10cm}
\input{temporalia/ps149-I.tex}

\rubrica{Hic non dicitur Gloria Patri.}

\vfill
\pagebreak

%
\scriptura{Psalmus 150.}

\initiumpsalmi{temporalia/ps150-initium-viii-G-auto.gtex}

%\psalmusEtTranslatioT{temporalia/ps150-I-comb.tex}{10cm}
\input{temporalia/ps150-I.tex}

\vfill

\vspace{-6mm}

\antiphona{}{temporalia/ant-alleluia2.gtex} % repeat the antiphon - new page

\vfill
\pagebreak

\pars{Capitulum.} \scriptura{Ac. 7, 12}

\grechangedim{interwordspacetext}{0.12 cm plus 0.15 cm minus 0.05 cm}{scalable}%
\cuminitiali{}{temporalia/capitulum-Benedictio.gtex}
\grechangedim{interwordspacetext}{0.22 cm plus 0.15 cm minus 0.05 cm}{scalable}

% preklad Jeruz. bible
%\trCapituliI

\vfill

\pars{Responsorium breve.} \scriptura{Ps. 118, 36-37}

\cuminitiali{IV}{temporalia/resp-inclinacormeum.gtex}

%\trResp

\vfill
\pagebreak

\pars{Hymnus} \scriptura{Gregorius Magnus (\olddag{} 604)}

\cuminitiali{IV}{temporalia/hym-EcceJamNoctis.gtex}
\vspace{-3mm}
%\input{hym-EcceJamNocis-bohtext.tex}

\vfill
%\pagebreak

\pars{Versus.} \scriptura{Ps. 92, 1}

% Versus. %%%
\sineinitiali{temporalia/versus-dominusregnavit.gtex}

%\noindent \trVersus

\vfill
\pagebreak

\benedictus

\vspace{-1cm}

\vfill
\pagebreak

%\sideThumbs{{\scriptsize{}Fine horarum}}

\anteOrationem

\pagebreak

% Oratio. %%%
\oratioLaudes

\vspace{-1mm}
%\trOrationisI

\vfill

\rubrica{Hebdomadarius dicit iterum Dominus vobiscum, vel cantor dicit:}

\vspace{2mm}

\sineinitiali{temporalia/domineexaudi.gtex}

\rubrica{Postea cantatur a cantore:}

\vspace{2mm}

\cuminitiali{I}{temporalia/benedicamus-dominica-perannum.gtex}

\vspace{1mm}

\vfill
\pagebreak

\hora{Ad II. Vesperas.} %%%%%%%%%%%%%%%%%%%%%%%%%%%%%%%%%%%%%%%%%%%%%%%%%%%%%
%\sideThumbs{II. Vesperæ}

\cantusSineNeumas

%\vspace{0.5cm}
\grechangedim{interwordspacetext}{0.18 cm plus 0.15 cm minus 0.05 cm}{scalable}%
\cuminitiali{}{temporalia/deusinadiutorium-solemnis.gtex}
\grechangedim{interwordspacetext}{0.22 cm plus 0.15 cm minus 0.05 cm}{scalable}%

\vfill
%\pagebreak

\vspace{-2mm}

\pars{Psalmus 1.} \scriptura{Ps. 109, 1; \textbf{H91}}

\vspace{-4mm}

\antiphona{VII c\textsuperscript{2}}{temporalia/ant-dixitdominus.gtex}

\vspace{-4mm}

\scriptura{Psalmus 109.}

\initiumpsalmi{temporalia/ps109-initium-vii-c2-auto.gtex}

%\psalmusEtTranslatioT{temporalia/ps109-I-comb.tex}{10cm}
\input{temporalia/ps109-I.tex} \Abardot{}

\vspace{-1cm}

\vfill
\pagebreak

\pars{Psalmus 2.} \scriptura{Ps. 110, 8; \textbf{H91}}

\vspace{-4mm}

\antiphona{IV g}{temporalia/ant-fideliaomnia.gtex}

\scriptura{Psalmus 110.}

\initiumpsalmi{temporalia/ps110-initium-iv-g-auto.gtex}

%\psalmusEtTranslatioT{temporalia/ps110-I-comb.tex}{10cm}
\input{temporalia/ps110-I.tex} \Abardot{}

\vfill
\pagebreak

\pars{Psalmus 3.} \scriptura{Ps. 111, 1; \textbf{H92}}

\vspace{-4mm}

\antiphona{IV a}{temporalia/ant-inmandatis.gtex}

\scriptura{Psalmus 111.}

\initiumpsalmi{temporalia/ps111-initium-iv-a-auto.gtex}

%\psalmusEtTranslatioT{temporalia/ps111-I-comb.tex}{10cm}
\input{temporalia/ps111-I.tex} \Abardot{}

\vfill
\pagebreak

\pars{Psalmus 4.} \scriptura{Ps. 112, 2; \textbf{H92}}

\vspace{-4mm}

\antiphona{VII c}{temporalia/ant-sitnomendomini.gtex}

\scriptura{Psalmus 112.}

\initiumpsalmi{temporalia/ps112-initium-vii-c-auto.gtex}

%\psalmusEtTranslatioT{temporalia/ps112-I-comb.tex}{10cm}
\input{temporalia/ps112-I.tex} \Abardot{}

\vfill
\pagebreak

\pars{Capitulum.} \scriptura{2 Cor. 1, 3-4}

\grechangedim{interwordspacetext}{0.12 cm plus 0.15 cm minus 0.05 cm}{scalable}%
\cuminitiali{}{temporalia/capitulum-BenedictusDeus.gtex}
\grechangedim{interwordspacetext}{0.22 cm plus 0.15 cm minus 0.05 cm}{scalable}

% preklad Jeruz. bible
%\trCapituliI

\vfill

\pars{Responsorium breve.} \scriptura{Ps. 103, 24}

\cuminitiali{VI}{temporalia/resp-quammagnificata.gtex}

%\trResp

\vfill
\pagebreak

\pars{Hymnus} \scriptura{Gregorius Magnus (\olddag{} 604)}

\cuminitiali{I}{temporalia/hym-LucisCreator-aestivalis.gtex}
\vspace{-3mm}
%\begin{translatioMulticol}{3}
Tvůrce světa předobrý,\\
tys ustanovil denní řád\\
a proudy světla rozhodil,\\
když světu základy jsi klad.\\
\\
A spojils ráno s večerem\\
a dnem tu dobu nazýváš;\\
hle padá temné noci stín -\\
slyš prosbu, vyslyš nářek náš.\columnbreak

Ach, nedej, by nás stihla smrt,\\
když svědomí nám tíží hřích,\\
když nemyslíme na věčnost\\
v té síti hříchů šalebných.\\
\\
Vzbuď naši touhu po nebi,\\
kde věčný život čeká nás,\\
a pomoz odložit vše zlé\\
a smýti z duše každý kaz.\columnbreak

To splň nám, dobrý Otče náš,\\
i ty, jenž rovné božství máš,\\
i Duchu, který těšíš nás\\
a vládneš, Bože, v každý čas.\\
Amen. 
\end{translatioMulticol}


\vfill
%\pagebreak

\pars{Versus.} \scriptura{Ps. 140, 2}

% Versus. %%%
\sineinitiali{temporalia/versus-dirigatur.gtex}

%\noindent \trVersus

\vfill
\pagebreak

\magnificatii

\vfill
\pagebreak

%\sideThumbs{{\scriptsize{}Fine horarum}}

\anteOrationem

\pagebreak

% Oratio. %%%
\oratioLaudes

\vspace{-1mm}
%\trOrationisI

\vfill

\rubrica{Hebdomadarius dicit iterum Dominus vobiscum, vel cantor dicit:}

\vspace{2mm}

\sineinitiali{temporalia/domineexaudi.gtex}

\rubrica{Postea cantatur a cantore:}

\vspace{2mm}

\cuminitiali{I}{temporalia/benedicamus-dominica-perannum.gtex}

\vspace{1mm}

\end{document}

