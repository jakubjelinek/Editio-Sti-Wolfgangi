\newcommand{\titulus}{\nomenFesti{Dominica II in Quadragesima.}
\celebratio{}}
\newcommand{\tempquad}{Tempore Quadragesimae}
\newcommand{\oratio}{\pars{Oratio.}

\noindent Deus, qui nobis diléctum Fílium tuum audíre præcepísti, verbo tuo intérius nos páscere dignéris, ut, spiritáli purificáto intúitu, glóriæ tuæ lætémur aspéctu.

\pars{Pro pace in universo mundo.} \scriptura{Sir. 50, 25; 2 Esdr. 4, 20; \textbf{H416}}

\vspace{-4mm}

\antiphona{II D}{temporalia/ant-dapacemdomine.gtex}

\vfill

\noindent Deus, a quo sancta desidéria, recta consília et iusta sunt ópera: da servis tuis illam, quam mundus dare non potest, pacem; ut et corda nostra mandátis tuis dédita, et hóstium subláta formídine, témpora sint tua protectióne tranquílla.

\noindent Per Dóminum nostrum Iesum Christum, Fílium tuum, qui tecum vivit et regnat in unitáte Spíritus Sancti, Deus, per ómnia sǽcula sæculórum.

\noindent \Rbardot{} Amen.}
\newcommand{\capitulum}{\pars{Capitulum.} \scriptura{1 Thess. 4, 1}

\grechangedim{interwordspacetext}{0.12 cm plus 0.15 cm minus 0.05 cm}{scalable}%
\cuminitiali{}{temporalia/capitulum-FratresRogamus.gtex}
\grechangedim{interwordspacetext}{0.22 cm plus 0.15 cm minus 0.05 cm}{scalable}}
\newcommand{\magnificati}{\pars{Canticum B. Mariæ V.} \scriptura{Mt. 17, 9; \textbf{H149}}

\vspace{-4mm}

{
\grechangedim{interwordspacetext}{0.18 cm plus 0.15 cm minus 0.05 cm}{scalable}%
\antiphona{I f}{temporalia/ant-visionemquamvidistis.gtex}
\grechangedim{interwordspacetext}{0.22 cm plus 0.15 cm minus 0.05 cm}{scalable}%
}

\vspace{-2mm}

\scriptura{Lc. 1, 46-55}

\vspace{-2mm}

\cantusSineNeumas
\initiumpsalmi{temporalia/magnificat-initium-isoll-f.gtex}

%\vspace{-1.5mm}

\input{temporalia/magnificat-isoll-f.tex} \Abardot{}}
\newcommand{\invitatorium}{\pars{Invitatorium.}

\vspace{-4mm}

\antiphona{IV*}{temporalia/inv-christumdominum-cumdox.gtex}}
\newcommand{\hymnusmatutinum}{\pars{Hymnus.} \scriptura{Gregorius Magnus (\olddag{} 604)}

\vspace{-5mm}

\antiphona{I}{temporalia/hym-ExMore.gtex}}
\newcommand{\nocturnoii}{\vspace{-4mm}

\pars{Psalmus 4.}

\vspace{-4mm}

\antiphona{III a}{temporalia/ant-confessionemetdecorem.gtex}

%\vspace{-2mm}

\scriptura{Ps. 103, 1-12}

%\vspace{-2mm}

\initiumpsalmi{temporalia/ps103i-initium-iii-a-auto.gtex}

\input{temporalia/ps103i-iii-a.tex} \Abardot{}

\vfill
\pagebreak

\pars{Psalmus 5.}

\vspace{-4mm}

\antiphona{II D}{temporalia/ant-dominumdeumadoremus.gtex}

%\vspace{-2mm}

\scriptura{Ps. 103, 13-23}

\initiumpsalmi{temporalia/ps103ii-initium-ii-D-auto.gtex}

\input{temporalia/ps103ii-ii-D.tex} \Abardot{}

\vfill
\pagebreak

\pars{Psalmus 6.} \scriptura{Ps. 103, 24}

\vspace{-4mm}

\antiphona{E}{temporalia/ant-quammagnificatasunt.gtex}

\vspace{-4mm}

\scriptura{Ps. 103, 24-35}

%\vspace{-2mm}

\initiumpsalmi{temporalia/ps103iii-initium-e.gtex}

\vspace{-1.5mm}

\input{temporalia/ps103iii-e.tex} \Abardot{}

\vfill
\pagebreak}
\newcommand{\nocturnoiii}{\pars{Cantica.} \scriptura{Ier. 14, 19.20}

\vspace{-4mm}

\antiphona{I d}{temporalia/ant-sustinuimuspacem.gtex}

%\vspace{-2mm}

\scriptura{Canticum Ieremiæ, Ier. 14, 17-21}

%\vspace{-2mm}

\initiumpsalmi{temporalia/jeremiae2-initium-i-d.gtex}

\input{temporalia/jeremiae2-i-d.tex} \hfill \rubrica{Hic non dicitur antiphona.}

\vfill
\pagebreak

\scriptura{Canticum Ezechiæ, Ez. 36, 24-28}

%\vspace{-2mm}

\initiumpsalmi{temporalia/ezechiae2-initium-i-d-auto.gtex}

\input{temporalia/ezechiae2-i-d.tex}

\vfill
\pagebreak

\scriptura{Canticum, Lam. 5, 1-7.15-17.19-21}

%\vspace{-2mm}

\initiumpsalmi{temporalia/lamentatio-initium-i-d-auto.gtex}

\input{temporalia/lamentatio-i-d.tex}

\vfill
\pagebreak

\antiphona{}{temporalia/ant-sustinuimuspacem.gtex}

\vfill
\pagebreak}
\newcommand{\matversusi}{\pars{Versus.}

\noindent \Vbardot{} Ipse liberávit me de láqueo venántium.

\noindent \Rbardot{} Et a verbo áspero.}
\newcommand{\matversusii}{\pars{Versus.}

\noindent \Vbardot{} Vox Patris de nube audíta est.

\noindent \Rbardot{} Hic est Fílius meus diléctus: ipsum audíte.}
\newcommand{\lectioi}{\pars{Lectio I.} \scriptura{Gn. 27, 1-10}

\noindent De libro Génesis.

\noindent Sénuit autem Isaac, et calligavérunt óculi eius, et vidére non póterat. Vocavítque Esau fílium suum maiórem, et dixit ei: "Fili mi!"

\noindent Qui respóndit: "Adsum."

\noindent Cui pater: "Vides, inquit, quod senúerim et ignórem diem mortis meæ. Sume arma tua, pháretram, et arcum, et egrédere foras; cumque venátu áliquid apprehénderis, fac mihi inde pulméntum sicut velle me nosti, et affer ut cómedam: et benedícat tibi ánima mea ántequam móriar."

\noindent Quod cum audísset Rebécca, et ille abiísset in agrum ut iussiónem patris impléret, dixit fílio suo Iacob: "Audívi patrem tuum loquéntem cum Esau fratre tuo, et dicéntem ei: Affer mihi de venatióne tua, et fac cibos ut cómedam, et benedícam tibi coram Dómino ántequam móriar. Nunc ergo fili mi, acquiésce consíliis meis; et pergens ad gregem, affer mihi duos hædos óptimos, ut fáciam ex eis escas patri tuo, quibus libénter véscitur. Quas cum intúleris, et coméderit, benedícat tibi priúsquam moriátur."}
\newcommand{\responsoriumi}{\pars{Responsorium 1.} \scriptura{\Rbardot{} Gn. 27, 3-4 \Vbardot{} ibid.; \textbf{H149}}

\vspace{-5mm}

\responsorium{VII}{temporalia/resp-tollearmatua-CROCHU.gtex}{}}
\newcommand{\lectioii}{\pars{Lectio II.} \scriptura{Gn. 27, 11-20}

\noindent Cui ille respóndit: "Nosti quod Esau frater meus homo pilósus sit, et ego lenis. Si attractáverit me pater meus, et sénserit, tímeo ne putet sibi voluísse illúdere, et indúcam super me maledictiónem pro benedictióne."

\noindent Ad quem mater: "In me sit, ait, ista maledíctio, fili mi: tantum audi vocem meam, et pergens, affer quæ dixi."

\noindent Abiit, et áttulit, dedítque matri. Parávit illa cibos, sicut velle nóverat patrem illíus. Et véstibus Esau valde bonis, quas apud se habébat domi, índuit eum: pelliculásque hædórum circúmdedit mánibus, et colli nuda protéxit. Dedítque pulméntum, et panes, quos cóxerat, trádidit.

\noindent Quibus illátis, dixit: "Pater mi!" At ille respóndit: "Audio. Quis es tu, fili mi?" Dixítque Iacob: "Ego sum primogénitus tuus Esau; feci sicut præcepísti mihi. Surge, sede, et cómede de venatióne mea, ut benedícat mihi ánima tua."

\noindent Rursúmque Isaac ad fílium suum: "Quómodo, inquit, tam cito inveníre potuísti, fili mi?"

\noindent Qui respóndit: "Volúntas Dei fuit ut cito occúrreret mihi quod volébam."}
\newcommand{\responsoriumii}{\pars{Responsorium 2.} \scriptura{\Rbardot{} Gn. 27, 27-28 \Vbardot{} ibid., 29; \textbf{H149}}

\vspace{-5mm}

\responsorium{VII}{temporalia/resp-ecceodorfiliimei-CROCHU.gtex}{}}
\newcommand{\lectioiii}{\pars{Lectio III.} \scriptura{Gn. 27, 21-29}

\noindent Dixítque Isaac: "Accéde huc ut tangam te, fili mi, et probem utrum tu sis fílius meus Esau, an non."

\noindent Accéssit ille ad patrem, et, palpáto eo, dixit Isaac: "Vox quidem, vox Iacob est: sed manus, manus sunt Esau."

\noindent Et non cognóvit eum, quia pilósæ manus similitúdinem maióris exprésserant.

\noindent Benedícens ergo illi, Ait: "Tu es fílius meus Esau?"

\noindent Respóndit: "Ego sum."

\noindent At ille: "Affer mihi, inquit, cibos de venatióne tua, fili mi, ut benedícat tibi ánima mea."

\noindent Quos cum oblátos comedísset, óbtulit ei étiam vinum. Quo hausto, dixit ad eum: "Accéde ad me, et da mihi ósculum, fili mi." Accéssit, et osculátus est eum.

\noindent Statímque ut sensit vestimentórum illíus flagrántiam, benedícens illi, ait: "Ecce odor fílii mei sicut odor agri pleni, cui benedíxit Dóminus. Det tibi Deus de rore cæli, et de pinguédine terræ abundántiam fruménti et vini. Et sérviant tibi pópuli, et adórent te tribus: esto dóminus fratrum tuórum, et incurvéntur ante te fílii matris tuæ. Qui maledíxerit tibi, sit ille maledíctus: et qui benedíxerit tibi, benedictiónibus repleátur."}
\newcommand{\responsoriumiii}{\pars{Responsorium 3.} \scriptura{\Rbardot{} Gn. 28, 28-29 \Vbardot{} ibid.; \textbf{H150}}

\vspace{-5mm}

\responsorium{VII}{temporalia/resp-dettibideus-CROCHU-cumdox.gtex}{}}
\newcommand{\lectioiv}{\pars{Lectio IV.} \scriptura{Cap. 10, tom. 4; post init.}

\noindent Ex libro sancti Augustíni Epíscopi contra mendácium.

\noindent Iacob quod matre fecit auctóre, ut patrem fállere viderétur, si diligénter et fidéliter attendátur, non est mendácium, sed mystérium.

\noindent Quæ si mendácia dixérimus, omnes étiam parábolæ ac figúræ significandárum quarumcúmque rerum, quæ non ad proprietátem accipiéndæ sunt, sed in eis áliud ex álio est intelligéndum, dicéntur esse mendácia: quod absit omníno.

\noindent Nam qui hoc putat, trópicis étiam tam multis locutiónibus ómnibus potest hanc importáre calúmniam; ita ut et hæc ipsa, quæ appellátur metáphora, hoc est, de re própria ad rem non própriam verbi alicúius usurpáta translátio, possit ista ratióne mendácium nuncupári.}
\newcommand{\responsoriumiv}{\pars{Responsorium 4.} \scriptura{\Rbardot{} Gn. 28, 17 \Vbardot{} ibid., 16; \textbf{H150}}

\vspace{-5mm}

\responsorium{I}{temporalia/resp-dumexiretiacob-CROCHU.gtex}{}}
\newcommand{\lectiov}{\pars{Lectio V.}

\noindent Quæ significántur enim, útique ipsa dicúntur: putántur autem mendácia, quóniam non ea quæ vere significántur, dicta intelligúntur; sed ea, quæ falsa sunt, dicta esse credúntur.

\noindent Hoc ut exémplis fiat plánius, idípsum quod Iacob fecit, atténde. Hædínis certe péllibus membra contéxit.

\noindent Si causam próximam requirámus, mentítum putábimus: hoc enim fecit, ut putarétur esse qui non erat.

\noindent Si autem hoc factum ad illud, propter quod significándum revéra factum est, referátur: per hædínas pelles, peccáta; per eum vero, qui eis se opéruit, ille significátus est, qui non sua, sed aliéna peccáta portávit.}
\newcommand{\responsoriumv}{\pars{Responsorium 5.} \scriptura{\Rbardot{} Gn. 28, 20-22 \Vbardot{} ibid., 16; \textbf{H150}}

\vspace{-5mm}

\responsorium{I}{temporalia/resp-sidominusdeusmeus-CROCHU.gtex}{}}
\newcommand{\lectiovi}{\pars{Lectio VI.}

\noindent Verax ergo significátio nullo modo mendácium recte dici potest: ut autem in facto, ita et in verbo. Nam cum ei pater dixísset: Quis es tu, fili? ille respóndit: Ego sum Esau primogénitus tuus.

\noindent Hoc si referátur ad duos illos géminos, mendácium vidébitur: si autem ad illud, propter quod significándum ista gesta dictáque conscrípta sunt; ille est hic intelligéndus in córpore suo, quod est eius Ecclésia, qui de hac re loquens, ait:

\noindent {\color{gray} \emph{Cum vidéritis Abraham, et Isaac et Iacob et omnes Prophétas in regno Dei, vos autem expélli foras.}

\noindent Et, \emph{Vénient ab Oriénte et Occidénte, et Aquilóne et Austro, et accúmbent in regno Dei.}

\noindent Et,} \emph{Ecce sunt novíssimi qui erant primi: et sunt primi, qui erant novíssimi.} Sic enim quodámmodo minor maióris primátum frater ábstulit, atque in se tránstulit fratris.}
\newcommand{\responsoriumvi}{\pars{Responsorium 6.} \scriptura{\Rbardot{} Gn. 28, 21-22 \Vbardot{} ibid., 20; \textbf{H150}}

\vspace{-5mm}

\responsorium{I}{temporalia/resp-eritmihidominusindeum-CROCHU-cumdox.gtex}{}}
\newcommand{\evangelium}{\pars{Versus.}

\noindent \Vbardot{} Scuto circúmdabit te véritas eius.

\noindent \Rbardot{} Non timébis a timóre noctúrno.

\vspace{5mm}

\sineinitiali{temporalia/oratiodominica-mat.gtex}

\vspace{5mm}

\pars{Absolutio.}

\cuminitiali{}{temporalia/absolutio-avinculis.gtex}

\vfill
\pagebreak

\cuminitiali{}{temporalia/benedictio-solemn-evangelica.gtex}

\vspace{7mm}

\pars{Evangelium} \scriptura{Lc. 9, 28-36}

\noindent Léctio sancti Evangélii secúndum Lucam.

\noindent In illo témpore: Assúmpsit Iesus Petrum et Ioánnem et Iacóbum et ascéndit in montem, ut oráret. Et facta est, dum oráret, spécies vultus eius áltera, et vestítus eius albus, refúlgens. Et ecce duo viri loquebántur cum illo, et erant Móyses et Elías, qui visi in glória dicébant éxodum eius, quam completúrus erat in Ierúsalem. Petrus vero et qui cum illo graváti erant somno; et evigilántes vidérunt glóriam eius et duos viros, qui stabant cum illo.

\noindent Et factum est, cum discéderent ab illo, ait Petrus ad Iesum: «Præcéptor, bonum est nos hic esse; et faciámus tria tabernácula: unum tibi et unum Móysi et unum Elíæ», nésciens quid díceret. Hæc autem illo loquénte, facta est nubes et obumbrávit eos; et timuérunt intrántibus illis in nubem.

\noindent Et vox facta est de nube dicens: «Hic est Fílius meus eléctus; ipsum audíte». Et dum fíeret vox, invéntus est Iesus solus.

\noindent Et ipsi tacuérunt et némini dixérunt in illis diébus quidquam ex his, quæ víderant.

\vspace{5mm}

\scriptura{Sermo 51, 3-4. 8: PL 54, 310.311. 313}

\noindent Ex Sermónibus sancti Leónis Magni papæ.

\noindent Aperit Dóminus coram eléctis téstibus glóriam suam, et commúnem illam cum céteris córporis formam tanto splendóre claríficat, ut et fácies eius solis fulgóri símilis, et vestítus candóri nívium esset æquális. In qua transfiguratióne illud quidem principáliter agebátur, ut de córdibus discipulórum crucis scándalum tollerétur; nec conturbáret eórum fidem voluntáriæ humílitas passiónis, quibus reveláta esset abscónditæ excelléntia dignitátis.

\noindent Sed non minóre providéntia spes sanctæ Ecclésiæ fundabátur, ut totum corpus Christi agnósceret quali esset commutatióne donándum, et eius sibi honóris consórtium membra promítterent, qui in cápite præfulsísset.

\noindent De quo idem Dóminus díxerat, cum de advéntus sui maiestáte loquerétur: \emph{Tunc iusti fulgébunt sicut sol in regno Patris sui;} protestánte hoc ipsum beáto Paulo apóstolo et dicénte: \emph{Exístimo enim quod non sunt condígnæ passiónes huius témporis ad futúram glóriam, quæ revelábitur in nobis;} et íterum: \emph{Mórtui enim estis, et vita vestra abscóndita est cum Christo in Deo. Cum enim Christus apparúerit vita vestra, tunc et vos apparébitis cum ipso in glória.}

\noindent {\color{gray} Confirmándis vero apóstolis et ad omnem sciéntiam provehéndis, ália quoque in illo miráculo accéssit instrúctio. Móyses enim et Elías, lex scílicet et prophétæ, apparuérunt cum Dómino loquéntes, ut veríssime in illa quinque virórum præséntia complerétur quod dictum est: \emph{In duóbus vel tribus téstibus stat omne verbum.}

\noindent Quid hoc stabílius, quid fírmius verbo, in cuius prædicatióne véteris et novi testaménti cóncinit tuba et cum evangélica doctrína antiquárum protestatiónum instruménta concúrrunt?

\noindent Astipulántur enim sibi ínvicem utriúsque fœ́deris páginæ; et, quem sub velámine mysteriórum præcedéntia signa promíserant, maniféstum atque perspícuum præséntis glóriæ splendor osténdit; quia, sicut ait beátus Ioánnes, \emph{lex per Móysen data est, grátia autem et véritas per Iesum Christum facta est}; in quo et propheticárum promíssio impléta est figurárum et legálium rátio præceptórum, dum et veram docet prophetíam per sui præséntiam, et possibília facit mandáta per grátiam.}

\noindent Confirmétur ergo secúndum prædicatiónem sacratíssimi Evangélii ómnium fides, et nemo de Christi cruce, per quam mundus redémptus est, erubéscat.

\noindent Nec ídeo quisquam aut pati pro iustítia tímeat, aut de promissórum retributióne diffídat, quia per labórem ad réquiem, et per mortem transítur ad vitam; cum omnem humilitátis nostræ infirmitátem ille suscéperit, in quo, si in confessióne et in dilectióne ipsíus permaneámus, et quod vicit víncimus, et quod promísit accípimus.

\noindent Quia sive ad faciénda mandáta, sive ad toleránda advérsa, præmíssa Patris vox debet semper áuribus nostris insonáre, dicéntis: \emph{Hic est Fílius meus diléctus, in quo mihi bene complácui: ipsum audíte.}

\vfill
\pagebreak

\pars{Responsorium 7.} \scriptura{\Rbardot{} Gn. 32, 30 \Vbardot{} ibid., 28; \textbf{H151}}

\vspace{-5mm}

\grechangedim{interwordspacetext}{0.26 cm plus 0.15 cm minus 0.05 cm}{scalable}%
\responsorium{VI}{temporalia/resp-vididominumfacieadfaciem-CROCHU-cumdox.gtex}{}
\grechangedim{interwordspacetext}{0.22 cm plus 0.15 cm minus 0.05 cm}{scalable}

\vfill
\pagebreak
}
\newcommand{\laudes}{\pars{Psalmus 1.} \scriptura{Ps. 117, 16; \textbf{H151}}

\vspace{-4mm}

\antiphona{VIII c}{temporalia/ant-dexteradomini.gtex}

\scriptura{Psalmus 117.}

\initiumpsalmi{temporalia/ps117-initium-viii-C-auto.gtex}

\input{temporalia/ps117-viii-C.tex}

\vfill

\antiphona{}{temporalia/ant-dexteradomini.gtex}

\vfill
\pagebreak

\pars{Psalmus 2.} \scriptura{Cf. Dn. 3, 51; \textbf{H151}}

\vspace{-4mm}

\antiphona{VIII G}{temporalia/ant-triumpuerorum.gtex}

\scriptura{Canticum Danielis, Dan. 3, 52-57}

%\vspace{-3mm}

\initiumpsalmi{temporalia/dan33-initium-viii-G-auto.gtex}

\input{temporalia/dan33-viii-G.tex} \Abardot{}

\vfill
\pagebreak

\pars{Psalmus 3.} \scriptura{Ps. 150, 1}

\vspace{-4mm}

\antiphona{VII a}{temporalia/ant-infirmamento.gtex}

\scriptura{Psalmus 150.}

\initiumpsalmi{temporalia/ps150-initium-vii-a-auto.gtex}

\input{temporalia/ps150-vii-a.tex} \Abardot{}

\vfill
\pagebreak}
\newcommand{\lectiobrevis}{\pars{Lectio brevis.} \scriptura{Neh. 8, 9.10}

\noindent Dies iste sanctificátus est Dómino Deo nostro! Nolíte lugére et nolíte flere. Quia sanctus dies Dómini nostri est; et nolíte contristári, gáudium étenim Dómini est fortitúdo vestra.}
\newcommand{\responsoriumbreve}{\pars{Responsorium breve.}

\cuminitiali{IV}{temporalia/resp-christefilideiquiattritus-tq.gtex}}
\newcommand{\hymnuslaudes}{\pars{Hymnus} \scriptura{Gregorius Magnus (\olddag{} 604)}

\cuminitiali{II}{temporalia/hym-PrecemurOmnes.gtex}}
\newcommand{\preces}{\noindent Deum glorificémus, cuius bonitátis infinítus est thesáurus,\gredagger{} et per Iesum Christum, qui est semper vivens ad interpellándum pro nobis,\grestar{} eum deprecémur, dicéntes:

\Rbardot{} Accénde in nobis ignem tui amóris.

\noindent Deus misericórdiæ, fac ut hódie abundémus in opéribus pietátis,~\grestar{} atque omnes nostram experiántur humanitátem.

\Rbardot{} Accénde in nobis ignem tui amóris.

\noindent Qui in dilúvio Noe per arcam salvásti,~\grestar{} salva catechúmenos in aqua baptísmatis.

\Rbardot{} Accénde in nobis ignem tui amóris.

\noindent Præsta nos non solo pane satiári,~\grestar{} sed omni verbo, quod procédit de ore tuo.

\Rbardot{} Accénde in nobis ignem tui amóris.

\noindent Fac ut omnes dissensiónes componámus,~\grestar{} ut pace et caritáte, te donánte, gaudeámus.

\Rbardot{} Accénde in nobis ignem tui amóris.}
\newcommand{\benedictus}{\pars{Canticum Zachariæ.} \scriptura{Cf. Mt. 17, 1.2; \textbf{H149}}

\vspace{-4mm}

{
\grechangedim{interwordspacetext}{0.18 cm plus 0.15 cm minus 0.05 cm}{scalable}%
\antiphona{II A}{temporalia/ant-assumpsit.gtex}
\grechangedim{interwordspacetext}{0.22 cm plus 0.15 cm minus 0.05 cm}{scalable}%
}

%\vspace{-2mm}

\scriptura{Lc. 1, 68-79}

%\vspace{-2mm}

\cantusSineNeumas
\initiumpsalmi{temporalia/benedictus-initium-iisoll-A-auto.gtex}

%\vspace{-1.5mm}

\input{temporalia/benedictus-iisoll-A.tex} \Abardot{}}
\newcommand{\magnificatii}{\pars{Canticum B. Mariæ V.} \scriptura{Mt. 17, 9; \textbf{H149}}

\vspace{-4mm}

{
\grechangedim{interwordspacetext}{0.18 cm plus 0.15 cm minus 0.05 cm}{scalable}%
\antiphona{I f}{temporalia/ant-visionemquamvidistis.gtex}
\grechangedim{interwordspacetext}{0.22 cm plus 0.15 cm minus 0.05 cm}{scalable}%
}

\vspace{-2mm}

\scriptura{Lc. 1, 46-55}

\vspace{-2mm}

\cantusSineNeumas
\initiumpsalmi{temporalia/magnificat-initium-isoll-f.gtex}

%\vspace{-1.5mm}

\input{temporalia/magnificat-isoll-f.tex} \Abardot{}}
\newcommand{\hebdomada}{infra Hebdom. II per Annum.}
\newcommand{\matub}{Matutinum Hebdomadae B}
\newcommand{\laudb}{Laudes Hebdomadae B}
\newcommand{\laudbd}{Laudes Hebdomadae B vel D}

% LuaLaTeX

\documentclass[a4paper, twoside, 12pt]{article}
\usepackage[latin]{babel}
%\usepackage[landscape, left=3cm, right=1.5cm, top=2cm, bottom=1cm]{geometry} % okraje stranky
%\usepackage[landscape, a4paper, mag=1166, truedimen, left=2cm, right=1.5cm, top=1.6cm, bottom=0.95cm]{geometry} % okraje stranky
\usepackage[landscape, a4paper, mag=1400, truedimen, left=0.5cm, right=0.5cm, top=0.5cm, bottom=0.5cm]{geometry} % okraje stranky

\usepackage{fontspec}
\setmainfont[FeatureFile={junicode.fea}, Ligatures={Common, TeX}, RawFeature=+fixi]{Junicode}
%\setmainfont{Junicode}

% shortcut for Junicode without ligatures (for the Czech texts)
\newfontfamily\nlfont[FeatureFile={junicode.fea}, Ligatures={Common, TeX}, RawFeature=+fixi]{Junicode}

\usepackage{multicol}
\usepackage{color}
\usepackage{lettrine}
\usepackage{fancyhdr}

% usual packages loading:
\usepackage{luatextra}
\usepackage{graphicx} % support the \includegraphics command and options
\usepackage{gregoriotex} % for gregorio score inclusion
\usepackage{gregoriosyms}
\usepackage{wrapfig} % figures wrapped by the text
\usepackage{parcolumns}
\usepackage[contents={},opacity=1,scale=1,color=black]{background}
\usepackage{tikzpagenodes}
\usepackage{calc}
\usepackage{longtable}
\usetikzlibrary{calc}

\setlength{\headheight}{14.5pt}

\input{conventuscommune.tex} % Often used macros
%%%% Preklady jednotlivych zpevu (nektere se opakuji, a je dobre mit je
% vsechny na jedne hromade)

% HOURS ---

\newcommand{\trAntI}{\translatioCantus{Muž boží měl kožený toulec, pečlivě
zavázaný, jenž mu visel na šíji a~často se ho dotýkal.}}

\newcommand{\trAntII}{\translatioCantus{Klíč od~něho tak dobře střežil, že
dokud žil v~těle, nikdo z~jeho žáků nezvěděl, co je uvnitř.}}

\newcommand{\trAntIII}{\translatioCantus{Ale když se odebral z~tohoto
života, schránku otevřeli a~objevili v~ní žíněné roucho a~měděný řetěz
potřísněný krví.}}

\newcommand{\trAntIV}{\translatioCantus{A když prohlédli mistrovo tělo,
nalezli jeho tělo na čtyřech místech hluboce zbrázděno ranami od řetězu.}}

\newcommand{\trAntV}{\translatioCantus{Krev vytékající z~těch ran, místy
prostoupila i~žíněným rouchem.}}

\newcommand{\trCapituli}{\translatioCantus{
Miláčkovi Boha a~lidí,
Mojžíšovi požehnané paměti,~\gredagger{}
dopřál slávu rovnou slávě svatých~\grestar{}
učinil ho mocným na postrach nepřátelům
a~jeho slovy zastavil divy.}}

\newcommand{\trLectioBrevis}{\translatioCantus{
Pamatujte na své představené,
kteří vám hlásali Boží slovo.
Uvažte, jak oni skončili život, a~napodobujte jejich víru.
Ježíš Kristus je stejný včera i~dnes i~navěky.
Nenechte se svést věelijakými cizími naukami.}}

\newcommand{\trRespLaud}{\translatioCantus{Spravedlivého vodil Hospodin~\grestar{}
po přímých stezkách. \Vbardot{} A~ukázal mu Boží království.}}

\newcommand{\trRespLaudB}{\translatioCantus{Na tvých hradbách, Jeruzaléme,
ustanovil jsem strážné;~\grestar{}
budou bdít nad mým lidem. \Vbardot{} Ani ve dne, ani v~noci nesmějí nikdy
mlčet.}}

\newcommand{\trVersus}{\translatioCantus{\Vbardot{} Ústa spravedlivého šeptají moudrost, aleluja.
\Rbardot{} A~jeho jazyk ohlašuje právo, aleluja.}}

\newcommand{\trAntBenedictus}{\translatioCantus{Když na bujné oře vložili
nosítka a~sňali jim uzdu, vydali se přímo k~cele božího muže.}}

\newcommand{\trPreces}{\translatioCantus{
\noindent S vděčností chvalme Krista, dobrého Pastýře, \gredagger{} který dal život za své ovce, \grestar{} a~pokorně ho prosme: \Rbardot{} Pane, buď pastýřem svého lidu.

\noindent Kriste, ty dáváš církvi pastýře, a~jejich službou se ujímáš svého lidu, \grestar{} dej, ať v~lásce těch, kteří nás vedou, poznáváme, jak nás miluješ. \Rbardot{} Pane, buď pastýřem svého lidu.

\noindent Ty stále konáš skrze své zástupce službu pastýře a~učitele, \grestar{} nepřestávej nás nikdy vést prostřednictvím svých služebníků. \Rbardot{} Pane, buď pastýřem svého lidu.

\noindent Ty prokazuješ svému lidu skrze jeho pastýře službu lékaře duše i~těla, \grestar{} ochraňuj náš život a~veď nás ke svatosti. \Rbardot{} Pane, buď pastýřem svého lidu.

\noindent Ty posíláš své svaté, aby slovem i~příkladem vedli tvůj lid k~tobě, \grestar{} na jejich přímluvu nás posiluj, abychom vytrvali na cestě, která vede k~věčnému životu. \Rbardot{} Pane, buď pastýřem svého lidu.}}

\newcommand{\trOrationis}{\translatioCantus{Bože, jenž nám dopřáváš radovat
se z~výroční slavnosti svatého tvého vyznavače Havla, uděl dobrotivě,
abychom když slavíme jeho narození, též se řídili podobou jeho skutků.
Skrze…}}
 % Czech translations of the proper texts

\newcommand{\annusEditionis}{2020}

%%%% Vicekrat opakovane kousky

\newcommand{\anteOrationem}{
  \rubrica{Ante Orationem, cantatur a Superiore:}

  \pars{Supplicatio Litaniæ.}

  \cuminitiali{}{temporalia/supplicatiolitaniae.gtex}

  \pars{Oratio Dominica.}

  \cuminitiali{}{temporalia/oratiodominica.gtex}

  \rubrica{Deinde dicitur ab Hebdomadario:}

  \cuminitiali{}{temporalia/dominusvobiscum-solemnis.gtex}

  \rubrica{In choro monialium loco Dominus vobiscum dicitur:}

  \sineinitiali{temporalia/domineexaudi.gtex}
}

\setlength{\columnsep}{30pt} % prostor mezi sloupci

%%%%%%%%%%%%%%%%%%%%%%%%%%%%%%%%%%%%%%%%%%%%%%%%%%%%%%%%%%%%%%%%%%%%%%%%%%%%%%%%%%%%%%%%%%%%%%%%%%%%%%%%%%%%%
\begin{document}

% Here we set the space around the initial.
% Please report to http://home.gna.org/gregorio/gregoriotex/details for more details and options
\grechangedim{afterinitialshift}{2.2mm}{scalable}
\grechangedim{beforeinitialshift}{2.2mm}{scalable}
\grechangedim{interwordspacetext}{0.22 cm plus 0.15 cm minus 0.05 cm}{scalable}%
\grechangedim{annotationraise}{-0.2cm}{scalable}

% Here we set the initial font. Change 38 if you want a bigger initial.
% Emit the initials in red.
\grechangestyle{initial}{\color{red}\fontsize{38}{38}\selectfont}

\pagestyle{empty}

%%%% Titulni stranka
\begin{titulusOfficii}
\titulus{}
\end{titulusOfficii}

% graphic
%\vspace{1.5cm}
%\begin{center}
%\includegraphics[width=8cm]{emmaus.jpg}
%\end{center}

\vfill

\begin{center}
%Ad usum et secundum consuetudines chori \guillemotright{}Conventus Choralis\guillemotleft.

%Editio Sancti Wolfgangi \annusEditionis
\end{center}

\pagebreak

\renewcommand{\headrulewidth}{0pt} % no horiz. rule at the header
\fancyhf{}
\pagestyle{fancy}

\pars{Oratio ante divinum Officium.}

\lettrine{{\color{red}A}}{peri,} Dómine, os meum ad benedicéndum nomen sanctum tuum:
munda quoque cor meum ab ómnibus vanis, pervérsis, et aliénis
cogitatiónibus:
intelléctum illúmina, afféctum inflámma,
ut digne, atténte ac devóte hoc Offícium recitáre váleam,
et exaudíri mérear ante conspéctum Divínæ Maiestátis tuæ.
Per Christum, Dóminum nostrum.
\Rbardot{} Amen.

Dómine, in unióne illíus divínæ intentiónis,
qua ipse in terris laudes Deo persolvísti,
has tibi Horas \rubricatum{(vel \textnormal{hanc tibi Horam})} persólvo.

%\trOratioAnteOfficium

\vfill

\pars{Oratio post divinum Officium.}

\rubrica{
  Orationem sequentem devote post Officium recitantibus
  Leo Papa X. defectus, et culpas in eo persolvendo ex humana
  fragilitate contractas, indulsit, et dicitur flexis genibus.
}

\lettrine{{\color{red}S}}{acrosánctæ} et indivíduæ Trinitáti,
crucifíxi Dómini nostri Iesu Christi humanitáti,
beatíssimæ et gloriosíssimæ sempérque Vírginis Maríæ
fecúndæ integritáti, 
et ómnium Sanctórum universitáti
sit sempitérna laus, honor, virtus et glória
ab omni creatúra,
nobísque remíssio ómnium peccatórum,
per infiníta sǽcula sæculórum.
\Rbardot{} Amen.

\noindent \Vbardot{} Beáta víscera Maríæ Virginis, quæ portavérunt
ætérni Patris Fílium.\\
\Rbardot{} Et beáta úbera, quæ lactavérunt Christum Dominum.

\rubrica{Et dicitur secreto \textnormal{Pater noster.} et \textnormal{Ave María.}}

%\trOratioPostOfficium

\vfill

\hora{Ad I. Vesperas.} %%%%%%%%%%%%%%%%%%%%%%%%%%%%%%%%%%%%%%%%%%%%%%%%%%%%%
%\sideThumbs{I. Vesperæ}

\cantusSineNeumas

\vspace{0.5cm}
\grechangedim{interwordspacetext}{0.18 cm plus 0.15 cm minus 0.05 cm}{scalable}%
\cuminitiali{}{temporalia/deusinadiutorium-solemnis.gtex}
\grechangedim{interwordspacetext}{0.22 cm plus 0.15 cm minus 0.05 cm}{scalable}%

\vfill
\pagebreak

\pars{Psalmus 1.} \scriptura{Ps. 144, 13; \textbf{H100}}

\vspace{-4mm}

\antiphona{VII c\textsuperscript{2}}{temporalia/ant-regnumtuum.gtex}

\scriptura{Psalmus 144, 10-21.}

\initiumpsalmi{temporalia/ps144ii-initium-vii-c2-auto.gtex}

%\psalmusEtTranslatioT{temporalia/ps144ii-VII-comb.tex}{10cm}
\input{temporalia/ps144ii-VII.tex} \Abardot{}

\vspace{-1cm}

\vfill
\pagebreak

\pars{Psalmus 2.} \scriptura{Ps. 145, 2; \textbf{H100}}

\vspace{-4mm}

\antiphona{IV E}{temporalia/ant-laudabodeum.gtex}

\scriptura{Psalmus 145.}

\initiumpsalmi{temporalia/ps145-initium-iv-E-auto.gtex}

%\psalmusEtTranslatioT{temporalia/ps145-VII-comb.tex}{10cm}
\input{temporalia/ps145-VII.tex} \Abardot{}

\vfill
\pagebreak

\pars{Psalmus 3.} \scriptura{Ps. 146, 1; \textbf{H101}}

\vspace{-4mm}

\antiphona{VIII a}{temporalia/ant-deonostro.gtex}

\scriptura{Psalmus 146.}

\initiumpsalmi{temporalia/ps146-initium-viii-A-auto.gtex}

%\psalmusEtTranslatioT{temporalia/ps146-VII-comb.tex}{10cm}
\input{temporalia/ps146-VII.tex} \Abardot{}

\vfill
\pagebreak

\pars{Psalmus 4.} \scriptura{Ps. 147, 1}

\vspace{-4mm}

\antiphona{E}{temporalia/ant-laudajerusalem.gtex}

\scriptura{Psalmus 147.}

\initiumpsalmi{temporalia/ps147-initium-e-auto.gtex}

%\psalmusEtTranslatioT{temporalia/ps147-VII-comb.tex}{10cm}
\input{temporalia/ps147-VII.tex} \Abardot{}

\vfill
\pagebreak

\pars{Capitulum.} \scriptura{Rom. 11, 33}

\grechangedim{interwordspacetext}{0.12 cm plus 0.15 cm minus 0.05 cm}{scalable}%
\cuminitiali{}{temporalia/capitulum-OAltitudo.gtex}
\grechangedim{interwordspacetext}{0.22 cm plus 0.15 cm minus 0.05 cm}{scalable}

% preklad Jeruz. bible
%\trCapituliI

\vfill

\pars{Responsorium breve.} \scriptura{Ps. 146, 5}

\cuminitiali{VI}{temporalia/resp-magnusdominusnoster.gtex}

%\trResp

\vfill
\pagebreak

\pars{Hymnus} \scriptura{Ambrosius (\olddag{} 397)}

\cuminitiali{I}{temporalia/hym-OLuxBeata-aestivalis.gtex}
\vspace{-3mm}
%\input{hym-OLuxBeata-bohtext.tex}

\vfill
%\pagebreak

\pars{Versus.}

% Versus. %%%
\sineinitiali{temporalia/versus-vespertina.gtex}

%\noindent \trVersus

\vfill
\pagebreak

\magnificati

\vfill
\pagebreak

%\sideThumbs{{\scriptsize{}Fine horarum}}

\anteOrationem

\pagebreak

% Oratio. %%%
\oratioLaudes

\vspace{-1mm}
%\trOrationisI

\vfill

\rubrica{Hebdomadarius dicit iterum Dominus vobiscum, vel cantor dicit:}

\vspace{2mm}

\sineinitiali{temporalia/domineexaudi.gtex}

\rubrica{Postea cantatur a cantore:}

\vspace{2mm}

\cuminitiali{I}{temporalia/benedicamus-dominica-perannum.gtex}

\vspace{1mm}

\vfill
\pagebreak

\hora{Ad Matutinum.} %%%%%%%%%%%%%%%%%%%%%%%%%%%%%%%%%%%%%%%%%%%%%%%%%%%%%
%\sideThumbs{Matutinum}

\vspace{2mm}

\cuminitiali{}{temporalia/dominelabiamea.gtex}

\vspace{2mm}

\pars{Invitatorium.} \scriptura{Ps. 94, 1; Psalmus 94}

\vspace{-6mm}

\antiphona{E}{temporalia/inv-veniteexsultemus.gtex}

\vfill
\pagebreak

\pars{Hymnus.} \scriptura{Adamus Sancti Victoris (\olddag 1146)}

\vspace{-5mm}

\antiphona{VII}{temporalia/hym-SalveDies.gtex}

\scriptura{Non dicitur \textnormal{Amen} in fine.}
%{
%\vspace{-5mm}
%\setlength{\columnsep}{0pt} % prostor mezi sloupci
%\input{hym-SalveDies-bohtext.tex}
%\setlength{\columnsep}{30pt} % prostor mezi sloupci
%}

\vfill
\pagebreak

\subhora{In I. Nocturno}

\pars{Psalmus 1.} \scriptura{Ps. 1, 1}

\vspace{-4mm}

\antiphona{VIII G}{temporalia/ant-beatusvir.gtex}

%\vspace{-5mm}

\scriptura{Ps. 1}

%\vspace{-2mm}

\initiumpsalmi{temporalia/ps1-initium-viii-G-auto.gtex}

%\psalmusEtTranslatioT{temporalia/ps1-I-comb.tex}{10cm}
\input{temporalia/ps1-I.tex} \Abardot{}

\vfill
\pagebreak

\pars{Psalmus 2.} \scriptura{Ps. 2, 11; \textbf{H93}}

\vspace{-4mm}

\antiphona{VII a}{temporalia/ant-servitedomino.gtex}

\vspace{-3mm}

\scriptura{Ps. 2}

\vspace{-2mm}

\initiumpsalmi{temporalia/ps2-initium-vii-a-auto.gtex}

%\psalmusEtTranslatioT{temporalia/ps2-I-comb.tex}{10cm}
\input{temporalia/ps2-I.tex} \Abardot{}

\vfill
\pagebreak

\pars{Psalmus 3.} \scriptura{Ps. 3, 7}

\vspace{-4mm}

\antiphona{VI F}{temporalia/ant-exsurgedominesalvum.gtex}

%\vspace{-5mm}

\scriptura{Ps. 3}

\initiumpsalmi{temporalia/ps3-initium-vi-F-auto.gtex}

%\psalmusEtTranslatioT{temporalia/ps3-I-comb.tex}{10cm}
\input{temporalia/ps3-I.tex} \Abardot{}

\vfill
\pagebreak

\pars{Versus.} \scriptura{Ps. 118, 55}

% Versus. %%%
\sineinitiali{temporalia/versus-memorfui.gtex}

\vspace{5mm}

\sineinitiali{temporalia/oratiodominica-mat.gtex}

\vspace{5mm}

\pars{Absolutio.}

\cuminitiali{}{temporalia/absolutio-exaudi.gtex}

\vfill
\pagebreak

\cuminitiali{}{temporalia/benedictio-solemn-benedictione.gtex}

\vspace{7mm}

\lectioi

\noindent \Vbardot{} Tu autem, Dómine, miserére nobis.
\noindent \Rbardot{} Deo grátias.

\vfill
\pagebreak

\responsoriumi

\vfill
\pagebreak

\cuminitiali{}{temporalia/benedictio-solemn-unigenitus.gtex}

\vspace{7mm}

\lectioii

\noindent \Vbardot{} Tu autem, Dómine, miserére nobis.
\noindent \Rbardot{} Deo grátias.

\vfill
\pagebreak

\responsoriumii

\vfill
\pagebreak

\cuminitiali{}{temporalia/benedictio-solemn-spiritus.gtex}

\vspace{7mm}

\lectioiii

\noindent \Vbardot{} Tu autem, Dómine, miserére nobis.
\noindent \Rbardot{} Deo grátias.

\vfill
\pagebreak

\responsoriumiii

\vfill
\pagebreak

\subhora{In II. Nocturno}

\pars{Psalmus 4.} \scriptura{Ps. 8, 2}

\vspace{-4mm}

\antiphona{I g}{temporalia/ant-quamadmirabileest.gtex}

%\vspace{-5mm}

\scriptura{Ps. 8}

%A\vspace{-2mm}

\initiumpsalmi{temporalia/ps8-initium-i-g-auto.gtex}

%\psalmusEtTranslatioT{temporalia/ps8-I-comb.tex}{10cm}
\input{temporalia/ps8-I.tex} \Abardot{}

\vfill
\pagebreak

\pars{Psalmus 5.} \scriptura{Ps. 9, 5}

\vspace{-4mm}

\antiphona{VIII G}{temporalia/ant-sedistisuperthronum.gtex}

%\vspace{-5mm}

\scriptura{Ps. 9, 2-11}

\initiumpsalmi{temporalia/ps9ii_xi-initium-viii-G-auto.gtex}

%\psalmusEtTranslatioT{temporalia/ps9ii_xi-I-comb.tex}{10cm}
\input{temporalia/ps9ii_xi-I.tex} \Abardot{}

\vfill
\pagebreak

\pars{Psalmus 6.} \scriptura{Ps. 9, 20}

\vspace{-4mm}

\antiphona{I g\textsuperscript{3}}{temporalia/ant-exsurgedominenon.gtex}

%\vspace{-5mm}

\scriptura{Ps. 9, 12-21}

\initiumpsalmi{temporalia/ps9xii_xxi-initium-i-g3-auto.gtex}

%\psalmusEtTranslatioT{temporalia/ps9xii_xxi-I-comb.tex}{10cm}
\input{temporalia/ps9xii_xxi-I.tex} \Abardot{}

\vfill
\pagebreak

\pars{Versus.} \scriptura{Ps. 118, 62}

% Versus. %%%
\sineinitiali{temporalia/versus-medianocte.gtex}

\vspace{5mm}

\sineinitiali{temporalia/oratiodominica-mat.gtex}

\vspace{5mm}

\pars{Absolutio.}

\cuminitiali{}{temporalia/absolutio-ipsius.gtex}

\vfill
\pagebreak

\cuminitiali{}{temporalia/benedictio-solemn-deus.gtex}

\vspace{7mm}

\lectioiv

\noindent \Vbardot{} Tu autem, Dómine, miserére nobis.
\noindent \Rbardot{} Deo grátias.

\vfill
\pagebreak

\responsoriumiv

\vfill
\pagebreak

\cuminitiali{}{temporalia/benedictio-solemn-christus.gtex}

\vspace{7mm}

\lectiov

\noindent \Vbardot{} Tu autem, Dómine, miserére nobis.
\noindent \Rbardot{} Deo grátias.

\vfill
\pagebreak

\responsoriumv

\vfill
\pagebreak

\cuminitiali{}{temporalia/benedictio-solemn-ignem.gtex}

\vspace{7mm}

\lectiovi

\noindent \Vbardot{} Tu autem, Dómine, miserére nobis.
\noindent \Rbardot{} Deo grátias.

\vfill
\pagebreak

\responsoriumvi

\vfill
\pagebreak

\subhora{In III. Nocturno}

\pars{Psalmus 7.} \scriptura{Ps. 9, 22}

\vspace{-4mm}

\antiphona{II D}{temporalia/ant-utquiddomine.gtex}

\vspace{-4mm}

\scriptura{Ps. 9, 22-32}

%\vspace{-2mm}

\initiumpsalmi{temporalia/ps9xxii_xxxii-initium-ii-D-auto.gtex}

%\psalmusEtTranslatioT{temporalia/ps9xxii_xxxii-I-comb.tex}{10cm}
\input{temporalia/ps9xxii_xxxii-I.tex} \Abardot{}

\vfill
\pagebreak

\pars{Psalmus 8.}\scriptura{Ex. 15, 18}

\vspace{-4mm}

\antiphona{IV* e}{temporalia/ant-inaeternum.gtex}

%\vspace{-4mm}

\scriptura{Ps. 9, 33-39}

\initiumpsalmi{temporalia/ps9xxxiii_xxxix-initium-iv_-e-auto.gtex}

%\psalmusEtTranslatioT{temporalia/ps9xxxiii_xxxix-I-comb.tex}{10cm}
\input{temporalia/ps9xxxiii_xxxix-I.tex} \Abardot{}

\vfill
\pagebreak

\pars{Psalmus 9.} \scriptura{Ps. 10, 8}

\vspace{-4mm}

\antiphona{II* f}{temporalia/ant-justusdominus.gtex}

%\vspace{-4mm}

\scriptura{Ps. 10}

%\initiumpsalmi{temporalia/ps10-initium-iv-c-auto.gtex}
\initiumpsalmi{temporalia/ps10-initium-ii_-f.gtex}

%\psalmusEtTranslatioT{temporalia/ps10-I-comb.tex}{10cm}
\input{temporalia/ps10-I.tex} \Abardot{}

\vfill
\pagebreak

\pars{Versus.} \scriptura{Ps. 118, 148}

% Versus. %%%
\sineinitiali{temporalia/versus-praevenerunt.gtex}

\vspace{5mm}

\sineinitiali{temporalia/oratiodominica-mat.gtex}

\vspace{5mm}

\pars{Absolutio.}

\cuminitiali{}{temporalia/absolutio-avinculis.gtex}

\vfill
\pagebreak

\cuminitiali{}{temporalia/benedictio-solemn-evangelica.gtex}

\vspace{7mm}

\lectiovii

\noindent \Vbardot{} Tu autem, Dómine, miserére nobis.
\noindent \Rbardot{} Deo grátias.

\vfill
\pagebreak

\responsoriumvii

\vfill
\pagebreak

\cuminitiali{}{temporalia/benedictio-solemn-divinum.gtex}

\vspace{7mm}

\lectioviii

\noindent \Vbardot{} Tu autem, Dómine, miserére nobis.
\noindent \Rbardot{} Deo grátias.

\vfill
\pagebreak

\responsoriumviii

\vfill
\pagebreak

\cuminitiali{}{temporalia/benedictio-solemn-adsocietatem.gtex}

\vspace{7mm}

\lectioix

\noindent \Vbardot{} Tu autem, Dómine, miserére nobis.
\noindent \Rbardot{} Deo grátias.

\vfill
\pagebreak

% Te Deum

{
\pars{Hymnus Ambrosianus} \scriptura{Tonus Solemnis}

\vspace{-2mm}

\grechangedim{interwordspacetext}{0.26 cm plus 0.15 cm minus 0.05 cm}{scalable}%
\cuminitiali{III}{temporalia/tedeum-solemnis-gn.gtex}
\grechangedim{interwordspacetext}{0.22 cm plus 0.15 cm minus 0.05 cm}{scalable}%
}

\vfill
\pagebreak

\rubrica{Reliqua omittuntur, nisi Laudes separandæ sint.}

\pars{Oratio}

\noindent \Vbardot{} Dómine, exáudi oratiónem meam.

\noindent \Rbardot{} Et clamor meus ad te véniat.

Orémus:

\oratioLaudes

\vspace{7mm}

\pars{Conclusio}

\noindent \Vbardot{} Dómine, exáudi oratiónem meam.

\noindent \Rbardot{} Et clamor meus ad te véniat.

\noindent \Vbardot{} Benedicámus Dómino, allelúia, allelúia.

\noindent \Rbardot{} Deo grátias, allelúia, allelúia.

\noindent \Vbardot{} Fidélium ánimæ per misericórdiam Dei requiéscant in pace.

\noindent \Rbardot{} Amen.

\vfill
\pagebreak

\hora{Ad Laudes.} %%%%%%%%%%%%%%%%%%%%%%%%%%%%%%%%%%%%%%%%%%%%%%%%%%%%%
%\sideThumbs{Laudes}

\cantusSineNeumas

\vspace{0.5cm}
\grechangedim{interwordspacetext}{0.18 cm plus 0.15 cm minus 0.05 cm}{scalable}%
\cuminitiali{}{temporalia/deusinadiutorium-alter.gtex}
\grechangedim{interwordspacetext}{0.22 cm plus 0.15 cm minus 0.05 cm}{scalable}%

\vfill
%\pagebreak

\pars{Psalmus 1.}

\vspace{-4mm}

\antiphona{VI F}{temporalia/ant-alleluia1.gtex}

\scriptura{Psalmus 50.}

\initiumpsalmi{temporalia/ps50-initium-vi-F-auto.gtex}

%\psalmusEtTranslatioT{temporalia/ps50-I-comb.tex}{10cm}
\input{temporalia/ps50-I.tex}

\vfill
\pagebreak

\pars{Psalmus 2.}

\scriptura{Psalmus 117.}

\initiumpsalmi{temporalia/ps117-initium-vi-F-auto.gtex}

%\psalmusEtTranslatioT{temporalia/ps117-I-comb.tex}{10cm}
\input{temporalia/ps117-I.tex}

\vfill
\pagebreak

\pars{Psalmus 3.}

\scriptura{Psalmus 62.}

\initiumpsalmi{temporalia/ps62-initium-vi-F-auto.gtex}

%\psalmusEtTranslatioT{temporalia/ps62-I-comb.tex}{10cm}
\input{temporalia/ps62-I.tex}

\vfill

\vspace{-6mm}

\antiphona{}{temporalia/ant-alleluia1.gtex} % repeat the antiphon - new page

\vfill
\pagebreak

\pars{Psalmus 4.} \scriptura{Dan. 3, 22-26; \textbf{H422}}

\vspace{-4mm}

\antiphona{VIII G}{temporalia/ant-trespueri.gtex}

\scriptura{Canticum trium puerorum, Dan. 3, 57-88 et 56}

\initiumpsalmi{temporalia/dan3-initium-viii-G-auto.gtex}

%\psalmusEtTranslatioT{temporalia/dan3-comb.tex}{10cm}
\input{temporalia/dan3.tex}

\rubrica{Hic non dicitur Gloria Patri, neque Amen.}

\vfill

\vspace{-6mm}

\antiphona{}{temporalia/ant-trespueri.gtex} % repeat the antiphon - new page

\vfill
\pagebreak

\pars{Psalmus 5.}

\vspace{-4mm}

\antiphona{VIII G}{temporalia/ant-alleluia2.gtex}

\scriptura{Psalmus 148.}

\initiumpsalmi{temporalia/ps148-initium-viii-G-auto.gtex}

%\psalmusEtTranslatioT{temporalia/ps148-I-comb.tex}{10cm}
\input{temporalia/ps148-I.tex}

\rubrica{Hic non dicitur Gloria Patri.}

\vfill
\pagebreak

%
\scriptura{Psalmus 149.}

\initiumpsalmi{temporalia/ps149-initium-viii-G-auto.gtex}

%\psalmusEtTranslatioT{temporalia/ps149-I-comb.tex}{10cm}
\input{temporalia/ps149-I.tex}

\rubrica{Hic non dicitur Gloria Patri.}

\vfill
\pagebreak

%
\scriptura{Psalmus 150.}

\initiumpsalmi{temporalia/ps150-initium-viii-G-auto.gtex}

%\psalmusEtTranslatioT{temporalia/ps150-I-comb.tex}{10cm}
\input{temporalia/ps150-I.tex}

\vfill

\vspace{-6mm}

\antiphona{}{temporalia/ant-alleluia2.gtex} % repeat the antiphon - new page

\vfill
\pagebreak

\pars{Capitulum.} \scriptura{Ac. 7, 12}

\grechangedim{interwordspacetext}{0.12 cm plus 0.15 cm minus 0.05 cm}{scalable}%
\cuminitiali{}{temporalia/capitulum-Benedictio.gtex}
\grechangedim{interwordspacetext}{0.22 cm plus 0.15 cm minus 0.05 cm}{scalable}

% preklad Jeruz. bible
%\trCapituliI

\vfill

\pars{Responsorium breve.} \scriptura{Ps. 118, 36-37}

\cuminitiali{IV}{temporalia/resp-inclinacormeum.gtex}

%\trResp

\vfill
\pagebreak

\pars{Hymnus} \scriptura{Gregorius Magnus (\olddag{} 604)}

\cuminitiali{IV}{temporalia/hym-EcceJamNoctis.gtex}
\vspace{-3mm}
%\input{hym-EcceJamNocis-bohtext.tex}

\vfill
%\pagebreak

\pars{Versus.} \scriptura{Ps. 92, 1}

% Versus. %%%
\sineinitiali{temporalia/versus-dominusregnavit.gtex}

%\noindent \trVersus

\vfill
\pagebreak

\benedictus

\vspace{-1cm}

\vfill
\pagebreak

%\sideThumbs{{\scriptsize{}Fine horarum}}

\anteOrationem

\pagebreak

% Oratio. %%%
\oratioLaudes

\vspace{-1mm}
%\trOrationisI

\vfill

\rubrica{Hebdomadarius dicit iterum Dominus vobiscum, vel cantor dicit:}

\vspace{2mm}

\sineinitiali{temporalia/domineexaudi.gtex}

\rubrica{Postea cantatur a cantore:}

\vspace{2mm}

\cuminitiali{I}{temporalia/benedicamus-dominica-perannum.gtex}

\vspace{1mm}

\vfill
\pagebreak

\hora{Ad II. Vesperas.} %%%%%%%%%%%%%%%%%%%%%%%%%%%%%%%%%%%%%%%%%%%%%%%%%%%%%
%\sideThumbs{II. Vesperæ}

\cantusSineNeumas

%\vspace{0.5cm}
\grechangedim{interwordspacetext}{0.18 cm plus 0.15 cm minus 0.05 cm}{scalable}%
\cuminitiali{}{temporalia/deusinadiutorium-solemnis.gtex}
\grechangedim{interwordspacetext}{0.22 cm plus 0.15 cm minus 0.05 cm}{scalable}%

\vfill
%\pagebreak

\vspace{-2mm}

\pars{Psalmus 1.} \scriptura{Ps. 109, 1; \textbf{H91}}

\vspace{-4mm}

\antiphona{VII c\textsuperscript{2}}{temporalia/ant-dixitdominus.gtex}

\vspace{-4mm}

\scriptura{Psalmus 109.}

\initiumpsalmi{temporalia/ps109-initium-vii-c2-auto.gtex}

%\psalmusEtTranslatioT{temporalia/ps109-I-comb.tex}{10cm}
\input{temporalia/ps109-I.tex} \Abardot{}

\vspace{-1cm}

\vfill
\pagebreak

\pars{Psalmus 2.} \scriptura{Ps. 110, 8; \textbf{H91}}

\vspace{-4mm}

\antiphona{IV g}{temporalia/ant-fideliaomnia.gtex}

\scriptura{Psalmus 110.}

\initiumpsalmi{temporalia/ps110-initium-iv-g-auto.gtex}

%\psalmusEtTranslatioT{temporalia/ps110-I-comb.tex}{10cm}
\input{temporalia/ps110-I.tex} \Abardot{}

\vfill
\pagebreak

\pars{Psalmus 3.} \scriptura{Ps. 111, 1; \textbf{H92}}

\vspace{-4mm}

\antiphona{IV a}{temporalia/ant-inmandatis.gtex}

\scriptura{Psalmus 111.}

\initiumpsalmi{temporalia/ps111-initium-iv-a-auto.gtex}

%\psalmusEtTranslatioT{temporalia/ps111-I-comb.tex}{10cm}
\input{temporalia/ps111-I.tex} \Abardot{}

\vfill
\pagebreak

\pars{Psalmus 4.} \scriptura{Ps. 112, 2; \textbf{H92}}

\vspace{-4mm}

\antiphona{VII c}{temporalia/ant-sitnomendomini.gtex}

\scriptura{Psalmus 112.}

\initiumpsalmi{temporalia/ps112-initium-vii-c-auto.gtex}

%\psalmusEtTranslatioT{temporalia/ps112-I-comb.tex}{10cm}
\input{temporalia/ps112-I.tex} \Abardot{}

\vfill
\pagebreak

\pars{Capitulum.} \scriptura{2 Cor. 1, 3-4}

\grechangedim{interwordspacetext}{0.12 cm plus 0.15 cm minus 0.05 cm}{scalable}%
\cuminitiali{}{temporalia/capitulum-BenedictusDeus.gtex}
\grechangedim{interwordspacetext}{0.22 cm plus 0.15 cm minus 0.05 cm}{scalable}

% preklad Jeruz. bible
%\trCapituliI

\vfill

\pars{Responsorium breve.} \scriptura{Ps. 103, 24}

\cuminitiali{VI}{temporalia/resp-quammagnificata.gtex}

%\trResp

\vfill
\pagebreak

\pars{Hymnus} \scriptura{Gregorius Magnus (\olddag{} 604)}

\cuminitiali{I}{temporalia/hym-LucisCreator-aestivalis.gtex}
\vspace{-3mm}
%\begin{translatioMulticol}{3}
Tvůrce světa předobrý,\\
tys ustanovil denní řád\\
a proudy světla rozhodil,\\
když světu základy jsi klad.\\
\\
A spojils ráno s večerem\\
a dnem tu dobu nazýváš;\\
hle padá temné noci stín -\\
slyš prosbu, vyslyš nářek náš.\columnbreak

Ach, nedej, by nás stihla smrt,\\
když svědomí nám tíží hřích,\\
když nemyslíme na věčnost\\
v té síti hříchů šalebných.\\
\\
Vzbuď naši touhu po nebi,\\
kde věčný život čeká nás,\\
a pomoz odložit vše zlé\\
a smýti z duše každý kaz.\columnbreak

To splň nám, dobrý Otče náš,\\
i ty, jenž rovné božství máš,\\
i Duchu, který těšíš nás\\
a vládneš, Bože, v každý čas.\\
Amen. 
\end{translatioMulticol}


\vfill
%\pagebreak

\pars{Versus.} \scriptura{Ps. 140, 2}

% Versus. %%%
\sineinitiali{temporalia/versus-dirigatur.gtex}

%\noindent \trVersus

\vfill
\pagebreak

\magnificatii

\vfill
\pagebreak

%\sideThumbs{{\scriptsize{}Fine horarum}}

\anteOrationem

\pagebreak

% Oratio. %%%
\oratioLaudes

\vspace{-1mm}
%\trOrationisI

\vfill

\rubrica{Hebdomadarius dicit iterum Dominus vobiscum, vel cantor dicit:}

\vspace{2mm}

\sineinitiali{temporalia/domineexaudi.gtex}

\rubrica{Postea cantatur a cantore:}

\vspace{2mm}

\cuminitiali{I}{temporalia/benedicamus-dominica-perannum.gtex}

\vspace{1mm}

\end{document}

