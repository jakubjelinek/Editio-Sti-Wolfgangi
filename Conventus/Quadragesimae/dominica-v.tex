\newcommand{\titulus}{\nomenFesti{Dominica V in Quadragesima.}
\nomenFesti{Dominica de Passione.}
%\celebratio{1. Classis. Semiduplex}
}
\newcommand{\tempquad}{Tempore Passionis}
\newcommand{\oratio}{\pars{Oratio.}

\noindent Quǽsumus, Dómine Deus noster, ut in illa caritáte, qua Fílius tuus díligens mundum morti se trádidit, inveniámur ipsi, te opitulánte, alácriter ambulántes.

\pars{Pro pace in Ucraina.} \scriptura{Sir. 50, 25; 2 Esdr. 4, 20; \textbf{H416}}

\vspace{-4mm}

\antiphona{II D}{temporalia/ant-dapacemdomine.gtex}

\vfill

\noindent Deus, a quo sancta desidéria, recta consília et iusta sunt ópera: da servis tuis illam, quam mundus dare non potest, pacem; ut et corda nostra mandátis tuis dédita, et hóstium subláta formídine, témpora sint tua protectióne tranquílla.

\noindent Per Dóminum nostrum Iesum Christum, Fílium tuum, qui tecum vivit et regnat in unitáte Spíritus Sancti, Deus, per ómnia sǽcula sæculórum.

\noindent \Rbardot{} Amen.}
\newcommand{\capitulum}{\pars{Capitulum.} \scriptura{Heb. 9, 11-12}

\grechangedim{interwordspacetext}{0.12 cm plus 0.15 cm minus 0.05 cm}{scalable}%
\cuminitiali{}{temporalia/capitulum-FratresChristus.gtex}
\grechangedim{interwordspacetext}{0.22 cm plus 0.15 cm minus 0.05 cm}{scalable}}
\newcommand{\magnificati}{\pars{Canticum B.  Mariæ V.} \scriptura{Io. 8, 18; \textbf{H164}}

\vspace{-4mm}

{
\grechangedim{interwordspacetext}{0.18 cm plus 0.15 cm minus 0.05 cm}{scalable}%
\antiphona{VIII G}{temporalia/ant-egosumquitestimonium.gtex}
\grechangedim{interwordspacetext}{0.22 cm plus 0.15 cm minus 0.05 cm}{scalable}%
}

%\vspace{-3mm}

\scriptura{Lc. 1, 46-55}

%\vspace{-2mm}

\cantusSineNeumas
\initiumpsalmi{temporalia/magnificat-initium-viiisoll-G.gtex}

%\vspace{-1.5mm}

\input{temporalia/magnificat-viiisoll-G.tex} \Abardot{}}
\newcommand{\invitatorium}{\pars{Invitatorium.} \scriptura{Ps. 94, 8; Psalmus 94; \textbf{H143}}

\vspace{-4mm}

\antiphona{IV* c}{temporalia/inv-hodiesivocem-CROCHU.gtex}}
\iftrue
\newcommand{\hymnusmatutinum}{\pars{Hymnus.} \scriptura{Venantius Fortunatus (sæc. VI)}

\vspace{-5mm}

\antiphona{I}{temporalia/hym-PangeLingua.gtex}}
\else
\newcommand{\hymnusmatutinum}{\pars{Hymnus.} \scriptura{Gregorius Magnus (\olddag{} 604)}

\vspace{-5mm}

\antiphona{I}{temporalia/hym-ExMore.gtex}}
\fi
\newcommand{\nocturnoi}{\pars{Psalmus 1.} \scriptura{Ps. 1, 2; \textbf{H372}}

\vspace{-4mm}

\antiphona{I g}{temporalia/ant-inlegedomini.gtex}

%\vspace{-2mm}

\scriptura{Ps. 1}

%\vspace{-2mm}

\initiumpsalmi{temporalia/ps1-initium-i-g-auto.gtex}

\input{temporalia/ps1-i-g.tex} \Abardot{}

\vfill
\pagebreak

\pars{Psalmus 2.} \scriptura{Ps. 2, 6}

\vspace{-4mm}

\antiphona{I a\textsuperscript{2}}{temporalia/ant-praedicanspraeceptum.gtex}

\vspace{-1mm}

\scriptura{Ps. 2}

\vspace{-2mm}

\initiumpsalmi{temporalia/ps2-initium-i-a4-auto.gtex}

\input{temporalia/ps2-i-a4.tex} \Abardot{}

\vfill
\pagebreak

\pars{Psalmus 3.} \scriptura{Ps. 3, 4-5; \textbf{H376}}

\vspace{-4mm}

\antiphona{VIII G\textsuperscript{2}}{temporalia/ant-tuesgloriamea.gtex}

%\vspace{-2mm}

\scriptura{Ps. 3}

\initiumpsalmi{temporalia/ps3-initium-viii-G5-auto.gtex}

\input{temporalia/ps3-viii-G5.tex} \Abardot{}

\vfill
\pagebreak
}
\newcommand{\nocturnoii}{\vspace{-4mm}

\pars{Psalmus 4.} \scriptura{Ps. 102, 1; \textbf{H99}}

\vspace{-4mm}

\antiphona{VIII c}{temporalia/ant-benedicanimamea.gtex}

%\vspace{-2mm}

\scriptura{Ps. 102, 1-7}

%\vspace{-2mm}

\initiumpsalmi{temporalia/ps102i-initium-viii-C-auto.gtex}

\input{temporalia/ps102i-viii-C.tex} \Abardot{}

\vfill
\pagebreak

\pars{Psalmus 5.} \scriptura{Ps. 102, 11}

\vspace{-4mm}

\antiphona{I d}{temporalia/ant-supertimentesdominum.gtex}

%\vspace{-2mm}

\scriptura{Ps. 102, 8-16}

%\vspace{-2mm}

\initiumpsalmi{temporalia/ps102ii-initium-i-d-auto.gtex}

\input{temporalia/ps102ii-i-d.tex} \Abardot{}

\vfill
\pagebreak

\pars{Psalmus 6.} \scriptura{Ps. 102, 20; \textbf{H332}}

\vspace{-4mm}

\antiphona{III g}{temporalia/ant-benedicitedomino.gtex}

%\vspace{-5mm}

\scriptura{Ps. 102, 17-22}

%\vspace{-2mm}

\initiumpsalmi{temporalia/ps102iii-initium-iii-g.gtex}

\input{temporalia/ps102iii-iii-g.tex} \Abardot{}

\vfill
\pagebreak}
\newcommand{\nocturnoiii}{\pars{Cantica.} \scriptura{Ier. 14, 19.20}

\vspace{-4mm}

\antiphona{I d}{temporalia/ant-sustinuimuspacem.gtex}

%\vspace{-2mm}

\scriptura{Canticum Ieremiæ, Ier. 14, 17-21}

%\vspace{-2mm}

\initiumpsalmi{temporalia/jeremiae2-initium-i-d.gtex}

\input{temporalia/jeremiae2-i-d.tex} \hfill \rubrica{Hic non dicitur antiphona.}

\vfill
\pagebreak

\scriptura{Canticum Ezechiæ, Ez. 36, 24-28}

%\vspace{-2mm}

\initiumpsalmi{temporalia/ezechiae2-initium-i-d-auto.gtex}

\input{temporalia/ezechiae2-i-d.tex}

\vfill
\pagebreak

\scriptura{Canticum, Lam. 5, 1-7.15-17.19-21}

%\vspace{-2mm}

\initiumpsalmi{temporalia/lamentatio-initium-i-d-auto.gtex}

\input{temporalia/lamentatio-i-d.tex}

\vfill
\pagebreak

\antiphona{}{temporalia/ant-sustinuimuspacem.gtex}

\vfill
\pagebreak}
\newcommand{\matversusi}{\pars{Versus.}

\noindent \Vbardot{} Erue a frámea, Deus, ánimam meam.

\noindent \Rbardot{} Et de manu canis únicam meam.}
\newcommand{\matversusii}{\pars{Versus.}

\noindent \Vbardot{} De ore leónis líbera me, Dómine.

\noindent \Rbardot{} Et a córnibus unicórnium humilitátem meam.}
\newcommand{\lectioi}{\pars{Lectio I.} \scriptura{Ier. 1, 1-6}

\noindent Incipit liber Ieremíæ Prophétæ.

\noindent Verba Ieremíæ fílii Helcíæ, de sacerdótibus, qui fuérunt in Anathoth, in terra Béniamin. Quod factum est verbum Dómini ad eum in diébus Iosíæ fílii Amon regis Iuda, in tertiodécimo anno regni eius. Et factum est in diébus Ióakim fílii Iosíæ regis Iuda, usque ad consummatiónem undécimi anni Sedecíæ fílii Iosíæ regis Iuda, usque ad transmigratiónem Ierúsalem, in mense quinto. Et factum est verbum Dómini ad me, dicens: Priúsquam te formárem in útero, novi te: et ántequam exíres de vulva, sanctificávi te, et prophétam in géntibus dedi te. Et dixi: A, a, a, Dómine Deus: ecce néscio loqui, quia puer ego sum.}
\newcommand{\responsoriumi}{\pars{Responsorium 1.} \scriptura{\Rbardot{} Lv. 23, 4; \textbf{H164}}

\vspace{-5mm}

\responsorium{VII}{temporalia/resp-istisuntdies-CROCHU.gtex}{}}
\newcommand{\lectioii}{\pars{Lectio II.} \scriptura{Ier. 1, 7-13}

\noindent Et dixit Dóminus ad me: Noli dícere: Puer sum: quóniam ad ómnia, quæ mittam te, ibis: et univérsa, quæcúmque mandávero tibi, loquéris. Ne tímeas a fácie eórum: quia tecum ego sum, ut éruam te, dicit Dóminus. Et misit Dóminus manum suam, et tétigit os meum: et dixit Dóminus ad me: Ecce dedi verba mea in ore tuo: Ecce constítui te hódie super gentes, et super regna, ut evéllas, et déstruas, et dispérdas, et díssipes, et ædífices, et plantes. Et factum est verbum Dómini ad me, dicens: Quid tu vides, Ieremía? Et dixi: Virgam vigilántem ego vídeo. Et dixit Dóminus ad me: Bene vidísti, quia vigilábo ego super verbo meo, ut fáciam illud. Et factum est verbum Dómini secúndo ad me, dicens: Quid tu vides? Et dixi: Ollam succénsam ego vídeo, et fáciem eius a fácie Aquilónis.}
\newcommand{\responsoriumii}{\pars{Responsorium 2.} \scriptura{\Rbardot{} Ps. 3, 2.3.7 \Vbardot{} Ps. 12, 5; \textbf{H164}}

\vspace{-5mm}

\responsorium{VIII}{temporalia/resp-multiplicatisunt-CROCHU.gtex}{}}
\newcommand{\lectioiii}{\pars{Lectio III.} \scriptura{Ier. 1, 14-19}

\noindent Et dixit Dóminus ad me: Ab Aquilóne pandétur malum super omnes habitatóres terræ. Quia ecce ego convocábo omnes cognatiónes regnórum Aquilónis, ait Dóminus: et vénient et ponent unusquísque sólium suum in intróitu portárum Ierúsalem, et super omnes muros eius in circúitu, et super univérsas urbes Iuda. Et loquar iudícia mea cum eis super omnem malítiam eórum, qui dereliquérunt me, et libavérunt diis aliénis, et adoravérunt opus mánuum suárum. Tu ergo, accínge lumbos tuos, et surge, et lóquere ad eos ómnia quæ ego præcípio tibi. Ne formídes a fácie eórum: nec enim timére te fáciam vultum eórum. Ego quippe dedi te hódie in civitátem munítam, et in colúmnam férream, et in murum ǽreum, super omnem terram, régibus Iuda, princípibus eius, et sacerdótibus et pópulo terræ. Et bellábunt advérsum te, et non prævalébunt: quia ego tecum sum, ait Dóminus, ut líberem te.}
\newcommand{\responsoriumiii}{\pars{Responsorium 3.} \scriptura{\Rbardot{} Ps. 70, 10-12 \Vbardot{} ibid., 8-9; \textbf{H164}}

\vspace{-5mm}

\responsorium{I}{temporalia/resp-quicustodiebant-CROCHU.gtex}{}

\rubrica{Omittitur Versus \textnormal{Gloria Patri}, repetitur integrum Responsorium usque ad Versum.}}
\newcommand{\lectioiv}{\pars{Lectio IV.} \scriptura{Sermo 9 de Quadragesima}

\noindent Sermo sancti Leónis Papæ.

\noindent In ómnibus, dilectíssimi, solemnitátibus christiánis non ignorámus paschále sacraméntum esse præcípuum: cui condígne et cóngrue suscipiéndo, totíus quidem nos témporis institúta refórmant: sed devotiónem nostram præséntes vel máxime dies éxigunt, quos illi sublimíssimo divínæ misericórdiæ sacraménto scimus esse contíguos. In quibus mérito a sanctis Apóstolis per doctrínam Spíritus Sancti maióra sunt ordináta ieiúnia: ut per commúne consórtium crucis Christi, étiam nos áliquid in eo quod propter nos gessit, agerémus, sicut Apóstolus ait: Si compátimur, et conglorificábimur. Certa atque secúra est exspectátio promíssæ beatitúdinis, ubi est participátio Domínicæ passiónis.}
\newcommand{\responsoriumiv}{\pars{Responsorium 4.} \scriptura{\Rbardot{} Ps. 21, 11 \Vbardot{} ibid., 2; \textbf{H165}}

\vspace{-5mm}

\responsorium{VIII}{temporalia/resp-deusmeusestune-CROCHU.gtex}{}}
\newcommand{\lectiov}{\pars{Lectio V.}

\noindent Nemo est, dilectíssimi, cui per conditiónem témporis socíetas huius glóriæ denegétur, tamquam tranquíllitas pacis vácua sit occasióne virtútis. Apóstolus enim prædícat, dicens: Omnes qui pie volunt vívere in Christo, persecutiónem patiéntur: et ídeo numquam deest tribulátio persecutiónis, si numquam desit observántia pietátis. Dóminus enim in exhortatiónibus suis dicit: Qui non áccipit crucem suam, et séquitur me, non est me dignus. Nec dubitáre debémus, hanc vocem non solum ad discípulos Christi, sed ad cunctos fidéles, totámque Ecclésiam pertinére, quæ salutáre suum in his qui áderant, universáliter audiébat.}
\newcommand{\responsoriumv}{\pars{Responsorium 5.} \scriptura{\Rbardot{} Ps. 21, 11 \Vbardot{} ibid., 21; \textbf{H166}}

\vspace{-5mm}

\responsorium{II}{temporalia/resp-inteiactatussum-CROCHU.gtex}{}}
\newcommand{\lectiovi}{\pars{Lectio VI.}

\noindent Sicut ergo totíus est córporis pie vívere, ita totíus est témporis crucem ferre: quæ mérito ferri unicuíque suadétur, quia própriis modis atque mensúris ab unoquóque tolerátur. Unum nomen est persecutiónis, sed non una est causa certáminis: et plus plerúmque perículi est in insidiatóre occúlto, quam in hoste manifésto. Beátus Iob alternántibus bonis ac malis mundi huius erudítus, pie veracitérque dicébat: Nonne tentátio est vita hóminis super terram? Quóniam non solis dolóribus córporis atque supplíciis ánima fidélis impétitur, verum étiam, salva incolumitáte membrórum, gravi morbo urgétur, si carnis voluptáte mollítur. Sed cum caro concupíscit advérsus spíritum, spíritus autem advérsus carnem; præsídio crucis Christi mens rationális instrúitur, nec cupiditátibus nóxiis illécta conséntit, quóniam continéntiæ clavis et Dei timóre transfígitur.}
\newcommand{\responsoriumvi}{\pars{Responsorium 6.} \scriptura{\Rbardot{} Ps. 21, 11.17.21 \Vbardot{} ibid., 2; \textbf{H166}}

\vspace{-5mm}

\responsorium{VII}{temporalia/resp-inproximoest-CROCHU.gtex}{}

\rubrica{Omittitur Versus \textnormal{Gloria Patri}, repetitur integrum Responsorium usque ad Versum.}}
\newcommand{\evangelium}{\pars{Versus.}

\noindent \Vbardot{} Ne perdas cum ímpiis, Deus, ánimam meam.

\noindent \Rbardot{} Et cum viris sánguinum vitam meam.

\vspace{5mm}

\sineinitiali{temporalia/oratiodominica-mat.gtex}

\vspace{5mm}

\pars{Absolutio.}

\cuminitiali{}{temporalia/absolutio-avinculis.gtex}

\vfill
\pagebreak

\cuminitiali{}{temporalia/benedictio-solemn-evangelica.gtex}

\vspace{7mm}

\pars{Evangelium} \scriptura{Io. 8, 1-11}
 
\noindent Léctio sancti Evangélii secúndum Ioánnem.
 
\noindent In illo témpore, perréxit Iesus in montem Olivéti. Dilúculo autem íterum venit in templum, et omnis pópulus veniébat ad eum, et sedens docébat eos. Addúcunt autem scribæ et pharisǽi mulíerem in adultério deprehénsam et statuérunt eam in médio et dicunt ei: «Magíster, hæc múlier manifésto deprehénsa est in adultério. In lege autem Móyses mandávit nobis huiúsmodi lapidáre; tu ergo quid dicis?». Hoc autem dicébant tentántes eum, ut possent accusáre eum.
  
\noindent Iesus autem inclínans se deórsum dígito scribébat in terra. Cum autem perseverárent interrogántes eum, eréxit se et dixit eis: «Qui sine peccáto est vestrum, primus in illam lápidem mittat»; et íterum se inclínans scribébat in terra. Audiéntes autem unus post unum exíbant, incipiéntes a senióribus, et remánsit solus, et múlier in médio stans.

\noindent Erigens autem se Iesus dixit ei: «Múlier, ubi sunt? Nemo te condemnávit?».

\noindent Quæ dixit: «Nemo, Dómine».

\noindent Dixit autem Iesus: «Nec ego te condémno; vade et ámplius iam noli peccáre».

\vspace{5mm}

\scriptura{Tract. 33,4-5: CCL 36, 307-309}

\noindent Ex Tractátibus sancti Augustíni epíscopi in Ioánnem.

\noindent {\color{gray} Nunc iam atténdite ubi ab inimícis tentáta sit Dómini mansuetúdo. \emph{Addúcunt autem illi scribæ et pharisǽi mulíerem in adultério deprehénsam, tentántes eum, ut possent accusáre eum.} Unde accusáre? Quid est ergo: \emph{tentántes eum ut possent accusáre eum?} Intellegémus, fratres, admirábilem mansuetúdinem in Dómino præeminuísse. Animadvertérunt eum nímium esse mitem, nímium esse mansuétum; de illo quippe fúerat ante prædíctum: \emph{Accíngere gládio tuo circa femur tuum, potentíssime; spécie tua et pulchritúdine tua inténde, próspere procéde, et regna propter veritátem et mansuetúdinem et iustítiam.} Ergo áttulit veritátem ut doctor, mansuetúdinem ut liberátor, iustítiam ut cógnitor.

\noindent Propter hæc eum esse regnatúrum in Spíritu Sancto prophéta prædíxerat. Cum loquerétur, véritas agnoscebátur; cum advérsus inimícos non moverétur, mansuetúdo laudabátur. Cum ergo de duóbus istis, id est de veritáte et mansuetúdine eius, inimíci livóre et invídia torqueréntur, in tértio, id est iustítia, scándalum posuérunt. Quare? Quia lex iússerat adúlteros lapidári; et útique lex quod iniústum erat iubére non póterat; si quis áliud díceret quam lex iússerat, iniústus deprehenderétur.}

\noindent Dixérunt ergo apud semetípsos: «Verax putátur, mansuétus vidétur, de iustítia illi quærénda calúmnia est; offerámus ei mulíerem in adultério deprehénsam, dicámus quid de illa in lege præcéptum sit: si eam iússerit lapidári, mansuetúdinem non habébit; si eam dimítti censúerit, iustítiam non tenébit.» Posset his verbis atque his senténtiis inflammári invídia, fervére accusátio, flagitári damnátio. Sed cui hoc? Pervérsitas rectitúdini, fálsitas veritáti, corrúptum cor cordi recto, stultítia sapiéntiæ.

\noindent Quid respóndit Dóminus Iesus? Quid respóndit véritas? Quid respóndit sapiéntia? Quid respóndit ipsa cui calúmnia parabátur iustítia? Vidéte quam plenum sit iustítia, plenum mansuetúdine et veritáte! \emph{Qui sine peccáto est vestrum, prior in illam lápidem mittat.} O respónsio sapiéntiæ! Hæc vox iustítiæ est: puniátur peccátrix, sed non a peccatóribus; impleátur lex, sed non a prævaricatóribus legis. Hæc vox omníno iustítiæ est; qua iustítia illi tamquam trabáli percússi, sese inspiciéntes et reos inveniéntes, \emph{unus post unum omnes recessérunt.} Relícti sunt duo: mísera et misericórdia.

\vfill
\pagebreak

\pars{Responsorium 7.} \scriptura{\Rbardot{} Ps. 118, 114 \Vbardot{} Ps. 58, 2; \textbf{H165}}

\vspace{-5mm}

\responsorium{IV}{temporalia/resp-adiutoretsusceptormeus-CROCHU.gtex}{}

\rubrica{Omittitur Versus \textnormal{Gloria Patri}, repetitur integrum Responsorium usque ad Versum.}

\vfill
\pagebreak
}
\newcommand{\laudes}{\pars{Psalmus 1.} \scriptura{Ps. 62, 2; \textbf{H138}}

\vspace{-4mm}

\antiphona{I f}{temporalia/ant-deusdeusmeus.gtex}

%\vspace{-2mm}

\scriptura{Psalmus 62}

%\vspace{-2mm}

\initiumpsalmi{temporalia/ps62-initium-i-f-auto.gtex}

%\vspace{-1.5mm}

\input{temporalia/ps62-i-f.tex} \Abardot{}

\vfill
\pagebreak

\pars{Psalmus 2.} \scriptura{Dn. 3, 43; \textbf{H416}}

\vspace{-4mm}

\antiphona{VIII G}{temporalia/ant-liberanosdeus.gtex}

\scriptura{Canticum trium puerorum, Dan. 3, 57-88 et 56}

\initiumpsalmi{temporalia/dan3-initium-viii-G-auto.gtex}

\input{temporalia/dan3-viii-G-sinedox.tex}

\rubrica{Hic non dicitur Gloria Patri, neque Amen.}

\vfill

\vspace{-6mm}

\antiphona{}{temporalia/ant-liberanosdeus.gtex} % repeat the antiphon - new page

\vfill
\pagebreak

\pars{Psalmus 3.} \scriptura{Ps. 149, 8; \textbf{H177}}

\vspace{-4mm}

\antiphona{I g\textsuperscript{3}}{temporalia/ant-alligadomine.gtex}

\scriptura{Psalmus 149.}

\vspace{-2mm}

\initiumpsalmi{temporalia/ps149-initium-i-g3-auto.gtex}

\vspace{-1.5mm}

\input{temporalia/ps149-i-g3.tex} \Abardot{}

\vfill
\pagebreak}
\newcommand{\lectiobrevis}{\pars{Lectio brevis.} \scriptura{Lv. 23, 4-7}

\noindent Hæ sunt ergo fériæ Dómini, convéntus sancti, quas celebráre debétis tempóribus suis. Mense primo, quarta décima die mensis ad vésperum Pascha Dómini est. Et quinta décima die mensis huius sollémnitas Azymórum Dómini est. Septem diébus ázyma comedétis. Die primo erit vobis convéntus sanctus; omne opus servíle non faciétis in eo.}
\iftrue
\newcommand{\responsoriumbreve}{\pars{Responsorium breve.} \scriptura{Ps. 21, 21}

\cuminitiali{IV}{temporalia/resp-erueaframea.gtex}}
\newcommand{\hymnuslaudes}{\pars{Hymnus} \scriptura{Venantius Fortunatus (sæc. VI)}

\vspace{-2mm}

\cuminitiali{I}{temporalia/hym-HicAccetum.gtex}}
\else
\newcommand{\responsoriumbreve}{\pars{Responsorium breve.}

\cuminitiali{IV}{temporalia/resp-christefilidei-tq.gtex}}
\newcommand{\hymnuslaudes}{\pars{Hymnus} \scriptura{Gregorius Magnus (\olddag{} 604)}

\cuminitiali{II}{temporalia/hym-PrecemurOmnes.gtex}}
\fi
\newcommand{\benedictus}{\pars{Canticum Zachariæ.} \scriptura{Io. 8, 6.7; \textbf{H158}}

\vspace{-4mm}

{
\grechangedim{interwordspacetext}{0.18 cm plus 0.15 cm minus 0.05 cm}{scalable}%
\antiphona{I g}{temporalia/ant-inclinavitseiesus.gtex}
\grechangedim{interwordspacetext}{0.22 cm plus 0.15 cm minus 0.05 cm}{scalable}%
}

\vspace{-2mm}

\scriptura{Lc. 1, 68-79}

\vspace{-2mm}

\initiumpsalmi{temporalia/benedictus-initium-i-g-auto.gtex}

%\vspace{-1mm}

\input{temporalia/benedictus-i-g.tex} \Abardot{}}
\newcommand{\magnificatii}{\pars{Canticum B. Mariæ V.} \scriptura{Io. 8, 56; \textbf{H167}}

\vspace{-4mm}

{
\grechangedim{interwordspacetext}{0.18 cm plus 0.15 cm minus 0.05 cm}{scalable}%
\antiphona{II D}{temporalia/ant-abrahampatervester.gtex}
\grechangedim{interwordspacetext}{0.22 cm plus 0.15 cm minus 0.05 cm}{scalable}%
}

\vspace{-2mm}

\scriptura{Lc. 1, 46-55}

\vspace{-2mm}

\cantusSineNeumas
\initiumpsalmi{temporalia/magnificat-initium-iisoll-D.gtex}

\vspace{-1.5mm}

\input{temporalia/magnificat-iisoll-D.tex} \Abardot{}}
\newcommand{\hebdomada}{infra Hebdom. V post Pentecosten.}
\newcommand{\oratioLaudes}{\cuminitiali{}{temporalia/oratio5.gtex}}

% LuaLaTeX

\documentclass[a4paper, twoside, 12pt]{article}
\usepackage[latin]{babel}
%\usepackage[landscape, left=3cm, right=1.5cm, top=2cm, bottom=1cm]{geometry} % okraje stranky
%\usepackage[landscape, a4paper, mag=1166, truedimen, left=2cm, right=1.5cm, top=1.6cm, bottom=0.95cm]{geometry} % okraje stranky
\usepackage[landscape, a4paper, mag=1400, truedimen, left=0.5cm, right=0.5cm, top=0.5cm, bottom=0.5cm]{geometry} % okraje stranky

\usepackage{fontspec}
\setmainfont[FeatureFile={junicode.fea}, Ligatures={Common, TeX}, RawFeature=+fixi]{Junicode}
%\setmainfont{Junicode}

% shortcut for Junicode without ligatures (for the Czech texts)
\newfontfamily\nlfont[FeatureFile={junicode.fea}, Ligatures={Common, TeX}, RawFeature=+fixi]{Junicode}

\usepackage{multicol}
\usepackage{color}
\usepackage{lettrine}
\usepackage{fancyhdr}

% usual packages loading:
\usepackage{luatextra}
\usepackage{graphicx} % support the \includegraphics command and options
\usepackage{gregoriotex} % for gregorio score inclusion
\usepackage{gregoriosyms}
\usepackage{wrapfig} % figures wrapped by the text
\usepackage{parcolumns}
\usepackage[contents={},opacity=1,scale=1,color=black]{background}
\usepackage{tikzpagenodes}
\usepackage{calc}
\usepackage{longtable}
\usetikzlibrary{calc}

\setlength{\headheight}{14.5pt}

\input{conventuscommune.tex} % Often used macros
%%%% Preklady jednotlivych zpevu (nektere se opakuji, a je dobre mit je
% vsechny na jedne hromade)

% HOURS ---

\newcommand{\trAntI}{\translatioCantus{Muž boží měl kožený toulec, pečlivě
zavázaný, jenž mu visel na šíji a~často se ho dotýkal.}}

\newcommand{\trAntII}{\translatioCantus{Klíč od~něho tak dobře střežil, že
dokud žil v~těle, nikdo z~jeho žáků nezvěděl, co je uvnitř.}}

\newcommand{\trAntIII}{\translatioCantus{Ale když se odebral z~tohoto
života, schránku otevřeli a~objevili v~ní žíněné roucho a~měděný řetěz
potřísněný krví.}}

\newcommand{\trAntIV}{\translatioCantus{A když prohlédli mistrovo tělo,
nalezli jeho tělo na čtyřech místech hluboce zbrázděno ranami od řetězu.}}

\newcommand{\trAntV}{\translatioCantus{Krev vytékající z~těch ran, místy
prostoupila i~žíněným rouchem.}}

\newcommand{\trCapituli}{\translatioCantus{
Miláčkovi Boha a~lidí,
Mojžíšovi požehnané paměti,~\gredagger{}
dopřál slávu rovnou slávě svatých~\grestar{}
učinil ho mocným na postrach nepřátelům
a~jeho slovy zastavil divy.}}

\newcommand{\trLectioBrevis}{\translatioCantus{
Pamatujte na své představené,
kteří vám hlásali Boží slovo.
Uvažte, jak oni skončili život, a~napodobujte jejich víru.
Ježíš Kristus je stejný včera i~dnes i~navěky.
Nenechte se svést věelijakými cizími naukami.}}

\newcommand{\trRespLaud}{\translatioCantus{Spravedlivého vodil Hospodin~\grestar{}
po přímých stezkách. \Vbardot{} A~ukázal mu Boží království.}}

\newcommand{\trRespLaudB}{\translatioCantus{Na tvých hradbách, Jeruzaléme,
ustanovil jsem strážné;~\grestar{}
budou bdít nad mým lidem. \Vbardot{} Ani ve dne, ani v~noci nesmějí nikdy
mlčet.}}

\newcommand{\trVersus}{\translatioCantus{\Vbardot{} Ústa spravedlivého šeptají moudrost, aleluja.
\Rbardot{} A~jeho jazyk ohlašuje právo, aleluja.}}

\newcommand{\trAntBenedictus}{\translatioCantus{Když na bujné oře vložili
nosítka a~sňali jim uzdu, vydali se přímo k~cele božího muže.}}

\newcommand{\trPreces}{\translatioCantus{
\noindent S vděčností chvalme Krista, dobrého Pastýře, \gredagger{} který dal život za své ovce, \grestar{} a~pokorně ho prosme: \Rbardot{} Pane, buď pastýřem svého lidu.

\noindent Kriste, ty dáváš církvi pastýře, a~jejich službou se ujímáš svého lidu, \grestar{} dej, ať v~lásce těch, kteří nás vedou, poznáváme, jak nás miluješ. \Rbardot{} Pane, buď pastýřem svého lidu.

\noindent Ty stále konáš skrze své zástupce službu pastýře a~učitele, \grestar{} nepřestávej nás nikdy vést prostřednictvím svých služebníků. \Rbardot{} Pane, buď pastýřem svého lidu.

\noindent Ty prokazuješ svému lidu skrze jeho pastýře službu lékaře duše i~těla, \grestar{} ochraňuj náš život a~veď nás ke svatosti. \Rbardot{} Pane, buď pastýřem svého lidu.

\noindent Ty posíláš své svaté, aby slovem i~příkladem vedli tvůj lid k~tobě, \grestar{} na jejich přímluvu nás posiluj, abychom vytrvali na cestě, která vede k~věčnému životu. \Rbardot{} Pane, buď pastýřem svého lidu.}}

\newcommand{\trOrationis}{\translatioCantus{Bože, jenž nám dopřáváš radovat
se z~výroční slavnosti svatého tvého vyznavače Havla, uděl dobrotivě,
abychom když slavíme jeho narození, též se řídili podobou jeho skutků.
Skrze…}}
 % Czech translations of the proper texts

\newcommand{\annusEditionis}{2020}

%%%% Vicekrat opakovane kousky

\newcommand{\anteOrationem}{
  \rubrica{Ante Orationem, cantatur a Superiore:}

  \pars{Supplicatio Litaniæ.}

  \cuminitiali{}{temporalia/supplicatiolitaniae.gtex}

  \pars{Oratio Dominica.}

  \cuminitiali{}{temporalia/oratiodominica.gtex}

  \rubrica{Deinde dicitur ab Hebdomadario:}

  \cuminitiali{}{temporalia/dominusvobiscum-solemnis.gtex}

  \rubrica{In choro monialium loco Dominus vobiscum dicitur:}

  \sineinitiali{temporalia/domineexaudi.gtex}
}

\setlength{\columnsep}{30pt} % prostor mezi sloupci

%%%%%%%%%%%%%%%%%%%%%%%%%%%%%%%%%%%%%%%%%%%%%%%%%%%%%%%%%%%%%%%%%%%%%%%%%%%%%%%%%%%%%%%%%%%%%%%%%%%%%%%%%%%%%
\begin{document}

% Here we set the space around the initial.
% Please report to http://home.gna.org/gregorio/gregoriotex/details for more details and options
\grechangedim{afterinitialshift}{2.2mm}{scalable}
\grechangedim{beforeinitialshift}{2.2mm}{scalable}
\grechangedim{interwordspacetext}{0.22 cm plus 0.15 cm minus 0.05 cm}{scalable}%
\grechangedim{annotationraise}{-0.2cm}{scalable}

% Here we set the initial font. Change 38 if you want a bigger initial.
% Emit the initials in red.
\grechangestyle{initial}{\color{red}\fontsize{38}{38}\selectfont}

\pagestyle{empty}

%%%% Titulni stranka
\begin{titulusOfficii}
\titulus{}
\end{titulusOfficii}

% graphic
%\vspace{1.5cm}
%\begin{center}
%\includegraphics[width=8cm]{emmaus.jpg}
%\end{center}

\vfill

\begin{center}
%Ad usum et secundum consuetudines chori \guillemotright{}Conventus Choralis\guillemotleft.

%Editio Sancti Wolfgangi \annusEditionis
\end{center}

\pagebreak

\renewcommand{\headrulewidth}{0pt} % no horiz. rule at the header
\fancyhf{}
\pagestyle{fancy}

\pars{Oratio ante divinum Officium.}

\lettrine{{\color{red}A}}{peri,} Dómine, os meum ad benedicéndum nomen sanctum tuum:
munda quoque cor meum ab ómnibus vanis, pervérsis, et aliénis
cogitatiónibus:
intelléctum illúmina, afféctum inflámma,
ut digne, atténte ac devóte hoc Offícium recitáre váleam,
et exaudíri mérear ante conspéctum Divínæ Maiestátis tuæ.
Per Christum, Dóminum nostrum.
\Rbardot{} Amen.

Dómine, in unióne illíus divínæ intentiónis,
qua ipse in terris laudes Deo persolvísti,
has tibi Horas \rubricatum{(vel \textnormal{hanc tibi Horam})} persólvo.

%\trOratioAnteOfficium

\vfill

\pars{Oratio post divinum Officium.}

\rubrica{
  Orationem sequentem devote post Officium recitantibus
  Leo Papa X. defectus, et culpas in eo persolvendo ex humana
  fragilitate contractas, indulsit, et dicitur flexis genibus.
}

\lettrine{{\color{red}S}}{acrosánctæ} et indivíduæ Trinitáti,
crucifíxi Dómini nostri Iesu Christi humanitáti,
beatíssimæ et gloriosíssimæ sempérque Vírginis Maríæ
fecúndæ integritáti, 
et ómnium Sanctórum universitáti
sit sempitérna laus, honor, virtus et glória
ab omni creatúra,
nobísque remíssio ómnium peccatórum,
per infiníta sǽcula sæculórum.
\Rbardot{} Amen.

\noindent \Vbardot{} Beáta víscera Maríæ Virginis, quæ portavérunt
ætérni Patris Fílium.\\
\Rbardot{} Et beáta úbera, quæ lactavérunt Christum Dominum.

\rubrica{Et dicitur secreto \textnormal{Pater noster.} et \textnormal{Ave María.}}

%\trOratioPostOfficium

\vfill

\hora{Ad I. Vesperas.} %%%%%%%%%%%%%%%%%%%%%%%%%%%%%%%%%%%%%%%%%%%%%%%%%%%%%
%\sideThumbs{I. Vesperæ}

\cantusSineNeumas

\vspace{0.5cm}
\grechangedim{interwordspacetext}{0.18 cm plus 0.15 cm minus 0.05 cm}{scalable}%
\cuminitiali{}{temporalia/deusinadiutorium-solemnis.gtex}
\grechangedim{interwordspacetext}{0.22 cm plus 0.15 cm minus 0.05 cm}{scalable}%

\vfill
\pagebreak

\pars{Psalmus 1.} \scriptura{Ps. 144, 13; \textbf{H100}}

\vspace{-4mm}

\antiphona{VII c\textsuperscript{2}}{temporalia/ant-regnumtuum.gtex}

\scriptura{Psalmus 144, 10-21.}

\initiumpsalmi{temporalia/ps144ii-initium-vii-c2-auto.gtex}

%\psalmusEtTranslatioT{temporalia/ps144ii-VII-comb.tex}{10cm}
\input{temporalia/ps144ii-VII.tex} \Abardot{}

\vspace{-1cm}

\vfill
\pagebreak

\pars{Psalmus 2.} \scriptura{Ps. 145, 2; \textbf{H100}}

\vspace{-4mm}

\antiphona{IV E}{temporalia/ant-laudabodeum.gtex}

\scriptura{Psalmus 145.}

\initiumpsalmi{temporalia/ps145-initium-iv-E-auto.gtex}

%\psalmusEtTranslatioT{temporalia/ps145-VII-comb.tex}{10cm}
\input{temporalia/ps145-VII.tex} \Abardot{}

\vfill
\pagebreak

\pars{Psalmus 3.} \scriptura{Ps. 146, 1; \textbf{H101}}

\vspace{-4mm}

\antiphona{VIII a}{temporalia/ant-deonostro.gtex}

\scriptura{Psalmus 146.}

\initiumpsalmi{temporalia/ps146-initium-viii-A-auto.gtex}

%\psalmusEtTranslatioT{temporalia/ps146-VII-comb.tex}{10cm}
\input{temporalia/ps146-VII.tex} \Abardot{}

\vfill
\pagebreak

\pars{Psalmus 4.} \scriptura{Ps. 147, 1}

\vspace{-4mm}

\antiphona{E}{temporalia/ant-laudajerusalem.gtex}

\scriptura{Psalmus 147.}

\initiumpsalmi{temporalia/ps147-initium-e-auto.gtex}

%\psalmusEtTranslatioT{temporalia/ps147-VII-comb.tex}{10cm}
\input{temporalia/ps147-VII.tex} \Abardot{}

\vfill
\pagebreak

\pars{Capitulum.} \scriptura{Rom. 11, 33}

\grechangedim{interwordspacetext}{0.12 cm plus 0.15 cm minus 0.05 cm}{scalable}%
\cuminitiali{}{temporalia/capitulum-OAltitudo.gtex}
\grechangedim{interwordspacetext}{0.22 cm plus 0.15 cm minus 0.05 cm}{scalable}

% preklad Jeruz. bible
%\trCapituliI

\vfill

\pars{Responsorium breve.} \scriptura{Ps. 146, 5}

\cuminitiali{VI}{temporalia/resp-magnusdominusnoster.gtex}

%\trResp

\vfill
\pagebreak

\pars{Hymnus} \scriptura{Ambrosius (\olddag{} 397)}

\cuminitiali{I}{temporalia/hym-OLuxBeata-aestivalis.gtex}
\vspace{-3mm}
%\input{hym-OLuxBeata-bohtext.tex}

\vfill
%\pagebreak

\pars{Versus.}

% Versus. %%%
\sineinitiali{temporalia/versus-vespertina.gtex}

%\noindent \trVersus

\vfill
\pagebreak

\magnificati

\vfill
\pagebreak

%\sideThumbs{{\scriptsize{}Fine horarum}}

\anteOrationem

\pagebreak

% Oratio. %%%
\oratioLaudes

\vspace{-1mm}
%\trOrationisI

\vfill

\rubrica{Hebdomadarius dicit iterum Dominus vobiscum, vel cantor dicit:}

\vspace{2mm}

\sineinitiali{temporalia/domineexaudi.gtex}

\rubrica{Postea cantatur a cantore:}

\vspace{2mm}

\cuminitiali{I}{temporalia/benedicamus-dominica-perannum.gtex}

\vspace{1mm}

\vfill
\pagebreak

\hora{Ad Matutinum.} %%%%%%%%%%%%%%%%%%%%%%%%%%%%%%%%%%%%%%%%%%%%%%%%%%%%%
%\sideThumbs{Matutinum}

\vspace{2mm}

\cuminitiali{}{temporalia/dominelabiamea.gtex}

\vspace{2mm}

\pars{Invitatorium.} \scriptura{Ps. 94, 1; Psalmus 94}

\vspace{-6mm}

\antiphona{E}{temporalia/inv-veniteexsultemus.gtex}

\vfill
\pagebreak

\pars{Hymnus.} \scriptura{Adamus Sancti Victoris (\olddag 1146)}

\vspace{-5mm}

\antiphona{VII}{temporalia/hym-SalveDies.gtex}

\scriptura{Non dicitur \textnormal{Amen} in fine.}
%{
%\vspace{-5mm}
%\setlength{\columnsep}{0pt} % prostor mezi sloupci
%\input{hym-SalveDies-bohtext.tex}
%\setlength{\columnsep}{30pt} % prostor mezi sloupci
%}

\vfill
\pagebreak

\subhora{In I. Nocturno}

\pars{Psalmus 1.} \scriptura{Ps. 1, 1}

\vspace{-4mm}

\antiphona{VIII G}{temporalia/ant-beatusvir.gtex}

%\vspace{-5mm}

\scriptura{Ps. 1}

%\vspace{-2mm}

\initiumpsalmi{temporalia/ps1-initium-viii-G-auto.gtex}

%\psalmusEtTranslatioT{temporalia/ps1-I-comb.tex}{10cm}
\input{temporalia/ps1-I.tex} \Abardot{}

\vfill
\pagebreak

\pars{Psalmus 2.} \scriptura{Ps. 2, 11; \textbf{H93}}

\vspace{-4mm}

\antiphona{VII a}{temporalia/ant-servitedomino.gtex}

\vspace{-3mm}

\scriptura{Ps. 2}

\vspace{-2mm}

\initiumpsalmi{temporalia/ps2-initium-vii-a-auto.gtex}

%\psalmusEtTranslatioT{temporalia/ps2-I-comb.tex}{10cm}
\input{temporalia/ps2-I.tex} \Abardot{}

\vfill
\pagebreak

\pars{Psalmus 3.} \scriptura{Ps. 3, 7}

\vspace{-4mm}

\antiphona{VI F}{temporalia/ant-exsurgedominesalvum.gtex}

%\vspace{-5mm}

\scriptura{Ps. 3}

\initiumpsalmi{temporalia/ps3-initium-vi-F-auto.gtex}

%\psalmusEtTranslatioT{temporalia/ps3-I-comb.tex}{10cm}
\input{temporalia/ps3-I.tex} \Abardot{}

\vfill
\pagebreak

\pars{Versus.} \scriptura{Ps. 118, 55}

% Versus. %%%
\sineinitiali{temporalia/versus-memorfui.gtex}

\vspace{5mm}

\sineinitiali{temporalia/oratiodominica-mat.gtex}

\vspace{5mm}

\pars{Absolutio.}

\cuminitiali{}{temporalia/absolutio-exaudi.gtex}

\vfill
\pagebreak

\cuminitiali{}{temporalia/benedictio-solemn-benedictione.gtex}

\vspace{7mm}

\lectioi

\noindent \Vbardot{} Tu autem, Dómine, miserére nobis.
\noindent \Rbardot{} Deo grátias.

\vfill
\pagebreak

\responsoriumi

\vfill
\pagebreak

\cuminitiali{}{temporalia/benedictio-solemn-unigenitus.gtex}

\vspace{7mm}

\lectioii

\noindent \Vbardot{} Tu autem, Dómine, miserére nobis.
\noindent \Rbardot{} Deo grátias.

\vfill
\pagebreak

\responsoriumii

\vfill
\pagebreak

\cuminitiali{}{temporalia/benedictio-solemn-spiritus.gtex}

\vspace{7mm}

\lectioiii

\noindent \Vbardot{} Tu autem, Dómine, miserére nobis.
\noindent \Rbardot{} Deo grátias.

\vfill
\pagebreak

\responsoriumiii

\vfill
\pagebreak

\subhora{In II. Nocturno}

\pars{Psalmus 4.} \scriptura{Ps. 8, 2}

\vspace{-4mm}

\antiphona{I g}{temporalia/ant-quamadmirabileest.gtex}

%\vspace{-5mm}

\scriptura{Ps. 8}

%A\vspace{-2mm}

\initiumpsalmi{temporalia/ps8-initium-i-g-auto.gtex}

%\psalmusEtTranslatioT{temporalia/ps8-I-comb.tex}{10cm}
\input{temporalia/ps8-I.tex} \Abardot{}

\vfill
\pagebreak

\pars{Psalmus 5.} \scriptura{Ps. 9, 5}

\vspace{-4mm}

\antiphona{VIII G}{temporalia/ant-sedistisuperthronum.gtex}

%\vspace{-5mm}

\scriptura{Ps. 9, 2-11}

\initiumpsalmi{temporalia/ps9ii_xi-initium-viii-G-auto.gtex}

%\psalmusEtTranslatioT{temporalia/ps9ii_xi-I-comb.tex}{10cm}
\input{temporalia/ps9ii_xi-I.tex} \Abardot{}

\vfill
\pagebreak

\pars{Psalmus 6.} \scriptura{Ps. 9, 20}

\vspace{-4mm}

\antiphona{I g\textsuperscript{3}}{temporalia/ant-exsurgedominenon.gtex}

%\vspace{-5mm}

\scriptura{Ps. 9, 12-21}

\initiumpsalmi{temporalia/ps9xii_xxi-initium-i-g3-auto.gtex}

%\psalmusEtTranslatioT{temporalia/ps9xii_xxi-I-comb.tex}{10cm}
\input{temporalia/ps9xii_xxi-I.tex} \Abardot{}

\vfill
\pagebreak

\pars{Versus.} \scriptura{Ps. 118, 62}

% Versus. %%%
\sineinitiali{temporalia/versus-medianocte.gtex}

\vspace{5mm}

\sineinitiali{temporalia/oratiodominica-mat.gtex}

\vspace{5mm}

\pars{Absolutio.}

\cuminitiali{}{temporalia/absolutio-ipsius.gtex}

\vfill
\pagebreak

\cuminitiali{}{temporalia/benedictio-solemn-deus.gtex}

\vspace{7mm}

\lectioiv

\noindent \Vbardot{} Tu autem, Dómine, miserére nobis.
\noindent \Rbardot{} Deo grátias.

\vfill
\pagebreak

\responsoriumiv

\vfill
\pagebreak

\cuminitiali{}{temporalia/benedictio-solemn-christus.gtex}

\vspace{7mm}

\lectiov

\noindent \Vbardot{} Tu autem, Dómine, miserére nobis.
\noindent \Rbardot{} Deo grátias.

\vfill
\pagebreak

\responsoriumv

\vfill
\pagebreak

\cuminitiali{}{temporalia/benedictio-solemn-ignem.gtex}

\vspace{7mm}

\lectiovi

\noindent \Vbardot{} Tu autem, Dómine, miserére nobis.
\noindent \Rbardot{} Deo grátias.

\vfill
\pagebreak

\responsoriumvi

\vfill
\pagebreak

\subhora{In III. Nocturno}

\pars{Psalmus 7.} \scriptura{Ps. 9, 22}

\vspace{-4mm}

\antiphona{II D}{temporalia/ant-utquiddomine.gtex}

\vspace{-4mm}

\scriptura{Ps. 9, 22-32}

%\vspace{-2mm}

\initiumpsalmi{temporalia/ps9xxii_xxxii-initium-ii-D-auto.gtex}

%\psalmusEtTranslatioT{temporalia/ps9xxii_xxxii-I-comb.tex}{10cm}
\input{temporalia/ps9xxii_xxxii-I.tex} \Abardot{}

\vfill
\pagebreak

\pars{Psalmus 8.}\scriptura{Ex. 15, 18}

\vspace{-4mm}

\antiphona{IV* e}{temporalia/ant-inaeternum.gtex}

%\vspace{-4mm}

\scriptura{Ps. 9, 33-39}

\initiumpsalmi{temporalia/ps9xxxiii_xxxix-initium-iv_-e-auto.gtex}

%\psalmusEtTranslatioT{temporalia/ps9xxxiii_xxxix-I-comb.tex}{10cm}
\input{temporalia/ps9xxxiii_xxxix-I.tex} \Abardot{}

\vfill
\pagebreak

\pars{Psalmus 9.} \scriptura{Ps. 10, 8}

\vspace{-4mm}

\antiphona{II* f}{temporalia/ant-justusdominus.gtex}

%\vspace{-4mm}

\scriptura{Ps. 10}

%\initiumpsalmi{temporalia/ps10-initium-iv-c-auto.gtex}
\initiumpsalmi{temporalia/ps10-initium-ii_-f.gtex}

%\psalmusEtTranslatioT{temporalia/ps10-I-comb.tex}{10cm}
\input{temporalia/ps10-I.tex} \Abardot{}

\vfill
\pagebreak

\pars{Versus.} \scriptura{Ps. 118, 148}

% Versus. %%%
\sineinitiali{temporalia/versus-praevenerunt.gtex}

\vspace{5mm}

\sineinitiali{temporalia/oratiodominica-mat.gtex}

\vspace{5mm}

\pars{Absolutio.}

\cuminitiali{}{temporalia/absolutio-avinculis.gtex}

\vfill
\pagebreak

\cuminitiali{}{temporalia/benedictio-solemn-evangelica.gtex}

\vspace{7mm}

\lectiovii

\noindent \Vbardot{} Tu autem, Dómine, miserére nobis.
\noindent \Rbardot{} Deo grátias.

\vfill
\pagebreak

\responsoriumvii

\vfill
\pagebreak

\cuminitiali{}{temporalia/benedictio-solemn-divinum.gtex}

\vspace{7mm}

\lectioviii

\noindent \Vbardot{} Tu autem, Dómine, miserére nobis.
\noindent \Rbardot{} Deo grátias.

\vfill
\pagebreak

\responsoriumviii

\vfill
\pagebreak

\cuminitiali{}{temporalia/benedictio-solemn-adsocietatem.gtex}

\vspace{7mm}

\lectioix

\noindent \Vbardot{} Tu autem, Dómine, miserére nobis.
\noindent \Rbardot{} Deo grátias.

\vfill
\pagebreak

% Te Deum

{
\pars{Hymnus Ambrosianus} \scriptura{Tonus Solemnis}

\vspace{-2mm}

\grechangedim{interwordspacetext}{0.26 cm plus 0.15 cm minus 0.05 cm}{scalable}%
\cuminitiali{III}{temporalia/tedeum-solemnis-gn.gtex}
\grechangedim{interwordspacetext}{0.22 cm plus 0.15 cm minus 0.05 cm}{scalable}%
}

\vfill
\pagebreak

\rubrica{Reliqua omittuntur, nisi Laudes separandæ sint.}

\pars{Oratio}

\noindent \Vbardot{} Dómine, exáudi oratiónem meam.

\noindent \Rbardot{} Et clamor meus ad te véniat.

Orémus:

\oratioLaudes

\vspace{7mm}

\pars{Conclusio}

\noindent \Vbardot{} Dómine, exáudi oratiónem meam.

\noindent \Rbardot{} Et clamor meus ad te véniat.

\noindent \Vbardot{} Benedicámus Dómino, allelúia, allelúia.

\noindent \Rbardot{} Deo grátias, allelúia, allelúia.

\noindent \Vbardot{} Fidélium ánimæ per misericórdiam Dei requiéscant in pace.

\noindent \Rbardot{} Amen.

\vfill
\pagebreak

\hora{Ad Laudes.} %%%%%%%%%%%%%%%%%%%%%%%%%%%%%%%%%%%%%%%%%%%%%%%%%%%%%
%\sideThumbs{Laudes}

\cantusSineNeumas

\vspace{0.5cm}
\grechangedim{interwordspacetext}{0.18 cm plus 0.15 cm minus 0.05 cm}{scalable}%
\cuminitiali{}{temporalia/deusinadiutorium-alter.gtex}
\grechangedim{interwordspacetext}{0.22 cm plus 0.15 cm minus 0.05 cm}{scalable}%

\vfill
%\pagebreak

\pars{Psalmus 1.}

\vspace{-4mm}

\antiphona{VI F}{temporalia/ant-alleluia1.gtex}

\scriptura{Psalmus 50.}

\initiumpsalmi{temporalia/ps50-initium-vi-F-auto.gtex}

%\psalmusEtTranslatioT{temporalia/ps50-I-comb.tex}{10cm}
\input{temporalia/ps50-I.tex}

\vfill
\pagebreak

\pars{Psalmus 2.}

\scriptura{Psalmus 117.}

\initiumpsalmi{temporalia/ps117-initium-vi-F-auto.gtex}

%\psalmusEtTranslatioT{temporalia/ps117-I-comb.tex}{10cm}
\input{temporalia/ps117-I.tex}

\vfill
\pagebreak

\pars{Psalmus 3.}

\scriptura{Psalmus 62.}

\initiumpsalmi{temporalia/ps62-initium-vi-F-auto.gtex}

%\psalmusEtTranslatioT{temporalia/ps62-I-comb.tex}{10cm}
\input{temporalia/ps62-I.tex}

\vfill

\vspace{-6mm}

\antiphona{}{temporalia/ant-alleluia1.gtex} % repeat the antiphon - new page

\vfill
\pagebreak

\pars{Psalmus 4.} \scriptura{Dan. 3, 22-26; \textbf{H422}}

\vspace{-4mm}

\antiphona{VIII G}{temporalia/ant-trespueri.gtex}

\scriptura{Canticum trium puerorum, Dan. 3, 57-88 et 56}

\initiumpsalmi{temporalia/dan3-initium-viii-G-auto.gtex}

%\psalmusEtTranslatioT{temporalia/dan3-comb.tex}{10cm}
\input{temporalia/dan3.tex}

\rubrica{Hic non dicitur Gloria Patri, neque Amen.}

\vfill

\vspace{-6mm}

\antiphona{}{temporalia/ant-trespueri.gtex} % repeat the antiphon - new page

\vfill
\pagebreak

\pars{Psalmus 5.}

\vspace{-4mm}

\antiphona{VIII G}{temporalia/ant-alleluia2.gtex}

\scriptura{Psalmus 148.}

\initiumpsalmi{temporalia/ps148-initium-viii-G-auto.gtex}

%\psalmusEtTranslatioT{temporalia/ps148-I-comb.tex}{10cm}
\input{temporalia/ps148-I.tex}

\rubrica{Hic non dicitur Gloria Patri.}

\vfill
\pagebreak

%
\scriptura{Psalmus 149.}

\initiumpsalmi{temporalia/ps149-initium-viii-G-auto.gtex}

%\psalmusEtTranslatioT{temporalia/ps149-I-comb.tex}{10cm}
\input{temporalia/ps149-I.tex}

\rubrica{Hic non dicitur Gloria Patri.}

\vfill
\pagebreak

%
\scriptura{Psalmus 150.}

\initiumpsalmi{temporalia/ps150-initium-viii-G-auto.gtex}

%\psalmusEtTranslatioT{temporalia/ps150-I-comb.tex}{10cm}
\input{temporalia/ps150-I.tex}

\vfill

\vspace{-6mm}

\antiphona{}{temporalia/ant-alleluia2.gtex} % repeat the antiphon - new page

\vfill
\pagebreak

\pars{Capitulum.} \scriptura{Ac. 7, 12}

\grechangedim{interwordspacetext}{0.12 cm plus 0.15 cm minus 0.05 cm}{scalable}%
\cuminitiali{}{temporalia/capitulum-Benedictio.gtex}
\grechangedim{interwordspacetext}{0.22 cm plus 0.15 cm minus 0.05 cm}{scalable}

% preklad Jeruz. bible
%\trCapituliI

\vfill

\pars{Responsorium breve.} \scriptura{Ps. 118, 36-37}

\cuminitiali{IV}{temporalia/resp-inclinacormeum.gtex}

%\trResp

\vfill
\pagebreak

\pars{Hymnus} \scriptura{Gregorius Magnus (\olddag{} 604)}

\cuminitiali{IV}{temporalia/hym-EcceJamNoctis.gtex}
\vspace{-3mm}
%\input{hym-EcceJamNocis-bohtext.tex}

\vfill
%\pagebreak

\pars{Versus.} \scriptura{Ps. 92, 1}

% Versus. %%%
\sineinitiali{temporalia/versus-dominusregnavit.gtex}

%\noindent \trVersus

\vfill
\pagebreak

\benedictus

\vspace{-1cm}

\vfill
\pagebreak

%\sideThumbs{{\scriptsize{}Fine horarum}}

\anteOrationem

\pagebreak

% Oratio. %%%
\oratioLaudes

\vspace{-1mm}
%\trOrationisI

\vfill

\rubrica{Hebdomadarius dicit iterum Dominus vobiscum, vel cantor dicit:}

\vspace{2mm}

\sineinitiali{temporalia/domineexaudi.gtex}

\rubrica{Postea cantatur a cantore:}

\vspace{2mm}

\cuminitiali{I}{temporalia/benedicamus-dominica-perannum.gtex}

\vspace{1mm}

\vfill
\pagebreak

\hora{Ad II. Vesperas.} %%%%%%%%%%%%%%%%%%%%%%%%%%%%%%%%%%%%%%%%%%%%%%%%%%%%%
%\sideThumbs{II. Vesperæ}

\cantusSineNeumas

%\vspace{0.5cm}
\grechangedim{interwordspacetext}{0.18 cm plus 0.15 cm minus 0.05 cm}{scalable}%
\cuminitiali{}{temporalia/deusinadiutorium-solemnis.gtex}
\grechangedim{interwordspacetext}{0.22 cm plus 0.15 cm minus 0.05 cm}{scalable}%

\vfill
%\pagebreak

\vspace{-2mm}

\pars{Psalmus 1.} \scriptura{Ps. 109, 1; \textbf{H91}}

\vspace{-4mm}

\antiphona{VII c\textsuperscript{2}}{temporalia/ant-dixitdominus.gtex}

\vspace{-4mm}

\scriptura{Psalmus 109.}

\initiumpsalmi{temporalia/ps109-initium-vii-c2-auto.gtex}

%\psalmusEtTranslatioT{temporalia/ps109-I-comb.tex}{10cm}
\input{temporalia/ps109-I.tex} \Abardot{}

\vspace{-1cm}

\vfill
\pagebreak

\pars{Psalmus 2.} \scriptura{Ps. 110, 8; \textbf{H91}}

\vspace{-4mm}

\antiphona{IV g}{temporalia/ant-fideliaomnia.gtex}

\scriptura{Psalmus 110.}

\initiumpsalmi{temporalia/ps110-initium-iv-g-auto.gtex}

%\psalmusEtTranslatioT{temporalia/ps110-I-comb.tex}{10cm}
\input{temporalia/ps110-I.tex} \Abardot{}

\vfill
\pagebreak

\pars{Psalmus 3.} \scriptura{Ps. 111, 1; \textbf{H92}}

\vspace{-4mm}

\antiphona{IV a}{temporalia/ant-inmandatis.gtex}

\scriptura{Psalmus 111.}

\initiumpsalmi{temporalia/ps111-initium-iv-a-auto.gtex}

%\psalmusEtTranslatioT{temporalia/ps111-I-comb.tex}{10cm}
\input{temporalia/ps111-I.tex} \Abardot{}

\vfill
\pagebreak

\pars{Psalmus 4.} \scriptura{Ps. 112, 2; \textbf{H92}}

\vspace{-4mm}

\antiphona{VII c}{temporalia/ant-sitnomendomini.gtex}

\scriptura{Psalmus 112.}

\initiumpsalmi{temporalia/ps112-initium-vii-c-auto.gtex}

%\psalmusEtTranslatioT{temporalia/ps112-I-comb.tex}{10cm}
\input{temporalia/ps112-I.tex} \Abardot{}

\vfill
\pagebreak

\pars{Capitulum.} \scriptura{2 Cor. 1, 3-4}

\grechangedim{interwordspacetext}{0.12 cm plus 0.15 cm minus 0.05 cm}{scalable}%
\cuminitiali{}{temporalia/capitulum-BenedictusDeus.gtex}
\grechangedim{interwordspacetext}{0.22 cm plus 0.15 cm minus 0.05 cm}{scalable}

% preklad Jeruz. bible
%\trCapituliI

\vfill

\pars{Responsorium breve.} \scriptura{Ps. 103, 24}

\cuminitiali{VI}{temporalia/resp-quammagnificata.gtex}

%\trResp

\vfill
\pagebreak

\pars{Hymnus} \scriptura{Gregorius Magnus (\olddag{} 604)}

\cuminitiali{I}{temporalia/hym-LucisCreator-aestivalis.gtex}
\vspace{-3mm}
%\begin{translatioMulticol}{3}
Tvůrce světa předobrý,\\
tys ustanovil denní řád\\
a proudy světla rozhodil,\\
když světu základy jsi klad.\\
\\
A spojils ráno s večerem\\
a dnem tu dobu nazýváš;\\
hle padá temné noci stín -\\
slyš prosbu, vyslyš nářek náš.\columnbreak

Ach, nedej, by nás stihla smrt,\\
když svědomí nám tíží hřích,\\
když nemyslíme na věčnost\\
v té síti hříchů šalebných.\\
\\
Vzbuď naši touhu po nebi,\\
kde věčný život čeká nás,\\
a pomoz odložit vše zlé\\
a smýti z duše každý kaz.\columnbreak

To splň nám, dobrý Otče náš,\\
i ty, jenž rovné božství máš,\\
i Duchu, který těšíš nás\\
a vládneš, Bože, v každý čas.\\
Amen. 
\end{translatioMulticol}


\vfill
%\pagebreak

\pars{Versus.} \scriptura{Ps. 140, 2}

% Versus. %%%
\sineinitiali{temporalia/versus-dirigatur.gtex}

%\noindent \trVersus

\vfill
\pagebreak

\magnificatii

\vfill
\pagebreak

%\sideThumbs{{\scriptsize{}Fine horarum}}

\anteOrationem

\pagebreak

% Oratio. %%%
\oratioLaudes

\vspace{-1mm}
%\trOrationisI

\vfill

\rubrica{Hebdomadarius dicit iterum Dominus vobiscum, vel cantor dicit:}

\vspace{2mm}

\sineinitiali{temporalia/domineexaudi.gtex}

\rubrica{Postea cantatur a cantore:}

\vspace{2mm}

\cuminitiali{I}{temporalia/benedicamus-dominica-perannum.gtex}

\vspace{1mm}

\end{document}

