\newcommand{\oratio}{\pars{Oratio.}

\noindent Inchoáta pæniténtiæ ópera, quǽsumus, Dómine, benígno favóre proséquere, ut observántiam, quam corporáliter exercémus, méntibus étiam valeámus implére sincéris.

\pars{Pro commemoratione Sanctarum Perpetuæ \& Felicitatis, Martyrum.} \scriptura{Lc. 12, 4; \textbf{H364}}

\vspace{-4mm}

\antiphona{III a}{temporalia/ant-dicoautemvobis.gtex}

\vfill

\noindent Deus, cuius urgénte caritáte, beátæ mártyres Perpétua et Felícitas torméntum mortis, contémpto persecutóre, vicérunt, da nobis, quǽsumus, eárum précibus, ut in tua semper dilectióne crescámus.

\pars{Pro pace in universo mundo.} \scriptura{Sir. 50, 25; 2 Esdr. 4, 20; \textbf{H416}}

\vspace{-4mm}

\antiphona{II D}{temporalia/ant-dapacemdomine.gtex}

\vfill

\noindent Deus, a quo sancta desidéria, recta consília et iusta sunt ópera: da servis tuis illam, quam mundus dare non potest, pacem; ut et corda nostra mandátis tuis dédita, et hóstium subláta formídine, témpora sint tua protectióne tranquílla.

\noindent Per Dóminum nostrum Iesum Christum, Fílium tuum, qui tecum vivit et regnat in unitáte Spíritus Sancti, Deus, per ómnia sǽcula sæculórum.

\noindent \Rbardot{} Amen.}
\newcommand{\invitatorium}{\pars{Invitatorium.} \scriptura{Ps. 94, 8; Psalmus 94; \textbf{H143}}

\vspace{-4mm}

\antiphona{E}{temporalia/inv-hodiesivocem.gtex}}
\newcommand{\hymnusmatutinum}{\pars{Hymnus}

\cuminitiali{I}{temporalia/hym-NuncTempus.gtex}}
\newcommand{\matutinum}{\pars{Psalmus 1.} \scriptura{Ps. 77, 1}

\vspace{-4mm}

\antiphona{I g\textsuperscript{3}}{temporalia/ant-inclinateaurem.gtex}

%\vspace{-2mm}

\scriptura{Ps. 77, 1-16}

%\vspace{-2mm}

\initiumpsalmi{temporalia/ps77i_xvi-initium-i-g3-auto.gtex}

\input{temporalia/ps77i_xvi-i-g3.tex}

\vfill

\antiphona{}{temporalia/ant-inclinateaurem.gtex}

\vfill
\pagebreak

\pars{Psalmus 2.} \scriptura{Sap. 16, 20}

\vspace{-4mm}

\antiphona{II D}{temporalia/ant-angelorumesca.gtex}

%\vspace{-2mm}

\scriptura{Ps. 77, 17-31}

%\vspace{-2mm}

\initiumpsalmi{temporalia/ps77iii-initium-ii-D-auto.gtex}

\input{temporalia/ps77iii-ii-D.tex}

\vfill

\antiphona{}{temporalia/ant-angelorumesca.gtex}

\vfill
\pagebreak

\pars{Psalmus 3.} \scriptura{Ps. 53, 6; \textbf{H223}}

\vspace{-4mm}

\antiphona{VIII G}{temporalia/ant-deusadiuvatme.gtex}

%\vspace{-2mm}

\scriptura{Ps. 77, 32-39}

%\vspace{-2mm}

\initiumpsalmi{temporalia/ps77xxxii_xxxix-initium-viii-G-auto.gtex}

\input{temporalia/ps77xxxii_xxxix-viii-G.tex} \Abardot{}

\vfill
\pagebreak}
\newcommand{\matversus}{\noindent \Vbardot{} Convertímini ad Dóminum Deum vestrum.

\noindent \Rbardot{} Quia benígnus et miséricors est.}
\newcommand{\lectioi}{\vspace{-4mm}

\pars{Lectio I.} \scriptura{Gn. 18, 16-25}

\noindent De libro Génesis.

\noindent Cum ergo surrexíssent inde viri, direxérunt óculos contra Sódomam: et Abraham simul gradiebátur, dedúcens eos.

\noindent Dixítque Dóminus: "Num celáre pótero Abraham quæ gestúrus sum: cum futúrus sit in gentem magnam, ac robustíssimam, et benedicéndæ sint in illo omnes natiónes terræ?

\noindent Scio enim quod præceptúrus sit fíliis suis, et dómui suæ post se ut custódiant viam Dómini, et fáciant iudícium et iustítiam: ut addúcat Dóminus propter Abraham ómnia quæ locútus est ad eum."

\noindent Dixit ítaque Dóminus: "Clamor Sodomórum et Gomórrhæ multiplicátus est, et peccátum eórum aggravátum est nimis. Descéndam, et vidébo utrum clamórem qui venit ad me, ópere compléverint; an non est ita, ut sciam."

\noindent Converteruntque se inde, et abiérunt Sódomam; Abraham vero adhuc stabat coram Dómino. Et appropínquans ait: "Numquid perdes iustum cum ímpio? si fúerint quinquagínta iusti in civitáte, períbunt simul? et non parces loco illi propter quinquagínta iustos, si fúerint in eo?

\noindent Absit a te ut rem hanc fácias, et occídas iustum cum ímpio, fiátque iustus sicut ímpius, non est hoc tuum: qui iúdicas omnem terram, nequáquam fácies iudícium hoc."}
\newcommand{\responsoriumi}{\pars{Responsorium 1.} \scriptura{\Rbardot{} Gn.22, 1 \Vbardot{} Ps. 49, 14; \textbf{H140}}

\vspace{-5mm}

\responsorium{II}{temporalia/resp-tentavitdeusabraham-CROCHU.gtex}{}}
\newcommand{\lectioii}{\pars{Lectio II.} \scriptura{Gn. 18, 26-32}

\noindent Dixítque Dóminus ad eum: "Si invénero Sódomis quinquagínta iustos in médio civitátis, dimíttam omni loco propter eos."

\noindent Respondénsque Abraham, ait: "Quia semel cœpi, loquar ad Dóminum meum, cum sim pulvis et cinis. Quid si minus quinquagínta iustis quinque fúerint? delébis, propter quadragínta quinque, univérsam urbem?"

\noindent Et ait: "Non delébo, si invénero ibi quadragínta quinque."

\noindent Rursúmque locútus est ad eum: "Sin autem quadragínta ibi invénti fúerint, quid fácies?"

\noindent Ait: "Non percútiam propter quadragínta."

\noindent "Ne quæso, inquit, indignéris, Dómine, si loquar: quid si ibi invénti fúerint trigínta?"

\noindent Respóndit: "Non fáciam, si invénero ibi trigínta."

\noindent "Quia semel, ait, cœpi loquar ad Dóminum meum: quid si ibi invénti fúerint vigínti?"

\noindent Ait: "Non interfíciam propter vigínti."

\noindent "Obsecro, inquit, ne irascáris, Dómine, si loquar adhuc semel: quid si invénti fúerint ibi decem?"

\noindent Et dixit: "Non delébo propter decem."

\noindent Abiitque Dóminus, postquam cessávit loqui ad Abraham: et ille revérsus est in locum suum.}
\newcommand{\responsoriumii}{\pars{Responsorium 2.} \scriptura{\Rbardot{} Gn. 22, 11-12 \Vbardot{} ibid., 18; \textbf{H140}}

\vspace{-5mm}

\responsorium{II}{temporalia/resp-angelusdominivocavit-CROCHU.gtex}{}}
\newcommand{\lectioiii}{\pars{Lectio III.} \scriptura{Supp., Hom. 6 De precatione: PG 64, 462-466}

\noindent Ex Homíliis pseudo-Chrysóstomi.

\noindent Summum bonum est precátio et collóquium cum Deo; nam est consociátio et únio cum Deo: et sícuti córporis óculi lucem vidéntes illustrántur, sic étiam ánimus in Deum inténtus ineffábili eius lúmine illustrátur. Precatiónem, inquam, quæ non sit in hábitu, sed fiat ex ánimo; quæ non certis tempóribus horarúmve discrimínibus circumscribátur, sed noctu diúque contínuo perficiátur.

\noindent Etenim non solum tunc opórtet ánimum repénte in Deum inténdere, cum precatiónem meditétur, sed opórtet étiam tunc, cum offíciis quibúsdam occupátus sit, vel cura circa egénos, vel curis áliis, vel utílibus munificéntiæ opéribus, desidérium et memóriam Dei commiscére, ut, ceu sale, Dei amóre condíta, cibus dulcíssimus Dómino univérsi fiant. Sed licet nobis emoluménto inde redundánte frui per totam perpétuo vitam, si plúrimum témporis ei tribúimus.

\noindent Precátio lumen est ánimi, vera Dei cognítio, Dei et hóminum mediátrix. Animus, per eam sursum elátus in cælos, ampléctitur Dóminum compléxibus ineffabílibus, sícuti infans ad suam matrem lácrimans clamat, divínum lac áppetens; éxpetit vero própria vota, et áccipit dona melióra omni visíbili natúra.

\noindent Nam internúntia venerábilis coram Deo adest precátio, exhílarat ánimum, tranquíllat eius afféctum. Precatiónem síquidem dico, ne putes verba esse. Desidérium est Dei, píetas ineffábilis, non ab homínibus prǽstita, sed a divína grátia effécta, de qua étiam Apóstolus dicit: \emph{Quid enim orémus, ut fíeri debet, nescímus; sed ipse Spíritus intercédit pro nobis gemítibus ineffabílibus.}

\noindent Talem supplicatiónem si cui largiátur Dóminus, opuléntia est non auferénda, et cibus cæléstis, sáturans ánimum: qui eum gustávit, Dómini incénditur desidério ætérno, tamquam igne ardentíssimo, eius ánimum inflammánte.

\noindent {\color{gray} Hanc vero origináliter perfíciens, modéstia et humiliatióne pinge domum tuam, spléndidam redde iustítiæ lúmine; bonis opéribus, tamquam bráctea probáta, exórna domum tuam eámque loco murórum et lapillórum fide et ánimi magnitúdine condécora; super ómnia precatiónem tamquam fastígium ad perfectiónem domus impónens ædifício, ut absolútam domum tuam prǽpares Dómino, et tamquam in domo régia et spléndida Dóminum excípias, tamquam simulácrum iámiam ipsum in templo ánimi collocátum póssidens per eius grátiam.}}
\newcommand{\responsoriumiii}{\pars{Responsorium 3.} \scriptura{\Rbardot{} Gn. 22, 15-17 \Vbardot{} ibid., 18; \textbf{H141}}

\vspace{-5mm}

\responsorium{VIII}{temporalia/resp-vocavitangelusdomini-CROCHU-cumdox.gtex}{}}
\newcommand{\lectiobrevis}{\pars{Lectio Brevis.} \scriptura{Is. 53, 11-12}

\noindent Iustificábit iustus servus meus multos et iniquitátes eórum ipse portábit. Ideo dispértiam ei multos, et cum fórtibus dívidet spólia, pro eo quod trádidit in mortem ánimam suam et cum scelerátis reputátus est; et ipse peccátum multórum tulit et pro transgressóribus rogat.}
\newcommand{\responsoriumbreve}{\pars{Responsorium breve.} \scriptura{Ps. 90, 3}

\cuminitiali{IV}{temporalia/resp-ipseliberavitme.gtex}}
\newcommand{\hymnuslaudes}{\pars{Hymnus}

\cuminitiali{D}{temporalia/hym-IamChriste.gtex}}
\newcommand{\laudes}{\pars{Psalmus 1.} \scriptura{Ps. 50, 3; \textbf{H142}}

\vspace{-4mm}

\antiphona{I f}{temporalia/ant-secundummultitudinem.gtex}

\scriptura{Psalmus 50.}

\initiumpsalmi{temporalia/ps50-initium-i-f-auto.gtex}

\input{temporalia/ps50-i-f.tex}

\vfill

\antiphona{}{temporalia/ant-secundummultitudinem.gtex}

\vfill
\pagebreak

\pars{Psalmus 2.}

\vspace{-4mm}

\antiphona{II D}{temporalia/ant-aedificansierusalem.gtex}

%\vspace{-2mm}

\scriptura{Canticum Tobiæ, Tob. 13, 10-18}

%\vspace{-2mm}

\initiumpsalmi{temporalia/tobiae2-initium-ii-D-auto.gtex}

\input{temporalia/tobiae2-ii-D.tex} \Abardot{}

\vfill
\pagebreak

\pars{Psalmus 3.} \scriptura{Ps. 147, 13; \textbf{H101}}

\vspace{-4mm}

\antiphona{VI F}{temporalia/ant-benedixitfiliistuis.gtex}

\vspace{-2mm}

\scriptura{Psalmus 147.}

%\vspace{-2mm}

\initiumpsalmi{temporalia/ps147-initium-vi-F-auto.gtex}

\input{temporalia/ps147-vi-F.tex} \Abardot{}

\vfill
\pagebreak}
\newcommand{\preces}{\noindent Christum salvatórem,~\gredagger{} qui per mortem et resurrectiónem suam nos redémit,~\grestar{} implorémus:

\Rbardot{} Dómine, miserére nostri.

\noindent Qui Ierúsalem ascendísti ad passiónem subeúndam,~\gredagger{} ut intráres in glóriam,~\grestar{} perduc Ecclésiam tuam in Pascha æternitátis.

\Rbardot{} Dómine, miserére nostri.

\noindent Qui, in cruce exaltátus,~\gredagger{} láncea mílitis transfígi voluísti,~\grestar{} sana vúlnera nostra.

\Rbardot{} Dómine, miserére nostri.

\noindent Qui crucem tuam árborem vitæ constituísti,~\grestar{} fructus eiúsdem baptísmate renátis largíre.

\Rbardot{} Dómine, miserére nostri.

\noindent Qui, in ligno pendens,~\gredagger{} latróni pæniténti pepercísti,~\grestar{} nobis peccatóribus ignósce.

\Rbardot{} Dómine, miserére nostri.}
\newcommand{\benedictus}{\pars{Canticum Zachariæ.} \scriptura{Is. 58, 7-8}

\vspace{-4mm}

{
\grechangedim{interwordspacetext}{0.18 cm plus 0.15 cm minus 0.05 cm}{scalable}%
\antiphona{VIII G}{temporalia/ant-cumviderisnudum.gtex}
\grechangedim{interwordspacetext}{0.22 cm plus 0.15 cm minus 0.05 cm}{scalable}%
}

\vspace{-3mm}

\scriptura{Lc. 1, 68-79}

\vspace{-2mm}

\initiumpsalmi{temporalia/benedictus-initium-viii-G-auto.gtex}

\vspace{-1.5mm}

\input{temporalia/benedictus-viii-G.tex} \Abardot{}}
\newcommand{\hebdomada}{post Cinerum.}
\newcommand{\matud}{Matutinum Hebdomadae D}
\newcommand{\matubd}{Matutinum Hebdomadae B vel D}
\newcommand{\laudd}{Laudes Hebdomadae D}
\newcommand{\laudbd}{Laudes Hebdomadae B vel D}
\newcommand{\hiemalis}{Hiemalis.}
\newcommand{\postcinerum}{Post cinerum.}

% LuaLaTeX

\documentclass[a4paper, twoside, 12pt]{article}
\usepackage[latin]{babel}
%\usepackage[landscape, left=3cm, right=1.5cm, top=2cm, bottom=1cm]{geometry} % okraje stranky
%\usepackage[landscape, a4paper, mag=1166, truedimen, left=2cm, right=1.5cm, top=1.6cm, bottom=0.95cm]{geometry} % okraje stranky
\usepackage[landscape, a4paper, mag=1400, truedimen, left=0.5cm, right=0.5cm, top=0.5cm, bottom=0.5cm]{geometry} % okraje stranky

\usepackage{fontspec}
\setmainfont[FeatureFile={junicode.fea}, Ligatures={Common, TeX}, RawFeature=+fixi]{Junicode}
%\setmainfont{Junicode}

% shortcut for Junicode without ligatures (for the Czech texts)
\newfontfamily\nlfont[FeatureFile={junicode.fea}, Ligatures={Common, TeX}, RawFeature=+fixi]{Junicode}

\usepackage{multicol}
\usepackage{color}
\usepackage{lettrine}
\usepackage{fancyhdr}

% usual packages loading:
\usepackage{luatextra}
\usepackage{graphicx} % support the \includegraphics command and options
\usepackage{gregoriotex} % for gregorio score inclusion
\usepackage{gregoriosyms}
\usepackage{wrapfig} % figures wrapped by the text
\usepackage{parcolumns}
\usepackage[contents={},opacity=1,scale=1,color=black]{background}
\usepackage{tikzpagenodes}
\usepackage{calc}
\usepackage{longtable}
\usetikzlibrary{calc}

\setlength{\headheight}{14.5pt}

\input{conventuscommune.tex} % Often used macros

\newcommand{\annusEditionis}{2021}

%%%% Vicekrat opakovane kousky

\newcommand{\anteOrationem}{
  \rubrica{Ante Orationem, cantatur a Superiore:}

  \pars{Supplicatio Litaniæ.}

  \cuminitiali{}{temporalia/supplicatiolitaniae.gtex}

  \pars{Oratio Dominica.}

  \cuminitiali{}{temporalia/oratiodominica.gtex}

  \rubrica{Deinde dicitur ab Hebdomadario:}

  \cuminitiali{}{temporalia/dominusvobiscum-solemnis.gtex}

  \rubrica{In choro monialium loco Dominus vobiscum dicitur:}

  \sineinitiali{temporalia/domineexaudi.gtex}
}

\setlength{\columnsep}{30pt} % prostor mezi sloupci

%%%%%%%%%%%%%%%%%%%%%%%%%%%%%%%%%%%%%%%%%%%%%%%%%%%%%%%%%%%%%%%%%%%%%%%%%%%%%%%%%%%%%%%%%%%%%%%%%%%%%%%%%%%%%
\begin{document}

% Here we set the space around the initial.
% Please report to http://home.gna.org/gregorio/gregoriotex/details for more details and options
\grechangedim{afterinitialshift}{2.2mm}{scalable}
\grechangedim{beforeinitialshift}{2.2mm}{scalable}
\grechangedim{interwordspacetext}{0.22 cm plus 0.15 cm minus 0.05 cm}{scalable}%
\grechangedim{annotationraise}{-0.2cm}{scalable}

% Here we set the initial font. Change 38 if you want a bigger initial.
% Emit the initials in red.
\grechangestyle{initial}{\color{red}\fontsize{38}{38}\selectfont}

\pagestyle{empty}

%%%% Titulni stranka
\begin{titulusOfficii}
\ifx\titulus\undefined
\nomenFesti{Feria VI \hebdomada{}}
\else
\titulus
\fi
\end{titulusOfficii}

\vfill

\begin{center}
%Ad usum et secundum consuetudines chori \guillemotright{}Conventus Choralis\guillemotleft.

%Editio Sancti Wolfgangi \annusEditionis
\end{center}

\scriptura{}

\pars{}

\pagebreak

\renewcommand{\headrulewidth}{0pt} % no horiz. rule at the header
\fancyhf{}
\pagestyle{fancy}

\cantusSineNeumas

\hora{Ad Matutinum.} %%%%%%%%%%%%%%%%%%%%%%%%%%%%%%%%%%%%%%%%%%%%%%%%%%%%%

\vspace{2mm}

\cuminitiali{}{temporalia/dominelabiamea.gtex}

\vfill
%\pagebreak

\vspace{2mm}

\ifx\invitatorium\undefined
\pars{Invitatorium.} \scriptura{Lc. 24, 34; Psalmus 94; \textbf{H232}}

\antiphona{VI}{temporalia/inv-surrexitdominusvere.gtex}
\else
\invitatorium
\fi

\vfill
\pagebreak

\ifx\hymnusmatutinum\undefined
\pars{Hymnus.}

\cuminitiali{VIII}{temporalia/hym-LaetareCaelum.gtex}
\else
\hymnusmatutinum
\fi

\vspace{-3mm}

\vfill
\pagebreak

\ifx\matutinum\undefined
\ifx\matua\undefined
\else
% MAT A
\pars{Psalmus 1.}

\vspace{-4mm}

\antiphona{I a\textsuperscript{3}}{temporalia/ant-alleluia-turco24.gtex}

%\vspace{-2mm}

\scriptura{Ps. 34, 1-10}

%\vspace{-2mm}

\initiumpsalmi{temporalia/ps34i-initium-i-a5-auto.gtex}

\input{temporalia/ps34i-i-a5.tex}

\vfill
\pagebreak

\pars{Psalmus 2.} \scriptura{Ps. 34, 11-17}

%\vspace{-2mm}

\initiumpsalmi{temporalia/ps34ii-initium-i-a5-auto.gtex}

\input{temporalia/ps34ii-i-a5.tex}

\vfill
\pagebreak

\pars{Psalmus 3.} \scriptura{Ps. 34, 18-28}

\vspace{-2mm}

\initiumpsalmi{temporalia/ps34iii-initium-i-a5-auto.gtex}

\input{temporalia/ps34iii-i-a5.tex}

\vfill

\antiphona{}{temporalia/ant-alleluia-turco24.gtex}

\vfill
\pagebreak
\fi
\ifx\matub\undefined
\else
% MAT B
\pars{Psalmus 1.}

\vspace{-4mm}

\antiphona{D}{temporalia/ant-alleluia-turco2.gtex}

%\vspace{-2mm}

\scriptura{Ps. 37, 2-5}

%\vspace{-2mm}

\initiumpsalmi{temporalia/ps37ii_v-initium-d-g-auto.gtex}

\input{temporalia/ps37ii_v-d-g.tex}

\vfill
\pagebreak

\pars{Psalmus 2.}

\scriptura{Ps. 37, 6-13}

%\vspace{-2mm}

\initiumpsalmi{temporalia/ps37vi_xiii-initium-d-g-auto.gtex}

\input{temporalia/ps37vi_xiii-d-g.tex}

\vfill
\pagebreak

\pars{Psalmus 3.}

\scriptura{Ps. 37, 14-23}

%\vspace{-2mm}

\initiumpsalmi{temporalia/ps37xiv_xxiii-initium-d-g-auto.gtex}

\input{temporalia/ps37xiv_xxiii-d-g.tex}

\vfill

\antiphona{}{temporalia/ant-alleluia-turco2.gtex}

\vfill
\pagebreak
\fi
\ifx\matuc\undefined
\else
% MAT C
\pars{Psalmus 1.}

\vspace{-4mm}

\antiphona{I d\textsuperscript{3}}{temporalia/ant-alleluia-auglx5.gtex}

%\vspace{-3mm}

\scriptura{Ps. 68, 2-13}

%\vspace{-2mm}

\initiumpsalmi{temporalia/ps68ii_xiii-initium-i-d-auto.gtex}

%\vspace{-1.5mm}

\input{temporalia/ps68ii_xiii-i-d.tex}

\vfill
\pagebreak

\pars{Psalmus 2.}

\scriptura{Ps. 68, 14-22}

%\vspace{-2mm}

\initiumpsalmi{temporalia/ps68xiv_xxii-initium-i-d-auto.gtex}

\input{temporalia/ps68xiv_xxii-i-d.tex}

\vfill
\pagebreak

\pars{Psalmus 3.}

\scriptura{Ps. 68, 30-37}

%\vspace{-2mm}

\initiumpsalmi{temporalia/ps68iii-initium-i-d-auto.gtex}

\input{temporalia/ps68iii-i-d.tex}

\vfill

\antiphona{}{temporalia/ant-alleluia-auglx5.gtex}

\vfill
\pagebreak
\fi
\ifx\matud\undefined
\else
% MAT D
\pars{Psalmus 1.}

\vspace{-4mm}

\antiphona{I a\textsuperscript{2}}{temporalia/ant-alleluia-turco24.gtex}

%\vspace{-3mm}

\scriptura{Ps. 77, 1-16}

%\vspace{-2mm}

\initiumpsalmi{temporalia/ps77i_xvi-initium-i-a4-auto.gtex}

\input{temporalia/ps77i_xvi-i-a4.tex}

\vfill
\pagebreak

\pars{Psalmus 2.} \scriptura{Ps. 77, 17-31}

%\vspace{-2mm}

\initiumpsalmi{temporalia/ps77iii-initium-i-a4-auto.gtex}

\input{temporalia/ps77iii-i-a4.tex}

\vfill
\pagebreak

\pars{Psalmus 3.} \scriptura{Ps. 77, 32-39}

%\vspace{-2mm}

\initiumpsalmi{temporalia/ps77xxxii_xxxix-initium-i-a4-auto.gtex}

\input{temporalia/ps77xxxii_xxxix-i-a4.tex}

\vfill

\antiphona{}{temporalia/ant-alleluia-turco24.gtex}

\vfill
\pagebreak
\fi
\else
\matutinum
\fi

\pars{Versus.}

\ifx\matversus\undefined
\noindent \Vbardot{} In resurrectióne tua, Christe, allelúia.

\noindent \Rbardot{} Cæli et terra læténtur, allelúia.
\else
\matversus
\fi

\vspace{5mm}

\sineinitiali{temporalia/oratiodominica-mat.gtex}

\vspace{5mm}

\pars{Absolutio.}

\cuminitiali{}{temporalia/absolutio-ipsius.gtex}

\vfill
\pagebreak

\cuminitiali{}{temporalia/benedictio-solemn-deus.gtex}

\vspace{7mm}

\lectioi

\noindent \Vbardot{} Tu autem, Dómine, miserére nobis.
\noindent \Rbardot{} Deo grátias.

\vfill
\pagebreak

\responsoriumi

\vfill
\pagebreak

\cuminitiali{}{temporalia/benedictio-solemn-christus.gtex}

\vspace{7mm}

\lectioii

\noindent \Vbardot{} Tu autem, Dómine, miserére nobis.
\noindent \Rbardot{} Deo grátias.

\vfill
\pagebreak

\responsoriumii

\vfill
\pagebreak

\cuminitiali{}{temporalia/benedictio-solemn-ignem.gtex}

\vspace{7mm}

\lectioiii

\noindent \Vbardot{} Tu autem, Dómine, miserére nobis.
\noindent \Rbardot{} Deo grátias.

\vfill
\pagebreak

\responsoriumiii

\vfill
\pagebreak

\rubrica{Reliqua omittuntur, nisi Laudes separandæ sint.}

\sineinitiali{temporalia/domineexaudi.gtex}

\vfill

\oratio

\vfill

\noindent \Vbardot{} Dómine, exáudi oratiónem meam.
\Rbardot{} Et clamor meus ad te véniat.

\vfill

\noindent \Vbardot{} Benedicámus Dómino.
\noindent \Rbardot{} Deo grátias.

\vfill

\noindent \Vbardot{} Fidélium ánimæ per misericórdiam Dei requiéscant in pace.
\Rbardot{} Amen.

\vfill
\pagebreak

\hora{Ad Laudes.} %%%%%%%%%%%%%%%%%%%%%%%%%%%%%%%%%%%%%%%%%%%%%%%%%%%%%

\cantusSineNeumas

\vspace{0.5cm}
\grechangedim{interwordspacetext}{0.18 cm plus 0.15 cm minus 0.05 cm}{scalable}%
\cuminitiali{}{temporalia/deusinadiutorium-communis.gtex}
\grechangedim{interwordspacetext}{0.22 cm plus 0.15 cm minus 0.05 cm}{scalable}%

\vfill
\pagebreak

\ifx\hymnuslaudes\undefined
\ifx\laudac\undefined
\else
\pars{Hymnus}

\cuminitiali{I}{temporalia/hym-ChorusNovae-praglia.gtex}
\vspace{-3mm}
\fi
\ifx\laudbd\undefined
\else
\pars{Hymnus}

\cuminitiali{I}{temporalia/hym-ChorusNovae.gtex}
\vspace{-3mm}
\fi
\else
\hymnuslaudes
\fi

\vfill
\pagebreak

\ifx\laudes\undefined
\ifx\lauda\undefined
\else
\pars{Psalmus 1.}

\vspace{-4mm}

\antiphona{VI F}{temporalia/ant-alleluia-turco6.gtex}

\scriptura{Psalmus 50.}

\initiumpsalmi{temporalia/ps50-initium-vi-F-auto.gtex}

\input{temporalia/ps50-vi-F.tex}

\vfill

\antiphona{}{temporalia/ant-alleluia-turco6.gtex}

\vfill
\pagebreak

\pars{Psalmus 2.} \scriptura{Is. 45, 25}

\vspace{-4mm}

\antiphona{V a}{temporalia/ant-indominoiustificabitur-tp.gtex}

\scriptura{Canticum Isaiæ, Is. 45, 15-30}

%\vspace{-2mm}

\initiumpsalmi{temporalia/isaiae2-initium-v-a-auto.gtex}

\input{temporalia/isaiae2-v-a.tex}

\vfill

\antiphona{}{temporalia/ant-indominoiustificabitur-tp.gtex}

\vfill
\pagebreak

\pars{Psalmus 3.}

\vspace{-4mm}

\antiphona{IV* e}{temporalia/ant-alleluia-turco9.gtex}

\scriptura{Psalmus 99.}

\initiumpsalmi{temporalia/ps99-initium-iv_-e-auto.gtex}

\input{temporalia/ps99-iv_-e.tex} \Abardot{}

\vfill
\pagebreak
\fi
\ifx\laudb\undefined
\else
\pars{Psalmus 1.}

\vspace{-4mm}

\antiphona{VII a}{temporalia/ant-alleluia-turco29.gtex}

\scriptura{Psalmus 50.}

\initiumpsalmi{temporalia/ps50-initium-vii-a-auto.gtex}

\input{temporalia/ps50-vii-a.tex}

\vfill

\antiphona{}{temporalia/ant-alleluia-turco29.gtex}

\vfill
\pagebreak

\pars{Psalmus 2.} \scriptura{Hab. 3, 2; \textbf{H99}}

\vspace{-6mm}

\antiphona{IV* e}{temporalia/ant-domineaudivi-tp.gtex}

\vspace{-2mm}

\scriptura{Canticum Habacuc, Hab. 3, 2-19}

%\vspace{-2mm}

%\initiumpsalmi{temporalia/habacuc-initium-iv_-e-auto.gtex}
\initiumpsalmi{temporalia/habacuc-initium-iv_-e.gtex}

\input{temporalia/habacuc-iv_-e.tex}

\vfill

\antiphona{}{temporalia/ant-domineaudivi-tp.gtex}

\vfill
\pagebreak

\pars{Psalmus 3.}

\vspace{-4mm}

\antiphona{E}{temporalia/ant-alleluia-turco4.gtex}

\vspace{-2mm}

\scriptura{Psalmus 147.}

%\vspace{-3mm}

%\initiumpsalmi{temporalia/ps147-initium-e-auto.gtex}
\initiumpsalmi{temporalia/ps147-initium-e.gtex}

\input{temporalia/ps147-e.tex} \Abardot{}

\vfill
\pagebreak
\fi
\ifx\laudc\undefined
\else
\pars{Psalmus 1.}

\vspace{-4mm}

\antiphona{VIII G\textsuperscript{2}}{temporalia/ant-alleluia-turco13.gtex}

\scriptura{Psalmus 50.}

\initiumpsalmi{temporalia/ps50-initium-viii-G5-auto.gtex}

\input{temporalia/ps50-viii-G5.tex}

\vfill

\antiphona{}{temporalia/ant-alleluia-turco13.gtex}

\vfill
\pagebreak

\pars{Psalmus 2.}

\vspace{-4mm}

\antiphona{VIII G}{temporalia/ant-nonnosderelinquas-tp.gtex}

%\vspace{-2mm}

\scriptura{Canticum Ieremiæ, Ier. 14, 17-31}

%\vspace{-2mm}

\initiumpsalmi{temporalia/jeremiae2-initium-viii-G.gtex}

\input{temporalia/jeremiae2-viii-G.tex} \Abardot{}

\vfill
\pagebreak

\pars{Psalmus 3.}

\vspace{-4mm}

\antiphona{E}{temporalia/ant-alleluia-praglia-e2.gtex}

\vspace{-2mm}

\scriptura{Psalmus 99.}

%\vspace{-2mm}

\initiumpsalmi{temporalia/ps99-initium-e-auto.gtex}

\input{temporalia/ps99-e.tex} \Abardot{}

\vfill
\pagebreak
\fi
\ifx\laudd\undefined
\else
\pars{Psalmus 1.}

\vspace{-4mm}

\antiphona{I f}{temporalia/ant-alleluia-turco20.gtex}

\scriptura{Psalmus 50.}

\initiumpsalmi{temporalia/ps50-initium-i-f-auto.gtex}

\input{temporalia/ps50-i-f.tex}

\vfill

\antiphona{}{temporalia/ant-alleluia-turco20.gtex}

\vfill
\pagebreak

\pars{Psalmus 2.} \scriptura{Ac. 22, 14}

\vspace{-4mm}

\antiphona{VIII G}{temporalia/ant-beatiquilavantstolas.gtex}

%\vspace{-2mm}

\scriptura{Canticum Tobiæ, Tob. 13, 10-18}

%\vspace{-2mm}

\initiumpsalmi{temporalia/tobiae2-initium-viii-G-auto.gtex}

\input{temporalia/tobiae2-viii-G.tex} \Abardot{}

\vfill
\pagebreak

\pars{Psalmus 3.}

\vspace{-4mm}

\antiphona{VI F}{temporalia/ant-alleluia-turco5.gtex}

\vspace{-2mm}

\scriptura{Psalmus 147.}

%\vspace{-2mm}

\initiumpsalmi{temporalia/ps147-initium-vi-F-auto.gtex}

\input{temporalia/ps147-vi-F.tex} \Abardot{}

\vfill
\pagebreak
\fi
\else
\laudes
\fi

\ifx\lectiobrevis\undefined
\pars{Lectio Brevis.} \scriptura{Ac. 5, 30-32}

\noindent Deus patrum nostrórum suscitávit Iesum, quem vos interemístis suspendéntes in ligno; hunc Deus Príncipem et Salvatórem exaltávit déxtera sua ad dandam pæniténtiam Israel et remissiónem peccatórum. Et nos sumus testes horum verbórum, et Spíritus Sanctus, quem dedit Deus obœdiéntibus sibi.
\else
\lectiobrevis
\fi

\vfill

\ifx\responsoriumbreve\undefined
\pars{Responsorium breve.} \scriptura{Cf. Mt. 28, 6; Cf. Gal. 3, 13}

\cuminitiali{VI}{temporalia/resp-surrexitdominusdesepulcro.gtex}
\else
\responsoriumbreve
\fi

\vfill
\pagebreak

\benedictus

\vspace{-1cm}

\vfill
\pagebreak

\pars{Preces.}

\sineinitiali{}{temporalia/tonusprecum.gtex}

\ifx\preces\undefined
\ifx\lauda\undefined
\else
\noindent Deum Patrem, qui vitam novam per Christi resurrectiónem cóntulit nobis,~\gredagger{} súpplices exorémus:

\Rbardot{} Clarífica nos claritáte Christi.

\noindent Deus, qui opéribus tuis antíquam dispensatiónem manifestásti, terram creásti et fidélis es in ómnibus generatiónibus,~\gredagger{} exáudi nos, clementíssime Pater.

\Rbardot{} Clarífica nos claritáte Christi.

\noindent Purífica nos puritáte veritátis tuæ, et gressus nostros dírige in cordis sanctitáte,~\gredagger{} ut quod iustum est tibíque plácitum agámus.

\Rbardot{} Clarífica nos claritáte Christi.

\noindent Illúmina vultum tuum super nos,~\gredagger{} ut a peccáto liberáti bonis domus tuæ repleámur.

\Rbardot{} Clarífica nos claritáte Christi.

\noindent Qui per Christum nos tibi reconciliásti,~\gredagger{} pacem nobis largíre omnibúsque in orbe terrárum degéntibus.

\Rbardot{} Clarífica nos claritáte Christi.
\fi
\ifx\laudb\undefined
\else
\noindent Deus Pater Christum per Spíritum suscitávit, et étiam mortália córpora nostra vivificábit.~\gredagger{} Quare clamémus:

\Rbardot{} Dómine, vivífica nos Spíritu Sancto tuo.

\noindent Pater sancte, qui accepísti holocáustum Fílii tui, resúscitans eum ex mórtuis,~\gredagger{} súscipe hodiérnam nostram oblatiónem et perduc nos in vitam ætérnam.

\Rbardot{} Dómine, vivífica nos Spíritu Sancto tuo.

\noindent Opera nostra hódie propítius intuére,~\gredagger{} ut fiant ad glóriam tuam et ad ómnium sanctificatiónem.

\Rbardot{} Dómine, vivífica nos Spíritu Sancto tuo.

\noindent Opus nostrum hódie non sit vanum, sed univérsis homínibus insérviat~\gredagger{} et sic operántes ad regnum tuum fac nos perveníre.

\Rbardot{} Dómine, vivífica nos Spíritu Sancto tuo.

\noindent Aperi hódie óculos nostros et cor nostrum ad fratres,~\gredagger{} ut nos ínvicem amémus nobísque serviámus.

\Rbardot{} Dómine, vivífica nos Spíritu Sancto tuo.
\fi
\ifx\laudc\undefined
\else
\noindent Deum Patrem, qui vitam novam per Christi resurrectiónem cóntulit nobis,~\gredagger{} súpplices exorémus:

\Rbardot{} Clarífica nos claritáte Christi.

\noindent Deus, qui opéribus tuis antíquam dispensatiónem manifestásti, terram creásti et fidélis es in ómnibus generatiónibus,~\gredagger{} exáudi nos, clementíssime Pater.

\Rbardot{} Clarífica nos claritáte Christi.

\noindent Purífica nos puritáte veritátis tuæ, et gressus nostros dírige in cordis sanctitáte,~\gredagger{} ut quod iustum est tibíque plácitum agámus.

\Rbardot{} Clarífica nos claritáte Christi.

\noindent Illúmina vultum tuum super nos,~\gredagger{} ut a peccáto liberáti bonis domus tuæ repleámur.

\Rbardot{} Clarífica nos claritáte Christi.

\noindent Qui per Christum nos tibi reconciliásti,~\gredagger{} pacem nobis largíre omnibúsque in orbe terrárum degéntibus.

\Rbardot{} Clarífica nos claritáte Christi.
\fi
\ifx\laudd\undefined
\else
\noindent Deus Pater Christum per Spíritum suscitávit, et étiam mortália córpora nostra vivificábit.~\gredagger{} Quare clamémus:

\Rbardot{} Dómine, vivífica nos Spíritu Sancto tuo.

\noindent Pater sancte, qui accepísti holocáustum Fílii tui, resúscitans eum ex mórtuis,~\gredagger{} súscipe hodiérnam nostram oblatiónem et perduc nos in vitam ætérnam.

\Rbardot{} Dómine, vivífica nos Spíritu Sancto tuo.

\noindent Opera nostra hódie propítius intuére,~\gredagger{} ut fiant ad glóriam tuam et ad ómnium sanctificatiónem.

\Rbardot{} Dómine, vivífica nos Spíritu Sancto tuo.

\noindent Opus nostrum hódie non sit vanum, sed univérsis homínibus insérviat~\gredagger{} et sic operántes ad regnum tuum fac nos perveníre.

\Rbardot{} Dómine, vivífica nos Spíritu Sancto tuo.

\noindent Aperi hódie óculos nostros et cor nostrum ad fratres,~\gredagger{} ut nos ínvicem amémus nobísque serviámus.

\Rbardot{} Dómine, vivífica nos Spíritu Sancto tuo.
\fi 
\else
\preces
\fi

\vfill

\pars{Oratio Dominica.}

\cuminitiali{}{temporalia/oratiodominicaalt.gtex}

\vfill
\pagebreak

\rubrica{vel:}

\pars{Supplicatio Litaniæ.}

\cuminitiali{}{temporalia/supplicatiolitaniae.gtex}

\vfill

\pars{Oratio Dominica.}

\cuminitiali{}{temporalia/oratiodominica.gtex}

\vfill
\pagebreak

% Oratio. %%%
\oratio

\vspace{-1mm}

\vfill

\rubrica{Hebdomadarius dicit Dominus vobiscum, vel, absente sacerdote vel diacono, sic concluditur:}

\vspace{2mm}

\antiphona{C}{temporalia/dominusnosbenedicat.gtex}

\rubrica{Postea cantatur a cantore:}

\vspace{2mm}

\cuminitiali{VII}{temporalia/benedicamus-tempore-paschali.gtex}

\vspace{1mm}

\vfill
\pagebreak

\end{document}

