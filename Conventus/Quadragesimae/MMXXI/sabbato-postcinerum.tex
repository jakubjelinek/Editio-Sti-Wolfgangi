\newcommand{\sineobmv}{Sine Officium B.M.V. in Sabbato.}
\newcommand{\oratio}{\pars{Oratio.}

\noindent Omnípotens sempitérne Deus, infirmitátem nostram propítius réspice, atque ad protegéndum nos déxteram tuæ maiestátis exténde.

\vfill

\pars{Pro commemoratione B.M.V.} \scriptura{Ct. 4, 11; \textbf{H309}}

\vspace{-4mm}

\antiphona{III a2}{temporalia/ant-favusdistillans.gtex}

\vfill

\noindent Concéde nos fámulos tuos, quǽsumus, Dómine Deus, perpétua mentis et córporis sanitáte gaudére et, gloriósa Beátæ Maríæ semper Vírginis intercessióne, a præsénti liberári tristítia et ætérna pérfrui lætítia.

\noindent Per Dóminum nostrum Iesum Christum, Fílium tuum, qui tecum vivit et regnat in unitáte Spíritus Sancti, Deus, per ómnia sǽcula sæculórum.

\noindent \Rbardot{} Amen.}
\newcommand{\invitatorium}{\pars{Invitatorium.} \scriptura{Ps. 94, 8; Psalmus 94; \textbf{H143}}

\vspace{-4mm}

\antiphona{E}{temporalia/inv-hodiesivocem.gtex}}
\newcommand{\hymnusmatutinum}{\pars{Hymnus}

\cuminitiali{I}{temporalia/hym-NuncTempus.gtex}}
\newcommand{\matutinum}{\pars{Psalmus 1.} \scriptura{Ps. 77, 42}

\vspace{-4mm}

\antiphona{III a}{temporalia/ant-redemiteos.gtex}

%\vspace{-2mm}

\scriptura{Ps. 77, 40-51}

%\vspace{-2mm}

\initiumpsalmi{temporalia/ps77xl_li-initium-iii-a-auto.gtex}

\input{temporalia/ps77xl_li-iii-a.tex} \Abardot{}

\vfill
\pagebreak

\pars{Psalmus 2.} \scriptura{Ps. 76, 15; \textbf{H97}}

\vspace{-5mm}

\antiphona{C}{temporalia/ant-tuesdeus.gtex}

\vspace{-2mm}

\scriptura{Ps. 77, 52-64}

\vspace{-2mm}

\initiumpsalmi{temporalia/ps77lii_lxiv-initium-c-c2-auto.gtex}

\input{temporalia/ps77lii_lxiv-c-c2.tex} \Abardot{}

\vfill
\pagebreak

\pars{Psalmus 3.} \scriptura{Ps. 131, 13}

\vspace{-4mm}

\antiphona{VIII G}{temporalia/ant-elegitdominus.gtex}

%\vspace{-2mm}

\scriptura{Ps. 77, 65-72}

%\vspace{-2mm}

\initiumpsalmi{temporalia/ps77lxv_lxxii-initium-viii-G-auto.gtex}

\input{temporalia/ps77lxv_lxxii-viii-G.tex} \Abardot{}

\vfill
\pagebreak}
\newcommand{\matversus}{\noindent \Vbardot{} Qui facit veritátem, venit ad lucem.

\noindent \Rbardot{} Ut manifesténtur ópera eius.}
\newcommand{\lectioi}{\vspace{-4mm}

\pars{Lectio I.} \scriptura{Gn. 19, 1-3}

\noindent De libro Génesis.

\noindent Venérunt duo ángeli Sódomam véspere, et sedénte Loth in fóribus civitátis. Qui, cum vidísset, surréxit, et ivit óbviam eis: adoravítque pronus in terra, et dixit: Obsecro, dómini, declináte in domum púeri vestri, et manéte ibi: laváte pedes vestros, et mane proficiscímini in viam vestram. Qui dixérunt: Mínime, sed in plátea manébimus. Cómpulit illos óppido ut divérterent ad eum: ingressísque domum illíus fecit convívium, coxit ázyma, et comedérunt.}
\newcommand{\responsoriumi}{\pars{Responsorium 1.} \scriptura{\Rbardot{} Gn. 22, 15-17 \Vbardot{} ibid., 18; \textbf{H141}}

\vspace{-5mm}

\responsorium{VIII}{temporalia/resp-vocavitangelusdomini-CROCHU.gtex}{}}
\newcommand{\lectioii}{\pars{Lectio II.} \scriptura{Gn. 19, 12-17}

\noindent Dixérunt autem ad Loth: Habes hic tuórum quémpiam? génerum, aut fílios, aut fílias, omnes qui tui sunt, educ de urbe hac: delébimus enim locum istum, eo quod incréverit clamor eórum coram Dómino, qui misit nos ut perdámus illos. Egréssus ítaque Loth, locútus est ad géneros suos qui acceptúri erant fílias eius, et dixit: Súrgite, egredímini de loco isto: quia delébit Dóminus civitátem hanc. Et visus est eis quasi ludens loqui. Cumque esset mane, cogébant eum ángeli dicéntes: Surge et tolle uxórem tuam, et duas fílias quas habes: ne et tu páriter péreas in scélere civitátis. Dissimulánte illo, apprehendérunt manum eius, et manum uxóris, ac duárum filiárum eius, eo quod párceret Dóminus illi. Eduxéruntque eum, et posuérunt extra civitátem: ibi locútus est ad eum: Salva ánimam tuam: noli respícere post tergum, nec stes in omni circa regióne: sed in monte salvum te fac, ne et tu simul péreas.}
\newcommand{\responsoriumii}{\pars{Responsorium 2.} \scriptura{\Rbardot{} Gn. 24, 42 \Vbardot{} ibid., 12; \textbf{H141}}

\vspace{-5mm}

\responsorium{VIII}{temporalia/resp-deusdominimeiabraham-CROCHU.gtex}{}}
\newcommand{\lectioiii}{\pars{Lectio III.} \scriptura{Gn. 19, 18-21.23-29}

\noindent Dixítque Loth ad eos: Quæso, Dómine mi, quia invénit servus tuus grátiam coram te, et magnificásti misericórdiam tuam, quam fecísti mecum, ut salváres ánimam meam, nec possum in monte salvári, ne forte apprehéndat me malum, et móriar: est cívitas hæc iuxta, ad quam possum fúgere, parva, et salvábor in ea: numquid non módica est, et vivet ánima mea? Dixítque ad eum: Ecce étiam in hoc suscépi preces tuas, ut non subvértam urbem pro qua locútus es. Festína, et salváre ibi: quia non pótero fácere quidquam donec ingrediáris illuc. Idcírco vocátum est nomen urbis illíus Segor. Sol egréssus est super terram, et Loth ingréssus est in Segor. Igitur Dóminus pluit super Sódomam et Gomórram sulphur et ignem a Dómino de cælo: et subvértit civitátes has et omnem circa regiónem, univérsos habitatóres úrbium, et cuncta terræ viréntia. Respiciénsque uxor eius post se, versa est in státuam salis. Abraham autem consúrgens mane, ubi stéterat prius cum Dómino, intúitus est Sódomam et Gomórram, et univérsam terram regiónis illíus: vidítque ascendéntem favíllam de terra quasi fornácis fumum. Cum enim subvérteret Deus civitátes regiónis illíus, recordátus est Abrahæ et liberávit Loth de subversióne úrbium in quibus habitáverat.}
\newcommand{\responsoriumiii}{\pars{Responsorium 3.} \scriptura{\Rbardot{} Lc. 18, 35.38.41 \Vbardot{} ibid., 39; \textbf{H141}}

\vspace{-5mm}

\responsorium{VIII}{temporalia/resp-caecussedebatsecusviam-CROCHU-cumdox.gtex}{}}
\newcommand{\lectiobrevis}{\pars{Lectio Brevis.} \scriptura{Is. 1, 16-18}

\noindent Lavámini, mundi estóte, auférte malum cogitatiónum vestrárum ab óculis meis; quiéscite ágere pervérse, díscite benefácere: quǽrite iudícium, subveníte opprésso, iudicáte pupíllo, deféndite víduam. Et veníte et iudício contendámus, dicit Dóminus. Si fúerint peccáta vestra ut cóccinum, quasi nix dealbabúntur; et si fúerint rubra quasi vermículus, velut lana erunt.}
\newcommand{\responsoriumbreve}{\pars{Responsorium breve.} \scriptura{Ps. 90, 3}

\cuminitiali{IV}{temporalia/resp-ipseliberavitme.gtex}}
\newcommand{\hymnuslaudes}{\pars{Hymnus}

\cuminitiali{D}{temporalia/hym-IamChriste.gtex}}
\newcommand{\preces}{\noindent Semper et ubíque grátias Christo agámus, qui salvat nos,~\gredagger{} eíque fidénter supplicémus:

\Rbardot{} Súbveni nobis, Dómine, grátia tua.

\noindent Tríbue nos córpora nostra incontamináta serváre,~\gredagger{} ut possit Spíritus Sanctus illic habitáre.

\Rbardot{} Súbveni nobis, Dómine, grátia tua.

\noindent Doce nos iam mane pro frátribus nosmetípsos impéndere~\gredagger{} et tota die in ómnibus tuam implére voluntátem.

\Rbardot{} Súbveni nobis, Dómine, grátia tua.

\noindent Da nobis quǽrere panem, qui permáneat in vitam ætérnam,~\gredagger{} quem tu præstas nobis.

\Rbardot{} Súbveni nobis, Dómine, grátia tua.

\noindent Mater tua, refúgium peccatórum, pro nobis intercédat,~\gredagger{} ut peccátis nostris benígnus ignóscas.

\Rbardot{} Súbveni nobis, Dómine, grátia tua.}
\newcommand{\benedictus}{\pars{Canticum Zachariæ.} \scriptura{Mt. 6, 20; \textbf{H142}}

%\vspace{-4mm}

{
\grechangedim{interwordspacetext}{0.18 cm plus 0.15 cm minus 0.05 cm}{scalable}%
\antiphona{II* a}{temporalia/ant-thesaurizatevobis.gtex}
\grechangedim{interwordspacetext}{0.22 cm plus 0.15 cm minus 0.05 cm}{scalable}%
}

\vspace{-3mm}

\scriptura{Lc. 1, 68-79}

\vspace{-2mm}

\initiumpsalmi{temporalia/benedictus-initium-ii_-a-auto.gtex}

\vspace{-1.5mm}

\input{temporalia/benedictus-ii_-a.tex} \Abardot{}}

\newcommand{\hebdomada}{post Cinerum.}
\newcommand{\matud}{Matutinum Hebdomadae D}
\newcommand{\matubd}{Matutinum Hebdomadae B vel D}
\newcommand{\laudd}{Laudes Hebdomadae D}
\newcommand{\laudbd}{Laudes Hebdomadae B vel D}
\newcommand{\hiemalis}{Hiemalis.}
\newcommand{\postcinerum}{Post cinerum.}

% LuaLaTeX

\documentclass[a4paper, twoside, 12pt]{article}
\usepackage[latin]{babel}
%\usepackage[landscape, left=3cm, right=1.5cm, top=2cm, bottom=1cm]{geometry} % okraje stranky
%\usepackage[landscape, a4paper, mag=1166, truedimen, left=2cm, right=1.5cm, top=1.6cm, bottom=0.95cm]{geometry} % okraje stranky
\usepackage[landscape, a4paper, mag=1400, truedimen, left=0.5cm, right=0.5cm, top=0.5cm, bottom=0.5cm]{geometry} % okraje stranky

\usepackage{fontspec}
\setmainfont[FeatureFile={junicode.fea}, Ligatures={Common, TeX}, RawFeature=+fixi]{Junicode}
%\setmainfont{Junicode}

% shortcut for Junicode without ligatures (for the Czech texts)
\newfontfamily\nlfont[FeatureFile={junicode.fea}, Ligatures={Common, TeX}, RawFeature=+fixi]{Junicode}

% Hebrew font: http://scripts.sil.org/cms/scripts/page.php?site_id=nrsi&id=SILHebrUnic2
\newfontfamily\hebfont[Scale=1]{Ezra SIL}

\usepackage{multicol}
\usepackage{color}
\usepackage{lettrine}
\usepackage{fancyhdr}

% usual packages loading:
\usepackage{luatextra}
\usepackage{graphicx} % support the \includegraphics command and options
\usepackage{gregoriotex} % for gregorio score inclusion
\usepackage{gregoriosyms}
\usepackage{wrapfig} % figures wrapped by the text
\usepackage{parcolumns}
\usepackage[contents={},opacity=1,scale=1,color=black]{background}
\usepackage{tikzpagenodes}
\usepackage{calc}
\usepackage{longtable}
\usetikzlibrary{calc}

\setlength{\headheight}{14.5pt}

\input{conventuscommune.tex} % Often used macros

\newcommand{\annusEditionis}{2022}

\def\hebinitial#1{%
\leavevmode{\newbox\hebbox\setbox\hebbox\hbox{\hebfont{#1}\hskip 1mm}\kern -\wd\hebbox\hbox{\hebfont{#1}\hskip 1mm}}%
}

%%%% Vicekrat opakovane kousky

\newcommand{\anteOrationem}{
  \rubrica{Ante Orationem, cantatur a Superiore:}

  \pars{Supplicatio Litaniæ.}

  \cuminitiali{}{temporalia/supplicatiolitaniae.gtex}

  \pars{Oratio Dominica.}

  \cuminitiali{}{temporalia/oratiodominica.gtex}

  \rubrica{Deinde dicitur ab Hebdomadario:}

  \cuminitiali{}{temporalia/dominusvobiscum-solemnis.gtex}

  \rubrica{In choro monialium loco Dominus vobiscum dicitur:}

  \sineinitiali{temporalia/domineexaudi.gtex}
}

\setlength{\columnsep}{30pt} % prostor mezi sloupci

%%%%%%%%%%%%%%%%%%%%%%%%%%%%%%%%%%%%%%%%%%%%%%%%%%%%%%%%%%%%%%%%%%%%%%%%%%%%%%%%%%%%%%%%%%%%%%%%%%%%%%%%%%%%%
\begin{document}

% Here we set the space around the initial.
% Please report to http://home.gna.org/gregorio/gregoriotex/details for more details and options
\grechangedim{afterinitialshift}{2.2mm}{scalable}
\grechangedim{beforeinitialshift}{2.2mm}{scalable}
\grechangedim{interwordspacetext}{0.22 cm plus 0.15 cm minus 0.05 cm}{scalable}%
\grechangedim{annotationraise}{-0.2cm}{scalable}

% Here we set the initial font. Change 38 if you want a bigger initial.
% Emit the initials in red.
\grechangestyle{initial}{\color{red}\fontsize{38}{38}\selectfont}

\pagestyle{empty}

%%%% Titulni stranka
\begin{titulusOfficii}
\ifx\titulus\undefined
\nomenFesti{Sabbato \hebdomada{}}
\else
\titulus
\fi
\end{titulusOfficii}

\vfill

\begin{center}
%Ad usum et secundum consuetudines chori \guillemotright{}Conventus Choralis\guillemotleft.

%Editio Sancti Wolfgangi \annusEditionis
\end{center}

\scriptura{}

\pars{}

\pagebreak

\renewcommand{\headrulewidth}{0pt} % no horiz. rule at the header
\fancyhf{}
\pagestyle{fancy}

\cantusSineNeumas

\hora{Ad Matutinum.} %%%%%%%%%%%%%%%%%%%%%%%%%%%%%%%%%%%%%%%%%%%%%%%%%%%%%

\vspace{2mm}

\cuminitiali{}{temporalia/dominelabiamea.gtex}

\vfill
%\pagebreak

\vspace{2mm}

\ifx\invitatorium\undefined
\pars{Invitatorium.} \scriptura{Lc. 24, 34; Psalmus 94; \textbf{H232}}

\vspace{-4mm}

\antiphona{VI}{temporalia/inv-surrexitdominusvere.gtex}
\else
\invitatorium
\fi

\vfill
\pagebreak

\ifx\hymnusmatutinum\undefined
\pars{Hymnus.}

\cuminitiali{VIII}{temporalia/hym-LaetareCaelum.gtex}
\else
\hymnusmatutinum
\fi

\vspace{-3mm}

\vfill
\pagebreak

\ifx\matutinum\undefined
\ifx\matua\undefined
\else
% MAT A
\pars{Psalmus 1.}

\vspace{-4mm}

\antiphona{VIII G\textsuperscript{5}}{temporalia/ant-alleluia-turco15.gtex}

\vspace{-3mm}

\scriptura{Ps. 104, 1-15}

\vspace{-2mm}

\initiumpsalmi{temporalia/ps104i-initium-viii-g5.gtex}

\vspace{-1.5mm}

\input{temporalia/ps104i-viii-g.tex}

\vfill
\pagebreak

\pars{Psalmus 2.} \scriptura{Ps. 104, 16-27}

%\vspace{-2mm}

\initiumpsalmi{temporalia/ps104ii-initium-viii-g5.gtex}

\input{temporalia/ps104ii-viii-g.tex}

\vfill
\pagebreak

\pars{Psalmus 3.} \scriptura{Ps. 104, 28-45}

%\vspace{-2mm}

\initiumpsalmi{temporalia/ps104iii-initium-viii-g5.gtex}

\input{temporalia/ps104iii-viii-g.tex}

\vfill

\antiphona{}{temporalia/ant-alleluia-turco15.gtex}

\vfill
\pagebreak
\fi
\ifx\matub\undefined
\else
% MAT B
\pars{Psalmus 1.}

\vspace{-4mm}

\antiphona{t. pereg.}{temporalia/ant-alleluia-turco3.gtex}

%\vspace{-2mm}

\scriptura{Ps. 105, 1-15}

%\vspace{-2mm}

\initiumpsalmi{temporalia/ps105i-initium-per-auto.gtex}

\input{temporalia/ps105i-per.tex}

\vfill
\pagebreak

\pars{Psalmus 2.} \scriptura{Ps. 105, 16-31}

\vspace{-2.5mm}

\initiumpsalmi{temporalia/ps105ii-initium-per-auto.gtex}

\vspace{-1.5mm}

\input{temporalia/ps105ii-per.tex}

\vfill
\pagebreak

\pars{Psalmus 3.} \scriptura{Ps. 105, 32-48}

%\vspace{-2mm}

\initiumpsalmi{temporalia/ps105iii-initium-per-auto.gtex}

\input{temporalia/ps105iii-per.tex}

\vfill

\antiphona{}{temporalia/ant-alleluia-turco3.gtex}

\vfill
\pagebreak
\fi
\ifx\matuc\undefined
\else
% MAT C
\pars{Psalmus 1.} \scriptura{Ps. 106, 8}

\vspace{-4mm}

\antiphona{IV e}{temporalia/ant-alleluia-fo2.gtex}

%\vspace{-2mm}

\scriptura{Ps. 106, 1-14}

%\vspace{-2mm}

\initiumpsalmi{temporalia/ps106i-initium-iv-e2-auto.gtex}

\input{temporalia/ps106i-iv-e2.tex}

\vfill
\pagebreak

\pars{Psalmus 2.} \scriptura{Ps. 106, 15-30}

%\vspace{-2mm}

\initiumpsalmi{temporalia/ps106ii-initium-iv-e2-auto.gtex}

\input{temporalia/ps106ii-iv-e2.tex}

\vfill
\pagebreak

\pars{Psalmus 3.} \scriptura{Ps. 106, 31-43}

%\vspace{-2mm}

\initiumpsalmi{temporalia/ps106iii-initium-iv-e2-auto.gtex}

\input{temporalia/ps106iii-iv-e2.tex}

\vfill
\pagebreak

\antiphona{}{temporalia/ant-alleluia-fo2.gtex}

\vfill
\pagebreak
\fi
\ifx\matud\undefined
\else
% MAT D
\pars{Psalmus 1.}

\vspace{-4mm}

\antiphona{III g}{temporalia/ant-alleluia-turco26.gtex}

%\vspace{-2mm}

\scriptura{Ps. 77, 40-51}

%\vspace{-2mm}

\initiumpsalmi{temporalia/ps77xl_li-initium-iii-g-auto.gtex}

\input{temporalia/ps77xl_li-iii-g.tex}

\vfill
\pagebreak

\pars{Psalmus 2.} \scriptura{Ps. 77, 52-64}

\vspace{-2mm}

\initiumpsalmi{temporalia/ps77lii_lxiv-initium-iii-g-auto.gtex}

\input{temporalia/ps77lii_lxiv-iii-g.tex}

\vfill
\pagebreak

\pars{Psalmus 3.} \scriptura{Ps. 77, 65-72}

%\vspace{-2mm}

\initiumpsalmi{temporalia/ps77lxv_lxxii-initium-iii-g-auto.gtex}

\input{temporalia/ps77lxv_lxxii-iii-g.tex}

\vfill

\antiphona{}{temporalia/ant-alleluia-turco26.gtex}

\vfill
\pagebreak
\fi
\else
\matutinum
\fi

\pars{Versus.}

\ifx\matversus\undefined
\noindent \Vbardot{} Deus regenerávit nos in spem vivam, allelúia.

\noindent \Rbardot{} Per resurrectiónem Iesu Christi ex mórtuis, allelúia.
\else
\matversus
\fi

\vspace{5mm}

\sineinitiali{temporalia/oratiodominica-mat.gtex}

\vspace{5mm}

\pars{Absolutio.}

\cuminitiali{}{temporalia/absolutio-avinculis.gtex}

\vfill
\pagebreak

\cuminitiali{}{temporalia/benedictio-solemn-ille.gtex}

\vspace{7mm}

\lectioi

\noindent \Vbardot{} Tu autem, Dómine, miserére nobis.
\noindent \Rbardot{} Deo grátias.

\vfill
\pagebreak

\responsoriumi

\vfill
\pagebreak

\cuminitiali{}{temporalia/benedictio-solemn-divinum.gtex}

\vspace{7mm}

\lectioii

\noindent \Vbardot{} Tu autem, Dómine, miserére nobis.
\noindent \Rbardot{} Deo grátias.

\vfill
\pagebreak

\responsoriumii

\vfill
\pagebreak

\cuminitiali{}{temporalia/benedictio-solemn-adsocietatem.gtex}

\vspace{7mm}

\lectioiii

\noindent \Vbardot{} Tu autem, Dómine, miserére nobis.
\noindent \Rbardot{} Deo grátias.

\vfill
\pagebreak

\responsoriumiii

\vfill
\pagebreak

\rubrica{Reliqua omittuntur, nisi Laudes separandæ sint.}

\sineinitiali{temporalia/domineexaudi.gtex}

\vfill

\oratio

\vfill

\noindent \Vbardot{} Dómine, exáudi oratiónem meam.
\Rbardot{} Et clamor meus ad te véniat.

\vfill

\noindent \Vbardot{} Benedicámus Dómino.
\noindent \Rbardot{} Deo grátias.

\vfill

\noindent \Vbardot{} Fidélium ánimæ per misericórdiam Dei requiéscant in pace.
\Rbardot{} Amen.

\vfill
\pagebreak

\hora{Ad Laudes.} %%%%%%%%%%%%%%%%%%%%%%%%%%%%%%%%%%%%%%%%%%%%%%%%%%%%%

\cantusSineNeumas

\vspace{0.5cm}
\grechangedim{interwordspacetext}{0.18 cm plus 0.15 cm minus 0.05 cm}{scalable}%
\cuminitiali{}{temporalia/deusinadiutorium-communis.gtex}
\grechangedim{interwordspacetext}{0.22 cm plus 0.15 cm minus 0.05 cm}{scalable}%

\vfill
\pagebreak

\ifx\hymnuslaudes\undefined
\ifx\laudac\undefined
\else
\pars{Hymnus}

\cuminitiali{I}{temporalia/hym-ChorusNovae-praglia.gtex}
\vspace{-3mm}
\fi
\ifx\laudbd\undefined
\else
\pars{Hymnus}

\cuminitiali{I}{temporalia/hym-ChorusNovae.gtex}
\vspace{-3mm}
\fi
\else
\hymnuslaudes
\fi

\vfill
\pagebreak

\ifx\laudes\undefined
\ifx\lauda\undefined
\else
\pars{Psalmus 1.}

\vspace{-4mm}

\antiphona{VII a}{temporalia/ant-alleluia-turco29.gtex}

\scriptura{Psalmus 118, 145-152; \hspace{5mm} \hebinitial{ק}}

\initiumpsalmi{temporalia/ps118xix-initium-vii-a-auto.gtex}

\input{temporalia/ps118xix-vii-a.tex} \Abardot{}

\vfill
\pagebreak

\pars{Psalmus 2.} \scriptura{Ex. 15, 2}

\vspace{-4mm}

\antiphona{IV e}{temporalia/ant-fortitudomeaetlausmea.gtex}

\scriptura{Canticum Moysis, Ex. 15, 1-4a.7b-13.17-19}

\initiumpsalmi{temporalia/moysis1-initium-iv-e2-auto.gtex}

\input{temporalia/moysis1-iv-e2.tex}

\antiphona{}{temporalia/ant-fortitudomeaetlausmea.gtex}

\vfill
\pagebreak

\pars{Psalmus 3.}

\vspace{-4mm}

\antiphona{E}{temporalia/ant-alleluia-praglia-e2.gtex}

\scriptura{Psalmus 116.}

\initiumpsalmi{temporalia/ps116-initium-e-auto.gtex}

\input{temporalia/ps116-e.tex} \Abardot{}

\vfill
\pagebreak
\fi
\ifx\laudb\undefined
\else
\pars{Psalmus 1.}

\vspace{-4.5mm}

\antiphona{E}{temporalia/ant-alleluia-praglia-e2.gtex}

\vspace{-3mm}

\scriptura{Psalmus 91.}

\vspace{-2mm}

\initiumpsalmi{temporalia/ps91-initium-e-auto.gtex}

\vspace{-1.5mm}

\input{temporalia/ps91-e.tex} \Abardot{}

\vfill
\pagebreak

\pars{Psalmus 2.} \scriptura{Eccli. 39, 19}

\vspace{-4mm}

\antiphona{VII c\textsuperscript{2}}{temporalia/ant-effrondeteingratia.gtex}

\vspace{-2mm}

\scriptura{Canticum Moysi, Dt. 32, 1-32}

\vspace{-2mm}

\initiumpsalmi{temporalia/moysis2i_xii-initium-vii-c2-auto.gtex}

\input{temporalia/moysis2i_xii-vii-c2.tex}

\vfill

\antiphona{}{temporalia/ant-effrondeteingratia.gtex}

\vfill
\pagebreak

\pars{Psalmus 3.}

\vspace{-4mm}

\antiphona{I a\textsuperscript{2}}{temporalia/ant-alleluia-turco23.gtex}

%\vspace{-2mm}

\scriptura{Ps. 8}

%\vspace{-2mm}

\initiumpsalmi{temporalia/ps8-initium-i-a2-auto.gtex}

\input{temporalia/ps8-i-a2.tex} \Abardot{}

\vfill
\pagebreak
\fi
\ifx\laudc\undefined
\else
\pars{Psalmus 1.}

\vspace{-4mm}

\antiphona{E}{temporalia/ant-alleluia-praglia-e2.gtex}

%\vspace{-2mm}

\scriptura{Psalmus 118, 145-152.}

%\vspace{-2mm}

\initiumpsalmi{temporalia/ps118xix-initium-e-auto.gtex}

%\vspace{-1.5mm}

\input{temporalia/ps118xix-e.tex} \Abardot{}

\vfill
\pagebreak

\pars{Psalmus 2.}

\vspace{-4mm}

\antiphona{V a}{temporalia/ant-mecumsitdomine-tp.gtex}

%\vspace{-2mm}

\scriptura{Canticum Sapientiæ, Sap. 9, 1-6.9-11}

\initiumpsalmi{temporalia/sapientia-initium-v-a-auto.gtex}

\input{temporalia/sapientia-v-a.tex} \Abardot{}

\vfill
\pagebreak

\pars{Psalmus 3.}

\vspace{-4mm}

\antiphona{II* a}{temporalia/ant-alleluia-turco18.gtex}

%\vspace{-2mm}

\scriptura{Ps. 116}

%\vspace{-2mm}

\initiumpsalmi{temporalia/ps116-initium-ii_-a-auto.gtex}

\input{temporalia/ps116-ii_-a.tex} \Abardot{}

\vfill
\pagebreak
\fi
\ifx\laudd\undefined
\else
\pars{Psalmus 1.}

\vspace{-4.5mm}

\antiphona{VIII G\textsuperscript{2}}{temporalia/ant-alleluia-turco12.gtex}

\vspace{-3mm}

\scriptura{Psalmus 91.}

\vspace{-2mm}

\initiumpsalmi{temporalia/ps91-initium-viii-G5-auto.gtex}

\vspace{-1.5mm}

\input{temporalia/ps91-viii-G5.tex} \Abardot{}

\vfill
\pagebreak

\pars{Psalmus 2.} \scriptura{Heb. 13, 8}

\vspace{-4mm}

\antiphona{II D}{temporalia/ant-iesuschristusheriethodie.gtex}

%\vspace{-2mm}

\scriptura{Canticum Ezechiæ, Ez. 36, 24-28}

\initiumpsalmi{temporalia/ezechiae2-initium-ii-D-auto.gtex}

\input{temporalia/ezechiae2-ii-D.tex} \Abardot{}

\vfill
\pagebreak

\pars{Psalmus 3.}

\vspace{-4mm}

\antiphona{I a\textsuperscript{2}}{temporalia/ant-alleluia-turco23.gtex}

%\vspace{-2mm}

\scriptura{Ps. 8}

%\vspace{-2mm}

\initiumpsalmi{temporalia/ps8-initium-i-a4-auto.gtex}

\input{temporalia/ps8-i-a4.tex} \Abardot{}

\vfill
\pagebreak
\fi
\else
\laudes
\fi

\ifx\lectiobrevis\undefined
\pars{Lectio Brevis.} \scriptura{Rom. 14, 7-9}

\noindent Nemo nostrum sibi vivit et nemo sibi móritur; sive enim vívimus, Dómino vívimus, sive mórimur, Dómino mórimur. Sive ergo vívimus, sive mórimur, Dómini sumus. In hoc enim Christus et mórtuus est et vixit, ut et mortuórum et vivórum dominétur.
\else
\lectiobrevis
\fi

\vfill

\ifx\responsoriumbreve\undefined
\pars{Responsorium breve.} \scriptura{Cf. Mt. 28, 6; Cf. Gal. 3, 13}

\cuminitiali{VI}{temporalia/resp-surrexitdominusdesepulcro.gtex}
\else
\responsoriumbreve
\fi

\vfill
\pagebreak

\benedictus

\vspace{-1cm}

\vfill
\pagebreak

\ifx\precestotum\undefined
\pars{Preces.}

\sineinitiali{}{temporalia/tonusprecumnovum.gtex}

\ifx\preces\undefined
\ifx\lauda\undefined
\else
\noindent Christum, panem vitæ, \gredagger{} qui mensa verbi et córporis sui fruéntes suscitábit in novíssimo die, \grestar{} læti deprecémur:

\Rbardot{} Da nobis, Dómine, pacem et gáudium.

\noindent Fili Dei, qui, suscitátus a mórtuis, princeps es vitæ, \grestar{} nos omnésque fratres tuos bénedic et sanctífica.

\Rbardot{} Da nobis, Dómine, pacem et gáudium.

\noindent Tu, qui pacem et gáudium ómnibus in te credéntibus largíris, \grestar{} da nos sicut fílios lucis ambuláre et de victória tua lætári.

\Rbardot{} Da nobis, Dómine, pacem et gáudium.

\noindent Adáuge fidem Ecclésiæ peregrinántis in terra, \grestar{} ut resurrectiónis tuæ testimónium mundo perhíbeat.

\Rbardot{} Da nobis, Dómine, pacem et gáudium.

\noindent Tu qui, multa passus, \gredagger{} in glóriam Patris intrásti, \grestar{} luctum mæréntium convérte in gáudium.

\Rbardot{} Da nobis, Dómine, pacem et gáudium.
\fi
\ifx\laudb\undefined
\else
\noindent Christum, qui vitam ætérnam nobis manifestávit, \grestar{} devóta mente rogémus, clamántes:

\Rbardot{} Resurréctio tua locuplétet nos grátia, Dómine.

\noindent Pastor ætérne, \gredagger{} réspice gregem tuum e somno surgéntem \grestar{} et pasce nos verbi et panis tui ubérrimo alimónio.

\Rbardot{} Resurréctio tua locuplétet nos grátia, Dómine.

\noindent Ne permíttas nos a lupo rapi vel a mercenário perdi, \grestar{} sed fac, ut vocem tuam fidéliter audiámus.

\Rbardot{} Resurréctio tua locuplétet nos grátia, Dómine.

\noindent Tu, qui cum prædicatóribus ubíque cooperáris eorúmque sermónem confírmas, \grestar{} fac, ut hódie resurrectiónem tuam móribus et vita proclamémus.

\Rbardot{} Resurréctio tua locuplétet nos grátia, Dómine.

\noindent Esto ipse gáudium nostrum, \gredagger{} quod nemo tollat a nobis, \grestar{} ut, reiécta tristítia peccáti, vitam appetámus ætérnam.

\Rbardot{} Resurréctio tua locuplétet nos grátia, Dómine.
\fi
\ifx\laudc\undefined
\else
\noindent Christum, panem vitæ, \gredagger{} qui mensa verbi et córporis sui fruéntes suscitábit in novíssimo die, \grestar{} læti deprecémur:

\Rbardot{} Da nobis, Dómine, pacem et gáudium.

\noindent Fili Dei, qui, suscitátus a mórtuis, princeps es vitæ, \grestar{} nos omnésque fratres tuos bénedic et sanctífica.

\Rbardot{} Da nobis, Dómine, pacem et gáudium.

\noindent Tu, qui pacem et gáudium ómnibus in te credéntibus largíris, \grestar{} da nos sicut fílios lucis ambuláre et de victória tua lætári.

\Rbardot{} Da nobis, Dómine, pacem et gáudium.

\noindent Adáuge fidem Ecclésiæ peregrinántis in terra, \grestar{} ut resurrectiónis tuæ testimónium mundo perhíbeat.

\Rbardot{} Da nobis, Dómine, pacem et gáudium.

\noindent Tu qui, multa passus, \gredagger{} in glóriam Patris intrásti, \grestar{} luctum mæréntium convérte in gáudium.

\Rbardot{} Da nobis, Dómine, pacem et gáudium.
\fi
\ifx\laudd\undefined
\else
\noindent Christum, qui vitam ætérnam nobis manifestávit, \grestar{} devóta mente rogémus, clamántes:

\Rbardot{} Resurréctio tua locuplétet nos grátia, Dómine.

\noindent Pastor ætérne, \gredagger{} réspice gregem tuum e somno surgéntem \grestar{} et pasce nos verbi et panis tui ubérrimo alimónio.

\Rbardot{} Resurréctio tua locuplétet nos grátia, Dómine.

\noindent Ne permíttas nos a lupo rapi vel a mercenário perdi, \grestar{} sed fac, ut vocem tuam fidéliter audiámus.

\Rbardot{} Resurréctio tua locuplétet nos grátia, Dómine.

\noindent Tu, qui cum prædicatóribus ubíque cooperáris eorúmque sermónem confírmas, \grestar{} fac, ut hódie resurrectiónem tuam móribus et vita proclamémus.

\Rbardot{} Resurréctio tua locuplétet nos grátia, Dómine.

\noindent Esto ipse gáudium nostrum, \gredagger{} quod nemo tollat a nobis, \grestar{} ut, reiécta tristítia peccáti, vitam appetámus ætérnam.

\Rbardot{} Resurréctio tua locuplétet nos grátia, Dómine.
\fi
\else
\preces
\fi

\vfill

\pars{Oratio Dominica.}

\cuminitiali{}{temporalia/oratiodominicaalt.gtex}

\vfill
\pagebreak

\rubrica{vel:}

\pars{Deprecatio Gelasii}

\vspace{-5mm}

\grechangedim{interwordspacetext}{0.16 cm plus 0.15 cm minus 0.05 cm}{scalable}%
\antiphona{D\textsuperscript{1}}{temporalia/deprecatio4-propace.gtex}
\grechangedim{interwordspacetext}{0.22 cm plus 0.15 cm minus 0.05 cm}{scalable}%

\vfill

\pars{Oratio Dominica.}

\cuminitiali{D}{temporalia/oratiodominica-d.gtex}
\else
\precestotum
\fi

\vfill
\pagebreak

% Oratio. %%%
\oratio

\vspace{-1mm}

\vfill

\rubrica{Hebdomadarius dicit Dominus vobiscum, vel, absente sacerdote vel diacono, sic concluditur:}

\vspace{2mm}

\ifx\dominusnosbenedicat\undefined
\antiphona{C}{temporalia/dominusnosbenedicat.gtex}
\else
\dominusnosbenedicat
\fi

\rubrica{Postea cantatur a cantore:}

\vspace{2mm}

\cuminitiali{VII}{temporalia/benedicamus-tempore-paschali.gtex}

\vspace{1mm}

\vfill
\pagebreak

\end{document}

