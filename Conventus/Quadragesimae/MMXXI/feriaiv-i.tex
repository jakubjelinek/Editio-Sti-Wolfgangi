\newcommand{\oratio}{\pars{Oratio.}

\noindent Devotiónem pópuli tui, quǽsumus, Dómine, benígnus inténde, ut, qui per abstinéntiam temperántur in córpore, per fructum boni óperis reficiántur in mente.

\noindent Per Dóminum nostrum Iesum Christum, Fílium tuum, qui tecum vivit et regnat in unitáte Spíritus Sancti, Deus, per ómnia sǽcula sæculórum.

\noindent \Rbardot{} Amen.}
\newcommand{\invitatorium}{\pars{Invitatorium.} \scriptura{Ps. 94, 8; Psalmus 94; \textbf{H143}}

\vspace{-4mm}

\antiphona{E}{temporalia/inv-hodiesivocem.gtex}}
\newcommand{\hymnusmatutinum}{\pars{Hymnus}

\cuminitiali{I}{temporalia/hym-NuncTempus.gtex}}
\newcommand{\matversus}{\noindent \Vbardot{} Convertímini et ágite pæniténtiam. 

\noindent \Rbardot{} Fácite vobis cor novum et spíritum novum.}
\newcommand{\lectioi}{\vspace{-4mm}

\pars{Lectio I.} \scriptura{Gn. 23, 1-4.13-16.19}

\noindent De libro Génesis.

\noindent Vixit autem Sara centum vigínti septem annis. Et mórtua est in civitáte Arbée, quæ est Hebron, in terra Chánaan: venítque Abraham ut plángeret et fleret eam. Cumque surrexísset ab offício fúneris, locútus est ad fílios Heth, dicens: Advena sum et peregrínus apud vos: date mihi ius sepúlchri vobíscum, ut sepéliam mórtuum meum. Surréxit Abraham, et locútus est ad Ephron circumstánte plebe: Quæso ut áudias me: dabo pecúniam pro agro: súscipe eam, et sic sepéliam mórtuum meum in eo. Respondítque Ephron: Dómine mi, audi me: terra, quam póstulas, quadringéntis siclis argénti valet: istud est prétium inter me et te: sed quantum est hoc? sépeli mórtuum tuum. Quod cum audísset Abraham, appéndit pecúniam, quam Ephron postuláverat. Atque ita sepelívit Abraham Saram uxórem suam in spelúnca agri dúplici, quæ respiciébat Mambre. Hæc est Hebron in terra Chánaan.}
\newcommand{\responsoriumi}{\pars{Responsorium 1.} \scriptura{\Rbardot{} 2 Cor. 6, 2.4 \Vbardot{} ibid., 4.3; \textbf{H144}}

\vspace{-5mm}

\responsorium{III}{temporalia/resp-eccenunctempus-CROCHU.gtex}{}}
\newcommand{\lectioii}{\pars{Lectio II.} \scriptura{Gn. 24, 1-5}

\noindent Erat autem Abraham senex, dierúmque multórum: et Dóminus in cunctis benedíxerat ei. Dixítque ad servum seniórem domus suæ, qui prǽerat ómnibus quæ habébat: Pone manum tuam subter femur meum, ut adiúrem te per Dóminum Deum cæli et terræ, ut non accípias uxórem fílio meo de filiábus Chananæórum, inter quos hábito: sed ad terram et cognatiónem meam proficiscáris et inde accípias uxórem fílio meo Isaac. Respóndit servus: Si nolúerit múlier veníre mecum in terram hanc, numquid redúcere débeo fílium tuum ad locum, de quo egréssus es?}
\newcommand{\responsoriumii}{\pars{Responsorium 2.} \scriptura{\Vbardot{} 2 Cor. 6, 2.3; \textbf{H143}}

\vspace{-5mm}

\responsorium{VIII}{temporalia/resp-paradisiportas-CROCHU.gtex}{}}
\newcommand{\lectioiii}{\pars{Lectio III.} \scriptura{Gn. 24, 6-10}

\noindent Dixítque Abraham: Cave nequándo redúcas fílium meum illuc. Dóminus Deus cæli, qui tulit me de domo patris mei, et de terra nativitátis meæ, qui locútus est mihi, et iurávit mihi, dicens: Sémini tuo dabo terram hanc: ipse mittet ángelum suum coram te, et accípies inde uxórem fílio meo: sin autem múlier nolúerit sequi te, non tenéberis iuraménto: fílium meum tantum ne redúcas illuc. Pósuit ergo servus manum sub fémore Abraham dómini sui, et iurávit illi super sermóne hoc. Tulítque decem camélos de grege dómini sui, et ábiit, ex ómnibus bonis eius portans secum, profectúsque perréxit in Mesopotámiam ad urbem Nachor.}
\newcommand{\responsoriumiii}{\pars{Responsorium 3.} \scriptura{\Rbardot{} Ps. 90, 11-12 \Vbardot{} ibid., 13; \textbf{H145}}

\vspace{-5mm}

\responsorium{I}{temporalia/resp-angelissuismandavit-CROCHU-cumdox.gtex}{}}
\newcommand{\lectiobrevis}{\pars{Lectio Brevis.} \scriptura{Dt. 7, 6.8-9}

\noindent Te elégit Dóminus Deus tuus, ut sis ei pópulus peculiáris de cunctis pópulis, qui sunt super terram, quia diléxit vos Dóminus et custodívit iuraméntum, quod iurávit pátribus vestris, edúxit vos in manu forti et redémit te de domo servitútis, de manu pharaónis regis Ægýpti. Et scies quia Dóminus Deus tuus ipse est Deus, Deus fidélis, custódiens pactum et misericórdiam diligéntibus se et his, qui custódiunt mandáta eius, in mille generatiónes.}
\newcommand{\responsoriumbreve}{\pars{Responsorium breve.} \scriptura{Ps. 90, 3}

\cuminitiali{IV}{temporalia/resp-ipseliberavitme.gtex}}
\newcommand{\hymnuslaudes}{\pars{Hymnus}

\cuminitiali{D}{temporalia/hym-IamChriste.gtex}}
\newcommand{\preces}{\noindent Benedicátur Auctor salútis nostræ, qui vult hómines fíeri in se novam creatúram, ut vétera tránseant et ómnia renovéntur.~\gredagger{} Quaprópter viva spe fulti eum rogémus:

\Rbardot{} Rénova nos, Dómine, in Spíritu tuo.

\noindent Dómine, qui cælum novum terrámque novam promisísti, semper nos rénova per Spíritum tuum,~\gredagger{} ut in cælésti Ierúsalem te iúgiter perfruámur.

\Rbardot{} Rénova nos, Dómine, in Spíritu tuo.

\noindent Da nos tecum operári ut hunc mundum Spíritu tuo imbuámus~\gredagger{} atque in iustítia, caritáte et pace cívitas terréna finem suum efficácius assequátur.

\Rbardot{} Rénova nos, Dómine, in Spíritu tuo.

\noindent Tríbue nos omnes desídias et neglegéntias castigáre~\gredagger{} et supérnis delectári munéribus.

\Rbardot{} Rénova nos, Dómine, in Spíritu tuo.

\noindent Líbera nos a malo~\gredagger{} nosque a fascinatióne nugacitátis, quæ bona obscúrat, defénde.

\Rbardot{} Rénova nos, Dómine, in Spíritu tuo.}
\newcommand{\benedictus}{\pars{Canticum Zachariæ.} \scriptura{Mt. 12, 39; \textbf{H148}}

\vspace{-4mm}

{
\grechangedim{interwordspacetext}{0.18 cm plus 0.15 cm minus 0.05 cm}{scalable}%
\antiphona{VIII G}{temporalia/ant-generatiohaec.gtex}
\grechangedim{interwordspacetext}{0.22 cm plus 0.15 cm minus 0.05 cm}{scalable}%
}

%\vspace{-2mm}

\scriptura{Lc. 1, 68-79}

%\vspace{-2mm}

\initiumpsalmi{temporalia/benedictus-initium-viii-G-auto.gtex}

%\vspace{-1mm}

\input{temporalia/benedictus-viii-G.tex} \Abardot{}}
\newcommand{\magnificat}{\pars{Canticum B. Mariæ V.} \scriptura{Mt. 12, 40; \textbf{H148}}

\vspace{-4mm}

{
\grechangedim{interwordspacetext}{0.18 cm plus 0.15 cm minus 0.05 cm}{scalable}%
\antiphona{II* b}{temporalia/ant-sicutfuitionas.gtex}
\grechangedim{interwordspacetext}{0.22 cm plus 0.15 cm minus 0.05 cm}{scalable}%
}

%\vspace{-2mm}

\scriptura{Lc. 1, 46-55}

%\vspace{-2mm}

\cantusSineNeumas
\initiumpsalmi{temporalia/magnificat-initium-ii_-B.gtex}

%\vspace{-1.5mm}

\input{temporalia/magnificat-ii_-B.tex} \Abardot{}}
\newcommand{\oratiovesperas}{\pars{Oratio.}

\noindent Mentes nostras, quǽsumus Dómine, lúmine tuæ claritátis illústra:~\grestar{} ut vidére possímus quæ agénda sunt; et quæ recta sunt, ágere valeámus.

\noindent Per Dóminum nostrum Iesum Christum, Fílium tuum, qui tecum vivit et regnat in unitáte Spíritus Sancti, Deus, per ómnia sǽcula sæculórum.

\noindent \Rbardot{} Amen.}
\newcommand{\hebdomada}{I}
\newcommand{\oratioMatutinum}{\noindent Præsta, quǽsumus, omnípotens Deus: \gredagger{} ut qui paschália festa perégimus, \grestar{} hæc, te largiénte, móribus et vita teneámus. Per Dóminum.}
\newcommand{\oratioLaudes}{\cuminitiali{}{temporalia/oratio.gtex}}


% LuaLaTeX

\documentclass[a4paper, twoside, 12pt]{article}
\usepackage[latin]{babel} 
%\usepackage[landscape, left=3cm, right=1.5cm, top=2cm, bottom=1cm]{geometry} % okraje stranky
%\usepackage[landscape, a4paper, mag=1166, truedimen, left=2cm, right=1.5cm, top=1.6cm, bottom=0.95cm]{geometry} % okraje stranky
\usepackage[landscape, a4paper, mag=1400, truedimen, left=0.5cm, right=0.5cm, top=0.5cm, bottom=0.5cm]{geometry} % okraje stranky

\usepackage{fontspec}
\setmainfont[FeatureFile={junicode.fea}, Ligatures={Common, TeX}, RawFeature=+fixi]{Junicode}
%\setmainfont{Junicode}

% shortcut for Junicode without ligatures (for the Czech texts)
\newfontfamily\nlfont[FeatureFile={junicode.fea}, Ligatures={Common, TeX}, RawFeature=+fixi]{Junicode}

% Hebrew font:
% http://scripts.sil.org/cms/scripts/page.php?site_id=nrsi&id=SILHebrUnic2
\newfontfamily\hebfont[Scale=1]{Ezra SIL}

\usepackage{multicol}
\usepackage{color}
\usepackage{lettrine}
\usepackage{fancyhdr}

% usual packages loading:
\usepackage{luatextra}
\usepackage{graphicx} % support the \includegraphics command and options
\usepackage{gregoriotex} % for gregorio score inclusion
\usepackage{gregoriosyms}
\usepackage{wrapfig} % figures wrapped by the text
\usepackage{parcolumns}
\usepackage[contents={},opacity=1,scale=1,color=black]{background}
\usepackage{tikzpagenodes}
\usepackage{calc}
\usepackage{longtable}
\usetikzlibrary{calc}

\setlength{\headheight}{14.5pt}

\input{conventuscommune.tex} % Often used macros
%%%% Preklady jednotlivych zpevu (nektere se opakuji, a je dobre mit je
% vsechny na jedne hromade)

% HOURS ---

\newcommand{\trAntI}{\translatioCantus{Muž boží měl kožený toulec, pečlivě
zavázaný, jenž mu visel na šíji a~často se ho dotýkal.}}

\newcommand{\trAntII}{\translatioCantus{Klíč od~něho tak dobře střežil, že
dokud žil v~těle, nikdo z~jeho žáků nezvěděl, co je uvnitř.}}

\newcommand{\trAntIII}{\translatioCantus{Ale když se odebral z~tohoto
života, schránku otevřeli a~objevili v~ní žíněné roucho a~měděný řetěz
potřísněný krví.}}

\newcommand{\trAntIV}{\translatioCantus{A když prohlédli mistrovo tělo,
nalezli jeho tělo na čtyřech místech hluboce zbrázděno ranami od řetězu.}}

\newcommand{\trAntV}{\translatioCantus{Krev vytékající z~těch ran, místy
prostoupila i~žíněným rouchem.}}

\newcommand{\trCapituli}{\translatioCantus{
Miláčkovi Boha a~lidí,
Mojžíšovi požehnané paměti,~\gredagger{}
dopřál slávu rovnou slávě svatých~\grestar{}
učinil ho mocným na postrach nepřátelům
a~jeho slovy zastavil divy.}}

\newcommand{\trLectioBrevis}{\translatioCantus{
Pamatujte na své představené,
kteří vám hlásali Boží slovo.
Uvažte, jak oni skončili život, a~napodobujte jejich víru.
Ježíš Kristus je stejný včera i~dnes i~navěky.
Nenechte se svést věelijakými cizími naukami.}}

\newcommand{\trRespLaud}{\translatioCantus{Spravedlivého vodil Hospodin~\grestar{}
po přímých stezkách. \Vbardot{} A~ukázal mu Boží království.}}

\newcommand{\trRespLaudB}{\translatioCantus{Na tvých hradbách, Jeruzaléme,
ustanovil jsem strážné;~\grestar{}
budou bdít nad mým lidem. \Vbardot{} Ani ve dne, ani v~noci nesmějí nikdy
mlčet.}}

\newcommand{\trVersus}{\translatioCantus{\Vbardot{} Ústa spravedlivého šeptají moudrost, aleluja.
\Rbardot{} A~jeho jazyk ohlašuje právo, aleluja.}}

\newcommand{\trAntBenedictus}{\translatioCantus{Když na bujné oře vložili
nosítka a~sňali jim uzdu, vydali se přímo k~cele božího muže.}}

\newcommand{\trPreces}{\translatioCantus{
\noindent S vděčností chvalme Krista, dobrého Pastýře, \gredagger{} který dal život za své ovce, \grestar{} a~pokorně ho prosme: \Rbardot{} Pane, buď pastýřem svého lidu.

\noindent Kriste, ty dáváš církvi pastýře, a~jejich službou se ujímáš svého lidu, \grestar{} dej, ať v~lásce těch, kteří nás vedou, poznáváme, jak nás miluješ. \Rbardot{} Pane, buď pastýřem svého lidu.

\noindent Ty stále konáš skrze své zástupce službu pastýře a~učitele, \grestar{} nepřestávej nás nikdy vést prostřednictvím svých služebníků. \Rbardot{} Pane, buď pastýřem svého lidu.

\noindent Ty prokazuješ svému lidu skrze jeho pastýře službu lékaře duše i~těla, \grestar{} ochraňuj náš život a~veď nás ke svatosti. \Rbardot{} Pane, buď pastýřem svého lidu.

\noindent Ty posíláš své svaté, aby slovem i~příkladem vedli tvůj lid k~tobě, \grestar{} na jejich přímluvu nás posiluj, abychom vytrvali na cestě, která vede k~věčnému životu. \Rbardot{} Pane, buď pastýřem svého lidu.}}

\newcommand{\trOrationis}{\translatioCantus{Bože, jenž nám dopřáváš radovat
se z~výroční slavnosti svatého tvého vyznavače Havla, uděl dobrotivě,
abychom když slavíme jeho narození, též se řídili podobou jeho skutků.
Skrze…}}
 % Czech translations of the proper texts

\newcommand{\annusEditionis}{2020}

\def\hebinitial#1{%
\leavevmode{\newbox\hebbox\setbox\hebbox\hbox{\hebfont{#1}\hskip 1mm}\kern -\wd\hebbox\hbox{\hebfont{#1}\hskip 1mm}}%
}

%%%% Vicekrat opakovane kousky

\newcommand{\anteOrationem}{
  \rubrica{Ante Orationem, cantatur a Superiore:}

  \pars{Supplicatio Litaniæ.}

  \cuminitiali{}{temporalia/supplicatiolitaniae.gtex}

  \pars{Oratio Dominica.}

  \cuminitiali{}{temporalia/oratiodominica.gtex}

  \rubrica{Deinde dicitur ab Hebdomadario:}

  \cuminitiali{}{temporalia/dominusvobiscum-solemnis.gtex}

  \rubrica{In choro monialium loco Dominus vobiscum dicitur:}

  \sineinitiali{temporalia/domineexaudi.gtex}
}

\setlength{\columnsep}{30pt} % prostor mezi sloupci

%%%%%%%%%%%%%%%%%%%%%%%%%%%%%%%%%%%%%%%%%%%%%%%%%%%%%%%%%%%%%%%%%%%%%%%%%%%%%%%%%%%%%%%%%%%%%%%%%%%%%%%%%%%%%
\begin{document}

% Here we set the space around the initial.
% Please report to http://home.gna.org/gregorio/gregoriotex/details for more details and options
\grechangedim{afterinitialshift}{2.2mm}{scalable}
\grechangedim{beforeinitialshift}{2.2mm}{scalable}

\grechangedim{interwordspacetext}{0.32 cm plus 0.15 cm minus 0.05 cm}{scalable}%
\grechangedim{annotationraise}{-0.2cm}{scalable}

% Here we set the initial font. Change 38 if you want a bigger initial.
% Emit the initials in red.
\grechangestyle{initial}{\color{red}\fontsize{38}{38}\selectfont}

\pagestyle{empty}

%%%% Titulni stranka
\begin{titulusOfficii}
\nomenFesti{Feria IV \hebdomada{}}
\end{titulusOfficii}

\pagebreak

% graphic
\renewcommand{\headrulewidth}{0pt} % no horiz. rule at the header
\fancyhf{}
\pagestyle{fancy}

\cantusSineNeumas

\hora{Ad Matutinum.}

\vspace{2mm}

\cuminitiali{}{temporalia/dominelabiamea.gtex}

\vspace{2mm}

\pars{Invitatorium.} \scriptura{Lc. 24, 34; Psalmus 94; \textbf{H232}}

\vspace{-6mm}

\antiphona{VI}{temporalia/inv-surrexitdominusvere.gtex}

\vfill
\pagebreak

\pars{Hymnus.}

\vspace{-5mm}

\scriptura{\textbf{AR454}}

{
\grechangedim{interwordspacetext}{0.30 cm plus 0.15 cm minus 0.05 cm}{scalable}%
\antiphona{IV}{temporalia/hym-RexSempiterne.gtex}
\grechangedim{interwordspacetext}{0.32 cm plus 0.15 cm minus 0.05 cm}{scalable}%
}
%{
%\vspace{-5mm}
%\setlength{\columnsep}{0pt} % prostor mezi sloupci
%\input{hym-RexSempiterne-bohtext.tex}
%\setlength{\columnsep}{30pt} % prostor mezi sloupci
%}

\vfill
\pagebreak

\pars{Psalmus 1.}

%\vspace{-5mm}

\antiphona{I g}{temporalia/ant-alleluia-fiv-matutinum.gtex}

%\vspace{-5mm}

\scriptura{Ps. 44, 2-10}

%\vspace{-2mm}

\initiumpsalmi{temporalia/ps44i-initium-i-g-auto.gtex}

%\psalmusEtTranslatioT{temporalia/ps44i-III-comb.tex}{10cm}

\input{temporalia/ps44i-III.tex}

\vfill
\pagebreak

\pars{Psalmus 2.} \scriptura{Ps. 44, 11-18}

%\vspace{-2mm}

\initiumpsalmi{temporalia/ps44ii-initium-i-g-auto.gtex}

%\psalmusEtTranslatioT{temporalia/ps44i-III-comb.tex}{10cm}

\input{temporalia/ps44ii-III.tex}

\vfill
\pagebreak

\pars{Psalmus 3.} \scriptura{Ps. 45}

%\vspace{-2mm}

\initiumpsalmi{temporalia/ps45-initium-i-g-auto.gtex}

%\psalmusEtTranslatioT{temporalia/ps45-III-comb.tex}{10cm}

\input{temporalia/ps45-III.tex}

\vfill
\pagebreak

\pars{Psalmus 4.} \scriptura{Ps. 47}

%\vspace{-2mm}

\initiumpsalmi{temporalia/ps47-initium-i-g-auto.gtex}

%\psalmusEtTranslatioT{temporalia/ps47-III-comb.tex}{10cm}

\input{temporalia/ps47-III.tex}

\vfill
\pagebreak

\pars{Psalmus 5.} \scriptura{Ps. 48, 2-13}

%\vspace{-2mm}

\initiumpsalmi{temporalia/ps48i-initium-i-g-auto.gtex}

%\psalmusEtTranslatioT{temporalia/ps48i-III-comb.tex}{10cm}

\input{temporalia/ps48i-III.tex}

\vfill
\pagebreak

\pars{Psalmus 6.} \scriptura{Ps. 48, 14-21}

%\vspace{-2mm}

\initiumpsalmi{temporalia/ps48ii-initium-i-g-auto.gtex}

%\psalmusEtTranslatioT{temporalia/ps48ii-III-comb.tex}{10cm}

\input{temporalia/ps48ii-III.tex}

\vfill
\pagebreak

\pars{Psalmus 7.} \scriptura{Ps. 49, 1-15}

%\vspace{-2mm}

\initiumpsalmi{temporalia/ps49i-initium-i-g-auto.gtex}

%\psalmusEtTranslatioT{temporalia/ps49i-III-comb.tex}{10cm}

\input{temporalia/ps49i-III.tex}

\vfill
\pagebreak

\pars{Psalmus 8.} \scriptura{Ps. 49, 16-23}

%\vspace{-2mm}

\initiumpsalmi{temporalia/ps49ii-initium-i-g-auto.gtex}

%\psalmusEtTranslatioT{temporalia/ps49ii-III-comb.tex}{10cm}

\input{temporalia/ps49ii-III.tex}

\vfill
\pagebreak

\pars{Psalmus 9.} \scriptura{Ps. 50}

%\vspace{-2mm}

\initiumpsalmi{temporalia/ps50-initium-i-g-auto.gtex}

%\psalmusEtTranslatioT{temporalia/ps50-VI-comb.tex}{10cm}

\input{temporalia/ps50-VI.tex}

\vfill
%\pagebreak

\antiphona{}{temporalia/ant-alleluia-fiv-matutinum.gtex}

\vfill
\pagebreak

\noindent \Vbardot{} Gavísi sunt discípuli, allelúia.
\noindent \Rbardot{} Viso Dómino, allelúia.

\noindent Pater noster.

\pars{Absolutio.}

\cuminitiali{}{temporalia/absolutio-avinculis.gtex}

\vfill
\pagebreak

\ifx\magnificat\undefined
\cuminitiali{}{temporalia/benedictio-solemn-evangelica.gtex}
\else
\cuminitiali{}{temporalia/benedictio-solemn-ille.gtex}
\fi

\vspace{7mm}

\lectioi

\noindent \Vbardot{} Tu autem, Dómine, miserére nobis.
\noindent \Rbardot{} Deo grátias.

\vfill
\pagebreak

\responsoriumi

\vfill
\pagebreak

\cuminitiali{}{temporalia/benedictio-solemn-divinum.gtex}

\vspace{7mm}

\lectioii

\noindent \Vbardot{} Tu autem, Dómine, miserére nobis.
\noindent \Rbardot{} Deo grátias.

\vfill
\pagebreak

\responsoriumii

\vfill
\pagebreak

\ifx\magnificat\undefined
\cuminitiali{}{temporalia/benedictio-solemn-adsocietatem.gtex}
\else
\cuminitiali{}{temporalia/benedictio-solemn-ignem.gtex}
\fi

\vspace{7mm}

\lectioiii

\noindent \Vbardot{} Tu autem, Dómine, miserére nobis.
\noindent \Rbardot{} Deo grátias.

\vfill
\pagebreak

% Te Deum

%\pars{Hymnus Ambrosianus}

\vspace{-5mm}

{
\grechangedim{interwordspacetext}{0.22 cm plus 0.15 cm minus 0.05 cm}{scalable}%
\cuminitiali{III}{temporalia/tedeum-solemnis.gtex}
\grechangedim{interwordspacetext}{0.32 cm plus 0.15 cm minus 0.05 cm}{scalable}%
}

\vfill
\pagebreak

\rubrica{Reliqua omittuntur, nisi Laudes separandæ sint.}

\pars{Oratio}

\noindent \Vbardot{} Dómine, exáudi oratiónem meam.

\noindent \Rbardot{} Et clamor meus ad te véniat.

Orémus:

\oratioMatutinum

\noindent \Rbardot{} Amen.

\vspace{7mm}

\pars{Conclusio}

\noindent \Vbardot{} Dómine, exáudi oratiónem meam.

\noindent \Rbardot{} Et clamor meus ad te véniat.

\noindent \Vbardot{} Benedicámus Dómino, allelúia, allelúia.

\noindent \Rbardot{} Deo grátias, allelúia, allelúia.

\noindent \Vbardot{} Fidélium ánimæ per misericórdiam Dei requiéscant in pace.

\noindent \Rbardot{} Amen.

\vfill
\pagebreak

\hora{Ad Laudes.} %%%%%%%%%%%%%%%%%%%%%%%%%%%%%%%%%%%%%%%%%%%%%%%%%%%%%
%\sideThumbs{Laudes}

\cantusSineNeumas

\vspace{0.5cm}
\grechangedim{interwordspacetext}{0.18 cm plus 0.15 cm minus 0.05 cm}{scalable}%
\cuminitiali{}{temporalia/deusinadiutorium-communis.gtex}
\grechangedim{interwordspacetext}{0.32 cm plus 0.15 cm minus 0.05 cm}{scalable}%

\vfill
%\pagebreak

\pars{Psalmus 1.}

\vspace{-0.4cm}

\antiphona{VII a}{temporalia/ant-alleluia-fiv-laudes-1.gtex}

\scriptura{Psalmus 50.}

\initiumpsalmi{temporalia/ps50-initium-vii-a-auto.gtex}

%\psalmusEtTranslatioT{temporalia/ps50-III-comb.tex}{10cm}
\input{temporalia/ps50-III.tex}

\vspace{-1cm}

\vfill
\pagebreak

\pars{Psalmus 2.} \scriptura{Psalmus 63.}

\initiumpsalmi{temporalia/ps63-initium-vii-a-auto.gtex}

%\psalmusEtTranslatioT{temporalia/ps63-III-comb.tex}{10cm}
\input{temporalia/ps63-III.tex}

\vfill
\pagebreak

\pars{Psalmus 3.} \scriptura{Psalmus 64.}

\initiumpsalmi{temporalia/ps64-initium-vii-a-auto.gtex}

%\psalmusEtTranslatioT{temporalia/ps64-III-comb.tex}{10cm}
\input{temporalia/ps64-III.tex}

\vfill

\vspace{-6mm}

\antiphona{}{temporalia/ant-alleluia-fiv-laudes-1.gtex} % repeat the antiphon - new page

\vfill
\pagebreak

\pars{Psalmus 4.} \scriptura{1 Sam. 2, 10; \textbf{H96}}

\vspace{-7mm}

\antiphona{I g\textsuperscript{2}}{temporalia/ant-dominusjudicabit-tp.gtex}

%\vspace{-4mm}

\scriptura{Canticum Annæ, 1 Reg. 2, 1-10}

%\vspace{-3mm}

\initiumpsalmi{temporalia/anna-initium-i-g2-auto.gtex}

%\psalmusEtTranslatioT{temporalia/anna-comb.tex}{10cm}
\input{temporalia/anna.tex}

%\vfill

\antiphona{}{temporalia/ant-dominusjudicabit-tp.gtex}

\vfill
\pagebreak

\pars{Psalmus 5.}

\vspace{-0.4cm}

\antiphona{II D}{temporalia/ant-alleluia-fiv-laudes-2.gtex}

\scriptura{Psalmus 148.}

\initiumpsalmi{temporalia/ps148-initium-ii-D-auto.gtex}

%\psalmusEtTranslatioT{temporalia/ps148-III-comb.tex}{10cm}
\input{temporalia/ps148-III.tex}

\rubrica{Hic non dicitur Gloria Patri.}

\vfill
\pagebreak

%
\scriptura{Psalmus 149.}

\initiumpsalmi{temporalia/ps149-initium-ii-D-auto.gtex}

%\psalmusEtTranslatioT{temporalia/ps149-III-comb.tex}{10cm}
\input{temporalia/ps149-III.tex}

\rubrica{Hic non dicitur Gloria Patri.}

\vfill
\pagebreak

%
\scriptura{Psalmus 150.}

\initiumpsalmi{temporalia/ps150-initium-ii-D-auto.gtex}

%\psalmusEtTranslatioT{temporalia/ps150-III-comb.tex}{10cm}
\input{temporalia/ps150-III.tex}

\vfill

\vspace{-6mm}

\antiphona{}{temporalia/ant-alleluia-fiv-laudes-2.gtex} % repeat the antiphon - new page

\vfill
\pagebreak

\pars{Capitulum.} \scriptura{Rom. 6, 9-10}

\grechangedim{interwordspacetext}{0.12 cm plus 0.15 cm minus 0.05 cm}{scalable}%
\cuminitiali{}{temporalia/capitulum-ChristusResurgens.gtex}
\grechangedim{interwordspacetext}{0.32 cm plus 0.15 cm minus 0.05 cm}{scalable}%

% preklad Jeruz. bible
%\trCapituliI

\vfill

\pars{Responsorium breve.} \scriptura{Cf. Mt. 28, 6; Cf. Gal. 3, 13}

\cuminitiali{VI}{temporalia/respbr-laud.gtex}

%\trResp

\vfill
\pagebreak

\pars{Hymnus}

\cuminitiali{VIII}{temporalia/hym-AuroraLucis.gtex}
\vspace{-3mm}
%\input{hym-AuroraLucis-bohtext.tex}

\vfill
%\pagebreak

\pars{Versus.}

% Versus. %%%
\sineinitiali{temporalia/versus-inresurrectione.gtex}

%\noindent \trVersus

\vfill
\pagebreak

\benedictus

\vspace{-1cm}

\vfill
\pagebreak

%\sideThumbs{{\scriptsize{}Fine horarum}}

\anteOrationem

\pagebreak

% Oratio. %%%
\oratioLaudes

\vspace{-1mm}
%\trOrationisI

\vfill

\rubrica{Hebdomadarius dicit iterum Dominus vobiscum. Postea cantatur a cantore:}
\vspace{2mm}

\cuminitiali{VII}{temporalia/benedicamus-tempore-paschali.gtex}

\vspace{1mm}

\ifx\magnificat\undefined
\else
\vfill
\pagebreak

\hora{Ad Vesperas.} %%%%%%%%%%%%%%%%%%%%%%%%%%%%%%%%%%%%%%%%%%%%%%%%%%%%%
%\sideThumbs{Vesperæ}

\cantusSineNeumas

%\vspace{0.5cm}
\grechangedim{interwordspacetext}{0.18 cm plus 0.15 cm minus 0.05 cm}{scalable}%
\cuminitiali{}{temporalia/deusinadiutorium-communis.gtex}
\grechangedim{interwordspacetext}{0.32 cm plus 0.15 cm minus 0.05 cm}{scalable}%

\vfill
%\pagebreak

\vspace{4mm}

\pars{Psalmus 1.}

\vspace{-0.4cm}

\antiphona{III g}{temporalia/ant-alleluia-fiv-vesperas.gtex}

\vspace{-4mm}

\scriptura{Psalmus 134.}

\initiumpsalmi{temporalia/ps134-initium-iii-g-auto.gtex}

%\psalmusEtTranslatioT{temporalia/ps134-III-comb.tex}{10cm}
\input{temporalia/ps134-III.tex}

\vspace{-1cm}

\vfill
\pagebreak

\pars{Psalmus 2.} \scriptura{Psalmus 135.}

\initiumpsalmi{temporalia/ps135-initium-iii-g-auto.gtex}

%\psalmusEtTranslatioT{temporalia/ps135-III-comb.tex}{10cm}
\input{temporalia/ps135-III.tex}

\vfill
\pagebreak

\pars{Psalmus 3.} \scriptura{Psalmus 136.}

\initiumpsalmi{temporalia/ps136-initium-iii-g-auto.gtex}

%\psalmusEtTranslatioT{temporalia/ps136-III-comb.tex}{10cm}
\input{temporalia/ps136-III.tex}

\vfill
\pagebreak

\pars{Psalmus 4.} \scriptura{Psalmus 137.}

\initiumpsalmi{temporalia/ps137-initium-iii-g-auto.gtex}

%\psalmusEtTranslatioT{temporalia/ps137-III-comb.tex}{10cm}
\input{temporalia/ps137-III.tex}

\vfill

\vspace{-6mm}

\antiphona{}{temporalia/ant-alleluia-fiv-vesperas.gtex} % repeat the antiphon - new page

\vfill
\pagebreak

\pars{Capitulum.} \scriptura{Rom. 6, 9-10}

\grechangedim{interwordspacetext}{0.12 cm plus 0.15 cm minus 0.05 cm}{scalable}%
\cuminitiali{}{temporalia/capitulum-ChristusResurgens.gtex}
\grechangedim{interwordspacetext}{0.32 cm plus 0.15 cm minus 0.05 cm}{scalable}%

% preklad Jeruz. bible
%\trCapituliI

\vfill

\pars{Responsorium breve.} \scriptura{Lc. 24, 34}

\cuminitiali{VI}{temporalia/respbr-vesp.gtex}

%\trResp

\vfill
\pagebreak

\pars{Hymnus}

\cuminitiali{VIII}{temporalia/hym-AdCoenam.gtex}
\vspace{-3mm}
%\begin{translatioMulticol}{4}
U~Beránkovy hostiny\\
oděni rouchy bílými,\\
když Rudým mořem prošli jsme,\\
Vladaři Kristu zpívejme.\\
\\
Když jeho tělem posvátným,\\
na kříži obětovaným,\\
se sytíme a~pijeme\\
jeho krev, v~Bohu žijeme.\columnbreak

Chráněni tímto pokrmem\\
před smrtonosným andělem,\\
svrhli jsme z~beder kruté jho\\
tyrana bezohledného.\\
\\
Kristus je naší paschou teď,\\
on sám se vydal za oběť\\
a~místo přesnic našim rtům\\
své tělo dává za pokrm.\columnbreak

Tys, nejčistější Oběti,\\
zlomila vládu podsvětí.\\
Z~otroctví lid je vykoupen,\\
odměna žití kyne všem.\\
\\
Hle, Kristus, když vstal ze hrobu,\\
jde z~pekel v~slavném průvodu\\
a~brány nebes otevřev,\\
vládce tmy vleče v~okovech.\columnbreak

Buď věčně, Kriste, věrným svým\\
plesáním velikonočním.\\
Nás, milostí tvou vzkříšené,\\
vem k~oslavě své vítězné. \\
\\
Sláva tobě, Pane,\\
jenž jsi vstal z~mrtvých,\\
s~Otcem i~Svatým Duchem\\
na věčné věky.\\
Amen.
\end{translatioMulticol}


\vfill
\pagebreak

\pars{Versus.} \scriptura{Lc. 24, 29}

% Versus. %%%
\sineinitiali{temporalia/versus-mane.gtex}

%\noindent \trVersus

\vfill
\pagebreak

\magnificat

\vspace{-1cm}

\vfill
\pagebreak

%\sideThumbs{{\scriptsize{}Fine horarum}}

\anteOrationem

\pagebreak

% Oratio. %%%
\oratioLaudes

\vspace{-1mm}
%\trOrationisI

\vfill

\rubrica{Hebdomadarius dicit iterum Dominus vobiscum. Postea cantatur a cantore:}
\vspace{2mm}

\cuminitiali{VII}{temporalia/benedicamus-tempore-paschali.gtex}

\vspace{1mm}
\fi

\end{document}

