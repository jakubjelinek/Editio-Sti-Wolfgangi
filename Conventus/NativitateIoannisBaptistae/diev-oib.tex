\newcommand{\titulus}{\nomenFesti{Dominica infra Octavam Nativitatis S. Ioannis Baptistæ.}
\celebratio{Semiduplex.}}
\newcommand{\solemnis}{Die V}
\newcommand{\aequus}{Die V}
\newcommand{\lectioi}{\pars{Lectio I.} \scriptura{1 Reg. 16, 1-5}

\noindent De libro primo Regum.

\noindent Dixítque Dóminus ad Samuélem: Usquequo tu luges Saul, cum ego proiécerim eum ne regnet super Israël? Imple cornu tuum óleo, et veni, ut mittam te ad Isai Bethlehemítem: provídi enim in fíliis eius mihi regem. Et ait Sámuel: Quómodo vadam? áudiet enim Saul, et interfíciet me. Et ait Dóminus: Vítulum de arménto tolles in manu tua, et dices: Ad immolándum Dómino veni. Et vocábis Isai ad víctimam, et ego osténdam tibi quid fácias, et unges quemcúmque monstrávero tibi. Fecit ergo Sámuel sicut locútus est ei Dóminus, venítque in Béthlehem, et admiráti sunt senióres civitátis occurréntes ei: dixerúntque: Pacificúsne est ingréssus tuus? Et ait: Pacíficus: ad immolándum Dómino veni: sanctificámini, et veníte mecum ut ímmolem. Sanctificávit ergo Isai et fílios eius, et vocávit eos ad sacrifícium. Nec enim discumbémus priúsquam huc ille véniat.}
\newcommand{\lectioii}{\pars{Lectio II.} \scriptura{1 Reg. 16, 6-11}

\noindent Cumque ingréssi essent, vidit Eliab, et ait: Num coram Dómino est Christus eius? Et dixit Dóminus ad Samuélem: Ne respícias vultum eius, neque altitúdinem statúræ eius: quóniam abiéci eum, nec iuxta intúitum hóminis ego iúdico: homo enim videt ea quæ parent, Dóminus autem intuétur cor. Et vocávit Isai Abínadab, et addúxit eum coram Samuéle. Qui dixit: Nec hunc elégit Dóminus. Addúxit autem Isai Samma, de quo ait: Etiam hunc non elégit Dóminus. Addúxit ítaque Isai septem fílios suos coram Samuéle: et ait Sámuel ad Isai: Non elégit Dóminus ex istis. Dixítque Sámuel ad Isai: Numquid iam compléti sunt fílii? Qui respóndit: Adhuc réliquus est párvulus, et pascit oves. Et ait Sámuel ad Isai: Mitte, et adduc eum.}
\newcommand{\lectioiii}{\pars{Lectio III.} \scriptura{1 Reg. 16, 12-19}

\noindent Misit ergo, et addúxit eum. Erat autem rufus, et pulcher aspéctu, decoráque fácie: et ait Dóminus: Surge, unge eum: ipse est enim. Tulit ergo Sámuel cornu ólei, et unxit eum in médio fratrum eius: et diréctus est spíritus Dómini a die illa in David, et deínceps. Surgénsque Sámuel ábiit in Ramatha. Spíritus autem Dómini recéssit a Saul, et exagitábat eum spíritus nequam a Dómino. Dixerúntque servi Saul ad eum: Ecce spíritus Dei malus exágitat te. Iúbeat dóminus noster, et servi tui qui coram te sunt quærent hóminem sciéntem psállere cíthara, ut quando arripúerit te spíritus Dómini malus, psallat manu sua, et lévius feras. Et ait Saul ad servos suos: Providéte ergo mihi áliquem bene psalléntem, et addúcite eum ad me. Et respóndens unus de púeris, ait: Ecce vidi fílium Isai Bethlehemítem sciéntem psállere, et fortíssimum róbore, et virum bellicósum, et prudéntem in verbis, et virum pulchrum: et Dóminus est cum eo. Misit ergo Saul núntios ad Isai, dicens: Mitte ad me David fílium tuum, qui est in páscuis.}
\newcommand{\lectioiv}{\pars{Lectio IV.} \scriptura{Homilia 2 in Psalm. 28}

\noindent Sermo sancti Basilíi Magni.

\noindent Vox Dómini super aquas. Qualis vox? super quas aquas? Velut prophetíam accipiámus quod dictum est. Memíneris Ioánnis, qui interrogátus a Iudǽis: Tu quis es? quod respónsum dábimus iis, qui misérunt nos? respóndit: Ego vox clamántis in desérto. Igitur vox Dómini est Ioánnes, Angelus a Deo missus ante fáciem Dómini, ut paráret Dómino plebem perféctam. Hæc ígitur vox super aquas, erat super Iordánem, in quo baptizábat prǽdicans pœniténtiæ baptísmum, et non solum in Iordáne, sed étiam in Ænon prope Salim, quia aquæ multæ erant illic.}
\newcommand{\lectiov}{\pars{Lectio V.}

\noindent Igitur vox Dómini super aquas, Ioánnes est super baptísmum. Illic et Deus maiestátis intónuit: venit enim vox de cælo, dicens: Hic est Fílius meus diléctus, in quo mihi complácui. Tunc étiam Dóminus super aquas multas dignátus est descéndere in baptísma Ioánnis, ut compléret omnem iustítiam quæ in lege est. Vox Dómini in virtúte. Auferet enim debilitátes pópuli per pœniténtiæ baptísmum, per ipsum baptízans in aqua ad pœniténtiam. In virtúte est vox, dicens: Pœniténtiam ágite, appropinquávit enim regnum cælórum: et, Fácite fructus dignos pœniténtiæ.}
\newcommand{\lectiovi}{\pars{Lectio VI.}

\noindent Vox Dómini confringéntis cedros. Potest dici, quod parans Dómino pópulum perféctum, elátas impietátes et contra cognitiónem Dei exaltátas confríngens, ac cónterens, oblíqua faciébat recta. Qui enim omnem collem ac montem humíliat, hic erat qui confringébat cedros, et Dómino viam adæquábat, per hoc quod ad pœniténtiam inducébat altum, et elátum, et supérbum cor. Unde eius præparatiónem suscípiens Dóminus, suo advéntu confrégit oppósitas poténtias, cedros Líbani figuráte dictas. Opórtet enim Dóminum regnáre, donec ponat inimícos sub pedes suos, et cedros istas commínuat.}
\newcommand{\lectiovii}{\pars{Lectio VII.} \scriptura{Lc. 5, 1-11}

\noindent Léctio sancti Evangélii secúndum Lucam.

\noindent In illo témpore: Cum turbæ irrúerent in Iesum, ut audírent verbum Dei, et ipse stabat secus stagnum Genésareth. Et réliqua.

\scriptura{Liber 4 in Lucæ cap. 5 prope finem libri}

\noindent Homilía sancti Ambrósii Epíscopi.

\noindent Ubi Dóminus multis impartívit vária génera sanitátum, nec témpore, nec loco pótuit ab stúdio sanándi turba cohibéri. Vesper incúbuit sequebántur: stagnum occúrrit, urgébant: et ídeo ascéndit in Petri navim. Hæc est illa navis, quæ adhuc secúndum Matthǽum flúctuat, secúndum Lucam replétur píscibus: ut et princípia Ecclésiæ fluctuántis, et posterióra exuberántis agnóscas. Pisces enim sunt, qui hanc enávigant vitam. Ibi adhuc discípulis Christus dormit, hic prǽcipit; dormit enim tépidis, perféctis vígilat.}
\newcommand{\lectioviii}{\pars{Lectio VIII.}

\noindent Deníque et si áliis imperátur, ut laxent rétia sua, soli tamen Petro dícitur, Duc in altum: hocest in profúndum disputatiónum. Quid enim tam altum, quam altitúdinem divitiárum vidére, scire dei fílium, et professiónem divínæ generatiónis assúmere? Quam licet mens humána non queat plena ratiónis investigatióne comprehéndere, fídei tamen plenitúdo compléctitur. Nam etsi non licet mihi scire, quemádmodum natus sit; non licet tamen nescíre, quod natus sit. Sériem generatiónis ignóro, sed auctoritátem generatiónis agnósco. Non interfúimus cum ex patre dei fílius nascerétur; sed interfúimus, cum a patre dei fílius dicerétur.}
\newcommand{\lectioix}{\pars{Lectio IX.}

\noindent Si deo non crédimus, cui crédimus? Omnia enim, quæ crédimus, uel visu crédimus, uel audítu. Visus sæpe fállitur, audítus in fide est. An asseréntis persóna discútitur? Si viri boni dícerent, nefas putarémus non crédere: Deus ásserit, probat fílius, refúgiens sol faterétur, tremens terra testátur. In hoc altum disputatiónis Ecclésia a Petro dúcitur: ut vídeat hinc resurgéntem dei fílium, inde spíritum sanctum profluéntem. Quæ sunt autem apostolórum quæ laxári iubéntur rétia, nisi verbórum complexiónes, et quasi quidam oratiónis sinus, et disputatiónum recéssus, qui eos quos cœ́perint non amíttant? Et bene apostólica instruménta piscándi rétia sunt, quæ non captos périmunt sed resérvant, et de profúndo ad lumen éxtrahunt.}
% LuaLaTeX

\documentclass[a4paper, twoside, 12pt]{article}
\usepackage[latin]{babel}
%\usepackage[landscape, left=3cm, right=1.5cm, top=2cm, bottom=1cm]{geometry} % okraje stranky
%\usepackage[landscape, a4paper, mag=1166, truedimen, left=2cm, right=1.5cm, top=1.6cm, bottom=0.95cm]{geometry} % okraje stranky
\usepackage[landscape, a4paper, mag=1400, truedimen, left=0.5cm, right=0.5cm, top=0.5cm, bottom=0.5cm]{geometry} % okraje stranky

\usepackage{fontspec}
\setmainfont[FeatureFile={junicode.fea}, Ligatures={Common, TeX}, RawFeature=+fixi]{Junicode}
%\setmainfont{Junicode}

% shortcut for Junicode without ligatures (for the Czech texts)
\newfontfamily\nlfont[FeatureFile={junicode.fea}, Ligatures={Common, TeX}, RawFeature=+fixi]{Junicode}

\usepackage{multicol}
\usepackage{color}
\usepackage{lettrine}
\usepackage{fancyhdr}

% usual packages loading:
\usepackage{luatextra}
\usepackage{graphicx} % support the \includegraphics command and options
\usepackage{gregoriotex} % for gregorio score inclusion
\usepackage{gregoriosyms}
\usepackage{wrapfig} % figures wrapped by the text
\usepackage{parcolumns}
\usepackage[contents={},opacity=1,scale=1,color=black]{background}
\usepackage{tikzpagenodes}
\usepackage{calc}
\usepackage{longtable}
\usetikzlibrary{calc}

\setlength{\headheight}{14.5pt}

\input{conventuscommune.tex} % Often used macros
%%%% Preklady jednotlivych zpevu (nektere se opakuji, a je dobre mit je
% vsechny na jedne hromade)

% HOURS ---

\newcommand{\trAntI}{\translatioCantus{Muž boží měl kožený toulec, pečlivě
zavázaný, jenž mu visel na šíji a~často se ho dotýkal.}}

\newcommand{\trAntII}{\translatioCantus{Klíč od~něho tak dobře střežil, že
dokud žil v~těle, nikdo z~jeho žáků nezvěděl, co je uvnitř.}}

\newcommand{\trAntIII}{\translatioCantus{Ale když se odebral z~tohoto
života, schránku otevřeli a~objevili v~ní žíněné roucho a~měděný řetěz
potřísněný krví.}}

\newcommand{\trAntIV}{\translatioCantus{A když prohlédli mistrovo tělo,
nalezli jeho tělo na čtyřech místech hluboce zbrázděno ranami od řetězu.}}

\newcommand{\trAntV}{\translatioCantus{Krev vytékající z~těch ran, místy
prostoupila i~žíněným rouchem.}}

\newcommand{\trCapituli}{\translatioCantus{
Miláčkovi Boha a~lidí,
Mojžíšovi požehnané paměti,~\gredagger{}
dopřál slávu rovnou slávě svatých~\grestar{}
učinil ho mocným na postrach nepřátelům
a~jeho slovy zastavil divy.}}

\newcommand{\trLectioBrevis}{\translatioCantus{
Pamatujte na své představené,
kteří vám hlásali Boží slovo.
Uvažte, jak oni skončili život, a~napodobujte jejich víru.
Ježíš Kristus je stejný včera i~dnes i~navěky.
Nenechte se svést věelijakými cizími naukami.}}

\newcommand{\trRespLaud}{\translatioCantus{Spravedlivého vodil Hospodin~\grestar{}
po přímých stezkách. \Vbardot{} A~ukázal mu Boží království.}}

\newcommand{\trRespLaudB}{\translatioCantus{Na tvých hradbách, Jeruzaléme,
ustanovil jsem strážné;~\grestar{}
budou bdít nad mým lidem. \Vbardot{} Ani ve dne, ani v~noci nesmějí nikdy
mlčet.}}

\newcommand{\trVersus}{\translatioCantus{\Vbardot{} Ústa spravedlivého šeptají moudrost, aleluja.
\Rbardot{} A~jeho jazyk ohlašuje právo, aleluja.}}

\newcommand{\trAntBenedictus}{\translatioCantus{Když na bujné oře vložili
nosítka a~sňali jim uzdu, vydali se přímo k~cele božího muže.}}

\newcommand{\trPreces}{\translatioCantus{
\noindent S vděčností chvalme Krista, dobrého Pastýře, \gredagger{} který dal život za své ovce, \grestar{} a~pokorně ho prosme: \Rbardot{} Pane, buď pastýřem svého lidu.

\noindent Kriste, ty dáváš církvi pastýře, a~jejich službou se ujímáš svého lidu, \grestar{} dej, ať v~lásce těch, kteří nás vedou, poznáváme, jak nás miluješ. \Rbardot{} Pane, buď pastýřem svého lidu.

\noindent Ty stále konáš skrze své zástupce službu pastýře a~učitele, \grestar{} nepřestávej nás nikdy vést prostřednictvím svých služebníků. \Rbardot{} Pane, buď pastýřem svého lidu.

\noindent Ty prokazuješ svému lidu skrze jeho pastýře službu lékaře duše i~těla, \grestar{} ochraňuj náš život a~veď nás ke svatosti. \Rbardot{} Pane, buď pastýřem svého lidu.

\noindent Ty posíláš své svaté, aby slovem i~příkladem vedli tvůj lid k~tobě, \grestar{} na jejich přímluvu nás posiluj, abychom vytrvali na cestě, která vede k~věčnému životu. \Rbardot{} Pane, buď pastýřem svého lidu.}}

\newcommand{\trOrationis}{\translatioCantus{Bože, jenž nám dopřáváš radovat
se z~výroční slavnosti svatého tvého vyznavače Havla, uděl dobrotivě,
abychom když slavíme jeho narození, též se řídili podobou jeho skutků.
Skrze…}}
 % Czech translations of the proper texts

\newcommand{\annusEditionis}{2020}

%%%% Vicekrat opakovane kousky

\newcommand{\anteOrationem}{
  \rubrica{Ante Orationem, cantatur a Superiore:}

  \pars{Supplicatio Litaniæ.}

  \cuminitiali{}{temporalia/supplicatiolitaniae.gtex}

  \pars{Oratio Dominica.}

  \cuminitiali{}{temporalia/oratiodominica.gtex}

  \rubrica{Deinde dicitur ab Hebdomadario:}

  \cuminitiali{}{temporalia/dominusvobiscum-solemnis.gtex}

  \rubrica{In choro monialium loco Dominus vobiscum dicitur:}

  \sineinitiali{temporalia/domineexaudi.gtex}
}

\setlength{\columnsep}{30pt} % prostor mezi sloupci

%%%%%%%%%%%%%%%%%%%%%%%%%%%%%%%%%%%%%%%%%%%%%%%%%%%%%%%%%%%%%%%%%%%%%%%%%%%%%%%%%%%%%%%%%%%%%%%%%%%%%%%%%%%%%
\begin{document}

% Here we set the space around the initial.
% Please report to http://home.gna.org/gregorio/gregoriotex/details for more details and options
\grechangedim{afterinitialshift}{2.2mm}{scalable}
\grechangedim{beforeinitialshift}{2.2mm}{scalable}
\grechangedim{interwordspacetext}{0.22 cm plus 0.15 cm minus 0.05 cm}{scalable}%
\grechangedim{annotationraise}{-0.2cm}{scalable}

% Here we set the initial font. Change 38 if you want a bigger initial.
% Emit the initials in red.
\grechangestyle{initial}{\color{red}\fontsize{38}{38}\selectfont}

\pagestyle{empty}

%%%% Titulni stranka
\begin{titulusOfficii}
\titulus
\end{titulusOfficii}

% graphic
%\vspace{1.5cm}
%\begin{center}
%\includegraphics[width=8cm]{emmaus.jpg}
%\end{center}

\vfill

\begin{center}
%Ad usum et secundum consuetudines chori \guillemotright{}Conventus Choralis\guillemotleft.

%Editio Sancti Wolfgangi \annusEditionis
\end{center}

\pagebreak

\renewcommand{\headrulewidth}{0pt} % no horiz. rule at the header
\fancyhf{}
\pagestyle{fancy}

\cantusSineNeumas

\ifx\festumveldominica\undefined
\else
\pars{Oratio ante divinum Officium.}

\lettrine{{\color{red}A}}{peri,} Dómine, os meum ad benedicéndum nomen sanctum tuum:
munda quoque cor meum ab ómnibus vanis, pervérsis, et aliénis
cogitatiónibus:
intelléctum illúmina, afféctum inflámma,
ut digne, atténte ac devóte hoc Offícium recitáre váleam,
et exaudíri mérear ante conspéctum Divínæ Maiestátis tuæ.
Per Christum, Dóminum nostrum.
\Rbardot{} Amen.

Dómine, in unióne illíus divínæ intentiónis,
qua ipse in terris laudes Deo persolvísti,
has tibi Horas \rubricatum{(vel \textnormal{hanc tibi Horam})} persólvo.

%\trOratioAnteOfficium

\vfill

\pars{Oratio post divinum Officium.}

\rubrica{
  Orationem sequentem devote post Officium recitantibus
  Leo Papa X. defectus, et culpas in eo persolvendo ex humana
  fragilitate contractas, indulsit, et dicitur flexis genibus.
}

\lettrine{{\color{red}S}}{acrosánctæ} et indivíduæ Trinitáti,
crucifíxi Dómini nostri Iesu Christi humanitáti,
beatíssimæ et gloriosíssimæ sempérque Vírginis Maríæ
fecúndæ integritáti, 
et ómnium Sanctórum universitáti
sit sempitérna laus, honor, virtus et glória
ab omni creatúra,
nobísque remíssio ómnium peccatórum,
per infiníta sǽcula sæculórum.
\Rbardot{} Amen.

\noindent \Vbardot{} Beáta víscera Maríæ Virginis, quæ portavérunt
ætérni Patris Fílium.\\
\Rbardot{} Et beáta úbera, quæ lactavérunt Christum Dominum.

\rubrica{Et dicitur secreto \textnormal{Pater noster.} et \textnormal{Ave María.}}

%\trOratioPostOfficium

\vfill

\ifx\festum\undefined
\else
\hora{Ad I. Vesperas.} %%%%%%%%%%%%%%%%%%%%%%%%%%%%%%%%%%%%%%%%%%%%%%%%%%%%%
%\sideThumbs{I. Vesperæ}

\vspace{0.5cm}
\grechangedim{interwordspacetext}{0.18 cm plus 0.15 cm minus 0.05 cm}{scalable}%
\ifx\festum\undefined
\cuminitiali{}{temporalia/deusinadiutorium-alter.gtex}
\else
\cuminitiali{}{temporalia/deusinadiutorium-solemnis.gtex}
\fi
\grechangedim{interwordspacetext}{0.22 cm plus 0.15 cm minus 0.05 cm}{scalable}%

\vfill
\pagebreak

\pars{Psalmus 1.} \scriptura{Cf. Lc. 1, 11.13; \textbf{H273}}

\vspace{-0.4cm}

\antiphona{VII a}{temporalia/ant-descenditangelus.gtex}

\scriptura{Psalmus 112.}

\initiumpsalmi{temporalia/ps112-initium-vii-a-auto.gtex}

%\psalmusEtTranslatioT{temporalia/ps112-comb.tex}{10cm}
\input{temporalia/ps112.tex}

\vspace{-1cm}

\vfill
\pagebreak

\pars{Psalmus 2.} \scriptura{Psalmus 116.}

\initiumpsalmi{temporalia/ps116-initium-vii-a-auto.gtex}

%\psalmusEtTranslatioT{temporalia/ps116-comb.tex}{10cm}
\input{temporalia/ps116.tex}

\vfill
\pagebreak

\pars{Psalmus 3.} \scriptura{Psalmus 145.}

\initiumpsalmi{temporalia/ps145-initium-vii-a-auto.gtex}

%\psalmusEtTranslatioT{temporalia/ps145-comb.tex}{10cm}
\input{temporalia/ps145.tex}

\vfill
\pagebreak

\pars{Psalmus 4.} \scriptura{Psalmus 146.}

\initiumpsalmi{temporalia/ps146-initium-vii-a-auto.gtex}

%\psalmusEtTranslatioT{temporalia/ps146-comb.tex}{10cm}
\input{temporalia/ps146.tex} \Abardot{}

\vfill
\pagebreak

\pars{Psalmus 5.} \scriptura{Psalmus 147.}

\initiumpsalmi{temporalia/ps147-initium-vii-a-auto.gtex}

%\psalmusEtTranslatioT{temporalia/ps147-comb.tex}{10cm}
\input{temporalia/ps147.tex} \Abardot{}

\vfill

\antiphona{}{temporalia/ant-descenditangelus.gtex} % repeat the antiphon - new page

\vfill
\pagebreak

\pars{Capitulum.} \scriptura{Ier. 1, 5}

\grechangedim{interwordspacetext}{0.12 cm plus 0.15 cm minus 0.05 cm}{scalable}%
\cuminitiali{}{temporalia/capitulum-PriusquamTe.gtex}
\grechangedim{interwordspacetext}{0.22 cm plus 0.15 cm minus 0.05 cm}{scalable}

% preklad Jeruz. bible
%\trCapituliI

\vfill

\pars{Responsorium breve.} \scriptura{Cf. Lc. 7, 28}

\cuminitiali{VI}{temporalia/resp-internatos.gtex}

%\trResp

\vfill
\pagebreak

\pars{Hymnus}

\cuminitiali{II}{temporalia/hym-UtQueant.gtex}
\vspace{-3mm}
%\input{hym-UtQueant-bohtext.tex}

\vfill
%\pagebreak

\pars{Versus.} \scriptura{Ps. 91, 13}

% Versus. %%%
\sineinitiali{temporalia/versus-iustus.gtex}

%\noindent \trVersus

\vfill
\pagebreak

\pars{Canticum B. Mariæ V.} \scriptura{Lc. 1, 9.11}

\vspace{-3mm}

{
\grechangedim{interwordspacetext}{0.18 cm plus 0.15 cm minus 0.05 cm}{scalable}%
\antiphona{VIII G}{temporalia/ant-ingressozacharia.gtex}
\grechangedim{interwordspacetext}{0.22 cm plus 0.15 cm minus 0.05 cm}{scalable}%
}

%\trAntIMagnificat

%\vspace{-2mm}

\scriptura{Lc. 1, 46-55}

%\vspace{-2mm}

\cantusSineNeumas
\initiumpsalmi{temporalia/magnificat-initium-viiisoll-G.gtex}

%\psalmusEtTranslatioT{temporalia/magnificat-I-comb.tex}{10.2cm}
\input{temporalia/magnificat-I.tex} \Abardot{}

%\vspace{-1cm}

\vfill
\pagebreak

%\sideThumbs{{\scriptsize{}Fine horarum}}

\anteOrationem

\pagebreak

% Oratio. %%%
\cuminitiali{}{temporalia/oratio.gtex}

\vspace{-1mm}
%\trOrationisI

\vfill

\rubrica{Hebdomadarius dicit iterum Dominus vobiscum, vel cantor dicit:}

\vspace{2mm}

\sineinitiali{temporalia/domineexaudi.gtex}

\rubrica{Postea cantatur a cantore:}

\vspace{2mm}

\ifx\festum\undefined
\cuminitiali{II}{temporalia/benedicamus-semiduplex-vesp.gtex}
\else
\cuminitiali{II}{temporalia/benedicamus-solemnism-1vesp.gtex}
\fi

\vspace{1mm}

\vfill
\pagebreak
\fi

\iffalse
\hora{Ad Matutinum.} %%%%%%%%%%%%%%%%%%%%%%%%%%%%%%%%%%%%%%%%%%%%%%%%%%%%%
%\sideThumbs{Matutinum}

\vspace{2mm}

\cuminitiali{}{temporalia/dominelabiamea.gtex}

\vspace{2mm}

\pars{Invitatorium.} \scriptura{Cantor; \textbf{LU\textsubscript{918}}}

\vspace{-6mm}

\antiphona{IV}{temporalia/inv-christumregemadoremus.gtex}

\vfill
\pagebreak

\pars{Hymnus.}

\vspace{-5mm}

\scriptura{Thomas de Aquino; \textbf{LU\textsubscript{920}}}

\antiphona{IV}{temporalia/hym-SacrisSolemniis.gtex}
%{
%\vspace{-5mm}
%\setlength{\columnsep}{0pt} % prostor mezi sloupci
%\input{hym-SacrisSolemniis-bohtext.tex}
%\setlength{\columnsep}{30pt} % prostor mezi sloupci
%}

\vfill
\pagebreak

\subhora{In I. Nocturno}

\pars{Psalmus 1.} \scriptura{Ps. 1, 3; \textbf{LU\textsubscript{922}}}

\vspace{-2mm}

\antiphona{I D*}{temporalia/ant-fructumsalutiferum.gtex}

%\vspace{-5mm}

\scriptura{Ps. 1}

%\vspace{-2mm}

\initiumpsalmi{temporalia/ps1-initium-i-D_-auto.gtex}

%\psalmusEtTranslatioT{temporalia/ps1-comb.tex}{10cm}
\input{temporalia/ps1.tex} \Abardot{}

\vfill
\pagebreak

\pars{Psalmus 2.} \scriptura{Ps. 4, 8.9; \textbf{LU\textsubscript{923}}}

\vspace{-2mm}

\antiphona{II D}{temporalia/ant-afructufrumenti.gtex}

%\vspace{-5mm}

\scriptura{Ps. 4}

\initiumpsalmi{temporalia/ps4-initium-ii-D-auto.gtex}

%\psalmusEtTranslatioT{temporalia/ps4iiD-comb.tex}{10cm}
\input{temporalia/ps4iiD.tex} \Abardot{}

\vfill
\pagebreak

\pars{Psalmus 3.} \scriptura{Ps. 15, 4; \textbf{LU\textsubscript{924}}}

\vspace{-4mm}

\antiphona{III a\textsuperscript{3}}{temporalia/ant-communionecalicis.gtex}

%\vspace{-2mm}

\scriptura{Ps. 15}

\vspace{-2mm}

\initiumpsalmi{temporalia/ps15-initium-iii-a3-auto.gtex}

%\psalmusEtTranslatioT{temporalia/ps15-comb.tex}{10cm}
\input{temporalia/ps15.tex} \Abardot{}

\vfill
\pagebreak

\pars{Versus.} \scriptura{Sap. 16, 20; Ps. 77, 25}

% Versus. %%%
\sineinitiali{temporalia/versus-panemdecaelohomo-communis.gtex}

\vspace{5mm}

\sineinitiali{temporalia/oratiodominica-mat.gtex}

\vspace{5mm}

\pars{Absolutio.}

\cuminitiali{}{temporalia/absolutio-exaudi.gtex}

\vfill
\pagebreak

\cuminitiali{}{temporalia/benedictio-solemn-benedictione.gtex}

\vspace{7mm}

\lectioi

\noindent \Vbardot{} Tu autem, Dómine, miserére nobis.
\noindent \Rbardot{} Deo grátias.

\vfill
\pagebreak

\pars{Responsorium 1.} \scriptura{\Rbardot{} Ex. 12, 5.6.8 \Vbardot{} 1 Cor. 5, 7.8; \textbf{LU\textsubscript{926}}}

\vspace{-2mm}

\responsorium{I}{temporalia/resp-immolabithaedum.gtex}{}

\vfill
\pagebreak

\cuminitiali{}{temporalia/benedictio-solemn-unigenitus.gtex}

\vspace{7mm}

\lectioii

\noindent \Vbardot{} Tu autem, Dómine, miserére nobis.
\noindent \Rbardot{} Deo grátias.

\vfill
\pagebreak

\pars{Responsorium 2.} \scriptura{\Rbardot{} Ex. 16, 12.15 \Vbardot{} Io. 6, 21; \textbf{LU\textsubscript{927}}}

\vspace{-2mm}

\responsorium{II}{temporalia/resp-comedetiscarnes.gtex}{}

\vfill
\pagebreak

\cuminitiali{}{temporalia/benedictio-solemn-spiritus.gtex}

\vspace{7mm}

\lectioiii

\noindent \Vbardot{} Tu autem, Dómine, miserére nobis.
\noindent \Rbardot{} Deo grátias.

\vfill
\pagebreak

\pars{Responsorium 3.} \scriptura{\Rbardot{} 3 Reg. 19, 6.8 \Vbardot{} Io. 6, 52; \textbf{LU\textsubscript{927}}}

\vspace{-2mm}

\responsorium{III}{temporalia/resp-respexitelias.gtex}{}

\vfill
\pagebreak

\subhora{In II. Nocturno}

\pars{Psalmus 4.} \scriptura{Ps. 19, 4; \textbf{LU\textsubscript{928}}}

\vspace{-2mm}

\antiphona{IV E}{temporalia/ant-memorsit-FKP.gtex}

\vspace{-2mm}

\scriptura{Ps. 19}

\initiumpsalmi{temporalia/ps19-initium-iv-E-auto.gtex}

%\psalmusEtTranslatioT{temporalia/ps19-comb.tex}{10cm}
\input{temporalia/ps19.tex} \Abardot{}

\vfill
\pagebreak

\pars{Psalmus 5.} \scriptura{Ps. 22, 5; \textbf{LU\textsubscript{928}}}

\vspace{-2mm}

\antiphona{V a}{temporalia/ant-paraturnobis-FKP.gtex}

%\vspace{-3mm}

\scriptura{Ps. 22}

%\vspace{-2mm}

\initiumpsalmi{temporalia/ps22-initium-v-a.gtex}

%\vspace{-1.5mm}

%\psalmusEtTranslatioT{temporalia/ps22-comb.tex}{10cm}
\input{temporalia/ps22.tex} \Abardot{}

\vspace{-1cm}

\vfill
\pagebreak

\pars{Psalmus 6.} \scriptura{Ps. 41, 5; \textbf{LU\textsubscript{930}}}

\vspace{-2mm}

\antiphona{VI F}{temporalia/ant-invoceexsultationis-FKP.gtex}

%\vspace{-5mm}

\scriptura{Ps. 41}

\initiumpsalmi{temporalia/ps41-initium-vi-F-auto.gtex}

%\psalmusEtTranslatioT{temporalia/ps41-comb.tex}{10cm}
\input{temporalia/ps41.tex}

\vfill

\antiphona{}{temporalia/ant-invoceexsultationis-FKP.gtex}

\vfill
\pagebreak

\pars{Versus.} \scriptura{Ps. 80, 17}

% Versus. %%%
\sineinitiali{temporalia/versus-cibavit.gtex}

\vspace{5mm}

\sineinitiali{temporalia/oratiodominica-mat.gtex}

\vspace{5mm}

\pars{Absolutio.}

\cuminitiali{}{temporalia/absolutio-ipsius.gtex}

\vfill
\pagebreak

\cuminitiali{}{temporalia/benedictio-solemn-deus.gtex}

\vspace{7mm}

\lectioiv

\noindent \Vbardot{} Tu autem, Dómine, miserére nobis.
\noindent \Rbardot{} Deo grátias.

\vfill
\pagebreak

\pars{Responsorium 4.} \scriptura{\Rbardot{} Mt. 26, 26 \Vbardot{} Io. 31, 31; \textbf{LU\textsubscript{931}}}

\vspace{-2mm}

\responsorium{V}{temporalia/resp-coenantibus.gtex}{}

\vfill
\pagebreak

\cuminitiali{}{temporalia/benedictio-solemn-christus.gtex}

\vspace{7mm}

\lectiov

\noindent \Vbardot{} Tu autem, Dómine, miserére nobis.
\noindent \Rbardot{} Deo grátias.

\vfill
\pagebreak

\pars{Responsorium 5.} \scriptura{\Rbardot{} 1 Cor. 11, 25 \Vbardot{} Thren. 3, 20; \textbf{LU\textsubscript{932}}}

\vspace{-2mm}

\responsorium{VI}{temporalia/resp-accepitiesus.gtex}{}

\vfill
\pagebreak

\cuminitiali{}{temporalia/benedictio-solemn-ignem.gtex}

\vspace{7mm}

\lectiovi

\noindent \Vbardot{} Tu autem, Dómine, miserére nobis.
\noindent \Rbardot{} Deo grátias.

\vfill
\pagebreak

\pars{Responsorium 6.} \scriptura{\Rbardot{} Io. 6, 48 \Vbardot{} ibid. 6, 51.52; \textbf{LU\textsubscript{934}}}

\vspace{-2mm}

\responsorium{VII}{temporalia/resp-egosumpanisvitae.gtex}{}

\vfill
\pagebreak

\subhora{In III. Nocturno}

\pars{Psalmus 7.} \scriptura{Ps. 42, 4; \textbf{LU\textsubscript{934}}}

\vspace{-5mm}

\antiphona{VII a}{temporalia/ant-introibo-FKP.gtex}

\vspace{-4mm}

\scriptura{Ps. 42}

%\vspace{-2mm}

\initiumpsalmi{temporalia/ps42-initium-vii-a-auto.gtex}

%\psalmusEtTranslatioT{temporalia/ps42-comb.tex}{10cm}
\input{temporalia/ps42.tex} \Abardot{}

\vfill
\pagebreak

\pars{Psalmus 8.} \scriptura{Ps. 80, 17; \textbf{LU\textsubscript{935}}}

\vspace{-5mm}

\antiphona{VIII G}{temporalia/ant-cibavitnos-FKP.gtex}

\vspace{-3mm}

\scriptura{Ps. 80}

\vspace{-2mm}

\initiumpsalmi{temporalia/ps80-initium-viii-G-auto.gtex}

\vspace{-1mm}

%\psalmusEtTranslatioT{temporalia/ps80-comb.tex}{10cm}
\input{temporalia/ps80.tex} \Abardot{}

\vfill
\pagebreak

\pars{Psalmus 9.} \scriptura{Ps. 83, 3; \textbf{LU\textsubscript{936}}}

\vspace{-2mm}

\antiphona{VI F}{temporalia/ant-exaltari-FKP.gtex}

\vspace{-2mm}

\scriptura{Ps. 83}

\initiumpsalmi{temporalia/ps83-initium-vi-F-auto.gtex}

%\psalmusEtTranslatioT{temporalia/ps83-comb.tex}{10cm}
\input{temporalia/ps83.tex} \Abardot{}

\vfill
\pagebreak

\pars{Versus.} \scriptura{Ps. 103, 14-15}

% Versus. %%%
\sineinitiali{temporalia/versus-educas.gtex}

\vspace{5mm}

\sineinitiali{temporalia/oratiodominica-mat.gtex}

\vspace{5mm}

\pars{Absolutio.}

\cuminitiali{}{temporalia/absolutio-avinculis.gtex}

\vfill
\pagebreak

\cuminitiali{}{temporalia/benedictio-solemn-evangelica.gtex}

\vspace{7mm}

\lectiovii

\noindent \Vbardot{} Tu autem, Dómine, miserére nobis.
\noindent \Rbardot{} Deo grátias.

\vfill
\pagebreak

\pars{Responsorium 7.} \scriptura{\Rbardot{} Io. 6, 57 \Vbardot{} Dt. 4, 7; \textbf{LU\textsubscript{938}}}

\vspace{-2mm}

\responsorium{VII}{temporalia/resp-quimanducat.gtex}{}

\vfill
\pagebreak

\cuminitiali{}{temporalia/benedictio-solemn-divinum.gtex}

\vspace{7mm}

\lectioviii

\noindent \Vbardot{} Tu autem, Dómine, miserére nobis.
\noindent \Rbardot{} Deo grátias.

\vfill
\pagebreak

\ifx\dominica\undefined
\pars{Responsorium 8.} \scriptura{\Rbardot{} Io. 6, 58 \Vbardot{} Eccli. 15, 3; \textbf{LU\textsubscript{938}}}

\vspace{-2mm}

\responsorium{VIII}{temporalia/resp-misitmevivenspater.gtex}{}
\else
\pars{Responsorium 8.} \scriptura{\Rbardot{} Lc. 14, 16-17 \Vbardot{} Prv. 9, 5}

\vspace{-2mm}

\responsorium{VI}{temporalia/resp-homoquidamfecit.gtex}{}
\fi

\vfill
\pagebreak

\cuminitiali{}{temporalia/benedictio-solemn-adsocietatem.gtex}

\vspace{7mm}

\lectioix

\noindent \Vbardot{} Tu autem, Dómine, miserére nobis.
\noindent \Rbardot{} Deo grátias.

\vfill
\pagebreak

% Te Deum

%\pars{Hymnus Ambrosianus}

\vspace{-5mm}

\ifx\solemnis\undefined
\ifx\aequus\undefined
{
\pars{Hymnus Ambrosianus} \scriptura{Alio modo, iuxta morem Romanum}

\vspace{-2mm}

\grechangedim{interwordspacetext}{0.26 cm plus 0.15 cm minus 0.05 cm}{scalable}%
\cuminitiali{III}{temporalia/tedeum-romanum-gn.gtex}
\grechangedim{interwordspacetext}{0.22 cm plus 0.15 cm minus 0.05 cm}{scalable}%
}
\else
{
\pars{Hymnus Ambrosianus} \scriptura{Tonus Simplex}

\vspace{-2mm}

\grechangedim{interwordspacetext}{0.30 cm plus 0.15 cm minus 0.05 cm}{scalable}%
\cuminitiali{III}{temporalia/tedeum-simplex-gn.gtex}
\grechangedim{interwordspacetext}{0.22 cm plus 0.15 cm minus 0.05 cm}{scalable}%
}
\fi
\else
{
\pars{Hymnus Ambrosianus} \scriptura{Tonus Solemnis}

\vspace{-2mm}

\grechangedim{interwordspacetext}{0.26 cm plus 0.15 cm minus 0.05 cm}{scalable}%
\cuminitiali{III}{temporalia/tedeum-solemnis-gn.gtex}
\grechangedim{interwordspacetext}{0.22 cm plus 0.15 cm minus 0.05 cm}{scalable}%
}
\fi

\vfill
\pagebreak

\rubrica{Reliqua omittuntur, nisi Laudes separandæ sint.}

\sineinitiali{temporalia/domineexaudi.gtex}

\vfill

\pars{Oratio.}

\cuminitiali{}{temporalia/oratio2.gtex}

\vfill

\noindent \Vbardot{} Dómine, exáudi oratiónem meam.
\Rbardot{} Et clamor meus ad te véniat.

\vfill

% Nocturnale Romanum 2002, p. LXXVI Benedicamus Domino seems to match
% the one from Solemn Laudes.
\cuminitiali{V}{temporalia/benedicamus-solemnis-laud.gtex}

\vfill

\noindent \Vbardot{} Fidélium ánimæ per misericórdiam Dei requiéscant in pace.
\Rbardot{} Amen.

\vfill
\pagebreak
\fi

\hora{Ad Laudes.} %%%%%%%%%%%%%%%%%%%%%%%%%%%%%%%%%%%%%%%%%%%%%%%%%%%%%
%\sideThumbs{Laudes}

\cantusSineNeumas

\vspace{0.5cm}
\grechangedim{interwordspacetext}{0.18 cm plus 0.15 cm minus 0.05 cm}{scalable}%
\ifx\festumveldominica\undefined
\cuminitiali{}{temporalia/deusinadiutorium-communis.gtex}
\else
\cuminitiali{}{temporalia/deusinadiutorium-alter.gtex}
\fi
\grechangedim{interwordspacetext}{0.22 cm plus 0.15 cm minus 0.05 cm}{scalable}%

\vfill
%\pagebreak

\pars{Psalmus 1.} \scriptura{Lc. 1, 13; \textbf{H277}}

\vspace{-0.4cm}

\antiphona{III a}{temporalia/ant-elisabethzachariae.gtex}

\scriptura{Psalmus 92.}

\initiumpsalmi{temporalia/ps92-initium-iii-a-auto.gtex}

%\psalmusEtTranslatioT{temporalia/ps92-comb.tex}{10cm}
\input{temporalia/ps92.tex} \Abardot{}

\vfill
\pagebreak

\pars{Psalmus 2.} \scriptura{Lc. 1, 62.63; \textbf{H277}}

\vspace{-0.4cm}

\antiphona{IV E*}{temporalia/ant-innuebantpatriejus.gtex}

\scriptura{Psalmus 99.}

\initiumpsalmi{temporalia/ps99-initium-iv-E_-auto.gtex}

%\psalmusEtTranslatioT{temporalia/ps99-comb.tex}{10cm}
\input{temporalia/ps99.tex} \Abardot{}

\vfill
\pagebreak

\pars{Psalmus 3.} \scriptura{Lc. 1, 13.14; \textbf{H277}}

\vspace{-0.4cm}

\antiphona{I f}{temporalia/ant-joannesvocabitur.gtex}

\scriptura{Psalmus 62.}

\initiumpsalmi{temporalia/ps62-initium-i-f-auto.gtex}

%\psalmusEtTranslatioT{temporalia/ps62-comb.tex}{10cm}
\input{temporalia/ps62.tex} \Abardot{}

\vfill
\pagebreak

\pars{Psalmus 4.} \scriptura{Lc. 1, 15.14.63; \textbf{H275}}

\vspace{-0.4cm}

\antiphona{VIII G}{temporalia/ant-joannesest.gtex}

\scriptura{Canticum trium puerorum, Dan. 3, 57-88 et 56}

\initiumpsalmi{temporalia/dan3-initium-viii-G-auto.gtex}

%\psalmusEtTranslatioT{temporalia/dan3-comb.tex}{10cm}
\input{temporalia/dan3.tex}

\rubrica{Hic non dicitur Gloria Patri, neque Amen.}

\vfill

\vspace{-6mm}

\antiphona{}{temporalia/ant-joannesest.gtex} % repeat the antiphon - new page

\vfill
\pagebreak

\pars{Psalmus 5.} \scriptura{Lc. 1, 15.66; \textbf{H276}}

\vspace{-0.4cm}

\antiphona{VII a}{temporalia/ant-istepuer.gtex}

\scriptura{Psalmus 148.}

\initiumpsalmi{temporalia/ps148-initium-vii-a-auto.gtex}

%\psalmusEtTranslatioT{temporalia/ps148-comb.tex}{10cm}
\input{temporalia/ps148.tex}

\rubrica{Hic non dicitur Gloria Patri.}

\vfill
\pagebreak

%
\scriptura{Psalmus 149.}

\initiumpsalmi{temporalia/ps149-initium-vii-a-auto.gtex}

%\psalmusEtTranslatioT{temporalia/ps149-comb.tex}{10cm}
\input{temporalia/ps149.tex}

\rubrica{Hic non dicitur Gloria Patri.}

\vfill
\pagebreak

%
\scriptura{Psalmus 150.}

\initiumpsalmi{temporalia/ps150-initium-vii-a-auto.gtex}

%\psalmusEtTranslatioT{temporalia/ps150-comb.tex}{10cm}
\input{temporalia/ps150.tex}

\vfill

\vspace{-6mm}

\antiphona{}{temporalia/ant-istepuer.gtex} % repeat the antiphon - new page

\vfill
\pagebreak

\pars{Capitulum.} \scriptura{Is. 49, 1}

\grechangedim{interwordspacetext}{0.12 cm plus 0.15 cm minus 0.05 cm}{scalable}%
\cuminitiali{}{temporalia/capitulum-HaecDicit.gtex}
\grechangedim{interwordspacetext}{0.22 cm plus 0.15 cm minus 0.05 cm}{scalable}

% preklad Jeruz. bible
%\trCapituliI

\vfill

\pars{Responsorium breve.} \scriptura{Cf. Lc. 7, 28}

\ifx\festum\undefined
\cuminitiali{VI}{temporalia/resp-internatos-communis.gtex}
\else
\cuminitiali{VI}{temporalia/resp-internatos.gtex}
\fi

%\trResp

\vfill
\pagebreak

\pars{Hymnus}

\cuminitiali{IV}{temporalia/hym-ONimis.gtex}
\vspace{-3mm}
%\input{hym-ONimis-bohtext.tex}

\vfill
%\pagebreak

\pars{Versus.} \scriptura{Ps. 91, 13}

% Versus. %%%
\sineinitiali{temporalia/versus-iustus.gtex}

%\noindent \trVersus

\vfill
\pagebreak

\pars{Canticum Zachariæ.} \scriptura{Lc. 1, 64.67.68; \textbf{H277}}

\vspace{-4mm}

{
\grechangedim{interwordspacetext}{0.18 cm plus 0.15 cm minus 0.05 cm}{scalable}%
\antiphona{VIII G}{temporalia/ant-apertumestoszachariae.gtex}
\grechangedim{interwordspacetext}{0.22 cm plus 0.15 cm minus 0.05 cm}{scalable}%
}

%\trAntIMagnificat

\vspace{-2mm}

\scriptura{Lc. 1, 68-79}

\vspace{-1mm}

\cantusSineNeumas
\ifx\solemnis\undefined
\initiumpsalmi{temporalia/benedictus-initium-viii-G-auto.gtex}

%\vspace{-1.5mm}

%\psalmusEtTranslatioT{temporalia/benedictus-III-comb.tex}{10.2cm}
\input{temporalia/benedictus-III.tex} \Abardot{}
\else
\initiumpsalmi{temporalia/benedictus-initium-viiisoll-G-auto.gtex}

%\vspace{-1.5mm}

%\psalmusEtTranslatioT{temporalia/benedictus-I-comb.tex}{10.2cm}
\input{temporalia/benedictus-I.tex} \Abardot{}
\fi

\vspace{-1cm}

\vfill
\pagebreak

%\sideThumbs{{\scriptsize{}Fine horarum}}

\anteOrationem

\pagebreak

% Oratio. %%%
\cuminitiali{}{temporalia/oratio2.gtex}

\vspace{-1mm}
%\trOrationisI

\vfill

\rubrica{Hebdomadarius dicit iterum Dominus vobiscum, vel cantor dicit:}

\vspace{2mm}

\sineinitiali{temporalia/domineexaudi.gtex}

\rubrica{Postea cantatur a cantore:}

\vspace{2mm}

\ifx\festum\undefined
\ifx\octava\undefined
\cuminitiali{I}{temporalia/benedicamus-semiduplex-laud.gtex}
\else
\cuminitiali{VIII}{temporalia/benedicamus-duplexmajus-laudes.gtex}
\fi
\else
\cuminitiali{II}{temporalia/benedicamus-solemnism-laud.gtex}
\fi

\vspace{1mm}

\vfill
\pagebreak

\ifx\sabbatoveloctava\undefined
\ifx\festumveldominica\undefined
\hora{Ad Vesperas.} %%%%%%%%%%%%%%%%%%%%%%%%%%%%%%%%%%%%%%%%%%%%%%%%%%%%%
%\sideThumbs{Vesperæ}
\else
\hora{Ad II. Vesperas.} %%%%%%%%%%%%%%%%%%%%%%%%%%%%%%%%%%%%%%%%%%%%%%%%%%%%%
%\sideThumbs{II. Vesperæ}
\fi

\cantusSineNeumas

%\vspace{-2mm}
\grechangedim{interwordspacetext}{0.18 cm plus 0.15 cm minus 0.05 cm}{scalable}%
\ifx\festumveldominica\undefined
\cuminitiali{}{temporalia/deusinadiutorium-communis.gtex}
\else
\ifx\festum\undefined
\cuminitiali{}{temporalia/deusinadiutorium-alter.gtex}
\else
\cuminitiali{}{temporalia/deusinadiutorium-solemnis.gtex}
\fi
\fi
\grechangedim{interwordspacetext}{0.22 cm plus 0.15 cm minus 0.05 cm}{scalable}%

\vfill
%\pagebreak

%\vspace{-2mm}

\pars{Psalmus 1.} \scriptura{Lc. 1, 13; \textbf{H277}}

\vspace{-0.4cm}

\antiphona{III a}{temporalia/ant-elisabethzachariae.gtex}

\scriptura{Psalmus 109.}

\initiumpsalmi{temporalia/ps109-initium-iii-a-auto.gtex}

%\psalmusEtTranslatioT{temporalia/ps109-comb.tex}{10cm}
\input{temporalia/ps109.tex} \Abardot{}

\vfill
\pagebreak

\pars{Psalmus 2.} \scriptura{Lc. 1, 62.63; \textbf{H277}}

\vspace{-0.4cm}

\antiphona{IV E*}{temporalia/ant-innuebantpatriejus.gtex}

\scriptura{Psalmus 110.}

\initiumpsalmi{temporalia/ps110-initium-iv-E_-auto.gtex}

%\psalmusEtTranslatioT{temporalia/ps110-comb.tex}{10cm}
\input{temporalia/ps110.tex} \Abardot{}

\vfill
\pagebreak

\pars{Psalmus 3.} \scriptura{Lc. 1, 13.14; \textbf{H277}}

\vspace{-0.4cm}

\antiphona{I f}{temporalia/ant-joannesvocabitur.gtex}

\scriptura{Psalmus 111.}

\initiumpsalmi{temporalia/ps111-initium-i-f-auto.gtex}

%\psalmusEtTranslatioT{temporalia/ps111-comb.tex}{10cm}
\input{temporalia/ps111.tex} \Abardot{}

\vfill
\pagebreak

\pars{Psalmus 4.} \scriptura{Lc. 1, 15.14.63; \textbf{H275}}

\vspace{-0.4cm}

\antiphona{VIII G}{temporalia/ant-joannesest.gtex}

\scriptura{Psalmus 129.}

\initiumpsalmi{temporalia/ps129-initium-viii-G-auto.gtex}

%\psalmusEtTranslatioT{temporalia/ps129-comb.tex}{10cm}
\input{temporalia/ps129.tex} \Abardot{}

\vfill
\pagebreak

\pars{Psalmus 5.} \scriptura{Lc. 1, 15.66; \textbf{H276}}

\vspace{-0.4cm}

\antiphona{VII a}{temporalia/ant-istepuer.gtex}

\scriptura{Psalmus 131.}

\initiumpsalmi{temporalia/ps131-initium-vii-a-auto.gtex}

%\psalmusEtTranslatioT{temporalia/ps131-comb.tex}{10cm}
\input{temporalia/ps131.tex}

\vfill

\antiphona{}{temporalia/ant-istepuer.gtex}

\vfill
\pagebreak

\pars{Capitulum.} \scriptura{Is. 49, 7}

\grechangedim{interwordspacetext}{0.12 cm plus 0.15 cm minus 0.05 cm}{scalable}%
\cuminitiali{}{temporalia/capitulum-RegesVidebunt.gtex}
\grechangedim{interwordspacetext}{0.22 cm plus 0.15 cm minus 0.05 cm}{scalable}

% preklad Jeruz. bible
%\trCapituliI

\vfill

\pars{Responsorium breve.} \scriptura{Cf. Lc. 7, 28}

\cuminitiali{VI}{temporalia/resp-internatos.gtex}

%\trResp

\vfill
\pagebreak

\pars{Hymnus}

\cuminitiali{II}{temporalia/hym-UtQueant.gtex}
\vspace{-3mm}
%\input{hym-UtQueant-bohtext.tex}

\vfill
%\pagebreak

\pars{Versus.} \scriptura{Ps. 91, 13}

% Versus. %%%
\sineinitiali{temporalia/versus-iustus.gtex}

%\noindent \trVersus

\vfill
\pagebreak

\pars{Canticum B. Mariæ V.} \scriptura{Lc. 1, 59-60}

%\vspace{-5.5mm}

{
\grechangedim{interwordspacetext}{0.18 cm plus 0.15 cm minus 0.05 cm}{scalable}%
\antiphona{VIII G}{temporalia/ant-etfactumest.gtex}
\grechangedim{interwordspacetext}{0.22 cm plus 0.15 cm minus 0.05 cm}{scalable}%
}

%\trAntIMagnificat

\vspace{-3mm}

\scriptura{Lc. 1, 46-55}

\vspace{-2.5mm}

\cantusSineNeumas
\ifx\solemnis\undefined
\initiumpsalmi{temporalia/magnificat-initium-viii-G.gtex}

\vspace{-1.5mm}

%\psalmusEtTranslatioT{temporalia/magnificat-V-comb.tex}{10.2cm}
\input{temporalia/magnificat-V.tex} \Abardot{}
\else
\initiumpsalmi{temporalia/magnificat-initium-viiisoll-G.gtex}

\vspace{-1.5mm}

%\psalmusEtTranslatioT{temporalia/magnificat-II-comb.tex}{10.2cm}
\input{temporalia/magnificat-II.tex} \Abardot{}
\fi

\vspace{-1cm}

\vfill
\pagebreak

%\sideThumbs{{\scriptsize{}Fine horarum}}

\anteOrationem

\pagebreak

% Oratio. %%%
\cuminitiali{}{temporalia/oratio2.gtex}

\vspace{-1mm}
%\trOrationisI

\vfill

\rubrica{Hebdomadarius dicit iterum Dominus vobiscum, vel cantor dicit:}

\vspace{2mm}

\sineinitiali{temporalia/domineexaudi.gtex}

\rubrica{Postea cantatur a cantore:}

\vspace{2mm}

\ifx\festum\undefined
\cuminitiali{II}{temporalia/benedicamus-semiduplex-vesp.gtex}
\else
\cuminitiali{II}{temporalia/benedicamus-solemnism-2vesp.gtex}
\fi

\vspace{1mm}
\fi

\end{document}

