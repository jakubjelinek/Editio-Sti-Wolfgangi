\newcommand{\titulus}{\nomenFesti{In Nativitate S. Ioannis Baptistæ.}
\celebratio{Duplex 1. classis.}}
\newcommand{\festum}{Ioannes Baptista}
\newcommand{\aequus}{Ioannes Baptista}
\newcommand{\festumveldominica}{Ioannes Baptista}
\newcommand{\solemnis}{Ioannes Baptista}
\newcommand{\lectioi}{\pars{Lectio I.}

\noindent Sermo sancti Augustíni Epíscopi in natáli Iohánnis Baptístæ.

\noindent Natálem sancti Iohánnis, fratres caríssimi, hódie celebrámus, quod nulli unquam sanctórum légimus fuísse concéssum. Solíus enim Dómini et beáti Iohánnis dies nativitátis in univérso mundo celebrátur et cólitur. Illum enim stérilis péperit; istum virgo concépit. In Elísabeth sterílitas víncitur, in beáta María conceptiónis consuetúdo mutátur. Elísabeth virum cognoscéndo fílium génuit; María ángelo crédidit, et concépit. Hóminem concépit Elísabeth, et hóminem María; sed Elísabeth solum hóminem, María Deum et hóminem.}
\newcommand{\lectioii}{\pars{Lectio II.}

\noindent Quid sibi vult ergo Iohánnes? Unde interpósitus, unde præmíssus? Magnus ígitur Iohánnes, cuius magnitúdini étiam salvátor testimónium pérhibet, dicens: Non surréxit inter natos mulíerum maior Iohánne Baptísta. Præcéllit cunctis, éminet univérsis; antecéllit prophétas, supergréditur patriárchas; et quisquis de mulíere natus est, inférior est Iohánne. Dicit fortásse áliquis: Si inter natos mulíerum Iohánnes maior est, maior est salvatóre. Absit. Iohánnes enim natus mulíeris, Christus autem vírginis natus est; ille corruptíbilis úteri sínibus effúsus est, iste impollútæ vulvæ flore progénitus.}
\newcommand{\lectioiii}{\pars{Lectio III.}

\noindent Ideo autem cum Iohánnis nativitáte Dómini generátio deputátur, ne Dóminus extra veritátem videátur conditiónis humánæ: si comparétur homínibus Iohánnes, præmíssus est ante Deum. Tanta in illo excelléntia erat, tanta grátia, ut ipse putátus sit Christus. Quid ergo dixit de Christo? Nos omnes de plenitúdine eius accépimus. Quid est: nos omnes? Ergo prophétæ, patriárchæ, apóstoli, quotquot sancti, et ante incarnatiónem præmíssi, vel ab incarnáto missi, omnes nos de plenitúdine eius accépimus: nos vasa sumus, ille fons est. Si ergo intelléximus mystérium, fratres mei, Iohánnes homo est, Christus Deus est: humiliétur homo, et exaltétur Deus, secúndum illud, quod de Dómino ipse Iohánnes dixit: Illum opórtet créscere, me autem mínui. Ut humiliarétur homo, eo ei die natus est Iohánnes, quo incípiunt decréscere dies: ut exaltétur Deus, eo die natus est Christus, quo incípiunt créscere dies.}
\newcommand{\lectioiv}{\pars{Lectio IV.}

\noindent Magnum sacraméntum, fratres caríssimi. Ideo celebrámus natálem Iohánnis, sicut et Christi, quia et ipsa natívitas plena est mystério. Quo mystério, nisi humilitátis nostræ; sicut natívitas Christi plena est mystério altitúdinis nostræ? Ergo in hómine minuámur, ut in Deo crescámus; in nobis humiliémur, ut in illo exaltémur; humiliétur humána præsúmptio, ut crescat divína miserátio. Nam huius rei sacraméntum étiam in passiónibus ambórum implétum est. Ut minuétur homo, caput Iohánnis abscínditur; ut exaltétur Deus, Christus in ligno suspénditur. Quare autem beátum Iohánnem Dóminus et salvátor noster lucérnam esse díxerit, et quare eum mitti ante se volúerit, bréviter, si iubétis, caritátis vestræ áuribus intimábo.}
\newcommand{\lectiov}{\pars{Lectio V.}

\noindent Perfécta Christi grátia semper confirmétur in nobis. Præmíssus est enim velut vox ante verbum, lucérna ante solem, præco ante iúdicem, servus ante dóminum, amícus ante sponsum. Et quia univérsum mundum peccatórum ténebræ et nox infidelitátis opprésserat, et solem iustítiæ aspícere non valébat, beátus Iohánnes quasi lucérna præmíttitur, ut cordis óculi, qui lippitúdine iniquitátis oppréssi magnum et verum lumen vidére non póterant, ad lumen lucérnæ primum quasi ténuem splendórem vidére consuéscerent; et paulátim peccatórum núbilo remóto, et infidelitátis humóre digésto, adveniénte Christo, ab illo cælésti lúmine lætificáre possent pótius quam torquéri. Sicut enim lippiéntes óculos ad vidéndum próvocas, si exíguum splendórem lucérnæ osténderis; et ámplius crúcias, si lumen magnum ingésseris: ita Dóminus et Salvátor noster, qui est lumen verum, nisi prius beátum Iohánnem velut lucérnam prætermítteret, claritátem illíus totus mundus sustinére non posset.}
\newcommand{\lectiovi}{\pars{Lectio VI.}

\noindent Loquátur Iohánnes et dicat: ego vox clamántis in desérto. Vox erat, quia Verbi Dei Spíritu replebátur. Sicut sermo vocis quodam modo ministério ac vehículo ad audiéntem a loquénte transmíttitur, ita ille Christum sonans, Verbi erat miníster et pórtitor. Sanctus, inquam, Iohánnes typum in se legis, quæ Christum longe per signa et iudícia monstrábat, osténdit; et ídeo misit ad Christum duos de discípulis suis. Isti duo discípuli a Iohánne ad Christum missi, forte duo pópuli sunt, quorum unus ex Iudǽis crédidit, alter ex géntibus. Iohánnes dírigit ad Christum, lex mittit ad grátiam, et per evangélii fidem, véterem desíderat ástrui veritátem. Nos vero, fratres charíssimi, ut tam sanctam festivitátem non solum corporáli, sed étiam spirituáli cum gáudio celebráre possímus, secúndum vires nostros ad dandas eleemósynas, et ad tenéndam cum ómnibus pacem nostros ánimos præparémus: et ab omni scurrilitáte vel turpilóquio non solum nosmetipsos, sed et omnem famíliam nostram et univérsos ad nos pertinéntes pro amóre Dei et zelo sanctæ disciplínæ prohibére totis víribus laborémus, nec permittámus voluptuósos quosque solemnitátem sanctam cántica luxuriósa proferéndo pollúere.  Tunc enim pro nobis sanctus Iohánnes quidquid petiérimus póterit obtinére, si nos festivitátem suam pacíficos, sóbrios, castos, absque ullo turpilóquio cognóverit celebráre. Hæc ergo,  fratres charíssimi, pro patérna sollicitúdo súggero: nam Deo propítio ita de vestra devotióne confído, quod non solum vos ipsos, sed étiam omnes qui ad vos pértinent, cum omne honestáte castos sobriósque conservétis. Unde Deo grátias agens súpplico, ut qui vobis dedit ea, quæ sancta sunt fidéliter incípere, concédat vobis felícem perseverántiam custodíre, qui cum Patre et Spíritu Sancto vivit et regnat, Deus, in sǽcula sæculórum. Amen.}
\newcommand{\lectiovii}{\pars{Lectio VII.} \scriptura{Io. 6, 57-59}

\noindent Léctio sancti Evangélii secúndum Lucam.

\noindent In illo témpore: Elísabeth implétum est tempus pariéndi, et péperit fílium. Et audiérunt vicíni et cognáti eius, quod magnificávit Dóminus misericórdiam suam cum illa, et congratulabántur ei. Et réliqua.

\vspace{4mm}

\noindent Ex homilía venerábilis Bedæ presbýteri in nativitáte sancti Iohánnis Baptístæ.

\noindent Præcursóris Dómini natívitas, sicut sacratíssima lectiónis evangélicæ prodit história, multa miraculórum sublimitáte refúlget, quia nimírum decébat, ut ille, quo maior inter natos mulíerum nemo surréxit maióre præ céteris sanctis in ipso mox ortu virtútum iúbare clarésceret. Senes ac diu infecúndi paréntes dono nobilíssimæ prolis exsúltant. Ipsi patri, quem incredúlitas mutum reddíderat, ad salutándum novæ præcónem grátiæ os et lingua reserátur. Nec solum facúltas Deum benedicéndi restitúitur, sed de eo étiam prophetándi virtus augétur. Excitáti vero fama facti omnes vicíni admiratióne ac metu percellúntur ómnium, qui audiére circumquáque ad advéntum novi prophétæ corda percellúntur.}
\newcommand{\lectioviii}{\pars{Lectio VIII.}

\noindent Unde mérito sancta per orbem ecclésia, quæ tot beatórum mártyrum victórias, quibus ingréssum regni cæléstis meruére frequéntat, huius tantúmmodo post Dómini étiam nativitátis diem celebráre consuévit. Quod nullátenus sine evangélica auctoritáte in consuetúdinem venísse credéndum est, sed atténtius ánimo recordendum, quia sicut nato Dómino pastóribus appárens ángelus ait: Ecce evangelízo vobis gáudium magnum, quod erit omni pópulo, quia natus est nobis hódie salvátor, qui est Christus Dóminus, ita étiam ángelus nascitúrum Zacharíæ prǽdicans Iohánnem. Et erit tibi gáudium, inquit, et exsultátio, et multi in nativitáte eius gaudébunt. Et erit magnus coram Dómino. Iure ígitur utriúsque natívitas festa devotióne celebrátur, sed in illíus tamquam in Christi Dómini, tamquam in salvatóris mundi, tamquam in Fílii Dei omnipoténtis, tamquam in solis iustítiæ nativitáte omni pópulo gáudium evangelizátur, in huius autem tamquam in præcursóris Dómini, in servi Dómini exímii, in lucérnæ ardéntis et lucéntis exórtu multi gavísi esse memorántur.}
\newcommand{\lectioix}{\pars{Lectio IX.}

\noindent Iste magnus coram Dómino esse narrátur, de illo prophéta testátur, quóniam magnus Dóminus et laudábilis nimis, et magnitúdinis eius non est finis, Iste peccatórum consórtia declínans, ab omni quod inebriáre potest, abstinébat, ille inter peccatóres conversátus, peccáti omnis immúnis permánsit. Hic adhuc ex útero matris Spíritu Sancto replétus est, in illo hábitat omnis plenitúdo divinitátis corporáliter, qui dono sui Spíritus sedem sibi úteri virginális, in quo carnem suscíperet, ipse consecrávit. Iste multos filiórum Isráhel ad Dóminum suo témpore prædicándo convértit. Ille multos cotídie de univérsis per orbem natiónibus ad suam fidem et caritátem intérius illustrándo convértere non desístit. Hic in Spíritu et virtúte Helíæ præcéssit ante illum, ut plebem eius aqua baptízans ad suscipiéndum eum, ubi apparéret, docéret esse perféctam. Huic succéssit ille in Spíritu et virtúte Dei Patris, ut plebem suam Spíritu sancto et igne baptízans ad vidéndam fáciem patris sui donáret esse perféctam.}
% LuaLaTeX

\documentclass[a4paper, twoside, 12pt]{article}
\usepackage[latin]{babel}
%\usepackage[landscape, left=3cm, right=1.5cm, top=2cm, bottom=1cm]{geometry} % okraje stranky
%\usepackage[landscape, a4paper, mag=1166, truedimen, left=2cm, right=1.5cm, top=1.6cm, bottom=0.95cm]{geometry} % okraje stranky
\usepackage[landscape, a4paper, mag=1400, truedimen, left=0.5cm, right=0.5cm, top=0.5cm, bottom=0.5cm]{geometry} % okraje stranky

\usepackage{fontspec}
\setmainfont[FeatureFile={junicode.fea}, Ligatures={Common, TeX}, RawFeature=+fixi]{Junicode}
%\setmainfont{Junicode}

% shortcut for Junicode without ligatures (for the Czech texts)
\newfontfamily\nlfont[FeatureFile={junicode.fea}, Ligatures={Common, TeX}, RawFeature=+fixi]{Junicode}

\usepackage{multicol}
\usepackage{color}
\usepackage{lettrine}
\usepackage{fancyhdr}

% usual packages loading:
\usepackage{luatextra}
\usepackage{graphicx} % support the \includegraphics command and options
\usepackage{gregoriotex} % for gregorio score inclusion
\usepackage{gregoriosyms}
\usepackage{wrapfig} % figures wrapped by the text
\usepackage{parcolumns}
\usepackage[contents={},opacity=1,scale=1,color=black]{background}
\usepackage{tikzpagenodes}
\usepackage{calc}
\usepackage{longtable}
\usetikzlibrary{calc}

\setlength{\headheight}{14.5pt}

\input{conventuscommune.tex} % Often used macros
%%%% Preklady jednotlivych zpevu (nektere se opakuji, a je dobre mit je
% vsechny na jedne hromade)

% HOURS ---

\newcommand{\trAntI}{\translatioCantus{Muž boží měl kožený toulec, pečlivě
zavázaný, jenž mu visel na šíji a~často se ho dotýkal.}}

\newcommand{\trAntII}{\translatioCantus{Klíč od~něho tak dobře střežil, že
dokud žil v~těle, nikdo z~jeho žáků nezvěděl, co je uvnitř.}}

\newcommand{\trAntIII}{\translatioCantus{Ale když se odebral z~tohoto
života, schránku otevřeli a~objevili v~ní žíněné roucho a~měděný řetěz
potřísněný krví.}}

\newcommand{\trAntIV}{\translatioCantus{A když prohlédli mistrovo tělo,
nalezli jeho tělo na čtyřech místech hluboce zbrázděno ranami od řetězu.}}

\newcommand{\trAntV}{\translatioCantus{Krev vytékající z~těch ran, místy
prostoupila i~žíněným rouchem.}}

\newcommand{\trCapituli}{\translatioCantus{
Miláčkovi Boha a~lidí,
Mojžíšovi požehnané paměti,~\gredagger{}
dopřál slávu rovnou slávě svatých~\grestar{}
učinil ho mocným na postrach nepřátelům
a~jeho slovy zastavil divy.}}

\newcommand{\trLectioBrevis}{\translatioCantus{
Pamatujte na své představené,
kteří vám hlásali Boží slovo.
Uvažte, jak oni skončili život, a~napodobujte jejich víru.
Ježíš Kristus je stejný včera i~dnes i~navěky.
Nenechte se svést věelijakými cizími naukami.}}

\newcommand{\trRespLaud}{\translatioCantus{Spravedlivého vodil Hospodin~\grestar{}
po přímých stezkách. \Vbardot{} A~ukázal mu Boží království.}}

\newcommand{\trRespLaudB}{\translatioCantus{Na tvých hradbách, Jeruzaléme,
ustanovil jsem strážné;~\grestar{}
budou bdít nad mým lidem. \Vbardot{} Ani ve dne, ani v~noci nesmějí nikdy
mlčet.}}

\newcommand{\trVersus}{\translatioCantus{\Vbardot{} Ústa spravedlivého šeptají moudrost, aleluja.
\Rbardot{} A~jeho jazyk ohlašuje právo, aleluja.}}

\newcommand{\trAntBenedictus}{\translatioCantus{Když na bujné oře vložili
nosítka a~sňali jim uzdu, vydali se přímo k~cele božího muže.}}

\newcommand{\trPreces}{\translatioCantus{
\noindent S vděčností chvalme Krista, dobrého Pastýře, \gredagger{} který dal život za své ovce, \grestar{} a~pokorně ho prosme: \Rbardot{} Pane, buď pastýřem svého lidu.

\noindent Kriste, ty dáváš církvi pastýře, a~jejich službou se ujímáš svého lidu, \grestar{} dej, ať v~lásce těch, kteří nás vedou, poznáváme, jak nás miluješ. \Rbardot{} Pane, buď pastýřem svého lidu.

\noindent Ty stále konáš skrze své zástupce službu pastýře a~učitele, \grestar{} nepřestávej nás nikdy vést prostřednictvím svých služebníků. \Rbardot{} Pane, buď pastýřem svého lidu.

\noindent Ty prokazuješ svému lidu skrze jeho pastýře službu lékaře duše i~těla, \grestar{} ochraňuj náš život a~veď nás ke svatosti. \Rbardot{} Pane, buď pastýřem svého lidu.

\noindent Ty posíláš své svaté, aby slovem i~příkladem vedli tvůj lid k~tobě, \grestar{} na jejich přímluvu nás posiluj, abychom vytrvali na cestě, která vede k~věčnému životu. \Rbardot{} Pane, buď pastýřem svého lidu.}}

\newcommand{\trOrationis}{\translatioCantus{Bože, jenž nám dopřáváš radovat
se z~výroční slavnosti svatého tvého vyznavače Havla, uděl dobrotivě,
abychom když slavíme jeho narození, též se řídili podobou jeho skutků.
Skrze…}}
 % Czech translations of the proper texts

\newcommand{\annusEditionis}{2020}

%%%% Vicekrat opakovane kousky

\newcommand{\anteOrationem}{
  \rubrica{Ante Orationem, cantatur a Superiore:}

  \pars{Supplicatio Litaniæ.}

  \cuminitiali{}{temporalia/supplicatiolitaniae.gtex}

  \pars{Oratio Dominica.}

  \cuminitiali{}{temporalia/oratiodominica.gtex}

  \rubrica{Deinde dicitur ab Hebdomadario:}

  \cuminitiali{}{temporalia/dominusvobiscum-solemnis.gtex}

  \rubrica{In choro monialium loco Dominus vobiscum dicitur:}

  \sineinitiali{temporalia/domineexaudi.gtex}
}

\setlength{\columnsep}{30pt} % prostor mezi sloupci

%%%%%%%%%%%%%%%%%%%%%%%%%%%%%%%%%%%%%%%%%%%%%%%%%%%%%%%%%%%%%%%%%%%%%%%%%%%%%%%%%%%%%%%%%%%%%%%%%%%%%%%%%%%%%
\begin{document}

% Here we set the space around the initial.
% Please report to http://home.gna.org/gregorio/gregoriotex/details for more details and options
\grechangedim{afterinitialshift}{2.2mm}{scalable}
\grechangedim{beforeinitialshift}{2.2mm}{scalable}
\grechangedim{interwordspacetext}{0.22 cm plus 0.15 cm minus 0.05 cm}{scalable}%
\grechangedim{annotationraise}{-0.2cm}{scalable}

% Here we set the initial font. Change 38 if you want a bigger initial.
% Emit the initials in red.
\grechangestyle{initial}{\color{red}\fontsize{38}{38}\selectfont}

\pagestyle{empty}

%%%% Titulni stranka
\begin{titulusOfficii}
\titulus
\end{titulusOfficii}

% graphic
%\vspace{1.5cm}
%\begin{center}
%\includegraphics[width=8cm]{emmaus.jpg}
%\end{center}

\vfill

\begin{center}
%Ad usum et secundum consuetudines chori \guillemotright{}Conventus Choralis\guillemotleft.

%Editio Sancti Wolfgangi \annusEditionis
\end{center}

\pagebreak

\renewcommand{\headrulewidth}{0pt} % no horiz. rule at the header
\fancyhf{}
\pagestyle{fancy}

\cantusSineNeumas

\ifx\festumveldominica\undefined
\else
\pars{Oratio ante divinum Officium.}

\lettrine{{\color{red}A}}{peri,} Dómine, os meum ad benedicéndum nomen sanctum tuum:
munda quoque cor meum ab ómnibus vanis, pervérsis, et aliénis
cogitatiónibus:
intelléctum illúmina, afféctum inflámma,
ut digne, atténte ac devóte hoc Offícium recitáre váleam,
et exaudíri mérear ante conspéctum Divínæ Maiestátis tuæ.
Per Christum, Dóminum nostrum.
\Rbardot{} Amen.

Dómine, in unióne illíus divínæ intentiónis,
qua ipse in terris laudes Deo persolvísti,
has tibi Horas \rubricatum{(vel \textnormal{hanc tibi Horam})} persólvo.

%\trOratioAnteOfficium

\vfill

\pars{Oratio post divinum Officium.}

\rubrica{
  Orationem sequentem devote post Officium recitantibus
  Leo Papa X. defectus, et culpas in eo persolvendo ex humana
  fragilitate contractas, indulsit, et dicitur flexis genibus.
}

\lettrine{{\color{red}S}}{acrosánctæ} et indivíduæ Trinitáti,
crucifíxi Dómini nostri Iesu Christi humanitáti,
beatíssimæ et gloriosíssimæ sempérque Vírginis Maríæ
fecúndæ integritáti, 
et ómnium Sanctórum universitáti
sit sempitérna laus, honor, virtus et glória
ab omni creatúra,
nobísque remíssio ómnium peccatórum,
per infiníta sǽcula sæculórum.
\Rbardot{} Amen.

\noindent \Vbardot{} Beáta víscera Maríæ Virginis, quæ portavérunt
ætérni Patris Fílium.\\
\Rbardot{} Et beáta úbera, quæ lactavérunt Christum Dominum.

\rubrica{Et dicitur secreto \textnormal{Pater noster.} et \textnormal{Ave María.}}

%\trOratioPostOfficium

\vfill

\ifx\festum\undefined
\else
\hora{Ad I. Vesperas.} %%%%%%%%%%%%%%%%%%%%%%%%%%%%%%%%%%%%%%%%%%%%%%%%%%%%%
%\sideThumbs{I. Vesperæ}

\vspace{0.5cm}
\grechangedim{interwordspacetext}{0.18 cm plus 0.15 cm minus 0.05 cm}{scalable}%
\ifx\festum\undefined
\cuminitiali{}{temporalia/deusinadiutorium-alter.gtex}
\else
\cuminitiali{}{temporalia/deusinadiutorium-solemnis.gtex}
\fi
\grechangedim{interwordspacetext}{0.22 cm plus 0.15 cm minus 0.05 cm}{scalable}%

\vfill
\pagebreak

\pars{Psalmus 1.} \scriptura{Cf. Lc. 1, 11.13; \textbf{H273}}

\vspace{-0.4cm}

\antiphona{VII a}{temporalia/ant-descenditangelus.gtex}

\scriptura{Psalmus 112.}

\initiumpsalmi{temporalia/ps112-initium-vii-a-auto.gtex}

%\psalmusEtTranslatioT{temporalia/ps112-comb.tex}{10cm}
\input{temporalia/ps112.tex}

\vspace{-1cm}

\vfill
\pagebreak

\pars{Psalmus 2.} \scriptura{Psalmus 116.}

\initiumpsalmi{temporalia/ps116-initium-vii-a-auto.gtex}

%\psalmusEtTranslatioT{temporalia/ps116-comb.tex}{10cm}
\input{temporalia/ps116.tex}

\vfill
\pagebreak

\pars{Psalmus 3.} \scriptura{Psalmus 145.}

\initiumpsalmi{temporalia/ps145-initium-vii-a-auto.gtex}

%\psalmusEtTranslatioT{temporalia/ps145-comb.tex}{10cm}
\input{temporalia/ps145.tex}

\vfill
\pagebreak

\pars{Psalmus 4.} \scriptura{Psalmus 146.}

\initiumpsalmi{temporalia/ps146-initium-vii-a-auto.gtex}

%\psalmusEtTranslatioT{temporalia/ps146-comb.tex}{10cm}
\input{temporalia/ps146.tex} \Abardot{}

\vfill
\pagebreak

\pars{Psalmus 5.} \scriptura{Psalmus 147.}

\initiumpsalmi{temporalia/ps147-initium-vii-a-auto.gtex}

%\psalmusEtTranslatioT{temporalia/ps147-comb.tex}{10cm}
\input{temporalia/ps147.tex} \Abardot{}

\vfill

\antiphona{}{temporalia/ant-descenditangelus.gtex} % repeat the antiphon - new page

\vfill
\pagebreak

\pars{Capitulum.} \scriptura{Ier. 1, 5}

\grechangedim{interwordspacetext}{0.12 cm plus 0.15 cm minus 0.05 cm}{scalable}%
\cuminitiali{}{temporalia/capitulum-PriusquamTe.gtex}
\grechangedim{interwordspacetext}{0.22 cm plus 0.15 cm minus 0.05 cm}{scalable}

% preklad Jeruz. bible
%\trCapituliI

\vfill

\pars{Responsorium breve.} \scriptura{Cf. Lc. 7, 28}

\cuminitiali{VI}{temporalia/resp-internatos.gtex}

%\trResp

\vfill
\pagebreak

\pars{Hymnus}

\cuminitiali{II}{temporalia/hym-UtQueant.gtex}
\vspace{-3mm}
%\input{hym-UtQueant-bohtext.tex}

\vfill
%\pagebreak

\pars{Versus.} \scriptura{Ps. 91, 13}

% Versus. %%%
\sineinitiali{temporalia/versus-iustus.gtex}

%\noindent \trVersus

\vfill
\pagebreak

\pars{Canticum B. Mariæ V.} \scriptura{Lc. 1, 9.11}

\vspace{-3mm}

{
\grechangedim{interwordspacetext}{0.18 cm plus 0.15 cm minus 0.05 cm}{scalable}%
\antiphona{VIII G}{temporalia/ant-ingressozacharia.gtex}
\grechangedim{interwordspacetext}{0.22 cm plus 0.15 cm minus 0.05 cm}{scalable}%
}

%\trAntIMagnificat

%\vspace{-2mm}

\scriptura{Lc. 1, 46-55}

%\vspace{-2mm}

\cantusSineNeumas
\initiumpsalmi{temporalia/magnificat-initium-viiisoll-G.gtex}

%\psalmusEtTranslatioT{temporalia/magnificat-I-comb.tex}{10.2cm}
\input{temporalia/magnificat-I.tex} \Abardot{}

%\vspace{-1cm}

\vfill
\pagebreak

%\sideThumbs{{\scriptsize{}Fine horarum}}

\anteOrationem

\pagebreak

% Oratio. %%%
\cuminitiali{}{temporalia/oratio.gtex}

\vspace{-1mm}
%\trOrationisI

\vfill

\rubrica{Hebdomadarius dicit iterum Dominus vobiscum, vel cantor dicit:}

\vspace{2mm}

\sineinitiali{temporalia/domineexaudi.gtex}

\rubrica{Postea cantatur a cantore:}

\vspace{2mm}

\ifx\festum\undefined
\cuminitiali{II}{temporalia/benedicamus-semiduplex-vesp.gtex}
\else
\cuminitiali{II}{temporalia/benedicamus-solemnism-1vesp.gtex}
\fi

\vspace{1mm}

\vfill
\pagebreak
\fi

\iffalse
\hora{Ad Matutinum.} %%%%%%%%%%%%%%%%%%%%%%%%%%%%%%%%%%%%%%%%%%%%%%%%%%%%%
%\sideThumbs{Matutinum}

\vspace{2mm}

\cuminitiali{}{temporalia/dominelabiamea.gtex}

\vspace{2mm}

\pars{Invitatorium.} \scriptura{Cantor; \textbf{LU\textsubscript{918}}}

\vspace{-6mm}

\antiphona{IV}{temporalia/inv-christumregemadoremus.gtex}

\vfill
\pagebreak

\pars{Hymnus.}

\vspace{-5mm}

\scriptura{Thomas de Aquino; \textbf{LU\textsubscript{920}}}

\antiphona{IV}{temporalia/hym-SacrisSolemniis.gtex}
%{
%\vspace{-5mm}
%\setlength{\columnsep}{0pt} % prostor mezi sloupci
%\input{hym-SacrisSolemniis-bohtext.tex}
%\setlength{\columnsep}{30pt} % prostor mezi sloupci
%}

\vfill
\pagebreak

\subhora{In I. Nocturno}

\pars{Psalmus 1.} \scriptura{Ps. 1, 3; \textbf{LU\textsubscript{922}}}

\vspace{-2mm}

\antiphona{I D*}{temporalia/ant-fructumsalutiferum.gtex}

%\vspace{-5mm}

\scriptura{Ps. 1}

%\vspace{-2mm}

\initiumpsalmi{temporalia/ps1-initium-i-D_-auto.gtex}

%\psalmusEtTranslatioT{temporalia/ps1-comb.tex}{10cm}
\input{temporalia/ps1.tex} \Abardot{}

\vfill
\pagebreak

\pars{Psalmus 2.} \scriptura{Ps. 4, 8.9; \textbf{LU\textsubscript{923}}}

\vspace{-2mm}

\antiphona{II D}{temporalia/ant-afructufrumenti.gtex}

%\vspace{-5mm}

\scriptura{Ps. 4}

\initiumpsalmi{temporalia/ps4-initium-ii-D-auto.gtex}

%\psalmusEtTranslatioT{temporalia/ps4iiD-comb.tex}{10cm}
\input{temporalia/ps4iiD.tex} \Abardot{}

\vfill
\pagebreak

\pars{Psalmus 3.} \scriptura{Ps. 15, 4; \textbf{LU\textsubscript{924}}}

\vspace{-4mm}

\antiphona{III a\textsuperscript{3}}{temporalia/ant-communionecalicis.gtex}

%\vspace{-2mm}

\scriptura{Ps. 15}

\vspace{-2mm}

\initiumpsalmi{temporalia/ps15-initium-iii-a3-auto.gtex}

%\psalmusEtTranslatioT{temporalia/ps15-comb.tex}{10cm}
\input{temporalia/ps15.tex} \Abardot{}

\vfill
\pagebreak

\pars{Versus.} \scriptura{Sap. 16, 20; Ps. 77, 25}

% Versus. %%%
\sineinitiali{temporalia/versus-panemdecaelohomo-communis.gtex}

\vspace{5mm}

\sineinitiali{temporalia/oratiodominica-mat.gtex}

\vspace{5mm}

\pars{Absolutio.}

\cuminitiali{}{temporalia/absolutio-exaudi.gtex}

\vfill
\pagebreak

\cuminitiali{}{temporalia/benedictio-solemn-benedictione.gtex}

\vspace{7mm}

\lectioi

\noindent \Vbardot{} Tu autem, Dómine, miserére nobis.
\noindent \Rbardot{} Deo grátias.

\vfill
\pagebreak

\pars{Responsorium 1.} \scriptura{\Rbardot{} Ex. 12, 5.6.8 \Vbardot{} 1 Cor. 5, 7.8; \textbf{LU\textsubscript{926}}}

\vspace{-2mm}

\responsorium{I}{temporalia/resp-immolabithaedum.gtex}{}

\vfill
\pagebreak

\cuminitiali{}{temporalia/benedictio-solemn-unigenitus.gtex}

\vspace{7mm}

\lectioii

\noindent \Vbardot{} Tu autem, Dómine, miserére nobis.
\noindent \Rbardot{} Deo grátias.

\vfill
\pagebreak

\pars{Responsorium 2.} \scriptura{\Rbardot{} Ex. 16, 12.15 \Vbardot{} Io. 6, 21; \textbf{LU\textsubscript{927}}}

\vspace{-2mm}

\responsorium{II}{temporalia/resp-comedetiscarnes.gtex}{}

\vfill
\pagebreak

\cuminitiali{}{temporalia/benedictio-solemn-spiritus.gtex}

\vspace{7mm}

\lectioiii

\noindent \Vbardot{} Tu autem, Dómine, miserére nobis.
\noindent \Rbardot{} Deo grátias.

\vfill
\pagebreak

\pars{Responsorium 3.} \scriptura{\Rbardot{} 3 Reg. 19, 6.8 \Vbardot{} Io. 6, 52; \textbf{LU\textsubscript{927}}}

\vspace{-2mm}

\responsorium{III}{temporalia/resp-respexitelias.gtex}{}

\vfill
\pagebreak

\subhora{In II. Nocturno}

\pars{Psalmus 4.} \scriptura{Ps. 19, 4; \textbf{LU\textsubscript{928}}}

\vspace{-2mm}

\antiphona{IV E}{temporalia/ant-memorsit-FKP.gtex}

\vspace{-2mm}

\scriptura{Ps. 19}

\initiumpsalmi{temporalia/ps19-initium-iv-E-auto.gtex}

%\psalmusEtTranslatioT{temporalia/ps19-comb.tex}{10cm}
\input{temporalia/ps19.tex} \Abardot{}

\vfill
\pagebreak

\pars{Psalmus 5.} \scriptura{Ps. 22, 5; \textbf{LU\textsubscript{928}}}

\vspace{-2mm}

\antiphona{V a}{temporalia/ant-paraturnobis-FKP.gtex}

%\vspace{-3mm}

\scriptura{Ps. 22}

%\vspace{-2mm}

\initiumpsalmi{temporalia/ps22-initium-v-a.gtex}

%\vspace{-1.5mm}

%\psalmusEtTranslatioT{temporalia/ps22-comb.tex}{10cm}
\input{temporalia/ps22.tex} \Abardot{}

\vspace{-1cm}

\vfill
\pagebreak

\pars{Psalmus 6.} \scriptura{Ps. 41, 5; \textbf{LU\textsubscript{930}}}

\vspace{-2mm}

\antiphona{VI F}{temporalia/ant-invoceexsultationis-FKP.gtex}

%\vspace{-5mm}

\scriptura{Ps. 41}

\initiumpsalmi{temporalia/ps41-initium-vi-F-auto.gtex}

%\psalmusEtTranslatioT{temporalia/ps41-comb.tex}{10cm}
\input{temporalia/ps41.tex}

\vfill

\antiphona{}{temporalia/ant-invoceexsultationis-FKP.gtex}

\vfill
\pagebreak

\pars{Versus.} \scriptura{Ps. 80, 17}

% Versus. %%%
\sineinitiali{temporalia/versus-cibavit.gtex}

\vspace{5mm}

\sineinitiali{temporalia/oratiodominica-mat.gtex}

\vspace{5mm}

\pars{Absolutio.}

\cuminitiali{}{temporalia/absolutio-ipsius.gtex}

\vfill
\pagebreak

\cuminitiali{}{temporalia/benedictio-solemn-deus.gtex}

\vspace{7mm}

\lectioiv

\noindent \Vbardot{} Tu autem, Dómine, miserére nobis.
\noindent \Rbardot{} Deo grátias.

\vfill
\pagebreak

\pars{Responsorium 4.} \scriptura{\Rbardot{} Mt. 26, 26 \Vbardot{} Io. 31, 31; \textbf{LU\textsubscript{931}}}

\vspace{-2mm}

\responsorium{V}{temporalia/resp-coenantibus.gtex}{}

\vfill
\pagebreak

\cuminitiali{}{temporalia/benedictio-solemn-christus.gtex}

\vspace{7mm}

\lectiov

\noindent \Vbardot{} Tu autem, Dómine, miserére nobis.
\noindent \Rbardot{} Deo grátias.

\vfill
\pagebreak

\pars{Responsorium 5.} \scriptura{\Rbardot{} 1 Cor. 11, 25 \Vbardot{} Thren. 3, 20; \textbf{LU\textsubscript{932}}}

\vspace{-2mm}

\responsorium{VI}{temporalia/resp-accepitiesus.gtex}{}

\vfill
\pagebreak

\cuminitiali{}{temporalia/benedictio-solemn-ignem.gtex}

\vspace{7mm}

\lectiovi

\noindent \Vbardot{} Tu autem, Dómine, miserére nobis.
\noindent \Rbardot{} Deo grátias.

\vfill
\pagebreak

\pars{Responsorium 6.} \scriptura{\Rbardot{} Io. 6, 48 \Vbardot{} ibid. 6, 51.52; \textbf{LU\textsubscript{934}}}

\vspace{-2mm}

\responsorium{VII}{temporalia/resp-egosumpanisvitae.gtex}{}

\vfill
\pagebreak

\subhora{In III. Nocturno}

\pars{Psalmus 7.} \scriptura{Ps. 42, 4; \textbf{LU\textsubscript{934}}}

\vspace{-5mm}

\antiphona{VII a}{temporalia/ant-introibo-FKP.gtex}

\vspace{-4mm}

\scriptura{Ps. 42}

%\vspace{-2mm}

\initiumpsalmi{temporalia/ps42-initium-vii-a-auto.gtex}

%\psalmusEtTranslatioT{temporalia/ps42-comb.tex}{10cm}
\input{temporalia/ps42.tex} \Abardot{}

\vfill
\pagebreak

\pars{Psalmus 8.} \scriptura{Ps. 80, 17; \textbf{LU\textsubscript{935}}}

\vspace{-5mm}

\antiphona{VIII G}{temporalia/ant-cibavitnos-FKP.gtex}

\vspace{-3mm}

\scriptura{Ps. 80}

\vspace{-2mm}

\initiumpsalmi{temporalia/ps80-initium-viii-G-auto.gtex}

\vspace{-1mm}

%\psalmusEtTranslatioT{temporalia/ps80-comb.tex}{10cm}
\input{temporalia/ps80.tex} \Abardot{}

\vfill
\pagebreak

\pars{Psalmus 9.} \scriptura{Ps. 83, 3; \textbf{LU\textsubscript{936}}}

\vspace{-2mm}

\antiphona{VI F}{temporalia/ant-exaltari-FKP.gtex}

\vspace{-2mm}

\scriptura{Ps. 83}

\initiumpsalmi{temporalia/ps83-initium-vi-F-auto.gtex}

%\psalmusEtTranslatioT{temporalia/ps83-comb.tex}{10cm}
\input{temporalia/ps83.tex} \Abardot{}

\vfill
\pagebreak

\pars{Versus.} \scriptura{Ps. 103, 14-15}

% Versus. %%%
\sineinitiali{temporalia/versus-educas.gtex}

\vspace{5mm}

\sineinitiali{temporalia/oratiodominica-mat.gtex}

\vspace{5mm}

\pars{Absolutio.}

\cuminitiali{}{temporalia/absolutio-avinculis.gtex}

\vfill
\pagebreak

\cuminitiali{}{temporalia/benedictio-solemn-evangelica.gtex}

\vspace{7mm}

\lectiovii

\noindent \Vbardot{} Tu autem, Dómine, miserére nobis.
\noindent \Rbardot{} Deo grátias.

\vfill
\pagebreak

\pars{Responsorium 7.} \scriptura{\Rbardot{} Io. 6, 57 \Vbardot{} Dt. 4, 7; \textbf{LU\textsubscript{938}}}

\vspace{-2mm}

\responsorium{VII}{temporalia/resp-quimanducat.gtex}{}

\vfill
\pagebreak

\cuminitiali{}{temporalia/benedictio-solemn-divinum.gtex}

\vspace{7mm}

\lectioviii

\noindent \Vbardot{} Tu autem, Dómine, miserére nobis.
\noindent \Rbardot{} Deo grátias.

\vfill
\pagebreak

\ifx\dominica\undefined
\pars{Responsorium 8.} \scriptura{\Rbardot{} Io. 6, 58 \Vbardot{} Eccli. 15, 3; \textbf{LU\textsubscript{938}}}

\vspace{-2mm}

\responsorium{VIII}{temporalia/resp-misitmevivenspater.gtex}{}
\else
\pars{Responsorium 8.} \scriptura{\Rbardot{} Lc. 14, 16-17 \Vbardot{} Prv. 9, 5}

\vspace{-2mm}

\responsorium{VI}{temporalia/resp-homoquidamfecit.gtex}{}
\fi

\vfill
\pagebreak

\cuminitiali{}{temporalia/benedictio-solemn-adsocietatem.gtex}

\vspace{7mm}

\lectioix

\noindent \Vbardot{} Tu autem, Dómine, miserére nobis.
\noindent \Rbardot{} Deo grátias.

\vfill
\pagebreak

% Te Deum

%\pars{Hymnus Ambrosianus}

\vspace{-5mm}

\ifx\solemnis\undefined
\ifx\aequus\undefined
{
\pars{Hymnus Ambrosianus} \scriptura{Alio modo, iuxta morem Romanum}

\vspace{-2mm}

\grechangedim{interwordspacetext}{0.26 cm plus 0.15 cm minus 0.05 cm}{scalable}%
\cuminitiali{III}{temporalia/tedeum-romanum-gn.gtex}
\grechangedim{interwordspacetext}{0.22 cm plus 0.15 cm minus 0.05 cm}{scalable}%
}
\else
{
\pars{Hymnus Ambrosianus} \scriptura{Tonus Simplex}

\vspace{-2mm}

\grechangedim{interwordspacetext}{0.30 cm plus 0.15 cm minus 0.05 cm}{scalable}%
\cuminitiali{III}{temporalia/tedeum-simplex-gn.gtex}
\grechangedim{interwordspacetext}{0.22 cm plus 0.15 cm minus 0.05 cm}{scalable}%
}
\fi
\else
{
\pars{Hymnus Ambrosianus} \scriptura{Tonus Solemnis}

\vspace{-2mm}

\grechangedim{interwordspacetext}{0.26 cm plus 0.15 cm minus 0.05 cm}{scalable}%
\cuminitiali{III}{temporalia/tedeum-solemnis-gn.gtex}
\grechangedim{interwordspacetext}{0.22 cm plus 0.15 cm minus 0.05 cm}{scalable}%
}
\fi

\vfill
\pagebreak

\rubrica{Reliqua omittuntur, nisi Laudes separandæ sint.}

\sineinitiali{temporalia/domineexaudi.gtex}

\vfill

\pars{Oratio.}

\cuminitiali{}{temporalia/oratio2.gtex}

\vfill

\noindent \Vbardot{} Dómine, exáudi oratiónem meam.
\Rbardot{} Et clamor meus ad te véniat.

\vfill

% Nocturnale Romanum 2002, p. LXXVI Benedicamus Domino seems to match
% the one from Solemn Laudes.
\cuminitiali{V}{temporalia/benedicamus-solemnis-laud.gtex}

\vfill

\noindent \Vbardot{} Fidélium ánimæ per misericórdiam Dei requiéscant in pace.
\Rbardot{} Amen.

\vfill
\pagebreak
\fi

\hora{Ad Laudes.} %%%%%%%%%%%%%%%%%%%%%%%%%%%%%%%%%%%%%%%%%%%%%%%%%%%%%
%\sideThumbs{Laudes}

\cantusSineNeumas

\vspace{0.5cm}
\grechangedim{interwordspacetext}{0.18 cm plus 0.15 cm minus 0.05 cm}{scalable}%
\ifx\festumveldominica\undefined
\cuminitiali{}{temporalia/deusinadiutorium-communis.gtex}
\else
\cuminitiali{}{temporalia/deusinadiutorium-alter.gtex}
\fi
\grechangedim{interwordspacetext}{0.22 cm plus 0.15 cm minus 0.05 cm}{scalable}%

\vfill
%\pagebreak

\pars{Psalmus 1.} \scriptura{Lc. 1, 13; \textbf{H277}}

\vspace{-0.4cm}

\antiphona{III a}{temporalia/ant-elisabethzachariae.gtex}

\scriptura{Psalmus 92.}

\initiumpsalmi{temporalia/ps92-initium-iii-a-auto.gtex}

%\psalmusEtTranslatioT{temporalia/ps92-comb.tex}{10cm}
\input{temporalia/ps92.tex} \Abardot{}

\vfill
\pagebreak

\pars{Psalmus 2.} \scriptura{Lc. 1, 62.63; \textbf{H277}}

\vspace{-0.4cm}

\antiphona{IV E*}{temporalia/ant-innuebantpatriejus.gtex}

\scriptura{Psalmus 99.}

\initiumpsalmi{temporalia/ps99-initium-iv-E_-auto.gtex}

%\psalmusEtTranslatioT{temporalia/ps99-comb.tex}{10cm}
\input{temporalia/ps99.tex} \Abardot{}

\vfill
\pagebreak

\pars{Psalmus 3.} \scriptura{Lc. 1, 13.14; \textbf{H277}}

\vspace{-0.4cm}

\antiphona{I f}{temporalia/ant-joannesvocabitur.gtex}

\scriptura{Psalmus 62.}

\initiumpsalmi{temporalia/ps62-initium-i-f-auto.gtex}

%\psalmusEtTranslatioT{temporalia/ps62-comb.tex}{10cm}
\input{temporalia/ps62.tex} \Abardot{}

\vfill
\pagebreak

\pars{Psalmus 4.} \scriptura{Lc. 1, 15.14.63; \textbf{H275}}

\vspace{-0.4cm}

\antiphona{VIII G}{temporalia/ant-joannesest.gtex}

\scriptura{Canticum trium puerorum, Dan. 3, 57-88 et 56}

\initiumpsalmi{temporalia/dan3-initium-viii-G-auto.gtex}

%\psalmusEtTranslatioT{temporalia/dan3-comb.tex}{10cm}
\input{temporalia/dan3.tex}

\rubrica{Hic non dicitur Gloria Patri, neque Amen.}

\vfill

\vspace{-6mm}

\antiphona{}{temporalia/ant-joannesest.gtex} % repeat the antiphon - new page

\vfill
\pagebreak

\pars{Psalmus 5.} \scriptura{Lc. 1, 15.66; \textbf{H276}}

\vspace{-0.4cm}

\antiphona{VII a}{temporalia/ant-istepuer.gtex}

\scriptura{Psalmus 148.}

\initiumpsalmi{temporalia/ps148-initium-vii-a-auto.gtex}

%\psalmusEtTranslatioT{temporalia/ps148-comb.tex}{10cm}
\input{temporalia/ps148.tex}

\rubrica{Hic non dicitur Gloria Patri.}

\vfill
\pagebreak

%
\scriptura{Psalmus 149.}

\initiumpsalmi{temporalia/ps149-initium-vii-a-auto.gtex}

%\psalmusEtTranslatioT{temporalia/ps149-comb.tex}{10cm}
\input{temporalia/ps149.tex}

\rubrica{Hic non dicitur Gloria Patri.}

\vfill
\pagebreak

%
\scriptura{Psalmus 150.}

\initiumpsalmi{temporalia/ps150-initium-vii-a-auto.gtex}

%\psalmusEtTranslatioT{temporalia/ps150-comb.tex}{10cm}
\input{temporalia/ps150.tex}

\vfill

\vspace{-6mm}

\antiphona{}{temporalia/ant-istepuer.gtex} % repeat the antiphon - new page

\vfill
\pagebreak

\pars{Capitulum.} \scriptura{Is. 49, 1}

\grechangedim{interwordspacetext}{0.12 cm plus 0.15 cm minus 0.05 cm}{scalable}%
\cuminitiali{}{temporalia/capitulum-HaecDicit.gtex}
\grechangedim{interwordspacetext}{0.22 cm plus 0.15 cm minus 0.05 cm}{scalable}

% preklad Jeruz. bible
%\trCapituliI

\vfill

\pars{Responsorium breve.} \scriptura{Cf. Lc. 7, 28}

\ifx\festum\undefined
\cuminitiali{VI}{temporalia/resp-internatos-communis.gtex}
\else
\cuminitiali{VI}{temporalia/resp-internatos.gtex}
\fi

%\trResp

\vfill
\pagebreak

\pars{Hymnus}

\cuminitiali{IV}{temporalia/hym-ONimis.gtex}
\vspace{-3mm}
%\input{hym-ONimis-bohtext.tex}

\vfill
%\pagebreak

\pars{Versus.} \scriptura{Ps. 91, 13}

% Versus. %%%
\sineinitiali{temporalia/versus-iustus.gtex}

%\noindent \trVersus

\vfill
\pagebreak

\pars{Canticum Zachariæ.} \scriptura{Lc. 1, 64.67.68; \textbf{H277}}

\vspace{-4mm}

{
\grechangedim{interwordspacetext}{0.18 cm plus 0.15 cm minus 0.05 cm}{scalable}%
\antiphona{VIII G}{temporalia/ant-apertumestoszachariae.gtex}
\grechangedim{interwordspacetext}{0.22 cm plus 0.15 cm minus 0.05 cm}{scalable}%
}

%\trAntIMagnificat

\vspace{-2mm}

\scriptura{Lc. 1, 68-79}

\vspace{-1mm}

\cantusSineNeumas
\ifx\solemnis\undefined
\initiumpsalmi{temporalia/benedictus-initium-viii-G-auto.gtex}

%\vspace{-1.5mm}

%\psalmusEtTranslatioT{temporalia/benedictus-III-comb.tex}{10.2cm}
\input{temporalia/benedictus-III.tex} \Abardot{}
\else
\initiumpsalmi{temporalia/benedictus-initium-viiisoll-G-auto.gtex}

%\vspace{-1.5mm}

%\psalmusEtTranslatioT{temporalia/benedictus-I-comb.tex}{10.2cm}
\input{temporalia/benedictus-I.tex} \Abardot{}
\fi

\vspace{-1cm}

\vfill
\pagebreak

%\sideThumbs{{\scriptsize{}Fine horarum}}

\anteOrationem

\pagebreak

% Oratio. %%%
\cuminitiali{}{temporalia/oratio2.gtex}

\vspace{-1mm}
%\trOrationisI

\vfill

\rubrica{Hebdomadarius dicit iterum Dominus vobiscum, vel cantor dicit:}

\vspace{2mm}

\sineinitiali{temporalia/domineexaudi.gtex}

\rubrica{Postea cantatur a cantore:}

\vspace{2mm}

\ifx\festum\undefined
\ifx\octava\undefined
\cuminitiali{I}{temporalia/benedicamus-semiduplex-laud.gtex}
\else
\cuminitiali{VIII}{temporalia/benedicamus-duplexmajus-laudes.gtex}
\fi
\else
\cuminitiali{II}{temporalia/benedicamus-solemnism-laud.gtex}
\fi

\vspace{1mm}

\vfill
\pagebreak

\ifx\sabbatoveloctava\undefined
\ifx\festumveldominica\undefined
\hora{Ad Vesperas.} %%%%%%%%%%%%%%%%%%%%%%%%%%%%%%%%%%%%%%%%%%%%%%%%%%%%%
%\sideThumbs{Vesperæ}
\else
\hora{Ad II. Vesperas.} %%%%%%%%%%%%%%%%%%%%%%%%%%%%%%%%%%%%%%%%%%%%%%%%%%%%%
%\sideThumbs{II. Vesperæ}
\fi

\cantusSineNeumas

%\vspace{-2mm}
\grechangedim{interwordspacetext}{0.18 cm plus 0.15 cm minus 0.05 cm}{scalable}%
\ifx\festumveldominica\undefined
\cuminitiali{}{temporalia/deusinadiutorium-communis.gtex}
\else
\ifx\festum\undefined
\cuminitiali{}{temporalia/deusinadiutorium-alter.gtex}
\else
\cuminitiali{}{temporalia/deusinadiutorium-solemnis.gtex}
\fi
\fi
\grechangedim{interwordspacetext}{0.22 cm plus 0.15 cm minus 0.05 cm}{scalable}%

\vfill
%\pagebreak

%\vspace{-2mm}

\pars{Psalmus 1.} \scriptura{Lc. 1, 13; \textbf{H277}}

\vspace{-0.4cm}

\antiphona{III a}{temporalia/ant-elisabethzachariae.gtex}

\scriptura{Psalmus 109.}

\initiumpsalmi{temporalia/ps109-initium-iii-a-auto.gtex}

%\psalmusEtTranslatioT{temporalia/ps109-comb.tex}{10cm}
\input{temporalia/ps109.tex} \Abardot{}

\vfill
\pagebreak

\pars{Psalmus 2.} \scriptura{Lc. 1, 62.63; \textbf{H277}}

\vspace{-0.4cm}

\antiphona{IV E*}{temporalia/ant-innuebantpatriejus.gtex}

\scriptura{Psalmus 110.}

\initiumpsalmi{temporalia/ps110-initium-iv-E_-auto.gtex}

%\psalmusEtTranslatioT{temporalia/ps110-comb.tex}{10cm}
\input{temporalia/ps110.tex} \Abardot{}

\vfill
\pagebreak

\pars{Psalmus 3.} \scriptura{Lc. 1, 13.14; \textbf{H277}}

\vspace{-0.4cm}

\antiphona{I f}{temporalia/ant-joannesvocabitur.gtex}

\scriptura{Psalmus 111.}

\initiumpsalmi{temporalia/ps111-initium-i-f-auto.gtex}

%\psalmusEtTranslatioT{temporalia/ps111-comb.tex}{10cm}
\input{temporalia/ps111.tex} \Abardot{}

\vfill
\pagebreak

\pars{Psalmus 4.} \scriptura{Lc. 1, 15.14.63; \textbf{H275}}

\vspace{-0.4cm}

\antiphona{VIII G}{temporalia/ant-joannesest.gtex}

\scriptura{Psalmus 129.}

\initiumpsalmi{temporalia/ps129-initium-viii-G-auto.gtex}

%\psalmusEtTranslatioT{temporalia/ps129-comb.tex}{10cm}
\input{temporalia/ps129.tex} \Abardot{}

\vfill
\pagebreak

\pars{Psalmus 5.} \scriptura{Lc. 1, 15.66; \textbf{H276}}

\vspace{-0.4cm}

\antiphona{VII a}{temporalia/ant-istepuer.gtex}

\scriptura{Psalmus 131.}

\initiumpsalmi{temporalia/ps131-initium-vii-a-auto.gtex}

%\psalmusEtTranslatioT{temporalia/ps131-comb.tex}{10cm}
\input{temporalia/ps131.tex}

\vfill

\antiphona{}{temporalia/ant-istepuer.gtex}

\vfill
\pagebreak

\pars{Capitulum.} \scriptura{Is. 49, 7}

\grechangedim{interwordspacetext}{0.12 cm plus 0.15 cm minus 0.05 cm}{scalable}%
\cuminitiali{}{temporalia/capitulum-RegesVidebunt.gtex}
\grechangedim{interwordspacetext}{0.22 cm plus 0.15 cm minus 0.05 cm}{scalable}

% preklad Jeruz. bible
%\trCapituliI

\vfill

\pars{Responsorium breve.} \scriptura{Cf. Lc. 7, 28}

\cuminitiali{VI}{temporalia/resp-internatos.gtex}

%\trResp

\vfill
\pagebreak

\pars{Hymnus}

\cuminitiali{II}{temporalia/hym-UtQueant.gtex}
\vspace{-3mm}
%\input{hym-UtQueant-bohtext.tex}

\vfill
%\pagebreak

\pars{Versus.} \scriptura{Ps. 91, 13}

% Versus. %%%
\sineinitiali{temporalia/versus-iustus.gtex}

%\noindent \trVersus

\vfill
\pagebreak

\pars{Canticum B. Mariæ V.} \scriptura{Lc. 1, 59-60}

%\vspace{-5.5mm}

{
\grechangedim{interwordspacetext}{0.18 cm plus 0.15 cm minus 0.05 cm}{scalable}%
\antiphona{VIII G}{temporalia/ant-etfactumest.gtex}
\grechangedim{interwordspacetext}{0.22 cm plus 0.15 cm minus 0.05 cm}{scalable}%
}

%\trAntIMagnificat

\vspace{-3mm}

\scriptura{Lc. 1, 46-55}

\vspace{-2.5mm}

\cantusSineNeumas
\ifx\solemnis\undefined
\initiumpsalmi{temporalia/magnificat-initium-viii-G.gtex}

\vspace{-1.5mm}

%\psalmusEtTranslatioT{temporalia/magnificat-V-comb.tex}{10.2cm}
\input{temporalia/magnificat-V.tex} \Abardot{}
\else
\initiumpsalmi{temporalia/magnificat-initium-viiisoll-G.gtex}

\vspace{-1.5mm}

%\psalmusEtTranslatioT{temporalia/magnificat-II-comb.tex}{10.2cm}
\input{temporalia/magnificat-II.tex} \Abardot{}
\fi

\vspace{-1cm}

\vfill
\pagebreak

%\sideThumbs{{\scriptsize{}Fine horarum}}

\anteOrationem

\pagebreak

% Oratio. %%%
\cuminitiali{}{temporalia/oratio2.gtex}

\vspace{-1mm}
%\trOrationisI

\vfill

\rubrica{Hebdomadarius dicit iterum Dominus vobiscum, vel cantor dicit:}

\vspace{2mm}

\sineinitiali{temporalia/domineexaudi.gtex}

\rubrica{Postea cantatur a cantore:}

\vspace{2mm}

\ifx\festum\undefined
\cuminitiali{II}{temporalia/benedicamus-semiduplex-vesp.gtex}
\else
\cuminitiali{II}{temporalia/benedicamus-solemnism-2vesp.gtex}
\fi

\vspace{1mm}
\fi

\end{document}

