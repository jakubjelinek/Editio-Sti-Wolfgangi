\newcommand{\titulus}{\dies{Die 8. Ianuarii.}
\nomenFesti{Die Tertia infra Octavam Epiphaniæ.}}
\newcommand{\postoctavam}{Post octavam}
\newcommand{\invitatorium}{\pars{Invitatorium.}

%\vspace{-6mm}

\antiphona{E}{temporalia/inv-christusapparuit.gtex}}
\newcommand{\hymnusmatutinum}{\pars{Hymnus.}

{
\grechangedim{interwordspacetext}{0.10 cm plus 0.15 cm minus 0.05 cm}{scalable}%
\antiphona{III}{temporalia/hym-MagiVidentes.gtex}
\grechangedim{interwordspacetext}{0.22 cm plus 0.15 cm minus 0.05 cm}{scalable}%
}}
\newcommand{\matutinum}{\pars{Psalmus 1.} \scriptura{\textbf{H416}}

\vspace{-4mm}

\antiphona{VIII G}{temporalia/ant-extendedomine.gtex}

\vspace{-1mm}

\scriptura{Ps. 43, 2-9}

\vspace{-2mm}

\initiumpsalmi{temporalia/ps43i-initium-viii-G-auto.gtex}

\vspace{-1.5mm}

\input{temporalia/ps43i-viii-G.tex} \Abardot{}

\vfill
\pagebreak

\pars{Psalmus 2.} \scriptura{Ie. 17, 18; \textbf{H174}}

\vspace{-4mm}

\antiphona{II* a}{temporalia/ant-confundanturqui.gtex}

%\vspace{-2mm}

\scriptura{Ps. 43, 10-17}

\initiumpsalmi{temporalia/ps43ii-initium-ii_-a-auto.gtex}

\input{temporalia/ps43ii-ii_-a.tex} \Abardot{}

\vfill
\pagebreak

\pars{Psalmus 3.} \scriptura{2 Esr. 6, 14; Tb. 3, 13}

\vspace{-4mm}

\antiphona{II D}{temporalia/ant-mementodomine.gtex}

%\vspace{-2mm}

\scriptura{Ps. 43, 18-26}

%\vspace{-2mm}

\initiumpsalmi{temporalia/ps43iii-initium-ii-D-auto.gtex}

\input{temporalia/ps43iii-ii-D.tex} \Abardot{}

\vfill
\pagebreak

\pars{Versus.}

\noindent \Vbardot{} Lauda, Ierúsalem, Dóminum.

\noindent \Rbardot{} Qui emíttit elóquium suum in terra.

\vspace{5mm}

\sineinitiali{temporalia/oratiodominica-mat.gtex}

\vspace{5mm}

\pars{Absolutio.}

\cuminitiali{}{temporalia/absolutio-exaudi.gtex}

\vfill
\pagebreak

\cuminitiali{}{temporalia/benedictio-solemn-benedictione.gtex}

\vspace{7mm}

\pars{Lectio I.} \scriptura{Is. 62, 1-12}

\noindent De libro Isaíæ prophétæ.

\noindent Propter Sion non tacébo et propter Ierúsalem non quiéscam,

\noindent donec egrediátur ut splendor iustítia eius, et salus eius ut lampas accendátur.

\noindent Et vidébunt gentes iustítiam tuam et cuncti reges glóriam tuam;

\noindent et vocáberis nómine novo, quod os Dómini nominábit.

\noindent Et eris coróna glóriæ in manu Dómini et diadéma regni in manu Dei tui.

\noindent Non vocáberis ultra Derelícta, et terra tua non vocábitur ámplius Desoláta;

\noindent sed vocáberis Beneplácitum meum in ea, et terra tua Nupta, quia complácuit Dómino in te, et terra tua erit nupta.

\noindent Nam ut iúvenis uxórem ducit vírginem, ita ducent te fílii tui;

\noindent ut gaudet sponsus super sponsam, ita gaudébit super te Deus tuus.

\noindent Super muros tuos Ierúsalem constítui custódes; tota die et tota nocte, in perpétuo non tacébunt.

\noindent Qui commonétis Dóminum, ne taceátis et ne detis siléntium ei,

\noindent donec stabíliat et donec ponat Ierúsalem laudem in terra.

\noindent Iurávit Dóminus in déxtera sua et in bráchio fortitúdinis suæ:

\noindent «Non dabo tríticum tuum ultra cibum inimícis tuis,

\noindent neque bibent fílii aliéni vinum tuum, in quo laborásti.

\noindent Quia, qui collígerint illud, cómedent et laudábunt Dóminum;

\noindent et qui vindémiam fécerint, illud bibent in átriis sanctuárii mei.

\noindent Transíte, transíte per portas, paráte viam pópulo.

\noindent Stérnite, stérnite sémitam, elígite lápides, eleváte signum ad pópulos».

\noindent Ecce Dóminus audítum fecit in extrémis terræ:

\noindent «Dícite fíliæ Sion: Ecce salus tua venit, ecce merces eius cum eo et prǽmium eius coram illa.

\noindent Et vocábunt eos Pópulus sanctus, Redémpti a Dómino; tu autem vocáberis Quæsíta, Cívitas non derelícta».

\noindent \Vbardot{} Tu autem, Dómine, miserére nobis.
\noindent \Rbardot{} Deo grátias.

\vfill
\pagebreak

\pars{Responsorium 1.} \scriptura{\Rbar{} Mt. 2, 9-10 \Vbar{} ibid. 2, 11; \textbf{H73}}

\vspace{-5mm}

\responsorium{VIII}{temporalia/resp-stellaquamviderunt-CROCHU.gtex}{}

\vfill
\pagebreak

\cuminitiali{}{temporalia/benedictio-solemn-unigenitus.gtex}

\vspace{7mm}

\pars{Lectio II.} \scriptura{Nn. 2. 6-8. 10: PG 10, 854. 858-859. 862}

\noindent Ex Sermóne in sancta Theophanía sancto Hippólyto presbýtero attribúto.

\noindent Ad Ioánnem venit Iesus, et baptísmum ab eo suscépit.

\noindent O res digníssima admiratióne! Infinítum flumen, quod lætíficat civitátem Dei exígua ablúitur aqua.

\noindent Fons incomprehensíbilis, qui vitam ómnibus homínibus progérminat et término caret, a parvis et temporáriis aquis obrúitur.

\noindent Qui ubíque præsens est nec usquam abest, incomprehensíbilis ángelis et ab hóminum conspéctu remótus, ad baptísmum accédit ut ipsi plácuit.

\noindent \emph{Et ecce apérti sunt ei cæli et vox facta est dicens: Hic est Fílius meus diléctus, in quo complácui.}

\noindent Amátus amórem génerat, et lux immateriális lucem inaccessíbilem.

\noindent Hic est qui Ioséphi nominátus est fílius, et meus est Unigénitus secúndum divínam esséntiam.

\noindent Hic est Fílius meus diléctus:

\noindent esúriens ille, qui innúmera alit mília;

\noindent labórans, idémque récreans laborántes;

\noindent non habens ubi caput reclínet suum, et ómnia gerens manu;

\noindent qui pátitur, et ómnibus medétur passiónibus;

\noindent qui cǽditur cólaphis, et mundum donat libertáte;

\noindent qui in látere percútitur, et latus Adámi córrigit.

\noindent Sed mentem, quæso, mihi accuráte inténdite: volo enim recúrrere ad fontem vitæ, et fontem medélas scaturiéntem contemplári.

\noindent \Vbardot{} Tu autem, Dómine, miserére nobis.
\noindent \Rbardot{} Deo grátias.

\vfill
\pagebreak

\pars{Responsorium 2.} \scriptura{\Rbar{} Mt. 3, 16-17 \Vbar{} ibidem; \textbf{H75}}

\vspace{-5mm}

\responsorium{II}{temporalia/resp-incolumbaespecie-CROCHU.gtex}{}

\vfill
\pagebreak

\cuminitiali{}{temporalia/benedictio-solemn-spiritus.gtex}

\vspace{7mm}

\pars{Lectio III.}

\noindent Pater immortalitátis immortálem Fílium ac Verbum in mundum misit, qui venit ad hómines, lotúrus eos aqua et Spíritu:

\noindent et regeneratúrus ad ánimæ corporísque incorruptibilitátem, inspirávit in nos spíritum vitæ, et incorruptíbili armatúra nos índuit.

\noindent Si ígitur homo factus immortális est, deus étiam erit.

\noindent Si vero per aquam et Spíritum Sanctum a regeneratióne ex lavácro deus fit, comperítur étiam post resurrectiónem a mórtuis cohéres Christi esse.

\noindent Igitur præcónis voce proclámo: Veníte, omnes tribus géntium, ad baptísmatis immortalitátem.

\noindent Hæc est aqua cum Spíritu coniúncta, qua paradísus rigátur, terra pinguéscit, increméntum plantæ cápiunt, génerant animália;

\noindent atque ut ómnia compéndio ampléctar, per quam regenerátus homo vivificátur, qua Christus baptizátus est, in quam Spíritus Sanctus colúmbæ spécie descéndit.

\noindent Qui enim cum fide in hoc regeneratiónis lavácrum descéndit, renúntiat diábolo, et Christo se addícit;

\noindent hostem ábnegat, at Christum Deum esse confitétur;

\noindent servitútem éxuit, índuit adoptiónem;

\noindent redit ex baptísmo spléndidus ut sol, rádios iustítiæ effúlgurans;

\noindent quod vero máximum est, revértitur fílius Dei et Christi cohéres.

\noindent Ipsi glória et poténtia cum sanctíssimo, bono et vivífico eius Spíritu, nunc sit et semper et in ómnia sǽcula sæculórum. Amen.

\noindent \Vbardot{} Tu autem, Dómine, miserére nobis.
\noindent \Rbardot{} Deo grátias.

\vfill
\pagebreak

\pars{Responsorium 3.} \scriptura{\textbf{H75}}

\vspace{-5mm}

\responsorium{VIII}{temporalia/resp-diessanctificatusilluxit-CROCHU-cumdox.gtex}{}

\rubrica{vel ad libitum:}

\vspace{3mm}

\pars{Responsorium 3.} \scriptura{\Rbar{} Is. 60, 1 \Vbar{} ibid. 60, 3; \textbf{H74}}

\vspace{-5mm}

\responsorium{V}{temporalia/resp-illuminare-CROCHU-cumdox.gtex}{}

\vfill
\pagebreak}
\newcommand{\hymnuslaudes}{\pagebreak
\pars{Hymnus}

\grechangedim{interwordspacetext}{0.16 cm plus 0.15 cm minus 0.05 cm}{scalable}%
\cuminitiali{III}{temporalia/hym-QuicumqueChristum.gtex}
\grechangedim{interwordspacetext}{0.22 cm plus 0.15 cm minus 0.05 cm}{scalable}%
\vspace{-3mm}}
\newcommand{\laudes}{\pars{Psalmus 1.} \scriptura{Ps. 79, 3; \textbf{H19}}

\vspace{-4mm}

\antiphona{II* b}{temporalia/ant-tuamdomineexcita.gtex}

\vspace{-2mm}

\scriptura{Psalmus 79.}

\vspace{-1mm}

\initiumpsalmi{temporalia/ps79-initium-ii_-B-auto.gtex}

\input{temporalia/ps79-ii_-B.tex}

\vfill

\antiphona{}{temporalia/ant-tuamdomineexcita.gtex}

\vfill
\pagebreak

\pars{Psalmus 2.} \scriptura{Is. 12, 1; \textbf{H93}}

\vspace{-4mm}

\antiphona{VIII G}{temporalia/ant-conversusestfuror.gtex}

\scriptura{Canticum Isaiæ Prophetæ, Is. 12, 1-7}

\initiumpsalmi{temporalia/isaiae-initium-viii-G-auto.gtex}

\input{temporalia/isaiae-viii-G.tex} \Abardot{}

\vfill
\pagebreak

\pars{Psalmus 3.} \scriptura{Ps. 80, 2}

\vspace{-4.5mm}

\antiphona{I g\textsuperscript{5}}{temporalia/ant-exsultatedeo.gtex}

\vspace{-2.5mm}

\scriptura{Psalmus 80.}

\vspace{-2mm}

\initiumpsalmi{temporalia/ps80-initium-i-g5-auto.gtex}

\vspace{-1.5mm}

\input{temporalia/ps80-i-g5.tex} \Abardot{}

\vfill
\pagebreak
}
\newcommand{\lectiobrevis}{\pars{Lectio Brevis.} \scriptura{Is. 4, 2-3}

\noindent In die illa erit germen Dómini in splendórem et glóriam et fructus terræ sublímis et exsultátio his, qui salváti fúerint de Israel. Et erit: omnis qui relíctus fúerit in Sion et resíduus in Ierúsalem, sanctus vocábitur, omnis, qui scriptus est ad vitam in Ierúsalem.}
\newcommand{\responsoriumbreve}{\pars{Responsorium breve.} \scriptura{Ps. 71, 11}

\cuminitiali{VI}{temporalia/resp-adorabunteum.gtex}}
\newcommand{\oratio}{\pars{Oratio.}

\noindent Deus, cuius Unigénitus in substántia nostræ carnis appáruit, præsta, quǽsumus, ut per eum, quem símilem nobis foris agnóvimus, intus reformári mereámur.

%\noindent Qui tecum vivit et regnat in unitáte Spíritus Sancti, Deus, per ómnia sǽcula sæculórum.
\vfill

\pars{Pro pace in universo mundo.} \scriptura{Sir. 50, 25; 2 Esdr. 4, 20; \textbf{H416}}

\vspace{-4mm}

\antiphona{II D}{temporalia/ant-dapacemdomine.gtex}

\vfill

\noindent Deus, a quo sancta desidéria, recta consília et iusta sunt ópera: da servis tuis illam, quam mundus dare non potest, pacem; ut et corda nostra mandátis tuis dédita, et hóstium subláta formídine, témpora sint tua protectióne tranquílla.

\noindent Per Dóminum nostrum Iesum Christum, Fílium tuum, qui tecum vivit et regnat in unitáte Spíritus Sancti, Deus, per ómnia sǽcula sæculórum.

\noindent \Rbardot{} Amen.}
\newcommand{\benedictus}{\pars{Canticum Zachariæ.} \scriptura{Mt. 2, 11; \textbf{H74}}

\vspace{-4mm}

{
\grechangedim{interwordspacetext}{0.18 cm plus 0.15 cm minus 0.05 cm}{scalable}%
\antiphona{IV E}{temporalia/ant-triasuntmunera.gtex}
\grechangedim{interwordspacetext}{0.22 cm plus 0.15 cm minus 0.05 cm}{scalable}%
}

%\vspace{-3mm}

\scriptura{Lc. 1, 68-79}

%\vspace{-2mm}

\cantusSineNeumas
\initiumpsalmi{temporalia/benedictus-initium-iv-E-auto.gtex}

\vspace{-1.5mm}

\input{temporalia/benedictus-iv-E.tex} \Abardot{}}
\newcommand{\preces}{\noindent Misericórdiam Christi celebrémus,~\gredagger{} qui venit ut creatúra liberarétur a servitúte corruptiónis in libertátem filiórum Dei.~\grestar{} Hac divína freti pietáte, rogémus:

\Rbardot{} Per nativitátem tuam, líbera nos a malo.

\noindent Dómine, qui, ab ætérno exsístens, novam vitam ingréssus es,~\grestar{} rénova nos semper per mystérium natális tui.

\Rbardot{} Per nativitátem tuam, líbera nos a malo.

\noindent Qui divinitátem non amíttens, humanitátem mirabíliter assumpsísti,~\grestar{} præsta, ut vita nostra ad pleniórem divinitátis tuæ participatiónem nitátur.

\Rbardot{} Per nativitátem tuam, líbera nos a malo.

\noindent Qui véniens, lumen géntium et magíster sanctitátis factus es,~\grestar{} præsta, ut sermo tuus sit lucérna pédibus nostris.

\Rbardot{} Per nativitátem tuam, líbera nos a malo.

\noindent Verbum Dei, quod in sinu Maríæ Vírginis caro factum es et in hunc mundum venísti,~\grestar{} in córdibus nostris per fidem semper inhabitáre dignéris.

\Rbardot{} Per nativitátem tuam, líbera nos a malo.}
\newcommand{\sinevesperas}{Sine Vesperas}
% LuaLaTeX

\documentclass[a4paper, twoside, 12pt]{article}
\usepackage[latin]{babel}
%\usepackage[landscape, left=3cm, right=1.5cm, top=2cm, bottom=1cm]{geometry} % okraje stranky
%\usepackage[landscape, a4paper, mag=1166, truedimen, left=2cm, right=1.5cm, top=1.6cm, bottom=0.95cm]{geometry} % okraje stranky
\usepackage[landscape, a4paper, mag=1400, truedimen, left=0.5cm, right=0.5cm, top=0.5cm, bottom=0.5cm]{geometry} % okraje stranky

\usepackage{fontspec}
\setmainfont[FeatureFile={junicode.fea}, Ligatures={Common, TeX}, RawFeature=+fixi]{Junicode}
%\setmainfont{Junicode}

% shortcut for Junicode without ligatures (for the Czech texts)
\newfontfamily\nlfont[FeatureFile={junicode.fea}, Ligatures={Common, TeX}, RawFeature=+fixi]{Junicode}

\usepackage{multicol}
\usepackage{color}
\usepackage{lettrine}
\usepackage{fancyhdr}

% usual packages loading:
\usepackage{luatextra}
\usepackage{graphicx} % support the \includegraphics command and options
\usepackage{gregoriotex} % for gregorio score inclusion
\usepackage{gregoriosyms}
\usepackage{wrapfig} % figures wrapped by the text
\usepackage{parcolumns}
\usepackage[contents={},opacity=1,scale=1,color=black]{background}
\usepackage{tikzpagenodes}
\usepackage{calc}
\usepackage{longtable}
\usetikzlibrary{calc}

\setlength{\headheight}{14.5pt}

\input{conventuscommune.tex} % Often used macros
%%%% Preklady jednotlivych zpevu (nektere se opakuji, a je dobre mit je
% vsechny na jedne hromade)

% HOURS ---

\newcommand{\trAntI}{\translatioCantus{Muž boží měl kožený toulec, pečlivě
zavázaný, jenž mu visel na šíji a~často se ho dotýkal.}}

\newcommand{\trAntII}{\translatioCantus{Klíč od~něho tak dobře střežil, že
dokud žil v~těle, nikdo z~jeho žáků nezvěděl, co je uvnitř.}}

\newcommand{\trAntIII}{\translatioCantus{Ale když se odebral z~tohoto
života, schránku otevřeli a~objevili v~ní žíněné roucho a~měděný řetěz
potřísněný krví.}}

\newcommand{\trAntIV}{\translatioCantus{A když prohlédli mistrovo tělo,
nalezli jeho tělo na čtyřech místech hluboce zbrázděno ranami od řetězu.}}

\newcommand{\trAntV}{\translatioCantus{Krev vytékající z~těch ran, místy
prostoupila i~žíněným rouchem.}}

\newcommand{\trCapituli}{\translatioCantus{
Miláčkovi Boha a~lidí,
Mojžíšovi požehnané paměti,~\gredagger{}
dopřál slávu rovnou slávě svatých~\grestar{}
učinil ho mocným na postrach nepřátelům
a~jeho slovy zastavil divy.}}

\newcommand{\trLectioBrevis}{\translatioCantus{
Pamatujte na své představené,
kteří vám hlásali Boží slovo.
Uvažte, jak oni skončili život, a~napodobujte jejich víru.
Ježíš Kristus je stejný včera i~dnes i~navěky.
Nenechte se svést věelijakými cizími naukami.}}

\newcommand{\trRespLaud}{\translatioCantus{Spravedlivého vodil Hospodin~\grestar{}
po přímých stezkách. \Vbardot{} A~ukázal mu Boží království.}}

\newcommand{\trRespLaudB}{\translatioCantus{Na tvých hradbách, Jeruzaléme,
ustanovil jsem strážné;~\grestar{}
budou bdít nad mým lidem. \Vbardot{} Ani ve dne, ani v~noci nesmějí nikdy
mlčet.}}

\newcommand{\trVersus}{\translatioCantus{\Vbardot{} Ústa spravedlivého šeptají moudrost, aleluja.
\Rbardot{} A~jeho jazyk ohlašuje právo, aleluja.}}

\newcommand{\trAntBenedictus}{\translatioCantus{Když na bujné oře vložili
nosítka a~sňali jim uzdu, vydali se přímo k~cele božího muže.}}

\newcommand{\trPreces}{\translatioCantus{
\noindent S vděčností chvalme Krista, dobrého Pastýře, \gredagger{} který dal život za své ovce, \grestar{} a~pokorně ho prosme: \Rbardot{} Pane, buď pastýřem svého lidu.

\noindent Kriste, ty dáváš církvi pastýře, a~jejich službou se ujímáš svého lidu, \grestar{} dej, ať v~lásce těch, kteří nás vedou, poznáváme, jak nás miluješ. \Rbardot{} Pane, buď pastýřem svého lidu.

\noindent Ty stále konáš skrze své zástupce službu pastýře a~učitele, \grestar{} nepřestávej nás nikdy vést prostřednictvím svých služebníků. \Rbardot{} Pane, buď pastýřem svého lidu.

\noindent Ty prokazuješ svému lidu skrze jeho pastýře službu lékaře duše i~těla, \grestar{} ochraňuj náš život a~veď nás ke svatosti. \Rbardot{} Pane, buď pastýřem svého lidu.

\noindent Ty posíláš své svaté, aby slovem i~příkladem vedli tvůj lid k~tobě, \grestar{} na jejich přímluvu nás posiluj, abychom vytrvali na cestě, která vede k~věčnému životu. \Rbardot{} Pane, buď pastýřem svého lidu.}}

\newcommand{\trOrationis}{\translatioCantus{Bože, jenž nám dopřáváš radovat
se z~výroční slavnosti svatého tvého vyznavače Havla, uděl dobrotivě,
abychom když slavíme jeho narození, též se řídili podobou jeho skutků.
Skrze…}}
 % Czech translations of the proper texts

\newcommand{\annusEditionis}{2020}

%%%% Vicekrat opakovane kousky

\newcommand{\anteOrationem}{
  \rubrica{Ante Orationem, cantatur a Superiore:}

  \pars{Supplicatio Litaniæ.}

  \cuminitiali{}{temporalia/supplicatiolitaniae.gtex}

  \pars{Oratio Dominica.}

  \cuminitiali{}{temporalia/oratiodominica.gtex}

  \rubrica{Deinde dicitur ab Hebdomadario:}

  \cuminitiali{}{temporalia/dominusvobiscum-solemnis.gtex}

  \rubrica{In choro monialium loco Dominus vobiscum dicitur:}

  \sineinitiali{temporalia/domineexaudi.gtex}
}

\setlength{\columnsep}{30pt} % prostor mezi sloupci

%%%%%%%%%%%%%%%%%%%%%%%%%%%%%%%%%%%%%%%%%%%%%%%%%%%%%%%%%%%%%%%%%%%%%%%%%%%%%%%%%%%%%%%%%%%%%%%%%%%%%%%%%%%%%
\begin{document}

% Here we set the space around the initial.
% Please report to http://home.gna.org/gregorio/gregoriotex/details for more details and options
\grechangedim{afterinitialshift}{2.2mm}{scalable}
\grechangedim{beforeinitialshift}{2.2mm}{scalable}
\grechangedim{interwordspacetext}{0.22 cm plus 0.15 cm minus 0.05 cm}{scalable}%
\grechangedim{annotationraise}{-0.2cm}{scalable}

% Here we set the initial font. Change 38 if you want a bigger initial.
% Emit the initials in red.
\grechangestyle{initial}{\color{red}\fontsize{38}{38}\selectfont}

\pagestyle{empty}

%%%% Titulni stranka
\begin{titulusOfficii}
\titulus
\end{titulusOfficii}

\vfill

\begin{center}
%Ad usum et secundum consuetudines chori \guillemotright{}Conventus Choralis\guillemotleft.

%Editio Sancti Wolfgangi \annusEditionis
\end{center}

\scriptura{}

\pars{}

\pagebreak

\renewcommand{\headrulewidth}{0pt} % no horiz. rule at the header
\fancyhf{}
\pagestyle{fancy}

\cantusSineNeumas

\pars{Oratio ante divinum Officium.}

\lettrine{{\color{red}A}}{peri,} Dómine, os meum ad benedicéndum nomen sanctum tuum:
munda quoque cor meum ab ómnibus vanis, pervérsis, et aliénis
cogitatiónibus:
intelléctum illúmina, afféctum inflámma,
ut digne, atténte ac devóte hoc Offícium recitáre váleam,
et exaudíri mérear ante conspéctum Divínæ Maiestátis tuæ.
Per Christum, Dóminum nostrum.
\Rbardot{} Amen.

Dómine, in unióne illíus divínæ intentiónis,
qua ipse in terris laudes Deo persolvísti,
has tibi Horas \rubricatum{(vel \textnormal{hanc tibi Horam})} persólvo.

%\trOratioAnteOfficium

\vfill

\pars{Oratio post divinum Officium.}

\rubrica{
  Orationem sequentem devote post Officium recitantibus
  Leo Papa X. defectus, et culpas in eo persolvendo ex humana
  fragilitate contractas, indulsit, et dicitur flexis genibus.
}

\lettrine{{\color{red}S}}{acrosánctæ} et indivíduæ Trinitáti,
crucifíxi Dómini nostri Iesu Christi humanitáti,
beatíssimæ et gloriosíssimæ sempérque Vírginis Maríæ
fecúndæ integritáti, 
et ómnium Sanctórum universitáti
sit sempitérna laus, honor, virtus et glória
ab omni creatúra,
nobísque remíssio ómnium peccatórum,
per infiníta sǽcula sæculórum.
\Rbardot{} Amen.

\noindent \Vbardot{} Beáta víscera Maríæ Virginis, quæ portavérunt
ætérni Patris Fílium.\\
\Rbardot{} Et beáta úbera, quæ lactavérunt Christum Dominum.

\rubrica{Et dicitur secreto \textnormal{Pater noster.} et \textnormal{Ave María.}}

%\trOratioPostOfficium

\vfill

\cantusSineNeumas

\pars{} \scriptura{}

\hora{Ad Matutinum.} %%%%%%%%%%%%%%%%%%%%%%%%%%%%%%%%%%%%%%%%%%%%%%%%%%%%%
%\sideThumbs{Matutinum}

\vspace{2mm}

\cuminitiali{}{temporalia/dominelabiamea.gtex}

\vfill
\pagebreak

\vspace{2mm}

\pars{Invitatorium.}

\vspace{-6mm}

%\antiphona{IV**}{temporalia/inv-christusnatusest.gtex}
\antiphona{E}{temporalia/inv-christusnatusest-simplex.gtex}

\vfill
\pagebreak

\pars{Hymnus.}

{
\grechangedim{interwordspacetext}{0.10 cm plus 0.15 cm minus 0.05 cm}{scalable}%
\antiphona{IV}{temporalia/hym-CandorAEternae-simplex.gtex}
\grechangedim{interwordspacetext}{0.22 cm plus 0.15 cm minus 0.05 cm}{scalable}%
}

\vspace{-3mm}

\vfill
\pagebreak

\matutinum

\ifx\postoctavam\undefined
% Te Deum

%\pars{Hymnus Ambrosianus}

\vspace{-5mm}

\pars{Hymnus Ambrosianus} \scriptura{Alio modo, iuxta morem Romanum}

\vspace{-2mm}

{
\grechangedim{interwordspacetext}{0.26 cm plus 0.15 cm minus 0.05 cm}{scalable}%
\cuminitiali{III}{temporalia/tedeum-romanum-gn.gtex}
\grechangedim{interwordspacetext}{0.22 cm plus 0.15 cm minus 0.05 cm}{scalable}%
}

\vfill
\pagebreak
\fi

\rubrica{Reliqua omittuntur, nisi Laudes separandæ sint.}

\sineinitiali{temporalia/domineexaudi.gtex}

\vfill

\oratio

\vfill

\noindent \Vbardot{} Dómine, exáudi oratiónem meam.
\Rbardot{} Et clamor meus ad te véniat.

\vfill

% Nocturnale Romanum 2002, p. LXXVI Benedicamus Domino seems to match
% the one from Solemn Laudes.
\cuminitiali{V}{temporalia/benedicamus-solemnis-laud.gtex}

\vfill

\noindent \Vbardot{} Fidélium ánimæ per misericórdiam Dei requiéscant in pace.
\Rbardot{} Amen.

\vfill
\pagebreak

\hora{Ad Laudes.} %%%%%%%%%%%%%%%%%%%%%%%%%%%%%%%%%%%%%%%%%%%%%%%%%%%%%
%\sideThumbs{Laudes}

\cantusSineNeumas

\vspace{0.5cm}
\grechangedim{interwordspacetext}{0.18 cm plus 0.15 cm minus 0.05 cm}{scalable}%
\ifx\postoctavam\undefined
\cuminitiali{}{temporalia/deusinadiutorium-alter.gtex}
\else
\cuminitiali{}{temporalia/deusinadiutorium-communis.gtex}
\fi
\grechangedim{interwordspacetext}{0.22 cm plus 0.15 cm minus 0.05 cm}{scalable}%

\vfill
%\pagebreak

\pars{Hymnus} \scriptura{Sedulius}

\grechangedim{interwordspacetext}{0.16 cm plus 0.15 cm minus 0.05 cm}{scalable}%
\cuminitiali{III}{temporalia/hym-ASolisOrtus.gtex}
\grechangedim{interwordspacetext}{0.22 cm plus 0.15 cm minus 0.05 cm}{scalable}%
\vspace{-3mm}
%\input{hym-ASolisOrtus-bohtext.tex}
\vfill
%\pagebreak

\vfill
\pagebreak

\pars{Psalmus 1.} \scriptura{Lc. 2, 8.11.13.18; \textbf{H50}}

\vspace{-4mm}

\antiphona{II D}{temporalia/ant-quemvidistis.gtex}

\vspace{-2mm}

\scriptura{Psalmus 62.}

\vspace{-1mm}

\initiumpsalmi{temporalia/ps62-initium-ii-D-auto.gtex}

%\psalmusEtTranslatioT{temporalia/ps62-comb.tex}{10cm}
\input{temporalia/ps62.tex} \Abardot{}

\vfill
\pagebreak

\pars{Psalmus 2.} \scriptura{Lc. 2, 10.11; \textbf{H50}}

\vspace{-4mm}

\antiphona{VII d}{temporalia/ant-angelusadpastores.gtex}

\scriptura{Canticum trium puerorum, Dan. 3, 57-88 et 56}

\initiumpsalmi{temporalia/dan3-initium-vii-d-auto.gtex}

%\psalmusEtTranslatioT{temporalia/dan3-comb.tex}{10cm}
\input{temporalia/dan3.tex}

\rubrica{Hic non dicitur Gloria Patri, neque Amen.}

\vfill

\vspace{-6mm}

\antiphona{}{temporalia/ant-angelusadpastores.gtex} % repeat the antiphon - new page

\vfill
\pagebreak

\pars{Psalmus 3.} \scriptura{Is. 9, 6; \textbf{H51}}

\vspace{-4mm}

\antiphona{VIII G\textsuperscript{2}}{temporalia/ant-parvulusfilius.gtex}

\scriptura{Psalmus 149.}

\initiumpsalmi{temporalia/ps149-initium-viii-G2-auto.gtex}

%\psalmusEtTranslatioT{temporalia/ps149-comb.tex}{10cm}
\input{temporalia/ps149gp.tex} \Abardot{}

\vfill
\pagebreak

\lectiobrevis

% preklad Jeruz. bible
%\trCapituliI

\vfill

\pars{Responsorium breve.} \scriptura{Ps. 97, 2}

\cuminitiali{VI}{temporalia/resp-notumfecit.gtex}

%\trResp

\vfill
\pagebreak

\benedictus

\vspace{-1cm}

\vfill
\pagebreak

%\sideThumbs{{\scriptsize{}Fine horarum}}

\pars{Preces.}

\sineinitiali{}{temporalia/tonusprecum.gtex}

\preces

\vfill

\pars{Oratio Dominica.}

\cuminitiali{}{temporalia/oratiodominicaalt.gtex}

\vfill
\pagebreak

\rubrica{vel:}

\pars{Supplicatio Litaniæ.}

\cuminitiali{}{temporalia/supplicatiolitaniae.gtex}

\vfill

\pars{Oratio Dominica.}

\cuminitiali{}{temporalia/oratiodominica.gtex}

\vfill
\pagebreak

% Oratio. %%%
\oratio

\vspace{-1mm}
%\trOrationisI

\vfill

\ifx\commemoratio\undefined
\else
\commemoratio
\fi

\rubrica{Hebdomadarius dicit Dominus vobiscum, vel, absente sacerdote vel diacono, sic concluditur:}

\vspace{2mm}

\antiphona{C}{temporalia/dominusnosbenedicat.gtex}

\rubrica{Postea cantatur a cantore:}

\vspace{2mm}

\ifx\postoctavam\undefined
\cuminitiali{II}{temporalia/benedicamus-solemnism-laud.gtex}
\else
\cuminitiali{IV}{temporalia/benedicamus-feria-laudes.gtex}
\fi

\vspace{1mm}

\vfill
\pagebreak

\ifx\sinevesperas\undefined
\hora{Ad Vesperas.} %%%%%%%%%%%%%%%%%%%%%%%%%%%%%%%%%%%%%%%%%%%%%%%%%%%%%
%\sideThumbs{Vesperæ}

\cantusSineNeumas

%\vspace{-2mm}
\grechangedim{interwordspacetext}{0.18 cm plus 0.15 cm minus 0.05 cm}{scalable}%
\ifx\postoctavam\undefined
\cuminitiali{}{temporalia/deusinadiutorium-solemnis.gtex}
\else
\cuminitiali{}{temporalia/deusinadiutorium-communis.gtex}
\fi
\grechangedim{interwordspacetext}{0.22 cm plus 0.15 cm minus 0.05 cm}{scalable}%

\vfill
%\pagebreak

\vspace{-2mm}

\pars{Psalmus 1.} \scriptura{Ps. 109, 3; \textbf{H52}}

\vspace{-5mm}

\antiphona{I g}{temporalia/ant-tecumprincipium.gtex}

\scriptura{Psalmus 109.}

\initiumpsalmi{temporalia/ps109-initium-i-g-auto.gtex}

\vspace{-1.5mm}

%\psalmusEtTranslatioT{temporalia/ps109-comb.tex}{10cm}
\input{temporalia/ps109.tex} \Abardot{}

\vfill
\pagebreak

\pars{Psalmus 2.} \scriptura{Ps. 110, 9; \textbf{H52}}

\vspace{-4mm}

\antiphona{VII a}{temporalia/ant-redemptionemmisit.gtex}

\scriptura{Psalmus 110.}

\initiumpsalmi{temporalia/ps110-initium-vii-a-auto.gtex}

%\psalmusEtTranslatioT{temporalia/ps110-comb.tex}{10cm}
\input{temporalia/ps110.tex} \Abardot{}

\vfill
\pagebreak

\pars{Psalmus 3.} \scriptura{Ps. 111, 4; \textbf{H52}}

\vspace{-4mm}

\antiphona{VII d}{temporalia/ant-exortumest.gtex}

\scriptura{Psalmus 111.}

\initiumpsalmi{temporalia/ps111-initium-vii-d-auto.gtex}

%\psalmusEtTranslatioT{temporalia/ps111-comb.tex}{10cm}
\input{temporalia/ps111.tex} \Abardot{}

\vfill
\pagebreak

\ifx\impar\undefined
\pars{Psalmus 4.} \scriptura{Ps. 131, 11; \textbf{H52}}

\vspace{-4mm}

\antiphona{VIII G}{temporalia/ant-defructuventris.gtex}

\scriptura{Psalmus 131.}

\initiumpsalmi{temporalia/ps131-initium-viii-G-auto.gtex}

%\psalmusEtTranslatioT{temporalia/ps131-comb.tex}{10cm}
\input{temporalia/ps131.tex}

\vfill

\antiphona{}{temporalia/ant-defructuventris.gtex}
\else
\pars{Psalmus 4.} \scriptura{Ps. 129, 7; \textbf{H52}}

\vspace{-4mm}

\antiphona{II* b}{temporalia/ant-apuddominum.gtex}

\scriptura{Psalmus 129.}

\initiumpsalmi{temporalia/ps129-initium-ii_-B-auto.gtex}

%\psalmusEtTranslatioT{temporalia/ps129-comb.tex}{10cm}
\input{temporalia/ps129.tex} \Abardot{}
\fi

\vfill
\pagebreak

\pars{Capitulum.} \scriptura{Hebr. 1, 1-2}

\grechangedim{interwordspacetext}{0.12 cm plus 0.15 cm minus 0.05 cm}{scalable}%
\cuminitiali{}{temporalia/capitulum-Multifariam.gtex}
\grechangedim{interwordspacetext}{0.22 cm plus 0.15 cm minus 0.05 cm}{scalable}

% preklad Jeruz. bible
%\trCapituliI

\vfill

\pars{Responsorium breve.} \scriptura{Io. 1, 14}

\cuminitiali{VI}{temporalia/resp-verbumcaro-simplex.gtex}

%\trResp

\vfill
\pagebreak

\pars{Hymnus}

\cuminitiali{I}{temporalia/hym-ChristeRedemptor.gtex}
\vspace{-3mm}
%\input{hym-ChristeRedemptor-bohtext.tex}

\vfill
%\pagebreak

\pars{Versus.} \scriptura{Ps. 97, 2}

% Versus. %%%
\sineinitiali{temporalia/versus-notumfecit-communis.gtex}

%\noindent \trVersus

\vfill
\pagebreak

\magnificat

\vfill
\pagebreak

%\sideThumbs{{\scriptsize{}Fine horarum}}

\anteOrationem

\pagebreak

% Oratio. %%%
\cuminitiali{}{temporalia/oratio.gtex}

\vspace{-1mm}
%\trOrationisI

\vfill

\rubrica{Hebdomadarius dicit iterum Dominus vobiscum, vel cantor dicit:}

\vspace{2mm}

\sineinitiali{temporalia/domineexaudi.gtex}

\rubrica{Postea cantatur a cantore:}

\vspace{2mm}

\ifx\postoctavam\undefined
\cuminitiali{II}{temporalia/benedicamus-solemnism-2vesp.gtex}
\else
\cuminitiali{I}{temporalia/benedicamus-feria-vesperae.gtex}
\fi

\vspace{1mm}
\fi

\end{document}

