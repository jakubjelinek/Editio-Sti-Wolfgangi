% LuaLaTeX

\documentclass[a4paper, twoside, 12pt]{article}
\usepackage[latin]{babel}
%\usepackage[landscape, left=3cm, right=1.5cm, top=2cm, bottom=1cm]{geometry} % okraje stranky
\usepackage[portrait, a4paper, mag=1300, truedimen, left=0.8cm, right=0.8cm, top=0.8cm, bottom=0.8cm]{geometry} % okraje stranky

\usepackage{fontspec}
\setmainfont[FeatureFile={junicode.fea}, Ligatures={Common, TeX}, RawFeature=+fixi]{Junicode}
%\setmainfont{Junicode}

% shortcut for Junicode without ligatures (for the Czech texts)
\newfontfamily\nlfont[FeatureFile={junicode.fea}, Ligatures={Common, TeX}, RawFeature=+fixi]{Junicode}

\usepackage{multicol}
\usepackage{color}
\usepackage{lettrine}
\usepackage{fancyhdr}

% usual packages loading:
\usepackage{luatextra}
\usepackage{graphicx} % support the \includegraphics command and options
\usepackage{gregoriotex} % for gregorio score inclusion
\usepackage{gregoriosyms}
\usepackage{wrapfig} % figures wrapped by the text
\usepackage{parcolumns}
\usepackage[contents={},opacity=1,scale=1,color=black]{background}
\usepackage{tikzpagenodes}
\usepackage{calc}
\usepackage{longtable}

\setlength{\headheight}{12pt}

\input{conventuscommune.tex} % Often used macros
%%%% Preklady jednotlivych zpevu (nektere se opakuji, a je dobre mit je
% vsechny na jedne hromade)

% HOURS ---

\newcommand{\trAntI}{\translatioCantus{Muž boží měl kožený toulec, pečlivě
zavázaný, jenž mu visel na šíji a~často se ho dotýkal.}}

\newcommand{\trAntII}{\translatioCantus{Klíč od~něho tak dobře střežil, že
dokud žil v~těle, nikdo z~jeho žáků nezvěděl, co je uvnitř.}}

\newcommand{\trAntIII}{\translatioCantus{Ale když se odebral z~tohoto
života, schránku otevřeli a~objevili v~ní žíněné roucho a~měděný řetěz
potřísněný krví.}}

\newcommand{\trAntIV}{\translatioCantus{A když prohlédli mistrovo tělo,
nalezli jeho tělo na čtyřech místech hluboce zbrázděno ranami od řetězu.}}

\newcommand{\trAntV}{\translatioCantus{Krev vytékající z~těch ran, místy
prostoupila i~žíněným rouchem.}}

\newcommand{\trCapituli}{\translatioCantus{
Miláčkovi Boha a~lidí,
Mojžíšovi požehnané paměti,~\gredagger{}
dopřál slávu rovnou slávě svatých~\grestar{}
učinil ho mocným na postrach nepřátelům
a~jeho slovy zastavil divy.}}

\newcommand{\trLectioBrevis}{\translatioCantus{
Pamatujte na své představené,
kteří vám hlásali Boží slovo.
Uvažte, jak oni skončili život, a~napodobujte jejich víru.
Ježíš Kristus je stejný včera i~dnes i~navěky.
Nenechte se svést věelijakými cizími naukami.}}

\newcommand{\trRespLaud}{\translatioCantus{Spravedlivého vodil Hospodin~\grestar{}
po přímých stezkách. \Vbardot{} A~ukázal mu Boží království.}}

\newcommand{\trRespLaudB}{\translatioCantus{Na tvých hradbách, Jeruzaléme,
ustanovil jsem strážné;~\grestar{}
budou bdít nad mým lidem. \Vbardot{} Ani ve dne, ani v~noci nesmějí nikdy
mlčet.}}

\newcommand{\trVersus}{\translatioCantus{\Vbardot{} Ústa spravedlivého šeptají moudrost, aleluja.
\Rbardot{} A~jeho jazyk ohlašuje právo, aleluja.}}

\newcommand{\trAntBenedictus}{\translatioCantus{Když na bujné oře vložili
nosítka a~sňali jim uzdu, vydali se přímo k~cele božího muže.}}

\newcommand{\trPreces}{\translatioCantus{
\noindent S vděčností chvalme Krista, dobrého Pastýře, \gredagger{} který dal život za své ovce, \grestar{} a~pokorně ho prosme: \Rbardot{} Pane, buď pastýřem svého lidu.

\noindent Kriste, ty dáváš církvi pastýře, a~jejich službou se ujímáš svého lidu, \grestar{} dej, ať v~lásce těch, kteří nás vedou, poznáváme, jak nás miluješ. \Rbardot{} Pane, buď pastýřem svého lidu.

\noindent Ty stále konáš skrze své zástupce službu pastýře a~učitele, \grestar{} nepřestávej nás nikdy vést prostřednictvím svých služebníků. \Rbardot{} Pane, buď pastýřem svého lidu.

\noindent Ty prokazuješ svému lidu skrze jeho pastýře službu lékaře duše i~těla, \grestar{} ochraňuj náš život a~veď nás ke svatosti. \Rbardot{} Pane, buď pastýřem svého lidu.

\noindent Ty posíláš své svaté, aby slovem i~příkladem vedli tvůj lid k~tobě, \grestar{} na jejich přímluvu nás posiluj, abychom vytrvali na cestě, která vede k~věčnému životu. \Rbardot{} Pane, buď pastýřem svého lidu.}}

\newcommand{\trOrationis}{\translatioCantus{Bože, jenž nám dopřáváš radovat
se z~výroční slavnosti svatého tvého vyznavače Havla, uděl dobrotivě,
abychom když slavíme jeho narození, též se řídili podobou jeho skutků.
Skrze…}}
 % Czech translations of the proper texts

\setlength{\columnsep}{15pt} % prostor mezi sloupci

%%%%%%%%%%%%%%%%%%%%%%%%%%%%%%%%%%%%%%%%%%%%%%%%%%%%%%%%%%%%%%%%%%%%%%%%%%%%%%%%%%%%%%%%%%%%%%%%%%%%%%%%%%%%%
\begin{document}

% Here we set the space around the initial.
% Please report to http://home.gna.org/gregorio/gregoriotex/details for more details and options
\grechangedim{afterinitialshift}{2.2mm}{scalable}
\grechangedim{beforeinitialshift}{2.2mm}{scalable}
\grechangedim{interwordspacetext}{0.20 cm plus 0.15 cm minus 0.05 cm}{scalable}%
\grechangedim{annotationraise}{-0.2cm}{scalable}

% Here we set the initial font. Change 38 if you want a bigger initial.
% Emit the initials in red.
\grechangestyle{initial}{\color{red}\fontsize{38}{38}\selectfont}

\renewcommand{\headrulewidth}{0pt} % no horiz. rule at the header
\pagestyle{empty}

\grechangedim{spaceabovelines}{0.2cm}{scalable}%

\begin{titulusOfficii}
\nomenFesti{In Dominicis Adventus.}
\textbf{Ad Vesperas}
\end{titulusOfficii}

\vfill

\cantusSineNeumas

\pars{Antiphona} \scriptura{\Abardot{} Is. 45, 8; \Vbardot{} Cf. ibid. 64,
9-11; 64, 5-7; 16, 1; 40, 1; 41, 4}

\vspace{-0.3cm}

{
\grechangedim{interwordspacetext}{0.20 cm plus 0.15 cm minus 0.05
cm}{scalable}%
\antiphona{I}{temporalia/ant-rorate-ant.gtex}
\grechangedim{interwordspacetext}{0.32 cm plus 0.15 cm minus 0.05
cm}{scalable}%
}

\begin{translatioMulticol}{2}
Ne irascáris Dómine,\\
ne ultra memíneris iniquitátis:\\
ecce cívitas Sancti facta est desérta:\\
Sion desérta facta est: Jerúsalem desoláta est:\\
domus sanctificatiónis tuæ et glóriæ tuæ,\\
ubi laudavérunt te patres nostri.\\
\\
Peccávimus, et facti sumus tamquam immúndus nos,\\
et cecídimus quasi fólium univérsi:\\
et iniquitátes nostræ quasi ventus abstulérunt nos:\\
abscondísti fáciem tuam a nobis,\\
et allisísti nos in manu iniquitátis nostræ.\columnbreak

Vide Dómine, afflictiónem pópuli tui,\\
et mitte quem missúrus es:\\
emítte Agnum dominatórem terræ,\\
de Petra desérti ad montem fíliæ Sion:\\
ut áuferat ipse jugum captivitátis nostræ.\\
\\
Consolámini, consolámini, pópule meus:\\
cito véniet salus tua:\\
quare mœróre consúmeris,\\
quia innovávit te dolor?\\
Salvábo te, noli timére,\\
ego enim sum Dóminus Deus tuus,\\
Sanctus Israel, Redémptor tuus.
\end{translatioMulticol}


\vfill

\begin{translatioMulticol}{2}
Dej rosu, nebe nad námi,\\
ať z oblak skane spása.\\
\\
Odlož hněv svůj, ó Pane náš\\
a zapomeň už na naše nepravosti.\\
Hle, tvé svaté město je pouští,\\
opuštěný je Sión, Jeruzalém je liduprázdný,\\
to místo tobě zasvěcené, dům tvé slávy,\\
kde k chvále tvé zpívali otcové naši.\\
\\
Pro hříchy své stali jsme se lidem nečistým\\
a odpadli jsme jako zvadlé listí.\\
Jako vichr nás uchvátily naše viny,\\
když jsi před námi ukryl svou tvář\\
a vydal nás napospas nepravosti naši.\columnbreak

Pohlédni, Pane, na ponížení lidu svého,\\
ať příjde ten, jenž přijít má.\\
Pošli Beránka, ať vládne zemi,\\
od Skály na poušti až k hoře siónské dcery.\\
Ať on sám sejme jho poroby naší.\\
\\
Přijmi útěchu, přijmi útěchu, můj lide drahý,\\
neboť blízko je tvoje spása.\\
Proč se stále trápíš v úzkostech,\\
proč tě bolest svírá?\\
Zachráním tě, neboj se, doufej!\\
Vždyť já to jsem, já Hospodin, Pán a Bůh tvůj,\\
izraelův Svatý a Spása tvoje.
\end{translatioMulticol}


\vfill
	
\pars{Supplicatio Litaniæ.}

\vspace{0.1cm}

\cuminitiali{}{temporalia/supplicatiolitaniae.gtex}

%\vspace{0.2cm}
\vfill
\pagebreak

\pars{Oratio Dominica.}

\vspace{0.1cm}

\cuminitiali{}{temporalia/oratiodominica.gtex}

\vfill

\rubrica{Deinde dicitur ab Hebdomadario:}

\vspace{0.1cm}

\cuminitiali{}{temporalia/dominusvobiscum-solemnis.gtex}

\vfill
%\pagebreak

\pars{Antiphona finalis B. M. V.}

\antiphona{V}{temporalia/an_alma_redemptoris_mater-simplex.gtex}

\trAlmaRedemptoris

\end{document}
