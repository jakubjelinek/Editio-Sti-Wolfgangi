% LuaLaTeX

\documentclass[a4paper, twoside, 12pt]{article}
\usepackage[latin]{babel}
%\usepackage[landscape, left=3cm, right=1.5cm, top=2cm, bottom=1cm]{geometry} % okraje stranky
\usepackage[portrait, a4paper, mag=1300, truedimen, left=0.8cm, right=0.8cm, top=0.8cm, bottom=0.8cm]{geometry} % okraje stranky

\usepackage{fontspec}
\setmainfont[FeatureFile={junicode.fea}, Ligatures={Common, TeX}, RawFeature=+fixi]{Junicode}
%\setmainfont{Junicode}

% shortcut for Junicode without ligatures (for the Czech texts)
\newfontfamily\nlfont[FeatureFile={junicode.fea}, Ligatures={Common, TeX}, RawFeature=+fixi]{Junicode}

\usepackage{multicol}
\usepackage{color}
\usepackage{lettrine}
\usepackage{fancyhdr}

% usual packages loading:
\usepackage{luatextra}
\usepackage{graphicx} % support the \includegraphics command and options
\usepackage{gregoriotex} % for gregorio score inclusion
\usepackage{gregoriosyms}
\usepackage{wrapfig} % figures wrapped by the text
\usepackage{parcolumns}
\usepackage[contents={},opacity=1,scale=1,color=black]{background}
\usepackage{tikzpagenodes}
\usepackage{calc}
\usepackage{longtable}

\setlength{\headheight}{12pt}

\input{conventuscommune.tex} % Often used macros
%%%% Preklady jednotlivych zpevu (nektere se opakuji, a je dobre mit je
% vsechny na jedne hromade)

% HOURS ---

\newcommand{\trAntI}{\translatioCantus{Muž boží měl kožený toulec, pečlivě
zavázaný, jenž mu visel na šíji a~často se ho dotýkal.}}

\newcommand{\trAntII}{\translatioCantus{Klíč od~něho tak dobře střežil, že
dokud žil v~těle, nikdo z~jeho žáků nezvěděl, co je uvnitř.}}

\newcommand{\trAntIII}{\translatioCantus{Ale když se odebral z~tohoto
života, schránku otevřeli a~objevili v~ní žíněné roucho a~měděný řetěz
potřísněný krví.}}

\newcommand{\trAntIV}{\translatioCantus{A když prohlédli mistrovo tělo,
nalezli jeho tělo na čtyřech místech hluboce zbrázděno ranami od řetězu.}}

\newcommand{\trAntV}{\translatioCantus{Krev vytékající z~těch ran, místy
prostoupila i~žíněným rouchem.}}

\newcommand{\trCapituli}{\translatioCantus{
Miláčkovi Boha a~lidí,
Mojžíšovi požehnané paměti,~\gredagger{}
dopřál slávu rovnou slávě svatých~\grestar{}
učinil ho mocným na postrach nepřátelům
a~jeho slovy zastavil divy.}}

\newcommand{\trLectioBrevis}{\translatioCantus{
Pamatujte na své představené,
kteří vám hlásali Boží slovo.
Uvažte, jak oni skončili život, a~napodobujte jejich víru.
Ježíš Kristus je stejný včera i~dnes i~navěky.
Nenechte se svést věelijakými cizími naukami.}}

\newcommand{\trRespLaud}{\translatioCantus{Spravedlivého vodil Hospodin~\grestar{}
po přímých stezkách. \Vbardot{} A~ukázal mu Boží království.}}

\newcommand{\trRespLaudB}{\translatioCantus{Na tvých hradbách, Jeruzaléme,
ustanovil jsem strážné;~\grestar{}
budou bdít nad mým lidem. \Vbardot{} Ani ve dne, ani v~noci nesmějí nikdy
mlčet.}}

\newcommand{\trVersus}{\translatioCantus{\Vbardot{} Ústa spravedlivého šeptají moudrost, aleluja.
\Rbardot{} A~jeho jazyk ohlašuje právo, aleluja.}}

\newcommand{\trAntBenedictus}{\translatioCantus{Když na bujné oře vložili
nosítka a~sňali jim uzdu, vydali se přímo k~cele božího muže.}}

\newcommand{\trPreces}{\translatioCantus{
\noindent S vděčností chvalme Krista, dobrého Pastýře, \gredagger{} který dal život za své ovce, \grestar{} a~pokorně ho prosme: \Rbardot{} Pane, buď pastýřem svého lidu.

\noindent Kriste, ty dáváš církvi pastýře, a~jejich službou se ujímáš svého lidu, \grestar{} dej, ať v~lásce těch, kteří nás vedou, poznáváme, jak nás miluješ. \Rbardot{} Pane, buď pastýřem svého lidu.

\noindent Ty stále konáš skrze své zástupce službu pastýře a~učitele, \grestar{} nepřestávej nás nikdy vést prostřednictvím svých služebníků. \Rbardot{} Pane, buď pastýřem svého lidu.

\noindent Ty prokazuješ svému lidu skrze jeho pastýře službu lékaře duše i~těla, \grestar{} ochraňuj náš život a~veď nás ke svatosti. \Rbardot{} Pane, buď pastýřem svého lidu.

\noindent Ty posíláš své svaté, aby slovem i~příkladem vedli tvůj lid k~tobě, \grestar{} na jejich přímluvu nás posiluj, abychom vytrvali na cestě, která vede k~věčnému životu. \Rbardot{} Pane, buď pastýřem svého lidu.}}

\newcommand{\trOrationis}{\translatioCantus{Bože, jenž nám dopřáváš radovat
se z~výroční slavnosti svatého tvého vyznavače Havla, uděl dobrotivě,
abychom když slavíme jeho narození, též se řídili podobou jeho skutků.
Skrze…}}
 % Czech translations of the proper texts

%%%% Vicekrat opakovane kousky

\newcommand{\anteOrationem}{
  \rubrica{Ante Orationem, cantatur a Superiore:}

  \pars{Supplicatio Litaniæ.}

  \cuminitiali{}{temporalia/supplicatiolitaniae.gtex}

  \vspace{5mm}

  \pars{Oratio Dominica.}

  \cuminitiali{}{temporalia/oratiodominica.gtex}

  \vspace{5mm}

  \rubrica{Deinde dicitur ab Hebdomadario:}

  \cuminitiali{}{temporalia/dominusvobiscum-solemnis.gtex}

  \rubrica{In choro monialium loco Dominus vobiscum dicitur:}

  \sineinitiali{temporalia/domineexaudi.gtex}
}

\setlength{\columnsep}{15pt} % prostor mezi sloupci

%%%%%%%%%%%%%%%%%%%%%%%%%%%%%%%%%%%%%%%%%%%%%%%%%%%%%%%%%%%%%%%%%%%%%%%%%%%%%%%%%%%%%%%%%%%%%%%%%%%%%%%%%%%%%
\begin{document}

% Here we set the space around the initial.
% Please report to http://home.gna.org/gregorio/gregoriotex/details for more details and options
\grechangedim{afterinitialshift}{2.2mm}{scalable}
\grechangedim{beforeinitialshift}{2.2mm}{scalable}
\grechangedim{interwordspacetext}{0.20 cm plus 0.15 cm minus 0.05 cm}{scalable}%
\grechangedim{annotationraise}{-0.2cm}{scalable}

% Here we set the initial font. Change 38 if you want a bigger initial.
% Emit the initials in red.
\grechangestyle{initial}{\color{red}\fontsize{38}{38}\selectfont}

\renewcommand{\headrulewidth}{0pt} % no horiz. rule at the header
\pagestyle{empty}

\grechangedim{spaceabovelines}{0.2cm}{scalable}%

\begin{titulusOfficii}
\nomenFesti{Dominica I Adventus.}
\celebratio{I Classis. Semiduplex.}
\textbf{Ad Vesperas}
\end{titulusOfficii}

% Psalmi festivi (AM33, pg. 721):
% 66 // 92, 99, 62, Dan3, 148+149+150

\cantusSineNeumas

{
\grechangedim{interwordspacetext}{0.16 cm plus 0.15 cm minus 0.05 cm}{scalable}%
\cuminitiali{}{temporalia/deusinadiutorium-alter.gtex}
\grechangedim{interwordspacetext}{0.20 cm plus 0.15 cm minus 0.05 cm}{scalable}%
}

\vspace{0.3cm}

\cantusSineNeumas

\pars{Psalmus 1.} \scriptura{Ioel 3, 18; \textbf{H18}}

\vspace{-0.5cm}

{
\grechangedim{interwordspacetext}{0.10 cm plus 0.15 cm minus 0.05cm}{scalable}%
\antiphona{VIII G}{temporalia/antI1.gtex}
\grechangedim{interwordspacetext}{0.20 cm plus 0.15 cm minus 0.05cm}{scalable}%
}

\trAntIDI

\vspace{-0.2cm}

\scriptura{Ps. 109}

\initiumpsalmi{temporalia/ps109-initium-viii-G-auto.gtex}

\vspace{-0.4cm}

\psalmusEtTranslatioT{temporalia/ps109-I-comb.tex}{6.5cm}

\vfill
\pagebreak

\pars{Psalmus 2.} \scriptura{\textbf{H18}}

\vspace{-0.5cm}

\antiphona{VIII G\textsuperscript{2}}{temporalia/antI2.gtex}

\trAntIDII

\scriptura{Ps. 110}

\initiumpsalmi{temporalia/ps110-initium-viii-G2-auto.gtex}

\psalmusEtTranslatioT{temporalia/ps110-I-comb.tex}{6.5cm}

\vfill
\pagebreak

\pars{Psalmus 3.} \scriptura{Cf. Zach. 14, 5-6; \textbf{H18}}

\vspace{-0.5cm}

\antiphona{V a}{temporalia/antI3.gtex}

\trAntIDIII

\scriptura{Ps. 111}

\initiumpsalmi{temporalia/ps111-initium-v-a-auto.gtex}

\psalmusEtTranslatioT{temporalia/ps111-I-comb.tex}{6.5cm}

\vfill
\pagebreak

\pars{Psalmus 4.} \scriptura{Is. 55, 1.6; \textbf{H18}}

\vspace{-0.5cm}

\antiphona{VII c}{temporalia/antI4.gtex}

\trAntIDIV

\scriptura{Ps. 112}

\initiumpsalmi{temporalia/ps112-initium-vii-c-auto.gtex}

\psalmusEtTranslatioT{temporalia/ps112-I-comb.tex}{6.5cm}

\vfill
\pagebreak

\pars{Psalmus 5.} \scriptura{\textbf{H18}}

\vspace{-0.5cm}

\antiphona{IV A\textsuperscript{*}}{temporalia/antI5.gtex}

\trAntIDV

\scriptura{Ps. 113}

\initiumpsalmi{temporalia/ps113-initium-iv-A-auto.gtex}

\psalmusEtTranslatioT{temporalia/ps113-I-comb.tex}{6.5cm}

\antiphona{}{temporalia/antI5.gtex} % repeat the antiphon - new page

\vfill
\pagebreak

\pars{Capitulum.} \scriptura{Rom. 13, 11}

\cuminitiali{}{temporalia/capitulum-FratresHora.gtex}

% preklad Jeruz. bible
\trCapituliI

\vfill

\pars{Responsorium breve.} \scriptura{Ps. 84, 8; \textbf{H20}}

\cuminitiali{IV}{temporalia/resp-vesp.gtex}

\trRespVesp

\vfill
\pagebreak

% Hymnus. %%%
\pars{Hymnus.}

\cuminitiali{IV}{temporalia/hym-ConditorAlme.gtex}

\input{../AdventusDominicaII/cantus/amon33/hym-ConditorAlme-bohtext.tex}

\vfill

\pars{Versus.} \scriptura{Is. 45, 8}

% Versus. %%%
\sineinitiali{temporalia/versus-rorate.gtex}
\noindent \trVersusVesp

\vfill
\pagebreak

\cantusCumNeumis

\pars{Canticum B. Mariæ V.} \scriptura{Lc. 1, 30-31; \textbf{H19}}

\vspace{-0.6cm}

\antiphona{VIII G}{temporalia/antI-magn-vesp2.gtex}

\trAntIDMagnificat

\scriptura{Lc. 1, 46-55}

\cantusSineNeumas
\initiumpsalmi{temporalia/magnificat-initium-viii-G.gtex}

\vspace{-1cm}

\psalmusEtTranslatioT{temporalia/magnificat-I-comb.tex}{6.5cm}

\vspace{-1cm}

\cantusSineNeumas

\anteOrationem

\vspace{5mm}

% Oratio. %%%
\pars{Oratio.}

\cuminitiali{}{temporalia/oratioI.gtex}
\trOrationisI

\vspace{1cm}
\rubrica{Hebdomadarius dicit iterum Dominus vobiscum. Postea cantatur a cantore:}
\vspace{2mm}

\cuminitiali{IV}{temporalia/benedicamus-dominica-advequad.gtex}

\vfill

\end{document}
