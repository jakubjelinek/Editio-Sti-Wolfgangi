\newcommand{\titulus}{\nomenFesti{Die V infra Octavam Sanctissimi Cordis Iesu.}
\celebratio{Semiduplex.}}
\newcommand{\aequus}{Feria III}
\newcommand{\lectioi}{\pars{Lectio I.} \scriptura{1 Reg. 12, 1-5}

\noindent De libro primo Regum.

\noindent Dixit autem Sámuel ad univérsum Israël: Ecce audívi vocem vestram iuxta ómnia quæ locúti estis ad me, et constítui super vos regem. Et nunc rex gráditur ante vos: ego autem sénui, et incánui: porro fílii mei vobíscum sunt: ítaque conversátus coram vobis ab adolescéntia mea usque ad hanc diem, ecce præsto sum. Loquímini de me coram Dómino, et coram christo eius, utrum bovem cuiúsquam túlerim, aut ásinum: si quémpiam calumniátus sum, si oppréssi áliquem, si de manu cuiúsquam munus accépi: et contémnam illud hódie, restituámque vobis. Et dixérunt: Non es calumniátus nos, neque oppressísti, neque tulísti de manu alicúius quíppiam. Dixítque ad eos: Testis est Dóminus advérsum vos, et testis Christus eius in die hac, quia non invenéritis in manu mea quíppiam. Et dixérunt: Testis.}
\newcommand{\lectioii}{\pars{Lectio II.} \scriptura{1 Reg. 12, 6-9}

\noindent Et ait Sámuel ad pópulum: Dóminus, qui fecit Móysen et Aaron, et edúxit patres nostros de terra Ægýpti. Nunc ergo state, ut iudício conténdam advérsum vos coram Dómino de ómnibus misericórdiis Dómini quas fecit vobíscum et cum pátribus vestris: quómodo Iacob ingréssus est in Ægýptum, et clamavérunt patres vestri ad Dóminum: et misit Dóminus Móysen et Aaron, et edúxit patres vestros de Ægýpto, et collocávit eos in loco hoc. Qui oblíti sunt Dómini Dei sui, et trádidit eos in manu Sísaræ magístri milítiæ Hasor, et in manu Philisthinórum, et in manu regis Moab: et pugnavérunt advérsum eos.}
\newcommand{\lectioiii}{\pars{Lectio III.} \scriptura{1 Reg. 12, 10-14}

\noindent Póstea autem clamavérunt ad Dóminum, et dixérunt: Peccávimus, quia derelíquimus Dóminum, et servívimus Báalim et Astaroth: nunc ergo érue nos de manu inimicórum nostrórum, et serviémus tibi. Et misit Dóminus Ieróbaal, et Badan, et Iephte, et Sámuel, et éruit vos de manu inimicórum vestrórum per circúitum, et habitástis confidénter. Vidéntes autem quod Naas rex filiórum Ammon venísset advérsum vos, dixístis mihi: Nequáquam, sed rex imperábit nobis: cum Dóminus Deus vester regnáret in vobis. Nunc ergo præsto est rex vester, quem elegístis et petístis: ecce dedit vobis Dóminus regem. Si timuéritis Dóminum, et serviéritis ei, et audiéritis vocem eius, et non exasperavéritis os Dómini, éritis et vos, et rex qui ímperat vobis, sequéntes Dóminum Deum vestrum.}
\newcommand{\lectioiv}{\pars{Lectio IV.} \scriptura{Encýclica Miserentíssimus Redémptor}

\noindent Ex lítteris Encýclicis Pii Papæ undécimi.

\noindent Quo autem perféctius oblátio nostra nostrúmque sacrifícium sacrifício domínico respónderit, idest amórem nostri cupiditatésque nostras immolavérimus et carnem crucifixérimus crucifixióne ea mýstica, de qua lóquitur Apóstolus, eo uberióres propitiatiónis atque expiatiónis pro nobis aliísque percipiémus fructus. Mirífica enim viget fidélium ómnium cum Christo necessitúdo, qualis inter caput et cétera córporis membra intercédit, itémque arcána illa, quam fide cathólica profitémur, Sanctórum communióne, cum sínguli hómines tum pópuli non modo coniungúntur inter se, sed étiam cum eódem qui est caput Christus, ex quo totum corpus compáctum et connéxum per omnem iunctúram subministratiónis, secúndum operatiónem in mensúram uniuscuiúsque membri augméntum córporis sui facit in ædificatiónem sui in caritáte. Quod quidem Mediátor ipse Dei et hóminum Christus Iesus, morti próximus, a Patre postulárat: Ego in eis et tu in me ut sint consummáti in unum.}
\newcommand{\lectiov}{\pars{Lectio V.}

\noindent Quemádmodum ígitur uniónem cum Christo profitétur ac firmat consecrátio, ita expiátio eámdem uniónem, et culpas detergéndo ínchoat, et Christi passiónes participándo pérficit, et víctimas pro frátribus offeréndo consúmmat. Atque id sane miseréntis Iesu consílium fuit, cum Cor nobis suum, insígnia passiónis prǽferens ac flammas amóris osténtans, patére vóluit, scílicet ut hinc infinítam peccáti malítiam coniectántes, illinc Reparatiónis caritátem infinítam admiráti, et peccátum veheméntius detestarémur et caritátis ardéntius vicem redderémus. Et vere expiatiónis potíssimum seu reparatiónis spíritus primas semper potiorésque partes hábuit in cultu Sacratíssimo Cordi Iesu exhibéndo, nihílque eo congruéntius orígini, índoli, virtúti, indústriis quæ huic religiónis formæ sunt própriæ, ut rerum memória et usus, sacra item litúrgia atque Summórum Pontíficum acta confírmant.}
\newcommand{\lectiovi}{\pars{Lectio VI.}

\noindent Síquidem cum se conspiciéndum Margarítæ Maríæ exhibéret Christus, caritátis suæ infinitátem prǽdicans, simul, mæréntis instar, tot tantásque sibi inústas ab ingrátis homínibus iniúrias in hæc verba conquéstus est, quæ útinam in piórum ánimis insidérent nulláque unquam oblivióne deleréntur: « En Cor illud, inquit, quod tantópere hómines amávit beneficiísque ómnibus cumulávit, quodque amóri suo infiníto non tantum rédditam grátiam nullam invénit, at contra, obliviónem, negléctum, contumélias, eásque ab iis étiam illátas nonnúnquam, qui amóris peculiáris débito officióque teneréntur ».}
\newcommand{\lectiovii}{\pars{Lectio VII.} \scriptura{Io. 19, 31-37}

\noindent Léctio sancti Evangélii secúndum Ioánnem.

\noindent In illo témpore: Iudǽi, quóniam parascéve erat, ut non remanérent in cruce córpora sábbato (erat enim magnus dies ille sábbati), rogavérunt Pilátum, ut frangeréntur eórum crura, et tolleréntur. Et réliqua.

\vspace{4mm}

\noindent Homilia sancti Bernardíni Senénsis.

\noindent Ioánnes subdit: Unus mílitum láncea latus eius apéruit et contínuo exívit sanguis et aqua. O amor qui ómnia liquas! quómodo pro redemptióne nostra reliquísti dilectórem nostrum? Nam, ut úndique inundáret amóris dilúvium, super nos ruptæ sunt abýssi magnæ; scílicet, penetrália Cordis Iesu, quibus, ad íntima progrédiens, dira láncea non pepércit. Sanguis exívit et aqua. Sanguis in redemptiónem, sed étiam in ablutiónem aqua deflúxit; unde formáta est Ecclésia ex látere Christi, ut ætérne únicam atque diléctam a Christo se discat, et ut recognóscat quam displícuit culpa pro qua sanguis divínus ex hómine Deo vivo et mórtuo ita deflúxit. Non enim parva quantitáte constámus, si pro nobis sanguis divínus effúnditur.}
\newcommand{\lectioviii}{\pars{Lectio VIII.}

\noindent Aqua ad lítteram non cum sánguine indistíncta deflúxit. Neque enim potuísset ab insipiéntibus comprehéndi, si mixta cum sánguine defluxísset. Et forte totus sanguis deflúxit ex illo divíno córpore in signum totíus amóris effúsi, post quem humor áqueus egréssus est. Quod quidem est alto mystério factum, ut prius egrederétur ex eódem córpore rédimens prétium, deínde aqua in qua multitúdo populórum redémpta significátur. Sunt enim aquæ multæ, pópuli multi; tamen qui ad christiánam fidem pértinent unus fidélis pópulus sunt, ut non sint aquæ, sed aqua quæ manávit ex látere Christi, sicut prima Corinthiórum cápite décimo Apóstolus ait: Unus panis, et unum corpus multi sumus omnes, qui de uno pane et de uno cálice participámus. Et íterum ad Ephésios, cápite quarto, inquit: Unus Deus, una fides, unum baptísma.}
\newcommand{\lectioix}{\pars{Lectio IX.}

\noindent Notánter tamen adverténdum est quod latus Christi apértum dícitur, non vulnerátum; quóniam próprie vulnus prætérquam in vivo córpore fíeri nequit. Ait enim Evangelísta Ioánnes: Unus mílitum láncea latus eius apéruit; ut, apérto látere, cognoscámus dilectiónem Cordis sui usque ad mortem, et ad illum ineffábilem amórem eius ingrediámur quo ille ad nos procéssit. Accedámus ergo ad Cor eius, Cor altum, Cor secrétum, Cor ómnia cógitans, Cor ómnia sciens, Cor díligens, immo amóre ardens; et apértam portam intelligámus saltem in amóris veheméntia; cordifórmes ingrediámur ad secrétum ab ætérno abscónditum, nunc vero in morte quasi apérto látere revelátum; quóniam apértio láteris ætérni templi apertiónem demónstrat, ubi ómnium exsisténtium consummáta est felícitas ætérna.}
% LuaLaTeX

\documentclass[a4paper, twoside, 12pt]{article}
\usepackage[latin]{babel}
%\usepackage[landscape, left=3cm, right=1.5cm, top=2cm, bottom=1cm]{geometry} % okraje stranky
%\usepackage[landscape, a4paper, mag=1166, truedimen, left=2cm, right=1.5cm, top=1.6cm, bottom=0.95cm]{geometry} % okraje stranky
\usepackage[landscape, a4paper, mag=1400, truedimen, left=0.5cm, right=0.5cm, top=0.5cm, bottom=0.5cm]{geometry} % okraje stranky

\usepackage{fontspec}
\setmainfont[FeatureFile={junicode.fea}, Ligatures={Common, TeX}, RawFeature=+fixi]{Junicode}
%\setmainfont{Junicode}

% shortcut for Junicode without ligatures (for the Czech texts)
\newfontfamily\nlfont[FeatureFile={junicode.fea}, Ligatures={Common, TeX}, RawFeature=+fixi]{Junicode}

\usepackage{multicol}
\usepackage{color}
\usepackage{lettrine}
\usepackage{fancyhdr}

% usual packages loading:
\usepackage{luatextra}
\usepackage{graphicx} % support the \includegraphics command and options
\usepackage{gregoriotex} % for gregorio score inclusion
\usepackage{gregoriosyms}
\usepackage{wrapfig} % figures wrapped by the text
\usepackage{parcolumns}
\usepackage[contents={},opacity=1,scale=1,color=black]{background}
\usepackage{tikzpagenodes}
\usepackage{calc}
\usepackage{longtable}
\usetikzlibrary{calc}

\setlength{\headheight}{14.5pt}

\input{conventuscommune.tex} % Often used macros
%%%% Preklady jednotlivych zpevu (nektere se opakuji, a je dobre mit je
% vsechny na jedne hromade)

% HOURS ---

\newcommand{\trAntI}{\translatioCantus{Muž boží měl kožený toulec, pečlivě
zavázaný, jenž mu visel na šíji a~často se ho dotýkal.}}

\newcommand{\trAntII}{\translatioCantus{Klíč od~něho tak dobře střežil, že
dokud žil v~těle, nikdo z~jeho žáků nezvěděl, co je uvnitř.}}

\newcommand{\trAntIII}{\translatioCantus{Ale když se odebral z~tohoto
života, schránku otevřeli a~objevili v~ní žíněné roucho a~měděný řetěz
potřísněný krví.}}

\newcommand{\trAntIV}{\translatioCantus{A když prohlédli mistrovo tělo,
nalezli jeho tělo na čtyřech místech hluboce zbrázděno ranami od řetězu.}}

\newcommand{\trAntV}{\translatioCantus{Krev vytékající z~těch ran, místy
prostoupila i~žíněným rouchem.}}

\newcommand{\trCapituli}{\translatioCantus{
Miláčkovi Boha a~lidí,
Mojžíšovi požehnané paměti,~\gredagger{}
dopřál slávu rovnou slávě svatých~\grestar{}
učinil ho mocným na postrach nepřátelům
a~jeho slovy zastavil divy.}}

\newcommand{\trLectioBrevis}{\translatioCantus{
Pamatujte na své představené,
kteří vám hlásali Boží slovo.
Uvažte, jak oni skončili život, a~napodobujte jejich víru.
Ježíš Kristus je stejný včera i~dnes i~navěky.
Nenechte se svést věelijakými cizími naukami.}}

\newcommand{\trRespLaud}{\translatioCantus{Spravedlivého vodil Hospodin~\grestar{}
po přímých stezkách. \Vbardot{} A~ukázal mu Boží království.}}

\newcommand{\trRespLaudB}{\translatioCantus{Na tvých hradbách, Jeruzaléme,
ustanovil jsem strážné;~\grestar{}
budou bdít nad mým lidem. \Vbardot{} Ani ve dne, ani v~noci nesmějí nikdy
mlčet.}}

\newcommand{\trVersus}{\translatioCantus{\Vbardot{} Ústa spravedlivého šeptají moudrost, aleluja.
\Rbardot{} A~jeho jazyk ohlašuje právo, aleluja.}}

\newcommand{\trAntBenedictus}{\translatioCantus{Když na bujné oře vložili
nosítka a~sňali jim uzdu, vydali se přímo k~cele božího muže.}}

\newcommand{\trPreces}{\translatioCantus{
\noindent S vděčností chvalme Krista, dobrého Pastýře, \gredagger{} který dal život za své ovce, \grestar{} a~pokorně ho prosme: \Rbardot{} Pane, buď pastýřem svého lidu.

\noindent Kriste, ty dáváš církvi pastýře, a~jejich službou se ujímáš svého lidu, \grestar{} dej, ať v~lásce těch, kteří nás vedou, poznáváme, jak nás miluješ. \Rbardot{} Pane, buď pastýřem svého lidu.

\noindent Ty stále konáš skrze své zástupce službu pastýře a~učitele, \grestar{} nepřestávej nás nikdy vést prostřednictvím svých služebníků. \Rbardot{} Pane, buď pastýřem svého lidu.

\noindent Ty prokazuješ svému lidu skrze jeho pastýře službu lékaře duše i~těla, \grestar{} ochraňuj náš život a~veď nás ke svatosti. \Rbardot{} Pane, buď pastýřem svého lidu.

\noindent Ty posíláš své svaté, aby slovem i~příkladem vedli tvůj lid k~tobě, \grestar{} na jejich přímluvu nás posiluj, abychom vytrvali na cestě, která vede k~věčnému životu. \Rbardot{} Pane, buď pastýřem svého lidu.}}

\newcommand{\trOrationis}{\translatioCantus{Bože, jenž nám dopřáváš radovat
se z~výroční slavnosti svatého tvého vyznavače Havla, uděl dobrotivě,
abychom když slavíme jeho narození, též se řídili podobou jeho skutků.
Skrze…}}
 % Czech translations of the proper texts

\newcommand{\annusEditionis}{2020}

%%%% Vicekrat opakovane kousky

\newcommand{\anteOrationem}{
  \rubrica{Ante Orationem, cantatur a Superiore:}

  \pars{Supplicatio Litaniæ.}

  \cuminitiali{}{temporalia/supplicatiolitaniae.gtex}

  \pars{Oratio Dominica.}

  \cuminitiali{}{temporalia/oratiodominica.gtex}

  \rubrica{Deinde dicitur ab Hebdomadario:}

  \cuminitiali{}{temporalia/dominusvobiscum-solemnis.gtex}

  \rubrica{In choro monialium loco Dominus vobiscum dicitur:}

  \sineinitiali{temporalia/domineexaudi.gtex}
}

\ifx\dominica\undefined
\newcommand{\capitulumLaudes}{\pars{Capitulum.} \scriptura{Eph. 3, 14-17}

\grechangedim{interwordspacetext}{0.12 cm plus 0.15 cm minus 0.05 cm}{scalable}%
\cuminitiali{}{temporalia/capitulum-FratresMihi.gtex}
\grechangedim{interwordspacetext}{0.22 cm plus 0.15 cm minus 0.05 cm}{scalable}}
\else
\newcommand{\capitulumLaudes}{\pars{Capitulum.} \scriptura{1 Ptr. 5, 6-7}

\grechangedim{interwordspacetext}{0.12 cm plus 0.15 cm minus 0.05 cm}{scalable}%
\cuminitiali{}{temporalia/capitulum-CarissimiHumiliamini.gtex}
\grechangedim{interwordspacetext}{0.22 cm plus 0.15 cm minus 0.05 cm}{scalable}}
\fi

\setlength{\columnsep}{30pt} % prostor mezi sloupci

%%%%%%%%%%%%%%%%%%%%%%%%%%%%%%%%%%%%%%%%%%%%%%%%%%%%%%%%%%%%%%%%%%%%%%%%%%%%%%%%%%%%%%%%%%%%%%%%%%%%%%%%%%%%%
\begin{document}

% Here we set the space around the initial.
% Please report to http://home.gna.org/gregorio/gregoriotex/details for more details and options
\grechangedim{afterinitialshift}{2.2mm}{scalable}
\grechangedim{beforeinitialshift}{2.2mm}{scalable}
\grechangedim{interwordspacetext}{0.22 cm plus 0.15 cm minus 0.05 cm}{scalable}%
\grechangedim{annotationraise}{-0.2cm}{scalable}

% Here we set the initial font. Change 38 if you want a bigger initial.
% Emit the initials in red.
\grechangestyle{initial}{\color{red}\fontsize{38}{38}\selectfont}

\pagestyle{empty}

%%%% Titulni stranka
\begin{titulusOfficii}
\titulus
\end{titulusOfficii}

% graphic
%\vspace{1.5cm}
%\begin{center}
%\includegraphics[width=8cm]{emmaus.jpg}
%\end{center}

\vfill

\begin{center}
%Ad usum et secundum consuetudines chori \guillemotright{}Conventus Choralis\guillemotleft.

%Editio Sancti Wolfgangi \annusEditionis
\end{center}

\pagebreak

\renewcommand{\headrulewidth}{0pt} % no horiz. rule at the header
\fancyhf{}
\pagestyle{fancy}

\cantusSineNeumas

\ifx\festumveldominica\undefined
\else
\pars{Oratio ante divinum Officium.}

\lettrine{{\color{red}A}}{peri,} Dómine, os meum ad benedicéndum nomen sanctum tuum:
munda quoque cor meum ab ómnibus vanis, pervérsis, et aliénis
cogitatiónibus:
intelléctum illúmina, afféctum inflámma,
ut digne, atténte ac devóte hoc Offícium recitáre váleam,
et exaudíri mérear ante conspéctum Divínæ Maiestátis tuæ.
Per Christum, Dóminum nostrum.
\Rbardot{} Amen.

Dómine, in unióne illíus divínæ intentiónis,
qua ipse in terris laudes Deo persolvísti,
has tibi Horas \rubricatum{(vel \textnormal{hanc tibi Horam})} persólvo.

%\trOratioAnteOfficium

\vfill

\pars{Oratio post divinum Officium.}

\rubrica{
  Orationem sequentem devote post Officium recitantibus
  Leo Papa X. defectus, et culpas in eo persolvendo ex humana
  fragilitate contractas, indulsit, et dicitur flexis genibus.
}

\lettrine{{\color{red}S}}{acrosánctæ} et indivíduæ Trinitáti,
crucifíxi Dómini nostri Iesu Christi humanitáti,
beatíssimæ et gloriosíssimæ sempérque Vírginis Maríæ
fecúndæ integritáti,
et ómnium Sanctórum universitáti
sit sempitérna laus, honor, virtus et glória
ab omni creatúra,
nobísque remíssio ómnium peccatórum,
per infiníta sǽcula sæculórum.
\Rbardot{} Amen.

\noindent \Vbardot{} Beáta víscera Maríæ Virginis, quæ portavérunt
ætérni Patris Fílium.\\
\Rbardot{} Et beáta úbera, quæ lactavérunt Christum Dominum.

\rubrica{Et dicitur secreto \textnormal{Pater noster.} et \textnormal{Ave María.}}

%\trOratioPostOfficium

\vfill

\hora{Ad I. Vesperas.} %%%%%%%%%%%%%%%%%%%%%%%%%%%%%%%%%%%%%%%%%%%%%%%%%%%%%
%\sideThumbs{I. Vesperæ}

\vspace{0.5cm}
\grechangedim{interwordspacetext}{0.18 cm plus 0.15 cm minus 0.05 cm}{scalable}%
\ifx\festum\undefined
\cuminitiali{}{temporalia/deusinadiutorium-alter.gtex}
\else
\cuminitiali{}{temporalia/deusinadiutorium-solemnis.gtex}
\fi
\grechangedim{interwordspacetext}{0.22 cm plus 0.15 cm minus 0.05 cm}{scalable}%

\vfill
\pagebreak

\pars{Psalmus 1.} \scriptura{Mt. 11, 30; Ps. 109, 2}

\vspace{-0.4cm}

\antiphona{I a\textsuperscript{2}}{temporalia/ant-suavijugo.gtex}

\scriptura{Psalmus 109.}

\initiumpsalmi{temporalia/ps109-initium-i-a2-auto.gtex}

%\psalmusEtTranslatioT{temporalia/ps109-comb.tex}{10cm}
\input{temporalia/ps109.tex} \Abardot{}

\vspace{-1cm}

\vfill
\pagebreak

\pars{Psalmus 2.} \scriptura{Ps. 110, 4-5}

\vspace{-0.4cm}

\antiphona{II D}{temporalia/ant-misericorsetmiserator.gtex}

\scriptura{Psalmus 110.}

\initiumpsalmi{temporalia/ps110-initium-ii-D-auto.gtex}

%\psalmusEtTranslatioT{temporalia/ps110-comb.tex}{10cm}
\input{temporalia/ps110.tex} \Abardot{}

\vfill
\pagebreak

\pars{Psalmus 3.} \scriptura{Ps. 111, 4}

\vspace{-0.4cm}

\antiphona{VII d}{temporalia/ant-exortumestintenebris.gtex}

\scriptura{Psalmus 111.}

\initiumpsalmi{temporalia/ps111-initium-vii-d-auto.gtex}

%\psalmusEtTranslatioT{temporalia/ps111-comb.tex}{10cm}
\input{temporalia/ps111.tex} \Abardot{}

\vfill
\pagebreak

\pars{Psalmus 4.} \scriptura{Ps. 129, 4.7}

\vspace{-0.4cm}

\antiphona{IV A*}{temporalia/ant-apuddominumpropitiatio.gtex}

\scriptura{Psalmus 129.}

\initiumpsalmi{temporalia/ps129-initium-iv-A_-auto.gtex}

%\psalmusEtTranslatioT{temporalia/ps129-comb.tex}{10cm}
\input{temporalia/ps129.tex} \Abardot{}

%\vfill

%\vspace{-6mm}

%\antiphona{}{temporalia/ant-apuddominumpropitiatio.gtex} % repeat the antiphon - new page

\vfill
\pagebreak

\capitulumLaudes

% preklad Jeruz. bible
%\trCapituliI

\vfill

\ifx\dominica\undefined
\pars{Responsorium.} \scriptura{Mt. 11, 29-30; \textbf{H361}}

\cuminitiali{VII}{temporalia/resp-tollite-cumdox.gtex}
\else
\pars{Responsorium breve.} \scriptura{Ps. 40, 5}

\cuminitiali{VI}{temporalia/resp-egodixi.gtex}
\fi

%\trResp

\vfill
\pagebreak

\pars{Hymnus}

\cuminitiali{I}{temporalia/hym-AuctorBeate.gtex}
\vspace{-3mm}
%\input{hym-AuctorBeate-bohtext.tex}

\vfill
%\pagebreak

\pars{Versus.} \scriptura{Mt. 11, 29}

% Versus. %%%
\ifx\festum\undefined
\sineinitiali{temporalia/versus-tollite-communis.gtex}
\else
\sineinitiali{temporalia/versus-tollite.gtex}
\fi

%\noindent \trVersus

\vfill
\pagebreak

\ifx\dominica\undefined
\pars{Canticum B. Mariæ V.} \scriptura{Lc. 12, 49}

\vspace{-4mm}

{
\grechangedim{interwordspacetext}{0.18 cm plus 0.15 cm minus 0.05 cm}{scalable}%
\antiphona{I D*}{temporalia/ant-ignemvenimittere.gtex}
\grechangedim{interwordspacetext}{0.22 cm plus 0.15 cm minus 0.05 cm}{scalable}%
}

%\trAntIMagnificat

\vspace{-2mm}

\scriptura{Lc. 1, 46-55}

\vspace{-2mm}

\cantusSineNeumas
\initiumpsalmi{temporalia/magnificat-initium-isoll-D_.gtex}

%\psalmusEtTranslatioT{temporalia/magnificat-I-comb.tex}{10.2cm}
\input{temporalia/magnificat-I.tex} \Abardot{}
\else
\pars{Canticum B. Mariæ V.} \scriptura{1 Sam. 3, 20; \textbf{H397}}

\vspace{-3mm}

{
\grechangedim{interwordspacetext}{0.18 cm plus 0.15 cm minus 0.05 cm}{scalable}%
\antiphona{I g}{temporalia/ant-cognoveruntomnes.gtex}
\grechangedim{interwordspacetext}{0.22 cm plus 0.15 cm minus 0.05 cm}{scalable}%
}

%\trAntIMagnificat

%\vspace{-3mm}

\scriptura{Lc. 1, 46-55}

%\vspace{-2mm}

\cantusSineNeumas
\initiumpsalmi{temporalia/magnificat-initium-isoll-g.gtex}

%\vspace{-1.5mm}

%\psalmusEtTranslatioT{temporalia/magnificat-III-comb.tex}{10.2cm}
\input{temporalia/magnificat-III.tex} \Abardot{}
\fi

\vspace{-1cm}

\vfill
\pagebreak

%\sideThumbs{{\scriptsize{}Fine horarum}}

\anteOrationem

\pagebreak

% Oratio. %%%
\ifx\dominica\undefined
\cuminitiali{}{temporalia/oratio.gtex}
\else
\cuminitiali{}{temporalia/oratio2.gtex}
\fi

\vspace{-1mm}
%\trOrationisI

\vfill

\rubrica{Hebdomadarius dicit iterum Dominus vobiscum, vel cantor dicit:}

\vspace{2mm}

\sineinitiali{temporalia/domineexaudi.gtex}

\rubrica{Postea cantatur a cantore:}

\vspace{2mm}

\ifx\festum\undefined
\cuminitiali{II}{temporalia/benedicamus-semiduplex-vesp.gtex}
\else
\cuminitiali{II}{temporalia/benedicamus-solemnism-1vesp.gtex}
\fi

\vspace{1mm}

\vfill
\pagebreak
\fi

\ifx\festum\undefined
\else
\hora{Ad Completorium.} %%%%%%%%%%%%%%%%%%%%%%%%%%%%%%%%%%%%%%%%%%%%%%%%%%%%%%%%%%
%\sideThumbs{{\scriptsize{}Completorium}}

\rubrica{Lector petit benedictionem, dicens:}

\cuminitiali{}{temporalia/jubedomnebenedicere.gtex}

%\trJubeDomne

\vfill

\pars{Benedictio.}

\cuminitiali{}{temporalia/benedictio-noctemquietam.gtex}

%\trComplBenedictio

\vfill

\pars{Lectio brevis.} \scriptura{1Ptr. 5, 8-9}

\cuminitiali{}{temporalia/lectiobrevis-fratressobrii.gtex}

%\trComplLectioBr

\vfill

\noindent \Vbardot{} Adiutórium nostrum in nómine Dómini.

\noindent \Rbardot{} Qui fecit cælum, et terram.

\vfill
\pagebreak

\pars{Confessio.}

\noindent Confíteor Deo omnipoténti, beátæ Maríæ semper Vírgini, beáto
Michaéli Archángelo, beáto Ioánni Baptístæ, sanctis Apóstolis Petro
et Paulo, ómnibus Sanctis, et vobis fratres: quia peccávi nimis cogitatióne,
verbo et ópere: mea culpa, mea culpa, mea máxima culpa.
Ideo precor beátam Maríam semper Vírginem, beátum Michaélem
Archángelum, beátum Ioánnem Baptístam, sanctos Apóstolos Petrum
et Paulum, omnes Sanctos, et vos fratres, oráre pro me ad Dóminum
Deum nostrum.

\vfill

\noindent \Vbardot{} Misereátur nostri omnípotens Deus, et, dimíssis peccátis nostris, perdúcat
nos ad vitam ætérnam. \Rbardot{} Amen.

\vfill

\noindent \Vbardot{} Indulgéntiam, absolutiónem et remissiónem peccatórum nostrórum tríbuat nobis
omnípotens et miséricors Dóminus. \Rbardot{} Amen.

\vfill

\rubrica{Et facta absolutione dicitur:}

\sineinitiali{temporalia/convertenosdeus.gtex}

\vfill

\cuminitiali{}{temporalia/deusinadiutorium-communis.gtex}

\vfill
\pagebreak

\pars{Psalmus 1.}

\antiphona{VIII G}{temporalia/ant-miserere.gtex}

\scriptura{Ps. 4}

\initiumpsalmi{temporalia/ps4-initium-viii-G-auto.gtex}

%\psalmusEtTranslatioT{temporalia/ps4-comb.tex}{10cm}
\input{temporalia/ps4.tex}

\vfill
\pagebreak

\pars{Psalmus 2.} \scriptura{Ps. 90}

\initiumpsalmi{temporalia/ps90-initium-viii-G-auto.gtex}

%\psalmusEtTranslatioT{temporalia/ps90-comb.tex}{10cm}
\input{temporalia/ps90.tex}

\pagebreak

\pars{Psalmus 3.} \scriptura{Ps. 133}

\initiumpsalmi{temporalia/ps133-initium-viii-G-auto.gtex}

%\psalmusEtTranslatioT{temporalia/ps133-comb.tex}{10cm}
\input{temporalia/ps133.tex}

\vfill

\antiphona{}{temporalia/ant-miserere.gtex}

\vfill

\pars{Hymnus.}

\antiphona{II}{temporalia/hym-TeLucis.gtex}
%\input{hym-TeLucis-bohtext.tex}

\pagebreak

\pars{Capitulum.} \scriptura{Ier. 14, 9}

\cuminitiali{}{temporalia/capitulum-tuautem.gtex}

% preklad Jeruz. bible
%\trComplCapituli

\vfill

\pars{Responsorium breve.} \scriptura{Ps. 30, 6}

\cuminitiali{VI}{temporalia/resp-inmanus.gtex}

%\trRespCompl
\vfill

\pars{Versus.} \scriptura{Ps. 16, 8}

\sineinitiali{temporalia/versus-custodi.gtex}

%\noindent \trComplVersus

\vfill
\pagebreak

\cantusCumNeumis

\pars{Canticum Simeonis.}

\vspace{-3mm}

\antiphona{III a}{temporalia/ant-salvanos-antiquo.gtex}

%\trAntSalvaNos

%\vspace{-1mm}

\scriptura{Lc. 2, 29-32}

\vspace{-2mm}

\initiumpsalmi{temporalia/nuncdimittis-initium-iii-a-auto.gtex}

%\psalmusEtTranslatioT{temporalia/nuncdimittis-comb.tex}{10cm}
\input{temporalia/nuncdimittis.tex} \Abardot{}

\vfill

\rubrica{Ante Orationem, cantatur a Superiore:}

\vspace{3mm}

\pars{Supplicatio Litaniæ.}

\cuminitiali{}{temporalia/supplicatiolitaniae.gtex}

\vspace{7mm}

\pars{Oratio Dominica.}

\noindent Pater noster.

\vfill
\pagebreak

\sineinitiali{temporalia/domineexaudi-simplex.gtex}

\vspace{7mm}

\pars{Oratio.}

\cantusSineNeumas

\cuminitiali{}{temporalia/oratio-visita.gtex}

%\trComplOrationis

\vfill

%\sineinitiali{temporalia/domineexaudi-communis.gtex}

\noindent \Vbardot{} Dómine, exáudi oratiónem meam. \Rbardot{} Et clamor meus ad te véniat.

\vfill

%\vfill

\sineinitiali{temporalia/benedicamus-minor.gtex}

\vfill

\pars{Benedictio.}

\noindent Benedícat et custódiat nos omnípotens et miséricors Dóminus, \gredagger{}
Pater, et Fílius, et Spíritus Sanctus. \Rbardot{} Amen.

\vfill
\pagebreak

\pars{Antiphona finalis B. M. V.}

\vspace{-4mm}

\antiphona{I}{temporalia/an_salve_regina.gtex}

\rubrica{vel:}

\vspace{-4mm}

\antiphona{V}{temporalia/ant-salveregina-simplex.gtex}

\vfill
\pagebreak

\rubrica{vel:}

\antiphona{VII}{temporalia/ant-subtuum.gtex}

\vfill
\pagebreak
\fi

\hora{Ad Matutinum.} %%%%%%%%%%%%%%%%%%%%%%%%%%%%%%%%%%%%%%%%%%%%%%%%%%%%%
%\sideThumbs{Matutinum}

\vspace{2mm}

\cuminitiali{}{temporalia/dominelabiamea.gtex}

\vspace{2mm}

\pars{Invitatorium.} \scriptura{Cantor; \textbf{LH118}}

\vspace{-6mm}

\antiphona{IV}{temporalia/inv-coriesuamorenostri.gtex}

\vfill
\pagebreak

\ifx\breviori\undefined
\pars{Hymnus.} \scriptura{\textbf{LH120}}

\vspace{-5mm}

\antiphona{III}{temporalia/hym-JesuAuctor.gtex}
\else
\pars{Hymnus}

\cuminitiali{I}{temporalia/hym-CorArca.gtex}
\fi

\vfill
\pagebreak

\subhora{In I. Nocturno}

\ifx\breviori\undefined
\pars{Psalmus 1.} \scriptura{Ps. 8, 6-7}

\vspace{-2mm}

\antiphona{I a\textsuperscript{2}}{temporalia/ant-gloriaethonorecoronasti.gtex}

%\vspace{-5mm}

\scriptura{Ps. 8}

%\vspace{-2mm}

\initiumpsalmi{temporalia/ps8-initium-i-a2-auto.gtex}

%\psalmusEtTranslatioT{temporalia/ps8-comb.tex}{10cm}
\input{temporalia/ps8.tex} \Abardot{}

\vfill
\pagebreak

\pars{Psalmus 2.} \scriptura{Ps. 18, 15}

\vspace{-4mm}

\antiphona{I g\textsuperscript{3}}{temporalia/ant-meditatiocordismei.gtex}

\vspace{-2mm}

\scriptura{Ps. 18}

\vspace{-2mm}

\initiumpsalmi{temporalia/ps18-initium-i-g3-auto.gtex}

%\vspace{-1.5mm}

%\psalmusEtTranslatioT{temporalia/ps18-comb.tex}{10cm}
\input{temporalia/ps18.tex}

\vfill

\antiphona{}{temporalia/ant-meditatiocordismei.gtex}

\vfill
\pagebreak

\pars{Psalmus 3.} \scriptura{Ps. 23, 7.9}

\vspace{-4mm}

\antiphona{V a}{temporalia/ant-elevamini.gtex}

%\vspace{-2mm}

\scriptura{Ps. 23}

\vspace{-2mm}

\initiumpsalmi{temporalia/ps23-initium-v-a-auto.gtex}

%\psalmusEtTranslatioT{temporalia/ps23-comb.tex}{10cm}
\input{temporalia/ps23.tex} \Abardot{}

\vfill
\pagebreak
\else
\pars{Psalmus 1.} \scriptura{Ps. 18, 15}

\vspace{-4mm}

\antiphona{I g\textsuperscript{3}}{temporalia/ant-meditatiocordismei.gtex}

\vspace{-2mm}

\scriptura{Ps. 18}

\vspace{-2mm}

\initiumpsalmi{temporalia/ps18-initium-i-g3-auto.gtex}

%\vspace{-1.5mm}

\input{temporalia/ps18.tex}

\vfill

\antiphona{}{temporalia/ant-meditatiocordismei.gtex}

\vfill
\pagebreak

\pars{Psalmus 2.} \scriptura{Ps. 35, 9-10}

\vspace{-4mm}

\antiphona{III a}{temporalia/ant-apudteestfonsvitae.gtex}

\vspace{-2mm}

\scriptura{Ps. 35}

\vspace{-2mm}

\initiumpsalmi{temporalia/ps35-initium-iii-a-auto.gtex}

%\vspace{-1.5mm}

\input{temporalia/ps35.tex} \Abardot{}

\vfill
\pagebreak

\pars{Psalmus 3.} \scriptura{Ps. 45, 5}

\vspace{-2mm}

\antiphona{V a}{temporalia/ant-sanctificavittabernaculum.gtex}

%\vspace{-3mm}

\scriptura{Ps. 45}

%\vspace{-2mm}

\initiumpsalmi{temporalia/ps45-initium-v-a_.gtex}

%\vspace{-1.5mm}

\input{temporalia/ps45.tex} \Abardot{}

\vspace{-1cm}

\vfill
\pagebreak
\fi

\pars{Versus.} \scriptura{Is. 26, 12}

% Versus. %%%
\sineinitiali{temporalia/versus-domine.gtex}

\vspace{5mm}

\sineinitiali{temporalia/oratiodominica-mat.gtex}

\vspace{5mm}

\pars{Absolutio.}

\cuminitiali{}{temporalia/absolutio-exaudi.gtex}

\vfill
\pagebreak

\cuminitiali{}{temporalia/benedictio-solemn-benedictione.gtex}

\vspace{7mm}

\lectioi

\noindent \Vbardot{} Tu autem, Dómine, miserére nobis.
\noindent \Rbardot{} Deo grátias.

\vfill
\pagebreak

\ifx\breviori\undefined
\pars{Responsorium 1.} \scriptura{\Rbardot{} Io. 6, 48 \Vbardot{} ibid. 6, 51.52}

\vspace{-2mm}

\responsorium{VII}{temporalia/resp-egosumpanisvitae-E611.gtex}{}
\else
\pars{Responsorium 1.} \scriptura{\Rbardot{} Io. 6, 58 \Vbardot{} Eccli. 15, 3}

\vspace{-2mm}

\responsorium{VIII}{temporalia/resp-misitmevivenspater2.gtex}{}
\fi

\vfill
\pagebreak

\cuminitiali{}{temporalia/benedictio-solemn-unigenitus.gtex}

\vspace{7mm}

\lectioii

\noindent \Vbardot{} Tu autem, Dómine, miserére nobis.
\noindent \Rbardot{} Deo grátias.

\vfill
\pagebreak

\ifx\breviori\undefined
\pars{Responsorium 2.} \scriptura{\Rbardot{} Is. 53, 2.5 \Vbardot{} ibid., 4; \textbf{H178}}

\vspace{-2mm}

\responsorium{V}{temporalia/resp-eccevidimus-CROCHU.gtex}{}
\else
\pars{Responsorium 2.} \scriptura{\Rbardot{} Is. 53, 7 \Vbardot{} ibid. 53, 12; \textbf{H224}}

\vspace{-2mm}

\responsorium{IV}{temporalia/resp-sicutovis-CROCHU.gtex}{}
\fi

\vfill
\pagebreak

\cuminitiali{}{temporalia/benedictio-solemn-spiritus.gtex}

\vspace{7mm}

\lectioiii

\noindent \Vbardot{} Tu autem, Dómine, miserére nobis.
\noindent \Rbardot{} Deo grátias.

\vfill
\pagebreak

\ifx\breviori\undefined
\pars{Responsorium 3.} \scriptura{\Rbardot{} Ps. 71, 18 \Vbardot{} ibid., 19; \textbf{H102}}

\vspace{-2mm}

\responsorium{II}{temporalia/resp-benedictus-dominus-deus2-cumdox.gtex}{}
\else
\pars{Responsorium 3.} \scriptura{\Rbardot{} Ps. 146, 5 \Vbardot{} ibid., 6; \textbf{H101}}

\vspace{-2mm}

\responsorium{II}{temporalia/resp-magnusdominusnoster-CROCHU-cumdox.gtex}{}
\fi

\vfill
\pagebreak

\subhora{In II. Nocturno}

\ifx\breviori\undefined
\pars{Psalmus 4.} \scriptura{Ps. 44, 18}

\vspace{-2mm}

\antiphona{VI F}{temporalia/ant-confitebuntur.gtex}

\vspace{-2mm}

\scriptura{Ps. 44}

\initiumpsalmi{temporalia/ps44-initium-vi-F-auto.gtex}

%\psalmusEtTranslatioT{temporalia/ps44-comb.tex}{10cm}
\input{temporalia/ps44.tex}

\vfill

\antiphona{}{temporalia/ant-confitebuntur.gtex}

\vfill
\pagebreak

\pars{Psalmus 5.} \scriptura{Ps. 45, 5}

\vspace{-2mm}

\antiphona{V a}{temporalia/ant-sanctificavittabernaculum.gtex}

%\vspace{-3mm}

\scriptura{Ps. 45}

%\vspace{-2mm}

\initiumpsalmi{temporalia/ps45-initium-v-a_.gtex}

%\vspace{-1.5mm}

%\psalmusEtTranslatioT{temporalia/ps45-comb.tex}{10cm}
\input{temporalia/ps45.tex} \Abardot{}

\vspace{-1cm}

\vfill
\pagebreak

\pars{Psalmus 6.} \scriptura{Ps. 86, 7}

\vspace{-2mm}

\antiphona{VII c}{temporalia/ant-sicutlaetantium.gtex}

%\vspace{-5mm}

\scriptura{Ps. 86}

\initiumpsalmi{temporalia/ps86-initium-vii-c-auto.gtex}

%\psalmusEtTranslatioT{temporalia/ps86-comb.tex}{10cm}
\input{temporalia/ps86.tex} \Abardot{}

\vfill
\pagebreak
\else
\pars{Psalmus 4.} \scriptura{Ps. 60, 3}

\vspace{-2mm}

\antiphona{IV* e}{temporalia/ant-dumanxiareturcorinpetra.gtex}

%\vspace{-5mm}

\scriptura{Ps. 60}

\initiumpsalmi{temporalia/ps60-initium-iv_-e-auto.gtex}

\input{temporalia/ps60.tex} \Abardot{}

\vfill
\pagebreak

\pars{Psalmus 5.} \scriptura{Ps. 86, 7}

\vspace{-2mm}

\antiphona{VII c}{temporalia/ant-sicutlaetantium.gtex}

%\vspace{-5mm}

\scriptura{Ps. 86}

\initiumpsalmi{temporalia/ps86-initium-vii-c-auto.gtex}

\input{temporalia/ps86.tex} \Abardot{}

\vfill
\pagebreak

\pars{Psalmus 6.} \scriptura{Ps. 97, 3}

\vspace{-2mm}

\antiphona{I a\textsuperscript{2}}{temporalia/ant-videruntomnes.gtex}

\vspace{-2mm}

\scriptura{Ps. 97}

\initiumpsalmi{temporalia/ps97-initium-i-a2-auto.gtex}

%\psalmusEtTranslatioT{temporalia/ps97-comb.tex}{10cm}
\input{temporalia/ps97.tex} \Abardot{}

\vfill
\pagebreak
\fi

\pars{Versus.} \scriptura{Is. 12, 1}

% Versus. %%%
\sineinitiali{temporalia/versus-conversus.gtex}

\vspace{5mm}

\sineinitiali{temporalia/oratiodominica-mat.gtex}

\vspace{5mm}

\pars{Absolutio.}

\cuminitiali{}{temporalia/absolutio-ipsius.gtex}

\vfill
\pagebreak

\cuminitiali{}{temporalia/benedictio-solemn-deus.gtex}

\vspace{7mm}

\lectioiv

\noindent \Vbardot{} Tu autem, Dómine, miserére nobis.
\noindent \Rbardot{} Deo grátias.

\vfill
\pagebreak

\pars{Responsorium 4.} \scriptura{\Rbardot{} Io. 6, 57 \Vbardot{} Dt. 4, 7}

\vspace{-2mm}

\responsorium{VII}{temporalia/resp-quimanducat-E611.gtex}{}

\vfill
\pagebreak

\cuminitiali{}{temporalia/benedictio-solemn-christus.gtex}

\vspace{7mm}

\lectiov

\noindent \Vbardot{} Tu autem, Dómine, miserére nobis.
\noindent \Rbardot{} Deo grátias.

\vfill
\pagebreak

\ifx\breviori\undefined
\pars{Responsorium 5.} \scriptura{\Rbardot{} Is. 53, 7 \Vbardot{} ibid. 53, 12; \textbf{H224}}

\vspace{-2mm}

\responsorium{IV}{temporalia/resp-sicutovis-CROCHU.gtex}{}
\else
\pars{Responsorium 5.} \scriptura{\Rbardot{} Io. 6, 48 \Vbardot{} ibid. 6, 51.52}

\vspace{-2mm}

\responsorium{VII}{temporalia/resp-egosumpanisvitae-E611.gtex}{}
\fi

\vfill
\pagebreak

\cuminitiali{}{temporalia/benedictio-solemn-ignem.gtex}

\vspace{7mm}

\lectiovi

\noindent \Vbardot{} Tu autem, Dómine, miserére nobis.
\noindent \Rbardot{} Deo grátias.

\vfill
\pagebreak

\ifx\breviori\undefined
\pars{Responsorium 6.} \scriptura{\Rbardot{} Ps. 146, 5 \Vbardot{} ibid., 6; \textbf{H101}}

\vspace{-2mm}

\responsorium{II}{temporalia/resp-magnusdominusnoster-CROCHU-cumdox.gtex}{}
\else
\pars{Responsorium 6.} \scriptura{\Rbardot{} Ap. 22, 1; \Vbardot{} ibid. 14, 6; \textbf{H248}}

\vspace{-5mm}

\responsorium{VII}{temporalia/resp-ostenditmihiangelus-CROCHU-cumdox.gtex}{}
\fi

\vfill
\pagebreak

\subhora{In III. Nocturno}

\ifx\breviori\undefined
\pars{Psalmus 7.} \scriptura{Ps. 95, 2}

\vspace{-2mm}

\antiphona{II D}{temporalia/ant-annuntiate.gtex}

\vspace{-2mm}

\scriptura{Ps. 95}

%\vspace{-2mm}

\initiumpsalmi{temporalia/ps95-initium-ii-D-auto.gtex}

%\psalmusEtTranslatioT{temporalia/ps95-comb.tex}{10cm}
\input{temporalia/ps95.tex} \Abardot{}

\vfill
\pagebreak

\pars{Psalmus 8.} \scriptura{Ps. 96, 12}

\vspace{-2mm}

\antiphona{V a}{temporalia/ant-laetamini.gtex}

\vspace{-2mm}

\scriptura{Ps. 96}

%\vspace{-2mm}

\initiumpsalmi{temporalia/ps96-initium-v-a-auto.gtex}

%\vspace{-1mm}

%\psalmusEtTranslatioT{temporalia/ps96-comb.tex}{10cm}
\input{temporalia/ps96.tex} \Abardot{}

\vfill
\pagebreak

\pars{Psalmus 9.} \scriptura{Ps. 97, 3}

\vspace{-2mm}

\antiphona{I a\textsuperscript{2}}{temporalia/ant-videruntomnes.gtex}

\vspace{-2mm}

\scriptura{Ps. 97}

\initiumpsalmi{temporalia/ps97-initium-i-a2-auto.gtex}

%\psalmusEtTranslatioT{temporalia/ps97-comb.tex}{10cm}
\input{temporalia/ps97.tex} \Abardot{}

\vfill
\pagebreak
\else
\pars{Psalmus 7.} \scriptura{Mt. 11, 29; \textbf{H364}}

\vspace{-2mm}

\antiphona{III a}{temporalia/ant-tolliteiugum.gtex}

\vspace{-2mm}

\scriptura{Canticum Isaiæ Prophetæ, Is. 12, 1-7}

%\vspace{-2mm}

\initiumpsalmi{temporalia/isaiae-initium-iii-a-auto.gtex}

\input{temporalia/isaiae.tex}

\vfill
\pagebreak

\scriptura{Canticum Annæ, 1 Reg. 2, 1-5}

%\vspace{-2mm}

\initiumpsalmi{temporalia/annai-initium-iii-a-auto.gtex}

%\vspace{-1mm}

\input{temporalia/annai.tex}

\vfill
\pagebreak

\scriptura{Canticum Annæ, 1 Reg. 2, 6-10}

%\vspace{-2mm}

\initiumpsalmi{temporalia/annaii-initium-iii-a-auto.gtex}

%\vspace{-1mm}

\input{temporalia/annaii.tex}

\vfill

\antiphona{}{temporalia/ant-tolliteiugum.gtex}

\vfill
\pagebreak
\fi

\pars{Versus.} \scriptura{Ps. 49, 2-3}

% Versus. %%%
\sineinitiali{temporalia/versus-exsion.gtex}

\vspace{5mm}

\sineinitiali{temporalia/oratiodominica-mat.gtex}

\vspace{5mm}

\pars{Absolutio.}

\cuminitiali{}{temporalia/absolutio-avinculis.gtex}

\vfill
\pagebreak

\cuminitiali{}{temporalia/benedictio-solemn-evangelica.gtex}

\vspace{7mm}

\lectiovii

\noindent \Vbardot{} Tu autem, Dómine, miserére nobis.
\noindent \Rbardot{} Deo grátias.

\vfill
\pagebreak

\ifx\breviori\undefined
\pars{Responsorium 7.} \scriptura{\Rbardot{} Io. 6, 58 \Vbardot{} Eccli. 15, 3}

\vspace{-2mm}

\responsorium{VIII}{temporalia/resp-misitmevivenspater2.gtex}{}
\else
\pars{Responsorium 7.} \scriptura{\Rbardot{} Cantor \Vbardot{} ibidem; \textbf{H223}}

\vspace{-2mm}

\responsorium{VII}{temporalia/resp-recesitpastornoster-CROCHU-cumdox.gtex}{}
\fi

\vfill
\pagebreak

\ifx\breviori\undefined
\cuminitiali{}{temporalia/benedictio-solemn-divinum.gtex}

\vspace{7mm}

\lectioviii

\noindent \Vbardot{} Tu autem, Dómine, miserére nobis.
\noindent \Rbardot{} Deo grátias.

\vfill
\pagebreak

\pars{Responsorium 8.} \scriptura{\Rbardot{} Cantor \Vbardot{} ibidem; \textbf{H223}}

\vspace{-2mm}

\responsorium{VII}{temporalia/resp-recesitpastornoster-CROCHU-cumdox.gtex}{}

\vfill
\pagebreak

\cuminitiali{}{temporalia/benedictio-solemn-adsocietatem.gtex}

\vspace{7mm}

\lectioix

\noindent \Vbardot{} Tu autem, Dómine, miserére nobis.
\noindent \Rbardot{} Deo grátias.
\fi

\vfill
\pagebreak

% Te Deum

%\pars{Hymnus Ambrosianus}

\vspace{-5mm}

\ifx\solemnis\undefined
\ifx\aequus\undefined
{
\pars{Hymnus Ambrosianus} \scriptura{Alio modo, iuxta morem Romanum}

\vspace{-2mm}

\grechangedim{interwordspacetext}{0.26 cm plus 0.15 cm minus 0.05 cm}{scalable}%
\cuminitiali{III}{temporalia/tedeum-romanum-gn.gtex}
\grechangedim{interwordspacetext}{0.22 cm plus 0.15 cm minus 0.05 cm}{scalable}%
}
\else
{
\pars{Hymnus Ambrosianus} \scriptura{Tonus Simplex}

\vspace{-2mm}

\grechangedim{interwordspacetext}{0.30 cm plus 0.15 cm minus 0.05 cm}{scalable}%
\cuminitiali{III}{temporalia/tedeum-simplex-gn.gtex}
\grechangedim{interwordspacetext}{0.22 cm plus 0.15 cm minus 0.05 cm}{scalable}%
}
\fi
\else
{
\pars{Hymnus Ambrosianus} \scriptura{Tonus Solemnis}

\vspace{-2mm}

\grechangedim{interwordspacetext}{0.26 cm plus 0.15 cm minus 0.05 cm}{scalable}%
\cuminitiali{III}{temporalia/tedeum-solemnis-gn.gtex}
\grechangedim{interwordspacetext}{0.22 cm plus 0.15 cm minus 0.05 cm}{scalable}%
}
\fi

\vfill
\pagebreak

\rubrica{Reliqua omittuntur, nisi Laudes separandæ sint.}

\sineinitiali{temporalia/domineexaudi.gtex}

\vfill

\pars{Oratio.}

\ifx\dominica\undefined
\cuminitiali{}{temporalia/oratio.gtex}
\else
\cuminitiali{}{temporalia/oratio2.gtex}
\fi

\vfill

\noindent \Vbardot{} Dómine, exáudi oratiónem meam.
\Rbardot{} Et clamor meus ad te véniat.

\vfill

% Nocturnale Romanum 2002, p. LXXVI Benedicamus Domino seems to match
% the one from Solemn Laudes.
\cuminitiali{V}{temporalia/benedicamus-solemnis-laud.gtex}

\vfill

\noindent \Vbardot{} Fidélium ánimæ per misericórdiam Dei requiéscant in pace.
\Rbardot{} Amen.

\vfill
\pagebreak

\hora{Ad Laudes.} %%%%%%%%%%%%%%%%%%%%%%%%%%%%%%%%%%%%%%%%%%%%%%%%%%%%%
%\sideThumbs{Laudes}

\cantusSineNeumas

\vspace{0.5cm}
\grechangedim{interwordspacetext}{0.18 cm plus 0.15 cm minus 0.05 cm}{scalable}%
\ifx\festumveldominica\undefined
\cuminitiali{}{temporalia/deusinadiutorium-communis.gtex}
\else
\cuminitiali{}{temporalia/deusinadiutorium-alter.gtex}
\fi
\grechangedim{interwordspacetext}{0.22 cm plus 0.15 cm minus 0.05 cm}{scalable}%

\vfill
%\pagebreak
\ifx\breviori\undefined

\pars{Psalmus 1.} \scriptura{Io. 19, 34}

\vspace{-0.4cm}

\antiphona{I g}{temporalia/ant-unusmilitum.gtex}

\scriptura{Psalmus 92.}

\initiumpsalmi{temporalia/ps92-initium-i-g-auto.gtex}

%\psalmusEtTranslatioT{temporalia/ps92-comb.tex}{10cm}
\input{temporalia/ps92.tex} \Abardot{}

\vfill
\pagebreak

\pars{Psalmus 2.} \scriptura{Io. 7, 37-38}

\vspace{-0.4cm}

\antiphona{\textit{IV d}}{temporalia/ant-quisititveniat.gtex}

\scriptura{Psalmus 99.}

\initiumpsalmi{temporalia/ps99-initium-iv-D-auto.gtex}

%\psalmusEtTranslatioT{temporalia/ps99-comb.tex}{10cm}
\input{temporalia/ps99.tex} \Abardot{}

\vfill
\pagebreak

\pars{Psalmus 3.} \scriptura{Cf. Ier. 31, 3}

\vspace{-0.4cm}

\antiphona{VIII G}{temporalia/ant-incaritateperpetua.gtex}

\scriptura{Psalmus 62.}

\initiumpsalmi{temporalia/ps62-initium-viii-G-auto.gtex}

%\psalmusEtTranslatioT{temporalia/ps62-comb.tex}{10cm}
\input{temporalia/ps62.tex} \Abardot{}

\vfill
\pagebreak

\pars{Psalmus 4.} \scriptura{Mt. 11, 28}

\vspace{-0.4cm}

\antiphona{IV E}{temporalia/ant-veniteadmeomnesqui.gtex}

\scriptura{Canticum trium puerorum, Dan. 3, 57-88 et 56}

\initiumpsalmi{temporalia/dan3-initium-iv-E-auto.gtex}

%\psalmusEtTranslatioT{temporalia/dan3-comb.tex}{10cm}
\input{temporalia/dan3.tex}

\rubrica{Hic non dicitur Gloria Patri, neque Amen.}

\vfill

\vspace{-6mm}

\antiphona{}{temporalia/ant-veniteadmeomnesqui.gtex} % repeat the antiphon - new page

\vfill
\pagebreak

\pars{Psalmus 5.} \scriptura{Prv. 23, 26}

\vspace{-0.4cm}

\antiphona{I g}{temporalia/ant-filipraebe.gtex}

\scriptura{Psalmus 148.}

\initiumpsalmi{temporalia/ps148-initium-i-g-auto.gtex}

%\psalmusEtTranslatioT{temporalia/ps148-comb.tex}{10cm}
\input{temporalia/ps148.tex}

\rubrica{Hic non dicitur Gloria Patri.}

\vfill
\pagebreak

%
\scriptura{Psalmus 149.}

\initiumpsalmi{temporalia/ps149-initium-i-g-auto.gtex}

%\psalmusEtTranslatioT{temporalia/ps149-comb.tex}{10cm}
\input{temporalia/ps149.tex}

\rubrica{Hic non dicitur Gloria Patri.}

\vfill
\pagebreak

%
\scriptura{Psalmus 150.}

\initiumpsalmi{temporalia/ps150-initium-i-g-auto.gtex}

%\psalmusEtTranslatioT{temporalia/ps150-comb.tex}{10cm}
\input{temporalia/ps150.tex}

\vfill

\vspace{-6mm}

\antiphona{}{temporalia/ant-filipraebe.gtex} % repeat the antiphon - new page

\vfill
\pagebreak

\capitulumLaudes
\else
\pagebreak

\pars{Hymnus.} \scriptura{\textbf{LH120}}

\vspace{-5mm}

\antiphona{III}{temporalia/hym-JesuAuctor.gtex}

\vfill
\pagebreak

\pars{Psalmus 1.} \scriptura{Io. 7, 37}

\vspace{-0.4cm}

\antiphona{VIII G\textsuperscript{5}}{temporalia/ant-indiemagno.gtex}

\scriptura{Psalmus 62.}

\initiumpsalmi{temporalia/ps62-initium-viii-g5.gtex}

\input{temporalia/ps62.tex} \Abardot{}

\vfill
\pagebreak

\pars{Psalmus 2.} \scriptura{Mt. 11, 28}

\vspace{-0.4cm}

\antiphona{IV E}{temporalia/ant-veniteadmeomnesqui.gtex}

\scriptura{Canticum trium puerorum, Dan. 3, 57-88 et 56}

\initiumpsalmi{temporalia/dan3-initium-iv-E-auto.gtex}

\input{temporalia/dan3.tex}

\rubrica{Hic non dicitur Gloria Patri, neque Amen.}

\vfill

\vspace{-6mm}

\antiphona{}{temporalia/ant-veniteadmeomnesqui.gtex} % repeat the antiphon - new page

\vfill
\pagebreak

\pars{Psalmus 3.} \scriptura{Prv. 23, 26}

\vspace{-0.4cm}

\antiphona{I g}{temporalia/ant-filipraebe.gtex}

\scriptura{Psalmus 149.}

\initiumpsalmi{temporalia/ps149-initium-i-g-auto.gtex}

\input{temporalia/ps149.tex}

\begin{psalmus}
Glória \textbf{Pa}\-tri et \textbf{Fí}\-lio,~\grestar{}
et Spirí\emph{\-tui }\textbf{Sanc}\-to.

Sicut erat in princípio, et \textbf{nunc} et \textbf{sem}\-per,~\grestar{}
et in sǽcula sæcu\emph{ló\-rum. }\textbf{A}\-men.
\end{psalmus} \Abardot{}

\vfill
\pagebreak

\pars{Lectio brevis.} \scriptura{Ier. 31, 33}

\noindent Hoc erit pactum, quod fériam cum domo Israel post dies illos, dicit Dóminus: Dabo legem meam in viscéribus eórum et in corde eórum scribam eam; et ero eis in Deum, et ipsi erunt mihi in pópulum.
\fi

% preklad Jeruz. bible
%\trCapituliI

\vfill

\pars{Responsorium breve.} \scriptura{Mt. 11, 29}

\cuminitiali{VI}{temporalia/resp-tollite.gtex}

%\trResp

\vfill
\pagebreak

\ifx\breviori\undefined
\pars{Hymnus}

\cuminitiali{I}{temporalia/hym-CorArca.gtex}
\vspace{-3mm}
%\input{hym-CorArca-bohtext.tex}

\vfill
%\pagebreak

\pars{Versus.} \scriptura{Is. 12, 3}

% Versus. %%%
\sineinitiali{temporalia/versus-haurietis-communis.gtex}

%\noindent \trVersus

\vfill
\pagebreak
\fi

\ifx\dominica\undefined
\ifx\breviori\undefined
\pars{Canticum Zachariæ.} \scriptura{Io. 19, 36-37}

\vspace{-4mm}

{
\grechangedim{interwordspacetext}{0.18 cm plus 0.15 cm minus 0.05 cm}{scalable}%
\antiphona{VIII G}{temporalia/ant-factasuntenimhaec.gtex}
\grechangedim{interwordspacetext}{0.22 cm plus 0.15 cm minus 0.05 cm}{scalable}%
}
\else
\pars{Canticum Zachariæ.} \scriptura{Lc. 1, 78}

\vspace{-4mm}

{
\grechangedim{interwordspacetext}{0.18 cm plus 0.15 cm minus 0.05 cm}{scalable}%
\antiphona{VIII G}{temporalia/ant-pervisceramisericordiae.gtex}
\grechangedim{interwordspacetext}{0.22 cm plus 0.15 cm minus 0.05 cm}{scalable}%
}
\fi

%\trAntIMagnificat

\vspace{-2mm}

\scriptura{Lc. 1, 68-79}

\vspace{-1mm}

\cantusSineNeumas
\ifx\solemnis\undefined
\initiumpsalmi{temporalia/benedictus-initium-viii-G-auto.gtex}

%\vspace{-1.5mm}

%\psalmusEtTranslatioT{temporalia/benedictus-III-comb.tex}{10.2cm}
\input{temporalia/benedictus-III.tex} \Abardot{}
\else
\initiumpsalmi{temporalia/benedictus-initium-viiisoll-G-auto.gtex}

%\vspace{-1.5mm}

%\psalmusEtTranslatioT{temporalia/benedictus-I-comb.tex}{10.2cm}
\input{temporalia/benedictus-I.tex} \Abardot{}
\fi
\else
\pars{Canticum Zachariæ.} \scriptura{Lc. 15, 4; \textbf{H427}}

\vspace{-7.5mm}

{
\grechangedim{interwordspacetext}{0.18 cm plus 0.15 cm minus 0.05 cm}{scalable}%
\antiphona{III e}{temporalia/ant-quisexvobishomo.gtex}
\grechangedim{interwordspacetext}{0.22 cm plus 0.15 cm minus 0.05 cm}{scalable}%
}

%\trAntIMagnificat

\vspace{-4mm}

\scriptura{Lc. 1, 68-79}

\vspace{-3mm}

\cantusSineNeumas
\initiumpsalmi{temporalia/benedictus-initium-iiisoll-e-auto.gtex}

\vspace{-1.5mm}

%\psalmusEtTranslatioT{temporalia/benedictus-II-comb.tex}{10.2cm}
\input{temporalia/benedictus-II.tex} \Abardot{}
\fi

\vspace{-1cm}

\vfill
\pagebreak

%\sideThumbs{{\scriptsize{}Fine horarum}}

\ifx\breviori\undefined
\anteOrationem

\pagebreak

% Oratio. %%%
\ifx\dominica\undefined
\cuminitiali{}{temporalia/oratio.gtex}
\else
\cuminitiali{}{temporalia/oratio2.gtex}
\fi

\vspace{-1mm}
%\trOrationisI

\vfill

\rubrica{Hebdomadarius dicit iterum Dominus vobiscum, vel cantor dicit:}

\vspace{2mm}

\sineinitiali{temporalia/domineexaudi.gtex}

\rubrica{Postea cantatur a cantore:}

\vspace{2mm}
\else
\ifx\precestotum\undefined
\pars{Preces.}

\sineinitiali{}{temporalia/tonusprecum.gtex}

\noindent Iesum, qui est mitis et húmilis corde, deprecémur, fratres, \gredagger{} eíque supplicémus:

\Rbardot{} Rex amantíssime, miserére.

\noindent Iesu, in quo hábitat omnis plenitúdo divinitátis, \gredagger{} divínæ consórtes natúræ tuæ nos éffice.

\Rbardot{} Rex amantíssime, miserére.

\noindent Iesu, in quo sunt omnes thesáuri sapiéntiæ et sciéntiæ, \gredagger{} multifórmem sapiéntiam Dei per Ecclésiam nobis revéla.

\Rbardot{} Rex amantíssime, miserére.

\noindent Iesu, in quo Pater sibi bene complácuit, \gredagger{} prædicatiónis tuæ fac nos auditóres perseverántes.

\Rbardot{} Rex amantíssime, miserére.

\noindent Iesu, de cuius plenitúdine omnes nos accépimus, \gredagger{} grátiam et veritátem Patris nobis abundánter largíre.

\Rbardot{} Rex amantíssime, miserére.

\noindent Iesu, fons vitæ et sanctitátis, \gredagger{} sanctos et immaculátos in caritáte nos redde.

\Rbardot{} Rex amantíssime, miserére.

\vfill

\pars{Oratio Dominica.}

\cuminitiali{}{temporalia/oratiodominicaalt.gtex}

\vfill
\pagebreak

\rubrica{vel:}

\pars{Supplicatio Litaniæ.}

\cuminitiali{}{temporalia/supplicatiolitaniae.gtex}

\vfill

\pars{Oratio Dominica.}

\cuminitiali{}{temporalia/oratiodominica.gtex}
\else
\precestotum
\fi

\vfill
\pagebreak

% Oratio. %%%
\pars{Oratio.}

\noindent Concéde, quǽsumus, omnípotens Deus, ut, qui dilécti Fílii tui corde gloriántes, eius præcípua in nos benefícia recólimus caritátis, de illo donórum fonte cælésti supereffluéntem grátiam mereámur accípere.

\ifx\precestotum\undefined
\else
\pars{Pro pace in Ucraina.} \scriptura{Sir. 50, 25; 2 Esdr. 4, 20; \textbf{H416}}

\vspace{-4mm}

\antiphona{II D}{temporalia/ant-dapacemdomine.gtex}

\vfill

\noindent Deus, a quo sancta desidéria, recta consília et iusta sunt ópera: da servis tuis illam, quam mundus dare non potest, pacem; ut et corda nostra mandátis tuis dédita, et hóstium subláta formídine, témpora sint tua protectióne tranquílla.
\fi

\noindent Per Dóminum nostrum Iesum Christum, Fílium tuum, qui tecum vivit et regnat in unitáte Spíritus Sancti, Deus, per ómnia sǽcula sæculórum.

\noindent \Rbardot{} Amen.

\vfill

\rubrica{Hebdomadarius dicit Dominus vobiscum, vel, absente sacerdote vel diacono, sic concluditur:}

\vspace{2mm}

\ifx\dominusnosbenedicat\undefined
\antiphona{C}{temporalia/dominusnosbenedicat.gtex}
\else
\dominusnosbenedicat
\fi

\rubrica{Postea cantatur a cantore:}

\vspace{2mm}
\fi

\ifx\festum\undefined
\ifx\octava\undefined
\cuminitiali{I}{temporalia/benedicamus-semiduplex-laud.gtex}
\else
\cuminitiali{VIII}{temporalia/benedicamus-duplexmajus-laudes.gtex}
\fi
\else
\cuminitiali{II}{temporalia/benedicamus-solemnism-laud.gtex}
\fi

\vspace{1mm}

\vfill
\pagebreak

\ifx\sabbato\undefined
\ifx\festumveldominica\undefined
\hora{Ad Vesperas.} %%%%%%%%%%%%%%%%%%%%%%%%%%%%%%%%%%%%%%%%%%%%%%%%%%%%%
%\sideThumbs{Vesperæ}
\else
\hora{Ad II. Vesperas.} %%%%%%%%%%%%%%%%%%%%%%%%%%%%%%%%%%%%%%%%%%%%%%%%%%%%%
%\sideThumbs{II. Vesperæ}
\fi

\cantusSineNeumas

%\vspace{-2mm}
\grechangedim{interwordspacetext}{0.18 cm plus 0.15 cm minus 0.05 cm}{scalable}%
\ifx\festumveldominica\undefined
\cuminitiali{}{temporalia/deusinadiutorium-communis.gtex}
\else
\ifx\festum\undefined
\cuminitiali{}{temporalia/deusinadiutorium-alter.gtex}
\else
\cuminitiali{}{temporalia/deusinadiutorium-solemnis.gtex}
\fi
\fi
\grechangedim{interwordspacetext}{0.22 cm plus 0.15 cm minus 0.05 cm}{scalable}%

\vfill
%\pagebreak

%\vspace{-2mm}

\pars{Psalmus 1.} \scriptura{Io. 19, 34}

\vspace{-0.4cm}

\antiphona{I g}{temporalia/ant-unusmilitum.gtex}

\scriptura{Psalmus 109.}

\initiumpsalmi{temporalia/ps109-initium-i-g-auto.gtex}

%\psalmusEtTranslatioT{temporalia/ps109-comb.tex}{10cm}
\input{temporalia/ps109.tex} \Abardot{}

\vfill
\pagebreak

\pars{Psalmus 2.} \scriptura{Io. 7, 37-38}

\vspace{-0.4cm}

\antiphona{\textit{IV d}}{temporalia/ant-quisititveniat.gtex}

\scriptura{Psalmus 110.}

\initiumpsalmi{temporalia/ps110-initium-iv-D-auto.gtex}

%\psalmusEtTranslatioT{temporalia/ps110ivD-comb.tex}{10cm}
\input{temporalia/ps110ivD.tex} \Abardot{}

\vfill
\pagebreak

\pars{Psalmus 3.} \scriptura{Cf. Ier. 31, 3}

\vspace{-0.4cm}

\antiphona{VIII G}{temporalia/ant-incaritateperpetua.gtex}

\scriptura{Psalmus 115.}

\initiumpsalmi{temporalia/ps115-initium-viii-G-auto.gtex}

%\psalmusEtTranslatioT{temporalia/ps115-comb.tex}{10cm}
\input{temporalia/ps115.tex} \Abardot{}

\vfill
\pagebreak

\pars{Psalmus 4.} \scriptura{Prv. 23, 26}

\vspace{-0.4cm}

\antiphona{I g}{temporalia/ant-filipraebe.gtex}

\scriptura{Psalmus 127.}

\initiumpsalmi{temporalia/ps127-initium-i-g-auto.gtex}

%\psalmusEtTranslatioT{temporalia/ps127-comb.tex}{10cm}
\input{temporalia/ps127.tex} \Abardot{}

\vfill
\pagebreak

\capitulumLaudes

% preklad Jeruz. bible
%\trCapituliI

\vfill

\ifx\festum\undefined
\pars{Responsorium breve.} \scriptura{Ps. 40, 5}

\cuminitiali{VI}{temporalia/resp-egodixi.gtex}
\else
\pars{Responsorium breve.} \scriptura{Cf. Ap. 5, 9-10}

\cuminitiali{II}{temporalia/resp-christus.gtex}
\fi

%\trResp

\vfill
\pagebreak

\pars{Hymnus}

\cuminitiali{I}{temporalia/hym-AuctorBeate.gtex}
\vspace{-3mm}
%\input{hym-AuctorBeate-bohtext.tex}

\vfill
%\pagebreak

\pars{Versus.} \scriptura{Is. 12, 3}

% Versus. %%%
\ifx\festum\undefined
\sineinitiali{temporalia/versus-haurietis-communis.gtex}
\else
\sineinitiali{temporalia/versus-haurietis.gtex}
\fi

%\noindent \trVersus

\vfill
\pagebreak

\ifx\dominica\undefined
\pars{Canticum B. Mariæ V.} \scriptura{Io. 19, 33-34}

%\vspace{-5.5mm}

{
\grechangedim{interwordspacetext}{0.18 cm plus 0.15 cm minus 0.05 cm}{scalable}%
\antiphona{I f}{temporalia/ant-adjesumautem.gtex}
\grechangedim{interwordspacetext}{0.22 cm plus 0.15 cm minus 0.05 cm}{scalable}%
}

%\trAntIMagnificat

\vspace{-3mm}

\scriptura{Lc. 1, 46-55}

\vspace{-2.5mm}

\cantusSineNeumas
\ifx\solemnis\undefined
\initiumpsalmi{temporalia/magnificat-initium-i-f.gtex}

\vspace{-1.5mm}

%\psalmusEtTranslatioT{temporalia/magnificat-V-comb.tex}{10.2cm}
\input{temporalia/magnificat-V.tex} \Abardot{}
\else
\initiumpsalmi{temporalia/magnificat-initium-isoll-f.gtex}

\vspace{-1.5mm}

%\psalmusEtTranslatioT{temporalia/magnificat-II-comb.tex}{10.2cm}
\input{temporalia/magnificat-II.tex} \Abardot{}
\fi
\else
\pars{Canticum B. Mariæ V.} \scriptura{Lc. 15, 8; \textbf{H427}}

\vspace{-5mm}

{
\grechangedim{interwordspacetext}{0.18 cm plus 0.15 cm minus 0.05 cm}{scalable}%
\antiphona{VI F}{temporalia/ant-quaemulier.gtex}
\grechangedim{interwordspacetext}{0.22 cm plus 0.15 cm minus 0.05 cm}{scalable}%
}

\vspace{-3mm}

%\trAntIMagnificat

\scriptura{Lc. 1, 46-55}

\vspace{-2mm}

\cantusSineNeumas
\initiumpsalmi{temporalia/magnificat-initium-visoll-F.gtex}

%\vspace{-1mm}

%\psalmusEtTranslatioT{temporalia/magnificat-IV-comb.tex}{10.2cm}
\input{temporalia/magnificat-IV.tex} \Abardot{}
\fi

\vspace{-1cm}

\vfill
\pagebreak

%\sideThumbs{{\scriptsize{}Fine horarum}}

\anteOrationem

\pagebreak

% Oratio. %%%
\ifx\dominica\undefined
\cuminitiali{}{temporalia/oratio.gtex}
\else
\cuminitiali{}{temporalia/oratio2.gtex}
\fi

\vspace{-1mm}
%\trOrationisI

\vfill

\rubrica{Hebdomadarius dicit iterum Dominus vobiscum, vel cantor dicit:}

\vspace{2mm}

\sineinitiali{temporalia/domineexaudi.gtex}

\rubrica{Postea cantatur a cantore:}

\vspace{2mm}

\ifx\festum\undefined
\cuminitiali{II}{temporalia/benedicamus-semiduplex-vesp.gtex}
\else
\cuminitiali{II}{temporalia/benedicamus-solemnism-2vesp.gtex}
\fi

\vspace{1mm}
\fi

\end{document}

