\newcommand{\titulus}{\nomenFesti{In Assumptione B.M.V.}
\celebratio{Duplex 1. classis.}}
\newcommand{\festum}{Assumptione B.M.V.}
\newcommand{\aequus}{Assumptione B.M.V.}
\newcommand{\festumveldominica}{Assumptione B.M.V.}
\newcommand{\solemnis}{Assumptione B.M.V.}
\newcommand{\mmxxi}{2021}
\newcommand{\lectioi}{\pars{Lectio I.} \scriptura{Ct. 1, 1-4}

\noindent Incípiunt Cántica canticórum.

\noindent \textit{\color{red}(Sponsa)} Osculétur me ósculo oris sui, quia melióra sunt úbera tua vino, fragrántia unguéntis óptimis. Oleum effúsum nomen tuum; ídeo adolescéntulæ dilexérunt te.

\noindent \textit{\color{red}(Chorus Adolescentulárum)} Trahe me: post te currémus in odórem unguentórum tuórum. Introdúxit me rex in cellária sua; exsultábimus et lætábimur in te mémores úberum tuórum super vinum. Recti díligunt te.

\noindent \textit{\color{red}(Sponsa)} Nigra sum, sed formósa, fíliæ Ierúsalem, sicut tabernácula Cedar, sicut pelles Salomónis.}
\newcommand{\lectioii}{\pars{Lectio II.} \scriptura{Ct. 1, 5-9}

\noindent Nolíte me consideráre quod fusca sim, quia decolorávit me sol. Fílii matris meæ pugnavérunt contra me, posuérunt me custódem in víneis, víneam meam non custodívi. Indica mihi, quem díligit ánima mea, ubi pascas, ubi cubes in merídie, ne vagári incípiam post greges sodálium tuórum.

\noindent \textit{\color{red}(Sponsus)} Si ignóras te, o pulchérrima inter mulíeres, egrédere et abi post vestígia gregum, et pasce hædos tuos iuxta tabernácula pastórum. Equitátui meo in cúrribus pharaónis assimilávi te, amíca mea. Pulchræ sunt genæ tuæ sicut túrturis, collum tuum sicut monília.}
\newcommand{\lectioiii}{\pars{Lectio III.} \scriptura{Ct. 1, 10-16}

\noindent Murénulas áureas faciémus tibi vermiculátas argénto.

\noindent \textit{\color{red}(Sponsa)} Dum esset rex in accúbitu suo, nardus mea dedit odórem suum. Fascículus myrrhæ diléctus meus mihi, inter úbera mea commorábitur. Botrus Cypri diléctus meus mihi in víneis Engáddi.

\noindent \textit{\color{red}(Sponsus)} Ecce tu pulchra es, amíca mea, ecce tu pulchra es; óculi tui columbárum.

\noindent \textit{\color{red}(Sponsa)} Ecce tu pulcher es, dilécte mi, et decórus. Léctulus noster flóridus, Tigna domórum nostrárum cédrina, laqueária nostra cypréssina.}
\newcommand{\lectioiv}{\pars{Lectio IV.} \scriptura{Orátio 2 de Dormit. B. M. V., post init}

\noindent Sermo sancti Ioánnis Damascéni.

\noindent Hódie sacra et animáta arca Dei vivéntis, quæ suum in útero concépit Creatórem, requiéscit in templo Dómini, quod nullis est exstrúctum mánibus. Et David exsúltat eius parens, et cum eo choros ducunt Angeli, célebrant Archángeli, Virtútes gloríficant, Principátus exsúltant, Potestátes collætántur, gaudent Dominatiónes, Throni festum diem agunt, laudant Chérubim, glóriam eius prǽdicant Séraphim. Hódie Eden novi Adam paradísum súscipit animátum, in quo solúta est condemnátio, in quo plantátum est lignum vitæ, in quo opérta fuit nostra núditas.}
\newcommand{\lectiov}{\pars{Lectio V.}

\noindent Hódie Virgo immaculáta, quæ nullis terrénis inquináta est afféctibus, sed cæléstibus educáta cogitatiónibus, non in terram revérsa est; sed, cum esset animátum cælum, in cæléstibus tabernáculis collocátur. Ex qua enim ómnibus vera vita manávit, quómodo illa mortem gustáret? Sed cedit legi latæ ab eo quem génuit; et, ut fília véteris Adam, véterem senténtiam súbiit (nam et eius Fílius, qui est vita ipsa, eam non recusávit); ut autem Dei vivéntis Mater, ad illum ipsum digne assúmitur.}
\newcommand{\lectiovi}{\pars{Lectio VI.}

\noindent Heva, quæ serpéntis suggestióni assénsum prǽbuit, partus dolóre, et mortis senténtia damnátur, et in inferórum collocátur penetrálibus. Hanc autem vere beátam, quæ Dei verbo aures prǽstitit, et Spíritus sancti operatióne repléta est, atque ad Archángeli spiritálem salutatiónem, sine voluptáte et viríli consórtio, Dei Fílium concépit, et sine dolóre áliquo péperit, ac totam se Deo consecrávit, quonam modo mors devoráret? quómodo ínferi suscíperent? quómodo corrúptio inváderet corpus illud, in quo vita suscépta est? Huic recta, plana, et fácilis ad cælum paráta est via. Si enim, Ubi ego sum, illic et miníster meus erit, inquit vita et véritas Christus; quómodo non pótius mater cum ipso erit?}
\newcommand{\lectiovii}{\pars{Lectio VII.} \scriptura{Lc. 1, 39-47}

\noindent Léctio sancti Evangélii secúndum Lucam.

\noindent Exsúrgens María in diébus illis ábiit in montána cum festinatióne in civitátem Iudæ. Et réliqua.

\scriptura{Sermo 34: PL 207, 663-665}

\noindent Ex Sermónibus Petri Blesénsis presbýteri.

\noindent Hódie in cúria cælésti inter príncipes et potestátes angélicas celebérrima est gratiárum áctio et vox laudis, quia in cælurn assúmitur Mater Dei. Nos autem pótius plángere quam pláudere decet, qui in hac valle lacrymárum privámur illíus præséntia, quæ nobis erat pública occásio gaudiórum. Speret tamen et exspéctet per eam peregrinátio nostra solátium tempestívum.

\noindent Assúmitur enim ut trahat nos post se in odórem unguentórum suórum; vadit paráre nobis locum; præcédit advocáta fidélis, et potens procuráre nostræ salútis negótium, vadit ad Fílium; amor dabit dona homínibus, tamquam mater Dei, dona regni, regína cæléstium potestátum, cæli et terræ póssidens principátum. Arca fœ́deris longam moram fécerat in Azóto; sed hódie, reducénte David, sollémni exsultatióne suscípitur in Ierúsalem, quia María de valle misériæ et ploratiónis assúmpta cæléstem ingréditur civitátem. Vere benedícta hæc inter mulíeres, et dum Christum in terris súscipit, et dum in cælis a Christo suscípitur. Vere beáta quæ secúre dícere potest: Beátam me dicent omnes generatiónes.}
\newcommand{\lectioviii}{\pars{Lectio VIII.}

\noindent Dum assumerétur beáta Virgo in cælum, mirabántur ángeli lucis et dicébant: Quæ est ista quæ procédit sicut auróra consúrgens, pulchra ut luna, elécta ut sol? O quam pulchra est ascénsio tua, pulchérrima mulíerum! O quam pulchri sunt gressus tui in calceaméntis, fília príncipis! Ea enim caritáte qua vóluit minorári paulo minus ab ángelis, vóluit et ipse matrem suam glorificári præ ángelis.}
\newcommand{\lectioix}{\pars{Lectio IX.}

\noindent Secúre iam potest in cælum ascéndere quæ in terris vitam duxit angélicam; relínquens homínibus pacem, firmans in eis fidem et inexstirpábilem caritátem. Gáudeant hódie Adam et Eva paréntes, aut pótius peremptóres nostri, quia, dum María cælum ingréditur, posteritáti eius reserátur ingréssus. Eva nos duxit in misériam, ista nos exáltat in glóriam. Evæ supérbia nobis ábstulit paradísum, Maríæ humílitas nos revéxit ad cælum.}
% LuaLaTeX

\documentclass[a4paper, twoside, 12pt]{article}
\usepackage[latin]{babel}
%\usepackage[landscape, left=3cm, right=1.5cm, top=2cm, bottom=1cm]{geometry} % okraje stranky
%\usepackage[landscape, a4paper, mag=1166, truedimen, left=2cm, right=1.5cm, top=1.6cm, bottom=0.95cm]{geometry} % okraje stranky
\usepackage[landscape, a4paper, mag=1400, truedimen, left=0.5cm, right=0.5cm, top=0.5cm, bottom=0.5cm]{geometry} % okraje stranky

\usepackage{fontspec}
\setmainfont[FeatureFile={junicode.fea}, Ligatures={Common, TeX}, RawFeature=+fixi]{Junicode}
%\setmainfont{Junicode}

% shortcut for Junicode without ligatures (for the Czech texts)
\newfontfamily\nlfont[FeatureFile={junicode.fea}, Ligatures={Common, TeX}, RawFeature=+fixi]{Junicode}

\usepackage{multicol}
\usepackage{color}
\usepackage{lettrine}
\usepackage{fancyhdr}

% usual packages loading:
\usepackage{luatextra}
\usepackage{graphicx} % support the \includegraphics command and options
\usepackage{gregoriotex} % for gregorio score inclusion
\usepackage{gregoriosyms}
\usepackage{wrapfig} % figures wrapped by the text
\usepackage{parcolumns}
\usepackage[contents={},opacity=1,scale=1,color=black]{background}
\usepackage{tikzpagenodes}
\usepackage{calc}
\usepackage{longtable}
\usetikzlibrary{calc}

\setlength{\headheight}{14.5pt}

\input{conventuscommune.tex} % Often used macros
%%%% Preklady jednotlivych zpevu (nektere se opakuji, a je dobre mit je
% vsechny na jedne hromade)

% HOURS ---

\newcommand{\trAntI}{\translatioCantus{Muž boží měl kožený toulec, pečlivě
zavázaný, jenž mu visel na šíji a~často se ho dotýkal.}}

\newcommand{\trAntII}{\translatioCantus{Klíč od~něho tak dobře střežil, že
dokud žil v~těle, nikdo z~jeho žáků nezvěděl, co je uvnitř.}}

\newcommand{\trAntIII}{\translatioCantus{Ale když se odebral z~tohoto
života, schránku otevřeli a~objevili v~ní žíněné roucho a~měděný řetěz
potřísněný krví.}}

\newcommand{\trAntIV}{\translatioCantus{A když prohlédli mistrovo tělo,
nalezli jeho tělo na čtyřech místech hluboce zbrázděno ranami od řetězu.}}

\newcommand{\trAntV}{\translatioCantus{Krev vytékající z~těch ran, místy
prostoupila i~žíněným rouchem.}}

\newcommand{\trCapituli}{\translatioCantus{
Miláčkovi Boha a~lidí,
Mojžíšovi požehnané paměti,~\gredagger{}
dopřál slávu rovnou slávě svatých~\grestar{}
učinil ho mocným na postrach nepřátelům
a~jeho slovy zastavil divy.}}

\newcommand{\trLectioBrevis}{\translatioCantus{
Pamatujte na své představené,
kteří vám hlásali Boží slovo.
Uvažte, jak oni skončili život, a~napodobujte jejich víru.
Ježíš Kristus je stejný včera i~dnes i~navěky.
Nenechte se svést věelijakými cizími naukami.}}

\newcommand{\trRespLaud}{\translatioCantus{Spravedlivého vodil Hospodin~\grestar{}
po přímých stezkách. \Vbardot{} A~ukázal mu Boží království.}}

\newcommand{\trRespLaudB}{\translatioCantus{Na tvých hradbách, Jeruzaléme,
ustanovil jsem strážné;~\grestar{}
budou bdít nad mým lidem. \Vbardot{} Ani ve dne, ani v~noci nesmějí nikdy
mlčet.}}

\newcommand{\trVersus}{\translatioCantus{\Vbardot{} Ústa spravedlivého šeptají moudrost, aleluja.
\Rbardot{} A~jeho jazyk ohlašuje právo, aleluja.}}

\newcommand{\trAntBenedictus}{\translatioCantus{Když na bujné oře vložili
nosítka a~sňali jim uzdu, vydali se přímo k~cele božího muže.}}

\newcommand{\trPreces}{\translatioCantus{
\noindent S vděčností chvalme Krista, dobrého Pastýře, \gredagger{} který dal život za své ovce, \grestar{} a~pokorně ho prosme: \Rbardot{} Pane, buď pastýřem svého lidu.

\noindent Kriste, ty dáváš církvi pastýře, a~jejich službou se ujímáš svého lidu, \grestar{} dej, ať v~lásce těch, kteří nás vedou, poznáváme, jak nás miluješ. \Rbardot{} Pane, buď pastýřem svého lidu.

\noindent Ty stále konáš skrze své zástupce službu pastýře a~učitele, \grestar{} nepřestávej nás nikdy vést prostřednictvím svých služebníků. \Rbardot{} Pane, buď pastýřem svého lidu.

\noindent Ty prokazuješ svému lidu skrze jeho pastýře službu lékaře duše i~těla, \grestar{} ochraňuj náš život a~veď nás ke svatosti. \Rbardot{} Pane, buď pastýřem svého lidu.

\noindent Ty posíláš své svaté, aby slovem i~příkladem vedli tvůj lid k~tobě, \grestar{} na jejich přímluvu nás posiluj, abychom vytrvali na cestě, která vede k~věčnému životu. \Rbardot{} Pane, buď pastýřem svého lidu.}}

\newcommand{\trOrationis}{\translatioCantus{Bože, jenž nám dopřáváš radovat
se z~výroční slavnosti svatého tvého vyznavače Havla, uděl dobrotivě,
abychom když slavíme jeho narození, též se řídili podobou jeho skutků.
Skrze…}}
 % Czech translations of the proper texts

\newcommand{\annusEditionis}{2020}

%%%% Vicekrat opakovane kousky

\newcommand{\anteOrationem}{
  \rubrica{Ante Orationem, cantatur a Superiore:}

  \pars{Supplicatio Litaniæ.}

  \cuminitiali{}{temporalia/supplicatiolitaniae.gtex}

  \pars{Oratio Dominica.}

  \cuminitiali{}{temporalia/oratiodominica.gtex}

  \rubrica{Deinde dicitur ab Hebdomadario:}

  \cuminitiali{}{temporalia/dominusvobiscum-solemnis.gtex}

  \rubrica{In choro monialium loco Dominus vobiscum dicitur:}

  \sineinitiali{temporalia/domineexaudi.gtex}
}

\setlength{\columnsep}{30pt} % prostor mezi sloupci

%%%%%%%%%%%%%%%%%%%%%%%%%%%%%%%%%%%%%%%%%%%%%%%%%%%%%%%%%%%%%%%%%%%%%%%%%%%%%%%%%%%%%%%%%%%%%%%%%%%%%%%%%%%%%
\begin{document}

% Here we set the space around the initial.
% Please report to http://home.gna.org/gregorio/gregoriotex/details for more details and options
\grechangedim{afterinitialshift}{2.2mm}{scalable}
\grechangedim{beforeinitialshift}{2.2mm}{scalable}
\grechangedim{interwordspacetext}{0.22 cm plus 0.15 cm minus 0.05 cm}{scalable}%
\grechangedim{annotationraise}{-0.2cm}{scalable}

% Here we set the initial font. Change 38 if you want a bigger initial.
% Emit the initials in red.
\grechangestyle{initial}{\color{red}\fontsize{38}{38}\selectfont}

\pagestyle{empty}

%%%% Titulni stranka
\begin{titulusOfficii}
\titulus
\end{titulusOfficii}

% graphic
%\vspace{1.5cm}
%\begin{center}
%\includegraphics[width=8cm]{emmaus.jpg}
%\end{center}

\vfill

\begin{center}
%Ad usum et secundum consuetudines chori \guillemotright{}Conventus Choralis\guillemotleft.

%Editio Sancti Wolfgangi \annusEditionis
\end{center}

\pagebreak

\renewcommand{\headrulewidth}{0pt} % no horiz. rule at the header
\fancyhf{}
\pagestyle{fancy}

\cantusSineNeumas

\ifx\festum\undefined
\else
\pars{Oratio ante divinum Officium.}

\lettrine{{\color{red}A}}{peri,} Dómine, os meum ad benedicéndum nomen sanctum tuum:
munda quoque cor meum ab ómnibus vanis, pervérsis, et aliénis
cogitatiónibus:
intelléctum illúmina, afféctum inflámma,
ut digne, atténte ac devóte hoc Offícium recitáre váleam,
et exaudíri mérear ante conspéctum Divínæ Maiestátis tuæ.
Per Christum, Dóminum nostrum.
\Rbardot{} Amen.

Dómine, in unióne illíus divínæ intentiónis,
qua ipse in terris laudes Deo persolvísti,
has tibi Horas \rubricatum{(vel \textnormal{hanc tibi Horam})} persólvo.

%\trOratioAnteOfficium

\vfill

\pars{Oratio post divinum Officium.}

\rubrica{
  Orationem sequentem devote post Officium recitantibus
  Leo Papa X. defectus, et culpas in eo persolvendo ex humana
  fragilitate contractas, indulsit, et dicitur flexis genibus.
}

\lettrine{{\color{red}S}}{acrosánctæ} et indivíduæ Trinitáti,
crucifíxi Dómini nostri Iesu Christi humanitáti,
beatíssimæ et gloriosíssimæ sempérque Vírginis Maríæ
fecúndæ integritáti, 
et ómnium Sanctórum universitáti
sit sempitérna laus, honor, virtus et glória
ab omni creatúra,
nobísque remíssio ómnium peccatórum,
per infiníta sǽcula sæculórum.
\Rbardot{} Amen.

\noindent \Vbardot{} Beáta víscera Maríæ Virginis, quæ portavérunt
ætérni Patris Fílium.\\
\Rbardot{} Et beáta úbera, quæ lactavérunt Christum Dominum.

\rubrica{Et dicitur secreto \textnormal{Pater noster.} et \textnormal{Ave María.}}

%\trOratioPostOfficium

\vfill

\hora{Ad I. Vesperas.} %%%%%%%%%%%%%%%%%%%%%%%%%%%%%%%%%%%%%%%%%%%%%%%%%%%%%
%\sideThumbs{I. Vesperæ}

\vspace{0.5cm}
\grechangedim{interwordspacetext}{0.18 cm plus 0.15 cm minus 0.05 cm}{scalable}%
\ifx\festum\undefined
\cuminitiali{}{temporalia/deusinadiutorium-alter.gtex}
\else
\cuminitiali{}{temporalia/deusinadiutorium-solemnis.gtex}
\fi
\grechangedim{interwordspacetext}{0.22 cm plus 0.15 cm minus 0.05 cm}{scalable}%

\vfill
\pagebreak

\pars{Psalmus 1.}

\vspace{-4mm}

\antiphona{I g\textsuperscript{4}}{temporalia/ant-osculeturme.gtex}

\scriptura{Psalmus 112.}

\initiumpsalmi{temporalia/ps112-initium-i-g4.gtex}

%\psalmusEtTranslatioT{temporalia/ps112-comb.tex}{10cm}
\input{temporalia/ps112.tex} \Abardot{}

\vspace{-1cm}

\vfill
\pagebreak

\pars{Psalmus 2.} \scriptura{Ap. 12, 1}

\vspace{-4mm}

\antiphona{I g}{temporalia/ant-mulieramictasole.gtex}

\scriptura{Psalmus 145.}

\initiumpsalmi{temporalia/ps145-initium-i-g-auto.gtex}

%\psalmusEtTranslatioT{temporalia/ps145-comb.tex}{10cm}
\input{temporalia/ps145.tex} \Abardot{}

\vfill
\pagebreak

\pars{Psalmus 3.}

\vspace{-4mm}

\antiphona{I f}{temporalia/ant-paradisiportaperevam.gtex}

\scriptura{Psalmus 146.}

\initiumpsalmi{temporalia/ps146-initium-i-f-auto.gtex}

%\psalmusEtTranslatioT{temporalia/ps146-comb.tex}{10cm}
\input{temporalia/ps146.tex} \Abardot{}

\vfill
\pagebreak

\pars{Psalmus 4.}

\vspace{-4mm}

\antiphona{I f}{temporalia/ant-exaltataestgloriosa.gtex}

\scriptura{Psalmus 147.}

\initiumpsalmi{temporalia/ps147-initium-i-f-auto.gtex}

%\psalmusEtTranslatioT{temporalia/ps147-comb.tex}{10cm}
\input{temporalia/ps147.tex} \Abardot{}

\vfill
\pagebreak

\pars{Capitulum.} \scriptura{Iudt. 13, 22}

\grechangedim{interwordspacetext}{0.12 cm plus 0.15 cm minus 0.05 cm}{scalable}%
\cuminitiali{}{temporalia/capitulum-BenedixitTe.gtex}
\grechangedim{interwordspacetext}{0.22 cm plus 0.15 cm minus 0.05 cm}{scalable}

% preklad Jeruz. bible
%\trCapituliI

\vfill

\ifx\festum\undefined
\pars{Responsorium breve.} \scriptura{}

\cuminitiali{VI}{temporalia/resp-inomnemterram.gtex}
\else
\pars{Responsorium.} \scriptura{\Vbardot{} Ct. 3, 6; \textbf{H296}}

\cuminitiali{III}{temporalia/resp-vidispeciosam.gtex}
\fi

%\trResp

\vfill
\pagebreak

\pars{Hymnus}

\cuminitiali{I}{temporalia/hym-GaudiumMundi.gtex}
\vspace{-3mm}
%\input{hym-GaudiumMundi-bohtext.tex}

\vfill
%\pagebreak

\pars{Versus.}

% Versus. %%%
\sineinitiali{temporalia/versus-elegit.gtex}

%\noindent \trVersus

\vfill
\pagebreak

\pars{Canticum B. Mariæ V.} \scriptura{Cf. Ct. 6, 9; 1, 4; \textbf{H298}}

\vspace{-6mm}

{
\grechangedim{interwordspacetext}{0.18 cm plus 0.15 cm minus 0.05 cm}{scalable}%
\antiphona{I g\textsuperscript{4}}{temporalia/ant-virgoprudentissima.gtex}
\grechangedim{interwordspacetext}{0.22 cm plus 0.15 cm minus 0.05 cm}{scalable}%
}

%\trAntIMagnificat

\vspace{-3mm}

\scriptura{Lc. 1, 46-55}

\vspace{-2mm}

\cantusSineNeumas
\initiumpsalmi{temporalia/magnificat-initium-isoll-g4.gtex}

\vspace{-1mm}

%\psalmusEtTranslatioT{temporalia/magnificat-I-comb.tex}{10.2cm}
\input{temporalia/magnificat-I.tex} \Abardot{}

%\vspace{-1cm}

\vfill
\pagebreak

%\sideThumbs{{\scriptsize{}Fine horarum}}

\anteOrationem

\pagebreak

% Oratio. %%%
\cuminitiali{}{temporalia/oratio.gtex}

\vspace{-1mm}
%\trOrationisI

\vfill

\rubrica{Hebdomadarius dicit iterum Dominus vobiscum, vel cantor dicit:}

\vspace{2mm}

\sineinitiali{temporalia/domineexaudi.gtex}

\rubrica{Postea cantatur a cantore:}

\vspace{2mm}

\cuminitiali{I}{temporalia/benedicamus-festis-bmv.gtex}

\vspace{1mm}

\vfill
\pagebreak
\fi

\hora{Ad Matutinum.} %%%%%%%%%%%%%%%%%%%%%%%%%%%%%%%%%%%%%%%%%%%%%%%%%%%%%
%\sideThumbs{Matutinum}

\vspace{2mm}

\cuminitiali{}{temporalia/dominelabiamea.gtex}

\vspace{2mm}

\pars{Invitatorium.}

\vspace{-6mm}

\antiphona{IV}{temporalia/inv-cuiushodie.gtex}

\vfill
\pagebreak

\pars{Hymnus.}

\cuminitiali{VIII}{temporalia/hym-AuroraVelut.gtex}
\vspace{-3mm}
%\input{hym-AuroraVelut-bohtext.tex}

\vfill
\pagebreak

\subhora{In I. Nocturno}

\pars{Psalmus 1.} \scriptura{Ct. 1, 14; \textbf{H309}}

\vspace{-4mm}

\antiphona{I g\textsuperscript{4}}{temporalia/ant-eccetupulchra.gtex}

%\vspace{-5mm}

\scriptura{Ps. 8}

%\vspace{-2mm}

\initiumpsalmi{temporalia/ps8-initium-i-g4.gtex}

%\psalmusEtTranslatioT{temporalia/ps8-comb.tex}{10cm}
\input{temporalia/ps8.tex} \Abardot{}

\vfill
\pagebreak

\pars{Psalmus 2.} \scriptura{Ct. 2, 2; \textbf{H309}}

\vspace{-4mm}

\antiphona{II D}{temporalia/ant-sicutlilium.gtex}

%\vspace{-5mm}

\scriptura{Ps. 18}

\initiumpsalmi{temporalia/ps18-initium-ii-D-auto.gtex}

%\psalmusEtTranslatioT{temporalia/ps18-comb.tex}{10cm}
\input{temporalia/ps18.tex}

\vfill

\antiphona{}{temporalia/ant-sicutlilium.gtex}

\vfill
\pagebreak

\pars{Psalmus 3.} \scriptura{Ct. 4, 11; \textbf{H309}}

\vspace{-4mm}

\antiphona{III a2}{temporalia/ant-favusdistillans.gtex}

%\vspace{-2mm}

\scriptura{Ps. 23}

%\vspace{-2mm}

\initiumpsalmi{temporalia/ps23-initium-iii-a2.gtex}

%\psalmusEtTranslatioT{temporalia/ps23-comb.tex}{10cm}
\input{temporalia/ps23.tex} \Abardot{}

\vfill
\pagebreak

\pars{Versus.}

% Versus. %%%
\sineinitiali{temporalia/versus-exaltata.gtex}

\vspace{5mm}

\sineinitiali{temporalia/oratiodominica-mat.gtex}

\vspace{5mm}

\pars{Absolutio.}

\cuminitiali{}{temporalia/absolutio-exaudi.gtex}

\vfill
\pagebreak

\cuminitiali{}{temporalia/benedictio-solemn-benedictione.gtex}

\vspace{7mm}

\lectioi

\noindent \Vbardot{} Tu autem, Dómine, miserére nobis.
\noindent \Rbardot{} Deo grátias.

\vfill
\pagebreak

\pars{Responsorium 1.} \scriptura{\Vbardot{} Ct. 3, 6; \textbf{H296}}

%\vspace{-5mm}

\cuminitiali{III}{temporalia/resp-vidispeciosam-sinedox.gtex}

\vfill
\pagebreak

\cuminitiali{}{temporalia/benedictio-solemn-unigenitus.gtex}

\vspace{7mm}

\lectioii

\noindent \Vbardot{} Tu autem, Dómine, miserére nobis.
\noindent \Rbardot{} Deo grátias.

\vfill
\pagebreak

\pars{Responsorium 2.} \scriptura{\Rbardot{} Sir. 24, 17 \Vbardot{} ibid. 20; \textbf{H296}}

\vspace{-5mm}

\responsorium{IV}{temporalia/resp-sicutcedrus.gtex}{}

\vfill
\pagebreak

\cuminitiali{}{temporalia/benedictio-solemn-spiritus.gtex}

\vspace{7mm}

\lectioiii

\noindent \Vbardot{} Tu autem, Dómine, miserére nobis.
\noindent \Rbardot{} Deo grátias.

\vfill
\pagebreak

\pars{Responsorium 3.} \scriptura{\Rbardot{} Ps. 44, 3.9-10 \Vbardot{} ibid. 5; \textbf{H297}}

\vspace{-5mm}

\responsorium{IV}{temporalia/resp-diffusaestgratia.gtex}{}

\vfill
\pagebreak

\subhora{In II. Nocturno}

\pars{Psalmus 4.} \scriptura{Ct. 4, 7.11; \textbf{H310}}

\vspace{-4mm}

\antiphona{IV E}{temporalia/ant-totapulchraes.gtex}

\scriptura{Ps. 44}

%\vspace{-2mm}

\initiumpsalmi{temporalia/ps44-initium-iv-E.gtex}

%\psalmusEtTranslatioT{temporalia/ps44-comb.tex}{10cm}
\input{temporalia/ps44.tex}

\vfill
\pagebreak

\antiphona{}{temporalia/ant-totapulchraes.gtex}

\vfill
\pagebreak

\pars{Psalmus 5.} \scriptura{Ct. 4, 15; \textbf{H309}}

\vspace{-4mm}

\antiphona{V a}{temporalia/ant-fonshortorum.gtex}

%\vspace{-4mm}

\scriptura{Ps. 45}

%\vspace{-2mm}

\initiumpsalmi{temporalia/ps45-initium-v-a.gtex}

%\vspace{-1.5mm}

%\psalmusEtTranslatioT{temporalia/ps45-comb.tex}{10cm}
\input{temporalia/ps45.tex} \Abardot{}

\vspace{-1cm}

\vfill
\pagebreak

\pars{Psalmus 6.} \scriptura{Ct. 5, 1; \textbf{H309}}

\vspace{-4mm}

\antiphona{VIII G\textsuperscript{2}}{temporalia/ant-comedifavum.gtex}

%\vspace{-5mm}

\scriptura{Ps. 86}

\initiumpsalmi{temporalia/ps86-initium-viii-G2-auto.gtex}

%\psalmusEtTranslatioT{temporalia/ps86-comb.tex}{10cm}
\input{temporalia/ps86.tex} \Abardot{}

%\vfill

%\antiphona{}{temporalia/ant-anuntiaverunt-FKP.gtex}

\vfill
\pagebreak

\pars{Versus.}

% Versus. %%%
\sineinitiali{temporalia/versus-assumpta.gtex}

\vspace{5mm}

\sineinitiali{temporalia/oratiodominica-mat.gtex}

\vspace{5mm}

\pars{Absolutio.}

\cuminitiali{}{temporalia/absolutio-ipsius.gtex}

\vfill
\pagebreak

\cuminitiali{}{temporalia/benedictio-solemn-deus.gtex}

\vspace{7mm}

\lectioiv

\noindent \Vbardot{} Tu autem, Dómine, miserére nobis.
\noindent \Rbardot{} Deo grátias.

\vfill
\pagebreak

\pars{Responsorium 4.} \scriptura{\Rbardot{} Is. 61, 10; Ps. 44, 12; Ct. 1, 2 \Vbardot{} Ps. 44, 10; \textbf{H297}}

\vspace{-5mm}

\responsorium{VIII}{temporalia/resp-ornataminmonilibus.gtex}{}

\vfill
\pagebreak

\cuminitiali{}{temporalia/benedictio-solemn-christus.gtex}

\vspace{7mm}

\lectiov

\noindent \Vbardot{} Tu autem, Dómine, miserére nobis.
\noindent \Rbardot{} Deo grátias.

\vfill
\pagebreak

\pars{Responsorium 5.} \scriptura{\Vbardot{} Lc. 1, 28; \textbf{H298}}

\vspace{-5mm}

\responsorium{VI}{temporalia/resp-beataesvirgomariadei.gtex}{}

\vfill
\pagebreak

\cuminitiali{}{temporalia/benedictio-solemn-ignem.gtex}

\vspace{7mm}

\lectiovi

\noindent \Vbardot{} Tu autem, Dómine, miserére nobis.
\noindent \Rbardot{} Deo grátias.

\vfill
\pagebreak

\pars{Responsorium 6.} \scriptura{\Vbardot{} Ps. 44, 5; \textbf{H298}}

\vspace{-5mm}

\responsorium{II}{temporalia/resp-istaestspeciosa.gtex}{}

\vfill
\pagebreak

\subhora{In III. Nocturno}

\pars{Psalmus 7.} \scriptura{Ct. 5, 6-8; \textbf{H310}}

\vspace{-4mm}

\antiphona{VII a}{temporalia/ant-animamealiquefacta.gtex}

%\vspace{-4mm}

\scriptura{Ps. 95}

%\vspace{-2mm}

\initiumpsalmi{temporalia/ps95-initium-vii-a.gtex}

%\psalmusEtTranslatioT{temporalia/ps95-comb.tex}{10cm}
\input{temporalia/ps95.tex}

\vfill

\antiphona{}{temporalia/ant-animamealiquefacta.gtex}

\vfill
\pagebreak

\pars{Psalmus 8.} \scriptura{Ct. 5, 16; \textbf{H309}}

\vspace{-4mm}

\antiphona{VI F}{temporalia/ant-talisestdilectus.gtex}

%\vspace{-3mm}

\scriptura{Ps. 96}

%\vspace{-2mm}

\initiumpsalmi{temporalia/ps96-initium-vi-F-auto.gtex}

%\vspace{-1mm}

%\psalmusEtTranslatioT{temporalia/ps96-comb.tex}{10cm}
\input{temporalia/ps96.tex} \Abardot{}

\vfill
\pagebreak

\pars{Psalmus 9.} \scriptura{Ct. 6, 10.12; \textbf{H309}}

\vspace{-4mm}

\antiphona{VII a}{temporalia/ant-descendiinhortum.gtex}

%\vspace{-3mm}

\scriptura{Ps. 97}

%\vspace{-3mm}

\initiumpsalmi{temporalia/ps97-initium-vii-a.gtex}

%\vspace{-1.5mm}

%\psalmusEtTranslatioT{temporalia/ps97-comb.tex}{10cm}
\input{temporalia/ps97.tex} \Abardot{}

\vfill
\pagebreak

\pars{Versus.}

% Versus. %%%
\sineinitiali{temporalia/versus-maria.gtex}

\vspace{5mm}

\sineinitiali{temporalia/oratiodominica-mat.gtex}

\vspace{5mm}

\pars{Absolutio.}

\cuminitiali{}{temporalia/absolutio-avinculis.gtex}

\vfill
\pagebreak

\cuminitiali{}{temporalia/benedictio-solemn-evangelica.gtex}

\vspace{7mm}

\lectiovii

\noindent \Vbardot{} Tu autem, Dómine, miserére nobis.
\noindent \Rbardot{} Deo grátias.

\vfill
\pagebreak

\pars{Responsorium 7.} \scriptura{\textbf{H189}}

\vspace{-5mm}

\responsorium{III}{temporalia/resp-supersalutem.gtex}{}

\vfill
\pagebreak

\cuminitiali{}{temporalia/benedictio-solemn-quorum.gtex}

\vspace{7mm}

\lectioviii

\noindent \Vbardot{} Tu autem, Dómine, miserére nobis.
\noindent \Rbardot{} Deo grátias.

\vfill
\pagebreak

\pars{Responsorium 8.} \scriptura{\Rbardot{} Lc. 1, 48-49 \Vbardot{} ibid. 50; \textbf{H297}}

\vspace{-5mm}

\responsorium{VIII}{temporalia/resp-beatammedicent.gtex}{}

\vfill
\pagebreak

\cuminitiali{}{temporalia/benedictio-solemn-adsocietatem.gtex}

\vspace{7mm}

\lectioix

\noindent \Vbardot{} Tu autem, Dómine, miserére nobis.
\noindent \Rbardot{} Deo grátias.

\vfill
\pagebreak

% Te Deum

%\pars{Hymnus Ambrosianus}

\vspace{-5mm}

\ifx\solemnis\undefined
\ifx\aequus\undefined
{
\pars{Hymnus Ambrosianus} \scriptura{Alio modo, iuxta morem Romanum}

\vspace{-2mm}

\grechangedim{interwordspacetext}{0.26 cm plus 0.15 cm minus 0.05 cm}{scalable}%
\cuminitiali{III}{temporalia/tedeum-romanum-gn.gtex}
\grechangedim{interwordspacetext}{0.22 cm plus 0.15 cm minus 0.05 cm}{scalable}%
}
\else
{
\pars{Hymnus Ambrosianus} \scriptura{Tonus Simplex}

\vspace{-2mm}

\grechangedim{interwordspacetext}{0.30 cm plus 0.15 cm minus 0.05 cm}{scalable}%
\cuminitiali{III}{temporalia/tedeum-simplex-gn.gtex}
\grechangedim{interwordspacetext}{0.22 cm plus 0.15 cm minus 0.05 cm}{scalable}%
}
\fi
\else
{
\pars{Hymnus Ambrosianus} \scriptura{Tonus Solemnis}

\vspace{-2mm}

\grechangedim{interwordspacetext}{0.26 cm plus 0.15 cm minus 0.05 cm}{scalable}%
\cuminitiali{III}{temporalia/tedeum-solemnis-gn.gtex}
\grechangedim{interwordspacetext}{0.22 cm plus 0.15 cm minus 0.05 cm}{scalable}%
}
\fi

\vfill
\pagebreak

\rubrica{Reliqua omittuntur, nisi Laudes separandæ sint.}

\sineinitiali{temporalia/domineexaudi.gtex}

\vfill

\pars{Oratio.}

\cuminitiali{}{temporalia/oratio.gtex}

\vfill

\noindent \Vbardot{} Dómine, exáudi oratiónem meam.
\Rbardot{} Et clamor meus ad te véniat.

\vfill

% Nocturnale Romanum 2002, p. LXXVI Benedicamus Domino seems to match
% the one from Solemn Laudes.
\cuminitiali{V}{temporalia/benedicamus-solemnis-laud.gtex}

\vfill

\noindent \Vbardot{} Fidélium ánimæ per misericórdiam Dei requiéscant in pace.
\Rbardot{} Amen.

\vfill
\pagebreak

\hora{Ad Laudes.} %%%%%%%%%%%%%%%%%%%%%%%%%%%%%%%%%%%%%%%%%%%%%%%%%%%%%
%\sideThumbs{Laudes}

\cantusSineNeumas

\vspace{0.5cm}
\grechangedim{interwordspacetext}{0.18 cm plus 0.15 cm minus 0.05 cm}{scalable}%
\ifx\festumveldominica\undefined
\cuminitiali{}{temporalia/deusinadiutorium-communis.gtex}
\else
\cuminitiali{}{temporalia/deusinadiutorium-alter.gtex}
\fi
\grechangedim{interwordspacetext}{0.22 cm plus 0.15 cm minus 0.05 cm}{scalable}%

\vfill
%\pagebreak

\pars{Psalmus 1.} \scriptura{\textbf{H298}}

\vspace{-6mm}

\antiphona{VII a}{temporalia/ant-assumptaestmaria.gtex}

\vspace{-2mm}

\scriptura{Psalmus 92.}

\vspace{-1mm}

\initiumpsalmi{temporalia/ps92-initium-vii-a-auto.gtex}

%\psalmusEtTranslatioT{temporalia/ps92-comb.tex}{10cm}
\input{temporalia/ps92.tex} \Abardot{}

\vfill
\pagebreak

\pars{Psalmus 2.} \scriptura{\textbf{H298}}

\vspace{-4mm}

\antiphona{VIII G}{temporalia/ant-mariavirgoassumptaest.gtex}

\scriptura{Psalmus 99.}

\initiumpsalmi{temporalia/ps99-initium-viii-G-auto.gtex}

%\psalmusEtTranslatioT{temporalia/ps99-comb.tex}{10cm}
\input{temporalia/ps99.tex} \Abardot{}

\vfill
\pagebreak

\pars{Psalmus 3.} \scriptura{Cf. Ct. 1, 2.3; \textbf{H299}}

\vspace{-4mm}

\antiphona{II* a}{temporalia/ant-inodore.gtex}

\scriptura{Psalmus 62.}

\initiumpsalmi{temporalia/ps62-initium-ii_-a.gtex}

%\psalmusEtTranslatioT{temporalia/ps62-comb.tex}{10cm}
\input{temporalia/ps62.tex} \Abardot{}

\vfill
\pagebreak

\pars{Psalmus 4.} \scriptura{Cf. Iudt. 13, 18; \textbf{H299}}

\vspace{-4mm}

\antiphona{VII c\textsuperscript{2}}{temporalia/ant-benedictafilia.gtex}

\scriptura{Canticum trium puerorum, Dan. 3, 57-88 et 56}

\initiumpsalmi{temporalia/dan3-initium-vii-c2-auto.gtex}

%\psalmusEtTranslatioT{temporalia/dan3-comb.tex}{10cm}
\input{temporalia/dan3.tex}

\rubrica{Hic non dicitur Gloria Patri, neque Amen.}

\vfill

\vspace{-6mm}

\antiphona{}{temporalia/ant-benedictafilia.gtex} % repeat the antiphon - new page

\vfill
\pagebreak

\pars{Psalmus 5.} \scriptura{Ct. 6, 3; \textbf{H299}}

\vspace{-4mm}

\antiphona{I g\textsuperscript{g3}}{temporalia/ant-pulchraesetdecora.gtex}

\scriptura{Psalmus 148.}

\initiumpsalmi{temporalia/ps148-initium-i-g3-auto.gtex}

%\psalmusEtTranslatioT{temporalia/ps148-comb.tex}{10cm}
\input{temporalia/ps148.tex}

\rubrica{Hic non dicitur Gloria Patri.}

\vfill
\pagebreak

%
\scriptura{Psalmus 149.}

\initiumpsalmi{temporalia/ps149-initium-i-g3-auto.gtex}

%\psalmusEtTranslatioT{temporalia/ps149-comb.tex}{10cm}
\input{temporalia/ps149.tex}

\rubrica{Hic non dicitur Gloria Patri.}

\vfill
\pagebreak

%
\scriptura{Psalmus 150.}

\initiumpsalmi{temporalia/ps150-initium-i-g3-auto.gtex}

%\psalmusEtTranslatioT{temporalia/ps150-comb.tex}{10cm}
\input{temporalia/ps150.tex}

\vfill

\vspace{-6mm}

\antiphona{}{temporalia/ant-pulchraesetdecora.gtex} % repeat the antiphon - new page

\vfill
\pagebreak

\pars{Capitulum.} \scriptura{Iudt.. 13, 22}

\grechangedim{interwordspacetext}{0.12 cm plus 0.15 cm minus 0.05 cm}{scalable}%
\cuminitiali{}{temporalia/capitulum-BenedixitTe.gtex}
\grechangedim{interwordspacetext}{0.22 cm plus 0.15 cm minus 0.05 cm}{scalable}

% preklad Jeruz. bible
%\trCapituliI

\vfill

\pars{Responsorium breve.}

\cuminitiali{VI}{temporalia/resp-hodiemariavirgo.gtex}

%\trResp

\vfill
\pagebreak

\pars{Hymnus}

\cuminitiali{I}{temporalia/hym-SolisOVirgo.gtex}
\vspace{-3mm}
%\input{hym-OSolisVirgo-bohtext.tex}
\vfill
%\pagebreak

\pars{Versus.}

% Versus. %%%
\sineinitiali{temporalia/versus-exaltata.gtex}

%\noindent \trVersus

\vfill
\pagebreak

\pars{Canticum Zachariæ.} \scriptura{Ct. 6, 9; \textbf{H299}}

\vspace{-6mm}

{
\grechangedim{interwordspacetext}{0.18 cm plus 0.15 cm minus 0.05 cm}{scalable}%
\antiphona{I g}{temporalia/ant-quaeestista.gtex}
\grechangedim{interwordspacetext}{0.22 cm plus 0.15 cm minus 0.05 cm}{scalable}%
}

%\trAntIMagnificat

\vspace{-4mm}

\scriptura{Lc. 1, 68-79}

\vspace{-2mm}

\cantusSineNeumas
\ifx\solemnis\undefined
\initiumpsalmi{temporalia/benedictus-initium-i-g-auto.gtex}

\vspace{-1.5mm}

%\psalmusEtTranslatioT{temporalia/benedictus-II-comb.tex}{10.2cm}
\input{temporalia/benedictus-II.tex} \Abardot{}
\else
\initiumpsalmi{temporalia/benedictus-initium-isoll-g-auto.gtex}

\vspace{-1.5mm}

%\psalmusEtTranslatioT{temporalia/benedictus-I-comb.tex}{10.2cm}
\input{temporalia/benedictus-I.tex} \Abardot{}
\fi

\vspace{-1cm}

\vfill
\pagebreak

%\sideThumbs{{\scriptsize{}Fine horarum}}

\anteOrationem

\pagebreak

% Oratio. %%%
\cuminitiali{}{temporalia/oratio.gtex}

\vspace{-1mm}
%\trOrationisI

\vfill

\rubrica{Hebdomadarius dicit iterum Dominus vobiscum, vel cantor dicit:}

\vspace{2mm}

\sineinitiali{temporalia/domineexaudi.gtex}

\rubrica{Postea cantatur a cantore:}

\vspace{2mm}

\cuminitiali{I}{temporalia/benedicamus-festis-bmv.gtex}

\vspace{1mm}

\vfill
\pagebreak

\ifx\sabbatoveloctava\undefined
\ifx\festumveldominica\undefined
\hora{Ad Vesperas.} %%%%%%%%%%%%%%%%%%%%%%%%%%%%%%%%%%%%%%%%%%%%%%%%%%%%%
%\sideThumbs{Vesperæ}
\else
\hora{Ad II. Vesperas.} %%%%%%%%%%%%%%%%%%%%%%%%%%%%%%%%%%%%%%%%%%%%%%%%%%%%%
%\sideThumbs{II. Vesperæ}
\fi

\cantusSineNeumas

%\vspace{-2mm}
\grechangedim{interwordspacetext}{0.18 cm plus 0.15 cm minus 0.05 cm}{scalable}%
\ifx\festumveldominica\undefined
\cuminitiali{}{temporalia/deusinadiutorium-communis.gtex}
\else
\ifx\festum\undefined
\cuminitiali{}{temporalia/deusinadiutorium-alter.gtex}
\else
\cuminitiali{}{temporalia/deusinadiutorium-solemnis.gtex}
\fi
\fi
\grechangedim{interwordspacetext}{0.22 cm plus 0.15 cm minus 0.05 cm}{scalable}%

\vfill
%\pagebreak

%\vspace{-2mm}

\pars{Psalmus 1.} \scriptura{\textbf{H298}}

\vspace{-4mm}

\antiphona{VII a}{temporalia/ant-assumptaestmaria.gtex}

\scriptura{Psalmus 109.}

\initiumpsalmi{temporalia/ps109-initium-vii-a-auto.gtex}

%\psalmusEtTranslatioT{temporalia/ps109-comb.tex}{10cm}
\input{temporalia/ps109.tex} \Abardot{}

\vfill
\pagebreak

\pars{Psalmus 2.} \scriptura{\textbf{H298}}

\vspace{-4mm}

\antiphona{VIII G}{temporalia/ant-mariavirgoassumptaest.gtex}

\scriptura{Psalmus 112.}

\initiumpsalmi{temporalia/ps112-initium-viii-G-auto.gtex}

%\psalmusEtTranslatioT{temporalia/ps112viiiG-comb.tex}{10cm}
\input{temporalia/ps112viiiG.tex} \Abardot{}

\vfill
\pagebreak

\pars{Psalmus 3.} \scriptura{Cf. Ct. 1, 2.3; \textbf{H299}}

\vspace{-4mm}

\antiphona{II* a}{temporalia/ant-inodore.gtex}

\scriptura{Psalmus 121.}

\initiumpsalmi{temporalia/ps121-initium-ii_-a-auto.gtex}

%\psalmusEtTranslatioT{temporalia/ps121-comb.tex}{10cm}
\input{temporalia/ps121.tex} \Abardot{}

\vfill
\pagebreak

\pars{Psalmus 4.} \scriptura{Ct. 6, 3; \textbf{H299}}

\vspace{-4mm}

\antiphona{I g\textsuperscript{3}}{temporalia/ant-pulchraesetdecora.gtex}

\scriptura{Psalmus 126.}

\initiumpsalmi{temporalia/ps126-initium-i-g3-auto.gtex}

%\psalmusEtTranslatioT{temporalia/ps126-comb.tex}{10cm}
\input{temporalia/ps126.tex} \Abardot{}

\vfill
\pagebreak

\pars{Capitulum.} \scriptura{Iudt. 13, 22}

\grechangedim{interwordspacetext}{0.12 cm plus 0.15 cm minus 0.05 cm}{scalable}%
\cuminitiali{}{temporalia/capitulum-BenedixitTe.gtex}
\grechangedim{interwordspacetext}{0.22 cm plus 0.15 cm minus 0.05 cm}{scalable}

% preklad Jeruz. bible
%\trCapituliI

\vfill

\pars{Responsorium breve.}

\cuminitiali{VI}{temporalia/resp-mariavirgoexaltataest.gtex}

%\trResp

\vfill
\pagebreak

\pars{Hymnus}

\cuminitiali{I}{temporalia/hym-GaudiumMundi.gtex}
\vspace{-3mm}
%\input{hym-GaudiumMundi-bohtext.tex}

\vfill
%\pagebreak

\pars{Versus.}

% Versus. %%%
\sineinitiali{temporalia/versus-elegit.gtex}

%\noindent \trVersus

\vfill
\pagebreak

\pars{Canticum B. Mariæ V.} \scriptura{\textbf{H300}}

\vspace{-4mm}

{
\grechangedim{interwordspacetext}{0.18 cm plus 0.15 cm minus 0.05 cm}{scalable}%
\antiphona{VIII G\textsuperscript{2}}{temporalia/ant-hodiemariavirgocaelos.gtex}
\grechangedim{interwordspacetext}{0.22 cm plus 0.15 cm minus 0.05 cm}{scalable}%
}

%\trAntIMagnificat

\vspace{-2mm}

\scriptura{Lc. 1, 46-55}

\vspace{-1mm}

\cantusSineNeumas
\ifx\solemnis\undefined
\initiumpsalmi{temporalia/magnificat-initium-viii-G2.gtex}

%\vspace{-1.5mm}

%\psalmusEtTranslatioT{temporalia/magnificat-III-comb.tex}{10.2cm}
\input{temporalia/magnificat-III.tex} \Abardot{}
\else
\initiumpsalmi{temporalia/magnificat-initium-viiisoll-G2.gtex}

%\vspace{-1.5mm}

%\psalmusEtTranslatioT{temporalia/magnificat-II-comb.tex}{10.2cm}
\input{temporalia/magnificat-II.tex} \Abardot{}
\fi

\vspace{-1cm}

\vfill
\pagebreak

%\sideThumbs{{\scriptsize{}Fine horarum}}

\anteOrationem

\pagebreak

% Oratio. %%%
\cuminitiali{}{temporalia/oratio.gtex}

\vspace{-1mm}
%\trOrationisI

\vfill

\rubrica{Hebdomadarius dicit iterum Dominus vobiscum, vel cantor dicit:}

\vspace{2mm}

\sineinitiali{temporalia/domineexaudi.gtex}

\rubrica{Postea cantatur a cantore:}

\vspace{2mm}

\cuminitiali{I}{temporalia/benedicamus-festis-bmv.gtex}

\vspace{1mm}
\fi

\end{document}

