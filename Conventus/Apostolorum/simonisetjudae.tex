\newcommand{\titulus}{\nomenFesti{Ss. Simonis et Judæ Apostolorum.}
\celebratio{Duplex 2. classis.}}
\newcommand{\lectioi}{\pars{Lectio I.} \scriptura{Ac. 3, 1-5}

\noindent De Actibus Apostolórum.

\noindent Petrus autem et Ioánnes ascendébant in templum ad horam oratiónis nonam. Et quidam vir, qui erat claudus ex útero matris suæ, baiulabátur; quem ponébant quotídie ad portam templi, quæ dícitur Speciósa, ut péteret eleemósynam ab introëúntibus in templum. Is, cum vidísset Petrum et Ioánnem incipiéntes introíre in templum, rogábat ut eleemósynam accíperet. Intuens autem in eum Petrus cum Ioánne, dixit: Réspice in nos. At ille intendébat in eos, sperans se áliquid acceptúrum ab eis.}
\newcommand{\lectioii}{\pars{Lectio II.} \scriptura{Ac. 3, 6-10}

\noindent Petrus autem dixit: Argéntum et aurum non est mihi: quod autem hábeo, hoc tibi do: In nómine Iesu Christi Nazaréni surge, et ámbula. Et, apprehénsa manu eius déxtera, allevávit eum, et prótinus consolidátæ sunt bases eius, et plantæ. Et exsíliens stetit, et ambulábat; et intrávit cum illis in templum ámbulans, et exsíliens, et laudans Deum. Et vidit omnis pópulus eum ambulántem, et laudántem Deum. Cognoscébant autem illum, quod ipse erat, qui ad eleemósynam sedébat ad Speciósam portam templi; et impléti sunt stupóre et éxtasi in eo, quod contígerat illi.}
\newcommand{\lectioiii}{\pars{Lectio III.} \scriptura{Ac. 3, 11-16}

\noindent Cum tenéret autem Petrum et Ioánnem, cucúrrit omnis pópulus ad eos ad pórticum, quæ appellátur Salomónis, stupéntes. Videns autem Petrus, respóndit ad pópulum: Viri Israëlítæ, quid mirámini in hoc, aut nos quid intuémini, quasi nostra virtúte aut potestáte fecérimus hunc ambuláre? Deus Abraham, et Deus Isaac, et Deus Iacob, Deus patrum nostrórum glorificávit Fílium suum Iesum, quem vos quidem tradidístis et negástis ante fáciem Piláti, iudicánte illo dimítti. Vos autem Sanctum et Iustum negástis, et petístis virum homicídam donári vobis: Auctórem vero vitæ interfecístis, quem Deus suscitávit a mórtuis, cuius nos testes sumus. Et in fide nóminis eius, hunc, quem vos vidístis et nostis, confirmávit nomen eius: et fides, quæ per eum est, dedit íntegram sanitátem istam in conspéctu ómnium vestrum.}
\newcommand{\lectioiv}{\pars{Lectio IV.} \scriptura{Sermo 1 in natáli App. Petri et Pauli}

\noindent Sermo sancti Leónis Papæ.

\noindent Omnium quidem sanctárum solemnitátum, dilectíssimi, totus mundus est párticeps, et uníus fídei píetas éxigit, ut quidquid pro salúte universórum gestum recólitur, commúnibus ubíque gáudiis celebrétur. Verúmtamen hodiérna festívitas, præter illam reveréntiam quam toto terrárum orbe proméruit, speciáli et própria nostræ Urbis exsultatióne veneránda est; ut, ubi præcipuórum Apostolórum glorificátus est éxitus, ibi in die martýrii eórum sit lætítiæ principátus. Isti enim sunt viri, per quos tibi Evangélium Christi, Roma, resplénduit; et, quæ eras magístra erróris, facta es discípula veritátis.}
\newcommand{\lectiov}{\pars{Lectio V.}

\noindent Isti sunt patres tui veríque pastóres, qui te regnis cæléstibus inseréndam multo mélius multóque felícius condidérunt, quam illi quorum stúdio prima mœ́nium tuórum fundaménta locáta sunt; ex quibus is qui tibi nomen dedit, fratérna te cæde fœdávit. Isti sunt, qui te ad hanc glóriam provexérunt, ut gens sancta, pópulus eléctus, cívitas sacerdotális et régia, per sacram beáti Petri Sedem caput orbis effécta, látius præsidéres religióne divína quam dominatióne terréna. Quamvis enim, multis aucta victóriis, ius impérii tui terra maríque protúleris; minus tamen est quod tibi béllicus labor súbdidit, quam quod pax christiána subiécit.}
\newcommand{\lectiovi}{\pars{Lectio VI.}

\noindent Disposíto namque divínitus óperi máxime congruébat, ut multa regna uno confœderaréntur império, et cito pérvios habéret pópulos prædicátio generális, quos uníus tenéret régimen civitátis. Hæc autem cívitas ignórans suæ provectiónis auctórem, cum pene ómnibus dominarétur géntibus, ómnium géntium serviébat erróribus; et magnam sibi videbátur assumpsísse religiónem, quia nullam respúerat falsitátem. Unde, quantum erat per diábolum tenácius illigáta, tantum per Christum est mirabílius absolúta.}
\newcommand{\lectiovii}{\pars{Lectio VII.} \scriptura{Mt. 16, 13-19}

\noindent Léctio sancti Evangélii secúndum Matthǽum.

\noindent In illo témpore: Venit Iesus in partes Cæsaréæ Philíppi, et interrogábat discípulos suos, dicens: Quem dicunt hómines esse Fílium hóminis? Et réliqua.

\scriptura{Lib. 3 Comment. in Matth. cap. 16}

\noindent Homilía sancti Hierónymi Presbýteri.

\noindent Pulchre intérrogat: Quem dicunt hómines esse Fílium hóminis? quia qui de fílio hóminis loquúntur, hómines sunt; qui vero divinitátem eius intéllegunt, non hómines, sed dii appellántur. At illi dixérunt: Alii Ioánnem Baptístam, álii autem Elíam. Miror quosdam intérpretes causas errórum inquírere singulórum, et disputatiónem longíssimam téxere, quare Dóminum nostrum Iesum Christum álii Ioánnem putáverint, álii Elíam, álii Ieremíam aut unum ex prophétis; cum sic potúerint erráre in Elía et Ieremía, quo modo Heródes errávit in Ioánne, dicens: Quem ego decollávi Ioánnem, ipse surréxit a mórtuis, et virtútes operántur in eo.}
\newcommand{\lectioviii}{\pars{Lectio VIII.}

\noindent Vos autem quem me esse dícitis? Prudens lector, atténde quod, ex consequéntibus textúque sermónis, Apóstoli nequáquam hómines, sed dii appellántur. Cum enim dixísset: Quem dicunt hómines esse Fílium hóminis? subiécit: Vos autem quem me esse dícitis? Illis, quia hómines sunt, humána opinántibus, vos qui estis dii, quem me esse existimátis? Petrus ex persóna ómnium Apostolórum profitétur: Tu es Christus Fílius Dei vivi. Deum vivum appéllat, ad distinctiónem eórum deórum, qui putántur dii, sed mórtui sunt.}
\newcommand{\lectioix}{\pars{Lectio IX.}

\noindent Respóndens autem Iesus, dixit ei: Beátus es, Simon Bar-Iona. Testimónio de se Apóstoli reddit vicem. Petrus díxerat: Tu es Christus Fílius Dei vivi; mercédem recépit vera conféssio: Beátus es, Simon Bar-Iona. Quare? Quia non revelávit tibi caro et sanguis, sed revelávit Pater. Quod caro et sanguis reveláre non pótuit, Spíritus Sancti grátia revelátum est. Ergo ex confessióne sortítur vocábulum, quod revelatiónem ex Spíritu Sancto hábeat, cuius et fílius appellándus sit. Síquidem Bar-Iona in nostra lingua sonat Fílius colúmbæ.}
\include{apostolorum}
