\newcommand{\titulus}{\nomenFesti{Dominica VI Paschæ (V post Pascha).}}
\newcommand{\tedeummonasticum}{Monasticum}
\newcommand{\oratioLaudes}{\cuminitiali{}{temporalia/oratiod6.gtex}}
\newcommand{\oratio}{\pars{Oratio.}

\noindent Fac nos, omnípotens Deus, hos lætítiæ dies, quos in honórem Dómini resurgéntis exséquimur, afféctu sédulo celebráre, ut, quod recordatióne percúrrimus, semper in ópere teneámus.

\pars{Pro pace in universo mundo.} \scriptura{Sir. 50, 25; 2 Esdr. 4, 20; \textbf{H416}}

\vspace{-4mm}

\antiphona{II D}{temporalia/ant-dapacemdomine.gtex}

\vfill

\noindent Deus, a quo sancta desidéria, recta consília et iusta sunt ópera: da servis tuis illam, quam mundus dare non potest, pacem; ut et corda nostra mandátis tuis dédita, et hóstium subláta formídine, témpora sint tua protectióne tranquílla.

\noindent Per Dóminum nostrum Iesum Christum, Fílium tuum, qui tecum vivit et regnat in unitáte Spíritus Sancti, Deus, per ómnia sǽcula sæculórum.

\noindent \Rbardot{} Amen.}
\newcommand{\invitatorium}{\pars{Invitatorium.} \scriptura{Lc. 24, 34; Psalmus 94; \textbf{H232}}

\vspace{-6mm}

\antiphona{VI}{temporalia/inv-surrexitdominusvere.gtex}}
\newcommand{\hymnusmatutinum}{\pars{Hymnus.}

\vspace{-5mm}

\antiphona{III}{temporalia/hym-HicEstDies.gtex}}
\newcommand{\nocturnoi}{\pars{Psalmus 1.}

\vspace{-4mm}

\antiphona{II D}{temporalia/ant-alleluia-bv21-n1.gtex}

%\vspace{-2mm}

\scriptura{Ps. 1}

%\vspace{-2mm}

%\initiumpsalmi{temporalia/ps1-initium-ii-D-auto.gtex}
\initiumpsalmi{temporalia/ps1-initium-ii-D.gtex}

%\input{temporalia/ps1-ii-D.tex} \Abardot{}
\input{temporalia/ps1-viii-G.tex} \Abardot{}

\vfill
\pagebreak

\pars{Psalmus 2.}

\vspace{-4mm}

\antiphona{VIII c}{temporalia/ant-alleluia-bv21-n2.gtex}

%\vspace{-2mm}

\scriptura{Ps. 2}

\initiumpsalmi{temporalia/ps2-initium-viii-c-auto.gtex}

\input{temporalia/ps2-viii-c.tex} \Abardot{}

\vfill
\pagebreak

\pars{Psalmus 3.}

%\vspace{-4mm}

\antiphona{VIII G\textsuperscript{3}}{temporalia/ant-alleluia-bv21-n3.gtex}

%\vspace{-2mm}

\scriptura{Ps. 3}

%\initiumpsalmi{temporalia/ps3-initium-viii-G2-auto.gtex}
\initiumpsalmi{temporalia/ps3-initium-viii-G3.gtex}

\input{temporalia/ps3-viii-G2.tex} \Abardot{}

\vfill
\pagebreak}
\newcommand{\nocturnoii}{\vspace{-6mm}

\pars{Psalmus 4.} \scriptura{Mt. 28, 2; Mc. 16, 4; \textbf{H230}}

\vspace{-6mm}

\antiphona{VII a}{temporalia/ant-alleluialapisrevolutus.gtex}

\vspace{-1mm}

\scriptura{Ps. 103, 1-12}

\vspace{-2mm}

\initiumpsalmi{temporalia/ps103i-initium-vii-a-auto.gtex}

\vspace{-1.5mm}

\input{temporalia/ps103i-vii-a.tex} \Abardot{}

\vfill
\pagebreak

\pars{Psalmus 5.} \scriptura{Io. 20, 15.17}

\vspace{-4mm}

\antiphona{VIII G\textsuperscript{2}}{temporalia/ant-marianoliiamflere.gtex}

%\vspace{-2mm}

\scriptura{Ps. 103, 13-23}

\initiumpsalmi{temporalia/ps103ii-initium-viii-G5-auto.gtex}

\input{temporalia/ps103ii-viii-G5.tex} \Abardot{}

\vfill
\pagebreak

\pars{Psalmus 6.} \scriptura{Io. 20, 18; \textbf{H238}}

\vspace{-6mm}

\antiphona{VII a}{temporalia/ant-venitmarianuntians.gtex}

\vspace{-3mm}

\scriptura{Ps. 103, 24-35}

\vspace{-2mm}

\initiumpsalmi{temporalia/ps103iii-initium-vii-a-auto.gtex}

\vspace{-1.5mm}

\input{temporalia/ps103iii-vii-a.tex} \Abardot{}

\vfill
\pagebreak}
\newcommand{\nocturnoiii}{\pars{Cantica.}

\vspace{-4mm}

\antiphona{IV E}{temporalia/ant-veniteomnesadoremus.gtex}

%\vspace{-2mm}

\scriptura{Canticum Isaiæ, Is. 63, 1-5}

%\vspace{-2mm}

\initiumpsalmi{temporalia/isaiae12-initium-iv-E-auto.gtex}

\input{temporalia/isaiae12-iv-E.tex} \hfill \rubrica{Hic non dicitur antiphona.}

\vfill
\pagebreak

\scriptura{Canticum Oseæ, Os. 6, 1-6}

%\vspace{-2mm}

\initiumpsalmi{temporalia/oseae-initium-iv-E-auto.gtex}

\input{temporalia/oseae-iv-E.tex}

\vfill
\pagebreak

\scriptura{Canticum Sophoniæ, Soph. 3, 8-13}

%\vspace{-2mm}

\initiumpsalmi{temporalia/sophoniae-initium-iv-E-auto.gtex}

\input{temporalia/sophoniae-iv-E.tex}

\vfill
\pagebreak

\antiphona{}{temporalia/ant-veniteomnesadoremus.gtex}

\vfill
\pagebreak}
\newcommand{\matversusi}{\pars{Versus.}

\noindent \Vbardot{} Surréxit Dóminus de sepúlcro, allelúia.

\noindent \Rbardot{} Qui pro nobis pepéndit in ligno, allelúia.}
\newcommand{\matversusii}{\pars{Versus.}

\noindent \Vbardot{} Surréxit Dóminus vere, allelúia.

\noindent \Rbardot{} Et appáruit Simóni, allelúia.}
\newcommand{\lectioi}{\pars{Lectio I.} \scriptura{1 Io. 1, 1-4}

\noindent Incipit Epístola prima beáti Ioánnis apóstoli.

\noindent Quod fuit ab inítio, quod audívimus, quod vídimus óculis nostris, quod perspéximus, et manus nostræ contrectavérunt de verbo vitæ - et vita appáruit, et vídimus et testámur et annuntiámus vobis vitam ætérnam, quæ erat coram Patre et appáruit nobis - quod vídimus et audívimus, annuntiámus et vobis, ut et vos communiónem habeátis nobíscum.

\noindent Commúnio autem nostra est cum Patre et cum Fílio eius Iesu Christo. Et hæc scríbimus nos, ut gáudium nostrum sit plenum.}
\newcommand{\responsoriumi}{\pars{Responsorium 1.} \scriptura{\Rbardot{} Ps. 136, 5-6 \Vbardot{} ibid., 1; \textbf{H250}}

\vspace{-5mm}

\responsorium{VIII}{temporalia/resp-sioblitusfuero-CROCHU.gtex}{}}
\newcommand{\lectioii}{\pars{Lectio II.} \scriptura{1 Io. 1, 5-7}

\noindent Et hæc est annuntiátio, quam audívimus ab eo et annuntiámus vobis, quóniam Deus lux est, et ténebræ in eo non sunt ullæ.

\noindent Si dixérimus quóniam communiónem habémus cum eo et in ténebris ambulámus, mentímur et non fácimus veritátem; si autem in luce ambulémus, sicut ipse est in luce, communiónem habémus ad ínvicem, et sanguis Iesu Fílii eius mundat nos ab omni peccáto.}
\newcommand{\responsoriumii}{\pars{Responsorium 2.} \scriptura{\Rbardot{} Ps. 21, 23 \Vbardot{} ibid., 24; \textbf{H250}}

\vspace{-5mm}

\responsorium{II}{temporalia/resp-narrabonomentuum-CROCHU.gtex}{}}
\newcommand{\lectioiii}{\pars{Lectio III.} \scriptura{1 Io. 1, 8-10}

\noindent Si dixérimus quóniam peccátum non habémus, nosmetípsos sedúcimus, et véritas in nobis non est.

\noindent Si confiteámur peccáta nostra, fidélis est et iustus, ut remíttat nobis peccáta et emúndet nos ab omni iniustítia.

\noindent Si dixérimus quóniam non peccávimus, mendácem fácimus eum, et verbum eius non est in nobis.}
\newcommand{\responsoriumiii}{\pars{Responsorium 3.} \scriptura{\Rbardot{} Ps. 136, 3-4 \Vbardot{} ibid., 1; \textbf{H251}}

\vspace{-5mm}

\responsorium{VIII}{temporalia/resp-hymnumcantatenobis-CROCHU-cumdox.gtex}{}}
\newcommand{\lectioiv}{\pars{Lectio IV.} \scriptura{Cap. 5, 5-6, 2: PG 74, 942-943}

\noindent Ex Commentário sancti Cyrílli Alexandríni epíscopi in Epístolam secúndam ad Corínthios.

\noindent Qui Spíritus arrham habent, et spe resurrectiónis sunt prǽditi, hi rem exspectátam tamquam præséntem tenéntes, aiunt se néminem abhinc agnóscere secúndum carnem: cuncti enim spiritáles sumus, et carnáli corruptióne aliéni.

\noindent Etenim illucescénte nobis Unigénito, in ipsum quod ómnia vivíficat Verbum transformámur.

\noindent Nempe sícuti mortis vínculis erámus obstrícti regnánte peccáto, ita Christi subintránte iustítia corruptélam abiécimus.}
\newcommand{\responsoriumiv}{\pars{Responsorium 4.} \scriptura{Sap. 8, 18; \textbf{H251}}

\vspace{-5mm}

\responsorium{III}{temporalia/resp-alleluiadelectatiobona-CROCHU.gtex}{}}
\newcommand{\lectiov}{\pars{Lectio V.}

\noindent Nemo ítaque iam est in carne, id est in infirmitáte carnáli, cuiúsmodi iure meritóque corrúptio inter cétera intellegénda est, addit: \emph{Nam etiámsi novérimus secúndum carnem Christum, nunc tamen iam non nóvimus.}

\noindent Tamquam si dícere vellet: \emph{Verbum caro factum est et habitávit in nobis,} et pro nostrum ómnium vita mortem secúndum carnem súbiit, atque ita ipsum nóvimus; verúmtamen abhinc iam non agnóscimus.

\noindent Etiámsi enim carnem rétinet, quippe qui tértia die revíxit, et apud Patrem in cælis versátur, nihilóminus supra carnem esse intellégitur: \emph{semel enim mórtuus, iam non móritur, mors illi ultérius non dominábitur.}

\noindent \emph{Quod enim mórtuus est, peccáto semel est mórtuus; quod autem vivit, Deo vivit.}}
\newcommand{\responsoriumv}{\pars{Responsorium 5.} \scriptura{\Rbardot{} Ps. 67, 5 \Vbardot{} ibid., 35; \textbf{H252}}

\vspace{-5mm}

\responsorium{I}{temporalia/resp-cantatedeoalleluia-CROCHU.gtex}{}}
\newcommand{\lectiovi}{\pars{Lectio VI.}

\noindent Igitur si huiuscémodi ille est qui se fecit vitæ nobis antesignánum, prorsus opórtet nos quoque vestígiis eius insisténtes, non tam in carne quam supra carnem esse reputári.

\noindent Ergo rectíssime divus Paulus: \emph{Si qua in Christo,} inquit, \emph{nova creatúra, vétera transiérunt, ecce nova facta sunt.}

\noindent Iustificáti enim fúimus per fidem in Christo, et maledictiónis vis désiit.

\noindent Quandóquidem ille nostri grátia revíxit, calcáta mortis poténtia; verúmque et suápte natúra Deum agnóvimus, cui in spíritu et veritáte cultum impéndimus, mediatóre Fílio, qui supérnas a Patre benedictiónes mundo impertítur.

\noindent Quámobrem sapientíssime divus Paulus: \emph{Omnia autem}, inquit, \emph{ex Deo qui sibi nos reconciliávit per Christum.}

\noindent Reápse enim haud præter Patris voluntátem est incarnatiónis mystérium et consectánea huic renovátio.

\noindent Quippe per Christum nacti áditum sumus cum nemo ad Patrem véniat, ut ípsemet ait: nisi per ipsum.

\noindent Igitur \emph{ómnia ex Deo sunt, qui nos per Christum reconciliávit et ministérium reconciliatiónis attríbuit.}}
\newcommand{\responsoriumvi}{\pars{Responsorium 6.} \scriptura{\Rbardot{} Ps. 118, 35-36 \Vbardot{} ibid., 37; \textbf{H251}}

\vspace{-5mm}

\responsorium{I}{temporalia/resp-deducmeinsemitam-CROCHU-cumdox.gtex}{}}
\newcommand{\evangelium}{\pars{Versus.}

\noindent \Vbardot{} Gavísi sunt discípuli, allelúia.

\noindent \Rbardot{} Viso Dómino, allelúia.

\vspace{5mm}

\sineinitiali{temporalia/oratiodominica-mat.gtex}

\vspace{5mm}

\pars{Absolutio.}

\cuminitiali{}{temporalia/absolutio-avinculis.gtex}

\vfill
\pagebreak

\cuminitiali{}{temporalia/benedictio-solemn-evangelica.gtex}

\vspace{7mm}

\pars{Evangelium} \scriptura{Io. 14, 15-21}

\noindent Léctio sancti Evangélii secúndum Ioánnem.

\noindent In illo témpore: Dixit Iesus discípulis suis:

\noindent «Si dilígitis me, mandáta mea servábitis; et ego rogábo Patrem, et álium Paráclitum dabit vobis, ut máneat vobíscum in ætérnum, Spíritum veritátis, quem mundus non potest accípere, quia non videt eum nec cognóscit. Vos cognóscitis eum, quia apud vos manet; et in vobis erit.

\noindent Non relínquam vos órphanos; vénio ad vos.

\noindent Adhuc módicum, et mundus me iam non videt; vos autem vidétis me, quia ego vivo et vos vivétis.

\noindent In illo die vos cognoscétis quia ego sum in Patre meo, et vos in me, et ego in vobis.

\noindent Qui habet mandáta mea et servat ea, ille est, qui díligit me; qui autem díligit me, diligétur a Patre meo, et ego díligam eum et manifestábo ei meípsum».

\vspace{5mm}

\scriptura{Tract. 74,1: CCL 36, 512-513}

\noindent Ex Tractátibus sancti Augustíni epíscopi in Ioánnem.

\noindent \emph{Si dilígitis me, mandáta mea serváte; ego rogábo Patrem, et álium Paraclétum dabit vobis, ut máneat vobíscum in ætérnum, Spíritum veritátis.} Hic est útique in Trinitáte Spíritus Sanctus quem Patri et Fílio consubstantiálem et coætérnum fides cathólica confitétur; ipse est de quo dicit Apóstolus: \emph{Cáritas Dei diffúsa est in córdibus nostris per Spíritum Sanctum qui datus est nobis.} Quómodo ergo Dóminus dicit: \emph{Si dilígitis me, mandáta mea serváte; et ego rogábo Patrem et álium Paraclétum dabit vobis,} cum dicat de Spíritu Sancto quem, nisi habeámus, nec dilígere Deum póssumus nec eius mandáta serváre?

\noindent {\color{gray} Quómodo dilígimus ut eum accipiámus quem, nisi habeámus, dilígere non valémus? Aut quómodo mandáta servábimus ut eum accipiámus quem, nisi habeámus, mandáta serváre non póssumus? An forte præcédit in nobis cáritas qua dilígimus Christum, ut diligéndo Christum eiúsque mandáta faciéndo, mereámur accípere Spíritum Sanctum, ut cáritas non Christi, quæ iam præcésserat, sed Dei Patris diffundátur in córdibus nostris per Spíritum Sanctum qui datus est nobis? Pervérsa est ista senténtia.

\noindent Qui enim se Fílium dilígere credit et Patrem non díligit, profécto nec Fílium díligit, sed quod sibi ipse confínxit. Deínde apostólica vox est: \emph{Nemo dicit: «Dóminus Iesus», nisi in Spíritu Sancto}; et quis Dóminum Iesum, nisi qui eum díligit, dicit, si eo modo dicit quo Apóstolus intéllegi vóluit? Multi enim voce dicunt, corde autem et factis negant; sicut de tálibus ait: \emph{Confiténtur enim se nosse Deum, factis autem negant.} Si negátur factis, procul dúbio étiam dícitur factis.}

\noindent \emph{Nemo ítaque dicit: «Dóminus Iesus», nisi in Spíritu Sancto}; et nemo sic dicit, nisi qui díligit. Iam ítaque apóstoli dicébant: «Dóminus Iesus»; et si eo modo dicébant, ut non ficte dícerent, ore confiténtes, corde et factis negántes; prorsus si veráciter hoc dicébant, procul dúbio diligébant. Quómodo ígitur diligébant, nisi in Spíritu Sancto? Et tamen eis prius imperátur ut díligant eum, et eius mandáta consérvent, ut accípiant Spíritum Sanctum: quem, nisi habérent, profécto dilígere et mandáta serváre non possent.

\vfill
\pagebreak

\pars{Responsorium 7.} \scriptura{\Rbardot{} Os, 81m 2 \Vbardot{} ibid., 3; \textbf{H252}}

\vspace{-5mm}

\responsorium{VII}{temporalia/resp-bonumestconfiteri-CROCHU-cumdox.gtex}{}

\vfill

\rubrica{vel ad libitum:}

\vspace{3mm}

\pars{Responsorium 7.} \scriptura{\Rbardot{} Io. 14, 1; 16, 16; 15, 26; 14, 26 \Vbardot{} ibid., 16, 7; \textbf{H264}}

\vspace{-5mm}

\responsorium{IV}{temporalia/resp-nonconturbeturetmittam-CROCHU-cumdox.gtex}{}

\vfill
\pagebreak}
\newcommand{\laudes}{\pars{Hymnus}

\cuminitiali{VIII}{temporalia/hym-AuroraLucis-alt.gtex}

\vfill
\pagebreak

\pars{Psalmus 1.} \scriptura{Ps. 117, 15}

\vspace{-4mm}

\antiphona{I f}{temporalia/ant-intabernaculisiustorumvox.gtex}

%\vspace{-2mm}

\scriptura{Psalmus 117}

%\vspace{-2mm}

\initiumpsalmi{temporalia/ps117-initium-i-f-auto.gtex}

%\vspace{-1.5mm}

\input{temporalia/ps117-i-f.tex}

\vfill

\antiphona{}{temporalia/ant-intabernaculisiustorumvox.gtex}

\vfill
\pagebreak

\pars{Psalmus 2.} \scriptura{Dan. 3, 56; \textbf{H138}}

\vspace{-4mm}

\antiphona{I a\textsuperscript{2}}{temporalia/ant-benedictusesinfirmamento-tp.gtex}

%\vspace{-2mm}

\scriptura{Canticum Danielis, Dan. 3, 52-57}

%\vspace{-3mm}

\initiumpsalmi{temporalia/dan33-initium-i-a2-auto.gtex}

\input{temporalia/dan33-i-a2.tex} \Abardot{}

\vfill
\pagebreak

\pars{Psalmus 3.} \scriptura{Ap. 4, 9}

\vspace{-4mm}

\antiphona{I a\textsuperscript{2}}{temporalia/ant-gloriaethonoretbenedictio.gtex}

\scriptura{Psalmus 150}

\initiumpsalmi{temporalia/ps150-initium-i-a2-auto.gtex}

\input{temporalia/ps150-i-a2.tex} \Abardot{}

\vfill
\pagebreak
}
\newcommand{\lectiobrevis}{\pars{Lectio brevis.} \scriptura{Ac. 10, 40-43}

\noindent Deus suscitávit Iesum tértia die et dedit eum maniféstum fíeri, non omni pópulo sed téstibus præordinátis a Deo, nobis, qui manducávimus et bíbimus cum illo postquam resurréxit a mórtuis; et præcépit nobis prædicáre pópulo et testificári quia ipse est, qui constitútus est a Deo iudex vivórum et mortuórum. Huic omnes prophétæ testimónium pérhibent remissiónem peccatórum accípere per nomen eius omnes, qui credunt in eum.}
\newcommand{\responsoriumbreve}{\pars{Responsorium breve.}

\cuminitiali{VI}{temporalia/resp-christefilideivivi-tp.gtex}}
\newcommand{\benedictus}{\pars{Canticum Zachariæ.} \scriptura{Io. 14, 16; \textbf{H266}}

\vspace{-4mm}

\antiphona{I g}{temporalia/ant-rogabopatremmeum.gtex}

\vspace{-2mm}

\scriptura{Lc. 1, 68-79}

\vspace{-2mm}

\cantusSineNeumas
\initiumpsalmi{temporalia/benedictus-initium-isoll-g-auto.gtex}

%\vspace{-1.5mm}

\input{temporalia/benedictus-isoll-g.tex} \Abardot{}}
\newcommand{\preces}{\noindent Deum Patrem omnipoténtem,~\gredagger{} qui Iesum, príncipem et salvatórem nostrum, suscitávit,~\grestar{} invocémus clamántes:

\Rbardot{} Claritáte Christi clarífica nos, Dómine.

\noindent Pater sancte,~\gredagger{} qui Iesum, diléctum tuum, de ténebris mortis ad lumen glóriæ tuæ transíre fecísti,~\grestar{} da nobis in admirábile lumen tuum veníre.

\Rbardot{} Claritáte Christi clarífica nos, Dómine.

\noindent Qui nos salvásti per fidem,~\grestar{} in fide baptísmatis nostri fac ut hódie vivámus.

\Rbardot{} Claritáte Christi clarífica nos, Dómine.

\noindent Tu, qui mandas ut quæ sursum sunt quærámus,~\gredagger{} ubi Christus est in déxtera tua sedens,~\grestar{} serva nos a peccáti blandítiis.

\Rbardot{} Claritáte Christi clarífica nos, Dómine.

\noindent Vita nostra, in te abscóndita cum Christo, lúceat in mundo,~\grestar{} ut cælum novum et terra nova prænuntiéntur.

\Rbardot{} Claritáte Christi clarífica nos, Dómine.}
\newcommand{\magnificatii}{\pars{Canticum B. Mariæ V.} \scriptura{Io. 1, 41}

\vspace{-6mm}

{
\grechangedim{interwordspacetext}{0.18 cm plus 0.15 cm minus 0.05 cm}{scalable}%
\antiphona{I d\textsuperscript{3}}{temporalia/ant-ambulansiesus.gtex}
\grechangedim{interwordspacetext}{0.22 cm plus 0.15 cm minus 0.05 cm}{scalable}%
}

\vspace{-1.5mm}

\scriptura{Lc. 1, 46-55}

\vspace{-2.5mm}

\cantusSineNeumas
\initiumpsalmi{temporalia/magnificat-initium-isoll-d3.gtex}

\vspace{-1.5mm}

\input{temporalia/magnificat-isoll-d3.tex} \Abardot{}}
%\newcommand{\hebdomada}{infra Hebdom. V post Pentecosten.}
\newcommand{\oratioLaudes}{\cuminitiali{}{temporalia/oratio5.gtex}}

% LuaLaTeX

\documentclass[a4paper, twoside, 12pt]{article}
\usepackage[latin]{babel}
%\usepackage[landscape, left=3cm, right=1.5cm, top=2cm, bottom=1cm]{geometry} % okraje stranky
%\usepackage[landscape, a4paper, mag=1166, truedimen, left=2cm, right=1.5cm, top=1.6cm, bottom=0.95cm]{geometry} % okraje stranky
\usepackage[landscape, a4paper, mag=1400, truedimen, left=0.5cm, right=0.5cm, top=0.5cm, bottom=0.5cm]{geometry} % okraje stranky

\usepackage{fontspec}
\setmainfont[FeatureFile={junicode.fea}, Ligatures={Common, TeX}, RawFeature=+fixi]{Junicode}
%\setmainfont{Junicode}

% shortcut for Junicode without ligatures (for the Czech texts)
\newfontfamily\nlfont[FeatureFile={junicode.fea}, Ligatures={Common, TeX}, RawFeature=+fixi]{Junicode}

\usepackage{multicol}
\usepackage{color}
\usepackage{lettrine}
\usepackage{fancyhdr}

% usual packages loading:
\usepackage{luatextra}
\usepackage{graphicx} % support the \includegraphics command and options
\usepackage{gregoriotex} % for gregorio score inclusion
\usepackage{gregoriosyms}
\usepackage{wrapfig} % figures wrapped by the text
\usepackage{parcolumns}
\usepackage[contents={},opacity=1,scale=1,color=black]{background}
\usepackage{tikzpagenodes}
\usepackage{calc}
\usepackage{longtable}
\usetikzlibrary{calc}

\setlength{\headheight}{14.5pt}

\input{conventuscommune.tex} % Often used macros
%%%% Preklady jednotlivych zpevu (nektere se opakuji, a je dobre mit je
% vsechny na jedne hromade)

% HOURS ---

\newcommand{\trAntI}{\translatioCantus{Muž boží měl kožený toulec, pečlivě
zavázaný, jenž mu visel na šíji a~často se ho dotýkal.}}

\newcommand{\trAntII}{\translatioCantus{Klíč od~něho tak dobře střežil, že
dokud žil v~těle, nikdo z~jeho žáků nezvěděl, co je uvnitř.}}

\newcommand{\trAntIII}{\translatioCantus{Ale když se odebral z~tohoto
života, schránku otevřeli a~objevili v~ní žíněné roucho a~měděný řetěz
potřísněný krví.}}

\newcommand{\trAntIV}{\translatioCantus{A když prohlédli mistrovo tělo,
nalezli jeho tělo na čtyřech místech hluboce zbrázděno ranami od řetězu.}}

\newcommand{\trAntV}{\translatioCantus{Krev vytékající z~těch ran, místy
prostoupila i~žíněným rouchem.}}

\newcommand{\trCapituli}{\translatioCantus{
Miláčkovi Boha a~lidí,
Mojžíšovi požehnané paměti,~\gredagger{}
dopřál slávu rovnou slávě svatých~\grestar{}
učinil ho mocným na postrach nepřátelům
a~jeho slovy zastavil divy.}}

\newcommand{\trLectioBrevis}{\translatioCantus{
Pamatujte na své představené,
kteří vám hlásali Boží slovo.
Uvažte, jak oni skončili život, a~napodobujte jejich víru.
Ježíš Kristus je stejný včera i~dnes i~navěky.
Nenechte se svést věelijakými cizími naukami.}}

\newcommand{\trRespLaud}{\translatioCantus{Spravedlivého vodil Hospodin~\grestar{}
po přímých stezkách. \Vbardot{} A~ukázal mu Boží království.}}

\newcommand{\trRespLaudB}{\translatioCantus{Na tvých hradbách, Jeruzaléme,
ustanovil jsem strážné;~\grestar{}
budou bdít nad mým lidem. \Vbardot{} Ani ve dne, ani v~noci nesmějí nikdy
mlčet.}}

\newcommand{\trVersus}{\translatioCantus{\Vbardot{} Ústa spravedlivého šeptají moudrost, aleluja.
\Rbardot{} A~jeho jazyk ohlašuje právo, aleluja.}}

\newcommand{\trAntBenedictus}{\translatioCantus{Když na bujné oře vložili
nosítka a~sňali jim uzdu, vydali se přímo k~cele božího muže.}}

\newcommand{\trPreces}{\translatioCantus{
\noindent S vděčností chvalme Krista, dobrého Pastýře, \gredagger{} který dal život za své ovce, \grestar{} a~pokorně ho prosme: \Rbardot{} Pane, buď pastýřem svého lidu.

\noindent Kriste, ty dáváš církvi pastýře, a~jejich službou se ujímáš svého lidu, \grestar{} dej, ať v~lásce těch, kteří nás vedou, poznáváme, jak nás miluješ. \Rbardot{} Pane, buď pastýřem svého lidu.

\noindent Ty stále konáš skrze své zástupce službu pastýře a~učitele, \grestar{} nepřestávej nás nikdy vést prostřednictvím svých služebníků. \Rbardot{} Pane, buď pastýřem svého lidu.

\noindent Ty prokazuješ svému lidu skrze jeho pastýře službu lékaře duše i~těla, \grestar{} ochraňuj náš život a~veď nás ke svatosti. \Rbardot{} Pane, buď pastýřem svého lidu.

\noindent Ty posíláš své svaté, aby slovem i~příkladem vedli tvůj lid k~tobě, \grestar{} na jejich přímluvu nás posiluj, abychom vytrvali na cestě, která vede k~věčnému životu. \Rbardot{} Pane, buď pastýřem svého lidu.}}

\newcommand{\trOrationis}{\translatioCantus{Bože, jenž nám dopřáváš radovat
se z~výroční slavnosti svatého tvého vyznavače Havla, uděl dobrotivě,
abychom když slavíme jeho narození, též se řídili podobou jeho skutků.
Skrze…}}
 % Czech translations of the proper texts

\newcommand{\annusEditionis}{2020}

%%%% Vicekrat opakovane kousky

\newcommand{\anteOrationem}{
  \rubrica{Ante Orationem, cantatur a Superiore:}

  \pars{Supplicatio Litaniæ.}

  \cuminitiali{}{temporalia/supplicatiolitaniae.gtex}

  \pars{Oratio Dominica.}

  \cuminitiali{}{temporalia/oratiodominica.gtex}

  \rubrica{Deinde dicitur ab Hebdomadario:}

  \cuminitiali{}{temporalia/dominusvobiscum-solemnis.gtex}

  \rubrica{In choro monialium loco Dominus vobiscum dicitur:}

  \sineinitiali{temporalia/domineexaudi.gtex}
}

\setlength{\columnsep}{30pt} % prostor mezi sloupci

%%%%%%%%%%%%%%%%%%%%%%%%%%%%%%%%%%%%%%%%%%%%%%%%%%%%%%%%%%%%%%%%%%%%%%%%%%%%%%%%%%%%%%%%%%%%%%%%%%%%%%%%%%%%%
\begin{document}

% Here we set the space around the initial.
% Please report to http://home.gna.org/gregorio/gregoriotex/details for more details and options
\grechangedim{afterinitialshift}{2.2mm}{scalable}
\grechangedim{beforeinitialshift}{2.2mm}{scalable}
\grechangedim{interwordspacetext}{0.22 cm plus 0.15 cm minus 0.05 cm}{scalable}%
\grechangedim{annotationraise}{-0.2cm}{scalable}

% Here we set the initial font. Change 38 if you want a bigger initial.
% Emit the initials in red.
\grechangestyle{initial}{\color{red}\fontsize{38}{38}\selectfont}

\pagestyle{empty}

%%%% Titulni stranka
\begin{titulusOfficii}
\titulus{}
\end{titulusOfficii}

% graphic
%\vspace{1.5cm}
%\begin{center}
%\includegraphics[width=8cm]{emmaus.jpg}
%\end{center}

\vfill

\begin{center}
%Ad usum et secundum consuetudines chori \guillemotright{}Conventus Choralis\guillemotleft.

%Editio Sancti Wolfgangi \annusEditionis
\end{center}

\pagebreak

\renewcommand{\headrulewidth}{0pt} % no horiz. rule at the header
\fancyhf{}
\pagestyle{fancy}

\pars{Oratio ante divinum Officium.}

\lettrine{{\color{red}A}}{peri,} Dómine, os meum ad benedicéndum nomen sanctum tuum:
munda quoque cor meum ab ómnibus vanis, pervérsis, et aliénis
cogitatiónibus:
intelléctum illúmina, afféctum inflámma,
ut digne, atténte ac devóte hoc Offícium recitáre váleam,
et exaudíri mérear ante conspéctum Divínæ Maiestátis tuæ.
Per Christum, Dóminum nostrum.
\Rbardot{} Amen.

Dómine, in unióne illíus divínæ intentiónis,
qua ipse in terris laudes Deo persolvísti,
has tibi Horas \rubricatum{(vel \textnormal{hanc tibi Horam})} persólvo.

%\trOratioAnteOfficium

\vfill

\pars{Oratio post divinum Officium.}

\rubrica{
  Orationem sequentem devote post Officium recitantibus
  Leo Papa X. defectus, et culpas in eo persolvendo ex humana
  fragilitate contractas, indulsit, et dicitur flexis genibus.
}

\lettrine{{\color{red}S}}{acrosánctæ} et indivíduæ Trinitáti,
crucifíxi Dómini nostri Iesu Christi humanitáti,
beatíssimæ et gloriosíssimæ sempérque Vírginis Maríæ
fecúndæ integritáti, 
et ómnium Sanctórum universitáti
sit sempitérna laus, honor, virtus et glória
ab omni creatúra,
nobísque remíssio ómnium peccatórum,
per infiníta sǽcula sæculórum.
\Rbardot{} Amen.

\noindent \Vbardot{} Beáta víscera Maríæ Virginis, quæ portavérunt
ætérni Patris Fílium.\\
\Rbardot{} Et beáta úbera, quæ lactavérunt Christum Dominum.

\rubrica{Et dicitur secreto \textnormal{Pater noster.} et \textnormal{Ave María.}}

%\trOratioPostOfficium

\vfill

\hora{Ad I. Vesperas.} %%%%%%%%%%%%%%%%%%%%%%%%%%%%%%%%%%%%%%%%%%%%%%%%%%%%%
%\sideThumbs{I. Vesperæ}

\cantusSineNeumas

\vspace{0.5cm}
\grechangedim{interwordspacetext}{0.18 cm plus 0.15 cm minus 0.05 cm}{scalable}%
\cuminitiali{}{temporalia/deusinadiutorium-solemnis.gtex}
\grechangedim{interwordspacetext}{0.22 cm plus 0.15 cm minus 0.05 cm}{scalable}%

\vfill
\pagebreak

\pars{Psalmus 1.} \scriptura{Ps. 144, 13; \textbf{H100}}

\vspace{-4mm}

\antiphona{VII c\textsuperscript{2}}{temporalia/ant-regnumtuum.gtex}

\scriptura{Psalmus 144, 10-21.}

\initiumpsalmi{temporalia/ps144ii-initium-vii-c2-auto.gtex}

%\psalmusEtTranslatioT{temporalia/ps144ii-VII-comb.tex}{10cm}
\input{temporalia/ps144ii-VII.tex} \Abardot{}

\vspace{-1cm}

\vfill
\pagebreak

\pars{Psalmus 2.} \scriptura{Ps. 145, 2; \textbf{H100}}

\vspace{-4mm}

\antiphona{IV E}{temporalia/ant-laudabodeum.gtex}

\scriptura{Psalmus 145.}

\initiumpsalmi{temporalia/ps145-initium-iv-E-auto.gtex}

%\psalmusEtTranslatioT{temporalia/ps145-VII-comb.tex}{10cm}
\input{temporalia/ps145-VII.tex} \Abardot{}

\vfill
\pagebreak

\pars{Psalmus 3.} \scriptura{Ps. 146, 1; \textbf{H101}}

\vspace{-4mm}

\antiphona{VIII a}{temporalia/ant-deonostro.gtex}

\scriptura{Psalmus 146.}

\initiumpsalmi{temporalia/ps146-initium-viii-A-auto.gtex}

%\psalmusEtTranslatioT{temporalia/ps146-VII-comb.tex}{10cm}
\input{temporalia/ps146-VII.tex} \Abardot{}

\vfill
\pagebreak

\pars{Psalmus 4.} \scriptura{Ps. 147, 1}

\vspace{-4mm}

\antiphona{E}{temporalia/ant-laudajerusalem.gtex}

\scriptura{Psalmus 147.}

\initiumpsalmi{temporalia/ps147-initium-e-auto.gtex}

%\psalmusEtTranslatioT{temporalia/ps147-VII-comb.tex}{10cm}
\input{temporalia/ps147-VII.tex} \Abardot{}

\vfill
\pagebreak

\pars{Capitulum.} \scriptura{Rom. 11, 33}

\grechangedim{interwordspacetext}{0.12 cm plus 0.15 cm minus 0.05 cm}{scalable}%
\cuminitiali{}{temporalia/capitulum-OAltitudo.gtex}
\grechangedim{interwordspacetext}{0.22 cm plus 0.15 cm minus 0.05 cm}{scalable}

% preklad Jeruz. bible
%\trCapituliI

\vfill

\pars{Responsorium breve.} \scriptura{Ps. 146, 5}

\cuminitiali{VI}{temporalia/resp-magnusdominusnoster.gtex}

%\trResp

\vfill
\pagebreak

\pars{Hymnus} \scriptura{Ambrosius (\olddag{} 397)}

\cuminitiali{I}{temporalia/hym-OLuxBeata-aestivalis.gtex}
\vspace{-3mm}
%\input{hym-OLuxBeata-bohtext.tex}

\vfill
%\pagebreak

\pars{Versus.}

% Versus. %%%
\sineinitiali{temporalia/versus-vespertina.gtex}

%\noindent \trVersus

\vfill
\pagebreak

\magnificati

\vfill
\pagebreak

%\sideThumbs{{\scriptsize{}Fine horarum}}

\anteOrationem

\pagebreak

% Oratio. %%%
\oratioLaudes

\vspace{-1mm}
%\trOrationisI

\vfill

\rubrica{Hebdomadarius dicit iterum Dominus vobiscum, vel cantor dicit:}

\vspace{2mm}

\sineinitiali{temporalia/domineexaudi.gtex}

\rubrica{Postea cantatur a cantore:}

\vspace{2mm}

\cuminitiali{I}{temporalia/benedicamus-dominica-perannum.gtex}

\vspace{1mm}

\vfill
\pagebreak

\hora{Ad Matutinum.} %%%%%%%%%%%%%%%%%%%%%%%%%%%%%%%%%%%%%%%%%%%%%%%%%%%%%
%\sideThumbs{Matutinum}

\vspace{2mm}

\cuminitiali{}{temporalia/dominelabiamea.gtex}

\vspace{2mm}

\pars{Invitatorium.} \scriptura{Ps. 94, 1; Psalmus 94}

\vspace{-6mm}

\antiphona{E}{temporalia/inv-veniteexsultemus.gtex}

\vfill
\pagebreak

\pars{Hymnus.} \scriptura{Adamus Sancti Victoris (\olddag 1146)}

\vspace{-5mm}

\antiphona{VII}{temporalia/hym-SalveDies.gtex}

\scriptura{Non dicitur \textnormal{Amen} in fine.}
%{
%\vspace{-5mm}
%\setlength{\columnsep}{0pt} % prostor mezi sloupci
%\input{hym-SalveDies-bohtext.tex}
%\setlength{\columnsep}{30pt} % prostor mezi sloupci
%}

\vfill
\pagebreak

\subhora{In I. Nocturno}

\pars{Psalmus 1.} \scriptura{Ps. 1, 1}

\vspace{-4mm}

\antiphona{VIII G}{temporalia/ant-beatusvir.gtex}

%\vspace{-5mm}

\scriptura{Ps. 1}

%\vspace{-2mm}

\initiumpsalmi{temporalia/ps1-initium-viii-G-auto.gtex}

%\psalmusEtTranslatioT{temporalia/ps1-I-comb.tex}{10cm}
\input{temporalia/ps1-I.tex} \Abardot{}

\vfill
\pagebreak

\pars{Psalmus 2.} \scriptura{Ps. 2, 11; \textbf{H93}}

\vspace{-4mm}

\antiphona{VII a}{temporalia/ant-servitedomino.gtex}

\vspace{-3mm}

\scriptura{Ps. 2}

\vspace{-2mm}

\initiumpsalmi{temporalia/ps2-initium-vii-a-auto.gtex}

%\psalmusEtTranslatioT{temporalia/ps2-I-comb.tex}{10cm}
\input{temporalia/ps2-I.tex} \Abardot{}

\vfill
\pagebreak

\pars{Psalmus 3.} \scriptura{Ps. 3, 7}

\vspace{-4mm}

\antiphona{VI F}{temporalia/ant-exsurgedominesalvum.gtex}

%\vspace{-5mm}

\scriptura{Ps. 3}

\initiumpsalmi{temporalia/ps3-initium-vi-F-auto.gtex}

%\psalmusEtTranslatioT{temporalia/ps3-I-comb.tex}{10cm}
\input{temporalia/ps3-I.tex} \Abardot{}

\vfill
\pagebreak

\pars{Versus.} \scriptura{Ps. 118, 55}

% Versus. %%%
\sineinitiali{temporalia/versus-memorfui.gtex}

\vspace{5mm}

\sineinitiali{temporalia/oratiodominica-mat.gtex}

\vspace{5mm}

\pars{Absolutio.}

\cuminitiali{}{temporalia/absolutio-exaudi.gtex}

\vfill
\pagebreak

\cuminitiali{}{temporalia/benedictio-solemn-benedictione.gtex}

\vspace{7mm}

\lectioi

\noindent \Vbardot{} Tu autem, Dómine, miserére nobis.
\noindent \Rbardot{} Deo grátias.

\vfill
\pagebreak

\responsoriumi

\vfill
\pagebreak

\cuminitiali{}{temporalia/benedictio-solemn-unigenitus.gtex}

\vspace{7mm}

\lectioii

\noindent \Vbardot{} Tu autem, Dómine, miserére nobis.
\noindent \Rbardot{} Deo grátias.

\vfill
\pagebreak

\responsoriumii

\vfill
\pagebreak

\cuminitiali{}{temporalia/benedictio-solemn-spiritus.gtex}

\vspace{7mm}

\lectioiii

\noindent \Vbardot{} Tu autem, Dómine, miserére nobis.
\noindent \Rbardot{} Deo grátias.

\vfill
\pagebreak

\responsoriumiii

\vfill
\pagebreak

\subhora{In II. Nocturno}

\pars{Psalmus 4.} \scriptura{Ps. 8, 2}

\vspace{-4mm}

\antiphona{I g}{temporalia/ant-quamadmirabileest.gtex}

%\vspace{-5mm}

\scriptura{Ps. 8}

%A\vspace{-2mm}

\initiumpsalmi{temporalia/ps8-initium-i-g-auto.gtex}

%\psalmusEtTranslatioT{temporalia/ps8-I-comb.tex}{10cm}
\input{temporalia/ps8-I.tex} \Abardot{}

\vfill
\pagebreak

\pars{Psalmus 5.} \scriptura{Ps. 9, 5}

\vspace{-4mm}

\antiphona{VIII G}{temporalia/ant-sedistisuperthronum.gtex}

%\vspace{-5mm}

\scriptura{Ps. 9, 2-11}

\initiumpsalmi{temporalia/ps9ii_xi-initium-viii-G-auto.gtex}

%\psalmusEtTranslatioT{temporalia/ps9ii_xi-I-comb.tex}{10cm}
\input{temporalia/ps9ii_xi-I.tex} \Abardot{}

\vfill
\pagebreak

\pars{Psalmus 6.} \scriptura{Ps. 9, 20}

\vspace{-4mm}

\antiphona{I g\textsuperscript{3}}{temporalia/ant-exsurgedominenon.gtex}

%\vspace{-5mm}

\scriptura{Ps. 9, 12-21}

\initiumpsalmi{temporalia/ps9xii_xxi-initium-i-g3-auto.gtex}

%\psalmusEtTranslatioT{temporalia/ps9xii_xxi-I-comb.tex}{10cm}
\input{temporalia/ps9xii_xxi-I.tex} \Abardot{}

\vfill
\pagebreak

\pars{Versus.} \scriptura{Ps. 118, 62}

% Versus. %%%
\sineinitiali{temporalia/versus-medianocte.gtex}

\vspace{5mm}

\sineinitiali{temporalia/oratiodominica-mat.gtex}

\vspace{5mm}

\pars{Absolutio.}

\cuminitiali{}{temporalia/absolutio-ipsius.gtex}

\vfill
\pagebreak

\cuminitiali{}{temporalia/benedictio-solemn-deus.gtex}

\vspace{7mm}

\lectioiv

\noindent \Vbardot{} Tu autem, Dómine, miserére nobis.
\noindent \Rbardot{} Deo grátias.

\vfill
\pagebreak

\responsoriumiv

\vfill
\pagebreak

\cuminitiali{}{temporalia/benedictio-solemn-christus.gtex}

\vspace{7mm}

\lectiov

\noindent \Vbardot{} Tu autem, Dómine, miserére nobis.
\noindent \Rbardot{} Deo grátias.

\vfill
\pagebreak

\responsoriumv

\vfill
\pagebreak

\cuminitiali{}{temporalia/benedictio-solemn-ignem.gtex}

\vspace{7mm}

\lectiovi

\noindent \Vbardot{} Tu autem, Dómine, miserére nobis.
\noindent \Rbardot{} Deo grátias.

\vfill
\pagebreak

\responsoriumvi

\vfill
\pagebreak

\subhora{In III. Nocturno}

\pars{Psalmus 7.} \scriptura{Ps. 9, 22}

\vspace{-4mm}

\antiphona{II D}{temporalia/ant-utquiddomine.gtex}

\vspace{-4mm}

\scriptura{Ps. 9, 22-32}

%\vspace{-2mm}

\initiumpsalmi{temporalia/ps9xxii_xxxii-initium-ii-D-auto.gtex}

%\psalmusEtTranslatioT{temporalia/ps9xxii_xxxii-I-comb.tex}{10cm}
\input{temporalia/ps9xxii_xxxii-I.tex} \Abardot{}

\vfill
\pagebreak

\pars{Psalmus 8.}\scriptura{Ex. 15, 18}

\vspace{-4mm}

\antiphona{IV* e}{temporalia/ant-inaeternum.gtex}

%\vspace{-4mm}

\scriptura{Ps. 9, 33-39}

\initiumpsalmi{temporalia/ps9xxxiii_xxxix-initium-iv_-e-auto.gtex}

%\psalmusEtTranslatioT{temporalia/ps9xxxiii_xxxix-I-comb.tex}{10cm}
\input{temporalia/ps9xxxiii_xxxix-I.tex} \Abardot{}

\vfill
\pagebreak

\pars{Psalmus 9.} \scriptura{Ps. 10, 8}

\vspace{-4mm}

\antiphona{II* f}{temporalia/ant-justusdominus.gtex}

%\vspace{-4mm}

\scriptura{Ps. 10}

%\initiumpsalmi{temporalia/ps10-initium-iv-c-auto.gtex}
\initiumpsalmi{temporalia/ps10-initium-ii_-f.gtex}

%\psalmusEtTranslatioT{temporalia/ps10-I-comb.tex}{10cm}
\input{temporalia/ps10-I.tex} \Abardot{}

\vfill
\pagebreak

\pars{Versus.} \scriptura{Ps. 118, 148}

% Versus. %%%
\sineinitiali{temporalia/versus-praevenerunt.gtex}

\vspace{5mm}

\sineinitiali{temporalia/oratiodominica-mat.gtex}

\vspace{5mm}

\pars{Absolutio.}

\cuminitiali{}{temporalia/absolutio-avinculis.gtex}

\vfill
\pagebreak

\cuminitiali{}{temporalia/benedictio-solemn-evangelica.gtex}

\vspace{7mm}

\lectiovii

\noindent \Vbardot{} Tu autem, Dómine, miserére nobis.
\noindent \Rbardot{} Deo grátias.

\vfill
\pagebreak

\responsoriumvii

\vfill
\pagebreak

\cuminitiali{}{temporalia/benedictio-solemn-divinum.gtex}

\vspace{7mm}

\lectioviii

\noindent \Vbardot{} Tu autem, Dómine, miserére nobis.
\noindent \Rbardot{} Deo grátias.

\vfill
\pagebreak

\responsoriumviii

\vfill
\pagebreak

\cuminitiali{}{temporalia/benedictio-solemn-adsocietatem.gtex}

\vspace{7mm}

\lectioix

\noindent \Vbardot{} Tu autem, Dómine, miserére nobis.
\noindent \Rbardot{} Deo grátias.

\vfill
\pagebreak

% Te Deum

{
\pars{Hymnus Ambrosianus} \scriptura{Tonus Solemnis}

\vspace{-2mm}

\grechangedim{interwordspacetext}{0.26 cm plus 0.15 cm minus 0.05 cm}{scalable}%
\cuminitiali{III}{temporalia/tedeum-solemnis-gn.gtex}
\grechangedim{interwordspacetext}{0.22 cm plus 0.15 cm minus 0.05 cm}{scalable}%
}

\vfill
\pagebreak

\rubrica{Reliqua omittuntur, nisi Laudes separandæ sint.}

\pars{Oratio}

\noindent \Vbardot{} Dómine, exáudi oratiónem meam.

\noindent \Rbardot{} Et clamor meus ad te véniat.

Orémus:

\oratioLaudes

\vspace{7mm}

\pars{Conclusio}

\noindent \Vbardot{} Dómine, exáudi oratiónem meam.

\noindent \Rbardot{} Et clamor meus ad te véniat.

\noindent \Vbardot{} Benedicámus Dómino, allelúia, allelúia.

\noindent \Rbardot{} Deo grátias, allelúia, allelúia.

\noindent \Vbardot{} Fidélium ánimæ per misericórdiam Dei requiéscant in pace.

\noindent \Rbardot{} Amen.

\vfill
\pagebreak

\hora{Ad Laudes.} %%%%%%%%%%%%%%%%%%%%%%%%%%%%%%%%%%%%%%%%%%%%%%%%%%%%%
%\sideThumbs{Laudes}

\cantusSineNeumas

\vspace{0.5cm}
\grechangedim{interwordspacetext}{0.18 cm plus 0.15 cm minus 0.05 cm}{scalable}%
\cuminitiali{}{temporalia/deusinadiutorium-alter.gtex}
\grechangedim{interwordspacetext}{0.22 cm plus 0.15 cm minus 0.05 cm}{scalable}%

\vfill
%\pagebreak

\pars{Psalmus 1.}

\vspace{-4mm}

\antiphona{VI F}{temporalia/ant-alleluia1.gtex}

\scriptura{Psalmus 50.}

\initiumpsalmi{temporalia/ps50-initium-vi-F-auto.gtex}

%\psalmusEtTranslatioT{temporalia/ps50-I-comb.tex}{10cm}
\input{temporalia/ps50-I.tex}

\vfill
\pagebreak

\pars{Psalmus 2.}

\scriptura{Psalmus 117.}

\initiumpsalmi{temporalia/ps117-initium-vi-F-auto.gtex}

%\psalmusEtTranslatioT{temporalia/ps117-I-comb.tex}{10cm}
\input{temporalia/ps117-I.tex}

\vfill
\pagebreak

\pars{Psalmus 3.}

\scriptura{Psalmus 62.}

\initiumpsalmi{temporalia/ps62-initium-vi-F-auto.gtex}

%\psalmusEtTranslatioT{temporalia/ps62-I-comb.tex}{10cm}
\input{temporalia/ps62-I.tex}

\vfill

\vspace{-6mm}

\antiphona{}{temporalia/ant-alleluia1.gtex} % repeat the antiphon - new page

\vfill
\pagebreak

\pars{Psalmus 4.} \scriptura{Dan. 3, 22-26; \textbf{H422}}

\vspace{-4mm}

\antiphona{VIII G}{temporalia/ant-trespueri.gtex}

\scriptura{Canticum trium puerorum, Dan. 3, 57-88 et 56}

\initiumpsalmi{temporalia/dan3-initium-viii-G-auto.gtex}

%\psalmusEtTranslatioT{temporalia/dan3-comb.tex}{10cm}
\input{temporalia/dan3.tex}

\rubrica{Hic non dicitur Gloria Patri, neque Amen.}

\vfill

\vspace{-6mm}

\antiphona{}{temporalia/ant-trespueri.gtex} % repeat the antiphon - new page

\vfill
\pagebreak

\pars{Psalmus 5.}

\vspace{-4mm}

\antiphona{VIII G}{temporalia/ant-alleluia2.gtex}

\scriptura{Psalmus 148.}

\initiumpsalmi{temporalia/ps148-initium-viii-G-auto.gtex}

%\psalmusEtTranslatioT{temporalia/ps148-I-comb.tex}{10cm}
\input{temporalia/ps148-I.tex}

\rubrica{Hic non dicitur Gloria Patri.}

\vfill
\pagebreak

%
\scriptura{Psalmus 149.}

\initiumpsalmi{temporalia/ps149-initium-viii-G-auto.gtex}

%\psalmusEtTranslatioT{temporalia/ps149-I-comb.tex}{10cm}
\input{temporalia/ps149-I.tex}

\rubrica{Hic non dicitur Gloria Patri.}

\vfill
\pagebreak

%
\scriptura{Psalmus 150.}

\initiumpsalmi{temporalia/ps150-initium-viii-G-auto.gtex}

%\psalmusEtTranslatioT{temporalia/ps150-I-comb.tex}{10cm}
\input{temporalia/ps150-I.tex}

\vfill

\vspace{-6mm}

\antiphona{}{temporalia/ant-alleluia2.gtex} % repeat the antiphon - new page

\vfill
\pagebreak

\pars{Capitulum.} \scriptura{Ac. 7, 12}

\grechangedim{interwordspacetext}{0.12 cm plus 0.15 cm minus 0.05 cm}{scalable}%
\cuminitiali{}{temporalia/capitulum-Benedictio.gtex}
\grechangedim{interwordspacetext}{0.22 cm plus 0.15 cm minus 0.05 cm}{scalable}

% preklad Jeruz. bible
%\trCapituliI

\vfill

\pars{Responsorium breve.} \scriptura{Ps. 118, 36-37}

\cuminitiali{IV}{temporalia/resp-inclinacormeum.gtex}

%\trResp

\vfill
\pagebreak

\pars{Hymnus} \scriptura{Gregorius Magnus (\olddag{} 604)}

\cuminitiali{IV}{temporalia/hym-EcceJamNoctis.gtex}
\vspace{-3mm}
%\input{hym-EcceJamNocis-bohtext.tex}

\vfill
%\pagebreak

\pars{Versus.} \scriptura{Ps. 92, 1}

% Versus. %%%
\sineinitiali{temporalia/versus-dominusregnavit.gtex}

%\noindent \trVersus

\vfill
\pagebreak

\benedictus

\vspace{-1cm}

\vfill
\pagebreak

%\sideThumbs{{\scriptsize{}Fine horarum}}

\anteOrationem

\pagebreak

% Oratio. %%%
\oratioLaudes

\vspace{-1mm}
%\trOrationisI

\vfill

\rubrica{Hebdomadarius dicit iterum Dominus vobiscum, vel cantor dicit:}

\vspace{2mm}

\sineinitiali{temporalia/domineexaudi.gtex}

\rubrica{Postea cantatur a cantore:}

\vspace{2mm}

\cuminitiali{I}{temporalia/benedicamus-dominica-perannum.gtex}

\vspace{1mm}

\vfill
\pagebreak

\hora{Ad II. Vesperas.} %%%%%%%%%%%%%%%%%%%%%%%%%%%%%%%%%%%%%%%%%%%%%%%%%%%%%
%\sideThumbs{II. Vesperæ}

\cantusSineNeumas

%\vspace{0.5cm}
\grechangedim{interwordspacetext}{0.18 cm plus 0.15 cm minus 0.05 cm}{scalable}%
\cuminitiali{}{temporalia/deusinadiutorium-solemnis.gtex}
\grechangedim{interwordspacetext}{0.22 cm plus 0.15 cm minus 0.05 cm}{scalable}%

\vfill
%\pagebreak

\vspace{-2mm}

\pars{Psalmus 1.} \scriptura{Ps. 109, 1; \textbf{H91}}

\vspace{-4mm}

\antiphona{VII c\textsuperscript{2}}{temporalia/ant-dixitdominus.gtex}

\vspace{-4mm}

\scriptura{Psalmus 109.}

\initiumpsalmi{temporalia/ps109-initium-vii-c2-auto.gtex}

%\psalmusEtTranslatioT{temporalia/ps109-I-comb.tex}{10cm}
\input{temporalia/ps109-I.tex} \Abardot{}

\vspace{-1cm}

\vfill
\pagebreak

\pars{Psalmus 2.} \scriptura{Ps. 110, 8; \textbf{H91}}

\vspace{-4mm}

\antiphona{IV g}{temporalia/ant-fideliaomnia.gtex}

\scriptura{Psalmus 110.}

\initiumpsalmi{temporalia/ps110-initium-iv-g-auto.gtex}

%\psalmusEtTranslatioT{temporalia/ps110-I-comb.tex}{10cm}
\input{temporalia/ps110-I.tex} \Abardot{}

\vfill
\pagebreak

\pars{Psalmus 3.} \scriptura{Ps. 111, 1; \textbf{H92}}

\vspace{-4mm}

\antiphona{IV a}{temporalia/ant-inmandatis.gtex}

\scriptura{Psalmus 111.}

\initiumpsalmi{temporalia/ps111-initium-iv-a-auto.gtex}

%\psalmusEtTranslatioT{temporalia/ps111-I-comb.tex}{10cm}
\input{temporalia/ps111-I.tex} \Abardot{}

\vfill
\pagebreak

\pars{Psalmus 4.} \scriptura{Ps. 112, 2; \textbf{H92}}

\vspace{-4mm}

\antiphona{VII c}{temporalia/ant-sitnomendomini.gtex}

\scriptura{Psalmus 112.}

\initiumpsalmi{temporalia/ps112-initium-vii-c-auto.gtex}

%\psalmusEtTranslatioT{temporalia/ps112-I-comb.tex}{10cm}
\input{temporalia/ps112-I.tex} \Abardot{}

\vfill
\pagebreak

\pars{Capitulum.} \scriptura{2 Cor. 1, 3-4}

\grechangedim{interwordspacetext}{0.12 cm plus 0.15 cm minus 0.05 cm}{scalable}%
\cuminitiali{}{temporalia/capitulum-BenedictusDeus.gtex}
\grechangedim{interwordspacetext}{0.22 cm plus 0.15 cm minus 0.05 cm}{scalable}

% preklad Jeruz. bible
%\trCapituliI

\vfill

\pars{Responsorium breve.} \scriptura{Ps. 103, 24}

\cuminitiali{VI}{temporalia/resp-quammagnificata.gtex}

%\trResp

\vfill
\pagebreak

\pars{Hymnus} \scriptura{Gregorius Magnus (\olddag{} 604)}

\cuminitiali{I}{temporalia/hym-LucisCreator-aestivalis.gtex}
\vspace{-3mm}
%\begin{translatioMulticol}{3}
Tvůrce světa předobrý,\\
tys ustanovil denní řád\\
a proudy světla rozhodil,\\
když světu základy jsi klad.\\
\\
A spojils ráno s večerem\\
a dnem tu dobu nazýváš;\\
hle padá temné noci stín -\\
slyš prosbu, vyslyš nářek náš.\columnbreak

Ach, nedej, by nás stihla smrt,\\
když svědomí nám tíží hřích,\\
když nemyslíme na věčnost\\
v té síti hříchů šalebných.\\
\\
Vzbuď naši touhu po nebi,\\
kde věčný život čeká nás,\\
a pomoz odložit vše zlé\\
a smýti z duše každý kaz.\columnbreak

To splň nám, dobrý Otče náš,\\
i ty, jenž rovné božství máš,\\
i Duchu, který těšíš nás\\
a vládneš, Bože, v každý čas.\\
Amen. 
\end{translatioMulticol}


\vfill
%\pagebreak

\pars{Versus.} \scriptura{Ps. 140, 2}

% Versus. %%%
\sineinitiali{temporalia/versus-dirigatur.gtex}

%\noindent \trVersus

\vfill
\pagebreak

\magnificatii

\vfill
\pagebreak

%\sideThumbs{{\scriptsize{}Fine horarum}}

\anteOrationem

\pagebreak

% Oratio. %%%
\oratioLaudes

\vspace{-1mm}
%\trOrationisI

\vfill

\rubrica{Hebdomadarius dicit iterum Dominus vobiscum, vel cantor dicit:}

\vspace{2mm}

\sineinitiali{temporalia/domineexaudi.gtex}

\rubrica{Postea cantatur a cantore:}

\vspace{2mm}

\cuminitiali{I}{temporalia/benedicamus-dominica-perannum.gtex}

\vspace{1mm}

\end{document}

