\newcommand{\titulus}{\nomenFesti{Ss. Philippi et Iacobi, Apostolorum.}
\dies{Die 3. Maii.}}
\newcommand{\oratio}{\pars{Oratio.}

\noindent Deus, qui nos ánnua apostolórum Philíppi et Iacóbi festivitáte lætíficas, da nobis, ipsórum précibus, in Unigéniti tui passióne et resurrectióne consórtium, ut ad perpétuam tui visiónem perveníre mereámur.

\noindent Per Dóminum nostrum Iesum Christum, Fílium tuum, qui tecum vivit et regnat in unitáte Spíritus Sancti, Deus, per ómnia sǽcula sæculórum.

\noindent \Rbardot{} Amen.}
\newcommand{\invitatorium}{\pars{Invitatorium.}

\vspace{-4mm}

\antiphona{VI*}{temporalia/inv-regemapostolorum-tp.gtex}}
\newcommand{\hymnusmatutinum}{\pars{Hymnus}

\cuminitiali{III}{temporalia/hym-PhilippeSummae.gtex}}
\newcommand{\matutinum}{\pars{Psalmus 1.} \scriptura{Cf. Ps. 91, 13; \textbf{H254}}

\vspace{-4mm}

\antiphona{I a}{temporalia/ant-sanctituidomine.gtex}

%\vspace{-2mm}

\scriptura{Ps. 18}

%\vspace{-2mm}

\initiumpsalmi{temporalia/ps18-initium-i-a-auto.gtex}

%\vspace{-1.5mm}

\input{temporalia/ps18-i-a.tex}

\vfill

\antiphona{}{temporalia/ant-sanctituidomine.gtex}

\vfill
\pagebreak

\pars{Psalmus 2.} \scriptura{Cf. Ps. 117, 15; Sap. 18, 1; \textbf{H254}}

\vspace{-4mm}

\antiphona{I d}{temporalia/ant-intabernaculisiustorum.gtex}

%\vspace{-2mm}

\scriptura{Ps. 63}

%\vspace{-2mm}

\initiumpsalmi{temporalia/ps63-initium-i-d-auto.gtex}

%\vspace{-1.5mm}

\input{temporalia/ps63-i-d.tex} \Abardot{}

\vfill
\pagebreak

\pars{Psalmus 3.} \scriptura{Ps. 117, 15; \textbf{H254}}

\vspace{-4mm}

\antiphona{VIII G\textsuperscript{2}}{temporalia/ant-voxlaetitiaeintabernaculis.gtex}

%\vspace{-2mm}

\scriptura{Ps. 96}

%\vspace{-2mm}

\initiumpsalmi{temporalia/ps96-initium-viii-G5-auto.gtex}

\input{temporalia/ps96-viii-G5.tex} \Abardot{}

\vfill
\pagebreak}
\newcommand{\matversus}{\noindent \Vbardot{} Sancti et iusti in Dómino gaudéte, allelúia.

\noindent \Rbardot{} Vos elégit Deus in hereditátem sibi, allelúia.}
\newcommand{\lectioi}{\pars{Lectio I.} \scriptura{Ac. 5, 12-21}

\noindent De Actibus Apostolórum.

\noindent In diébus illis: Per manus apostolórum fiébant signa et prodígia multa in plebe; et erant unanímiter omnes in pórticu Salomónis. Ceterórum autem nemo audébat coniúngere se illis, sed magnificábat eos pópulus; magis autem addebántur credéntes Dómino, multitúdines virórum ac mulíerum, ita ut et in platéas efférrent infírmos et pónerent in léctulis et grabátis, ut, veniénte Petro, saltem umbra illíus obumbráret quemquam eórum. Concurrébat autem et multitúdo vicinárum civitátum Ierúsalem, afferéntes ægros et vexátos a spirítibus immúndis, qui curabántur omnes. Exsúrgens autem princeps sacerdótum et omnes, qui cum illo erant, quæ est hǽresis sadducæórum, repléti sunt zelo et iniecérunt manus in apóstolos et posuérunt illos in custódia pública. Angelus autem Dómini per noctem apéruit iánuas cárceris et edúcens eos dixit: «Ite et stantes loquímini in templo plebi ómnia verba vitæ huius». Qui cum audíssent, intravérunt dilúculo in templum et docébant.}
\newcommand{\responsoriumi}{\pars{Responsorium 1.} \scriptura{\Rbardot{} Io. 16, 20 \Vbardot{} ibid.; \textbf{H252}}

\vspace{-5mm}

\responsorium{VIII}{temporalia/resp-tristitiavestra-CROCHU.gtex}{}}
\newcommand{\lectioii}{\pars{Lectio II.} \scriptura{Ac. 5, 21-25}

\noindent Advéniens autem princeps sacerdótum et, qui cum eo erant, convocavérunt concílium et omnes senióres filiórum Israel et misérunt in cárcerem, ut adduceréntur illi. Cum veníssent autem minístri, non invenérunt illos in cárcere; revérsi autem nuntiavérunt dicéntes: «Cárcerem invénimus clausum cum omni diligéntia et custódes stantes ad iánuas, aperiéntes autem intus néminem invénimus!». Ut audiérunt autem hos sermónes, magistrátus templi et príncipes sacerdótum ambigébant de illis quidnam fíeret illud. Advéniens autem quidam nuntiávit eis: «Ecce viri, quos posuístis in cárcere, sunt in templo stantes et docéntes pópulum».}
\newcommand{\responsoriumii}{\pars{Responsorium 2.} \scriptura{\Rbardot{} Ps. 115, 6 \Vbardot{} ibid., 9; \textbf{H252}}

\vspace{-5mm}

\responsorium{VII}{temporalia/resp-pretiosa-CROCHU.gtex}{}}
\newcommand{\lectioiii}{\pars{Lectio III.} \scriptura{Ac. 5, 26-32}

\noindent Tunc ábiens magistrátus cum minístris adducébat illos, non per vim, timébant enim pópulum, ne lapidaréntur. Et cum adduxíssent illos, statuérunt in concílio. Et interrogávit eos princeps sacerdótum dicens: «Nonne præcipiéndo præcépimus vobis, ne docerétis in nómine isto? Et ecce replevístis Ierúsalem doctrína vestra et vultis indúcere super nos sánguinem hóminis istíus». Respóndens autem Petrus et apóstoli dixérunt: «Obœdíre opórtet Deo magis quam homínibus. Deus patrum nostrórum suscitávit Iesum, quem vos interemístis suspendéntes in ligno; hunc Deus Príncipem et Salvatórem exaltávit déxtera sua ad dandam pæniténtiam Israel et remissiónem peccatórum. Et nos sumus testes horum verbórum, et Spíritus Sanctus, quem dedit Deus obœdiéntibus sibi».}
\newcommand{\responsoriumiii}{\pars{Responsorium 3.} \scriptura{\Vbardot{} Ps. 147, 13; \textbf{H252}}

\vspace{-5mm}

\responsorium{I}{temporalia/resp-filiaeierusalem-CROCHU-cumdox.gtex}{}

\vfill
\pagebreak

\pars{Hymnus Ambrosianus} \scriptura{Alio modo, iuxta morem Romanum}

\vspace{-2mm}

\grechangedim{interwordspacetext}{0.26 cm plus 0.15 cm minus 0.05 cm}{scalable}%
\cuminitiali{III}{temporalia/tedeum-romanum-gn.gtex}
\grechangedim{interwordspacetext}{0.22 cm plus 0.15 cm minus 0.05 cm}{scalable}%
}
\newcommand{\hymnuslaudes}{\pars{Hymnus}

\cuminitiali{III}{temporalia/hym-ClaroPaschali.gtex}}
\newcommand{\laudes}{\pars{Psalmus 1.} \scriptura{Io. 14, 8}

\vspace{-4mm}

\antiphona{VII c\textsuperscript{2}}{temporalia/ant-domineostende.gtex}

%\vspace{-2mm}

\scriptura{Psalmus 62}

%\vspace{-2mm}

\initiumpsalmi{temporalia/ps62-initium-vii-c2-auto.gtex}

%\vspace{-1.5mm}

\input{temporalia/ps62-vii-c2.tex} \Abardot{}

\vfill
\pagebreak

\pars{Psalmus 2.} \scriptura{Io. 14, 9; \textbf{H246}}

\vspace{-4mm}

\antiphona{III a trans.}{temporalia/ant-tantotemporevobiscum.gtex}

%\vspace{-2mm}

\scriptura{Canticum trium puerorum, Dan. 3, 57-88 et 56}

\initiumpsalmi{temporalia/dan3-initium-iii-a-trans.gtex}

\input{temporalia/dan3-iii-a-sinedox.tex}

\rubrica{Hic non dicitur Gloria Patri, neque Amen.}

\vfill

\antiphona{}{temporalia/ant-tantotemporevobiscum.gtex}

\vfill
\pagebreak

\pars{Psalmus 3.} \scriptura{Io. 14, 15; \textbf{H268}}

\vspace{-4mm}

\antiphona{III a}{temporalia/ant-sidiligitisme.gtex}

%\vspace{-2mm}

\scriptura{Psalmus 149}

%\vspace{-2mm}

\initiumpsalmi{temporalia/ps149-initium-iii-a-auto.gtex}

\input{temporalia/ps149-iii-a.tex} \Abardot{}

\vfill
\pagebreak}
\newcommand{\lectiobrevis}{\pars{Lectio Brevis.} \scriptura{Eph. 2, 19-22}

\noindent Iam non estis extránei et ádvenæ, sed estis concíves sanctórum et doméstici Dei, superædificáti super fundaméntum apostolórum et prophetárum, ipso summo angulári lápide Christo Iesu, in quo omnis ædificátio compácta crescit in templum sanctum in Dómino, in quo et vos coædificámini in habitáculum Dei in Spíritu.}
\newcommand{\responsoriumbreve}{\pars{Responsorium breve.} \scriptura{Ps. 44, 17.18}

\cuminitiali{VI}{temporalia/resp-constitueseosprincipes-tp.gtex}}
\newcommand{\preces}{\noindent Fratres caríssimi, hereditátem cæléstem ab Apóstolis habéntes, grátias agámus Patri nostro,~\gredagger{} pro ómnibus donis:

\Rbardot{} Te laudat Apostolórum chorus, Dómine.

\noindent Laus tibi, Dómine, pro mensa córporis et sánguinis, nobis ab Apóstolis trádita,~\gredagger{} qua refícimur et vívimus:

\Rbardot{} Te laudat Apostolórum chorus, Dómine.

\noindent Pro mensa verbi tui, nobis ab Apóstolis paráta,~\gredagger{} qua lumen et gáudium nobis dantur:

\Rbardot{} Te laudat Apostolórum chorus, Dómine.

\noindent Pro Ecclésia tua sancta, super Apóstolos ædificáta,~\gredagger{} qua in unum concorporámur:

\Rbardot{} Te laudat Apostolórum chorus, Dómine.

\noindent Pro lavácro baptísmi et pæniténtiæ, Apóstolis concrédito,~\gredagger{} quo ab ómnibus peccátis ablúimur:

\Rbardot{} Te laudat Apostolórum chorus, Dómine.}
\newcommand{\benedictus}{\pars{Canticum Zachariæ.} \scriptura{Io. 14, 6}

\vspace{-4mm}

\antiphona{VIII c}{temporalia/ant-egosumvia.gtex}

\vspace{-2mm}

\scriptura{Lc. 1, 68-79}

\vspace{-2mm}

\cantusSineNeumas
\initiumpsalmi{temporalia/benedictus-initium-viii-C-auto.gtex}

%\vspace{-1.5mm}

\input{temporalia/benedictus-viii-C.tex} \Abardot{}}
\newcommand{\hebdomada}{infra Hebdom. V post Pentecosten.}
\newcommand{\oratioLaudes}{\cuminitiali{}{temporalia/oratio5.gtex}}

% LuaLaTeX

\documentclass[a4paper, twoside, 12pt]{article}
\usepackage[latin]{babel}
%\usepackage[landscape, left=3cm, right=1.5cm, top=2cm, bottom=1cm]{geometry} % okraje stranky
%\usepackage[landscape, a4paper, mag=1166, truedimen, left=2cm, right=1.5cm, top=1.6cm, bottom=0.95cm]{geometry} % okraje stranky
\usepackage[landscape, a4paper, mag=1400, truedimen, left=0.5cm, right=0.5cm, top=0.5cm, bottom=0.5cm]{geometry} % okraje stranky

\usepackage{fontspec}
\setmainfont[FeatureFile={junicode.fea}, Ligatures={Common, TeX}, RawFeature=+fixi]{Junicode}
%\setmainfont{Junicode}

% shortcut for Junicode without ligatures (for the Czech texts)
\newfontfamily\nlfont[FeatureFile={junicode.fea}, Ligatures={Common, TeX}, RawFeature=+fixi]{Junicode}

\usepackage{multicol}
\usepackage{color}
\usepackage{lettrine}
\usepackage{fancyhdr}

% usual packages loading:
\usepackage{luatextra}
\usepackage{graphicx} % support the \includegraphics command and options
\usepackage{gregoriotex} % for gregorio score inclusion
\usepackage{gregoriosyms}
\usepackage{wrapfig} % figures wrapped by the text
\usepackage{parcolumns}
\usepackage[contents={},opacity=1,scale=1,color=black]{background}
\usepackage{tikzpagenodes}
\usepackage{calc}
\usepackage{longtable}
\usetikzlibrary{calc}

\setlength{\headheight}{14.5pt}

\input{conventuscommune.tex} % Often used macros

\newcommand{\annusEditionis}{2021}

%%%% Vicekrat opakovane kousky

\newcommand{\anteOrationem}{
  \rubrica{Ante Orationem, cantatur a Superiore:}

  \pars{Supplicatio Litaniæ.}

  \cuminitiali{}{temporalia/supplicatiolitaniae.gtex}

  \pars{Oratio Dominica.}

  \cuminitiali{}{temporalia/oratiodominica.gtex}

  \rubrica{Deinde dicitur ab Hebdomadario:}

  \cuminitiali{}{temporalia/dominusvobiscum-solemnis.gtex}

  \rubrica{In choro monialium loco Dominus vobiscum dicitur:}

  \sineinitiali{temporalia/domineexaudi.gtex}
}

\setlength{\columnsep}{30pt} % prostor mezi sloupci

%%%%%%%%%%%%%%%%%%%%%%%%%%%%%%%%%%%%%%%%%%%%%%%%%%%%%%%%%%%%%%%%%%%%%%%%%%%%%%%%%%%%%%%%%%%%%%%%%%%%%%%%%%%%%
\begin{document}

% Here we set the space around the initial.
% Please report to http://home.gna.org/gregorio/gregoriotex/details for more details and options
\grechangedim{afterinitialshift}{2.2mm}{scalable}
\grechangedim{beforeinitialshift}{2.2mm}{scalable}
\grechangedim{interwordspacetext}{0.22 cm plus 0.15 cm minus 0.05 cm}{scalable}%
\grechangedim{annotationraise}{-0.2cm}{scalable}

% Here we set the initial font. Change 38 if you want a bigger initial.
% Emit the initials in red.
\grechangestyle{initial}{\color{red}\fontsize{38}{38}\selectfont}

\pagestyle{empty}

%%%% Titulni stranka
\begin{titulusOfficii}
\ifx\titulus\undefined
\nomenFesti{Feria II \hebdomada{}}
\else
\titulus
\fi
\end{titulusOfficii}

\vfill

\begin{center}
%Ad usum et secundum consuetudines chori \guillemotright{}Conventus Choralis\guillemotleft.

%Editio Sancti Wolfgangi \annusEditionis
\end{center}

\scriptura{}

\pars{}

\pagebreak

\renewcommand{\headrulewidth}{0pt} % no horiz. rule at the header
\fancyhf{}
\pagestyle{fancy}

\cantusSineNeumas

\ifx\oratio\undefined
\ifx\laudb\undefined
\else
\newcommand{\oratio}{\pars{Oratio.}

\noindent Dómine Deus omnípotens, qui ad princípium huius diéi nos perveníre fecísti, tua nos hódie salva virtúte, ut in hac die ad nullum declinémus peccátum, sed semper ad tuam iustítiam faciéndam nostra procédant elóquia, dirigántur cogitatiónes et ópera.

\noindent Per Dóminum nostrum Iesum Christum, Fílium tuum, qui tecum vivit et regnat in unitáte Spíritus Sancti, Deus, per ómnia sǽcula sæculórum.

\noindent \Rbardot{} Amen.}
\fi
\fi

\hora{Ad Matutinum.} %%%%%%%%%%%%%%%%%%%%%%%%%%%%%%%%%%%%%%%%%%%%%%%%%%%%%
%\sideThumbs{Matutinum}

\vspace{2mm}

\cuminitiali{}{temporalia/dominelabiamea.gtex}

\vfill
%\pagebreak

\vspace{2mm}

\ifx\invitatorium\undefined
\pars{Invitatorium.} \scriptura{Ps. 94, 1; Psalmus 94; \textbf{H451}}

\vspace{-6mm}

\antiphona{VI}{temporalia/inv-jubilemusdeo.gtex}\else
\invitatorium
\fi

\vfill
\pagebreak

\ifx\hymnusmatutinum\undefined
\ifx\matua\undefined
\else
\pars{Hymnus.}

{
\grechangedim{interwordspacetext}{0.10 cm plus 0.15 cm minus 0.05 cm}{scalable}%
\antiphona{II}{temporalia/hym-IpsumNunc.gtex}
\grechangedim{interwordspacetext}{0.22 cm plus 0.15 cm minus 0.05 cm}{scalable}%
}
\fi
\else
\hymnusmatutinum
\fi

\vspace{-3mm}

\vfill
\pagebreak

\ifx\matub\undefined
\else
% MAT B
\pars{Psalmus 1.} \scriptura{Ps. 30, 2; \textbf{H90}}

\vspace{-4mm}

\antiphona{VIII G}{temporalia/ant-intuaiustitia.gtex}

%\vspace{-2mm}

\scriptura{Ps. 30, 2-9}

%\vspace{-2mm}

\initiumpsalmi{temporalia/ps30i-initium-viii-G-auto.gtex}

\vspace{-1.5mm}

\input{temporalia/ps30i-viii-G.tex} \Abardot{}

\vfill
\pagebreak

\pars{Psalmus 2.} \scriptura{Ps. 66, 2}

\vspace{-4mm}

\antiphona{E}{temporalia/ant-illuminadomine.gtex}

%\vspace{-2mm}

\scriptura{Ps. 30, 10-17}

%\vspace{-2mm}

\initiumpsalmi{temporalia/ps30ii-initium-e-a-auto.gtex}

\input{temporalia/ps30ii-e-a.tex} \Abardot{}

\vfill
\pagebreak

\pars{Psalmus 3.} \scriptura{Ps. 30, 24}

\vspace{-4mm}

\antiphona{II D}{temporalia/ant-diligitedominum.gtex}

%\vspace{-5mm}

\scriptura{Ps. 30, 20-25}

%\vspace{-2mm}

\initiumpsalmi{temporalia/ps30iii-initium-ii-D-auto.gtex}

\input{temporalia/ps30iii-ii-D.tex} \Abardot{}

\vfill
\pagebreak
\fi

\pars{Versus.}

\ifx\matversus\undefined
\ifx\matub\undefined
\else
\noindent \Vbardot{} Dírige me, Dómine, in veritáte tua, et doce me.

\noindent \Rbardot{} Quia tu es Deus salútis meæ.
\fi
\else
\matversus
\fi

\vspace{5mm}

\sineinitiali{temporalia/oratiodominica-mat.gtex}

\vspace{5mm}

\pars{Absolutio.}

\cuminitiali{}{temporalia/absolutio-exaudi.gtex}

\vfill
\pagebreak

\cuminitiali{}{temporalia/benedictio-solemn-benedictione.gtex}

\vspace{7mm}

\lectioi

\noindent \Vbardot{} Tu autem, Dómine, miserére nobis.
\noindent \Rbardot{} Deo grátias.

\vfill
\pagebreak

\responsoriumi

\vfill
\pagebreak

\cuminitiali{}{temporalia/benedictio-solemn-unigenitus.gtex}

\vspace{7mm}

\lectioii

\noindent \Vbardot{} Tu autem, Dómine, miserére nobis.
\noindent \Rbardot{} Deo grátias.

\vfill
\pagebreak

\responsoriumii

\vfill
\pagebreak

\cuminitiali{}{temporalia/benedictio-solemn-spiritus.gtex}

\vspace{7mm}

\lectioiii

\noindent \Vbardot{} Tu autem, Dómine, miserére nobis.
\noindent \Rbardot{} Deo grátias.

\vfill
\pagebreak

\responsoriumiii

\vfill
\pagebreak

\rubrica{Reliqua omittuntur, nisi Laudes separandæ sint.}

\sineinitiali{temporalia/domineexaudi.gtex}

\vfill

\oratio

\vfill

\noindent \Vbardot{} Dómine, exáudi oratiónem meam.
\Rbardot{} Et clamor meus ad te véniat.

\vfill

\noindent \Vbardot{} Benedicámus Dómino.
\noindent \Rbardot{} Deo grátias.

\vfill

\noindent \Vbardot{} Fidélium ánimæ per misericórdiam Dei requiéscant in pace.
\Rbardot{} Amen.

\vfill
\pagebreak

\hora{Ad Laudes.} %%%%%%%%%%%%%%%%%%%%%%%%%%%%%%%%%%%%%%%%%%%%%%%%%%%%%
%\sideThumbs{Laudes}

\cantusSineNeumas

\vspace{0.5cm}
\grechangedim{interwordspacetext}{0.18 cm plus 0.15 cm minus 0.05 cm}{scalable}%
\cuminitiali{}{temporalia/deusinadiutorium-communis.gtex}
\grechangedim{interwordspacetext}{0.22 cm plus 0.15 cm minus 0.05 cm}{scalable}%

\vfill
\pagebreak

\ifx\hymnuslaudes\undefined
\ifx\laudbd\undefined
\else
\pars{Hymnus} \scriptura{Hilarius (\olddag{} 367)}

\grechangedim{interwordspacetext}{0.16 cm plus 0.15 cm minus 0.05 cm}{scalable}%
\cuminitiali{IV}{temporalia/hym-LucisLargitor.gtex}
\grechangedim{interwordspacetext}{0.22 cm plus 0.15 cm minus 0.05 cm}{scalable}%
\vspace{-3mm}
\fi
\else
\hymnuslaudes
\fi

\vfill
\pagebreak

\ifx\laudb\undefined
\else
\pars{Psalmus 1.} \scriptura{Ps. 41, 3; \textbf{H391}}

\vspace{-4mm}

\antiphona{II D}{temporalia/ant-sitivitanima.gtex}

%\vspace{-2mm}

\scriptura{Psalmus 41}

%\vspace{-2mm}

\initiumpsalmi{temporalia/ps41-initium-ii-D-auto.gtex}

%\vspace{-1.5mm}

\input{temporalia/ps41-ii-D.tex}

\vfill

\antiphona{}{temporalia/ant-sitivitanima.gtex}

\vfill
\pagebreak

\pars{Psalmus 2.}

\vspace{-4mm}

\antiphona{III a}{temporalia/ant-ostendenobisdomine.gtex}

%\vspace{-2mm}

\scriptura{Canticum Ecclesiastici, Sir. 36, 1-7.13-16}

%\vspace{-3mm}

\initiumpsalmi{temporalia/ecclesiastici-initium-iii-a-auto.gtex}

\input{temporalia/ecclesiastici-iii-a.tex} \Abardot{}

\vfill
\pagebreak

\pars{Psalmus 3.}

\vspace{-4mm}

\antiphona{II D}{temporalia/ant-operamanuumeius.gtex}

\scriptura{Psalmus 18, 1-7}

\initiumpsalmi{temporalia/ps18i-initium-ii-D-auto.gtex}

\input{temporalia/ps18i-ii-D.tex} \Abardot{}

\vfill
\pagebreak
\fi

\ifx\lectiobrevis\undefined
\ifx\laudb\undefined
\else
\pars{Lectio Brevis.} \scriptura{Ier. 15, 16}

\noindent Invénti sunt sermónes tui, et comédi eos, et factum est mihi verbum tuum in gáudium et in lætítiam cordis mei, quóniam invocátum est nomen tuum super me, Dómine Deus exercítuum.
\fi
\else
\lectiobrevis
\fi

\vfill

\ifx\responsoriumbreve\undefined
\ifx\laudbd\undefined
\else
\pars{Responsorium breve.} \scriptura{Ps. 32, 1.3}

\cuminitiali{VI}{temporalia/resp-exsultateiusti.gtex}
\fi
\else
\responsoriumbreve
\fi

\vfill
\pagebreak

\ifx\benedictus\undefined
\ifx\laudbd\undefined
\else
\pars{Canticum Zachariæ.} \scriptura{Lc. 1, 68; \textbf{H422}}

\vspace{-4mm}

{
\grechangedim{interwordspacetext}{0.18 cm plus 0.15 cm minus 0.05 cm}{scalable}%
\antiphona{IV E}{temporalia/ant-benedictusdominus.gtex}
\grechangedim{interwordspacetext}{0.22 cm plus 0.15 cm minus 0.05 cm}{scalable}%
}

%\vspace{-3mm}

\scriptura{Lc. 1, 68-79}

%\vspace{-2mm}

\cantusSineNeumas
\initiumpsalmi{temporalia/benedictus-initium-iv-E-auto.gtex}

%\vspace{-1.5mm}

\input{temporalia/benedictus-iv-E.tex} \Abardot{}
\fi
\else
\benedictus
\fi

\vspace{-1cm}

\vfill
\pagebreak

%\sideThumbs{{\scriptsize{}Fine horarum}}

\pars{Preces.}

\sineinitiali{}{temporalia/tonusprecum.gtex}

\ifx\preces\undefined
\ifx\laudb\undefined
\else
\noindent Salvátor noster fecit nos regnum et sacerdótium, ut hóstias Deo acceptábiles offerámus. \gredagger{} Grati ígitur eum invocémus:

\Rbardot{} Serva nos in tuo ministério, Dómine.

\noindent Christe, sacérdos ætérne, qui sanctum pópulo tuo sacerdótium concessísti, \gredagger{} concéde, ut spiritáles hóstias Deo acceptábiles iúgiter offerámus.

\Rbardot{} Serva nos in tuo ministério, Dómine.

\noindent Spíritus tui fructus nobis largíre propítius, \gredagger{} patiéntiam, benignitátem et mansuetúdinem.

\Rbardot{} Serva nos in tuo ministério, Dómine.

\noindent Da nobis te amáre, ut te, qui es cáritas, possideámus, \gredagger{} et bene ágere, ut per vitam étiam nostram te laudémus.

\Rbardot{} Serva nos in tuo ministério, Dómine.

\noindent Quæ frátribus nostris sunt utília, nos quǽrere concéde, \gredagger{} ut salútem facílius consequántur.

\Rbardot{} Serva nos in tuo ministério, Dómine.
\fi
\else
\preces
\fi

\vfill

\pars{Oratio Dominica.}

\cuminitiali{}{temporalia/oratiodominicaalt.gtex}

\vfill
\pagebreak

\rubrica{vel:}

\pars{Supplicatio Litaniæ.}

\cuminitiali{}{temporalia/supplicatiolitaniae.gtex}

\vfill

\pars{Oratio Dominica.}

\cuminitiali{}{temporalia/oratiodominica.gtex}

\vfill
\pagebreak

% Oratio. %%%
\oratio

\vspace{-1mm}

\vfill

\rubrica{Hebdomadarius dicit Dominus vobiscum, vel, absente sacerdote vel diacono, sic concluditur:}

\vspace{2mm}

\antiphona{C}{temporalia/dominusnosbenedicat.gtex}

\rubrica{Postea cantatur a cantore:}

\vspace{2mm}

\cuminitiali{IV}{temporalia/benedicamus-feria-laudes.gtex}

\vspace{1mm}

\vfill
\pagebreak

\end{document}

