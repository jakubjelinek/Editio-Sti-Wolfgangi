\newcommand{\titulus}{\nomenFesti{S. Athanasii, Episcopi \& Ecclesiæ Doctoris.}
\dies{Die 2. Maii.}}
\newcommand{\oratio}{\pars{Oratio.}

\noindent Omnípotens sempitérne Deus, qui beátum Athanásium, epíscopum, divinitátis Fílii tui propugnatórem exímium suscitásti, concéde propítius, ut, eius doctrína et protectióne gaudéntes, in tui cognitióne et amóre sine intermissióne crescámus.

\pars{Pro pace in universo mundo.} \scriptura{Sir. 50, 25; 2 Esdr. 4, 20; \textbf{H416}}

\vspace{-4mm}

\antiphona{II D}{temporalia/ant-dapacemdomine.gtex}

\vfill

\noindent Deus, a quo sancta desidéria, recta consília et iusta sunt ópera: da servis tuis illam, quam mundus dare non potest, pacem; ut et corda nostra mandátis tuis dédita, et hóstium subláta formídine, témpora sint tua protectióne tranquílla.

\noindent Per Dóminum nostrum Iesum Christum, Fílium tuum, qui tecum vivit et regnat in unitáte Spíritus Sancti, Deus, per ómnia sǽcula sæculórum.

\noindent \Rbardot{} Amen.}
\newcommand{\invitatorium}{\pars{Invitatorium.}

\vspace{-4mm}

\antiphona{VI*}{temporalia/inv-fontemsapientiae-tp.gtex}}
\newcommand{\hymnusmatutinum}{\pars{Hymnus}

\cuminitiali{III}{temporalia/hym-ChristePastorum.gtex}}
\newcommand{\lectioi}{\pars{Lectio I.} \scriptura{Ap. 21, 9-18}

\noindent De libro Apocalýpsis beáti Ioánnis Apóstoli.

\noindent Venit unus de septem ángelis habéntibus septem phíalas plenas septem plagis novíssimis et locútus est mecum dicens: «Veni, osténdam tibi sponsam uxórem Agni». Et sústulit me in spíritu super montem magnum et altum et osténdit mihi civitátem sanctam Ierúsalem descendéntem de cælo a Deo, habéntem claritátem Dei; lumen eius símile lápidi pretiosíssimo, tamquam lápidi iáspidi, in modum crystálli; et habébat murum magnum et altum et habébat \emph{portas} duódecim et super portas ángelos duódecim et \emph{nómina} inscrípta, quæ sunt duódecim \emph{tríbuum filiórum Israel. Ab oriénte portæ tres et ab aquilóne portæ tres et ab austro portæ tres et ab occásu portæ tres;} et murus civitátis habens fundaménta duódecim, et super ipsis duódecim nómina duódecim apostolórum Agni.

\noindent {\color{gray}Et, qui loquebátur mecum, habébat mensúram arúndinem áuream, ut metirétur civitátem et portas eius et murum eius. Et cívitas in quadro pósita est, et longitúdo eius tanta est quanta et latitúdo. Et mensus est civitátem arúndine per stádia duódecim mília; longitúdo et latitúdo et altitúdo eius æquáles sunt. Et mensus est murum eius centum quadragínta quáttuor cubitórum, mensúra hóminis, quæ est ángeli. Et erat structúra muri eius ex iáspide, ipsa vero cívitas aurum mundum símile vitro mundo.}}
\newcommand{\responsoriumi}{\pars{Responsorium 1.} \scriptura{\Rbardot{} Ap. 21, 9-10; Is. 61, 10 \Vbardot{} Sir. 32, 18; \textbf{H247}}

\vspace{-5mm}

\responsorium{II}{temporalia/resp-locutusestadmeunus-CROCHU.gtex}{}}
\newcommand{\lectioii}{\pars{Lectio II.} \scriptura{Ap. 21, 19-27}

\noindent Fundaménta muri civitátis omni lápide pretióso ornáta:

\noindent {\color{gray} fundaméntum primum iaspis, secúndus sapphírus, tértius chalcedónius, quartus smarágdus, quintus sárdonyx, sextus sárdinus, séptimus chrysólithus, octávus berýllus, nonus topázius, décimus chrysóprasus, undécimus hyacínthus, duodécimus amethýstus.}

\noindent Et duódecim portæ duódecim margarítæ sunt, et síngulæ portæ erant ex síngulis margarítis. Et platéa civitátis aurum mundum tamquam vitrum perlúcidum.

\noindent Et templum non vidi in ea: Dóminus enim, Deus omnípotens, templum illíus est, et Agnus. Et cívitas non eget sole neque luna, ut lúceant ei, nam cláritas Dei illuminávit eam, et lucérna eius est Agnus.

\noindent Et ambulábunt gentes per lumen eius, et reges terræ áfferunt glóriam suam in illam; et portæ eius non claudéntur per diem, nox enim non erit illic; et áfferent glóriam et divítias géntium in illam. Nec intrábit in ea áliquid coinquinátum et fáciens abominatiónem et mendácium, nisi qui scripti sunt in libro vitæ Agni.}
\newcommand{\responsoriumii}{\pars{Responsorium 2.} \scriptura{\Rbardot{} Tob. 13, 20; \Vbardot{} Ps. 147, 2; \textbf{H248}}

\vspace{-5mm}

\responsorium{I}{temporalia/resp-plateaetuaeierusalem-CROCHU.gtex}{}

\rubrica{vel ad libitum:}

\vspace{3mm}

\pars{Responsorium 2.} \scriptura{\Rbardot{} Ap. 21, 14; \Vbardot{} ibid. 21, 2; \textbf{H249}}

\vspace{-5mm}

\responsorium{VII}{temporalia/resp-vidiportamcivitatis-CROCHU.gtex}{}}
\newcommand{\lectioiii}{\pars{Lectio III.} \scriptura{Oratio de incarnatione Verbi, 8-9: PG 25, 110-111}

\noindent Ex Oratiónibus sancti Athanásii epíscopi.

\noindent {\color{gray} Incorpóreum et corruptiónis ac matériæ expers, Dei Verbum in nostram regiónem advénit, quamquam ántea non procul áberat: nulla enim pars mundi illo umquam vácua fuit, sed una cum suo Patre exsístens ómnia ubíque implébat.

\noindent Venit vero sua erga nos benignitáte, et quátenus sese palam nobis exhíbuit. Nostri ipse géneris et infirmitátis misértus, nostráque motus corruptióne, nec mortem in nobis dominári ferens, ne scílicet quod factum fúerat períret, suíque Patris in hómine formándo opus vanum fíeret, sibi ipsi corpus accépit, idque nostri non dissímile; nec enim esse tantum in córpore, vel solum apparére vóluit.

\noindent Nam, si tantum apparére voluísset, potuísset sane áliud præstántius corpus assúmere; verum nostrum corpus accépit.}

\noindent Sibi in Vírgine templum, corpus scílicet, exstrúxit, illúdque tamquam instruméntum sibi próprium fecit, in quo se notum fáceret et habitáret: sic ergo símili córpore e nostris accépto, quia omnes mortis corruptióni subiécti erant, illud, pro ómnibus morti tráditum, Patri summa cum benignitáte óbtulit; cum, ut ómnibus in ipso moriéntibus lex de corruptióne contra hómines lata solverétur, útpote quæ in domínico córpore suam vim complevísset, nec proínde locum ámplius advérsus símiles hómines habéret; tum, ut hómines, in corruptiónem revérsos, íterum incorrúptos rédderet, atque a morte ad vitam revocáret, córpore quod sibi assúmpserat et resurrectiónis grátia mortem ab eis, non secus ac stípula ab igne consúmitur, pénitus ámovens.

\noindent Idcírco corpus quod mori posset sibi assúmpsit, ut illud, Verbi ómnium prǽsidis factum párticeps, morti pro ómnibus satis esset, atque, propter Verbum in se hábitans, incorrúptum permanéret, ac dénique in pósterum corrúptio resurrectiónis grátia ab ómnibus discéderet. Hinc corpus, quod sibi ipse accépit, velut hóstiam et víctimam omni puram mácula morti offeréndo, mortem statim ab ómnibus simílibus, suo pro áliis obláto, propulsávit.

\noindent Sic enim Dei Verbum, supérius ómnibus exsístens, suum templum et corpóreum instruméntum pro ómnibus, uti dicébat, devovéndo et offeréndo, id quod debebátur in morte solvit, atque ita, ómnibus per símile corpus coniúnctus, incorrúptus Dei Fílius omnes resurrectiónis promissióne incorrúptos iure et mérito réddidit. Ipsa síquidem mortis corrúptio nullam iam vim advérsus hómines habet ob Verbum, quod in illis per unum corpus hábitat.}
\newcommand{\responsoriumiii}{\pars{Responsorium 3.} \scriptura{\Rbardot{} Ps. 63, 11 \Vbardot{} ibid., 10; \textbf{H253}}

\vspace{-5mm}

\responsorium{V}{temporalia/resp-laetabituriustus-CROCHU-cumdox.gtex}{}}
\newcommand{\hymnuslaudes}{\pars{Hymnus}

\cuminitiali{VIII}{temporalia/hym-InclitusPastor-doctorum.gtex}}
\newcommand{\laudes}{\pars{Psalmus 1.}

\vspace{-4mm}

\antiphona{VIII G}{temporalia/ant-alleluia-turco13.gtex}

%\vspace{-2mm}

\scriptura{Psalmus 56}

%\vspace{-2mm}

\initiumpsalmi{temporalia/ps56-initium-viii-G-auto.gtex}

%\vspace{-1.5mm}

\input{temporalia/ps56-viii-G.tex} \Abardot{}

\vfill
\pagebreak

\pars{Psalmus 2.} \scriptura{Ap. 2, 7}

\vspace{-4mm}

\antiphona{VII c\textsuperscript{2}}{temporalia/ant-vincentidaboeiedere.gtex}

%\vspace{-2mm}

\scriptura{Canticum Ieremiæ, 1 Ier. 31, 10-14}

%\vspace{-3mm}

\initiumpsalmi{temporalia/jeremiae3-initium-vii-c2-auto.gtex}

\input{temporalia/jeremiae3-vii-c2.tex} \Abardot{}

\vfill
\pagebreak

\pars{Psalmus 3.}

\vspace{-4mm}

\antiphona{II* a}{temporalia/ant-alleluia-turco18.gtex}

\scriptura{Psalmus 47}

\initiumpsalmi{temporalia/ps47-initium-ii_-a-auto.gtex}

\input{temporalia/ps47-ii_-a.tex} \Abardot{}

\vfill
\pagebreak}
\newcommand{\benedictus}{\pars{Canticum Zachariæ.} \scriptura{Mt. 7, 15.16; \textbf{H429}}

\vspace{-6mm}

{
\grechangedim{interwordspacetext}{0.18 cm plus 0.15 cm minus 0.05 cm}{scalable}%
\antiphona{VIII G}{temporalia/ant-attenditeafalsisprophetis.gtex}
\grechangedim{interwordspacetext}{0.22 cm plus 0.15 cm minus 0.05 cm}{scalable}%
}

\vspace{-4mm}

\scriptura{Lc. 1, 68-79}

\vspace{-2mm}

\cantusSineNeumas
\initiumpsalmi{temporalia/benedictus-initium-viiisoll-G-auto.gtex}

\vspace{-1.5mm}

\input{temporalia/benedictus-viiisoll-G.tex} \Abardot{}
}
\newcommand{\hebdomada}{infra Hebdom. V post Pentecosten.}
\newcommand{\oratioLaudes}{\cuminitiali{}{temporalia/oratio5.gtex}}

% LuaLaTeX

\documentclass[a4paper, twoside, 12pt]{article}
\usepackage[latin]{babel}
%\usepackage[landscape, left=3cm, right=1.5cm, top=2cm, bottom=1cm]{geometry} % okraje stranky
%\usepackage[landscape, a4paper, mag=1166, truedimen, left=2cm, right=1.5cm, top=1.6cm, bottom=0.95cm]{geometry} % okraje stranky
\usepackage[landscape, a4paper, mag=1400, truedimen, left=0.5cm, right=0.5cm, top=0.5cm, bottom=0.5cm]{geometry} % okraje stranky

\usepackage{fontspec}
\setmainfont[FeatureFile={junicode.fea}, Ligatures={Common, TeX}, RawFeature=+fixi]{Junicode}
%\setmainfont{Junicode}

% shortcut for Junicode without ligatures (for the Czech texts)
\newfontfamily\nlfont[FeatureFile={junicode.fea}, Ligatures={Common, TeX}, RawFeature=+fixi]{Junicode}

\usepackage{multicol}
\usepackage{color}
\usepackage{lettrine}
\usepackage{fancyhdr}

% usual packages loading:
\usepackage{luatextra}
\usepackage{graphicx} % support the \includegraphics command and options
\usepackage{gregoriotex} % for gregorio score inclusion
\usepackage{gregoriosyms}
\usepackage{wrapfig} % figures wrapped by the text
\usepackage{parcolumns}
\usepackage[contents={},opacity=1,scale=1,color=black]{background}
\usepackage{tikzpagenodes}
\usepackage{calc}
\usepackage{longtable}
\usetikzlibrary{calc}

\setlength{\headheight}{14.5pt}

\input{conventuscommune.tex} % Often used macros

\newcommand{\annusEditionis}{2021}

%%%% Vicekrat opakovane kousky

\newcommand{\anteOrationem}{
  \rubrica{Ante Orationem, cantatur a Superiore:}

  \pars{Supplicatio Litaniæ.}

  \cuminitiali{}{temporalia/supplicatiolitaniae.gtex}

  \pars{Oratio Dominica.}

  \cuminitiali{}{temporalia/oratiodominica.gtex}

  \rubrica{Deinde dicitur ab Hebdomadario:}

  \cuminitiali{}{temporalia/dominusvobiscum-solemnis.gtex}

  \rubrica{In choro monialium loco Dominus vobiscum dicitur:}

  \sineinitiali{temporalia/domineexaudi.gtex}
}

\setlength{\columnsep}{30pt} % prostor mezi sloupci

%%%%%%%%%%%%%%%%%%%%%%%%%%%%%%%%%%%%%%%%%%%%%%%%%%%%%%%%%%%%%%%%%%%%%%%%%%%%%%%%%%%%%%%%%%%%%%%%%%%%%%%%%%%%%
\begin{document}

% Here we set the space around the initial.
% Please report to http://home.gna.org/gregorio/gregoriotex/details for more details and options
\grechangedim{afterinitialshift}{2.2mm}{scalable}
\grechangedim{beforeinitialshift}{2.2mm}{scalable}
\grechangedim{interwordspacetext}{0.22 cm plus 0.15 cm minus 0.05 cm}{scalable}%
\grechangedim{annotationraise}{-0.2cm}{scalable}

% Here we set the initial font. Change 38 if you want a bigger initial.
% Emit the initials in red.
\grechangestyle{initial}{\color{red}\fontsize{38}{38}\selectfont}

\pagestyle{empty}

%%%% Titulni stranka
\begin{titulusOfficii}
\ifx\titulus\undefined
\nomenFesti{Feria V \hebdomada{}}
\else
\titulus
\fi
\end{titulusOfficii}

\vfill

\begin{center}
%Ad usum et secundum consuetudines chori \guillemotright{}Conventus Choralis\guillemotleft.

%Editio Sancti Wolfgangi \annusEditionis
\end{center}

\scriptura{}

\pars{}

\pagebreak

\renewcommand{\headrulewidth}{0pt} % no horiz. rule at the header
\fancyhf{}
\pagestyle{fancy}

\cantusSineNeumas

\ifx\oratio\undefined
\ifx\lauda\undefined
\else
\newcommand{\oratio}{\pars{Oratio.}

\noindent Omnípotens sempitérne Deus, véspere, mane et merídie maiestátem tuam supplíciter deprecámur, ut, expúlsis de córdibus nostris peccatórum ténebris, ad veram lucem, quæ Christus est, nos fácias perveníre.

\noindent Qui tecum vivit et regnat in unitáte Spíritus Sancti, Deus, per ómnia sǽcula sæculórum.

\noindent \Rbardot{} Amen.}
\fi
\fi

\hora{Ad Matutinum.} %%%%%%%%%%%%%%%%%%%%%%%%%%%%%%%%%%%%%%%%%%%%%%%%%%%%%
%\sideThumbs{Matutinum}

\vspace{2mm}

\cuminitiali{}{temporalia/dominelabiamea.gtex}

\vfill
%\pagebreak

\vspace{2mm}

\ifx\invitatorium\undefined
\pars{Invitatorium.} \scriptura{Ps. 94, 6; Psalmus 94; \textbf{H136}}

\vspace{-6mm}

\antiphona{E}{temporalia/inv-adoremusdominum.gtex}
\else
\invitatorium
\fi

\vfill
\pagebreak

\ifx\hymnusmatutinum\undefined
\ifx\matuac\undefined
\else
\pars{Hymnus.} \scriptura{Gregorius Magnus (+604)}

{
\grechangedim{interwordspacetext}{0.10 cm plus 0.15 cm minus 0.05 cm}{scalable}%
\antiphona{IV}{temporalia/hym-NoxAtra.gtex}
\grechangedim{interwordspacetext}{0.22 cm plus 0.15 cm minus 0.05 cm}{scalable}%
}
\fi
\else
\hymnusmatutinum
\fi

\vspace{-3mm}

\vfill
\pagebreak

\ifx\matua\undefined
\else
% MAT A
\pars{Psalmus 1.} \scriptura{Ps. 17, 3; \textbf{H99}}

\vspace{-4mm}

\antiphona{VIII G}{temporalia/ant-dominusfirmamentum.gtex}

%\vspace{-2mm}

\scriptura{Ps. 17, 31-35}

%\vspace{-2mm}

\initiumpsalmi{temporalia/ps17xxxi_xxxv-initium-viii-G-auto.gtex}

\input{temporalia/ps17xxxi_xxxv-viii-G.tex} \Abardot{}

\vfill
\pagebreak

\pars{Psalmus 2.} \scriptura{Ps. 62, 9; \textbf{H393}}

\vspace{-4mm}

\antiphona{VII c trans.}{temporalia/ant-mesuscepit.gtex}

%\vspace{-2mm}

\scriptura{Ps. 17, 36-46}

%\vspace{-2mm}

\initiumpsalmi{temporalia/ps17xxxvi_xlvi-initium-vii-c-trans.gtex}

\input{temporalia/ps17xxxvi_xlvi-vii-c.tex} \Abardot{}

\vfill
\pagebreak

\pars{Psalmus 3.} \scriptura{Ps. 17, 47; \textbf{H100}}

\vspace{-4mm}

\antiphona{VII c\textsuperscript{2}}{temporalia/ant-vivitdominus.gtex}

%\vspace{-2mm}

\scriptura{Ps. 17, 47-51}

%\vspace{-2mm}

\initiumpsalmi{temporalia/ps17xlvii_li-initium-vii-c2-auto.gtex}

\input{temporalia/ps17xlvii_li-vii-c2.tex} \Abardot{}

\vfill
\pagebreak
\fi
\ifx\matuc\undefined
\else
% MAT C
\pars{Psalmus 1.} \scriptura{Lam. 1, 21; \textbf{H177}}

\vspace{-4mm}

\antiphona{VII a}{temporalia/ant-omnesinimici.gtex}

%\vspace{-2mm}

\scriptura{Ps. 88, 39-46}

%\vspace{-2mm}

\initiumpsalmi{temporalia/ps88xxxix_xlvi-initium-vii-a-auto.gtex}

\input{temporalia/ps88xxxix_xlvi-vii-a.tex} \Abardot{}

\vfill
\pagebreak

\pars{Psalmus 2.} \scriptura{Ps. 88, 53; \textbf{H98}}

\vspace{-4mm}

\antiphona{VI F}{temporalia/ant-benedictusdominusinaeternum.gtex}

%\vspace{-2mm}

\scriptura{Ps. 88, 47-53}

%\vspace{-2mm}

\initiumpsalmi{temporalia/ps88xlvii_liii-initium-vi-F-auto.gtex}

\input{temporalia/ps88xlvii_liii-vi-F.tex} \Abardot{}

\vfill
\pagebreak

\pars{Psalmus 3.} \scriptura{Ps. 89, 13}

\vspace{-4mm}

\antiphona{I g}{temporalia/ant-converteredomine.gtex}

%\vspace{-2mm}

\scriptura{Ps. 89}

%\vspace{-2mm}

\initiumpsalmi{temporalia/ps89-initium-i-g-auto.gtex}

\input{temporalia/ps89-i-g.tex}

\vfill

\antiphona{}{temporalia/ant-converteredomine.gtex}

\vfill
\pagebreak
\fi

\pars{Versus.}

\ifx\matversus\undefined
\ifx\matua\undefined
\else
\noindent \Vbardot{} Révela, Dómine, óculos meos.

\noindent \Rbardot{} Et considerábo mirabília de lege tua.
\fi
\ifx\matuc\undefined
\else
\noindent \Vbardot{} Audies de ore meo verbum.

\noindent \Rbardot{} Et annuntiábis eis ex me.
\fi
\else
\matversus
\fi

\vspace{5mm}

\sineinitiali{temporalia/oratiodominica-mat.gtex}

\vspace{5mm}

\pars{Absolutio.}

\cuminitiali{}{temporalia/absolutio-exaudi.gtex}

\vfill
\pagebreak

\cuminitiali{}{temporalia/benedictio-solemn-benedictione.gtex}

\vspace{7mm}

\lectioi

\noindent \Vbardot{} Tu autem, Dómine, miserére nobis.
\noindent \Rbardot{} Deo grátias.

\vfill
\pagebreak

\responsoriumi

\vfill
\pagebreak

\cuminitiali{}{temporalia/benedictio-solemn-unigenitus.gtex}

\vspace{7mm}

\lectioii

\noindent \Vbardot{} Tu autem, Dómine, miserére nobis.
\noindent \Rbardot{} Deo grátias.

\vfill
\pagebreak

\responsoriumii

\vfill
\pagebreak

\cuminitiali{}{temporalia/benedictio-solemn-spiritus.gtex}

\vspace{7mm}

\lectioiii

\noindent \Vbardot{} Tu autem, Dómine, miserére nobis.
\noindent \Rbardot{} Deo grátias.

\vfill
\pagebreak

\responsoriumiii

\vfill
\pagebreak

\rubrica{Reliqua omittuntur, nisi Laudes separandæ sint.}

\sineinitiali{temporalia/domineexaudi.gtex}

\vfill

\oratio

\vfill

\noindent \Vbardot{} Dómine, exáudi oratiónem meam.
\Rbardot{} Et clamor meus ad te véniat.

\vfill

\noindent \Vbardot{} Benedicámus Dómino.
\noindent \Rbardot{} Deo grátias.

\vfill

\noindent \Vbardot{} Fidélium ánimæ per misericórdiam Dei requiéscant in pace.
\Rbardot{} Amen.

\vfill
\pagebreak

\hora{Ad Laudes.} %%%%%%%%%%%%%%%%%%%%%%%%%%%%%%%%%%%%%%%%%%%%%%%%%%%%%
%\sideThumbs{Laudes}

\cantusSineNeumas

\vspace{0.5cm}
\grechangedim{interwordspacetext}{0.18 cm plus 0.15 cm minus 0.05 cm}{scalable}%
\cuminitiali{}{temporalia/deusinadiutorium-communis.gtex}
\grechangedim{interwordspacetext}{0.22 cm plus 0.15 cm minus 0.05 cm}{scalable}%

\vfill
\pagebreak

\ifx\hymnuslaudes\undefined
\ifx\laudac\undefined
\else
\pars{Hymnus}

\grechangedim{interwordspacetext}{0.16 cm plus 0.15 cm minus 0.05 cm}{scalable}%
\cuminitiali{I}{temporalia/hym-SolEcce.gtex}
\grechangedim{interwordspacetext}{0.22 cm plus 0.15 cm minus 0.05 cm}{scalable}%
\vspace{-3mm}
\fi
\else
\hymnuslaudes
\fi

\vfill
\pagebreak

\ifx\lauda\undefined
\else
\pars{Psalmus 1.}

\vspace{-4mm}

\antiphona{VIII G}{temporalia/ant-exsurgamdiluculo.gtex}

%\vspace{-2mm}

\scriptura{Psalmus 56}

%\vspace{-2mm}

\initiumpsalmi{temporalia/ps56-initium-viii-g-auto.gtex}

%\vspace{-1.5mm}

\input{temporalia/ps56-viii-g.tex} \Abardot{}

\vfill
\pagebreak

\pars{Psalmus 2.} \scriptura{Ier. 31, 14}

\vspace{-4mm}

\antiphona{IV* e}{temporalia/ant-populusmeusait.gtex}

%\vspace{-2mm}

\scriptura{Canticum Ieremiæ, 1 Ier. 31, 10-14}

%\vspace{-3mm}

\initiumpsalmi{temporalia/jeremiae3-initium-iv_-e-auto.gtex}

\input{temporalia/jeremiae3-iv_-e.tex} \Abardot{}

\vfill
\pagebreak

\pars{Psalmus 3.} \scriptura{Ps. 95, 4; \textbf{H94}}

\vspace{-4mm}

\antiphona{IV a}{temporalia/ant-magnusdominus.gtex}

\scriptura{Psalmus 47}

\initiumpsalmi{temporalia/ps47-initium-iv-a-auto.gtex}

\input{temporalia/ps47-iv-a.tex} \Abardot{}

\vfill
\pagebreak
\fi
\ifx\laudc\undefined
\else
\pars{Psalmus 1.} \scriptura{Ps. 86, 1; \textbf{H98}}

\vspace{-4mm}

\antiphona{I g}{temporalia/ant-fundamentaeius.gtex}

%\vspace{-2mm}

\scriptura{Psalmus 86}

%\vspace{-2mm}

\initiumpsalmi{temporalia/ps86-initium-i-g-auto.gtex}

%\vspace{-1.5mm}

\input{temporalia/ps86-i-g.tex} \Abardot{}

\vfill
\pagebreak

\pars{Psalmus 2.}

\vspace{-4mm}

\antiphona{II D}{temporalia/ant-eccedominusnosterbrachio.gtex}

%\vspace{-2mm}

\scriptura{Canticum Isaiæ, Is. 40, 10-17}

%\vspace{-3mm}

\initiumpsalmi{temporalia/isaiae9-initium-ii-D-auto.gtex}

\input{temporalia/isaiae9-ii-D.tex} \Abardot{}

\vfill
\pagebreak

\pars{Psalmus 3.} \scriptura{Ps. 144, 17}

\vspace{-4mm}

\antiphona{E}{temporalia/ant-iustusetsanctus.gtex}

\scriptura{Psalmus 98}

\initiumpsalmi{temporalia/ps98-initium-e.gtex}

\input{temporalia/ps98-e.tex} \Abardot{}

\vfill
\pagebreak
\fi

\ifx\lectiobrevis\undefined
\ifx\lauda\undefined
\else
\pars{Lectio Brevis.} \scriptura{Is. 66, 1-2}

\noindent Hæc dicit Dóminus: Cælum thronus meus, terra autem scabéllum pedum meórum. Quæ ista domus, quam ædificábitis mihi, et quis iste locus quiétis meæ? Omnia hæc manus mea fecit et mea sunt univérsa ista, dicit Dóminus. Ad hunc autem respíciam, ad paupérculum et contrítum spíritu et treméntem sermónes meos.
\fi
\else
\lectiobrevis
\fi

\vfill

\ifx\responsoriumbreve\undefined
\ifx\laudac\undefined
\else
\pars{Responsorium breve.} \scriptura{Ps. 118, 145}

\cuminitiali{VI}{temporalia/resp-clamaviintotocorde.gtex}
\fi
\else
\responsoriumbreve
\fi

\vfill
\pagebreak

\ifx\benedictus\undefined
\ifx\laudac\undefined
\else
\pars{Canticum Zachariæ.} \scriptura{Lc. 1, 74.75; \textbf{H423}}

%\vspace{-4mm}

{
\grechangedim{interwordspacetext}{0.18 cm plus 0.15 cm minus 0.05 cm}{scalable}%
\antiphona{VII a}{temporalia/ant-insanctitate.gtex}
\grechangedim{interwordspacetext}{0.22 cm plus 0.15 cm minus 0.05 cm}{scalable}%
}

%\vspace{-3mm}

\scriptura{Lc. 1, 68-79}

%\vspace{-2mm}

\cantusSineNeumas
\initiumpsalmi{temporalia/benedictus-initium-vii-a-auto.gtex}

%\vspace{-1.5mm}

\input{temporalia/benedictus-vii-a.tex} \Abardot{}
\fi
\else
\benedictus
\fi

\vspace{-1cm}

\vfill
\pagebreak

%\sideThumbs{{\scriptsize{}Fine horarum}}

\pars{Preces.}

\sineinitiali{}{temporalia/tonusprecum.gtex}

\ifx\preces\undefined
\ifx\lauda\undefined
\else
\noindent Grátias agámus Christo, qui lumen huius diéi nobis concédit, \gredagger{} et ad eum clamémus:

\Rbardot{} Bénedic et sanctífica nos, Dómine.

\noindent Qui te pro peccátis nostris hóstiam obtulísti, \gredagger{} incépta et propósita suscípias hodiérna.

\Rbardot{} Bénedic et sanctífica nos, Dómine.

\noindent Qui óculos nostros lucis dono lætíficas novæ, \gredagger{} lúcifer oriáris in córdibus nostris.

\Rbardot{} Bénedic et sanctífica nos, Dómine.

\noindent Tríbue hódie nos esse ómnibus longánimes, \gredagger{} ut imitatóres tui fíeri possímus.

\Rbardot{} Bénedic et sanctífica nos, Dómine.

\noindent Audítam, Dómine, fac nobis mane misericórdiam tuam. \gredagger{} Sit hódie gáudium tuum fortitúdo nostra.

\Rbardot{} Bénedic et sanctífica nos, Dómine.
\fi
\ifx\laudc\undefined
\else
\noindent Christo, bono pastóri, qui pro suis óvibus ánimam pósuit, \gredagger{} laudes grati exsolvámus et supplicémus, dicéntes:

\Rbardot{} Pasce pópulum tuum, Dómine.

\noindent Christe, qui in sanctis pastóribus misericórdiam et dilectiónem tuam dignátus es osténdere, \gredagger{} numquam désinas per eos nobíscum misericórditer ágere.

\Rbardot{} Pasce pópulum tuum, Dómine.

\noindent Qui múnere pastóris animárum fungi per tuos vicários pergis, \gredagger{} ne destíteris nos ipse per rectóres nostros dirígere.

\Rbardot{} Pasce pópulum tuum, Dómine.

\noindent Qui in sanctis tuis, populórum dúcibus, córporum animarúmque médicus exstitísti, \gredagger{} numquam cesses ministérium in nos vitæ et sanctitátis perágere.

\Rbardot{} Pasce pópulum tuum, Dómine.

\noindent Qui, prudéntia et caritáte sanctórum, tuum gregem erudísti, \gredagger{} nos in sanctitáte iúgiter per pastóres nostros ædífica.

\Rbardot{} Pasce pópulum tuum, Dómine.
\fi
\else
\preces
\fi

\vfill

\pars{Oratio Dominica.}

\cuminitiali{}{temporalia/oratiodominicaalt.gtex}

\vfill
\pagebreak

\rubrica{vel:}

\pars{Supplicatio Litaniæ.}

\cuminitiali{}{temporalia/supplicatiolitaniae.gtex}

\vfill

\pars{Oratio Dominica.}

\cuminitiali{}{temporalia/oratiodominica.gtex}

\vfill
\pagebreak

% Oratio. %%%
\oratio

\vspace{-1mm}

\vfill

\rubrica{Hebdomadarius dicit Dominus vobiscum, vel, absente sacerdote vel diacono, sic concluditur:}

\vspace{2mm}

\antiphona{C}{temporalia/dominusnosbenedicat.gtex}

\rubrica{Postea cantatur a cantore:}

\vspace{2mm}

\cuminitiali{IV}{temporalia/benedicamus-feria-laudes.gtex}

\vspace{1mm}

\vfill
\pagebreak

\end{document}

