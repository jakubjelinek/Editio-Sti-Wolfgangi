\newcommand{\oratio}{\pars{Oratio.}

\noindent Deus, qui errántibus, ut in viam possint redíre, veritátis tuæ lumen osténdis, da cunctis, qui christiána professióne censéntur, et illa respúere, quæ huic inimíca sunt nómini, et ea, quæ sunt apta, sectári.

\pars{Pro commemoratione Sancti Fidelis de Sigmaringen, Presbyteri \& Martyris.} \scriptura{Ps. 1, 1; \textbf{H254}}

\vspace{-4mm}

\antiphona{III a}{temporalia/ant-istesanctusprolege.gtex}

\vfill

\noindent Deus, qui beátum Fidélem, amóre tuo succénsum, in fídei propagatióne martýrii palma decoráre dignátus es, ipso interveniénte, concéde, ut, in caritáte fundáti, cum illo resurrectiónis Christi virtútem cognóscere mereámur.

\pars{Pro pace in universo mundo.} \scriptura{Sir. 50, 25; 2 Esdr. 4, 20; \textbf{H416}}

\vspace{-4mm}

\antiphona{II D}{temporalia/ant-dapacemdomine.gtex}

\vfill

\noindent Deus, a quo sancta desidéria, recta consília et iusta sunt ópera: da servis tuis illam, quam mundus dare non potest, pacem; ut et corda nostra mandátis tuis dédita, et hóstium subláta formídine, témpora sint tua protectióne tranquílla.

\noindent Per Dóminum nostrum Iesum Christum, Fílium tuum, qui tecum vivit et regnat in unitáte Spíritus Sancti, Deus, per ómnia sǽcula sæculórum.

\noindent \Rbardot{} Amen.}
\newcommand{\lectioi}{\pars{Lectio I.} \scriptura{Ap. 7, 1-8}

\noindent De libro Apocalýpsis beáti Ioánnis apóstoli.

\noindent Ego Ioánnes vidi quáttuor ángelos stantes super quáttuor ángulos terræ tenéntes quáttuor ventos terræ, ne flaret ventus super terram neque super mare neque in ullam árborem. Et vidi álterum ángelum ascendéntem ab ortu solis, habéntem sigíllum Dei vivi, et clamávit voce magna quáttuor ángelis, quibus datum est nocére terræ et mari, dicens: «Nolíte nocére terræ neque mari neque arbóribus, quoadúsque signémus servos Dei nostri in fróntibus eórum».

\noindent Et audívi númerum signatórum, centum quadragínta quáttuor mília, signáti ex omni tribu filiórum Israel:

\noindent {\color{gray} ex tribu Iudæ duódecim mília signáti, ex tribu Ruben duódecim mília, ex tribu Gad duódecim mília, ex tribu Aser duódecim mília, ex tribu Néphthali duódecim mília, ex tribu Manásse duódecim mília, ex tribu Símeon duódecim mília, ex tribu Levi duódecim mília, ex tribu Issachar duódecim mília, ex tribu Zábulon duódecim mília, ex tribu Ioseph duódecim mília, ex tribu Béniamin duódecim mília signáti.}}
\newcommand{\responsoriumi}{\pars{Responsorium 1.} \scriptura{\Rbardot{} Sap. 18, 24 \Vbardot{} Sir. 45, 14; \textbf{H249}}

\vspace{-5mm}

\responsorium{III}{temporalia/resp-indiademate-CROCHU.gtex}{}}
\newcommand{\lectioii}{\pars{Lectio II.} \scriptura{Ap. 7, 9-17}

\noindent \noindent Post hæc vidi: et ecce turba magna, quam dinumeráre nemo póterat, ex ómnibus géntibus et tríbubus et pópulis et linguis stantes ante thronum et in conspéctu Agni, amícti stolis albis, et palmæ in mánibus eórum; et clamant voce magna dicéntes: «Salus Deo nostro, qui sedet super thronum, et Agno».

\noindent Et omnes ángeli stabant in circúitu throni et seniórum et quáttuor animálium, et cecidérunt in conspéctu throni in fácies suas et adoravérunt Deum dicéntes: «Amen! Benedíctio et glória et sapiéntia et gratiárum áctio et honor et virtus et fortitúdo Deo nostro in sǽcula sæculórum. Amen».

\noindent Et respóndit unus de senióribus dicens mihi: «Hi qui amícti sunt stolis albis, qui sunt et unde venérunt?». Et dixi illi: «Dómine mi, tu scis». Et dixit mihi: «Hi sunt qui véniunt de tribulatióne magna et lavérunt stolas suas et dealbavérunt eas in sánguine Agni. Ideo sunt ante thronum Dei et sérviunt ei die ac nocte in templo eius; et, qui sedet in throno, habitábit super illos. \emph{Non esúrient} ámplius \emph{neque sítient} ámplius, \emph{neque cadet super illos sol neque} ullus \emph{æstus,} quóniam Agnus, qui in médio throni est, \emph{pascet illos et dedúcet eos ad vitæ fontes aquárum, et abstérget Deus omnem lácrimam} ex óculis eórum».}
\newcommand{\responsoriumii}{\pars{Responsorium 2.} \scriptura{\Rbardot{} Ap. 19, 6 \& Ap. 12, 10 \Vbardot{} ibid. 14, 6-7; \textbf{H247}}

\vspace{-5mm}

\responsorium{VII}{temporalia/resp-audivivocemincaelotamquam-CROCHU.gtex}{}

\vfill

\rubrica{vel ad libitum:}

\vspace{3mm}

\pars{Responsorium 2.} \scriptura{\Vbardot{} Ap. 7, 14; \textbf{H362}}

\vspace{-5mm}

\responsorium{IV}{temporalia/resp-istisunttriumphatores-sinedox.gtex}{}}
\newcommand{\lectioiii}{\pars{Lectio III.} \scriptura{Cap. 2: PL 93, 50-51}

\noindent Ex Commentário sancti Bedæ Venerábilis presbýteri in primam Epístolam Petri.

\noindent {\color{gray} \emph{Vos autem genus eléctum, regále sacerdótium.} Hoc laudis testimónium quondam antíquo Dei pópulo per Móysen datum est, quod nunc recte géntibus dat apóstolus Petrus, quia vidélicet in Christum credidérunt, qui velut lapis anguláris in eam, quam in se Israel habúerat, salútem gentes adunávit.

\noindent Quas \emph{genus eléctum} vocat propter fidem, ut distínguat ab eis, qui, lápidem vivum reprobándo, facti sunt ipsi réprobi.

\noindent \emph{Regále} autem \emph{sacerdótium}, quia illíus córpori sunt uníti, qui rex summus et sacérdos est verus, regnum suis tríbuens ut rex, et ut póntifex eórum peccáta sui sánguinis hóstia mundans.

\noindent \emph{Regále sacerdótium} eos nóminat, ut et regnum speráre perpétuum, et hóstias immaculátæ conversatiónis Deo semper offérre memínerint.

\noindent \emph{Gens} quoque \emph{sancta et pópulus acquisitiónis} vocántur, iuxta id quod apóstolus Paulus, prophétæ senténtiam expónens, dicit: \emph{Iustus autem meus ex fide vivit; quod si subtráxerit se, non placébit ánimæ meæ; nos autem,} inquit, \emph{non sumus subtractiónis in perditiónem, sed fídei in acquisitiónem ánimæ.} Et in Actibus Apostolórum, \emph{Vos Spíritus Sanctus pósuit epíscopos régere Ecclésiam Dómini, quam acquisívit sánguine suo}.}

\noindent \emph{Pópulus} ergo \emph{acquisitiónis} facti sumus in sánguine nostri Redemptóris, quod erat quondam pópulus Israel redémptus sánguine agni de Ægýpto.

\noindent Unde et in sequénti quoque versículo mýstice véteris recordátus históriæ, et hanc étiam novo Dei pópulo spiritáliter docet impléndam dicens: \emph{Ut virtútes annuntiétis eius.} Sicut enim hi, qui de Ægýptia servitúte liberáti sunt per Móysen, carmen triumphále post tránsitum maris Rubri et demérsum Pharaónis exércitum Dómino decantárunt, ita et nos opórtet post accéptam in baptísmo remissiónem peccatórum dignas benefíciis cæléstibus repéndere grátias.

\noindent Namque Ægýptii, qui pópulum Dei affligébant, qui étiam ténebræ vel tribulatiónes interpretántur, apte per sequéntia nos peccáta, sed in baptísmate deléta, signíficant.

\noindent Liberátio quoque filiórum Israel, et ad promíssam olim pátriam perdúctio, cóngruit mystério nostræ redemptiónis, per quam ad lucem supérnæ mansiónis, illustránte nos ac ducénte Christi grátia, téndimus; cuius lucem grátiæ étiam illa nubis et ignis colúmna monstrávit; quæ eos et in toto itínere illo a ténebris deféndit nóctium, et ad promíssas pátriæ sedes inenarrábili calle perdúxit.}
\newcommand{\responsoriumiii}{\pars{Responsorium 3.} \scriptura{\Rbardot{} Cf. Ct. 4, 11; \textbf{H249}}

\vspace{-5mm}

\responsorium{III}{temporalia/resp-veniensalibano-CROCHU-cumdox.gtex}{}}
\newcommand{\benedictus}{\pars{Canticum Zachariæ.} \scriptura{Io. 6, 27}

%\vspace{-4mm}

{
\grechangedim{interwordspacetext}{0.18 cm plus 0.15 cm minus 0.05 cm}{scalable}%
\antiphona{VIII G}{temporalia/ant-operamininoncibum.gtex}
\grechangedim{interwordspacetext}{0.22 cm plus 0.15 cm minus 0.05 cm}{scalable}%
}

%\vspace{-3mm}

\scriptura{Lc. 1, 68-79}

%\vspace{-2mm}

\cantusSineNeumas
\initiumpsalmi{temporalia/benedictus-initium-viii-G-auto.gtex}

%\vspace{-1.5mm}

\input{temporalia/benedictus-viii-G.tex} \Abardot{}}
\newcommand{\hebdomada}{infra Hebdom. III Adventus.}
\newcommand{\oratioLaudes}{\cuminitiali{}{temporalia/oratio3vo.gtex}}
\newcommand{\responsoriumbreve}{\pars{Responsorium breve.} \scriptura{Is. 60, 2; \textbf{H20}}

\cuminitiali{IV}{temporalia/resp-superte.gtex}}

% LuaLaTeX

\documentclass[a4paper, twoside, 12pt]{article}
\usepackage[latin]{babel}
%\usepackage[landscape, left=3cm, right=1.5cm, top=2cm, bottom=1cm]{geometry} % okraje stranky
%\usepackage[landscape, a4paper, mag=1166, truedimen, left=2cm, right=1.5cm, top=1.6cm, bottom=0.95cm]{geometry} % okraje stranky
\usepackage[landscape, a4paper, mag=1400, truedimen, left=0.5cm, right=0.5cm, top=0.5cm, bottom=0.5cm]{geometry} % okraje stranky

\usepackage{fontspec}
\setmainfont[FeatureFile={junicode.fea}, Ligatures={Common, TeX}, RawFeature=+fixi]{Junicode}
%\setmainfont{Junicode}

% shortcut for Junicode without ligatures (for the Czech texts)
\newfontfamily\nlfont[FeatureFile={junicode.fea}, Ligatures={Common, TeX}, RawFeature=+fixi]{Junicode}

\usepackage{multicol}
\usepackage{color}
\usepackage{lettrine}
\usepackage{fancyhdr}

% usual packages loading:
\usepackage{luatextra}
\usepackage{graphicx} % support the \includegraphics command and options
\usepackage{gregoriotex} % for gregorio score inclusion
\usepackage{gregoriosyms}
\usepackage{wrapfig} % figures wrapped by the text
\usepackage{parcolumns}
\usepackage[contents={},opacity=1,scale=1,color=black]{background}
\usepackage{tikzpagenodes}
\usepackage{calc}
\usepackage{longtable}
\usetikzlibrary{calc}

\setlength{\headheight}{14.5pt}

\input{conventuscommune.tex} % Often used macros

\newcommand{\annusEditionis}{2021}

%%%% Vicekrat opakovane kousky

\newcommand{\anteOrationem}{
  \rubrica{Ante Orationem, cantatur a Superiore:}

  \pars{Supplicatio Litaniæ.}

  \cuminitiali{}{temporalia/supplicatiolitaniae.gtex}

  \pars{Oratio Dominica.}

  \cuminitiali{}{temporalia/oratiodominica.gtex}

  \rubrica{Deinde dicitur ab Hebdomadario:}

  \cuminitiali{}{temporalia/dominusvobiscum-solemnis.gtex}

  \rubrica{In choro monialium loco Dominus vobiscum dicitur:}

  \sineinitiali{temporalia/domineexaudi.gtex}
}

\setlength{\columnsep}{30pt} % prostor mezi sloupci

%%%%%%%%%%%%%%%%%%%%%%%%%%%%%%%%%%%%%%%%%%%%%%%%%%%%%%%%%%%%%%%%%%%%%%%%%%%%%%%%%%%%%%%%%%%%%%%%%%%%%%%%%%%%%
\begin{document}

% Here we set the space around the initial.
% Please report to http://home.gna.org/gregorio/gregoriotex/details for more details and options
\grechangedim{afterinitialshift}{2.2mm}{scalable}
\grechangedim{beforeinitialshift}{2.2mm}{scalable}
\grechangedim{interwordspacetext}{0.22 cm plus 0.15 cm minus 0.05 cm}{scalable}%
\grechangedim{annotationraise}{-0.2cm}{scalable}

% Here we set the initial font. Change 38 if you want a bigger initial.
% Emit the initials in red.
\grechangestyle{initial}{\color{red}\fontsize{38}{38}\selectfont}

\pagestyle{empty}

%%%% Titulni stranka
\begin{titulusOfficii}
\ifx\titulus\undefined
\nomenFesti{Feria II \hebdomada{}}
\else
\titulus
\fi
\end{titulusOfficii}

\vfill

\begin{center}
%Ad usum et secundum consuetudines chori \guillemotright{}Conventus Choralis\guillemotleft.

%Editio Sancti Wolfgangi \annusEditionis
\end{center}

\scriptura{}

\pars{}

\pagebreak

\renewcommand{\headrulewidth}{0pt} % no horiz. rule at the header
\fancyhf{}
\pagestyle{fancy}

\cantusSineNeumas

\ifx\oratio\undefined
\ifx\laudb\undefined
\else
\newcommand{\oratio}{\pars{Oratio.}

\noindent Dómine Deus omnípotens, qui ad princípium huius diéi nos perveníre fecísti, tua nos hódie salva virtúte, ut in hac die ad nullum declinémus peccátum, sed semper ad tuam iustítiam faciéndam nostra procédant elóquia, dirigántur cogitatiónes et ópera.

\noindent Per Dóminum nostrum Iesum Christum, Fílium tuum, qui tecum vivit et regnat in unitáte Spíritus Sancti, Deus, per ómnia sǽcula sæculórum.

\noindent \Rbardot{} Amen.}
\fi
\fi

\hora{Ad Matutinum.} %%%%%%%%%%%%%%%%%%%%%%%%%%%%%%%%%%%%%%%%%%%%%%%%%%%%%
%\sideThumbs{Matutinum}

\vspace{2mm}

\cuminitiali{}{temporalia/dominelabiamea.gtex}

\vfill
%\pagebreak

\vspace{2mm}

\ifx\invitatorium\undefined
\pars{Invitatorium.} \scriptura{Ps. 94, 1; Psalmus 94; \textbf{H451}}

\vspace{-6mm}

\antiphona{VI}{temporalia/inv-jubilemusdeo.gtex}\else
\invitatorium
\fi

\vfill
\pagebreak

\ifx\hymnusmatutinum\undefined
\ifx\matua\undefined
\else
\pars{Hymnus.}

{
\grechangedim{interwordspacetext}{0.10 cm plus 0.15 cm minus 0.05 cm}{scalable}%
\antiphona{II}{temporalia/hym-IpsumNunc.gtex}
\grechangedim{interwordspacetext}{0.22 cm plus 0.15 cm minus 0.05 cm}{scalable}%
}
\fi
\else
\hymnusmatutinum
\fi

\vspace{-3mm}

\vfill
\pagebreak

\ifx\matub\undefined
\else
% MAT B
\pars{Psalmus 1.} \scriptura{Ps. 30, 2; \textbf{H90}}

\vspace{-4mm}

\antiphona{VIII G}{temporalia/ant-intuaiustitia.gtex}

%\vspace{-2mm}

\scriptura{Ps. 30, 2-9}

%\vspace{-2mm}

\initiumpsalmi{temporalia/ps30i-initium-viii-G-auto.gtex}

\vspace{-1.5mm}

\input{temporalia/ps30i-viii-G.tex} \Abardot{}

\vfill
\pagebreak

\pars{Psalmus 2.} \scriptura{Ps. 66, 2}

\vspace{-4mm}

\antiphona{E}{temporalia/ant-illuminadomine.gtex}

%\vspace{-2mm}

\scriptura{Ps. 30, 10-17}

%\vspace{-2mm}

\initiumpsalmi{temporalia/ps30ii-initium-e-a-auto.gtex}

\input{temporalia/ps30ii-e-a.tex} \Abardot{}

\vfill
\pagebreak

\pars{Psalmus 3.} \scriptura{Ps. 30, 24}

\vspace{-4mm}

\antiphona{II D}{temporalia/ant-diligitedominum.gtex}

%\vspace{-5mm}

\scriptura{Ps. 30, 20-25}

%\vspace{-2mm}

\initiumpsalmi{temporalia/ps30iii-initium-ii-D-auto.gtex}

\input{temporalia/ps30iii-ii-D.tex} \Abardot{}

\vfill
\pagebreak
\fi

\pars{Versus.}

\ifx\matversus\undefined
\ifx\matub\undefined
\else
\noindent \Vbardot{} Dírige me, Dómine, in veritáte tua, et doce me.

\noindent \Rbardot{} Quia tu es Deus salútis meæ.
\fi
\else
\matversus
\fi

\vspace{5mm}

\sineinitiali{temporalia/oratiodominica-mat.gtex}

\vspace{5mm}

\pars{Absolutio.}

\cuminitiali{}{temporalia/absolutio-exaudi.gtex}

\vfill
\pagebreak

\cuminitiali{}{temporalia/benedictio-solemn-benedictione.gtex}

\vspace{7mm}

\lectioi

\noindent \Vbardot{} Tu autem, Dómine, miserére nobis.
\noindent \Rbardot{} Deo grátias.

\vfill
\pagebreak

\responsoriumi

\vfill
\pagebreak

\cuminitiali{}{temporalia/benedictio-solemn-unigenitus.gtex}

\vspace{7mm}

\lectioii

\noindent \Vbardot{} Tu autem, Dómine, miserére nobis.
\noindent \Rbardot{} Deo grátias.

\vfill
\pagebreak

\responsoriumii

\vfill
\pagebreak

\cuminitiali{}{temporalia/benedictio-solemn-spiritus.gtex}

\vspace{7mm}

\lectioiii

\noindent \Vbardot{} Tu autem, Dómine, miserére nobis.
\noindent \Rbardot{} Deo grátias.

\vfill
\pagebreak

\responsoriumiii

\vfill
\pagebreak

\rubrica{Reliqua omittuntur, nisi Laudes separandæ sint.}

\sineinitiali{temporalia/domineexaudi.gtex}

\vfill

\oratio

\vfill

\noindent \Vbardot{} Dómine, exáudi oratiónem meam.
\Rbardot{} Et clamor meus ad te véniat.

\vfill

\noindent \Vbardot{} Benedicámus Dómino.
\noindent \Rbardot{} Deo grátias.

\vfill

\noindent \Vbardot{} Fidélium ánimæ per misericórdiam Dei requiéscant in pace.
\Rbardot{} Amen.

\vfill
\pagebreak

\hora{Ad Laudes.} %%%%%%%%%%%%%%%%%%%%%%%%%%%%%%%%%%%%%%%%%%%%%%%%%%%%%
%\sideThumbs{Laudes}

\cantusSineNeumas

\vspace{0.5cm}
\grechangedim{interwordspacetext}{0.18 cm plus 0.15 cm minus 0.05 cm}{scalable}%
\cuminitiali{}{temporalia/deusinadiutorium-communis.gtex}
\grechangedim{interwordspacetext}{0.22 cm plus 0.15 cm minus 0.05 cm}{scalable}%

\vfill
\pagebreak

\ifx\hymnuslaudes\undefined
\ifx\laudbd\undefined
\else
\pars{Hymnus} \scriptura{Hilarius (\olddag{} 367)}

\grechangedim{interwordspacetext}{0.16 cm plus 0.15 cm minus 0.05 cm}{scalable}%
\cuminitiali{IV}{temporalia/hym-LucisLargitor.gtex}
\grechangedim{interwordspacetext}{0.22 cm plus 0.15 cm minus 0.05 cm}{scalable}%
\vspace{-3mm}
\fi
\else
\hymnuslaudes
\fi

\vfill
\pagebreak

\ifx\laudb\undefined
\else
\pars{Psalmus 1.} \scriptura{Ps. 41, 3; \textbf{H391}}

\vspace{-4mm}

\antiphona{II D}{temporalia/ant-sitivitanima.gtex}

%\vspace{-2mm}

\scriptura{Psalmus 41}

%\vspace{-2mm}

\initiumpsalmi{temporalia/ps41-initium-ii-D-auto.gtex}

%\vspace{-1.5mm}

\input{temporalia/ps41-ii-D.tex}

\vfill

\antiphona{}{temporalia/ant-sitivitanima.gtex}

\vfill
\pagebreak

\pars{Psalmus 2.}

\vspace{-4mm}

\antiphona{III a}{temporalia/ant-ostendenobisdomine.gtex}

%\vspace{-2mm}

\scriptura{Canticum Ecclesiastici, Sir. 36, 1-7.13-16}

%\vspace{-3mm}

\initiumpsalmi{temporalia/ecclesiastici-initium-iii-a-auto.gtex}

\input{temporalia/ecclesiastici-iii-a.tex} \Abardot{}

\vfill
\pagebreak

\pars{Psalmus 3.}

\vspace{-4mm}

\antiphona{II D}{temporalia/ant-operamanuumeius.gtex}

\scriptura{Psalmus 18, 1-7}

\initiumpsalmi{temporalia/ps18i-initium-ii-D-auto.gtex}

\input{temporalia/ps18i-ii-D.tex} \Abardot{}

\vfill
\pagebreak
\fi

\ifx\lectiobrevis\undefined
\ifx\laudb\undefined
\else
\pars{Lectio Brevis.} \scriptura{Ier. 15, 16}

\noindent Invénti sunt sermónes tui, et comédi eos, et factum est mihi verbum tuum in gáudium et in lætítiam cordis mei, quóniam invocátum est nomen tuum super me, Dómine Deus exercítuum.
\fi
\else
\lectiobrevis
\fi

\vfill

\ifx\responsoriumbreve\undefined
\ifx\laudbd\undefined
\else
\pars{Responsorium breve.} \scriptura{Ps. 32, 1.3}

\cuminitiali{VI}{temporalia/resp-exsultateiusti.gtex}
\fi
\else
\responsoriumbreve
\fi

\vfill
\pagebreak

\ifx\benedictus\undefined
\ifx\laudbd\undefined
\else
\pars{Canticum Zachariæ.} \scriptura{Lc. 1, 68; \textbf{H422}}

\vspace{-4mm}

{
\grechangedim{interwordspacetext}{0.18 cm plus 0.15 cm minus 0.05 cm}{scalable}%
\antiphona{IV E}{temporalia/ant-benedictusdominus.gtex}
\grechangedim{interwordspacetext}{0.22 cm plus 0.15 cm minus 0.05 cm}{scalable}%
}

%\vspace{-3mm}

\scriptura{Lc. 1, 68-79}

%\vspace{-2mm}

\cantusSineNeumas
\initiumpsalmi{temporalia/benedictus-initium-iv-E-auto.gtex}

%\vspace{-1.5mm}

\input{temporalia/benedictus-iv-E.tex} \Abardot{}
\fi
\else
\benedictus
\fi

\vspace{-1cm}

\vfill
\pagebreak

%\sideThumbs{{\scriptsize{}Fine horarum}}

\pars{Preces.}

\sineinitiali{}{temporalia/tonusprecum.gtex}

\ifx\preces\undefined
\ifx\laudb\undefined
\else
\noindent Salvátor noster fecit nos regnum et sacerdótium, ut hóstias Deo acceptábiles offerámus. \gredagger{} Grati ígitur eum invocémus:

\Rbardot{} Serva nos in tuo ministério, Dómine.

\noindent Christe, sacérdos ætérne, qui sanctum pópulo tuo sacerdótium concessísti, \gredagger{} concéde, ut spiritáles hóstias Deo acceptábiles iúgiter offerámus.

\Rbardot{} Serva nos in tuo ministério, Dómine.

\noindent Spíritus tui fructus nobis largíre propítius, \gredagger{} patiéntiam, benignitátem et mansuetúdinem.

\Rbardot{} Serva nos in tuo ministério, Dómine.

\noindent Da nobis te amáre, ut te, qui es cáritas, possideámus, \gredagger{} et bene ágere, ut per vitam étiam nostram te laudémus.

\Rbardot{} Serva nos in tuo ministério, Dómine.

\noindent Quæ frátribus nostris sunt utília, nos quǽrere concéde, \gredagger{} ut salútem facílius consequántur.

\Rbardot{} Serva nos in tuo ministério, Dómine.
\fi
\else
\preces
\fi

\vfill

\pars{Oratio Dominica.}

\cuminitiali{}{temporalia/oratiodominicaalt.gtex}

\vfill
\pagebreak

\rubrica{vel:}

\pars{Supplicatio Litaniæ.}

\cuminitiali{}{temporalia/supplicatiolitaniae.gtex}

\vfill

\pars{Oratio Dominica.}

\cuminitiali{}{temporalia/oratiodominica.gtex}

\vfill
\pagebreak

% Oratio. %%%
\oratio

\vspace{-1mm}

\vfill

\rubrica{Hebdomadarius dicit Dominus vobiscum, vel, absente sacerdote vel diacono, sic concluditur:}

\vspace{2mm}

\antiphona{C}{temporalia/dominusnosbenedicat.gtex}

\rubrica{Postea cantatur a cantore:}

\vspace{2mm}

\cuminitiali{IV}{temporalia/benedicamus-feria-laudes.gtex}

\vspace{1mm}

\vfill
\pagebreak

\end{document}

