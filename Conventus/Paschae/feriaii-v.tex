\newcommand{\oratio}{\pars{Oratio.}

\noindent Deus, qui fidélium mentes uníus éfficis voluntátis, da pópulis tuis id amáre quod prǽcipis, id desideráre quod promíttis, ut, inter mundánas varietátes, ibi nostra fixa sint corda, ubi vera sunt gáudia.

\pars{Pro pace in Ucraina.} \scriptura{Sir. 50, 25; 2 Esdr. 4, 20; \textbf{H416}}

\vspace{-4mm}

\antiphona{II D}{temporalia/ant-dapacemdomine.gtex}

\vfill

\noindent Deus, a quo sancta desidéria, recta consília et iusta sunt ópera: da servis tuis illam, quam mundus dare non potest, pacem; ut et corda nostra mandátis tuis dédita, et hóstium subláta formídine, témpora sint tua protectióne tranquílla.

\noindent Per Dóminum nostrum Iesum Christum, Fílium tuum, qui tecum vivit et regnat in unitáte Spíritus Sancti, Deus, per ómnia sǽcula sæculórum.

\noindent \Rbardot{} Amen.}
\newcommand{\matutinum}{\pars{Psalmus 1.}

\vspace{-4mm}

\antiphona{VIII G\textsuperscript{3}}{temporalia/ant-alleluia-bv21-n4.gtex}

%\vspace{-2mm}

\scriptura{Ps. 6}

%\vspace{-2mm}

\initiumpsalmi{temporalia/ps6-initium-viii-G3.gtex}

\input{temporalia/ps6-viii-G2.tex}

\vfill
\pagebreak

\pars{Psalmus 2.}

\scriptura{Ps. 9, 2-11}

%\vspace{-2mm}

\initiumpsalmi{temporalia/ps9ii_xi-initium-viii-G3.gtex}

\input{temporalia/ps9ii_xi-viii-G2.tex}

\vfill
\pagebreak

\pars{Psalmus 3.}

\scriptura{Ps. 9, 12-21}

%\vspace{-2mm}

\initiumpsalmi{temporalia/ps9xii_xxi-initium-viii-G3.gtex}

\input{temporalia/ps9xii_xxi-viii-G2.tex}

\vfill

\antiphona{}{temporalia/ant-alleluia-bv21-n4.gtex}

\vfill
\pagebreak}
\newcommand{\lectioi}{\pars{Lectio I.} \scriptura{Ap. 19, 11-21}

\noindent De libro Apocalýpsis beáti Ioánnis apóstoli.

\noindent Ego Ioánnes vidi cælum apértum: et ecce equus albus; et, qui sedébat super eum, vocabátur Fidélis et Verax, et in iustítia iúdicat et pugnat.

\noindent Oculi autem eius sicut flamma ignis, et in cápite eius diadémata multa, habens nomen scriptum, quod nemo novit nisi ipse, et vestítus veste aspérsa sánguine, et vocátur nomen eius Verbum Dei. Et exércitus, qui sunt in cælo, sequebántur eum in equis albis, vestíti býssino albo mundo. Et de ore ipsíus procédit gládius acútus, ut in ipso percútiat gentes, et ipse \emph{reget eos in virga férrea}; et ipse calcat tórcular vini furóris iræ Dei omnipoténtis. Et habet super vestiméntum et super femur suum nomen scriptum: Rex regum et Dóminus dominórum.

\noindent Et vidi unum ángelum stantem in sole, et clamávit voce magna dicens ómnibus ávibus, quæ volábant per médium cæli: «Veníte, congregámini ad cenam magnam Dei, ut manducétis carnes regum et carnes tribunórum et carnes fórtium et carnes equórum et sedéntium in ipsis et carnes ómnium liberórum ac servórum et pusillórum ac magnórum».

\noindent Et vidi béstiam et reges terræ et exércitus eórum congregátos ad faciéndum prœ́lium cum illo, qui sedébat super equum, et cum exércitu eius. Et apprehénsa est béstia et cum illa pseudoprophéta, qui fecit signa coram ipsa, quibus sedúxit eos, qui accepérunt charactérem béstiæ et qui adórant imáginem eius; vivi missi sunt hi duo in stagnum ignis ardéntis súlphure. Et céteri occísi sunt in gládio sedéntis super equum, qui procédit de ore ipsíus, et omnes aves saturátæ sunt cárnibus eórum.}
\newcommand{\responsoriumi}{\pars{Responsorium 1.} \scriptura{\Rbardot{} Ps. 76, 17-18 \Vbardot{} ibid., 19; \textbf{H250}}

\vspace{-5mm}

\responsorium{II}{temporalia/resp-videruntteaquae-CROCHU.gtex}{}}
\newcommand{\lectioii}{\pars{Lectio II.} \scriptura{Oratio 1 in Christi resurrectionem: PG 46, 603-606. 626-627}

\noindent Ex Oratiónibus sancti Gregórii Nysséni epíscopi.

\noindent Venit vitæ regnum, et solútum est mortis impérium. Appáruit ália generátio, ália vita, álius vivéndi modus, ipsíus natúræ nostræ commutátio. Quænam generátio? \emph{Quæ non ex sanguínibus, neque ex voluntáte viri, neque ex voluntáte carnis, sed ex Deo facta est.}

\noindent Cuínam, ínquies, hoc fíeri potest? Audi, nam paucis explicábo. Fetus hic per fidem concípitur, per baptísmatis regeneratiónem in lucem éditur, nutrícem habet Ecclésiam, huius doctrínam et institúta sugit tamquam úbera, illi cæléstis panis est cibus; ætátis perféctio est sublímis vivéndi rátio; núptiæ, consuetúdo sapiéntiæ; líberi, spes; domus, regnum; heréditas et opuléntia, delíciæ paradísi; finis, non mors sed vita illa, quæ dignis paráta est, beatíssima et sempitérna.

\noindent \emph{Hæc est dies illa, quam fecit Dóminus,} a diébus illis longe divérsa, qui mundi procreatiónis inítio sunt constitúti, quos témporis cursus dimetítur. Alteríus hæc est procreatiónis inítium. In hac enim die cælum novum facit Deus et terram novam, ut ait Prophéta. Ecquod cælum? Firmaméntum eius quæ in Christo est fídei. Ecquam terram? Cor bonum, inquam, ut dixit Dóminus, terram quæ bibit super se veniéntem imbrem, et spicam multíplicem parit.}
\newcommand{\responsoriumii}{\pars{Responsorium 2.} \scriptura{Sap. 8, 18; \textbf{H251}}

\vspace{-5mm}

\responsorium{III}{temporalia/resp-alleluiadelectatiobona-CROCHU.gtex}{}}
\newcommand{\lectioiii}{\pars{Lectio III.}

\noindent In hac creatióne, sol quidem est vita munda; stellæ, virtútes; aer, præclára conversátio; mare, \emph{altitúdo divitiárum sapiéntiæ et sciéntiæ?} herbæ et gérmina, bona doctrína, divináque documénta, quæ pópulus páscuæ, hoc est, Dei grex carpit, atque depáscitur; árbores feréntes fructum, mandatórum observátio.

\noindent In hac die verus homo procreátur ad imáginem et similitúdinem Dei. Annon mundus tibi fit hoc princípium, \emph{Hæc dies, quam fecit Dóminus?} Quam neque diem esse dicit prophéta diébus áliis, neque noctem áliis nóctibus símilem?

\noindent Sed quod in hac grátia præstantíssimum est, nondum explicávimus. Hæc mortis dolóres dissólvit. Hæc mortuórum primogénitum édidit.

\noindent \emph{Ascéndo}, inquit, \emph{ad Patrem meum et Patrem vestrum, Deum meum et Deum vestrum.} O núntium præclárum et bonum! Qui pro nobis factus est homo, is cum sit Unigénitus, quo nos fratres effíciat suos, se hóminem ad verum Patrem addúcit, ut per se ipsum omnes cognátos secum trahat.}
\newcommand{\responsoriumiii}{\pars{Responsorium 3.} \scriptura{\Rbardot{} Ps. 131, 6-7 \Vbardot{} ibid., 8; \textbf{H251}}

\vspace{-5mm}

\responsorium{I}{temporalia/resp-alleluiaaudivimusea-CROCHU-cumdox.gtex}{}}
\newcommand{\laudes}{\pars{Psalmus 1.}

\vspace{-4mm}

\antiphona{VIII G}{temporalia/ant-alleluia-turco12.gtex}

\vspace{-3mm}

\scriptura{Psalmus 5.}

\vspace{-2mm}

\initiumpsalmi{temporalia/ps5-initium-viii-G-auto.gtex}

\vspace{-1.5mm}

\input{temporalia/ps5-viii-G.tex} \Abardot{}

\vspace{-5mm}

\vfill
\pagebreak

\pars{Psalmus 2.} \scriptura{Ap. 22, 13.16; 3, 7; \textbf{Cod. San. 387, f. 97}}

\vspace{-4mm}

\antiphona{VII a}{temporalia/ant-egosumalpha.gtex}

%\vspace{-2mm}

\scriptura{Canticum David, 1 Par. 29, 10-13}

%\vspace{-2mm}

\initiumpsalmi{temporalia/david-initium-vii-a-auto.gtex}

\input{temporalia/david-vii-a.tex} \Abardot{}

\vfill
\pagebreak

\pars{Psalmus 3.}

\vspace{-4mm}

\antiphona{II D}{temporalia/ant-alleluia-turco8.gtex}

\scriptura{Psalmus. 28}

\initiumpsalmi{temporalia/ps28-initium-ii-D-auto.gtex}

\input{temporalia/ps28-ii-D.tex} \Abardot{}

\vfill
\pagebreak}
\newcommand{\preces}{\noindent Iesum, quem Pater glorificávit et herédem ómnium géntium constítuit, \grestar{} exaltémus, orántes:

\Rbardot{} Per victóriam tuam salva nos, Dómine.

\noindent Christe, qui victória tua portas contrivísti infernáles, \gredagger{} peccátum delens et mortem, \grestar{} fac nos hódie peccáti victóres.

\Rbardot{} Per victóriam tuam salva nos, Dómine.

\noindent Tu, qui mortem evacuásti, \gredagger{} vitam nobis impértiens novam, \grestar{} da ut hódie in hac vitæ novitáte ambulémus.

\Rbardot{} Per victóriam tuam salva nos, Dómine.

\noindent Qui vitam mórtuis tribuísti, \gredagger{} totum genus humánum de morte ad vitam redúcens, \grestar{} ómnibus, qui nobis occúrrent, ætérnam vitam concéde.

\Rbardot{} Per victóriam tuam salva nos, Dómine.

\noindent Qui, sepúlcri tui custódes confúndens, \gredagger{} discípulos tuos lætificásti, \grestar{} plenam tibi serviéntibus largíre lætítiam.

\Rbardot{} Per victóriam tuam salva nos, Dómine.}
\newcommand{\benedictus}{\pars{Canticum Zachariæ.} \scriptura{Io. 14, 21}

\vspace{-4mm}

\antiphona{I f}{temporalia/ant-quiautemdiligitme.gtex}

\vspace{-2mm}

\scriptura{Lc. 1, 68-79}

\vspace{-2mm}

\cantusSineNeumas
\initiumpsalmi{temporalia/benedictus-initium-i-f-auto.gtex}

%\vspace{-1.5mm}

\input{temporalia/benedictus-i-f.tex} \Abardot{}}
\newcommand{\hebdomada}{infra Hebdom. V post Pentecosten.}
\newcommand{\oratioLaudes}{\cuminitiali{}{temporalia/oratio5.gtex}}

% LuaLaTeX

\documentclass[a4paper, twoside, 12pt]{article}
\usepackage[latin]{babel}
%\usepackage[landscape, left=3cm, right=1.5cm, top=2cm, bottom=1cm]{geometry} % okraje stranky
%\usepackage[landscape, a4paper, mag=1166, truedimen, left=2cm, right=1.5cm, top=1.6cm, bottom=0.95cm]{geometry} % okraje stranky
\usepackage[landscape, a4paper, mag=1400, truedimen, left=0.5cm, right=0.5cm, top=0.5cm, bottom=0.5cm]{geometry} % okraje stranky

\usepackage{fontspec}
\setmainfont[FeatureFile={junicode.fea}, Ligatures={Common, TeX}, RawFeature=+fixi]{Junicode}
%\setmainfont{Junicode}

% shortcut for Junicode without ligatures (for the Czech texts)
\newfontfamily\nlfont[FeatureFile={junicode.fea}, Ligatures={Common, TeX}, RawFeature=+fixi]{Junicode}

\usepackage{multicol}
\usepackage{color}
\usepackage{lettrine}
\usepackage{fancyhdr}

% usual packages loading:
\usepackage{luatextra}
\usepackage{graphicx} % support the \includegraphics command and options
\usepackage{gregoriotex} % for gregorio score inclusion
\usepackage{gregoriosyms}
\usepackage{wrapfig} % figures wrapped by the text
\usepackage{parcolumns}
\usepackage[contents={},opacity=1,scale=1,color=black]{background}
\usepackage{tikzpagenodes}
\usepackage{calc}
\usepackage{longtable}
\usetikzlibrary{calc}

\setlength{\headheight}{14.5pt}

\input{conventuscommune.tex} % Often used macros

\newcommand{\annusEditionis}{2021}

%%%% Vicekrat opakovane kousky

\newcommand{\anteOrationem}{
  \rubrica{Ante Orationem, cantatur a Superiore:}

  \pars{Supplicatio Litaniæ.}

  \cuminitiali{}{temporalia/supplicatiolitaniae.gtex}

  \pars{Oratio Dominica.}

  \cuminitiali{}{temporalia/oratiodominica.gtex}

  \rubrica{Deinde dicitur ab Hebdomadario:}

  \cuminitiali{}{temporalia/dominusvobiscum-solemnis.gtex}

  \rubrica{In choro monialium loco Dominus vobiscum dicitur:}

  \sineinitiali{temporalia/domineexaudi.gtex}
}

\setlength{\columnsep}{30pt} % prostor mezi sloupci

%%%%%%%%%%%%%%%%%%%%%%%%%%%%%%%%%%%%%%%%%%%%%%%%%%%%%%%%%%%%%%%%%%%%%%%%%%%%%%%%%%%%%%%%%%%%%%%%%%%%%%%%%%%%%
\begin{document}

% Here we set the space around the initial.
% Please report to http://home.gna.org/gregorio/gregoriotex/details for more details and options
\grechangedim{afterinitialshift}{2.2mm}{scalable}
\grechangedim{beforeinitialshift}{2.2mm}{scalable}
\grechangedim{interwordspacetext}{0.22 cm plus 0.15 cm minus 0.05 cm}{scalable}%
\grechangedim{annotationraise}{-0.2cm}{scalable}

% Here we set the initial font. Change 38 if you want a bigger initial.
% Emit the initials in red.
\grechangestyle{initial}{\color{red}\fontsize{38}{38}\selectfont}

\pagestyle{empty}

%%%% Titulni stranka
\begin{titulusOfficii}
\ifx\titulus\undefined
\nomenFesti{Feria II \hebdomada{}}
\else
\titulus
\fi
\end{titulusOfficii}

\vfill

\begin{center}
%Ad usum et secundum consuetudines chori \guillemotright{}Conventus Choralis\guillemotleft.

%Editio Sancti Wolfgangi \annusEditionis
\end{center}

\scriptura{}

\pars{}

\pagebreak

\renewcommand{\headrulewidth}{0pt} % no horiz. rule at the header
\fancyhf{}
\pagestyle{fancy}

\cantusSineNeumas

\ifx\oratio\undefined
\ifx\laudb\undefined
\else
\newcommand{\oratio}{\pars{Oratio.}

\noindent Dómine Deus omnípotens, qui ad princípium huius diéi nos perveníre fecísti, tua nos hódie salva virtúte, ut in hac die ad nullum declinémus peccátum, sed semper ad tuam iustítiam faciéndam nostra procédant elóquia, dirigántur cogitatiónes et ópera.

\noindent Per Dóminum nostrum Iesum Christum, Fílium tuum, qui tecum vivit et regnat in unitáte Spíritus Sancti, Deus, per ómnia sǽcula sæculórum.

\noindent \Rbardot{} Amen.}
\fi
\fi

\hora{Ad Matutinum.} %%%%%%%%%%%%%%%%%%%%%%%%%%%%%%%%%%%%%%%%%%%%%%%%%%%%%
%\sideThumbs{Matutinum}

\vspace{2mm}

\cuminitiali{}{temporalia/dominelabiamea.gtex}

\vfill
%\pagebreak

\vspace{2mm}

\ifx\invitatorium\undefined
\pars{Invitatorium.} \scriptura{Ps. 94, 1; Psalmus 94; \textbf{H451}}

\vspace{-6mm}

\antiphona{VI}{temporalia/inv-jubilemusdeo.gtex}\else
\invitatorium
\fi

\vfill
\pagebreak

\ifx\hymnusmatutinum\undefined
\ifx\matua\undefined
\else
\pars{Hymnus.}

{
\grechangedim{interwordspacetext}{0.10 cm plus 0.15 cm minus 0.05 cm}{scalable}%
\antiphona{II}{temporalia/hym-IpsumNunc.gtex}
\grechangedim{interwordspacetext}{0.22 cm plus 0.15 cm minus 0.05 cm}{scalable}%
}
\fi
\else
\hymnusmatutinum
\fi

\vspace{-3mm}

\vfill
\pagebreak

\ifx\matub\undefined
\else
% MAT B
\pars{Psalmus 1.} \scriptura{Ps. 30, 2; \textbf{H90}}

\vspace{-4mm}

\antiphona{VIII G}{temporalia/ant-intuaiustitia.gtex}

%\vspace{-2mm}

\scriptura{Ps. 30, 2-9}

%\vspace{-2mm}

\initiumpsalmi{temporalia/ps30i-initium-viii-G-auto.gtex}

\vspace{-1.5mm}

\input{temporalia/ps30i-viii-G.tex} \Abardot{}

\vfill
\pagebreak

\pars{Psalmus 2.} \scriptura{Ps. 66, 2}

\vspace{-4mm}

\antiphona{E}{temporalia/ant-illuminadomine.gtex}

%\vspace{-2mm}

\scriptura{Ps. 30, 10-17}

%\vspace{-2mm}

\initiumpsalmi{temporalia/ps30ii-initium-e-a-auto.gtex}

\input{temporalia/ps30ii-e-a.tex} \Abardot{}

\vfill
\pagebreak

\pars{Psalmus 3.} \scriptura{Ps. 30, 24}

\vspace{-4mm}

\antiphona{II D}{temporalia/ant-diligitedominum.gtex}

%\vspace{-5mm}

\scriptura{Ps. 30, 20-25}

%\vspace{-2mm}

\initiumpsalmi{temporalia/ps30iii-initium-ii-D-auto.gtex}

\input{temporalia/ps30iii-ii-D.tex} \Abardot{}

\vfill
\pagebreak
\fi

\pars{Versus.}

\ifx\matversus\undefined
\ifx\matub\undefined
\else
\noindent \Vbardot{} Dírige me, Dómine, in veritáte tua, et doce me.

\noindent \Rbardot{} Quia tu es Deus salútis meæ.
\fi
\else
\matversus
\fi

\vspace{5mm}

\sineinitiali{temporalia/oratiodominica-mat.gtex}

\vspace{5mm}

\pars{Absolutio.}

\cuminitiali{}{temporalia/absolutio-exaudi.gtex}

\vfill
\pagebreak

\cuminitiali{}{temporalia/benedictio-solemn-benedictione.gtex}

\vspace{7mm}

\lectioi

\noindent \Vbardot{} Tu autem, Dómine, miserére nobis.
\noindent \Rbardot{} Deo grátias.

\vfill
\pagebreak

\responsoriumi

\vfill
\pagebreak

\cuminitiali{}{temporalia/benedictio-solemn-unigenitus.gtex}

\vspace{7mm}

\lectioii

\noindent \Vbardot{} Tu autem, Dómine, miserére nobis.
\noindent \Rbardot{} Deo grátias.

\vfill
\pagebreak

\responsoriumii

\vfill
\pagebreak

\cuminitiali{}{temporalia/benedictio-solemn-spiritus.gtex}

\vspace{7mm}

\lectioiii

\noindent \Vbardot{} Tu autem, Dómine, miserére nobis.
\noindent \Rbardot{} Deo grátias.

\vfill
\pagebreak

\responsoriumiii

\vfill
\pagebreak

\rubrica{Reliqua omittuntur, nisi Laudes separandæ sint.}

\sineinitiali{temporalia/domineexaudi.gtex}

\vfill

\oratio

\vfill

\noindent \Vbardot{} Dómine, exáudi oratiónem meam.
\Rbardot{} Et clamor meus ad te véniat.

\vfill

\noindent \Vbardot{} Benedicámus Dómino.
\noindent \Rbardot{} Deo grátias.

\vfill

\noindent \Vbardot{} Fidélium ánimæ per misericórdiam Dei requiéscant in pace.
\Rbardot{} Amen.

\vfill
\pagebreak

\hora{Ad Laudes.} %%%%%%%%%%%%%%%%%%%%%%%%%%%%%%%%%%%%%%%%%%%%%%%%%%%%%
%\sideThumbs{Laudes}

\cantusSineNeumas

\vspace{0.5cm}
\grechangedim{interwordspacetext}{0.18 cm plus 0.15 cm minus 0.05 cm}{scalable}%
\cuminitiali{}{temporalia/deusinadiutorium-communis.gtex}
\grechangedim{interwordspacetext}{0.22 cm plus 0.15 cm minus 0.05 cm}{scalable}%

\vfill
\pagebreak

\ifx\hymnuslaudes\undefined
\ifx\laudbd\undefined
\else
\pars{Hymnus} \scriptura{Hilarius (\olddag{} 367)}

\grechangedim{interwordspacetext}{0.16 cm plus 0.15 cm minus 0.05 cm}{scalable}%
\cuminitiali{IV}{temporalia/hym-LucisLargitor.gtex}
\grechangedim{interwordspacetext}{0.22 cm plus 0.15 cm minus 0.05 cm}{scalable}%
\vspace{-3mm}
\fi
\else
\hymnuslaudes
\fi

\vfill
\pagebreak

\ifx\laudb\undefined
\else
\pars{Psalmus 1.} \scriptura{Ps. 41, 3; \textbf{H391}}

\vspace{-4mm}

\antiphona{II D}{temporalia/ant-sitivitanima.gtex}

%\vspace{-2mm}

\scriptura{Psalmus 41}

%\vspace{-2mm}

\initiumpsalmi{temporalia/ps41-initium-ii-D-auto.gtex}

%\vspace{-1.5mm}

\input{temporalia/ps41-ii-D.tex}

\vfill

\antiphona{}{temporalia/ant-sitivitanima.gtex}

\vfill
\pagebreak

\pars{Psalmus 2.}

\vspace{-4mm}

\antiphona{III a}{temporalia/ant-ostendenobisdomine.gtex}

%\vspace{-2mm}

\scriptura{Canticum Ecclesiastici, Sir. 36, 1-7.13-16}

%\vspace{-3mm}

\initiumpsalmi{temporalia/ecclesiastici-initium-iii-a-auto.gtex}

\input{temporalia/ecclesiastici-iii-a.tex} \Abardot{}

\vfill
\pagebreak

\pars{Psalmus 3.}

\vspace{-4mm}

\antiphona{II D}{temporalia/ant-operamanuumeius.gtex}

\scriptura{Psalmus 18, 1-7}

\initiumpsalmi{temporalia/ps18i-initium-ii-D-auto.gtex}

\input{temporalia/ps18i-ii-D.tex} \Abardot{}

\vfill
\pagebreak
\fi

\ifx\lectiobrevis\undefined
\ifx\laudb\undefined
\else
\pars{Lectio Brevis.} \scriptura{Ier. 15, 16}

\noindent Invénti sunt sermónes tui, et comédi eos, et factum est mihi verbum tuum in gáudium et in lætítiam cordis mei, quóniam invocátum est nomen tuum super me, Dómine Deus exercítuum.
\fi
\else
\lectiobrevis
\fi

\vfill

\ifx\responsoriumbreve\undefined
\ifx\laudbd\undefined
\else
\pars{Responsorium breve.} \scriptura{Ps. 32, 1.3}

\cuminitiali{VI}{temporalia/resp-exsultateiusti.gtex}
\fi
\else
\responsoriumbreve
\fi

\vfill
\pagebreak

\ifx\benedictus\undefined
\ifx\laudbd\undefined
\else
\pars{Canticum Zachariæ.} \scriptura{Lc. 1, 68; \textbf{H422}}

\vspace{-4mm}

{
\grechangedim{interwordspacetext}{0.18 cm plus 0.15 cm minus 0.05 cm}{scalable}%
\antiphona{IV E}{temporalia/ant-benedictusdominus.gtex}
\grechangedim{interwordspacetext}{0.22 cm plus 0.15 cm minus 0.05 cm}{scalable}%
}

%\vspace{-3mm}

\scriptura{Lc. 1, 68-79}

%\vspace{-2mm}

\cantusSineNeumas
\initiumpsalmi{temporalia/benedictus-initium-iv-E-auto.gtex}

%\vspace{-1.5mm}

\input{temporalia/benedictus-iv-E.tex} \Abardot{}
\fi
\else
\benedictus
\fi

\vspace{-1cm}

\vfill
\pagebreak

%\sideThumbs{{\scriptsize{}Fine horarum}}

\pars{Preces.}

\sineinitiali{}{temporalia/tonusprecum.gtex}

\ifx\preces\undefined
\ifx\laudb\undefined
\else
\noindent Salvátor noster fecit nos regnum et sacerdótium, ut hóstias Deo acceptábiles offerámus. \gredagger{} Grati ígitur eum invocémus:

\Rbardot{} Serva nos in tuo ministério, Dómine.

\noindent Christe, sacérdos ætérne, qui sanctum pópulo tuo sacerdótium concessísti, \gredagger{} concéde, ut spiritáles hóstias Deo acceptábiles iúgiter offerámus.

\Rbardot{} Serva nos in tuo ministério, Dómine.

\noindent Spíritus tui fructus nobis largíre propítius, \gredagger{} patiéntiam, benignitátem et mansuetúdinem.

\Rbardot{} Serva nos in tuo ministério, Dómine.

\noindent Da nobis te amáre, ut te, qui es cáritas, possideámus, \gredagger{} et bene ágere, ut per vitam étiam nostram te laudémus.

\Rbardot{} Serva nos in tuo ministério, Dómine.

\noindent Quæ frátribus nostris sunt utília, nos quǽrere concéde, \gredagger{} ut salútem facílius consequántur.

\Rbardot{} Serva nos in tuo ministério, Dómine.
\fi
\else
\preces
\fi

\vfill

\pars{Oratio Dominica.}

\cuminitiali{}{temporalia/oratiodominicaalt.gtex}

\vfill
\pagebreak

\rubrica{vel:}

\pars{Supplicatio Litaniæ.}

\cuminitiali{}{temporalia/supplicatiolitaniae.gtex}

\vfill

\pars{Oratio Dominica.}

\cuminitiali{}{temporalia/oratiodominica.gtex}

\vfill
\pagebreak

% Oratio. %%%
\oratio

\vspace{-1mm}

\vfill

\rubrica{Hebdomadarius dicit Dominus vobiscum, vel, absente sacerdote vel diacono, sic concluditur:}

\vspace{2mm}

\antiphona{C}{temporalia/dominusnosbenedicat.gtex}

\rubrica{Postea cantatur a cantore:}

\vspace{2mm}

\cuminitiali{IV}{temporalia/benedicamus-feria-laudes.gtex}

\vspace{1mm}

\vfill
\pagebreak

\end{document}

