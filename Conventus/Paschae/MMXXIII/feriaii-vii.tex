\newcommand{\oratio}{\pars{Oratio.}

\noindent Advéniat nobis, quǽsumus, Dómine, virtus Spíritus Sancti, qua voluntátem tuam fidéli mente retinére et pia conversatióne deprómere valeámus.

\pars{Pro commemoratione Sanctæ Ritæ Cascianensis.} \scriptura{Prv. 31, 17}

\vspace{-4mm}

\antiphona{VIII G}{temporalia/ant-accinxitfortitudine-tp.gtex}

\vfill

\noindent Largíre nobis, quǽsumus, Dómine, sapiéntiam crucis et fortitúdinem, quibus beátam Ritam ditáre dignátus es, ut, in tribulatióne cum Christo patiéntes, pascháli eius mystério intímius participáre valeámus.

\pars{Pro pace in Ucraina.} \scriptura{Sir. 50, 25; 2 Esdr. 4, 20; \textbf{H416}}

\vspace{-4mm}

\antiphona{II D}{temporalia/ant-dapacemdomine.gtex}

\vfill

\noindent Deus, a quo sancta desidéria, recta consília et iusta sunt ópera: da servis tuis illam, quam mundus dare non potest, pacem; ut et corda nostra mandátis tuis dédita, et hóstium subláta formídine, témpora sint tua protectióne tranquílla.

\noindent Per Dóminum nostrum Iesum Christum, Fílium tuum, qui tecum vivit et regnat in unitáte Spíritus Sancti, Deus, per ómnia sǽcula sæculórum.

\noindent \Rbardot{} Amen.}
\newcommand{\invitatorium}{\pars{Invitatorium.}

\vspace{-4mm}

\antiphona{VI**}{temporalia/inv-christumdominumquisanctum.gtex}}
\newcommand{\hymnusmatutinum}{\pars{Hymnus} \scriptura{Anonymus X. sæculi; \textbf{AR488}}

\antiphona{IV}{temporalia/hym-AEterneRex.gtex}}
\newcommand{\lectioi}{\pars{Lectio I.} \scriptura{1 Io. 4, 1-10}

\noindent De Epístola prima beáti Ioánnis apóstoli.

\noindent Caríssimi, nolíte omni spirítui crédere, sed probáte spíritus si ex Deo sint, quóniam multi pseudoprophétæ prodiérunt in mundum.

\noindent In hoc cognóscitis Spíritum Dei: omnis spíritus, qui confitétur Iesum Christum in carne venísse, ex Deo est.

\noindent Et omnis spíritus, qui non confitétur Iesum, ex Deo non est; et hoc est antichrísti, quod audístis quóniam venit, et nunc iam in mundo est.

\noindent Vos ex Deo estis, filíoli, et vicístis eos, quóniam maior est, qui in vobis est, quam qui in mundo. Ipsi ex mundo sunt; ídeo ex mundo loquúntur, et mundus eos audit. Nos ex Deo sumus.

\noindent Qui cognóscit Deum, audit nos; qui non est ex Deo, non audit nos. Ex hoc cognóscimus Spíritum veritátis et spíritum erróris.

\noindent Caríssimi, diligámus ínvicem, quóniam cáritas ex Deo est, et omnis, qui díligit, ex Deo natus est et cognóscit Deum.

\noindent Qui non díligit, non cognóvit Deum, quóniam Deus cáritas est.

\noindent In hoc appáruit cáritas Dei in nobis, quóniam Fílium suum unigénitum misit Deus in mundum, ut vivámus per eum.

\noindent In hoc est cáritas, non quasi nos dilexérimus Deum, sed quóniam ipse diléxit nos et misit Fílium suum propitiatiónem pro peccátis nostris.}
\newcommand{\responsoriumi}{\pars{Responsorium 1.} \scriptura{\Rbardot{} Lc. 24, 50 \Vbardot{} ibid., 51; \textbf{H264}}

\vspace{-5mm}

\responsorium{IV}{temporalia/resp-eduxitdominusiesus-CROCHU.gtex}{}}
\newcommand{\lectioii}{\pars{Lectio II.} \scriptura{Cat. 16, De Spiritu Sancto 1, 11-12. 16: PG 33, 931-935. 939-942}

\noindent Ex Catechésibus sancti Cyrílli Hierosolymitáni epíscopi.

\noindent \emph{Aqua, quam dabo ei, fiet in eo fons aquæ vivæ saliéntis in vitam ætérnam.} Novum aquæ genus, quæ vivit et salit; salit vero super dignos.

\noindent Quid autem causæ est quod Spíritus grátiam aquæ vocábulo nuncupávit?

\noindent Quod vidélicet per aquam ómnia consístunt; quod herbárum et animántium efféctrix est aqua; quod ex cælis ímbrium aqua descéndit;  quod uníus modi et formæ ipsa lábitur, multifórmes vero parit efféctus:

\noindent álius quidem exsístit in palma, álius rursum in vite, inque ómnibus ómnia;

\noindent cum sit uníus modi, nec álius a seípso exsístat;

\noindent non enim seípsum commútans imber, álius atque álius descéndit; sed, suscipiéntium se structúræ accómmodans, unicuíque id quod ei cómpetens est effícitur.

\noindent Ad eúndem modum et Spíritus Sanctus, cum unus sit, et uníus modi, et indivisíbilis, unicuíque grátiam prout vult dívidit.

\noindent Et quemádmodum lignum áridum, aquam concípiens, gérmina emíttit; sic et ánima peccátrix, per pæniténtiam Spíritus Sancti dono dignáta, iustítiæ racémos portat.

\noindent Cumque ille uníus et eiúsdem modi sit, multíplices tamen, Dei nutu et in Christi nómine, virtútes operátur.}
\newcommand{\responsoriumii}{\pars{Responsorium 2.} \scriptura{\Rbardot{} Io. 14, 27 \Vbardot{} ibid., 18; \textbf{H265} \& \textbf{E259}}

\vspace{-5mm}

\responsorium{V}{temporalia/resp-pacemmeamdovobis-CROCHU.gtex}{}

\vfill

\rubrica{vel ad libitum:}

\vspace{3mm}

\pars{Responsorium 2.} \scriptura{\Rbardot{} Ps. 103, 3; \Vbardot{} ibid., 4; \textbf{H264}}

\vspace{-5mm}

\responsorium{II}{temporalia/resp-ponitnubem-CROCHU.gtex}{}}
\newcommand{\lectioiii}{\pars{Lectio III.}

\noindent Nam altérius quidem útitur lingua ad sapiéntiam; altérius mentem prophetía illústrat; huic fugandórum dǽmonum potestátem impértit; illi divínas Scriptúras interpretándi donum largítur.

\noindent Altérius temperántiam corróborat, álium quæ ad misericórdiam pértinent docet; álium ieiunáre et ascéticæ vitæ exercitatiónes toleráre docet, álium res córporis contémnere; álium ad martýrium prǽparat; álius in áliis, ipse vero a se numquam álius, sicut scriptum est: \emph{Unicuíque vero datur manifestátio Spíritus ad id quod éxpedit.}

\noindent Mansuétus et lenis eius accéssus, suávis et fragrans eius sénsio, iugum levíssimum.

\noindent Advéntum eius antevértunt præmicántes lúminis ac sciéntiæ rádii.

\noindent Germáni tutóris viscéribus prǽditus venit: venit namque salváre, sanáre, docére, monére, roboráre, consolári, illustráre mentem, primum eius, a quo suscípitur, dehinc, eius ópera, aliórum.

\noindent Et quemádmodum is qui in ténebris prius versabátur, sole póstea derepénte conspécto, lucem in córporis óculo récipit, quæque ántea non vidébat perspícue cernit; ita et qui Spíritus Sancti dono dignus hábitus est, ánimo illuminátur, et supra hóminem evéctus videt quæ nesciébat.}
\newcommand{\responsoriumiii}{\pars{Responsorium 3.} \scriptura{\Rbardot{} Io. 14, 1; 16, 16; 15, 26; 14, 26 \Vbardot{} ibid., 16, 7; \textbf{H264}}

\vspace{-5mm}

\responsorium{IV}{temporalia/resp-nonconturbeturetmittam-CROCHU-cumdox.gtex}{}}
\newcommand{\hymnuslaudes}{\pars{Hymnus}

\cuminitiali{I}{temporalia/hym-OptatusVotis.gtex}}
\newcommand{\laudes}{\pars{Psalmus 1.}

\vspace{-4mm}

\antiphona{VIII G}{temporalia/ant-alleluia-turco12.gtex}

%\vspace{-2mm}

\scriptura{Psalmus 83}

%\vspace{-2mm}

\initiumpsalmi{temporalia/ps83-initium-viii-G-auto.gtex}

%\vspace{-1.5mm}

\input{temporalia/ps83-viii-G.tex} \Abardot{}

\vfill
\pagebreak

\pars{Psalmus 2.} \scriptura{Ap. 22, 13.16; 3, 7; \textbf{Cod. San. 387, f. 97}}

\vspace{-4mm}

\antiphona{VII a}{temporalia/ant-egosumalpha.gtex}

%\vspace{-2mm}

\scriptura{Canticum Isaiæ, Is. 2, 2-5}

%\vspace{-2mm}

\initiumpsalmi{temporalia/isaiae11-initium-vii-a-auto.gtex}

\input{temporalia/isaiae11-vii-a.tex} \Abardot{}

\vfill
\pagebreak

\pars{Psalmus 3.}

\vspace{-4mm}

\antiphona{II D}{temporalia/ant-alleluia-turco8.gtex}

\scriptura{Psalmus 95}

\initiumpsalmi{temporalia/ps95-initium-ii-D-auto.gtex}

\input{temporalia/ps95-ii-D.tex} \Abardot{}

\vfill
\pagebreak}
%\newcommand{\responsoriumbreve}{\pars{Responsorium breve.} \scriptura{Cf. Ps. 67, 19}
%
%\cuminitiali{VI}{temporalia/resp-ascendenschristus.gtex}}
\newcommand{\benedictus}{\pars{Canticum Zachariæ.} \scriptura{Io. 16, 29-30; \textbf{H245}}

\vspace{-4mm}

\antiphona{VIII G\textsuperscript{2}}{temporalia/ant-eccenuncpalam.gtex}

%\vspace{-3mm}

\scriptura{Lc. 1, 68-79}

%\vspace{-2mm}

\cantusSineNeumas
\initiumpsalmi{temporalia/benedictus-initium-viii-G5-auto.gtex}

%\vspace{-1.5mm}

\input{temporalia/benedictus-viii-G5.tex}

\vfill

\antiphona{}{temporalia/ant-eccenuncpalam.gtex}}
\newcommand{\preces}{\noindent Christum, qui Paráclitum a Patre in huius nómine se missúrum promísit, benedicámus,~\grestar{} et invocémus:

\Rbardot{} Da nobis Spíritum tuum.

\noindent Grátias ágimus tibi, Christe,~\gredagger{} et Patri per te in Spíritu Sancto;~\grestar{} ómnia in nómine tuo hódie verbo et ópere faciámus.

\Rbardot{} Da nobis Spíritum tuum.

\noindent Da nobis Spíritum tuum habére,~\grestar{} ut membra vivéntia córporis tui simus.

\Rbardot{} Da nobis Spíritum tuum.

\noindent Præsta, ne fratres nostros umquam iudicémus vel spernámus;~\grestar{} omnes enim stábimus aliquándo ante tribúnal tuum.

\Rbardot{} Da nobis Spíritum tuum.

\noindent Reple nos omni gáudio et pace in credéndo,~\grestar{} ut abundémus in spe et virtúte Spíritus Sancti.

\Rbardot{} Da nobis Spíritum tuum.}
\newcommand{\hebdomada}{infra Hebdom. VII Paschæ.}
\newcommand{\matuc}{Matutinum Hebdomadae C}
\newcommand{\matuac}{Matutinum Hebdomadae A vel C}
\newcommand{\laudc}{Laudes Hebdomadae C}
\newcommand{\laudac}{Laudes Hebdomadae A vel C}

% LuaLaTeX

\documentclass[a4paper, twoside, 12pt]{article}
\usepackage[latin]{babel}
%\usepackage[landscape, left=3cm, right=1.5cm, top=2cm, bottom=1cm]{geometry} % okraje stranky
%\usepackage[landscape, a4paper, mag=1166, truedimen, left=2cm, right=1.5cm, top=1.6cm, bottom=0.95cm]{geometry} % okraje stranky
\usepackage[landscape, a4paper, mag=1400, truedimen, left=0.5cm, right=0.5cm, top=0.5cm, bottom=0.5cm]{geometry} % okraje stranky

\usepackage{fontspec}
\setmainfont[FeatureFile={junicode.fea}, Ligatures={Common, TeX}, RawFeature=+fixi]{Junicode}
%\setmainfont{Junicode}

% shortcut for Junicode without ligatures (for the Czech texts)
\newfontfamily\nlfont[FeatureFile={junicode.fea}, Ligatures={Common, TeX}, RawFeature=+fixi]{Junicode}

\usepackage{multicol}
\usepackage{color}
\usepackage{lettrine}
\usepackage{fancyhdr}

% usual packages loading:
\usepackage{luatextra}
\usepackage{graphicx} % support the \includegraphics command and options
\usepackage{gregoriotex} % for gregorio score inclusion
\usepackage{gregoriosyms}
\usepackage{wrapfig} % figures wrapped by the text
\usepackage{parcolumns}
\usepackage[contents={},opacity=1,scale=1,color=black]{background}
\usepackage{tikzpagenodes}
\usepackage{calc}
\usepackage{longtable}
\usetikzlibrary{calc}

\setlength{\headheight}{14.5pt}

\input{conventuscommune.tex} % Often used macros

\newcommand{\annusEditionis}{2021}

%%%% Vicekrat opakovane kousky

\newcommand{\anteOrationem}{
  \rubrica{Ante Orationem, cantatur a Superiore:}

  \pars{Supplicatio Litaniæ.}

  \cuminitiali{}{temporalia/supplicatiolitaniae.gtex}

  \pars{Oratio Dominica.}

  \cuminitiali{}{temporalia/oratiodominica.gtex}

  \rubrica{Deinde dicitur ab Hebdomadario:}

  \cuminitiali{}{temporalia/dominusvobiscum-solemnis.gtex}

  \rubrica{In choro monialium loco Dominus vobiscum dicitur:}

  \sineinitiali{temporalia/domineexaudi.gtex}
}

\setlength{\columnsep}{30pt} % prostor mezi sloupci

%%%%%%%%%%%%%%%%%%%%%%%%%%%%%%%%%%%%%%%%%%%%%%%%%%%%%%%%%%%%%%%%%%%%%%%%%%%%%%%%%%%%%%%%%%%%%%%%%%%%%%%%%%%%%
\begin{document}

% Here we set the space around the initial.
% Please report to http://home.gna.org/gregorio/gregoriotex/details for more details and options
\grechangedim{afterinitialshift}{2.2mm}{scalable}
\grechangedim{beforeinitialshift}{2.2mm}{scalable}
\grechangedim{interwordspacetext}{0.22 cm plus 0.15 cm minus 0.05 cm}{scalable}%
\grechangedim{annotationraise}{-0.2cm}{scalable}

% Here we set the initial font. Change 38 if you want a bigger initial.
% Emit the initials in red.
\grechangestyle{initial}{\color{red}\fontsize{38}{38}\selectfont}

\pagestyle{empty}

%%%% Titulni stranka
\begin{titulusOfficii}
\ifx\titulus\undefined
\nomenFesti{Feria II \hebdomada{}}
\else
\titulus
\fi
\end{titulusOfficii}

\vfill

\begin{center}
%Ad usum et secundum consuetudines chori \guillemotright{}Conventus Choralis\guillemotleft.

%Editio Sancti Wolfgangi \annusEditionis
\end{center}

\scriptura{}

\pars{}

\pagebreak

\renewcommand{\headrulewidth}{0pt} % no horiz. rule at the header
\fancyhf{}
\pagestyle{fancy}

\cantusSineNeumas

\ifx\oratio\undefined
\ifx\laudb\undefined
\else
\newcommand{\oratio}{\pars{Oratio.}

\noindent Dómine Deus omnípotens, qui ad princípium huius diéi nos perveníre fecísti, tua nos hódie salva virtúte, ut in hac die ad nullum declinémus peccátum, sed semper ad tuam iustítiam faciéndam nostra procédant elóquia, dirigántur cogitatiónes et ópera.

\noindent Per Dóminum nostrum Iesum Christum, Fílium tuum, qui tecum vivit et regnat in unitáte Spíritus Sancti, Deus, per ómnia sǽcula sæculórum.

\noindent \Rbardot{} Amen.}
\fi
\fi

\hora{Ad Matutinum.} %%%%%%%%%%%%%%%%%%%%%%%%%%%%%%%%%%%%%%%%%%%%%%%%%%%%%
%\sideThumbs{Matutinum}

\vspace{2mm}

\cuminitiali{}{temporalia/dominelabiamea.gtex}

\vfill
%\pagebreak

\vspace{2mm}

\ifx\invitatorium\undefined
\pars{Invitatorium.} \scriptura{Ps. 94, 1; Psalmus 94; \textbf{H451}}

\vspace{-6mm}

\antiphona{VI}{temporalia/inv-jubilemusdeo.gtex}\else
\invitatorium
\fi

\vfill
\pagebreak

\ifx\hymnusmatutinum\undefined
\ifx\matua\undefined
\else
\pars{Hymnus.}

{
\grechangedim{interwordspacetext}{0.10 cm plus 0.15 cm minus 0.05 cm}{scalable}%
\antiphona{II}{temporalia/hym-IpsumNunc.gtex}
\grechangedim{interwordspacetext}{0.22 cm plus 0.15 cm minus 0.05 cm}{scalable}%
}
\fi
\else
\hymnusmatutinum
\fi

\vspace{-3mm}

\vfill
\pagebreak

\ifx\matub\undefined
\else
% MAT B
\pars{Psalmus 1.} \scriptura{Ps. 30, 2; \textbf{H90}}

\vspace{-4mm}

\antiphona{VIII G}{temporalia/ant-intuaiustitia.gtex}

%\vspace{-2mm}

\scriptura{Ps. 30, 2-9}

%\vspace{-2mm}

\initiumpsalmi{temporalia/ps30i-initium-viii-G-auto.gtex}

\vspace{-1.5mm}

\input{temporalia/ps30i-viii-G.tex} \Abardot{}

\vfill
\pagebreak

\pars{Psalmus 2.} \scriptura{Ps. 66, 2}

\vspace{-4mm}

\antiphona{E}{temporalia/ant-illuminadomine.gtex}

%\vspace{-2mm}

\scriptura{Ps. 30, 10-17}

%\vspace{-2mm}

\initiumpsalmi{temporalia/ps30ii-initium-e-a-auto.gtex}

\input{temporalia/ps30ii-e-a.tex} \Abardot{}

\vfill
\pagebreak

\pars{Psalmus 3.} \scriptura{Ps. 30, 24}

\vspace{-4mm}

\antiphona{II D}{temporalia/ant-diligitedominum.gtex}

%\vspace{-5mm}

\scriptura{Ps. 30, 20-25}

%\vspace{-2mm}

\initiumpsalmi{temporalia/ps30iii-initium-ii-D-auto.gtex}

\input{temporalia/ps30iii-ii-D.tex} \Abardot{}

\vfill
\pagebreak
\fi

\pars{Versus.}

\ifx\matversus\undefined
\ifx\matub\undefined
\else
\noindent \Vbardot{} Dírige me, Dómine, in veritáte tua, et doce me.

\noindent \Rbardot{} Quia tu es Deus salútis meæ.
\fi
\else
\matversus
\fi

\vspace{5mm}

\sineinitiali{temporalia/oratiodominica-mat.gtex}

\vspace{5mm}

\pars{Absolutio.}

\cuminitiali{}{temporalia/absolutio-exaudi.gtex}

\vfill
\pagebreak

\cuminitiali{}{temporalia/benedictio-solemn-benedictione.gtex}

\vspace{7mm}

\lectioi

\noindent \Vbardot{} Tu autem, Dómine, miserére nobis.
\noindent \Rbardot{} Deo grátias.

\vfill
\pagebreak

\responsoriumi

\vfill
\pagebreak

\cuminitiali{}{temporalia/benedictio-solemn-unigenitus.gtex}

\vspace{7mm}

\lectioii

\noindent \Vbardot{} Tu autem, Dómine, miserére nobis.
\noindent \Rbardot{} Deo grátias.

\vfill
\pagebreak

\responsoriumii

\vfill
\pagebreak

\cuminitiali{}{temporalia/benedictio-solemn-spiritus.gtex}

\vspace{7mm}

\lectioiii

\noindent \Vbardot{} Tu autem, Dómine, miserére nobis.
\noindent \Rbardot{} Deo grátias.

\vfill
\pagebreak

\responsoriumiii

\vfill
\pagebreak

\rubrica{Reliqua omittuntur, nisi Laudes separandæ sint.}

\sineinitiali{temporalia/domineexaudi.gtex}

\vfill

\oratio

\vfill

\noindent \Vbardot{} Dómine, exáudi oratiónem meam.
\Rbardot{} Et clamor meus ad te véniat.

\vfill

\noindent \Vbardot{} Benedicámus Dómino.
\noindent \Rbardot{} Deo grátias.

\vfill

\noindent \Vbardot{} Fidélium ánimæ per misericórdiam Dei requiéscant in pace.
\Rbardot{} Amen.

\vfill
\pagebreak

\hora{Ad Laudes.} %%%%%%%%%%%%%%%%%%%%%%%%%%%%%%%%%%%%%%%%%%%%%%%%%%%%%
%\sideThumbs{Laudes}

\cantusSineNeumas

\vspace{0.5cm}
\grechangedim{interwordspacetext}{0.18 cm plus 0.15 cm minus 0.05 cm}{scalable}%
\cuminitiali{}{temporalia/deusinadiutorium-communis.gtex}
\grechangedim{interwordspacetext}{0.22 cm plus 0.15 cm minus 0.05 cm}{scalable}%

\vfill
\pagebreak

\ifx\hymnuslaudes\undefined
\ifx\laudbd\undefined
\else
\pars{Hymnus} \scriptura{Hilarius (\olddag{} 367)}

\grechangedim{interwordspacetext}{0.16 cm plus 0.15 cm minus 0.05 cm}{scalable}%
\cuminitiali{IV}{temporalia/hym-LucisLargitor.gtex}
\grechangedim{interwordspacetext}{0.22 cm plus 0.15 cm minus 0.05 cm}{scalable}%
\vspace{-3mm}
\fi
\else
\hymnuslaudes
\fi

\vfill
\pagebreak

\ifx\laudb\undefined
\else
\pars{Psalmus 1.} \scriptura{Ps. 41, 3; \textbf{H391}}

\vspace{-4mm}

\antiphona{II D}{temporalia/ant-sitivitanima.gtex}

%\vspace{-2mm}

\scriptura{Psalmus 41}

%\vspace{-2mm}

\initiumpsalmi{temporalia/ps41-initium-ii-D-auto.gtex}

%\vspace{-1.5mm}

\input{temporalia/ps41-ii-D.tex}

\vfill

\antiphona{}{temporalia/ant-sitivitanima.gtex}

\vfill
\pagebreak

\pars{Psalmus 2.}

\vspace{-4mm}

\antiphona{III a}{temporalia/ant-ostendenobisdomine.gtex}

%\vspace{-2mm}

\scriptura{Canticum Ecclesiastici, Sir. 36, 1-7.13-16}

%\vspace{-3mm}

\initiumpsalmi{temporalia/ecclesiastici-initium-iii-a-auto.gtex}

\input{temporalia/ecclesiastici-iii-a.tex} \Abardot{}

\vfill
\pagebreak

\pars{Psalmus 3.}

\vspace{-4mm}

\antiphona{II D}{temporalia/ant-operamanuumeius.gtex}

\scriptura{Psalmus 18, 1-7}

\initiumpsalmi{temporalia/ps18i-initium-ii-D-auto.gtex}

\input{temporalia/ps18i-ii-D.tex} \Abardot{}

\vfill
\pagebreak
\fi

\ifx\lectiobrevis\undefined
\ifx\laudb\undefined
\else
\pars{Lectio Brevis.} \scriptura{Ier. 15, 16}

\noindent Invénti sunt sermónes tui, et comédi eos, et factum est mihi verbum tuum in gáudium et in lætítiam cordis mei, quóniam invocátum est nomen tuum super me, Dómine Deus exercítuum.
\fi
\else
\lectiobrevis
\fi

\vfill

\ifx\responsoriumbreve\undefined
\ifx\laudbd\undefined
\else
\pars{Responsorium breve.} \scriptura{Ps. 32, 1.3}

\cuminitiali{VI}{temporalia/resp-exsultateiusti.gtex}
\fi
\else
\responsoriumbreve
\fi

\vfill
\pagebreak

\ifx\benedictus\undefined
\ifx\laudbd\undefined
\else
\pars{Canticum Zachariæ.} \scriptura{Lc. 1, 68; \textbf{H422}}

\vspace{-4mm}

{
\grechangedim{interwordspacetext}{0.18 cm plus 0.15 cm minus 0.05 cm}{scalable}%
\antiphona{IV E}{temporalia/ant-benedictusdominus.gtex}
\grechangedim{interwordspacetext}{0.22 cm plus 0.15 cm minus 0.05 cm}{scalable}%
}

%\vspace{-3mm}

\scriptura{Lc. 1, 68-79}

%\vspace{-2mm}

\cantusSineNeumas
\initiumpsalmi{temporalia/benedictus-initium-iv-E-auto.gtex}

%\vspace{-1.5mm}

\input{temporalia/benedictus-iv-E.tex} \Abardot{}
\fi
\else
\benedictus
\fi

\vspace{-1cm}

\vfill
\pagebreak

%\sideThumbs{{\scriptsize{}Fine horarum}}

\pars{Preces.}

\sineinitiali{}{temporalia/tonusprecum.gtex}

\ifx\preces\undefined
\ifx\laudb\undefined
\else
\noindent Salvátor noster fecit nos regnum et sacerdótium, ut hóstias Deo acceptábiles offerámus. \gredagger{} Grati ígitur eum invocémus:

\Rbardot{} Serva nos in tuo ministério, Dómine.

\noindent Christe, sacérdos ætérne, qui sanctum pópulo tuo sacerdótium concessísti, \gredagger{} concéde, ut spiritáles hóstias Deo acceptábiles iúgiter offerámus.

\Rbardot{} Serva nos in tuo ministério, Dómine.

\noindent Spíritus tui fructus nobis largíre propítius, \gredagger{} patiéntiam, benignitátem et mansuetúdinem.

\Rbardot{} Serva nos in tuo ministério, Dómine.

\noindent Da nobis te amáre, ut te, qui es cáritas, possideámus, \gredagger{} et bene ágere, ut per vitam étiam nostram te laudémus.

\Rbardot{} Serva nos in tuo ministério, Dómine.

\noindent Quæ frátribus nostris sunt utília, nos quǽrere concéde, \gredagger{} ut salútem facílius consequántur.

\Rbardot{} Serva nos in tuo ministério, Dómine.
\fi
\else
\preces
\fi

\vfill

\pars{Oratio Dominica.}

\cuminitiali{}{temporalia/oratiodominicaalt.gtex}

\vfill
\pagebreak

\rubrica{vel:}

\pars{Supplicatio Litaniæ.}

\cuminitiali{}{temporalia/supplicatiolitaniae.gtex}

\vfill

\pars{Oratio Dominica.}

\cuminitiali{}{temporalia/oratiodominica.gtex}

\vfill
\pagebreak

% Oratio. %%%
\oratio

\vspace{-1mm}

\vfill

\rubrica{Hebdomadarius dicit Dominus vobiscum, vel, absente sacerdote vel diacono, sic concluditur:}

\vspace{2mm}

\antiphona{C}{temporalia/dominusnosbenedicat.gtex}

\rubrica{Postea cantatur a cantore:}

\vspace{2mm}

\cuminitiali{IV}{temporalia/benedicamus-feria-laudes.gtex}

\vspace{1mm}

\vfill
\pagebreak

\end{document}

