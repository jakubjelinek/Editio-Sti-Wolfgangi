\newcommand{\oratio}{\pars{Oratio.}

\noindent Spíritus tuus, quǽsumus, Dómine, spiritália nobis dona poténter infúndat, ut det nobis mentem, quæ tibi sit plácita, et aptet nos tuæ propítius voluntáti.

\pars{Pro commemoratione Sancti Bedæ Venerabilis, Presbyteri \& Ecclesiæ Doctoris.} \scriptura{Sir. 31, 8}

\vspace{-4mm}

\antiphona{III a}{temporalia/ant-beatusvirquiinventusest.gtex}

\vfill

\noindent Deus, qui Ecclésiam tuam beáti Bedæ, presbýteri, eruditióne claríficas, fámulis tuis concéde propítius, et eius semper illustrári sapiéntia, et méritis adiuvári.

\pars{Pro pace in Ucraina.} \scriptura{Sir. 50, 25; 2 Esdr. 4, 20; \textbf{H416}}

\vspace{-4mm}

\antiphona{II D}{temporalia/ant-dapacemdomine.gtex}

\vfill

\noindent Deus, a quo sancta desidéria, recta consília et iusta sunt ópera: da servis tuis illam, quam mundus dare non potest, pacem; ut et corda nostra mandátis tuis dédita, et hóstium subláta formídine, témpora sint tua protectióne tranquílla.

\noindent Per Dóminum nostrum Iesum Christum, Fílium tuum, qui tecum vivit et regnat in unitáte Spíritus Sancti, Deus, per ómnia sǽcula sæculórum.

\noindent \Rbardot{} Amen.}
\newcommand{\invitatorium}{\pars{Invitatorium.}

\vspace{-4mm}

\antiphona{VI**}{temporalia/inv-christumdominumquisanctum.gtex}}
\newcommand{\hymnusmatutinum}{\pars{Hymnus} \scriptura{Anonymus X. sæculi; \textbf{AR488}}

\antiphona{IV}{temporalia/hym-AEterneRex.gtex}}
\newcommand{\lectioi}{\pars{Lectio I.} \scriptura{1 Io. 5, 13-21}

\noindent De Epístola prima beáti Ioánnis apóstoli.

\noindent Caríssimi: Hæc scripsi vobis, ut sciátis quóniam vitam habétis ætérnam, qui créditis in nomen Fílii Dei.

\noindent Et hæc est fidúcia, quam habémus ad eum, quia si quid petiérimus secúndum voluntátem eius, audit nos. Et si scimus quóniam audit nos, quidquid petiérimus, scimus quóniam habémus petitiónes, quas postulávimus ab eo.

\noindent Si quis videt fratrem suum peccáre peccátum non ad mortem, petet, et dabit ei Deus vitam, peccántibus non ad mortem. Est peccátum ad mortem; non pro illo dico, ut roget. Omnis iniustítia peccátum est, et est peccátum non ad mortem.

\noindent Scimus quóniam omnis, qui natus est ex Deo, non peccat, sed ille qui génitus est ex Deo, consérvat eum, et Malígnus non tangit eum. Scimus quóniam ex Deo sumus, et mundus totus in Malígno pósitus est. Et scimus quóniam Fílius Dei venit et dedit nobis sensum, ut cognoscámus eum, qui verus est; et sumus in eo, qui verus est, in Fílio eius Iesu Christo. Hic est qui verus est, Deus et vita ætérna. Filíoli, custodíte vos a simulácris!}
\newcommand{\responsoriumi}{\pars{Responsorium 1.} \scriptura{\Rbardot{} Mc. 16, 15-16 \Vbardot{} Mt. 28, 19; \textbf{H264}}

\vspace{-5mm}

\responsorium{VI}{temporalia/resp-iteinorbemuniversum-CROCHU.gtex}{}}
\newcommand{\lectioii}{\pars{Lectio II.} \scriptura{Lib. 10, 16, 6-7: PG 74, 434}

\noindent Ex Commentário sancti Cyrílli Alexandríni epíscopi in Evangélium Ioánnis.

\noindent Perácta erant ómnia quæ in terris dispensánda erant; sed oportébat omníno fíeri nos consórtes et divínæ Verbi natúræ partícipes, seu, nostra vita relícta, in áliam quandam transformári et ad novitátem piæ conversatiónis instaurári; quod áliter quam participatióne Sancti Spíritus fíeri non póterat.

\noindent Tempus autem máxime opportúnum missióni Spíritus, et eius in nos illápsui, illud erat quod discéssum Christi nostri servatóris excépit.

\noindent Nam quámdiu Christus cum fidélibus adhuc in carne versabátur, omnis, ut reor, boni largítor eis apparébat; verum ubi ad cæléstem Patrem ascendéndi tempus et necéssitas áffuit, quómodo non oportébat eum cultóribus suis per Spíritum adésse, et per fidem in nostris córdibus habitáre, ut, ipsum habéntes in nobis, cum fidúcia clamarémus: «Abba, Pater», et ad omnem virtútem fácile currerémus, ac prætérea poténtes et invícti advérsus diáboli insídias et insúltus hóminum reperirémur, ut qui cunctipoténtem Spíritum habeámus?}
\newcommand{\responsoriumii}{\pars{Responsorium 2.} \scriptura{\Rbardot{} Lc. 24, 50 \Vbardot{} ibid., 51; \textbf{H264}}

\vspace{-5mm}

\responsorium{IV}{temporalia/resp-eduxitdominusiesus-CROCHU.gtex}{}

\rubrica{vel ad libitum:}

\vspace{3mm}

\pars{Responsorium 2.} \scriptura{\Rbardot{} Eph. 4, 8 \Vbardot{} Ps. 46, 6; \textbf{H263}}

\vspace{-5mm}

\responsorium{IV}{temporalia/resp-ascendensinaltum-CROCHU.gtex}{}}
\newcommand{\lectioiii}{\pars{Lectio III.}

\noindent Quod enim in álium quendam hábitum transfórmet eos in quibus fúerit et inhabitárit Spíritus, eósque in novitátem vitæ restáuret, nonne fácile est cuívis, cum ex véteris tum novæ Scriptúræ testimóniis, osténdere?

\noindent Divínus quippe Sámuel, cum ad Saulem verba fáceret: \emph{Et insíliet,} inquit, \emph{super te Spíritus Dómini, et vertéris in virum álium.} Beátus autem Paulus: \emph{Nos vero omnes, reveláta fácie glóriam Dómini speculántes, in eándem imáginem transformámur a claritáte in claritátem, tamquam a Dómini Spíritu. Dóminus autem Spíritus est.}

\noindent Vides ut Spíritus transfórmet quodámmodo in áliam imáginem eos in quibus inhábitat? Transfert enim fácile a rerum terrenárum sensu ad spectánda solum ea quæ sunt in cælis, et ex imbélli timiditáte ad strénuam et generosíssimam vim ánimi. Neque vero dúbium est quin ita afféctos reperiámus discípulos et sic per Spíritum confirmátos, ut persequéntium incúrsibus nequáquam vinceréntur, sed dilectióni in Christum mórdicus adhærérent.

\noindent Verum ígitur est quod ait Servátor: \emph{Expedit vobis, ut ego ad cælos revértar.} Tempus enim illud erat descensúri Spíritus.}
\newcommand{\responsoriumiii}{\pars{Responsorium 3.} \scriptura{\Rbardot{} Tob. 12, 20 \& Io. 14, 27; \Vbardot{} Io. 16, 7; \textbf{H263}}

\vspace{-5mm}

\responsorium{IV}{temporalia/resp-tempusest-CROCHU-cumdox.gtex}{}}
\newcommand{\hymnuslaudes}{\pars{Hymnus}

\cuminitiali{I}{temporalia/hym-OptatusVotis.gtex}}
\newcommand{\laudes}{\pars{Psalmus 1.}

\vspace{-4mm}

\antiphona{I a\textsuperscript{2}}{temporalia/ant-alleluia-turco21.gtex}

%\vspace{-2mm}

\scriptura{Psalmus 86}

%\vspace{-2mm}

\initiumpsalmi{temporalia/ps86-initium-i-a4-auto.gtex}

%\vspace{-1.5mm}

\input{temporalia/ps86-i-a4.tex} \Abardot{}

\vfill
\pagebreak

\pars{Psalmus 2.} \scriptura{Ap. 2, 7}

\vspace{-4mm}

\antiphona{VII c\textsuperscript{2}}{temporalia/ant-vincentidaboeiedere.gtex}

%\vspace{-2mm}

\scriptura{Canticum Isaiæ, Is. 40, 10-17}

%\vspace{-3mm}

\initiumpsalmi{temporalia/isaiae9-initium-vii-c2-auto.gtex}

\input{temporalia/isaiae9-vii-c2.tex} \Abardot{}

\vfill
\pagebreak

\pars{Psalmus 3.}

\vspace{-4mm}

\antiphona{IV* e}{temporalia/ant-alleluia-turco10.gtex}

\scriptura{Psalmus 98}

\initiumpsalmi{temporalia/ps98-initium-iv_-e-auto.gtex}

\input{temporalia/ps98-iv_-e.tex} \Abardot{}

\vfill
\pagebreak}
%\newcommand{\responsoriumbreve}{\pars{Responsorium breve.} \scriptura{Cf. Ps. 67, 19}
%
%\cuminitiali{VI}{temporalia/resp-ascendenschristus.gtex}}
\newcommand{\benedictus}{\pars{Canticum Zachariæ.} \scriptura{Mt. 28, 19}

\vspace{-4mm}

\antiphona{I f}{temporalia/ant-euntesinmundumdoceteomnes.gtex}

\vspace{-2mm}

\scriptura{Lc. 1, 68-79}

\vspace{-2mm}

\cantusSineNeumas
\initiumpsalmi{temporalia/benedictus-initium-i-f-auto.gtex}

%\vspace{-1.5mm}

\input{temporalia/benedictus-i-f.tex} \Abardot{}}
\newcommand{\preces}{\noindent Benedíctus Christus Dóminus,~\gredagger{} per quem omnes accéssum habémus in Spíritu Sancto ad Patrem.~\grestar{} Orémus:

\Rbardot{} Christe, audi nos.

\noindent Mitte Spíritum tuum, optátum hóspitem animárum,~\grestar{} nobísque tríbue eum numquam contristáre.

\Rbardot{} Christe, audi nos.

\noindent Qui surrexísti a mórtuis et ad déxteram Dei sedes,~\grestar{} Patrem semper interpélla pro nobis.

\Rbardot{} Christe, audi nos.

\noindent Per Spíritum tuum nos tibi coniúnge,~\grestar{} ne tribulátio, persecútio et perícula umquam a tua nos séparent caritáte.

\Rbardot{} Christe, audi nos.

\noindent Da nos ínvicem suscípere,~\grestar{} sicut tu nos suscepísti in honórem Dei.

\Rbardot{} Christe, audi nos.}
\newcommand{\hebdomada}{infra Hebdom. VII Paschæ.}
\newcommand{\matuc}{Matutinum Hebdomadae C}
\newcommand{\matuac}{Matutinum Hebdomadae A vel C}
\newcommand{\laudc}{Laudes Hebdomadae C}
\newcommand{\laudac}{Laudes Hebdomadae A vel C}

% LuaLaTeX

\documentclass[a4paper, twoside, 12pt]{article}
\usepackage[latin]{babel}
%\usepackage[landscape, left=3cm, right=1.5cm, top=2cm, bottom=1cm]{geometry} % okraje stranky
%\usepackage[landscape, a4paper, mag=1166, truedimen, left=2cm, right=1.5cm, top=1.6cm, bottom=0.95cm]{geometry} % okraje stranky
\usepackage[landscape, a4paper, mag=1400, truedimen, left=0.5cm, right=0.5cm, top=0.5cm, bottom=0.5cm]{geometry} % okraje stranky

\usepackage{fontspec}
\setmainfont[FeatureFile={junicode.fea}, Ligatures={Common, TeX}, RawFeature=+fixi]{Junicode}
%\setmainfont{Junicode}

% shortcut for Junicode without ligatures (for the Czech texts)
\newfontfamily\nlfont[FeatureFile={junicode.fea}, Ligatures={Common, TeX}, RawFeature=+fixi]{Junicode}

\usepackage{multicol}
\usepackage{color}
\usepackage{lettrine}
\usepackage{fancyhdr}

% usual packages loading:
\usepackage{luatextra}
\usepackage{graphicx} % support the \includegraphics command and options
\usepackage{gregoriotex} % for gregorio score inclusion
\usepackage{gregoriosyms}
\usepackage{wrapfig} % figures wrapped by the text
\usepackage{parcolumns}
\usepackage[contents={},opacity=1,scale=1,color=black]{background}
\usepackage{tikzpagenodes}
\usepackage{calc}
\usepackage{longtable}
\usetikzlibrary{calc}

\setlength{\headheight}{14.5pt}

\input{conventuscommune.tex} % Often used macros

\newcommand{\annusEditionis}{2021}

%%%% Vicekrat opakovane kousky

\newcommand{\anteOrationem}{
  \rubrica{Ante Orationem, cantatur a Superiore:}

  \pars{Supplicatio Litaniæ.}

  \cuminitiali{}{temporalia/supplicatiolitaniae.gtex}

  \pars{Oratio Dominica.}

  \cuminitiali{}{temporalia/oratiodominica.gtex}

  \rubrica{Deinde dicitur ab Hebdomadario:}

  \cuminitiali{}{temporalia/dominusvobiscum-solemnis.gtex}

  \rubrica{In choro monialium loco Dominus vobiscum dicitur:}

  \sineinitiali{temporalia/domineexaudi.gtex}
}

\setlength{\columnsep}{30pt} % prostor mezi sloupci

%%%%%%%%%%%%%%%%%%%%%%%%%%%%%%%%%%%%%%%%%%%%%%%%%%%%%%%%%%%%%%%%%%%%%%%%%%%%%%%%%%%%%%%%%%%%%%%%%%%%%%%%%%%%%
\begin{document}

% Here we set the space around the initial.
% Please report to http://home.gna.org/gregorio/gregoriotex/details for more details and options
\grechangedim{afterinitialshift}{2.2mm}{scalable}
\grechangedim{beforeinitialshift}{2.2mm}{scalable}
\grechangedim{interwordspacetext}{0.22 cm plus 0.15 cm minus 0.05 cm}{scalable}%
\grechangedim{annotationraise}{-0.2cm}{scalable}

% Here we set the initial font. Change 38 if you want a bigger initial.
% Emit the initials in red.
\grechangestyle{initial}{\color{red}\fontsize{38}{38}\selectfont}

\pagestyle{empty}

%%%% Titulni stranka
\begin{titulusOfficii}
\ifx\titulus\undefined
\nomenFesti{Feria V \hebdomada{}}
\else
\titulus
\fi
\end{titulusOfficii}

\vfill

\begin{center}
%Ad usum et secundum consuetudines chori \guillemotright{}Conventus Choralis\guillemotleft.

%Editio Sancti Wolfgangi \annusEditionis
\end{center}

\scriptura{}

\pars{}

\pagebreak

\renewcommand{\headrulewidth}{0pt} % no horiz. rule at the header
\fancyhf{}
\pagestyle{fancy}

\cantusSineNeumas

\ifx\oratio\undefined
\ifx\lauda\undefined
\else
\newcommand{\oratio}{\pars{Oratio.}

\noindent Omnípotens sempitérne Deus, véspere, mane et merídie maiestátem tuam supplíciter deprecámur, ut, expúlsis de córdibus nostris peccatórum ténebris, ad veram lucem, quæ Christus est, nos fácias perveníre.

\noindent Qui tecum vivit et regnat in unitáte Spíritus Sancti, Deus, per ómnia sǽcula sæculórum.

\noindent \Rbardot{} Amen.}
\fi
\fi

\hora{Ad Matutinum.} %%%%%%%%%%%%%%%%%%%%%%%%%%%%%%%%%%%%%%%%%%%%%%%%%%%%%
%\sideThumbs{Matutinum}

\vspace{2mm}

\cuminitiali{}{temporalia/dominelabiamea.gtex}

\vfill
%\pagebreak

\vspace{2mm}

\ifx\invitatorium\undefined
\pars{Invitatorium.} \scriptura{Ps. 94, 6; Psalmus 94; \textbf{H136}}

\vspace{-6mm}

\antiphona{E}{temporalia/inv-adoremusdominum.gtex}
\else
\invitatorium
\fi

\vfill
\pagebreak

\ifx\hymnusmatutinum\undefined
\ifx\matuac\undefined
\else
\pars{Hymnus.} \scriptura{Gregorius Magnus (+604)}

{
\grechangedim{interwordspacetext}{0.10 cm plus 0.15 cm minus 0.05 cm}{scalable}%
\antiphona{IV}{temporalia/hym-NoxAtra.gtex}
\grechangedim{interwordspacetext}{0.22 cm plus 0.15 cm minus 0.05 cm}{scalable}%
}
\fi
\else
\hymnusmatutinum
\fi

\vspace{-3mm}

\vfill
\pagebreak

\ifx\matua\undefined
\else
% MAT A
\pars{Psalmus 1.} \scriptura{Ps. 17, 3; \textbf{H99}}

\vspace{-4mm}

\antiphona{VIII G}{temporalia/ant-dominusfirmamentum.gtex}

%\vspace{-2mm}

\scriptura{Ps. 17, 31-35}

%\vspace{-2mm}

\initiumpsalmi{temporalia/ps17xxxi_xxxv-initium-viii-G-auto.gtex}

\input{temporalia/ps17xxxi_xxxv-viii-G.tex} \Abardot{}

\vfill
\pagebreak

\pars{Psalmus 2.} \scriptura{Ps. 62, 9; \textbf{H393}}

\vspace{-4mm}

\antiphona{VII c trans.}{temporalia/ant-mesuscepit.gtex}

%\vspace{-2mm}

\scriptura{Ps. 17, 36-46}

%\vspace{-2mm}

\initiumpsalmi{temporalia/ps17xxxvi_xlvi-initium-vii-c-trans.gtex}

\input{temporalia/ps17xxxvi_xlvi-vii-c.tex} \Abardot{}

\vfill
\pagebreak

\pars{Psalmus 3.} \scriptura{Ps. 17, 47; \textbf{H100}}

\vspace{-4mm}

\antiphona{VII c\textsuperscript{2}}{temporalia/ant-vivitdominus.gtex}

%\vspace{-2mm}

\scriptura{Ps. 17, 47-51}

%\vspace{-2mm}

\initiumpsalmi{temporalia/ps17xlvii_li-initium-vii-c2-auto.gtex}

\input{temporalia/ps17xlvii_li-vii-c2.tex} \Abardot{}

\vfill
\pagebreak
\fi
\ifx\matuc\undefined
\else
% MAT C
\pars{Psalmus 1.} \scriptura{Lam. 1, 21; \textbf{H177}}

\vspace{-4mm}

\antiphona{VII a}{temporalia/ant-omnesinimici.gtex}

%\vspace{-2mm}

\scriptura{Ps. 88, 39-46}

%\vspace{-2mm}

\initiumpsalmi{temporalia/ps88xxxix_xlvi-initium-vii-a-auto.gtex}

\input{temporalia/ps88xxxix_xlvi-vii-a.tex} \Abardot{}

\vfill
\pagebreak

\pars{Psalmus 2.} \scriptura{Ps. 88, 53; \textbf{H98}}

\vspace{-4mm}

\antiphona{VI F}{temporalia/ant-benedictusdominusinaeternum.gtex}

%\vspace{-2mm}

\scriptura{Ps. 88, 47-53}

%\vspace{-2mm}

\initiumpsalmi{temporalia/ps88xlvii_liii-initium-vi-F-auto.gtex}

\input{temporalia/ps88xlvii_liii-vi-F.tex} \Abardot{}

\vfill
\pagebreak

\pars{Psalmus 3.} \scriptura{Ps. 89, 13}

\vspace{-4mm}

\antiphona{I g}{temporalia/ant-converteredomine.gtex}

%\vspace{-2mm}

\scriptura{Ps. 89}

%\vspace{-2mm}

\initiumpsalmi{temporalia/ps89-initium-i-g-auto.gtex}

\input{temporalia/ps89-i-g.tex}

\vfill

\antiphona{}{temporalia/ant-converteredomine.gtex}

\vfill
\pagebreak
\fi

\pars{Versus.}

\ifx\matversus\undefined
\ifx\matua\undefined
\else
\noindent \Vbardot{} Révela, Dómine, óculos meos.

\noindent \Rbardot{} Et considerábo mirabília de lege tua.
\fi
\ifx\matuc\undefined
\else
\noindent \Vbardot{} Audies de ore meo verbum.

\noindent \Rbardot{} Et annuntiábis eis ex me.
\fi
\else
\matversus
\fi

\vspace{5mm}

\sineinitiali{temporalia/oratiodominica-mat.gtex}

\vspace{5mm}

\pars{Absolutio.}

\cuminitiali{}{temporalia/absolutio-exaudi.gtex}

\vfill
\pagebreak

\cuminitiali{}{temporalia/benedictio-solemn-benedictione.gtex}

\vspace{7mm}

\lectioi

\noindent \Vbardot{} Tu autem, Dómine, miserére nobis.
\noindent \Rbardot{} Deo grátias.

\vfill
\pagebreak

\responsoriumi

\vfill
\pagebreak

\cuminitiali{}{temporalia/benedictio-solemn-unigenitus.gtex}

\vspace{7mm}

\lectioii

\noindent \Vbardot{} Tu autem, Dómine, miserére nobis.
\noindent \Rbardot{} Deo grátias.

\vfill
\pagebreak

\responsoriumii

\vfill
\pagebreak

\cuminitiali{}{temporalia/benedictio-solemn-spiritus.gtex}

\vspace{7mm}

\lectioiii

\noindent \Vbardot{} Tu autem, Dómine, miserére nobis.
\noindent \Rbardot{} Deo grátias.

\vfill
\pagebreak

\responsoriumiii

\vfill
\pagebreak

\rubrica{Reliqua omittuntur, nisi Laudes separandæ sint.}

\sineinitiali{temporalia/domineexaudi.gtex}

\vfill

\oratio

\vfill

\noindent \Vbardot{} Dómine, exáudi oratiónem meam.
\Rbardot{} Et clamor meus ad te véniat.

\vfill

\noindent \Vbardot{} Benedicámus Dómino.
\noindent \Rbardot{} Deo grátias.

\vfill

\noindent \Vbardot{} Fidélium ánimæ per misericórdiam Dei requiéscant in pace.
\Rbardot{} Amen.

\vfill
\pagebreak

\hora{Ad Laudes.} %%%%%%%%%%%%%%%%%%%%%%%%%%%%%%%%%%%%%%%%%%%%%%%%%%%%%
%\sideThumbs{Laudes}

\cantusSineNeumas

\vspace{0.5cm}
\grechangedim{interwordspacetext}{0.18 cm plus 0.15 cm minus 0.05 cm}{scalable}%
\cuminitiali{}{temporalia/deusinadiutorium-communis.gtex}
\grechangedim{interwordspacetext}{0.22 cm plus 0.15 cm minus 0.05 cm}{scalable}%

\vfill
\pagebreak

\ifx\hymnuslaudes\undefined
\ifx\laudac\undefined
\else
\pars{Hymnus}

\grechangedim{interwordspacetext}{0.16 cm plus 0.15 cm minus 0.05 cm}{scalable}%
\cuminitiali{I}{temporalia/hym-SolEcce.gtex}
\grechangedim{interwordspacetext}{0.22 cm plus 0.15 cm minus 0.05 cm}{scalable}%
\vspace{-3mm}
\fi
\else
\hymnuslaudes
\fi

\vfill
\pagebreak

\ifx\lauda\undefined
\else
\pars{Psalmus 1.}

\vspace{-4mm}

\antiphona{VIII G}{temporalia/ant-exsurgamdiluculo.gtex}

%\vspace{-2mm}

\scriptura{Psalmus 56}

%\vspace{-2mm}

\initiumpsalmi{temporalia/ps56-initium-viii-g-auto.gtex}

%\vspace{-1.5mm}

\input{temporalia/ps56-viii-g.tex} \Abardot{}

\vfill
\pagebreak

\pars{Psalmus 2.} \scriptura{Ier. 31, 14}

\vspace{-4mm}

\antiphona{IV* e}{temporalia/ant-populusmeusait.gtex}

%\vspace{-2mm}

\scriptura{Canticum Ieremiæ, 1 Ier. 31, 10-14}

%\vspace{-3mm}

\initiumpsalmi{temporalia/jeremiae3-initium-iv_-e-auto.gtex}

\input{temporalia/jeremiae3-iv_-e.tex} \Abardot{}

\vfill
\pagebreak

\pars{Psalmus 3.} \scriptura{Ps. 95, 4; \textbf{H94}}

\vspace{-4mm}

\antiphona{IV a}{temporalia/ant-magnusdominus.gtex}

\scriptura{Psalmus 47}

\initiumpsalmi{temporalia/ps47-initium-iv-a-auto.gtex}

\input{temporalia/ps47-iv-a.tex} \Abardot{}

\vfill
\pagebreak
\fi
\ifx\laudc\undefined
\else
\pars{Psalmus 1.} \scriptura{Ps. 86, 1; \textbf{H98}}

\vspace{-4mm}

\antiphona{I g}{temporalia/ant-fundamentaeius.gtex}

%\vspace{-2mm}

\scriptura{Psalmus 86}

%\vspace{-2mm}

\initiumpsalmi{temporalia/ps86-initium-i-g-auto.gtex}

%\vspace{-1.5mm}

\input{temporalia/ps86-i-g.tex} \Abardot{}

\vfill
\pagebreak

\pars{Psalmus 2.}

\vspace{-4mm}

\antiphona{II D}{temporalia/ant-eccedominusnosterbrachio.gtex}

%\vspace{-2mm}

\scriptura{Canticum Isaiæ, Is. 40, 10-17}

%\vspace{-3mm}

\initiumpsalmi{temporalia/isaiae9-initium-ii-D-auto.gtex}

\input{temporalia/isaiae9-ii-D.tex} \Abardot{}

\vfill
\pagebreak

\pars{Psalmus 3.} \scriptura{Ps. 144, 17}

\vspace{-4mm}

\antiphona{E}{temporalia/ant-iustusetsanctus.gtex}

\scriptura{Psalmus 98}

\initiumpsalmi{temporalia/ps98-initium-e.gtex}

\input{temporalia/ps98-e.tex} \Abardot{}

\vfill
\pagebreak
\fi

\ifx\lectiobrevis\undefined
\ifx\lauda\undefined
\else
\pars{Lectio Brevis.} \scriptura{Is. 66, 1-2}

\noindent Hæc dicit Dóminus: Cælum thronus meus, terra autem scabéllum pedum meórum. Quæ ista domus, quam ædificábitis mihi, et quis iste locus quiétis meæ? Omnia hæc manus mea fecit et mea sunt univérsa ista, dicit Dóminus. Ad hunc autem respíciam, ad paupérculum et contrítum spíritu et treméntem sermónes meos.
\fi
\else
\lectiobrevis
\fi

\vfill

\ifx\responsoriumbreve\undefined
\ifx\laudac\undefined
\else
\pars{Responsorium breve.} \scriptura{Ps. 118, 145}

\cuminitiali{VI}{temporalia/resp-clamaviintotocorde.gtex}
\fi
\else
\responsoriumbreve
\fi

\vfill
\pagebreak

\ifx\benedictus\undefined
\ifx\laudac\undefined
\else
\pars{Canticum Zachariæ.} \scriptura{Lc. 1, 74.75; \textbf{H423}}

%\vspace{-4mm}

{
\grechangedim{interwordspacetext}{0.18 cm plus 0.15 cm minus 0.05 cm}{scalable}%
\antiphona{VII a}{temporalia/ant-insanctitate.gtex}
\grechangedim{interwordspacetext}{0.22 cm plus 0.15 cm minus 0.05 cm}{scalable}%
}

%\vspace{-3mm}

\scriptura{Lc. 1, 68-79}

%\vspace{-2mm}

\cantusSineNeumas
\initiumpsalmi{temporalia/benedictus-initium-vii-a-auto.gtex}

%\vspace{-1.5mm}

\input{temporalia/benedictus-vii-a.tex} \Abardot{}
\fi
\else
\benedictus
\fi

\vspace{-1cm}

\vfill
\pagebreak

%\sideThumbs{{\scriptsize{}Fine horarum}}

\pars{Preces.}

\sineinitiali{}{temporalia/tonusprecum.gtex}

\ifx\preces\undefined
\ifx\lauda\undefined
\else
\noindent Grátias agámus Christo, qui lumen huius diéi nobis concédit, \gredagger{} et ad eum clamémus:

\Rbardot{} Bénedic et sanctífica nos, Dómine.

\noindent Qui te pro peccátis nostris hóstiam obtulísti, \gredagger{} incépta et propósita suscípias hodiérna.

\Rbardot{} Bénedic et sanctífica nos, Dómine.

\noindent Qui óculos nostros lucis dono lætíficas novæ, \gredagger{} lúcifer oriáris in córdibus nostris.

\Rbardot{} Bénedic et sanctífica nos, Dómine.

\noindent Tríbue hódie nos esse ómnibus longánimes, \gredagger{} ut imitatóres tui fíeri possímus.

\Rbardot{} Bénedic et sanctífica nos, Dómine.

\noindent Audítam, Dómine, fac nobis mane misericórdiam tuam. \gredagger{} Sit hódie gáudium tuum fortitúdo nostra.

\Rbardot{} Bénedic et sanctífica nos, Dómine.
\fi
\ifx\laudc\undefined
\else
\noindent Christo, bono pastóri, qui pro suis óvibus ánimam pósuit, \gredagger{} laudes grati exsolvámus et supplicémus, dicéntes:

\Rbardot{} Pasce pópulum tuum, Dómine.

\noindent Christe, qui in sanctis pastóribus misericórdiam et dilectiónem tuam dignátus es osténdere, \gredagger{} numquam désinas per eos nobíscum misericórditer ágere.

\Rbardot{} Pasce pópulum tuum, Dómine.

\noindent Qui múnere pastóris animárum fungi per tuos vicários pergis, \gredagger{} ne destíteris nos ipse per rectóres nostros dirígere.

\Rbardot{} Pasce pópulum tuum, Dómine.

\noindent Qui in sanctis tuis, populórum dúcibus, córporum animarúmque médicus exstitísti, \gredagger{} numquam cesses ministérium in nos vitæ et sanctitátis perágere.

\Rbardot{} Pasce pópulum tuum, Dómine.

\noindent Qui, prudéntia et caritáte sanctórum, tuum gregem erudísti, \gredagger{} nos in sanctitáte iúgiter per pastóres nostros ædífica.

\Rbardot{} Pasce pópulum tuum, Dómine.
\fi
\else
\preces
\fi

\vfill

\pars{Oratio Dominica.}

\cuminitiali{}{temporalia/oratiodominicaalt.gtex}

\vfill
\pagebreak

\rubrica{vel:}

\pars{Supplicatio Litaniæ.}

\cuminitiali{}{temporalia/supplicatiolitaniae.gtex}

\vfill

\pars{Oratio Dominica.}

\cuminitiali{}{temporalia/oratiodominica.gtex}

\vfill
\pagebreak

% Oratio. %%%
\oratio

\vspace{-1mm}

\vfill

\rubrica{Hebdomadarius dicit Dominus vobiscum, vel, absente sacerdote vel diacono, sic concluditur:}

\vspace{2mm}

\antiphona{C}{temporalia/dominusnosbenedicat.gtex}

\rubrica{Postea cantatur a cantore:}

\vspace{2mm}

\cuminitiali{IV}{temporalia/benedicamus-feria-laudes.gtex}

\vspace{1mm}

\vfill
\pagebreak

\end{document}

