\newcommand{\titulus}{\nomenFesti{S. Marci, Evangelistæ.}
\dies{Die 25. Aprilis.}}
\newcommand{\oratio}{\pars{Oratio.}

\noindent Deus, qui beátum Marcum evangelístam tuum evangélicæ prædicatiónis grátia sublimásti, tríbue, quǽsumus, eius nos eruditióne ita profícere, ut vestígia Christi fidéliter sequámur.

\pars{Pro pace in Ucraina.} \scriptura{Sir. 50, 25; 2 Esdr. 4, 20; \textbf{H416}}

\vspace{-4mm}

\antiphona{II D}{temporalia/ant-dapacemdomine.gtex}

\vfill

\noindent Deus, a quo sancta desidéria, recta consília et iusta sunt ópera: da servis tuis illam, quam mundus dare non potest, pacem; ut et corda nostra mandátis tuis dédita, et hóstium subláta formídine, témpora sint tua protectióne tranquílla.

\noindent Per Dóminum nostrum Iesum Christum, Fílium tuum, qui tecum vivit et regnat in unitáte Spíritus Sancti, Deus, per ómnia sǽcula sæculórum.

\noindent \Rbardot{} Amen.}
\newcommand{\invitatorium}{\pars{Invitatorium.}

\vspace{-4mm}

\antiphona{VI*}{temporalia/inv-dominumloquentem-tp.gtex}}
\newcommand{\hymnusmatutinum}{\pars{Hymnus.}

\antiphona{III}{temporalia/hym-OVirBeate.gtex}}
\newcommand{\matutinum}{\pars{Psalmus 1.} \scriptura{Cf. Ps. 91, 13; \textbf{H254}}

\vspace{-4mm}

\antiphona{I a}{temporalia/ant-sanctituidomine.gtex}

\vspace{-2mm}

\scriptura{Ps. 18, 1-7}

\vspace{-2mm}

\initiumpsalmi{temporalia/ps18i-initium-i-a-auto.gtex}

%\vspace{-1.5mm}

\input{temporalia/ps18i-i-a.tex} \Abardot{}

\vfill
\pagebreak

\pars{Psalmus 2.} \scriptura{Cf. Ps. 117, 15; Sap. 18, 1; \textbf{H254}}

\vspace{-4mm}

\antiphona{I d}{temporalia/ant-intabernaculisiustorum.gtex}

%\vspace{-2mm}

\scriptura{Ps. 63}

%\vspace{-2mm}

\initiumpsalmi{temporalia/ps63-initium-i-d-auto.gtex}

%\vspace{-1.5mm}

\input{temporalia/ps63-i-d.tex} \Abardot{}

\vfill
\pagebreak

\pars{Psalmus 3.} \scriptura{Ps. 117, 15; \textbf{H254}}

\vspace{-4mm}

\antiphona{VIII G\textsuperscript{2}}{temporalia/ant-voxlaetitiaeintabernaculis.gtex}

%\vspace{-2mm}

\scriptura{Ps. 96}

%\vspace{-2mm}

\initiumpsalmi{temporalia/ps96-initium-viii-G5-auto.gtex}

\input{temporalia/ps96-viii-G5.tex} \Abardot{}

\vfill
\pagebreak}
\newcommand{\matversus}{\noindent \Vbardot{} Dóminus dabit verbum evangelizántibus, allelúia.

\noindent \Rbardot{} Virtúte multa, allelúia.}
\newcommand{\lectioi}{\pars{Lectio I.} \scriptura{Eph. 4, 1-16}

\noindent De Epístola beáti Pauli apóstoli ad Ephésios.

\noindent Fratres: Obsecro vos ego vinctus in Dómino, ut digne ambulétis vocatióne, qua vocáti estis, cum omni humilitáte et mansuetúdine, cum longanimitáte, supportántes ínvicem in caritáte, sollíciti serváre unitátem spíritus in vínculo pacis; unum corpus et unus Spíritus, sicut et vocáti estis in una spe vocatiónis vestræ; unus Dóminus, una fides, unum baptísma; unus Deus et Pater ómnium, qui super omnes et per ómnia et in ómnibus.

\noindent Unicuíque autem nostrum data est grátia secúndum mensúram donatiónis Christi. Propter quod dicit:

\noindent \emph{«Ascéndens in altum captívam duxit captivitátem, dedit dona homínibus».}

\noindent Illud autem «ascéndit» quid est, nisi quia et descéndit in inferióres partes terræ? Qui descéndit, ipse est et qui ascéndit super omnes cælos, ut impléret ómnia.

\noindent Et ipse dedit quosdam quidem apóstolos, quosdam autem prophétas, álios vero evangelístas, álios autem pastóres et doctóres ad instructiónem sanctórum in opus ministérii, in ædificatiónem córporis Christi, donec occurrámus omnes in unitátem fídei et agnitiónis Fílii Dei, in virum perféctum, in mensúram ætátis plenitúdinis Christi, ut iam non simus párvuli fluctuántes et circumácti omni vento doctrínæ in fallácia hóminum, in astútia ad circumventiónem erróris; veritátem autem faciéntes in caritáte crescámus in illum per ómnia, qui est caput Christus, ex quo totum corpus compáctum et conéxum per omnem iunctúram subministratiónis secúndum operatiónem in mensúra uniuscuiúsque partis augméntum córporis facit in ædificatiónem sui in caritáte.}
\newcommand{\responsoriumi}{\pars{Responsorium 1.} \scriptura{\Rbardot{} Ps. 111, 1 \Vbardot{} ibid., 2; \textbf{H252}}

\vspace{-5mm}

\responsorium{VIII}{temporalia/resp-beatusvir-CROCHU.gtex}{}}
\newcommand{\lectioii}{\pars{Lectio II.} \scriptura{Lib. 1, 10, 1-3: PG 7, 550-554}

\noindent Ex Tractátu sancti Irenǽi epíscopi Advérsus hǽreses.

\noindent Ecclésia, per univérsum orbem usque ad fines terræ dissemináta, et ab Apóstolis et a discípulis eórum accépit eam fidem, quæ est in unum Deum, Patrem omnipoténtem, \emph{qui fecit cælum et terram, et mare, et ómnia quæ in eis sunt}; et in unum Iesum Christum Fílium Dei, incarnátum pro nostra salúte; et in Spíritum Sanctum, qui per prophétas prædicávit dispositiónes Dei, et advéntum, et eam quæ est ex Vírgine generatiónem, et passiónem, et resurrectiónem a mórtuis, et in carne in cælos ascensiónem dilécti Iesu Christi Dómini nostri, et de cælis in glória Patris advéntum eius, ad \emph{recapitulánda univérsa}, et resuscitándam omnem carnem humáni géneris, ut Christo Iesu Dómino nostro et Deo et salvatóri et regi, secúndum plácitum Patris invisíbilis, \emph{omne genu curvet cæléstium et terréstrium et infernórum et omnis lingua confiteátur ei}, et iudícium iustum in ómnibus fáciat.}
\newcommand{\responsoriumii}{\pars{Responsorium 2.} \scriptura{\Vbardot{} Sap. 5, 16; \textbf{H253}}

\vspace{-5mm}

\responsorium{III}{temporalia/resp-luxperpetualucebit-CROCHU.gtex}{}}
\newcommand{\lectioiii}{\pars{Lectio III.}

\noindent Hanc prædicatiónem cum accéperit, et hanc fidem, quemádmodum prædíximus, Ecclésia, et quidem in univérsum mundum dissemináta, diligénter custódit, quasi unam domum inhábitans; et simíliter credit iis, vidélicet quasi unam ánimam habens, et unum cor, et consonánter hæc prǽdicat, et docet, et tradit, quasi unum póssidens os. Nam, etsi in mundo loquélæ dissímiles sunt, sed tamen virtus traditiónis una et éadem est.

\noindent {\color{gray} Et neque hæ quæ in Germánia sunt fundátæ Ecclésiæ áliter credunt aut áliter tradunt, neque hæ quæ in Hibériis sunt, neque hæ quæ in Celtis, neque hæ quæ in Oriénte, neque hæ quæ in Ægýpto, neque hæ quæ in Líbya, neque hæ quæ in médio mundi constitútæ; sed sicut sol, creatúra Dei, in univérso mundo unus et idem est, sic et prædicátio veritátis ubíque lucet, et illúminat omnes hómines, qui volunt ad cognitiónem veritátis perveníre.

\noindent Et neque is qui valde prǽvalet in sermóne, ex iis qui præsunt Ecclésiis, ália, quam hæc sunt, dicet (nemo enim super magístrum est); neque infírmus in dicéndo deminorábit traditiónem. Cum enim una et éadem fides sit, neque is qui multum de ea potest dícere ámpliat, neque is qui minus, demínorat.}}
\newcommand{\responsoriumiii}{\pars{Responsorium 3.} \scriptura{\Rbardot{} Io. 15, 5 \Vbardot{} ibid., 9; \textbf{H253}}

\vspace{-5mm}

\responsorium{V}{temporalia/resp-egosumvitisvera-CROCHU-cumdox.gtex}{}

\vfill
\pagebreak

\pars{Hymnus Ambrosianus} \scriptura{Alio modo, iuxta morem Romanum}

\vspace{-2mm}

\grechangedim{interwordspacetext}{0.26 cm plus 0.15 cm minus 0.05 cm}{scalable}%
\cuminitiali{III}{temporalia/tedeum-romanum-gn.gtex}
\grechangedim{interwordspacetext}{0.22 cm plus 0.15 cm minus 0.05 cm}{scalable}%
}
\newcommand{\hymnuslaudes}{\pars{Hymnus}

\cuminitiali{I}{temporalia/hym-IamNunc-Marcus.gtex}}
\newcommand{\laudes}{\pars{Psalmus 1.} \scriptura{Cf. Prv. 3, 20}

\vspace{-4mm}

\antiphona{IV d}{temporalia/ant-sapientiadominievangelii-tp.gtex}

\vspace{-2mm}

\scriptura{Psalmus 62.}

\vspace{-1mm}

\initiumpsalmi{temporalia/ps62-initium-iv-d-auto.gtex}

%\vspace{-1.5mm}

\input{temporalia/ps62-iv-d.tex} \Abardot{}

\vfill
\pagebreak

\pars{Psalmus 2.} \scriptura{\textbf{H333}}

\vspace{-4mm}

\antiphona{VIII G\textsuperscript{2}}{temporalia/ant-istisuntvirisanctiet-tp.gtex}

%\vspace{-2mm}

\scriptura{Canticum trium puerorum, Dan. 3, 57-88 et 56}

\initiumpsalmi{temporalia/dan3-initium-viii-G5-auto.gtex}

\input{temporalia/dan3-viii-G5-sinedox.tex}

\rubrica{Hic non dicitur Gloria Patri, neque Amen.}

\vfill

\antiphona{}{temporalia/ant-istisuntvirisanctiet-tp.gtex}

\vfill
\pagebreak

\pars{Psalmus 3.} \scriptura{Eccli. 39, 12}

\vspace{-4mm}

\antiphona{III a}{temporalia/ant-collaudabuntmultisapientiam-tp.gtex}

%\vspace{-2mm}

\scriptura{Psalmus 149}

%\vspace{-2mm}

\initiumpsalmi{temporalia/ps149-initium-iii-a-auto.gtex}

\input{temporalia/ps149-iii-a.tex} \Abardot{}

\vfill
\pagebreak}
\newcommand{\lectiobrevis}{\pars{Lectio Brevis.} \scriptura{1 Cor. 15, 1-2.3-4}

\noindent Notum vobis fácio, fratres, evangélium, quod evangelizávi vobis, quod et accepístis, in quo et statis, per quod et salvámini. Trádidi enim vobis in primis, quod et accépi, quóniam Christus mórtuus est pro peccátis nostris secúndum Scriptúras et quia sepúltus est et quia suscitátus est tértia die secúndum Scriptúras.}
\newcommand{\responsoriumbreve}{\pars{Responsorium breve.} \scriptura{Cf. Ps. 77, 4}

\cuminitiali{VI}{temporalia/resp-narraveruntlaudes-tp.gtex}}
\newcommand{\preces}{\noindent Salvatórem nostrum, qui, mortem déstruens, illuminávit vitam et incorruptiónem per Evangélium \gredagger{}, dignis láudibus celebrémus \grestar{}, humíliter deprecántes:

\Rbardot{} Ecclésiam tuam in fide et caritáte confírma.

\noindent Qui Ecclésiam tuam per sanctos et exímios doctóres mirabíliter illustrásti, \grestar{} fac, ut christiáni eódem semper lætificéntur splendóre.

\Rbardot{} Ecclésiam tuam in fide et caritáte confírma.

\noindent Qui, cum sancti te pastóres sicut Móyses orárent, \gredagger{} pópuli peccáta dimisísti, \grestar{} per intercessiónem eórum Ecclésiam tuam contínua purificatióne sanctífica.

\Rbardot{} Ecclésiam tuam in fide et caritáte confírma.

\noindent Qui sanctos tuos unxísti in médio fratrum \gredagger{} et Spíritum tuum in illos direxísti, \grestar{} reple Spíritu Sancto omnes pópuli tui rectóres.

\Rbardot{} Ecclésiam tuam in fide et caritáte confírma.

\noindent Qui pastórum sanctórum ipse posséssio exstitísti, \grestar{} tríbue nullum ex iis, quos sánguine acquisísti, sine te manére.

\Rbardot{} Ecclésiam tuam in fide et caritáte confírma.}
\newcommand{\benedictus}{\pars{Canticum Zachariæ.} \scriptura{Mc. 16, 15-16}

\vspace{-4mm}

\antiphona{I f}{temporalia/ant-euntesinmundumuniversum.gtex}

\vspace{-2mm}

\scriptura{Lc. 1, 68-79}

\vspace{-2mm}

\cantusSineNeumas
\initiumpsalmi{temporalia/benedictus-initium-i-f-auto.gtex}

%\vspace{-1.5mm}

\input{temporalia/benedictus-i-f.tex} \Abardot{}}
\newcommand{\hebdomada}{infra Hebdom. III Adventus.}
\newcommand{\oratioLaudes}{\cuminitiali{}{temporalia/oratio3vo.gtex}}
\newcommand{\responsoriumbreve}{\pars{Responsorium breve.} \scriptura{Is. 60, 2; \textbf{H20}}

\cuminitiali{IV}{temporalia/resp-superte.gtex}}

% LuaLaTeX

\documentclass[a4paper, twoside, 12pt]{article}
\usepackage[latin]{babel}
%\usepackage[landscape, left=3cm, right=1.5cm, top=2cm, bottom=1cm]{geometry} % okraje stranky
%\usepackage[landscape, a4paper, mag=1166, truedimen, left=2cm, right=1.5cm, top=1.6cm, bottom=0.95cm]{geometry} % okraje stranky
\usepackage[landscape, a4paper, mag=1400, truedimen, left=0.5cm, right=0.5cm, top=0.5cm, bottom=0.5cm]{geometry} % okraje stranky

\usepackage{fontspec}
\setmainfont[FeatureFile={junicode.fea}, Ligatures={Common, TeX}, RawFeature=+fixi]{Junicode}
%\setmainfont{Junicode}

% shortcut for Junicode without ligatures (for the Czech texts)
\newfontfamily\nlfont[FeatureFile={junicode.fea}, Ligatures={Common, TeX}, RawFeature=+fixi]{Junicode}

\usepackage{multicol}
\usepackage{color}
\usepackage{lettrine}
\usepackage{fancyhdr}

% usual packages loading:
\usepackage{luatextra}
\usepackage{graphicx} % support the \includegraphics command and options
\usepackage{gregoriotex} % for gregorio score inclusion
\usepackage{gregoriosyms}
\usepackage{wrapfig} % figures wrapped by the text
\usepackage{parcolumns}
\usepackage[contents={},opacity=1,scale=1,color=black]{background}
\usepackage{tikzpagenodes}
\usepackage{calc}
\usepackage{longtable}
\usetikzlibrary{calc}

\setlength{\headheight}{14.5pt}

\input{conventuscommune.tex} % Often used macros

\newcommand{\annusEditionis}{2021}

%%%% Vicekrat opakovane kousky

\newcommand{\anteOrationem}{
  \rubrica{Ante Orationem, cantatur a Superiore:}

  \pars{Supplicatio Litaniæ.}

  \cuminitiali{}{temporalia/supplicatiolitaniae.gtex}

  \pars{Oratio Dominica.}

  \cuminitiali{}{temporalia/oratiodominica.gtex}

  \rubrica{Deinde dicitur ab Hebdomadario:}

  \cuminitiali{}{temporalia/dominusvobiscum-solemnis.gtex}

  \rubrica{In choro monialium loco Dominus vobiscum dicitur:}

  \sineinitiali{temporalia/domineexaudi.gtex}
}

\setlength{\columnsep}{30pt} % prostor mezi sloupci

%%%%%%%%%%%%%%%%%%%%%%%%%%%%%%%%%%%%%%%%%%%%%%%%%%%%%%%%%%%%%%%%%%%%%%%%%%%%%%%%%%%%%%%%%%%%%%%%%%%%%%%%%%%%%
\begin{document}

% Here we set the space around the initial.
% Please report to http://home.gna.org/gregorio/gregoriotex/details for more details and options
\grechangedim{afterinitialshift}{2.2mm}{scalable}
\grechangedim{beforeinitialshift}{2.2mm}{scalable}
\grechangedim{interwordspacetext}{0.22 cm plus 0.15 cm minus 0.05 cm}{scalable}%
\grechangedim{annotationraise}{-0.2cm}{scalable}

% Here we set the initial font. Change 38 if you want a bigger initial.
% Emit the initials in red.
\grechangestyle{initial}{\color{red}\fontsize{38}{38}\selectfont}

\pagestyle{empty}

%%%% Titulni stranka
\begin{titulusOfficii}
\ifx\titulus\undefined
\nomenFesti{Feria III \hebdomada{}}
\else
\titulus
\fi
\end{titulusOfficii}

\vfill

\begin{center}
%Ad usum et secundum consuetudines chori \guillemotright{}Conventus Choralis\guillemotleft.

%Editio Sancti Wolfgangi \annusEditionis
\end{center}

\scriptura{}

\pars{}

\pagebreak

\renewcommand{\headrulewidth}{0pt} % no horiz. rule at the header
\fancyhf{}
\pagestyle{fancy}

\cantusSineNeumas

\ifx\oratio\undefined
\ifx\laudb\undefined
\else
\newcommand{\oratio}{\pars{Oratio.}

\noindent Dómine Iesu Christe, lux vera, qui omnes hómines illúminas ad salútem, nobis, quǽsumus, concéde virtútem, ut ante te vias pacis et iustítiæ præparémus.

\noindent Qui vivis et regnas cum Deo Patre in unitáte Spíritus Sancti, Deus, per ómnia sǽcula sæculórum.

\noindent \Rbardot{} Amen.}
\fi
\fi

\hora{Ad Matutinum.} %%%%%%%%%%%%%%%%%%%%%%%%%%%%%%%%%%%%%%%%%%%%%%%%%%%%%

\vspace{2mm}

\cuminitiali{}{temporalia/dominelabiamea.gtex}

\vfill
%\pagebreak

\vspace{2mm}

\ifx\invitatorium\undefined
\ifx\matuac\undefined
\else
\pars{Invitatorium.} \scriptura{Ps. 94, 1; Psalmus 94; \textbf{H451}}

\vspace{-6mm}

\antiphona{VI}{temporalia/inv-jubilemusdeo.gtex}
\fi
\ifx\matubd\undefined
\else
\pars{Invitatorium.} \scriptura{Cantor; Psalmus 94; \textbf{H449}}

\vspace{-6mm}

\antiphona{E}{temporalia/inv-regemmagnum.gtex}
\fi
\else
\invitatorium
\fi

\vfill
\pagebreak

\ifx\hymnusmatutinum\undefined
\ifx\matuac\undefined
\else
\pars{Hymnus}

\cuminitiali{IV}{temporalia/hym-SomnoRefectis.gtex}
\fi
\ifx\matubd\undefined
\else
\pars{Hymnus.} \scriptura{Gregorius Magnus (\olddag{} 604)}

{
\grechangedim{interwordspacetext}{0.10 cm plus 0.15 cm minus 0.05 cm}{scalable}%
\antiphona{I}{temporalia/hym-NocteSurgentes.gtex}
\grechangedim{interwordspacetext}{0.22 cm plus 0.15 cm minus 0.05 cm}{scalable}%
}
\fi
\else
\hymnusmatutinum
\fi

\vspace{-3mm}

\vfill
\pagebreak

\ifx\matub\undefined
\else
% MAT B
\pars{Psalmus 1.} \scriptura{Ps. 36, 5; \textbf{H93}}

\vspace{-4mm}

\antiphona{VI F}{temporalia/ant-reveladomino.gtex}

%\vspace{-2mm}

\scriptura{Ps. 36, 1-11}

%\vspace{-2mm}

\initiumpsalmi{temporalia/ps36i_xi-initium-vi-F-auto.gtex}

\input{temporalia/ps36i_xi-vi-F.tex} \Abardot{}

\vfill
\pagebreak

\pars{Psalmus 2.}

\vspace{-4mm}

\antiphona{II D}{temporalia/ant-iuniorfui.gtex}

\vspace{-2mm}

\scriptura{Ps. 36, 12-29}

\vspace{-2mm}

\initiumpsalmi{temporalia/ps36xii_xxix-initium-ii-D-auto.gtex}

\input{temporalia/ps36xii_xxix-ii-D.tex}

\vfill

\antiphona{}{temporalia/ant-iuniorfui.gtex}

\vfill
\pagebreak

\pars{Psalmus 3.} \scriptura{Ps. 36, 3}

\vspace{-4mm}

\antiphona{VI F}{temporalia/ant-speraindomino.gtex}

%\vspace{-2mm}

\scriptura{Ps. 36, 30-40}

%\vspace{-2mm}

\initiumpsalmi{temporalia/ps36iii-initium-vi-F-auto.gtex}

\input{temporalia/ps36iii-vi-F.tex} \Abardot{}

\vfill
\pagebreak
\fi
\ifx\matuc\undefined
\else
% MAT C
\pars{Psalmus 1.} \scriptura{Ps. 67, 2}

\vspace{-4mm}

\antiphona{VII a}{temporalia/ant-exsurgatdeus.gtex}

%\vspace{-2mm}

\scriptura{Ps. 67, 2-11}

\initiumpsalmi{temporalia/ps67i-initium-vii-a-auto.gtex}

\input{temporalia/ps67i-vii-a.tex} \Abardot{}

\vfill
\pagebreak

\pars{Psalmus 2.}

\vspace{-4mm}

\antiphona{I f}{temporalia/ant-deusnosterdeussalvos.gtex}

%\vspace{-2mm}

\scriptura{Ps. 67, 12-24}

%\vspace{-2mm}

\initiumpsalmi{temporalia/ps67ii-initium-i-f-auto.gtex}

\input{temporalia/ps67ii-i-f.tex} \Abardot{}

\vfill
\pagebreak

\pars{Psalmus 3.} \scriptura{Ps. 67, 27; \textbf{H96}}

\vspace{-4mm}

\antiphona{D}{temporalia/ant-inecclesiis.gtex}

%\vspace{-2mm}

\scriptura{Ps. 67, 25-36}

\initiumpsalmi{temporalia/ps67iii-initium-d-g2-auto.gtex}

\input{temporalia/ps67iii-d-g2.tex} \Abardot{}

\vfill
\pagebreak
\fi

\pars{Versus.}

\ifx\matversus\undefined
\ifx\matub\undefined
\else
\noindent \Vbardot{} Bonitátem et prudéntiam et sciéntiam doce me.

\noindent \Rbardot{} Quia præcéptis tuis crédidi.
\fi
\ifx\matuc\undefined
\else
\noindent \Vbardot{} Audiam quid loquátur Dóminus Deus.

\noindent \Rbardot{} Loquétur pacem ad plebem suam.
\fi
\else
\matversus
\fi

\vspace{5mm}

\sineinitiali{temporalia/oratiodominica-mat.gtex}

\vspace{5mm}

\pars{Absolutio.}

\cuminitiali{}{temporalia/absolutio-ipsius.gtex}

\vfill
\pagebreak

\cuminitiali{}{temporalia/benedictio-solemn-deus.gtex}

\vspace{7mm}

\lectioi

\noindent \Vbardot{} Tu autem, Dómine, miserére nobis.
\noindent \Rbardot{} Deo grátias.

\vfill
\pagebreak

\responsoriumi

\vfill
\pagebreak

\cuminitiali{}{temporalia/benedictio-solemn-christus.gtex}

\vspace{7mm}

\lectioii

\noindent \Vbardot{} Tu autem, Dómine, miserére nobis.
\noindent \Rbardot{} Deo grátias.

\vfill
\pagebreak

\responsoriumii

\vfill
\pagebreak

\cuminitiali{}{temporalia/benedictio-solemn-ignem.gtex}

\vspace{7mm}

\lectioiii

\noindent \Vbardot{} Tu autem, Dómine, miserére nobis.
\noindent \Rbardot{} Deo grátias.

\vfill
\pagebreak

\responsoriumiii

\vfill
\pagebreak

\rubrica{Reliqua omittuntur, nisi Laudes separandæ sint.}

\sineinitiali{temporalia/domineexaudi.gtex}

\vfill

\oratio

\vfill

\noindent \Vbardot{} Dómine, exáudi oratiónem meam.
\Rbardot{} Et clamor meus ad te véniat.

\vfill

\noindent \Vbardot{} Benedicámus Dómino.
\noindent \Rbardot{} Deo grátias.

\vfill

\noindent \Vbardot{} Fidélium ánimæ per misericórdiam Dei requiéscant in pace.
\Rbardot{} Amen.

\vfill
\pagebreak

\hora{Ad Laudes.} %%%%%%%%%%%%%%%%%%%%%%%%%%%%%%%%%%%%%%%%%%%%%%%%%%%%%

\cantusSineNeumas

\vspace{0.5cm}
\grechangedim{interwordspacetext}{0.18 cm plus 0.15 cm minus 0.05 cm}{scalable}%
\cuminitiali{}{temporalia/deusinadiutorium-communis.gtex}
\grechangedim{interwordspacetext}{0.22 cm plus 0.15 cm minus 0.05 cm}{scalable}%

\vfill
\pagebreak

\ifx\hymnuslaudes\undefined
\ifx\laudac\undefined
\else
\pars{Hymnus} \scriptura{Ambrosius (\olddag{} 397)}

\cuminitiali{I}{temporalia/hym-SplendorPaternae-hiemalis.gtex}
\fi
\ifx\laudbd\undefined
\else
\pars{Hymnus}

\grechangedim{interwordspacetext}{0.16 cm plus 0.15 cm minus 0.05 cm}{scalable}%
\cuminitiali{IV}{temporalia/hym-AEterneLucis.gtex}
\grechangedim{interwordspacetext}{0.22 cm plus 0.15 cm minus 0.05 cm}{scalable}%
\vspace{-3mm}
\fi
\else
\hymnuslaudes
\fi

\vfill
\pagebreak

\ifx\laudb\undefined
\else
\pars{Psalmus 1.} \scriptura{Ps. 42, 5; \textbf{H95}}

\vspace{-4mm}

\antiphona{VI F}{temporalia/ant-salutarevultusmei.gtex}

\scriptura{Psalmus 42.}

\initiumpsalmi{temporalia/ps42-initium-vi-F-auto.gtex}

\input{temporalia/ps42-vi-F.tex} \Abardot{}

\vfill
\pagebreak

\pars{Psalmus 2.} \scriptura{Is. 38, 20; \textbf{H95}}

\vspace{-7mm}

\antiphona{E}{temporalia/ant-cunctisdiebus.gtex}

\vspace{-4mm}

\scriptura{Canticum Ezechiæ, Is. 38, 10-20}

\vspace{-3mm}

\initiumpsalmi{temporalia/ezechiae-initium-e-auto.gtex}

\input{temporalia/ezechiae-e.tex} \Abardot{}

\vfill
\pagebreak

\pars{Psalmus 3.} \scriptura{Ps. 64, 2; \textbf{H96}}

\vspace{-4mm}

\antiphona{VIII a}{temporalia/ant-tedecet.gtex}

\vspace{-2mm}

\scriptura{Psalmus 64.}

\vspace{-2mm}

\initiumpsalmi{temporalia/ps64-initium-viii-A-auto.gtex}

\input{temporalia/ps64-viii-A.tex} \Abardot{}

\vfill
\pagebreak
\fi
\ifx\laudc\undefined
\else
\pars{Psalmus 1.} \scriptura{Ps. 83, 5}

\vspace{-4mm}

\antiphona{VIII G}{temporalia/ant-beatiquihabitant.gtex}

\vspace{-2mm}

\scriptura{Psalmus 84.}

\vspace{-2mm}

\initiumpsalmi{temporalia/ps84-initium-viii-G-auto.gtex}

\input{temporalia/ps84-viii-G.tex} \Abardot{}

\vfill
\pagebreak

\pars{Psalmus 2.}

\vspace{-4mm}

\antiphona{VII d}{temporalia/ant-denoctespiritusmeus.gtex}

\vspace{-2mm}

\scriptura{Canticum Isaiæ, Is. 26, 1-12}

\vspace{-2mm}

\initiumpsalmi{temporalia/isaiae3-initium-vii-d.gtex}

\input{temporalia/isaiae3-vii-d.tex} \Abardot{}

\vfill
\pagebreak

\pars{Psalmus 3.} \scriptura{Ps. 66, 2}

\vspace{-4mm}

\antiphona{E}{temporalia/ant-illuminadomine.gtex}

%\vspace{-2mm}

\scriptura{Psalmus 66.}

%\vspace{-2mm}

\initiumpsalmi{temporalia/ps66-initium-e.gtex}

\input{temporalia/ps66-e.tex} \Abardot{}

\vfill
\pagebreak
\fi

\ifx\lectiobrevis\undefined
\ifx\laudb\undefined
\else
\pars{Lectio Brevis.} \scriptura{1 Th. 5, 4-5}

\noindent Vos, fratres, non estis in ténebris, ut vos dies ille tamquam fur comprehéndat; omnes enim vos fílii lucis estis et fílii diéi. Non sumus noctis neque tenebrárum.
\fi
\ifx\laudc\undefined
\else
\pars{Lectio Brevis.} \scriptura{1 Io. 4, 14-15}

\noindent Nos vídimus et testificámur quóniam Pater misit Fílium salvatórem mundi. Quisque conféssus fúerit: Iesus est Fílius Dei, Deus in ipso manet, et ipse in Deo.
\fi
\else
\lectiobrevis
\fi

\vfill

\ifx\responsoriumbreve\undefined
\ifx\laudac\undefined
\else
\pars{Responsorium breve.}

\cuminitiali{VI}{temporalia/resp-benedictusdominus.gtex}
\fi
\ifx\laudbd\undefined
\else
\pars{Responsorium breve.} \scriptura{Ps. 118, 149.147}

\cuminitiali{VI}{temporalia/resp-vocemmeamaudi.gtex}
\fi
\else
\responsoriumbreve
\fi

\vfill
\pagebreak

\ifx\benedictus\undefined
\ifx\laudbd\undefined
\else
\pars{Canticum Zachariæ.} \scriptura{Lc. 1, 71; \textbf{H423}}

\vspace{-5mm}

{
\grechangedim{interwordspacetext}{0.18 cm plus 0.15 cm minus 0.05 cm}{scalable}%
\antiphona{I g\textsuperscript{5}}{temporalia/ant-demanuomnium.gtex}
\grechangedim{interwordspacetext}{0.22 cm plus 0.15 cm minus 0.05 cm}{scalable}%
}

%\vspace{-3mm}

\scriptura{Lc. 1, 68-79}

%\vspace{-1mm}

\initiumpsalmi{temporalia/benedictus-initium-i-g5-auto.gtex}

\input{temporalia/benedictus-i-g5.tex} \Abardot{}
\fi
\else
\benedictus
\fi

\vspace{-1cm}

\vfill
\pagebreak

\pars{Preces.}

\sineinitiali{}{temporalia/tonusprecum.gtex}

\ifx\preces\undefined
\ifx\laudb\undefined
\else
\noindent Salvatóri nostro benedicámus, qui sua resurrectióne mundum clarificávit, \gredagger{} et humíliter invocémus eum dicéntes:

\Rbardot{} Salva nos, Dómine, in sémita tua.

\noindent Resurrectiónem tuam, Dómine Iesu, oratióne cólimus matutína, \gredagger{} spes glóriæ tuæ diem nostrum illúminet.

\Rbardot{} Salva nos, Dómine, in sémita tua.

\noindent Súscipe, Dómine, vota et propósita nostra, \gredagger{} tamquam diéi nostri primítias.

\Rbardot{} Salva nos, Dómine, in sémita tua.

\noindent Tríbue in dilectióne tua nos hódie profícere, \gredagger{} ut ómnia in nostrum omniúmque bonum cooperéntur.

\Rbardot{} Salva nos, Dómine, in sémita tua.

\noindent Da, Dómine, sic lucére lucem nostram coram homínibus, \gredagger{} ut vídeant ópera nostra bona et Patrem gloríficent.

\Rbardot{} Salva nos, Dómine, in sémita tua.
\fi
\else
\preces
\fi

\vfill

\pars{Oratio Dominica.}

\cuminitiali{}{temporalia/oratiodominicaalt.gtex}

\vfill
\pagebreak

\rubrica{vel:}

\pars{Supplicatio Litaniæ.}

\cuminitiali{}{temporalia/supplicatiolitaniae.gtex}

\vfill

\pars{Oratio Dominica.}

\cuminitiali{}{temporalia/oratiodominica.gtex}

\vfill
\pagebreak

% Oratio. %%%
\oratio

\vspace{-1mm}

\vfill

\rubrica{Hebdomadarius dicit Dominus vobiscum, vel, absente sacerdote vel diacono, sic concluditur:}

\vspace{2mm}

\antiphona{C}{temporalia/dominusnosbenedicat.gtex}

\rubrica{Postea cantatur a cantore:}

\vspace{2mm}

\cuminitiali{IV}{temporalia/benedicamus-feria-laudes.gtex}

\vspace{1mm}

\vfill
\pagebreak

\end{document}

