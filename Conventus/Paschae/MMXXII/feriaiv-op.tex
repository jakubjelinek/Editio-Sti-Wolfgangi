\newcommand{\titulus}{\nomenFesti{Feria IV infra Octavam Paschæ.}
\celebratio{1. Classis. Duplex}}
\newcommand{\versusMatutinum}{\noindent \Vbardot{} Gavísi sunt discípuli, allelúia.
\noindent \Rbardot{} Viso Dómino, allelúia.}
\newcommand{\absolutio}{\cuminitiali{}{temporalia/absolutio-avinculis.gtex}}
\newcommand{\lectioi}{\pars{Lectio I.} \scriptura{Io. 21, 1-14}

\noindent Léctio sancti Evangélii secúndum Ioánnem.

\noindent In illo témpore: Manifestávit se íterum Iesus discípulis ad mare Tiberíadis. Manifestávit autem sic: Erant simul Simon Petrus, et Thomas, qui dícitur Dídymus. Et réliqua.

\vspace{7mm}

\noindent Homilía sancti Gregórii Papæ.

\scriptura{Homilia 24 in Evangelia}

\noindent Léctio sancti Evangélii, quæ modo in áuribus vestris lecta est, fratres mei, quæstióne ánimum pulsat, sed pulsatióne sua vim discretiónis índicat. Quæri étenim potest, cur Petrus, qui piscátor ante conversiónem fuit, post conversiónem ad piscatiónem rédiit: et cum Véritas dicat: Nemo mittens manum suam ad arátrum, et aspíciens retro, aptus est regno Dei: cur repétiit quod derelíquit? Sed si virtus discretiónis inspícitur, cítius vidétur: quia nimírum negótium, quod ante conversiónem sine peccáto éxstitit, hoc étiam post conversiónem repétere culpa non fuit.}
\newcommand{\responsoriumi}{\pars{Responsorium 1.} \scriptura{\Rbardot{} Ap. 5, 5; \Vbardot{} ibid. 5, 5.12; \textbf{H236}}

\vspace{-5mm}

\responsorium{VII}{temporalia/resp-eccevicitleo-CROCHU.gtex}{}}
\newcommand{\lectioii}{\pars{Lectio II.}

\noindent Nam piscatórem Petrum, Matthǽum vero teloneárium scimus: et post conversiónem suam ad piscatiónem Petrus rédiit, Matthǽus vero ad telónei negótium non resédit: quia áliud est victum per piscatiónem quǽrere, áliud autem telónei lucris pecúnias augére. Sunt enim pléraque negótia, quæ sine peccátis exhibéri aut vix, aut nullátenus possunt. Quæ ergo ad peccátum ímplicant, ad hæc necésse est, ut post conversiónem ánimus non recúrrat.}
\newcommand{\responsoriumii}{\pars{Responsorium 2.} \scriptura{\Vbardot{} Cf. Mt. 28, 7; \textbf{H236}}

\vspace{-5mm}

\responsorium{VII}{temporalia/resp-istisuntagninovelli-CROCHU.gtex}{}}
\newcommand{\lectioiii}{\pars{Lectio III.}

\noindent Quæri étiam potest, cur discípulis in mari laborántibus, post resurrectiónem suam Dóminus in líttore stetit, qui ante resurrectiónem suam coram discípulis in fluctibus maris ambulávit. Cuius rei rátio festíne cognóscitur, si ipsa, quæ tunc ínerat, causa pensétur. Quid enim mare, nisi præsens sǽculum signat, quod se cásuum tumúltibus, et undis vitæ corruptíbilis illídit? Quid per soliditátem líttoris, nisi illa perpetúitas quiétis ætérnæ figurátur? Quia ergo discípuli adhuc flúctibus mortális vitæ ínerant, in mari laborábant: quia autem Redémptor noster iam corruptiónem carnis excésserat, post resurrectiónem suam in líttore stabat.}
\newcommand{\responsoriumiii}{\pars{Responsorium 3.} \scriptura{\Rbardot{} Ct. 4, 11 \Vbardot{} Ps. 118, 103; \textbf{H236}}

\vspace{-5mm}

\responsorium{VII}{temporalia/resp-deoreprudentis-CROCHU-cumdox.gtex}{}}
\newcommand{\oratioMatutinum}{\noindent Deus, qui nos resurrectiónis Domínicæ ánnua solemnitáte lætíficas: \gredagger{} concéde propítius; \grestar{} ut per temporália festa quæ ágimus, perveníre ad gáudia ætérna mereámur. Per eúmdem Dóminum.}
%\newcommand{\oratioLaudes}{\cuminitiali{}{temporalia/oratio4.gtex}}
\newcommand{\oratioLaudes}{\pars{Oratio.}

\noindent Orémus. Deus, qui nos resurrectiónis Domínicæ ánnua solemnitáte lætíficas: \gredagger{} concéde propítius; \grestar{} ut per temporália festa quæ ágimus, perveníre ad gáudia ætérna mereámur.

\vfill

\pars{Pro pace in Ucraina.} \scriptura{Sir. 50, 25; 2 Esdr. 4, 20; \textbf{H416}}

\vspace{-4mm}

\antiphona{II D}{temporalia/ant-dapacemdomine.gtex}

\vfill

\noindent Deus, a quo sancta desidéria, recta consília et iusta sunt ópera: da servis tuis illam, quam mundus dare non potest, pacem; ut et corda nostra mandátis tuis dédita, et hóstium subláta formídine, témpora sint tua protectióne tranquílla.

\noindent Per Dóminum nostrum Iesum Christum, Fílium tuum, qui tecum vivit et regnat in unitáte Spíritus Sancti, Deus, per ómnia sǽcula sæculórum.

\noindent \Rbardot{} Amen.

\vspace{1mm}}
\newcommand{\benedictus}{\pars{Canticum Zachariæ.} \scriptura{\Abardot{} Io. 21, 6; \textbf{H236}}

\vspace{-4mm}

\antiphona{VII c}{temporalia/ant-mittite.gtex}

\vspace{-3mm}

\scriptura{Lc. 1, 68-79}

\initiumpsalmi{temporalia/benedictus-initium-viisoll-c-auto.gtex}

\input{temporalia/benedictus-viisoll-c.tex} \Abardot{}}
\newcommand{\magnificat}{\pars{Canticum B. Mariæ V.} \scriptura{\Abardot{} Io. 21, 10-11; \textbf{H236}}

\vspace{-6mm}

\antiphona{VIII G}{temporalia/ant-dixit.gtex}

\vspace{-3mm}

\scriptura{Lc. 1, 46-55}

\vspace{-2mm}

\initiumpsalmi{temporalia/magnificat-initium-viiisoll-g.gtex}

\vspace{-1mm}

\input{temporalia/magnificat-viiisoll-g.tex} \Abardot{}}
% LuaLaTeX

\documentclass[a4paper, twoside, 12pt]{article}
\usepackage[latin]{babel} 
%\usepackage[landscape, left=3cm, right=1.5cm, top=2cm, bottom=1cm]{geometry} % okraje stranky
%\usepackage[landscape, a4paper, mag=1166, truedimen, left=2cm, right=1.5cm, top=1.6cm, bottom=0.95cm]{geometry} % okraje stranky
\usepackage[landscape, a4paper, mag=1400, truedimen, left=0.5cm, right=0.5cm, top=0.5cm, bottom=0.5cm]{geometry} % okraje stranky

\usepackage{fontspec}
\setmainfont[FeatureFile={junicode.fea}, Ligatures={Common, TeX}, RawFeature=+fixi]{Junicode}
%\setmainfont{Junicode}

% shortcut for Junicode without ligatures (for the Czech texts)
\newfontfamily\nlfont[FeatureFile={junicode.fea}, Ligatures={Common, TeX}, RawFeature=+fixi]{Junicode}

% Hebrew font:
% http://scripts.sil.org/cms/scripts/page.php?site_id=nrsi&id=SILHebrUnic2
\newfontfamily\hebfont[Scale=1]{Ezra SIL}

\usepackage{multicol}
\usepackage{color}
\usepackage{lettrine}
\usepackage{fancyhdr}

% usual packages loading:
\usepackage{luatextra}
\usepackage{graphicx} % support the \includegraphics command and options
\usepackage{gregoriotex} % for gregorio score inclusion
\usepackage{gregoriosyms}
\usepackage{wrapfig} % figures wrapped by the text
\usepackage{parcolumns}
\usepackage[contents={},opacity=1,scale=1,color=black]{background}
\usepackage{tikzpagenodes}
\usepackage{calc}
\usepackage{longtable}
\usetikzlibrary{calc}

\setlength{\headheight}{14.5pt}

\input{conventuscommune.tex} % Often used macros
%%%% Preklady jednotlivych zpevu (nektere se opakuji, a je dobre mit je
% vsechny na jedne hromade)

% HOURS ---

\newcommand{\trAntI}{\translatioCantus{Muž boží měl kožený toulec, pečlivě
zavázaný, jenž mu visel na šíji a~často se ho dotýkal.}}

\newcommand{\trAntII}{\translatioCantus{Klíč od~něho tak dobře střežil, že
dokud žil v~těle, nikdo z~jeho žáků nezvěděl, co je uvnitř.}}

\newcommand{\trAntIII}{\translatioCantus{Ale když se odebral z~tohoto
života, schránku otevřeli a~objevili v~ní žíněné roucho a~měděný řetěz
potřísněný krví.}}

\newcommand{\trAntIV}{\translatioCantus{A když prohlédli mistrovo tělo,
nalezli jeho tělo na čtyřech místech hluboce zbrázděno ranami od řetězu.}}

\newcommand{\trAntV}{\translatioCantus{Krev vytékající z~těch ran, místy
prostoupila i~žíněným rouchem.}}

\newcommand{\trCapituli}{\translatioCantus{
Miláčkovi Boha a~lidí,
Mojžíšovi požehnané paměti,~\gredagger{}
dopřál slávu rovnou slávě svatých~\grestar{}
učinil ho mocným na postrach nepřátelům
a~jeho slovy zastavil divy.}}

\newcommand{\trLectioBrevis}{\translatioCantus{
Pamatujte na své představené,
kteří vám hlásali Boží slovo.
Uvažte, jak oni skončili život, a~napodobujte jejich víru.
Ježíš Kristus je stejný včera i~dnes i~navěky.
Nenechte se svést věelijakými cizími naukami.}}

\newcommand{\trRespLaud}{\translatioCantus{Spravedlivého vodil Hospodin~\grestar{}
po přímých stezkách. \Vbardot{} A~ukázal mu Boží království.}}

\newcommand{\trRespLaudB}{\translatioCantus{Na tvých hradbách, Jeruzaléme,
ustanovil jsem strážné;~\grestar{}
budou bdít nad mým lidem. \Vbardot{} Ani ve dne, ani v~noci nesmějí nikdy
mlčet.}}

\newcommand{\trVersus}{\translatioCantus{\Vbardot{} Ústa spravedlivého šeptají moudrost, aleluja.
\Rbardot{} A~jeho jazyk ohlašuje právo, aleluja.}}

\newcommand{\trAntBenedictus}{\translatioCantus{Když na bujné oře vložili
nosítka a~sňali jim uzdu, vydali se přímo k~cele božího muže.}}

\newcommand{\trPreces}{\translatioCantus{
\noindent S vděčností chvalme Krista, dobrého Pastýře, \gredagger{} který dal život za své ovce, \grestar{} a~pokorně ho prosme: \Rbardot{} Pane, buď pastýřem svého lidu.

\noindent Kriste, ty dáváš církvi pastýře, a~jejich službou se ujímáš svého lidu, \grestar{} dej, ať v~lásce těch, kteří nás vedou, poznáváme, jak nás miluješ. \Rbardot{} Pane, buď pastýřem svého lidu.

\noindent Ty stále konáš skrze své zástupce službu pastýře a~učitele, \grestar{} nepřestávej nás nikdy vést prostřednictvím svých služebníků. \Rbardot{} Pane, buď pastýřem svého lidu.

\noindent Ty prokazuješ svému lidu skrze jeho pastýře službu lékaře duše i~těla, \grestar{} ochraňuj náš život a~veď nás ke svatosti. \Rbardot{} Pane, buď pastýřem svého lidu.

\noindent Ty posíláš své svaté, aby slovem i~příkladem vedli tvůj lid k~tobě, \grestar{} na jejich přímluvu nás posiluj, abychom vytrvali na cestě, která vede k~věčnému životu. \Rbardot{} Pane, buď pastýřem svého lidu.}}

\newcommand{\trOrationis}{\translatioCantus{Bože, jenž nám dopřáváš radovat
se z~výroční slavnosti svatého tvého vyznavače Havla, uděl dobrotivě,
abychom když slavíme jeho narození, též se řídili podobou jeho skutků.
Skrze…}}
 % Czech translations of the proper texts

\newcommand{\annusEditionis}{2020}

\def\hebinitial#1{%
\leavevmode{\newbox\hebbox\setbox\hebbox\hbox{\hebfont{#1}\hskip 1mm}\kern -\wd\hebbox\hbox{\hebfont{#1}\hskip 1mm}}%
}

%%%% Vicekrat opakovane kousky

\newcommand{\anteOrationem}{
  \rubrica{Ante Orationem, cantatur a Superiore:}

  \pars{Supplicatio Litaniæ.}

  \cuminitiali{}{temporalia/supplicatiolitaniae.gtex}

  \pars{Oratio Dominica.}

  \cuminitiali{}{temporalia/oratiodominica.gtex}

  \rubrica{Deinde dicitur ab Hebdomadario:}

  \cuminitiali{}{temporalia/dominusvobiscum-solemnis.gtex}

  \rubrica{In choro monialium loco Dominus vobiscum dicitur:}

  \sineinitiali{temporalia/domineexaudi.gtex}
}

\setlength{\columnsep}{30pt} % prostor mezi sloupci

%%%%%%%%%%%%%%%%%%%%%%%%%%%%%%%%%%%%%%%%%%%%%%%%%%%%%%%%%%%%%%%%%%%%%%%%%%%%%%%%%%%%%%%%%%%%%%%%%%%%%%%%%%%%%
\begin{document}

% Here we set the space around the initial.
% Please report to http://home.gna.org/gregorio/gregoriotex/details for more details and options
\grechangedim{afterinitialshift}{2.2mm}{scalable}
\grechangedim{beforeinitialshift}{2.2mm}{scalable}

\grechangedim{interwordspacetext}{0.32 cm plus 0.15 cm minus 0.05 cm}{scalable}%
\grechangedim{annotationraise}{-0.2cm}{scalable}

% Here we set the initial font. Change 38 if you want a bigger initial.
% Emit the initials in red.
\grechangestyle{initial}{\color{red}\fontsize{38}{38}\selectfont}

\pagestyle{empty}

%%%% Titulni stranka
\begin{titulusOfficii}
\nomenFesti{\feria{}}
\celebratio{Duplex 1. classis.}
\end{titulusOfficii}

\pagebreak

% graphic
\renewcommand{\headrulewidth}{0pt} % no horiz. rule at the header
\fancyhf{}
\pagestyle{fancy}

\cantusSineNeumas

\hora{Ad Matutinum.}

\vspace{2mm}

\cuminitiali{}{temporalia/dominelabiamea.gtex}

\vspace{2mm}

\pars{Invitatorium.} \scriptura{Lc. 24, 34; Psalmus 94; \textbf{H232}}

\vspace{-6mm}

\antiphona{VI}{temporalia/inv-surrexitdominusvere.gtex}

\rubrica{Hymnus non dicatur.}

\vfill
\pagebreak

\pars{Psalmus 1.} \scriptura{\Abardot{} Ex. 3, 14; \textbf{H226}}

\vspace{-5mm}

\antiphona{I f}{temporalia/an-egosumquisum.gtex}

%\vspace{-5mm}

\scriptura{Ps. 1}

%\vspace{-2mm}

\initiumpsalmi{temporalia/ps1-initium-i-F-auto.gtex}

%\psalmusEtTranslatioT{temporalia/ps1-comb.tex}{10cm}
\input{temporalia/ps1.tex} \Abardot{}

\vfill
\pagebreak

\pars{Psalmus 2.} \scriptura{\Abardot{} Ps. 2, 8; \textbf{H226}}

\vspace{-2mm}

\antiphona{I f}{temporalia/an-postulavi.gtex}

\vspace{-2mm}

\scriptura{Ps. 2}

\initiumpsalmi{temporalia/ps2-initium-i-f-auto.gtex}

%\psalmusEtTranslatioT{temporalia/ps2-comb.tex}{10cm}
\input{temporalia/ps2.tex} \Abardot{}

\vfill
\pagebreak

\pars{Psalmus 3.} \scriptura{\Abardot{} Ps. 3, 6; \textbf{H226}}

%\vspace{-4mm}

\antiphona{VIII C}{temporalia/an-egodormivi.gtex}

%\vspace{-0.3cm}

\scriptura{Ps. 3}

\initiumpsalmi{temporalia/ps3-initium-viii-c-auto.gtex}

%\psalmusEtTranslatioT{temporalia/ps3-comb.tex}{10cm}
\input{temporalia/ps3.tex} \Abardot{}

\vfill
\pagebreak

\versusMatutinum

\noindent Pater noster.

\pars{Absolutio.}

\absolutio

\vfill
\pagebreak

\cuminitiali{}{temporalia/benedictio-solemn-evangelica.gtex}

\vspace{7mm}

\lectioi

\noindent \Vbardot{} Tu autem, Dómine, miserére nobis.
\noindent \Rbardot{} Deo grátias.

\vfill
\pagebreak

\responsoriumi

\vfill
\pagebreak

\cuminitiali{}{temporalia/benedictio-solemn-divinum.gtex}

\vspace{7mm}

\lectioii

\noindent \Vbardot{} Tu autem, Dómine, miserére nobis.
\noindent \Rbardot{} Deo grátias.

\vfill
\pagebreak

\responsoriumii

\vfill
\pagebreak

\cuminitiali{}{temporalia/benedictio-solemn-adsocietatem.gtex}

\vspace{7mm}

\pars{Lectio III.}

\noindent Et quóniam sermo huc noster evásit, considerémus qua grátia secúndum Ioánnem credíderint Apóstoli, qui gavísi sunt; secúndum Lucam quasi incréduli redarguántur: ibi Spíritum Sanctum accéperint, hic sedére in civitáte iubeántur, quoadúsque induántur virtúte ex alto. Et vidétur mihi ille quasi Apóstolus maióra et altióra tetigísse, hic sequéntia et humánis próxima: hic histórico usus circúitu, ille compéndio: quia et de illo dubitári non potest, qui testimónium pérhibet de iis, quibus ipse intérfuit, et verum est testimónium eius: et ab hoc quoque, qui Evangelísta esse méruit, vel negligéntiæ, vel mendácii suspiciónem æquum est propulsári. Et ídeo verum putámus utrúmque, non sententiárum varietáte, nec personárum diversitáte distínctum. Nam etsi primo Lucas eos non credidísse dicat, póstea tamen credidísse demónstrat: et si prima considerémus, contrária sunt: si sequéntia, certum est conveníre.

\noindent \Vbardot{} Tu autem, Dómine, miserére nobis.
\noindent \Rbardot{} Deo grátias.

\vfill
\pagebreak

% Te Deum

%\pars{Hymnus Ambrosianus}

\vspace{-5mm}

{
\grechangedim{interwordspacetext}{0.22 cm plus 0.15 cm minus 0.05 cm}{scalable}%
\cuminitiali{III}{temporalia/tedeum-solemnis.gtex}
\grechangedim{interwordspacetext}{0.32 cm plus 0.15 cm minus 0.05 cm}{scalable}%
}

\vfill
\pagebreak

\rubrica{Reliqua omittuntur, nisi Laudes separandæ sint.}

\pars{Oratio}

\noindent \Vbardot{} Dómine, exáudi oratiónem meam.

\noindent \Rbardot{} Et clamor meus ad te véniat.

Orémus:

\oratioMatutinum

\noindent \Rbardot{} Amen.

\vspace{7mm}

\pars{Conclusio}

\noindent \Vbardot{} Dómine, exáudi oratiónem meam.

\noindent \Rbardot{} Et clamor meus ad te véniat.

\noindent \Vbardot{} Benedicámus Dómino, allelúia, allelúia.

\noindent \Rbardot{} Deo grátias, allelúia, allelúia.

\noindent \Vbardot{} Fidélium ánimæ per misericórdiam Dei requiéscant in pace.

\noindent \Rbardot{} Amen.

\vfill
\pagebreak

\hora{Ad Laudes.} %%%%%%%%%%%%%%%%%%%%%%%%%%%%%%%%%%%%%%%%%%%%%%%%%%%%%
%\sideThumbs{Laudes}

\cantusSineNeumas

\vspace{0.5cm}
\grechangedim{interwordspacetext}{0.18 cm plus 0.15 cm minus 0.05 cm}{scalable}%
\cuminitiali{}{temporalia/deusinadiutorium-alter.gtex}
\grechangedim{interwordspacetext}{0.22 cm plus 0.15 cm minus 0.05 cm}{scalable}%

\vfill
%\pagebreak

\pars{Psalmus 1.} \scriptura{Mt. 28, 2; \textbf{H230}}

\vspace{-0.4cm}

\antiphona{VIII G}{temporalia/ant-angelusautemdomini.gtex}

\scriptura{Psalmus 92.}

\initiumpsalmi{temporalia/ps92-initium-viii-G-auto.gtex}

%\psalmusEtTranslatioT{temporalia/ps92-comb.tex}{10cm}
\input{temporalia/ps92.tex}

\vfill

\antiphona{}{temporalia/ant-angelusautemdomini.gtex}

\vspace{-1cm}

\vfill
\pagebreak

\pars{Psalmus 2.} \scriptura{Mt. 28, 2; \textbf{H226}}

\vspace{-0.4cm}

\antiphona{VII c}{temporalia/ant-etecceterraemotus.gtex}

\scriptura{Psalmus 99.}

\initiumpsalmi{temporalia/ps99-initium-vii-c-auto.gtex}

%\psalmusEtTranslatioT{temporalia/ps99-comb.tex}{10cm}
\input{temporalia/ps99.tex} \Abardot{}

\vfill
\pagebreak

\pars{Psalmus 3.} \scriptura{Mt. 28, 3; \textbf{H226}}

\vspace{-0.4cm}

\antiphona{VIII c}{temporalia/ant-eratautem.gtex}

\scriptura{Psalmus 62.}

\initiumpsalmi{temporalia/ps62-initium-viii-C-auto.gtex}

%\psalmusEtTranslatioT{temporalia/ps62-comb.tex}{10cm}
\input{temporalia/ps62.tex} \Abardot{}

\vfill
\pagebreak

\pars{Psalmus 4.} \scriptura{Mt. 28, 4; \textbf{H226}}

\vspace{-0.4cm}

\antiphona{VII a}{temporalia/ant-praetimoreautem.gtex}

\scriptura{Canticum trium puerorum, Dan. 3, 57-88 et 56}

\initiumpsalmi{temporalia/dan3-initium-vii-a-auto.gtex}

%\psalmusEtTranslatioT{temporalia/dan3-comb.tex}{10cm}
\input{temporalia/dan3.tex}

\rubrica{Hic non dicitur Gloria Patri, neque Amen.}

\vfill

\vspace{-6mm}

\antiphona{}{temporalia/ant-praetimoreautem.gtex} % repeat the antiphon - new page

\vfill
\pagebreak

\pars{Psalmus 5.} \scriptura{Mt. 28, 5; \textbf{H230}}

\vspace{-7mm}

\antiphona{VIII G}{temporalia/ant-respondensautemangelus.gtex}

\scriptura{Psalmus 148.}

%\vspace{-4mm}

\initiumpsalmi{temporalia/ps148-initium-viii-g-auto.gtex}

%\psalmusEtTranslatioT{temporalia/ps148-comb.tex}{10cm}
\input{temporalia/ps148.tex}

\rubrica{Hic non dicitur Gloria Patri.}

\vfill
\pagebreak

%
\scriptura{Psalmus 149.}

\initiumpsalmi{temporalia/ps149-initium-viii-g-auto.gtex}

%\psalmusEtTranslatioT{temporalia/ps149-comb.tex}{10cm}
\input{temporalia/ps149.tex}

\rubrica{Hic non dicitur Gloria Patri.}

\vfill
\pagebreak

%
\scriptura{Psalmus 150.}

\initiumpsalmi{temporalia/ps150-initium-viii-g-auto.gtex}

%\psalmusEtTranslatioT{temporalia/ps150-comb.tex}{10cm}
\input{temporalia/ps150.tex}

\vfill

\vspace{-6mm}

\antiphona{}{temporalia/ant-respondensautemangelus.gtex} % repeat the antiphon - new page

\vfill
\pagebreak

\newcommand{\capitulumLaudes}{\pars{Capitulum.} \scriptura{1 Cor. 5, 7}

\grechangedim{interwordspacetext}{0.12 cm plus 0.15 cm minus 0.05 cm}{scalable}%
\cuminitiali{}{temporalia/capitulum-FratresExpurgate.gtex}
\grechangedim{interwordspacetext}{0.22 cm plus 0.15 cm minus 0.05 cm}{scalable}}
\capitulumLaudes

% preklad Jeruz. bible
%\trCapituliI

\vfill

\pars{Responsorium breve.} \scriptura{Cf. Mt. 28, 6; Cf. Gal. 3, 13}

\cuminitiali{VI}{temporalia/respbr-laud.gtex}

%\trResp

\vfill
\pagebreak

\pars{Hymnus}

\cuminitiali{VIII}{temporalia/hym-AuroraLucis.gtex}
\vspace{-3mm}
%\input{hym-AuroraLucis-bohtext.tex}

\vfill
%\pagebreak

\pars{Versus.} \scriptura{Ps. 117, 24}

% Versus. %%%
\sineinitiali{temporalia/versus-haecdies.gtex}

%\noindent \trVersus

\vfill
\pagebreak

\benedictus

\vspace{-1cm}

\vfill
\pagebreak

%\sideThumbs{{\scriptsize{}Fine horarum}}

\anteOrationem

\pagebreak

% Oratio. %%%
\oratioLaudes

\vspace{-1mm}
%\trOrationisI

\vfill

\rubrica{Hebdomadarius dicit iterum Dominus vobiscum. Postea cantatur a cantore:}
\vspace{2mm}

\cuminitiali{}{temporalia/benedicamus-octava-paschae.gtex}

\vspace{1mm}

\ifx\magnificat\undefined
\else
\vfill
\pagebreak

\hora{Ad Vesperas.} %%%%%%%%%%%%%%%%%%%%%%%%%%%%%%%%%%%%%%%%%%%%%%%%%%%%%
%\sideThumbs{Vesperæ}

\cantusSineNeumas

\vspace{0.5cm}
\grechangedim{interwordspacetext}{0.18 cm plus 0.15 cm minus 0.05 cm}{scalable}%
\cuminitiali{}{temporalia/deusinadiutorium-solemnis.gtex}
\grechangedim{interwordspacetext}{0.22 cm plus 0.15 cm minus 0.05 cm}{scalable}%

\vfill
%\pagebreak

\vspace{-2mm}

\pars{Psalmus 1.} \scriptura{Mt. 28, 2; \textbf{H230}}

\vspace{-0.4cm}

\antiphona{VIII G}{temporalia/ant-angelusautemdomini.gtex}

\scriptura{Psalmus 109.}

\initiumpsalmi{temporalia/ps109-initium-viii-G-auto.gtex}

%\psalmusEtTranslatioT{temporalia/ps109-comb.tex}{10cm}
\input{temporalia/ps109.tex}

\antiphona{}{temporalia/ant-angelusautemdomini.gtex}

\vspace{-1cm}

\vfill
\pagebreak

\pars{Psalmus 2.} \scriptura{Mt. 28, 2; \textbf{H226}}

\vspace{-0.4cm}

\antiphona{VII c}{temporalia/ant-etecceterraemotus.gtex}

\scriptura{Psalmus 110.}

\initiumpsalmi{temporalia/ps110-initium-vii-c-auto.gtex}

%\psalmusEtTranslatioT{temporalia/ps110-comb.tex}{10cm}
\input{temporalia/ps110.tex} \Abardot{}

\vfill
\pagebreak

\pars{Psalmus 3.} \scriptura{Mt. 28, 3; \textbf{H226}}

\vspace{-0.4cm}

\antiphona{VIII c}{temporalia/ant-eratautem.gtex}

\scriptura{Psalmus 111.}

\initiumpsalmi{temporalia/ps111-initium-viii-C-auto.gtex}

%\psalmusEtTranslatioT{temporalia/ps111-comb.tex}{10cm}
\input{temporalia/ps111.tex} \Abardot{}

\vfill
\pagebreak

\pars{Psalmus 4.} \scriptura{Mt. 28, 5; \textbf{H230}}

\vspace{-0.4cm}

\antiphona{VIII G}{temporalia/ant-respondensautemangelus.gtex}

\scriptura{Psalmus 112.}

\initiumpsalmi{temporalia/ps112-initium-viii-g-auto.gtex}

%\psalmusEtTranslatioT{temporalia/ps112-comb.tex}{10cm}
\input{temporalia/ps112.tex} \Abardot{}

\vfill
\pagebreak

\capitulumLaudes

% preklad Jeruz. bible
%\trCapituliI

\vfill

\pars{Responsorium breve.} \scriptura{Lc. 24, 34}

\cuminitiali{VI}{temporalia/respbr-vesp.gtex}

%\trResp

\vfill
\pagebreak

\pars{Hymnus}

\cuminitiali{VIII}{temporalia/hym-AdCoenam.gtex}
\vspace{-3mm}
%\begin{translatioMulticol}{4}
U~Beránkovy hostiny\\
oděni rouchy bílými,\\
když Rudým mořem prošli jsme,\\
Vladaři Kristu zpívejme.\\
\\
Když jeho tělem posvátným,\\
na kříži obětovaným,\\
se sytíme a~pijeme\\
jeho krev, v~Bohu žijeme.\columnbreak

Chráněni tímto pokrmem\\
před smrtonosným andělem,\\
svrhli jsme z~beder kruté jho\\
tyrana bezohledného.\\
\\
Kristus je naší paschou teď,\\
on sám se vydal za oběť\\
a~místo přesnic našim rtům\\
své tělo dává za pokrm.\columnbreak

Tys, nejčistější Oběti,\\
zlomila vládu podsvětí.\\
Z~otroctví lid je vykoupen,\\
odměna žití kyne všem.\\
\\
Hle, Kristus, když vstal ze hrobu,\\
jde z~pekel v~slavném průvodu\\
a~brány nebes otevřev,\\
vládce tmy vleče v~okovech.\columnbreak

Buď věčně, Kriste, věrným svým\\
plesáním velikonočním.\\
Nás, milostí tvou vzkříšené,\\
vem k~oslavě své vítězné. \\
\\
Sláva tobě, Pane,\\
jenž jsi vstal z~mrtvých,\\
s~Otcem i~Svatým Duchem\\
na věčné věky.\\
Amen.
\end{translatioMulticol}


\vfill
\pagebreak

\pars{Versus.} \scriptura{Ps. 117, 24}

% Versus. %%%
\sineinitiali{temporalia/versus-haecdies.gtex}

%\noindent \trVersus

\vfill
\pagebreak

\magnificat

\vspace{-1cm}

\vfill
\pagebreak

%\sideThumbs{{\scriptsize{}Fine horarum}}

\anteOrationem

\pagebreak

% Oratio. %%%
\oratioLaudes

\vspace{-1mm}
%\trOrationisI

\vfill

\rubrica{Hebdomadarius dicit iterum Dominus vobiscum. Postea cantatur a cantore:}
\vspace{2mm}

\cuminitiali{}{temporalia/benedicamus-octava-paschae.gtex}

\vspace{1mm}
\fi

\end{document}

