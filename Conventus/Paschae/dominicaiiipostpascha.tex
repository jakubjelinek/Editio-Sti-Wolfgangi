\newcommand{\titulus}{\nomenFesti{Dominica IV Paschæ (III post Pascha).}}
\newcommand{\tedeummonasticum}{Monasticum}
\newcommand{\oratioLaudes}{\cuminitiali{}{temporalia/oratiod4.gtex}}
\newcommand{\oratio}{\pars{Oratio.}

\noindent Omnípotens sempitérne Deus, deduc nos ad societátem cæléstium gaudiórum, ut eo pervéniat humílitas gregis, quo procéssit fortitúdo pastóris.

\pars{Pro pace in Ucraina.} \scriptura{Sir. 50, 25; 2 Esdr. 4, 20; \textbf{H416}}

\vspace{-4mm}

\antiphona{II D}{temporalia/ant-dapacemdomine.gtex}

\vfill

\noindent Deus, a quo sancta desidéria, recta consília et iusta sunt ópera: da servis tuis illam, quam mundus dare non potest, pacem; ut et corda nostra mandátis tuis dédita, et hóstium subláta formídine, témpora sint tua protectióne tranquílla.

\noindent Per Dóminum nostrum Iesum Christum, Fílium tuum, qui tecum vivit et regnat in unitáte Spíritus Sancti, Deus, per ómnia sǽcula sæculórum.

\noindent \Rbardot{} Amen.}
\newcommand{\invitatorium}{\pars{Invitatorium.} \scriptura{Lc. 24, 34; Psalmus 94; \textbf{H232}}

\vspace{-6mm}

\antiphona{VI}{temporalia/inv-surrexitdominusvere.gtex}}
\newcommand{\hymnusmatutinum}{\pars{Hymnus.}

\vspace{-5mm}

\antiphona{III}{temporalia/hym-HicEstDies.gtex}}
\newcommand{\nocturnoi}{\pars{Psalmus 1.}

\vspace{-4mm}

\antiphona{II D}{temporalia/ant-alleluia-bv21-n1.gtex}

%\vspace{-2mm}

\scriptura{Ps. 1}

%\vspace{-2mm}

%\initiumpsalmi{temporalia/ps1-initium-ii-D-auto.gtex}
\initiumpsalmi{temporalia/ps1-initium-ii-D.gtex}

%\input{temporalia/ps1-ii-D.tex} \Abardot{}
\input{temporalia/ps1-viii-G.tex} \Abardot{}

\vfill
\pagebreak

\pars{Psalmus 2.}

\vspace{-4mm}

\antiphona{VIII c}{temporalia/ant-alleluia-bv21-n2.gtex}

%\vspace{-2mm}

\scriptura{Ps. 2}

\initiumpsalmi{temporalia/ps2-initium-viii-c-auto.gtex}

\input{temporalia/ps2-viii-c.tex} \Abardot{}

\vfill
\pagebreak

\pars{Psalmus 3.}

%\vspace{-4mm}

\antiphona{VIII G\textsuperscript{3}}{temporalia/ant-alleluia-bv21-n3.gtex}

%\vspace{-2mm}

\scriptura{Ps. 3}

%\initiumpsalmi{temporalia/ps3-initium-viii-G2-auto.gtex}
\initiumpsalmi{temporalia/ps3-initium-viii-G3.gtex}

\input{temporalia/ps3-viii-G2.tex} \Abardot{}

\vfill
\pagebreak}
\newcommand{\nocturnoii}{\vspace{-4mm}

\pars{Psalmus 4.} \scriptura{Mt. 28, 2; Mc. 16, 4; \textbf{H230}}

\vspace{-4mm}

\antiphona{VII a}{temporalia/ant-alleluialapisrevolutus.gtex}

%\vspace{-2mm}

\scriptura{Ps. 23}

%\vspace{-2mm}

\initiumpsalmi{temporalia/ps23-initium-vii-a-auto.gtex}

\input{temporalia/ps23-vii-a.tex} \Abardot{}

\vfill
\pagebreak

\pars{Psalmus 5.} \scriptura{Io. 20, 15.17}

\vspace{-4mm}

\antiphona{VIII G\textsuperscript{2}}{temporalia/ant-marianoliiamflere.gtex}

%\vspace{-2mm}

\scriptura{Ps. 65, 1-12}

\initiumpsalmi{temporalia/ps65i-initium-viii-G5-auto.gtex}

\input{temporalia/ps65i-viii-G5.tex} \Abardot{}

\vfill
\pagebreak

\pars{Psalmus 6.} \scriptura{Io. 20, 18; \textbf{H238}}

\vspace{-4mm}

\antiphona{VII a}{temporalia/ant-venitmarianuntians.gtex}

%\vspace{-4mm}

\scriptura{Ps. 65, 13-20}

%\vspace{-2mm}

\initiumpsalmi{temporalia/ps65ii-initium-vii-a-auto.gtex}

%\vspace{-1.5mm}

\input{temporalia/ps65ii-vii-a.tex} \Abardot{}

\vfill
\pagebreak}
\newcommand{\nocturnoiii}{\pars{Cantica.}

\vspace{-4mm}

\antiphona{IV E}{temporalia/ant-veniteomnesadoremus.gtex}

%\vspace{-2mm}

\scriptura{Canticum Isaiæ, Is. 63, 1-5}

%\vspace{-2mm}

\initiumpsalmi{temporalia/isaiae12-initium-iv-E-auto.gtex}

\input{temporalia/isaiae12-iv-E.tex} \hfill \rubrica{Hic non dicitur antiphona.}

\vfill
\pagebreak

\scriptura{Canticum Oseæ, Os. 6, 1-6}

%\vspace{-2mm}

\initiumpsalmi{temporalia/oseae-initium-iv-E-auto.gtex}

\input{temporalia/oseae-iv-E.tex}

\vfill
\pagebreak

\scriptura{Canticum Sophoniæ, Soph. 3, 8-13}

%\vspace{-2mm}

\initiumpsalmi{temporalia/sophoniae-initium-iv-E-auto.gtex}

\input{temporalia/sophoniae-iv-E.tex}

\vfill
\pagebreak

\antiphona{}{temporalia/ant-veniteomnesadoremus.gtex}

\vfill
\pagebreak}
\newcommand{\matversusi}{\pars{Versus.}

\noindent \Vbardot{} Surréxit Dóminus de sepúlcro, allelúia.

\noindent \Rbardot{} Qui pro nobis pepéndit in ligno, allelúia.}
\newcommand{\matversusii}{\pars{Versus.}

\noindent \Vbardot{} Surréxit Dóminus vere, allelúia.

\noindent \Rbardot{} Et appáruit Simóni, allelúia.}
\newcommand{\lectioi}{\pars{Lectio I.} \scriptura{Ap. 12, 1-6}

\noindent De libro Apocalýpsis beáti Ioánnis apóstoli.

\noindent Signum magnum appáruit in cælo: múlier amícta sole, et luna sub pédibus eius, et super caput eius coróna stellárum duódecim; et in útero habens, et clamat partúriens et cruciáta, ut páriat.

\noindent Et visum est áliud signum in cælo: et ecce draco rufus magnus, habens cápita septem et córnua decem, et super cápita sua septem diadémata; et cauda eius trahit tértiam partem stellárum cæli et misit eas in terram.

\noindent Et draco stetit ante mulíerem, quæ erat paritúra, ut, cum peperísset, fílium eius devoráret.

\noindent Et péperit fílium, másculum, qui rectúrus est omnes gentes in virga férrea; et raptus est fílius eius ad Deum et ad thronum eius.

\noindent Et múlier fugit in desértum, ubi habet locum parátum a Deo, ut ibi pascant illam diébus mille ducéntis sexagínta.}
\newcommand{\responsoriumi}{\pars{Responsorium 1.} \scriptura{\Rbardot{} Sap. 18, 24 \Vbardot{} Sir. 45, 14; \textbf{H249}}

\vspace{-5mm}

\responsorium{III}{temporalia/resp-indiademate-CROCHU.gtex}{}}
\newcommand{\lectioii}{\pars{Lectio II.} \scriptura{Ap. 12, 7-12}

\noindent Et factum est prœ́lium in cælo, Míchael et ángeli eius, ut prœliaréntur cum dracóne.

\noindent Et draco pugnávit et ángeli eius, et non váluit, neque locus invéntus est eórum ámplius in cælo. Et proiéctus est draco ille magnus, serpens antíquus, qui vocátur Diábolus et Sátanas, qui sedúcit univérsum orbem; proiéctus est in terram, et ángeli eius cum illo proiécti sunt.

\noindent Et audívi vocem magnam in cælo dicéntem:

\noindent «Nunc facta est salus et virtus et regnum Dei nostri

\noindent et potéstas Christi eius,

\noindent quia proiéctus est accusátor fratrum nostrórum,

\noindent qui accusábat illos ante conspéctum Dei nostri die ac nocte.

\noindent Et ipsi vicérunt illum propter sánguinem Agni

\noindent et propter verbum testimónii sui;

\noindent et non dilexérunt ánimam suam

\noindent usque ad mortem.

\noindent Proptérea lætámini, cæli

\noindent et qui habitátis in eis.

\noindent Væ terræ et mari, quia descéndit Diábolus ad vos habens iram magnam, sciens quod módicum tempus habet!».}
\newcommand{\responsoriumii}{\pars{Responsorium 2.} \scriptura{\Rbardot{} Cf. Ct. 4, 11; \textbf{H249}}

\vspace{-5mm}

\responsorium{III}{temporalia/resp-veniensalibano-CROCHU.gtex}{}}
\newcommand{\lectioiii}{\pars{Lectio III.} \scriptura{Ap. 1, 13-18}

\noindent Et postquam vidit draco quod proiéctus est in terram, persecútus est mulíerem, quæ péperit másculum.

\noindent Et datæ sunt mulíeri duæ alæ áquilæ magnæ, ut voláret in desértum in locum suum, ubi álitur per tempus et témpora et dimídium témporis a fácie serpéntis.

\noindent Et misit serpens ex ore suo post mulíerem aquam tamquam flumen, ut eam fáceret trahi a flúmine.

\noindent Et adiúvit terra mulíerem, et apéruit terra os suum et absórbuit flumen, quod misit draco de ore suo.

\noindent Et irátus est draco in mulíerem et ábiit fácere prœ́lium cum réliquis de sémine eius, qui custódiunt mandáta Dei et habent testimónium Iesu.

\noindent Et stetit super arénam maris.}
\newcommand{\responsoriumiii}{\pars{Responsorium 3.} \scriptura{\Rbardot{} Ap. 21, 2 \Vbardot{} Ps. 147, 2; \textbf{H249}}

\vspace{-5mm}

\responsorium{I}{temporalia/resp-haecestierusalem-CROCHU-cumdox.gtex}{}}
\newcommand{\lectioiv}{\pars{Lectio IV.} \scriptura{Hom. 14, 3-6: PL 76, 1129-1130}

\noindent Ex Homíliis sancti Gregórii Magni papæ in Evangélia.

\noindent \emph{Ego sum pastor bonus. Et cognósco oves meas,} hoc est díligo, \emph{et cognóscunt me meæ.} Ac si paténter dicat: Diligéntes obsequúntur. Qui enim veritátem non díligit, adhuc mínime cognóvit.

\noindent Quia ergo audístis, fratres caríssimi, perículum nostrum, pensáte in verbis domínicis étiam perículum vestrum.

\noindent Vidéte si oves eius estis, vidéte si eum cognóscitis, vidéte si lumen veritátis scitis.

\noindent Scitis autem, dico, non per fidem, sed per amórem.

\noindent Scitis, dico, non ex credulitáte, sed ex operatióne.

\noindent Nam idem ipse qui hoc lóquitur, Ioánnes evangelísta testátur, dicens: \emph{Qui dicit se nosse Deum, et mandáta eius non custódit, mendax est.}

\noindent Unde et in hoc loco Dóminus prótinus subdit: \emph{Sicut novit me Pater, et ego agnósco Patrem, et ánimam meam pono pro óvibus meis.}

\noindent Ac si apérte dicat: In hoc constat quia et ego agnósco Patrem, et cognóscor a Patre, quia ánimam meam pono pro óvibus meis; id est, ea caritáte qua pro óvibus mórior, quantum Patrem díligam osténdo.}
\newcommand{\responsoriumiv}{\pars{Responsorium 4.} \scriptura{\Rbardot{} Ap. 21, 9-10; Is. 61, 10 \Vbardot{} Sir. 32, 18; \textbf{H247}}

\emph{}
\vspace{-5mm}

\responsorium{II}{temporalia/resp-locutusestadmeunus-CROCHU.gtex}{}}
\newcommand{\lectiov}{\pars{Lectio V.}

\noindent De quibus profécto óvibus rursum dicit: \emph{Oves meæ vocem meam áudiunt, et ego cognósco eas, et sequúntur me, et ego vitam ætérnam do eis.}

\noindent De quibus et paulo supérius dicit: \emph{Per me si quis introíerit, salvábitur et ingrediétur et egrediétur et páscua invéniet.}

\noindent Ingrediétur quippe ad fidem, egrediétur vero a fide ad spéciem, a credulitáte ad contemplatiónem, páscua autem invéniet in ætérna refectióne.

\noindent Oves ergo eius páscua invéniunt, quia quisquis illum corde símplici séquitur, ætérnæ viriditátis pábulo nutrítur.

\noindent Quæ autem sunt istárum óvium páscua, nisi intérna gáudia semper viréntis paradísi?

\noindent Páscua namque electórum sunt vultus præsens Dei, qui dum sine deféctu conspícitur, sine fine mens vitæ cibo satiátur.}
\newcommand{\responsoriumv}{\pars{Responsorium 5.} \scriptura{\Rbardot{} Ap. 14, 7; \Vbardot{} ibid., 6; \textbf{H248}}

\vspace{-5mm}

\responsorium{I}{temporalia/resp-audivivocemincaeloangelorum-CROCHU.gtex}{}}
\newcommand{\lectiovi}{\pars{Lectio VI.}

\noindent Quærámus ergo, fratres caríssimi, hæc páscua, in quibus cum tantórum cívium sollemnitáte gaudeámus. Ipsa nos lætántium festívitas invítet.

\noindent Accendámus ergo ánimum, fratres, recaléscat fides in id quod crédidit, inardéscant ad supérna nostra desidéria, et sic amáre iam ire est.

\noindent Ab intérnæ sollemnitátis gáudio nulla nos advérsitas révocet, quia, et si quis ad locum propósitum ire desíderat, eius desidérium quǽlibet viæ aspéritas non immútat.

\noindent Nulla nos prospéritas blándiens sedúcat, quia stultus viátor est, qui, in itínere amœ́na prata conspíciens, oblivíscitur ire quo tendébat.}
%\newcommand{\responsoriumvi}{\pars{Responsorium 6.} \scriptura{\Rbardot{} Cf. 1 Chr. 13, 8 \Vbardot{} Ps. 98, 6; \textbf{H248}}
%
%\vspace{-5mm}
%
%\responsorium{VI}{temporalia/resp-decantabatpopulus-CROCHU-cumdox.gtex}{}}
\newcommand{\responsoriumvi}{\pars{Responsorium 6.} \scriptura{\Rbardot{} Io. 10, 11; \textbf{H237}}

\vspace{-5mm}

\responsorium{I}{temporalia/resp-surrexitpastor-CROCHU-cumdox.gtex}{}}
\newcommand{\evangelium}{\pars{Versus.}

\noindent \Vbardot{} Gavísi sunt discípuli, allelúia.

\noindent \Rbardot{} Viso Dómino, allelúia.

\vspace{5mm}

\sineinitiali{temporalia/oratiodominica-mat.gtex}

\vspace{5mm}

\pars{Absolutio.}

\cuminitiali{}{temporalia/absolutio-avinculis.gtex}

\vfill
\pagebreak

\cuminitiali{}{temporalia/benedictio-solemn-evangelica.gtex}

\vspace{7mm}

\pars{Evangelium} \scriptura{Io. 10, 1-10}

\noindent Léctio sancti Evangélii secúndum Ioánnem.

\noindent In illo témpore: Dixit Iesus:

\noindent «Amen, amen dico vobis: Qui non intrat per óstium in ovíle óvium, sed ascéndit aliúnde, ille fur est et latro; qui autem intrat per óstium, pastor est óvium. Huic ostiárius áperit, et oves vocem eius áudiunt, et próprias oves vocat nominátim et edúcit eas. Cum próprias omnes emíserit, ante eas vadit, et oves illum sequúntur, quia sciunt vocem eius; aliénum autem non sequéntur, sed fúgient ab eo, quia non novérunt vocem alienórum».

\noindent Hoc provérbium dixit eis Iesus; illi autem non cognovérunt quid esset, quod loquebátur eis.

\noindent Dixit ergo íterum Iesus: «Amen, amen dico vobis: Ego sum óstium óvium. Omnes, quotquot venérunt ante me, fures sunt et latrónes, sed non audiérunt eos oves. Ego sum óstium; per me, si quis introíerit, salvábitur et ingrediétur et egrediétur et páscua invéniet. Fur non venit, nisi ut furétur et mactet et perdat; ego veni, ut vitam hábeant et abundántius hábeant».

\iffalse
\vspace{5mm}

\scriptura{Sermo 138,5-6: PL 38, 765-766}

\noindent Ex Sermónibus sancti Augustíni epíscopi.

\noindent {\color{gray} Quid est quod pastóribus bonis comméndas unum pastórem, nisi quia in uno pastóre doces unitátem? Et expónit apértius ipse Dóminus per ministérium nostrum, ex ipso evangélio commémorans caritátem vestram, et dicens: « Audíte quid commendávi: \emph{Ego sum pastor bonus,} dixi: quia omnes céteri, omnes pastóres boni membra mea sunt.» Unum caput, unum corpus, unus Christus. Ergo et pastor pastórum, et pastóres pastóris, et oves cum pastóribus sub pastóre.

\noindent Quid sunt hæc, nisi quod dicit Apóstolus: \emph{Sicut enim corpus unum est, et membra habet multa; ómnia autem membra córporis cum sint multa, unum est corpus: sic et Christus?} Ergo si sic et Christus, mérito Christus in se habens omnes pastóres bonos, unum comméndat dicens:« \emph{Ego sum pastor bonus.} Ego sum, unus sum, mecum omnes in unitáte unum sunt. Qui extra me pascit, contra me pascit. \emph{Qui mecum non cólligit, spargit.} »}

\noindent Ergo audíte ipsam unitátem veheméntius commendátam: \emph{Hábeo,} inquit, \emph{álias oves quæ non sunt de hoc ovíli.} Loquebátur enim primo ovíli de génere carnis Israel. Erant autem álii de génere fídei ipsíus Israel, et extra erant adhuc, in géntibus erant, prædestináti, nondum congregáti. Hos nóverat qui prædestináverat: nóverat qui redímere sánguine suo fuso vénerat. Vidébat eos, nondum vidéntes eum: nóverat eos, nondum credéntes in eum.

\noindent \emph{Hábeo,} inquit, \emph{álias oves quæ non sunt de hoc ovíli:} quia non sunt de génere carnis Israel. Sed tamen non erunt extra hoc ovíle, quia \emph{opórtet me eas addúcere, ut sit unus grex et unus pastor.} Mérito huic pastóri pastórum, amáta eius, sponsa eius, pulchra eius, sed ab ipso pulchra facta, prius peccátis fœda, post indulgéntia et grátia formósa, lóquitur amans et ardens in eum et dicit ei: \emph{Ubi pascis?}

\noindent Et vidéte quemádmodum, quo afféctu hic erigátur amor spiritális. Mélius multo isto afféctu delectántur, qui áliquid ex huius amóris dulcédine gustavérunt. Illi hoc bene áudiunt, qui amant Christum.

\vfill
\pagebreak

\pars{Responsorium 7.} \scriptura{\Rbardot{} Io. 10, 11; \textbf{H237}}

\vspace{-5mm}

\responsorium{I}{temporalia/resp-surrexitpastor-CROCHU-cumdox.gtex}{}
\fi

\vfill
\pagebreak}
\newcommand{\laudes}{\pars{Hymnus}

\cuminitiali{VIII}{temporalia/hym-AuroraLucis-alt.gtex}

\vfill
\pagebreak

\pars{Psalmus 1.} \scriptura{Ps. 117, 17}

\vspace{-4mm}

\antiphona{VIII c}{temporalia/ant-nonmoriarsedvivam.gtex}

%\vspace{-2mm}

\scriptura{Psalmus 117}

%\vspace{-2mm}

\initiumpsalmi{temporalia/ps117-initium-viii-c-auto.gtex}

%\vspace{-1.5mm}

\input{temporalia/ps117-viii-c.tex}

\vfill

\antiphona{}{temporalia/ant-nonmoriarsedvivam.gtex}

\vfill
\pagebreak

\pars{Psalmus 2.} \scriptura{Dn. 3, 52}

\vspace{-4mm}

\antiphona{VIII c\textsuperscript{2}}{temporalia/ant-benedictumnomengloriae.gtex}

%\vspace{-2mm}

\scriptura{Canticum Danielis, Dan. 3, 52-57}

%\vspace{-3mm}

\initiumpsalmi{temporalia/dan33-initium-viii-c2-auto.gtex}

\input{temporalia/dan33-viii-c2.tex} \Abardot{}

\vfill
\pagebreak

\pars{Psalmus 3.} \scriptura{Ps. 148, 5}

\vspace{-4mm}

\antiphona{VIII G}{temporalia/ant-redemptiadomino.gtex}

\scriptura{Psalmus 150}

\initiumpsalmi{temporalia/ps150-initium-viii-G-auto.gtex}

\input{temporalia/ps150-viii-G.tex} \Abardot{}

\vfill
\pagebreak}
\newcommand{\lectiobrevis}{\pars{Lectio brevis.} \scriptura{Ac. 10, 40-43}

\noindent Deus suscitávit Iesum tértia die et dedit eum maniféstum fíeri, non omni pópulo sed téstibus præordinátis a Deo, nobis, qui manducávimus et bíbimus cum illo postquam resurréxit a mórtuis; et præcépit nobis prædicáre pópulo et testificári quia ipse est, qui constitútus est a Deo iudex vivórum et mortuórum. Huic omnes prophétæ testimónium pérhibent remissiónem peccatórum accípere per nomen eius omnes, qui credunt in eum.}
\newcommand{\responsoriumbreve}{\pars{Responsorium breve.}

\cuminitiali{VI}{temporalia/resp-christefilideivivi-tp.gtex}}
\newcommand{\benedictus}{\pars{Canticum Zachariæ.} \scriptura{Io. 10, 1-2; \textbf{H203}}

\vspace{-4mm}

\antiphona{III b}{temporalia/ant-amenamendicovobisquinon.gtex}

%\vspace{-2mm}

\scriptura{Lc. 1, 68-79}

%\vspace{-2mm}

\cantusSineNeumas
\initiumpsalmi{temporalia/benedictus-initium-iiisoll-b-auto.gtex}

%\vspace{-1.5mm}

\input{temporalia/benedictus-iiisoll-b.tex}

\vfill

\antiphona{}{temporalia/ant-amenamendicovobisquinon.gtex}

\vspace{-10mm}}
\newcommand{\preces}{\noindent Deum Patrem omnipoténtem,~\gredagger{} qui Iesum, príncipem et salvatórem nostrum, suscitávit,~\grestar{} invocémus clamántes:

\Rbardot{} Claritáte Christi clarífica nos, Dómine.

\noindent Pater sancte,~\gredagger{} qui Iesum, diléctum tuum, de ténebris mortis ad lumen glóriæ tuæ transíre fecísti,~\grestar{} da nobis in admirábile lumen tuum veníre.

\Rbardot{} Claritáte Christi clarífica nos, Dómine.

\noindent Qui nos salvásti per fidem,~\grestar{} in fide baptísmatis nostri fac ut hódie vivámus.

\Rbardot{} Claritáte Christi clarífica nos, Dómine.

\noindent Tu, qui mandas ut quæ sursum sunt quærámus,~\gredagger{} ubi Christus est in déxtera tua sedens,~\grestar{} serva nos a peccáti blandítiis.

\Rbardot{} Claritáte Christi clarífica nos, Dómine.

\noindent Vita nostra, in te abscóndita cum Christo, lúceat in mundo,~\grestar{} ut cælum novum et terra nova prænuntiéntur.

\Rbardot{} Claritáte Christi clarífica nos, Dómine.}
\newcommand{\magnificatii}{\pars{Canticum B. Mariæ V.} \scriptura{Io. 1, 41}

\vspace{-6mm}

{
\grechangedim{interwordspacetext}{0.18 cm plus 0.15 cm minus 0.05 cm}{scalable}%
\antiphona{I d\textsuperscript{3}}{temporalia/ant-ambulansiesus.gtex}
\grechangedim{interwordspacetext}{0.22 cm plus 0.15 cm minus 0.05 cm}{scalable}%
}

\vspace{-1.5mm}

\scriptura{Lc. 1, 46-55}

\vspace{-2.5mm}

\cantusSineNeumas
\initiumpsalmi{temporalia/magnificat-initium-isoll-d3.gtex}

\vspace{-1.5mm}

\input{temporalia/magnificat-isoll-d3.tex} \Abardot{}}
%\newcommand{\hebdomada}{infra Hebdom. IV post Pentecosten.}
\newcommand{\oratioLaudes}{\cuminitiali{}{temporalia/oratio4.gtex}}

% LuaLaTeX

\documentclass[a4paper, twoside, 12pt]{article}
\usepackage[latin]{babel}
%\usepackage[landscape, left=3cm, right=1.5cm, top=2cm, bottom=1cm]{geometry} % okraje stranky
%\usepackage[landscape, a4paper, mag=1166, truedimen, left=2cm, right=1.5cm, top=1.6cm, bottom=0.95cm]{geometry} % okraje stranky
\usepackage[landscape, a4paper, mag=1400, truedimen, left=0.5cm, right=0.5cm, top=0.5cm, bottom=0.5cm]{geometry} % okraje stranky

\usepackage{fontspec}
\setmainfont[FeatureFile={junicode.fea}, Ligatures={Common, TeX}, RawFeature=+fixi]{Junicode}
%\setmainfont{Junicode}

% shortcut for Junicode without ligatures (for the Czech texts)
\newfontfamily\nlfont[FeatureFile={junicode.fea}, Ligatures={Common, TeX}, RawFeature=+fixi]{Junicode}

\usepackage{multicol}
\usepackage{color}
\usepackage{lettrine}
\usepackage{fancyhdr}

% usual packages loading:
\usepackage{luatextra}
\usepackage{graphicx} % support the \includegraphics command and options
\usepackage{gregoriotex} % for gregorio score inclusion
\usepackage{gregoriosyms}
\usepackage{wrapfig} % figures wrapped by the text
\usepackage{parcolumns}
\usepackage[contents={},opacity=1,scale=1,color=black]{background}
\usepackage{tikzpagenodes}
\usepackage{calc}
\usepackage{longtable}
\usetikzlibrary{calc}

\setlength{\headheight}{14.5pt}

\input{conventuscommune.tex} % Often used macros
%%%% Preklady jednotlivych zpevu (nektere se opakuji, a je dobre mit je
% vsechny na jedne hromade)

% HOURS ---

\newcommand{\trAntI}{\translatioCantus{Muž boží měl kožený toulec, pečlivě
zavázaný, jenž mu visel na šíji a~často se ho dotýkal.}}

\newcommand{\trAntII}{\translatioCantus{Klíč od~něho tak dobře střežil, že
dokud žil v~těle, nikdo z~jeho žáků nezvěděl, co je uvnitř.}}

\newcommand{\trAntIII}{\translatioCantus{Ale když se odebral z~tohoto
života, schránku otevřeli a~objevili v~ní žíněné roucho a~měděný řetěz
potřísněný krví.}}

\newcommand{\trAntIV}{\translatioCantus{A když prohlédli mistrovo tělo,
nalezli jeho tělo na čtyřech místech hluboce zbrázděno ranami od řetězu.}}

\newcommand{\trAntV}{\translatioCantus{Krev vytékající z~těch ran, místy
prostoupila i~žíněným rouchem.}}

\newcommand{\trCapituli}{\translatioCantus{
Miláčkovi Boha a~lidí,
Mojžíšovi požehnané paměti,~\gredagger{}
dopřál slávu rovnou slávě svatých~\grestar{}
učinil ho mocným na postrach nepřátelům
a~jeho slovy zastavil divy.}}

\newcommand{\trLectioBrevis}{\translatioCantus{
Pamatujte na své představené,
kteří vám hlásali Boží slovo.
Uvažte, jak oni skončili život, a~napodobujte jejich víru.
Ježíš Kristus je stejný včera i~dnes i~navěky.
Nenechte se svést věelijakými cizími naukami.}}

\newcommand{\trRespLaud}{\translatioCantus{Spravedlivého vodil Hospodin~\grestar{}
po přímých stezkách. \Vbardot{} A~ukázal mu Boží království.}}

\newcommand{\trRespLaudB}{\translatioCantus{Na tvých hradbách, Jeruzaléme,
ustanovil jsem strážné;~\grestar{}
budou bdít nad mým lidem. \Vbardot{} Ani ve dne, ani v~noci nesmějí nikdy
mlčet.}}

\newcommand{\trVersus}{\translatioCantus{\Vbardot{} Ústa spravedlivého šeptají moudrost, aleluja.
\Rbardot{} A~jeho jazyk ohlašuje právo, aleluja.}}

\newcommand{\trAntBenedictus}{\translatioCantus{Když na bujné oře vložili
nosítka a~sňali jim uzdu, vydali se přímo k~cele božího muže.}}

\newcommand{\trPreces}{\translatioCantus{
\noindent S vděčností chvalme Krista, dobrého Pastýře, \gredagger{} který dal život za své ovce, \grestar{} a~pokorně ho prosme: \Rbardot{} Pane, buď pastýřem svého lidu.

\noindent Kriste, ty dáváš církvi pastýře, a~jejich službou se ujímáš svého lidu, \grestar{} dej, ať v~lásce těch, kteří nás vedou, poznáváme, jak nás miluješ. \Rbardot{} Pane, buď pastýřem svého lidu.

\noindent Ty stále konáš skrze své zástupce službu pastýře a~učitele, \grestar{} nepřestávej nás nikdy vést prostřednictvím svých služebníků. \Rbardot{} Pane, buď pastýřem svého lidu.

\noindent Ty prokazuješ svému lidu skrze jeho pastýře službu lékaře duše i~těla, \grestar{} ochraňuj náš život a~veď nás ke svatosti. \Rbardot{} Pane, buď pastýřem svého lidu.

\noindent Ty posíláš své svaté, aby slovem i~příkladem vedli tvůj lid k~tobě, \grestar{} na jejich přímluvu nás posiluj, abychom vytrvali na cestě, která vede k~věčnému životu. \Rbardot{} Pane, buď pastýřem svého lidu.}}

\newcommand{\trOrationis}{\translatioCantus{Bože, jenž nám dopřáváš radovat
se z~výroční slavnosti svatého tvého vyznavače Havla, uděl dobrotivě,
abychom když slavíme jeho narození, též se řídili podobou jeho skutků.
Skrze…}}
 % Czech translations of the proper texts

\newcommand{\annusEditionis}{2020}

%%%% Vicekrat opakovane kousky

\newcommand{\anteOrationem}{
  \rubrica{Ante Orationem, cantatur a Superiore:}

  \pars{Supplicatio Litaniæ.}

  \cuminitiali{}{temporalia/supplicatiolitaniae.gtex}

  \pars{Oratio Dominica.}

  \cuminitiali{}{temporalia/oratiodominica.gtex}

  \rubrica{Deinde dicitur ab Hebdomadario:}

  \cuminitiali{}{temporalia/dominusvobiscum-solemnis.gtex}

  \rubrica{In choro monialium loco Dominus vobiscum dicitur:}

  \sineinitiali{temporalia/domineexaudi.gtex}
}

\setlength{\columnsep}{30pt} % prostor mezi sloupci

%%%%%%%%%%%%%%%%%%%%%%%%%%%%%%%%%%%%%%%%%%%%%%%%%%%%%%%%%%%%%%%%%%%%%%%%%%%%%%%%%%%%%%%%%%%%%%%%%%%%%%%%%%%%%
\begin{document}

% Here we set the space around the initial.
% Please report to http://home.gna.org/gregorio/gregoriotex/details for more details and options
\grechangedim{afterinitialshift}{2.2mm}{scalable}
\grechangedim{beforeinitialshift}{2.2mm}{scalable}
\grechangedim{interwordspacetext}{0.22 cm plus 0.15 cm minus 0.05 cm}{scalable}%
\grechangedim{annotationraise}{-0.2cm}{scalable}

% Here we set the initial font. Change 38 if you want a bigger initial.
% Emit the initials in red.
\grechangestyle{initial}{\color{red}\fontsize{38}{38}\selectfont}

\pagestyle{empty}

%%%% Titulni stranka
\begin{titulusOfficii}
\titulus{}
\end{titulusOfficii}

% graphic
%\vspace{1.5cm}
%\begin{center}
%\includegraphics[width=8cm]{emmaus.jpg}
%\end{center}

\vfill

\begin{center}
%Ad usum et secundum consuetudines chori \guillemotright{}Conventus Choralis\guillemotleft.

%Editio Sancti Wolfgangi \annusEditionis
\end{center}

\pagebreak

\renewcommand{\headrulewidth}{0pt} % no horiz. rule at the header
\fancyhf{}
\pagestyle{fancy}

\pars{Oratio ante divinum Officium.}

\lettrine{{\color{red}A}}{peri,} Dómine, os meum ad benedicéndum nomen sanctum tuum:
munda quoque cor meum ab ómnibus vanis, pervérsis, et aliénis
cogitatiónibus:
intelléctum illúmina, afféctum inflámma,
ut digne, atténte ac devóte hoc Offícium recitáre váleam,
et exaudíri mérear ante conspéctum Divínæ Maiestátis tuæ.
Per Christum, Dóminum nostrum.
\Rbardot{} Amen.

Dómine, in unióne illíus divínæ intentiónis,
qua ipse in terris laudes Deo persolvísti,
has tibi Horas \rubricatum{(vel \textnormal{hanc tibi Horam})} persólvo.

%\trOratioAnteOfficium

\vfill

\pars{Oratio post divinum Officium.}

\rubrica{
  Orationem sequentem devote post Officium recitantibus
  Leo Papa X. defectus, et culpas in eo persolvendo ex humana
  fragilitate contractas, indulsit, et dicitur flexis genibus.
}

\lettrine{{\color{red}S}}{acrosánctæ} et indivíduæ Trinitáti,
crucifíxi Dómini nostri Iesu Christi humanitáti,
beatíssimæ et gloriosíssimæ sempérque Vírginis Maríæ
fecúndæ integritáti, 
et ómnium Sanctórum universitáti
sit sempitérna laus, honor, virtus et glória
ab omni creatúra,
nobísque remíssio ómnium peccatórum,
per infiníta sǽcula sæculórum.
\Rbardot{} Amen.

\noindent \Vbardot{} Beáta víscera Maríæ Virginis, quæ portavérunt
ætérni Patris Fílium.\\
\Rbardot{} Et beáta úbera, quæ lactavérunt Christum Dominum.

\rubrica{Et dicitur secreto \textnormal{Pater noster.} et \textnormal{Ave María.}}

%\trOratioPostOfficium

\vfill

\hora{Ad I. Vesperas.} %%%%%%%%%%%%%%%%%%%%%%%%%%%%%%%%%%%%%%%%%%%%%%%%%%%%%
%\sideThumbs{I. Vesperæ}

\cantusSineNeumas

\vspace{0.5cm}
\grechangedim{interwordspacetext}{0.18 cm plus 0.15 cm minus 0.05 cm}{scalable}%
\cuminitiali{}{temporalia/deusinadiutorium-solemnis.gtex}
\grechangedim{interwordspacetext}{0.22 cm plus 0.15 cm minus 0.05 cm}{scalable}%

\vfill
\pagebreak

\pars{Psalmus 1.} \scriptura{Ps. 144, 13; \textbf{H100}}

\vspace{-4mm}

\antiphona{VII c\textsuperscript{2}}{temporalia/ant-regnumtuum.gtex}

\scriptura{Psalmus 144, 10-21.}

\initiumpsalmi{temporalia/ps144ii-initium-vii-c2-auto.gtex}

%\psalmusEtTranslatioT{temporalia/ps144ii-VII-comb.tex}{10cm}
\input{temporalia/ps144ii-VII.tex} \Abardot{}

\vspace{-1cm}

\vfill
\pagebreak

\pars{Psalmus 2.} \scriptura{Ps. 145, 2; \textbf{H100}}

\vspace{-4mm}

\antiphona{IV E}{temporalia/ant-laudabodeum.gtex}

\scriptura{Psalmus 145.}

\initiumpsalmi{temporalia/ps145-initium-iv-E-auto.gtex}

%\psalmusEtTranslatioT{temporalia/ps145-VII-comb.tex}{10cm}
\input{temporalia/ps145-VII.tex} \Abardot{}

\vfill
\pagebreak

\pars{Psalmus 3.} \scriptura{Ps. 146, 1; \textbf{H101}}

\vspace{-4mm}

\antiphona{VIII a}{temporalia/ant-deonostro.gtex}

\scriptura{Psalmus 146.}

\initiumpsalmi{temporalia/ps146-initium-viii-A-auto.gtex}

%\psalmusEtTranslatioT{temporalia/ps146-VII-comb.tex}{10cm}
\input{temporalia/ps146-VII.tex} \Abardot{}

\vfill
\pagebreak

\pars{Psalmus 4.} \scriptura{Ps. 147, 1}

\vspace{-4mm}

\antiphona{E}{temporalia/ant-laudajerusalem.gtex}

\scriptura{Psalmus 147.}

\initiumpsalmi{temporalia/ps147-initium-e-auto.gtex}

%\psalmusEtTranslatioT{temporalia/ps147-VII-comb.tex}{10cm}
\input{temporalia/ps147-VII.tex} \Abardot{}

\vfill
\pagebreak

\pars{Capitulum.} \scriptura{Rom. 11, 33}

\grechangedim{interwordspacetext}{0.12 cm plus 0.15 cm minus 0.05 cm}{scalable}%
\cuminitiali{}{temporalia/capitulum-OAltitudo.gtex}
\grechangedim{interwordspacetext}{0.22 cm plus 0.15 cm minus 0.05 cm}{scalable}

% preklad Jeruz. bible
%\trCapituliI

\vfill

\pars{Responsorium breve.} \scriptura{Ps. 146, 5}

\cuminitiali{VI}{temporalia/resp-magnusdominusnoster.gtex}

%\trResp

\vfill
\pagebreak

\pars{Hymnus} \scriptura{Ambrosius (\olddag{} 397)}

\cuminitiali{I}{temporalia/hym-OLuxBeata-aestivalis.gtex}
\vspace{-3mm}
%\input{hym-OLuxBeata-bohtext.tex}

\vfill
%\pagebreak

\pars{Versus.}

% Versus. %%%
\sineinitiali{temporalia/versus-vespertina.gtex}

%\noindent \trVersus

\vfill
\pagebreak

\magnificati

\vfill
\pagebreak

%\sideThumbs{{\scriptsize{}Fine horarum}}

\anteOrationem

\pagebreak

% Oratio. %%%
\oratioLaudes

\vspace{-1mm}
%\trOrationisI

\vfill

\rubrica{Hebdomadarius dicit iterum Dominus vobiscum, vel cantor dicit:}

\vspace{2mm}

\sineinitiali{temporalia/domineexaudi.gtex}

\rubrica{Postea cantatur a cantore:}

\vspace{2mm}

\cuminitiali{I}{temporalia/benedicamus-dominica-perannum.gtex}

\vspace{1mm}

\vfill
\pagebreak

\hora{Ad Matutinum.} %%%%%%%%%%%%%%%%%%%%%%%%%%%%%%%%%%%%%%%%%%%%%%%%%%%%%
%\sideThumbs{Matutinum}

\vspace{2mm}

\cuminitiali{}{temporalia/dominelabiamea.gtex}

\vspace{2mm}

\pars{Invitatorium.} \scriptura{Ps. 94, 1; Psalmus 94}

\vspace{-6mm}

\antiphona{E}{temporalia/inv-veniteexsultemus.gtex}

\vfill
\pagebreak

\pars{Hymnus.} \scriptura{Adamus Sancti Victoris (\olddag 1146)}

\vspace{-5mm}

\antiphona{VII}{temporalia/hym-SalveDies.gtex}

\scriptura{Non dicitur \textnormal{Amen} in fine.}
%{
%\vspace{-5mm}
%\setlength{\columnsep}{0pt} % prostor mezi sloupci
%\input{hym-SalveDies-bohtext.tex}
%\setlength{\columnsep}{30pt} % prostor mezi sloupci
%}

\vfill
\pagebreak

\subhora{In I. Nocturno}

\pars{Psalmus 1.} \scriptura{Ps. 1, 1}

\vspace{-4mm}

\antiphona{VIII G}{temporalia/ant-beatusvir.gtex}

%\vspace{-5mm}

\scriptura{Ps. 1}

%\vspace{-2mm}

\initiumpsalmi{temporalia/ps1-initium-viii-G-auto.gtex}

%\psalmusEtTranslatioT{temporalia/ps1-I-comb.tex}{10cm}
\input{temporalia/ps1-I.tex} \Abardot{}

\vfill
\pagebreak

\pars{Psalmus 2.} \scriptura{Ps. 2, 11; \textbf{H93}}

\vspace{-4mm}

\antiphona{VII a}{temporalia/ant-servitedomino.gtex}

\vspace{-3mm}

\scriptura{Ps. 2}

\vspace{-2mm}

\initiumpsalmi{temporalia/ps2-initium-vii-a-auto.gtex}

%\psalmusEtTranslatioT{temporalia/ps2-I-comb.tex}{10cm}
\input{temporalia/ps2-I.tex} \Abardot{}

\vfill
\pagebreak

\pars{Psalmus 3.} \scriptura{Ps. 3, 7}

\vspace{-4mm}

\antiphona{VI F}{temporalia/ant-exsurgedominesalvum.gtex}

%\vspace{-5mm}

\scriptura{Ps. 3}

\initiumpsalmi{temporalia/ps3-initium-vi-F-auto.gtex}

%\psalmusEtTranslatioT{temporalia/ps3-I-comb.tex}{10cm}
\input{temporalia/ps3-I.tex} \Abardot{}

\vfill
\pagebreak

\pars{Versus.} \scriptura{Ps. 118, 55}

% Versus. %%%
\sineinitiali{temporalia/versus-memorfui.gtex}

\vspace{5mm}

\sineinitiali{temporalia/oratiodominica-mat.gtex}

\vspace{5mm}

\pars{Absolutio.}

\cuminitiali{}{temporalia/absolutio-exaudi.gtex}

\vfill
\pagebreak

\cuminitiali{}{temporalia/benedictio-solemn-benedictione.gtex}

\vspace{7mm}

\lectioi

\noindent \Vbardot{} Tu autem, Dómine, miserére nobis.
\noindent \Rbardot{} Deo grátias.

\vfill
\pagebreak

\responsoriumi

\vfill
\pagebreak

\cuminitiali{}{temporalia/benedictio-solemn-unigenitus.gtex}

\vspace{7mm}

\lectioii

\noindent \Vbardot{} Tu autem, Dómine, miserére nobis.
\noindent \Rbardot{} Deo grátias.

\vfill
\pagebreak

\responsoriumii

\vfill
\pagebreak

\cuminitiali{}{temporalia/benedictio-solemn-spiritus.gtex}

\vspace{7mm}

\lectioiii

\noindent \Vbardot{} Tu autem, Dómine, miserére nobis.
\noindent \Rbardot{} Deo grátias.

\vfill
\pagebreak

\responsoriumiii

\vfill
\pagebreak

\subhora{In II. Nocturno}

\pars{Psalmus 4.} \scriptura{Ps. 8, 2}

\vspace{-4mm}

\antiphona{I g}{temporalia/ant-quamadmirabileest.gtex}

%\vspace{-5mm}

\scriptura{Ps. 8}

%A\vspace{-2mm}

\initiumpsalmi{temporalia/ps8-initium-i-g-auto.gtex}

%\psalmusEtTranslatioT{temporalia/ps8-I-comb.tex}{10cm}
\input{temporalia/ps8-I.tex} \Abardot{}

\vfill
\pagebreak

\pars{Psalmus 5.} \scriptura{Ps. 9, 5}

\vspace{-4mm}

\antiphona{VIII G}{temporalia/ant-sedistisuperthronum.gtex}

%\vspace{-5mm}

\scriptura{Ps. 9, 2-11}

\initiumpsalmi{temporalia/ps9ii_xi-initium-viii-G-auto.gtex}

%\psalmusEtTranslatioT{temporalia/ps9ii_xi-I-comb.tex}{10cm}
\input{temporalia/ps9ii_xi-I.tex} \Abardot{}

\vfill
\pagebreak

\pars{Psalmus 6.} \scriptura{Ps. 9, 20}

\vspace{-4mm}

\antiphona{I g\textsuperscript{3}}{temporalia/ant-exsurgedominenon.gtex}

%\vspace{-5mm}

\scriptura{Ps. 9, 12-21}

\initiumpsalmi{temporalia/ps9xii_xxi-initium-i-g3-auto.gtex}

%\psalmusEtTranslatioT{temporalia/ps9xii_xxi-I-comb.tex}{10cm}
\input{temporalia/ps9xii_xxi-I.tex} \Abardot{}

\vfill
\pagebreak

\pars{Versus.} \scriptura{Ps. 118, 62}

% Versus. %%%
\sineinitiali{temporalia/versus-medianocte.gtex}

\vspace{5mm}

\sineinitiali{temporalia/oratiodominica-mat.gtex}

\vspace{5mm}

\pars{Absolutio.}

\cuminitiali{}{temporalia/absolutio-ipsius.gtex}

\vfill
\pagebreak

\cuminitiali{}{temporalia/benedictio-solemn-deus.gtex}

\vspace{7mm}

\lectioiv

\noindent \Vbardot{} Tu autem, Dómine, miserére nobis.
\noindent \Rbardot{} Deo grátias.

\vfill
\pagebreak

\responsoriumiv

\vfill
\pagebreak

\cuminitiali{}{temporalia/benedictio-solemn-christus.gtex}

\vspace{7mm}

\lectiov

\noindent \Vbardot{} Tu autem, Dómine, miserére nobis.
\noindent \Rbardot{} Deo grátias.

\vfill
\pagebreak

\responsoriumv

\vfill
\pagebreak

\cuminitiali{}{temporalia/benedictio-solemn-ignem.gtex}

\vspace{7mm}

\lectiovi

\noindent \Vbardot{} Tu autem, Dómine, miserére nobis.
\noindent \Rbardot{} Deo grátias.

\vfill
\pagebreak

\responsoriumvi

\vfill
\pagebreak

\subhora{In III. Nocturno}

\pars{Psalmus 7.} \scriptura{Ps. 9, 22}

\vspace{-4mm}

\antiphona{II D}{temporalia/ant-utquiddomine.gtex}

\vspace{-4mm}

\scriptura{Ps. 9, 22-32}

%\vspace{-2mm}

\initiumpsalmi{temporalia/ps9xxii_xxxii-initium-ii-D-auto.gtex}

%\psalmusEtTranslatioT{temporalia/ps9xxii_xxxii-I-comb.tex}{10cm}
\input{temporalia/ps9xxii_xxxii-I.tex} \Abardot{}

\vfill
\pagebreak

\pars{Psalmus 8.}\scriptura{Ex. 15, 18}

\vspace{-4mm}

\antiphona{IV* e}{temporalia/ant-inaeternum.gtex}

%\vspace{-4mm}

\scriptura{Ps. 9, 33-39}

\initiumpsalmi{temporalia/ps9xxxiii_xxxix-initium-iv_-e-auto.gtex}

%\psalmusEtTranslatioT{temporalia/ps9xxxiii_xxxix-I-comb.tex}{10cm}
\input{temporalia/ps9xxxiii_xxxix-I.tex} \Abardot{}

\vfill
\pagebreak

\pars{Psalmus 9.} \scriptura{Ps. 10, 8}

\vspace{-4mm}

\antiphona{II* f}{temporalia/ant-justusdominus.gtex}

%\vspace{-4mm}

\scriptura{Ps. 10}

%\initiumpsalmi{temporalia/ps10-initium-iv-c-auto.gtex}
\initiumpsalmi{temporalia/ps10-initium-ii_-f.gtex}

%\psalmusEtTranslatioT{temporalia/ps10-I-comb.tex}{10cm}
\input{temporalia/ps10-I.tex} \Abardot{}

\vfill
\pagebreak

\pars{Versus.} \scriptura{Ps. 118, 148}

% Versus. %%%
\sineinitiali{temporalia/versus-praevenerunt.gtex}

\vspace{5mm}

\sineinitiali{temporalia/oratiodominica-mat.gtex}

\vspace{5mm}

\pars{Absolutio.}

\cuminitiali{}{temporalia/absolutio-avinculis.gtex}

\vfill
\pagebreak

\cuminitiali{}{temporalia/benedictio-solemn-evangelica.gtex}

\vspace{7mm}

\lectiovii

\noindent \Vbardot{} Tu autem, Dómine, miserére nobis.
\noindent \Rbardot{} Deo grátias.

\vfill
\pagebreak

\responsoriumvii

\vfill
\pagebreak

\cuminitiali{}{temporalia/benedictio-solemn-divinum.gtex}

\vspace{7mm}

\lectioviii

\noindent \Vbardot{} Tu autem, Dómine, miserére nobis.
\noindent \Rbardot{} Deo grátias.

\vfill
\pagebreak

\responsoriumviii

\vfill
\pagebreak

\cuminitiali{}{temporalia/benedictio-solemn-adsocietatem.gtex}

\vspace{7mm}

\lectioix

\noindent \Vbardot{} Tu autem, Dómine, miserére nobis.
\noindent \Rbardot{} Deo grátias.

\vfill
\pagebreak

% Te Deum

{
\pars{Hymnus Ambrosianus} \scriptura{Tonus Solemnis}

\vspace{-2mm}

\grechangedim{interwordspacetext}{0.26 cm plus 0.15 cm minus 0.05 cm}{scalable}%
\cuminitiali{III}{temporalia/tedeum-solemnis-gn.gtex}
\grechangedim{interwordspacetext}{0.22 cm plus 0.15 cm minus 0.05 cm}{scalable}%
}

\vfill
\pagebreak

\rubrica{Reliqua omittuntur, nisi Laudes separandæ sint.}

\pars{Oratio}

\noindent \Vbardot{} Dómine, exáudi oratiónem meam.

\noindent \Rbardot{} Et clamor meus ad te véniat.

Orémus:

\oratioLaudes

\vspace{7mm}

\pars{Conclusio}

\noindent \Vbardot{} Dómine, exáudi oratiónem meam.

\noindent \Rbardot{} Et clamor meus ad te véniat.

\noindent \Vbardot{} Benedicámus Dómino, allelúia, allelúia.

\noindent \Rbardot{} Deo grátias, allelúia, allelúia.

\noindent \Vbardot{} Fidélium ánimæ per misericórdiam Dei requiéscant in pace.

\noindent \Rbardot{} Amen.

\vfill
\pagebreak

\hora{Ad Laudes.} %%%%%%%%%%%%%%%%%%%%%%%%%%%%%%%%%%%%%%%%%%%%%%%%%%%%%
%\sideThumbs{Laudes}

\cantusSineNeumas

\vspace{0.5cm}
\grechangedim{interwordspacetext}{0.18 cm plus 0.15 cm minus 0.05 cm}{scalable}%
\cuminitiali{}{temporalia/deusinadiutorium-alter.gtex}
\grechangedim{interwordspacetext}{0.22 cm plus 0.15 cm minus 0.05 cm}{scalable}%

\vfill
%\pagebreak

\pars{Psalmus 1.}

\vspace{-4mm}

\antiphona{VI F}{temporalia/ant-alleluia1.gtex}

\scriptura{Psalmus 50.}

\initiumpsalmi{temporalia/ps50-initium-vi-F-auto.gtex}

%\psalmusEtTranslatioT{temporalia/ps50-I-comb.tex}{10cm}
\input{temporalia/ps50-I.tex}

\vfill
\pagebreak

\pars{Psalmus 2.}

\scriptura{Psalmus 117.}

\initiumpsalmi{temporalia/ps117-initium-vi-F-auto.gtex}

%\psalmusEtTranslatioT{temporalia/ps117-I-comb.tex}{10cm}
\input{temporalia/ps117-I.tex}

\vfill
\pagebreak

\pars{Psalmus 3.}

\scriptura{Psalmus 62.}

\initiumpsalmi{temporalia/ps62-initium-vi-F-auto.gtex}

%\psalmusEtTranslatioT{temporalia/ps62-I-comb.tex}{10cm}
\input{temporalia/ps62-I.tex}

\vfill

\vspace{-6mm}

\antiphona{}{temporalia/ant-alleluia1.gtex} % repeat the antiphon - new page

\vfill
\pagebreak

\pars{Psalmus 4.} \scriptura{Dan. 3, 22-26; \textbf{H422}}

\vspace{-4mm}

\antiphona{VIII G}{temporalia/ant-trespueri.gtex}

\scriptura{Canticum trium puerorum, Dan. 3, 57-88 et 56}

\initiumpsalmi{temporalia/dan3-initium-viii-G-auto.gtex}

%\psalmusEtTranslatioT{temporalia/dan3-comb.tex}{10cm}
\input{temporalia/dan3.tex}

\rubrica{Hic non dicitur Gloria Patri, neque Amen.}

\vfill

\vspace{-6mm}

\antiphona{}{temporalia/ant-trespueri.gtex} % repeat the antiphon - new page

\vfill
\pagebreak

\pars{Psalmus 5.}

\vspace{-4mm}

\antiphona{VIII G}{temporalia/ant-alleluia2.gtex}

\scriptura{Psalmus 148.}

\initiumpsalmi{temporalia/ps148-initium-viii-G-auto.gtex}

%\psalmusEtTranslatioT{temporalia/ps148-I-comb.tex}{10cm}
\input{temporalia/ps148-I.tex}

\rubrica{Hic non dicitur Gloria Patri.}

\vfill
\pagebreak

%
\scriptura{Psalmus 149.}

\initiumpsalmi{temporalia/ps149-initium-viii-G-auto.gtex}

%\psalmusEtTranslatioT{temporalia/ps149-I-comb.tex}{10cm}
\input{temporalia/ps149-I.tex}

\rubrica{Hic non dicitur Gloria Patri.}

\vfill
\pagebreak

%
\scriptura{Psalmus 150.}

\initiumpsalmi{temporalia/ps150-initium-viii-G-auto.gtex}

%\psalmusEtTranslatioT{temporalia/ps150-I-comb.tex}{10cm}
\input{temporalia/ps150-I.tex}

\vfill

\vspace{-6mm}

\antiphona{}{temporalia/ant-alleluia2.gtex} % repeat the antiphon - new page

\vfill
\pagebreak

\pars{Capitulum.} \scriptura{Ac. 7, 12}

\grechangedim{interwordspacetext}{0.12 cm plus 0.15 cm minus 0.05 cm}{scalable}%
\cuminitiali{}{temporalia/capitulum-Benedictio.gtex}
\grechangedim{interwordspacetext}{0.22 cm plus 0.15 cm minus 0.05 cm}{scalable}

% preklad Jeruz. bible
%\trCapituliI

\vfill

\pars{Responsorium breve.} \scriptura{Ps. 118, 36-37}

\cuminitiali{IV}{temporalia/resp-inclinacormeum.gtex}

%\trResp

\vfill
\pagebreak

\pars{Hymnus} \scriptura{Gregorius Magnus (\olddag{} 604)}

\cuminitiali{IV}{temporalia/hym-EcceJamNoctis.gtex}
\vspace{-3mm}
%\input{hym-EcceJamNocis-bohtext.tex}

\vfill
%\pagebreak

\pars{Versus.} \scriptura{Ps. 92, 1}

% Versus. %%%
\sineinitiali{temporalia/versus-dominusregnavit.gtex}

%\noindent \trVersus

\vfill
\pagebreak

\benedictus

\vspace{-1cm}

\vfill
\pagebreak

%\sideThumbs{{\scriptsize{}Fine horarum}}

\anteOrationem

\pagebreak

% Oratio. %%%
\oratioLaudes

\vspace{-1mm}
%\trOrationisI

\vfill

\rubrica{Hebdomadarius dicit iterum Dominus vobiscum, vel cantor dicit:}

\vspace{2mm}

\sineinitiali{temporalia/domineexaudi.gtex}

\rubrica{Postea cantatur a cantore:}

\vspace{2mm}

\cuminitiali{I}{temporalia/benedicamus-dominica-perannum.gtex}

\vspace{1mm}

\vfill
\pagebreak

\hora{Ad II. Vesperas.} %%%%%%%%%%%%%%%%%%%%%%%%%%%%%%%%%%%%%%%%%%%%%%%%%%%%%
%\sideThumbs{II. Vesperæ}

\cantusSineNeumas

%\vspace{0.5cm}
\grechangedim{interwordspacetext}{0.18 cm plus 0.15 cm minus 0.05 cm}{scalable}%
\cuminitiali{}{temporalia/deusinadiutorium-solemnis.gtex}
\grechangedim{interwordspacetext}{0.22 cm plus 0.15 cm minus 0.05 cm}{scalable}%

\vfill
%\pagebreak

\vspace{-2mm}

\pars{Psalmus 1.} \scriptura{Ps. 109, 1; \textbf{H91}}

\vspace{-4mm}

\antiphona{VII c\textsuperscript{2}}{temporalia/ant-dixitdominus.gtex}

\vspace{-4mm}

\scriptura{Psalmus 109.}

\initiumpsalmi{temporalia/ps109-initium-vii-c2-auto.gtex}

%\psalmusEtTranslatioT{temporalia/ps109-I-comb.tex}{10cm}
\input{temporalia/ps109-I.tex} \Abardot{}

\vspace{-1cm}

\vfill
\pagebreak

\pars{Psalmus 2.} \scriptura{Ps. 110, 8; \textbf{H91}}

\vspace{-4mm}

\antiphona{IV g}{temporalia/ant-fideliaomnia.gtex}

\scriptura{Psalmus 110.}

\initiumpsalmi{temporalia/ps110-initium-iv-g-auto.gtex}

%\psalmusEtTranslatioT{temporalia/ps110-I-comb.tex}{10cm}
\input{temporalia/ps110-I.tex} \Abardot{}

\vfill
\pagebreak

\pars{Psalmus 3.} \scriptura{Ps. 111, 1; \textbf{H92}}

\vspace{-4mm}

\antiphona{IV a}{temporalia/ant-inmandatis.gtex}

\scriptura{Psalmus 111.}

\initiumpsalmi{temporalia/ps111-initium-iv-a-auto.gtex}

%\psalmusEtTranslatioT{temporalia/ps111-I-comb.tex}{10cm}
\input{temporalia/ps111-I.tex} \Abardot{}

\vfill
\pagebreak

\pars{Psalmus 4.} \scriptura{Ps. 112, 2; \textbf{H92}}

\vspace{-4mm}

\antiphona{VII c}{temporalia/ant-sitnomendomini.gtex}

\scriptura{Psalmus 112.}

\initiumpsalmi{temporalia/ps112-initium-vii-c-auto.gtex}

%\psalmusEtTranslatioT{temporalia/ps112-I-comb.tex}{10cm}
\input{temporalia/ps112-I.tex} \Abardot{}

\vfill
\pagebreak

\pars{Capitulum.} \scriptura{2 Cor. 1, 3-4}

\grechangedim{interwordspacetext}{0.12 cm plus 0.15 cm minus 0.05 cm}{scalable}%
\cuminitiali{}{temporalia/capitulum-BenedictusDeus.gtex}
\grechangedim{interwordspacetext}{0.22 cm plus 0.15 cm minus 0.05 cm}{scalable}

% preklad Jeruz. bible
%\trCapituliI

\vfill

\pars{Responsorium breve.} \scriptura{Ps. 103, 24}

\cuminitiali{VI}{temporalia/resp-quammagnificata.gtex}

%\trResp

\vfill
\pagebreak

\pars{Hymnus} \scriptura{Gregorius Magnus (\olddag{} 604)}

\cuminitiali{I}{temporalia/hym-LucisCreator-aestivalis.gtex}
\vspace{-3mm}
%\begin{translatioMulticol}{3}
Tvůrce světa předobrý,\\
tys ustanovil denní řád\\
a proudy světla rozhodil,\\
když světu základy jsi klad.\\
\\
A spojils ráno s večerem\\
a dnem tu dobu nazýváš;\\
hle padá temné noci stín -\\
slyš prosbu, vyslyš nářek náš.\columnbreak

Ach, nedej, by nás stihla smrt,\\
když svědomí nám tíží hřích,\\
když nemyslíme na věčnost\\
v té síti hříchů šalebných.\\
\\
Vzbuď naši touhu po nebi,\\
kde věčný život čeká nás,\\
a pomoz odložit vše zlé\\
a smýti z duše každý kaz.\columnbreak

To splň nám, dobrý Otče náš,\\
i ty, jenž rovné božství máš,\\
i Duchu, který těšíš nás\\
a vládneš, Bože, v každý čas.\\
Amen. 
\end{translatioMulticol}


\vfill
%\pagebreak

\pars{Versus.} \scriptura{Ps. 140, 2}

% Versus. %%%
\sineinitiali{temporalia/versus-dirigatur.gtex}

%\noindent \trVersus

\vfill
\pagebreak

\magnificatii

\vfill
\pagebreak

%\sideThumbs{{\scriptsize{}Fine horarum}}

\anteOrationem

\pagebreak

% Oratio. %%%
\oratioLaudes

\vspace{-1mm}
%\trOrationisI

\vfill

\rubrica{Hebdomadarius dicit iterum Dominus vobiscum, vel cantor dicit:}

\vspace{2mm}

\sineinitiali{temporalia/domineexaudi.gtex}

\rubrica{Postea cantatur a cantore:}

\vspace{2mm}

\cuminitiali{I}{temporalia/benedicamus-dominica-perannum.gtex}

\vspace{1mm}

\end{document}

