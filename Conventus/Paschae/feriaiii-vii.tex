\newcommand{\titulus}{\nomenFesti{In Visitatione Beatæ Mariæ Virginis.}
\dies{Die 31. Maii.}}
\newcommand{\oratio}{\pars{Oratio.}

\noindent Omnípotens sempitérne Deus, qui Beátam Vírginem Maríam, Fílium tuum gestántem, ad visitándam Elísabeth inspirásti, præsta, quǽsumus, ut, afflánti Spirítui obsequéntes, cum ipsa te semper magnificáre possímus.

\pars{Pro pace in Ucraina.} \scriptura{Sir. 50, 25; 2 Esdr. 4, 20; \textbf{H416}}

\vspace{-4mm}

\antiphona{II D}{temporalia/ant-dapacemdomine.gtex}

\vfill

\noindent Deus, a quo sancta desidéria, recta consília et iusta sunt ópera: da servis tuis illam, quam mundus dare non potest, pacem; ut et corda nostra mandátis tuis dédita, et hóstium subláta formídine, témpora sint tua protectióne tranquílla.

\noindent Per Dóminum nostrum Iesum Christum, Fílium tuum, qui tecum vivit et regnat in unitáte Spíritus Sancti, Deus, per ómnia sǽcula sæculórum.

\noindent \Rbardot{} Amen.}
\newcommand{\invitatorium}{\pars{Invitatorium.}

\vspace{-4mm}

\antiphona{II}{temporalia/inv-visitationemvirginis-tp.gtex}}
\newcommand{\hymnusmatutinum}{\pars{Hymnus}

\cuminitiali{VIII}{temporalia/hym-VeniPraecelsa.gtex}}
\newcommand{\matutinum}{\pars{Psalmus 1.} \scriptura{Lc. 1, 39}

\vspace{-4mm}

\antiphona{VIII G}{temporalia/ant-exsurgensmaria-tp.gtex}

%\vspace{-2mm}

\scriptura{Ps. 23}

%\vspace{-2mm}

\initiumpsalmi{temporalia/ps23-initium-viii-G-auto.gtex}

%\vspace{-1.5mm}

\input{temporalia/ps23-viii-G.tex} \Abardot{}

\vfill
\pagebreak

\pars{Psalmus 2.} \scriptura{Lc. 1, 40}

\vspace{-4mm}

\antiphona{II D}{temporalia/ant-intravitmaria-tp.gtex}

%\vspace{-2mm}

\scriptura{Ps. 45}

%\vspace{-2mm}

\initiumpsalmi{temporalia/ps45-initium-ii-D-auto.gtex}

%\vspace{-1.5mm}

\input{temporalia/ps45-ii-D.tex} \Abardot{}

\vfill
\pagebreak

\pars{Psalmus 3.} \scriptura{Ct. 7, 4; \textbf{H299}}

\vspace{-4mm}

\antiphona{VII b}{temporalia/ant-oculituisanctadeigenetrix.gtex}

\scriptura{Ps. 86}

%\vspace{-2mm}

\initiumpsalmi{temporalia/ps86-initium-vii-b-auto.gtex}

\input{temporalia/ps86-vii-b.tex} \Abardot{}

\vfill
\pagebreak}
\newcommand{\matversus}{\noindent \Vbardot{} Spécie tua et pulchritúdine tua.

\noindent \Rbardot{} Inténde, próspere procéde, et regna.}
\newcommand{\lectioi}{\pars{Lectio I.} \scriptura{Ct. 2, 8-14; 8, 6-7}

\noindent Vox dilécti mei! Ecce iste venit sáliens in móntibus, transíliens colles. Símilis est diléctus meus cápreæ hinnulóque cervórum. En ipse stat post paríetem nostrum respíciens per fenéstras, prospíciens per cancéllos.

\noindent En diléctus meus lóquitur mihi: «Surge, amíca mea, colúmba mea, formósa mea, et veni. Iam enim hiems tránsiit, imber ábiit et recéssit. Flores apparuérunt in terra, tempus putatiónis advénit; vox túrturis audíta est in terra nostra, ficus prótulit grossos suos, víneæ floréntes dedérunt odórem suum: surge, amíca mea, speciósa mea, et veni, colúmba mea, in foramínibus petræ, in cavérna abrúpta. Osténde mihi fáciem tuam, sonet vox tua in áuribus meis; vox enim tua dulcis, et fácies tua decóra».

\noindent Pone me ut signáculum super cor tuum, ut signáculum super bráchium tuum, quia fortis est ut mors diléctio, dura sicut inférnus æmulátio; lámpades eius lámpades ignis atque flammæ divínæ. Aquæ multæ non potuérunt exstínguere caritátem, nec flúmina óbruent illam; si déderit homo omnem substántiam domus suæ pro dilectióne, quasi nihil despícient eum.}
\newcommand{\responsoriumi}{\pars{Responsorium 1.} \scriptura{\Rbardot{} Ct. 6, 9 \Vbardot{} ibid. 3, 6; \textbf{H297}}

\vspace{-5mm}

\responsorium{IV}{temporalia/resp-quaeestista-CROCHU-tp.gtex}{}}
\newcommand{\lectioii}{\pars{Lectio II.} \scriptura{Med. 8: SC 163, 166-168}

\noindent Ex Meditatiónibus Guigónis secúndi Cartusiáni mónachi.

\noindent O grátia plena, quid est quod báiulas in sinu tuo? Dóminus est, \emph{ego ancílla Dómini. Fecit mihi magna qui potens est;} mérito miránda, quia magna, sed potens est qui fecit mihi magna.

\noindent Dóminus est, ego ancílla: ille ros, ego terra, inde fruméntum; ille manna, ego vas, inde vermículus. \emph{Ego,} inquit, \emph{sum vermis et non homo: homo} enim \emph{sicut fenum,} iste fruméntum. Magna sunt, sed potens est qui fecit. Unum granum fruménti de me náscitur, et de abundántia fruménti dícitur: \emph{Si autem mórtuum fúerit, multum fructum affert.} Móriens autem, abundántiam vini dedit; resúrgens et ascéndens, \emph{óleum effúdit, quam effúdit,} inquit Apóstolus, \emph{in nos abúnde.}}
\newcommand{\responsoriumii}{\pars{Responsorium 2.} \scriptura{\Vbardot{} Lc. 1, 28; \textbf{H118}}

\vspace{-5mm}

\responsorium{IV}{temporalia/resp-benedictaetvenerabilis-CROCHU-tp.gtex}{}}
\newcommand{\lectioiii}{\pars{Lectio III.}

\noindent Ecce abundántia fruménti, vini et ólei, de rore cæli et de pinguédine terræ. O pinguédo terræ, grátia plena: sicut adeps a carne, sic separáta es a massa peccatríce, grátia plena, plena fruménti, plena vini, plena ólei, plena et superabúndans ómnibus donis Spíritus Sancti.

\noindent \emph{Dóminus tecum:} tecum in cubículo cordis, tecum in thálamo ventris; tecum manet, tecum persevérat, et numquam a te recédit. \emph{Dóminus tecum:} Quid est tecum? Dóminus una tecum natúra est, omníno super ángelos elevánda. Hábitat in ángelis Deus, non cum ángelis; hábitat Deus in te, hábitat tecum. Sedet super ángelos Deus, \emph{sedet super thronos, sedet super chérubim et séraphim,} sedet et regnat in ómnibus his, sed in univérsis regnis non est opus símile sicut thronus iste de ébore grandis.}
\newcommand{\responsoriumiii}{\pars{Responsorium 3.} \scriptura{\Rbardot{} Lc. 1, 48 \Vbardot{} ibid. 1, 50; \textbf{H297}}

\vspace{-5mm}

\responsorium{VIII}{temporalia/resp-beatammedicent-CROCHU-tp-cumdox.gtex}{}

\vfill
\pagebreak

\pars{Hymnus Ambrosianus} \scriptura{Alio modo, iuxta morem Romanum}

\vspace{-2mm}

\grechangedim{interwordspacetext}{0.26 cm plus 0.15 cm minus 0.05 cm}{scalable}%
\cuminitiali{III}{temporalia/tedeum-romanum-gn.gtex}
\grechangedim{interwordspacetext}{0.22 cm plus 0.15 cm minus 0.05 cm}{scalable}%
}
\newcommand{\hymnuslaudes}{\pars{Hymnus}

\cuminitiali{II}{temporalia/hym-VeniensMater.gtex}}
\newcommand{\laudes}{\pars{Psalmus 1.} \scriptura{Lc. 1, 41}

\vspace{-4mm}

\antiphona{III e}{temporalia/ant-utaudivitsalutationem-tp.gtex}

%\vspace{-2mm}

\scriptura{Psalmus 62}

%\vspace{-2mm}

\initiumpsalmi{temporalia/ps62-initium-iii-e-auto.gtex}

%\vspace{-1.5mm}

\input{temporalia/ps62-iii-e.tex} \Abardot{}

\vfill
\pagebreak

\pars{Psalmus 2.} \scriptura{Lc. 1, 42; \textbf{H21}}

\vspace{-4mm}

\antiphona{II* a}{temporalia/ant-benedictatu-tp.gtex}

%\vspace{-2mm}

\scriptura{Canticum trium puerorum, Dan. 3, 57-88 et 56}

\initiumpsalmi{temporalia/dan3-initium-ii_-a-auto.gtex}

\input{temporalia/dan3-ii_-a-sinedox.tex}

\rubrica{Hic non dicitur Gloria Patri, neque Amen.}

\vfill

\antiphona{}{temporalia/ant-benedictatu-tp.gtex}

\vfill
\pagebreak

\pars{Psalmus 3.} \scriptura{Lc. 1, 44; \textbf{H39}}

\vspace{-4mm}

\antiphona{II* f}{temporalia/ant-exquofactaestvox.gtex}

%\vspace{-2mm}

\scriptura{Psalmus 149}

%\vspace{-2mm}

\initiumpsalmi{temporalia/ps149-initium-ii_-f.gtex}

\input{temporalia/ps149-ii_-f.tex} \Abardot{}

\vfill
\pagebreak}
\newcommand{\lectiobrevis}{\pars{Lectio Brevis.} \scriptura{Ioel 2, 27-28}

\noindent Sciétis quia in médio Israel ego sum, et ego Dóminus Deus vester et non est ámplius; et non confundétur pópulus meus in ætérnum. Et erit post hæc: effúndam spíritum meum super omnem carnem, et prophetábunt fílii vestri et fíliæ vestræ.}
\newcommand{\responsoriumbreve}{\pars{Responsorium breve.} \scriptura{Ps. 44, 3}

\antiphona{VI}{temporalia/resp-diffusaestgratia-tp.gtex}}
\newcommand{\preces}{\noindent Salvatórem nostrum celebrántes, qui ex María Vírgine nasci dignátus est, \gredagger{} exorémus dicéntes:

\Rbardot{} Intercédat pro nobis mater tua, Dómine.

\noindent O sol iustítiæ, quem Immaculáta Virgo ut lucens auróra præcéssit, \gredagger{} tríbue, ut in lúmine visitatiónis tuæ semper ambulémus.

\Rbardot{} Intercédat pro nobis mater tua, Dómine.

\noindent Tríbue nobis, Dómine, Maríam matrem tuam imitári, quæ óptimam partem elégit, \gredagger{} cibum quæréntes, qui pérmanet in vitam ætérnam.

\Rbardot{} Intercédat pro nobis mater tua, Dómine.

\noindent Salvátor mundi, qui redemptiónis tuæ virtúte ab omni peccáti labe matrem tuam præservásti, \gredagger{} serva nos mundos a peccáto.

\Rbardot{} Intercédat pro nobis mater tua, Dómine.

\noindent Redémptor noster, qui Vírginem Maríam thálamum puríssimum habitatiónis tuæ et Spíritus Sancti fecísti sacrárium, \gredagger{} nos templum fac perénne tui Spíritus.

\Rbardot{} Intercédat pro nobis mater tua, Dómine.}
\newcommand{\benedictus}{\pars{Canticum Zachariæ.} \scriptura{Cf. Lc. 1, 45; \textbf{H24}}

\vspace{-4mm}

\antiphona{VIII G}{temporalia/ant-beataesmariaquaecredisti.gtex}

%\vspace{-2mm}

\scriptura{Lc. 1, 68-79}

%\vspace{-2mm}

\cantusSineNeumas
\initiumpsalmi{temporalia/benedictus-initium-viii-G-auto.gtex}

%\vspace{-1.5mm}

\input{temporalia/benedictus-viii-G.tex} \Abardot{}}
\newcommand{\benedicamuslaudes}{\cuminitiali{I}{temporalia/benedicamus-festis-bmv.gtex}}
\newcommand{\precestotum}{\pars{Deprecatio Gelasii}

\vspace{-5mm}

\grechangedim{interwordspacetext}{0.16 cm plus 0.15 cm minus 0.05 cm}{scalable}%
\antiphona{D\textsuperscript{1}}{temporalia/deprecatio4-propace.gtex}
\grechangedim{interwordspacetext}{0.22 cm plus 0.15 cm minus 0.05 cm}{scalable}%

\vfill

\pars{Oratio Dominica.}

\cuminitiali{D}{temporalia/oratiodominica-d.gtex}}
\newcommand{\dominusnosbenedicat}{\antiphona{D}{temporalia/dominusnosbenedicat-d.gtex}}
\newcommand{\hebdomada}{infra Hebdom. VII Paschæ.}
\newcommand{\matuc}{Matutinum Hebdomadae C}
\newcommand{\matuac}{Matutinum Hebdomadae A vel C}
\newcommand{\laudc}{Laudes Hebdomadae C}
\newcommand{\laudac}{Laudes Hebdomadae A vel C}

% LuaLaTeX

\documentclass[a4paper, twoside, 12pt]{article}
\usepackage[latin]{babel}
%\usepackage[landscape, left=3cm, right=1.5cm, top=2cm, bottom=1cm]{geometry} % okraje stranky
%\usepackage[landscape, a4paper, mag=1166, truedimen, left=2cm, right=1.5cm, top=1.6cm, bottom=0.95cm]{geometry} % okraje stranky
\usepackage[landscape, a4paper, mag=1400, truedimen, left=0.5cm, right=0.5cm, top=0.5cm, bottom=0.5cm]{geometry} % okraje stranky

\usepackage{fontspec}
\setmainfont[FeatureFile={junicode.fea}, Ligatures={Common, TeX}, RawFeature=+fixi]{Junicode}
%\setmainfont{Junicode}

% shortcut for Junicode without ligatures (for the Czech texts)
\newfontfamily\nlfont[FeatureFile={junicode.fea}, Ligatures={Common, TeX}, RawFeature=+fixi]{Junicode}

\usepackage{multicol}
\usepackage{color}
\usepackage{lettrine}
\usepackage{fancyhdr}

% usual packages loading:
\usepackage{luatextra}
\usepackage{graphicx} % support the \includegraphics command and options
\usepackage{gregoriotex} % for gregorio score inclusion
\usepackage{gregoriosyms}
\usepackage{wrapfig} % figures wrapped by the text
\usepackage{parcolumns}
\usepackage[contents={},opacity=1,scale=1,color=black]{background}
\usepackage{tikzpagenodes}
\usepackage{calc}
\usepackage{longtable}
\usetikzlibrary{calc}

\setlength{\headheight}{14.5pt}

\input{conventuscommune.tex} % Often used macros

\newcommand{\annusEditionis}{2021}

%%%% Vicekrat opakovane kousky

\newcommand{\anteOrationem}{
  \rubrica{Ante Orationem, cantatur a Superiore:}

  \pars{Supplicatio Litaniæ.}

  \cuminitiali{}{temporalia/supplicatiolitaniae.gtex}

  \pars{Oratio Dominica.}

  \cuminitiali{}{temporalia/oratiodominica.gtex}

  \rubrica{Deinde dicitur ab Hebdomadario:}

  \cuminitiali{}{temporalia/dominusvobiscum-solemnis.gtex}

  \rubrica{In choro monialium loco Dominus vobiscum dicitur:}

  \sineinitiali{temporalia/domineexaudi.gtex}
}

\setlength{\columnsep}{30pt} % prostor mezi sloupci

%%%%%%%%%%%%%%%%%%%%%%%%%%%%%%%%%%%%%%%%%%%%%%%%%%%%%%%%%%%%%%%%%%%%%%%%%%%%%%%%%%%%%%%%%%%%%%%%%%%%%%%%%%%%%
\begin{document}

% Here we set the space around the initial.
% Please report to http://home.gna.org/gregorio/gregoriotex/details for more details and options
\grechangedim{afterinitialshift}{2.2mm}{scalable}
\grechangedim{beforeinitialshift}{2.2mm}{scalable}
\grechangedim{interwordspacetext}{0.22 cm plus 0.15 cm minus 0.05 cm}{scalable}%
\grechangedim{annotationraise}{-0.2cm}{scalable}

% Here we set the initial font. Change 38 if you want a bigger initial.
% Emit the initials in red.
\grechangestyle{initial}{\color{red}\fontsize{38}{38}\selectfont}

\pagestyle{empty}

%%%% Titulni stranka
\begin{titulusOfficii}
\ifx\titulus\undefined
\nomenFesti{Feria III \hebdomada{}}
\else
\titulus
\fi
\end{titulusOfficii}

\vfill

\begin{center}
%Ad usum et secundum consuetudines chori \guillemotright{}Conventus Choralis\guillemotleft.

%Editio Sancti Wolfgangi \annusEditionis
\end{center}

\scriptura{}

\pars{}

\pagebreak

\renewcommand{\headrulewidth}{0pt} % no horiz. rule at the header
\fancyhf{}
\pagestyle{fancy}

\cantusSineNeumas

\ifx\oratio\undefined
\ifx\laudb\undefined
\else
\newcommand{\oratio}{\pars{Oratio.}

\noindent Dómine Iesu Christe, lux vera, qui omnes hómines illúminas ad salútem, nobis, quǽsumus, concéde virtútem, ut ante te vias pacis et iustítiæ præparémus.

\noindent Qui vivis et regnas cum Deo Patre in unitáte Spíritus Sancti, Deus, per ómnia sǽcula sæculórum.

\noindent \Rbardot{} Amen.}
\fi
\fi

\hora{Ad Matutinum.} %%%%%%%%%%%%%%%%%%%%%%%%%%%%%%%%%%%%%%%%%%%%%%%%%%%%%

\vspace{2mm}

\cuminitiali{}{temporalia/dominelabiamea.gtex}

\vfill
%\pagebreak

\vspace{2mm}

\ifx\invitatorium\undefined
\ifx\matuac\undefined
\else
\pars{Invitatorium.} \scriptura{Ps. 94, 1; Psalmus 94; \textbf{H451}}

\vspace{-6mm}

\antiphona{VI}{temporalia/inv-jubilemusdeo.gtex}
\fi
\ifx\matubd\undefined
\else
\pars{Invitatorium.} \scriptura{Cantor; Psalmus 94; \textbf{H449}}

\vspace{-6mm}

\antiphona{E}{temporalia/inv-regemmagnum.gtex}
\fi
\else
\invitatorium
\fi

\vfill
\pagebreak

\ifx\hymnusmatutinum\undefined
\ifx\matuac\undefined
\else
\pars{Hymnus}

\cuminitiali{IV}{temporalia/hym-SomnoRefectis.gtex}
\fi
\ifx\matubd\undefined
\else
\pars{Hymnus.} \scriptura{Gregorius Magnus (\olddag{} 604)}

{
\grechangedim{interwordspacetext}{0.10 cm plus 0.15 cm minus 0.05 cm}{scalable}%
\antiphona{I}{temporalia/hym-NocteSurgentes.gtex}
\grechangedim{interwordspacetext}{0.22 cm plus 0.15 cm minus 0.05 cm}{scalable}%
}
\fi
\else
\hymnusmatutinum
\fi

\vspace{-3mm}

\vfill
\pagebreak

\ifx\matub\undefined
\else
% MAT B
\pars{Psalmus 1.} \scriptura{Ps. 36, 5; \textbf{H93}}

\vspace{-4mm}

\antiphona{VI F}{temporalia/ant-reveladomino.gtex}

%\vspace{-2mm}

\scriptura{Ps. 36, 1-11}

%\vspace{-2mm}

\initiumpsalmi{temporalia/ps36i_xi-initium-vi-F-auto.gtex}

\input{temporalia/ps36i_xi-vi-F.tex} \Abardot{}

\vfill
\pagebreak

\pars{Psalmus 2.}

\vspace{-4mm}

\antiphona{II D}{temporalia/ant-iuniorfui.gtex}

\vspace{-2mm}

\scriptura{Ps. 36, 12-29}

\vspace{-2mm}

\initiumpsalmi{temporalia/ps36xii_xxix-initium-ii-D-auto.gtex}

\input{temporalia/ps36xii_xxix-ii-D.tex}

\vfill

\antiphona{}{temporalia/ant-iuniorfui.gtex}

\vfill
\pagebreak

\pars{Psalmus 3.} \scriptura{Ps. 36, 3}

\vspace{-4mm}

\antiphona{VI F}{temporalia/ant-speraindomino.gtex}

%\vspace{-2mm}

\scriptura{Ps. 36, 30-40}

%\vspace{-2mm}

\initiumpsalmi{temporalia/ps36iii-initium-vi-F-auto.gtex}

\input{temporalia/ps36iii-vi-F.tex} \Abardot{}

\vfill
\pagebreak
\fi
\ifx\matuc\undefined
\else
% MAT C
\pars{Psalmus 1.} \scriptura{Ps. 67, 2}

\vspace{-4mm}

\antiphona{VII a}{temporalia/ant-exsurgatdeus.gtex}

%\vspace{-2mm}

\scriptura{Ps. 67, 2-11}

\initiumpsalmi{temporalia/ps67i-initium-vii-a-auto.gtex}

\input{temporalia/ps67i-vii-a.tex} \Abardot{}

\vfill
\pagebreak

\pars{Psalmus 2.}

\vspace{-4mm}

\antiphona{I f}{temporalia/ant-deusnosterdeussalvos.gtex}

%\vspace{-2mm}

\scriptura{Ps. 67, 12-24}

%\vspace{-2mm}

\initiumpsalmi{temporalia/ps67ii-initium-i-f-auto.gtex}

\input{temporalia/ps67ii-i-f.tex} \Abardot{}

\vfill
\pagebreak

\pars{Psalmus 3.} \scriptura{Ps. 67, 27; \textbf{H96}}

\vspace{-4mm}

\antiphona{D}{temporalia/ant-inecclesiis.gtex}

%\vspace{-2mm}

\scriptura{Ps. 67, 25-36}

\initiumpsalmi{temporalia/ps67iii-initium-d-g2-auto.gtex}

\input{temporalia/ps67iii-d-g2.tex} \Abardot{}

\vfill
\pagebreak
\fi

\pars{Versus.}

\ifx\matversus\undefined
\ifx\matub\undefined
\else
\noindent \Vbardot{} Bonitátem et prudéntiam et sciéntiam doce me.

\noindent \Rbardot{} Quia præcéptis tuis crédidi.
\fi
\ifx\matuc\undefined
\else
\noindent \Vbardot{} Audiam quid loquátur Dóminus Deus.

\noindent \Rbardot{} Loquétur pacem ad plebem suam.
\fi
\else
\matversus
\fi

\vspace{5mm}

\sineinitiali{temporalia/oratiodominica-mat.gtex}

\vspace{5mm}

\pars{Absolutio.}

\cuminitiali{}{temporalia/absolutio-ipsius.gtex}

\vfill
\pagebreak

\cuminitiali{}{temporalia/benedictio-solemn-deus.gtex}

\vspace{7mm}

\lectioi

\noindent \Vbardot{} Tu autem, Dómine, miserére nobis.
\noindent \Rbardot{} Deo grátias.

\vfill
\pagebreak

\responsoriumi

\vfill
\pagebreak

\cuminitiali{}{temporalia/benedictio-solemn-christus.gtex}

\vspace{7mm}

\lectioii

\noindent \Vbardot{} Tu autem, Dómine, miserére nobis.
\noindent \Rbardot{} Deo grátias.

\vfill
\pagebreak

\responsoriumii

\vfill
\pagebreak

\cuminitiali{}{temporalia/benedictio-solemn-ignem.gtex}

\vspace{7mm}

\lectioiii

\noindent \Vbardot{} Tu autem, Dómine, miserére nobis.
\noindent \Rbardot{} Deo grátias.

\vfill
\pagebreak

\responsoriumiii

\vfill
\pagebreak

\rubrica{Reliqua omittuntur, nisi Laudes separandæ sint.}

\sineinitiali{temporalia/domineexaudi.gtex}

\vfill

\oratio

\vfill

\noindent \Vbardot{} Dómine, exáudi oratiónem meam.
\Rbardot{} Et clamor meus ad te véniat.

\vfill

\noindent \Vbardot{} Benedicámus Dómino.
\noindent \Rbardot{} Deo grátias.

\vfill

\noindent \Vbardot{} Fidélium ánimæ per misericórdiam Dei requiéscant in pace.
\Rbardot{} Amen.

\vfill
\pagebreak

\hora{Ad Laudes.} %%%%%%%%%%%%%%%%%%%%%%%%%%%%%%%%%%%%%%%%%%%%%%%%%%%%%

\cantusSineNeumas

\vspace{0.5cm}
\grechangedim{interwordspacetext}{0.18 cm plus 0.15 cm minus 0.05 cm}{scalable}%
\cuminitiali{}{temporalia/deusinadiutorium-communis.gtex}
\grechangedim{interwordspacetext}{0.22 cm plus 0.15 cm minus 0.05 cm}{scalable}%

\vfill
\pagebreak

\ifx\hymnuslaudes\undefined
\ifx\laudac\undefined
\else
\pars{Hymnus} \scriptura{Ambrosius (\olddag{} 397)}

\cuminitiali{I}{temporalia/hym-SplendorPaternae-hiemalis.gtex}
\fi
\ifx\laudbd\undefined
\else
\pars{Hymnus}

\grechangedim{interwordspacetext}{0.16 cm plus 0.15 cm minus 0.05 cm}{scalable}%
\cuminitiali{IV}{temporalia/hym-AEterneLucis.gtex}
\grechangedim{interwordspacetext}{0.22 cm plus 0.15 cm minus 0.05 cm}{scalable}%
\vspace{-3mm}
\fi
\else
\hymnuslaudes
\fi

\vfill
\pagebreak

\ifx\laudb\undefined
\else
\pars{Psalmus 1.} \scriptura{Ps. 42, 5; \textbf{H95}}

\vspace{-4mm}

\antiphona{VI F}{temporalia/ant-salutarevultusmei.gtex}

\scriptura{Psalmus 42.}

\initiumpsalmi{temporalia/ps42-initium-vi-F-auto.gtex}

\input{temporalia/ps42-vi-F.tex} \Abardot{}

\vfill
\pagebreak

\pars{Psalmus 2.} \scriptura{Is. 38, 20; \textbf{H95}}

\vspace{-7mm}

\antiphona{E}{temporalia/ant-cunctisdiebus.gtex}

\vspace{-4mm}

\scriptura{Canticum Ezechiæ, Is. 38, 10-20}

\vspace{-3mm}

\initiumpsalmi{temporalia/ezechiae-initium-e-auto.gtex}

\input{temporalia/ezechiae-e.tex} \Abardot{}

\vfill
\pagebreak

\pars{Psalmus 3.} \scriptura{Ps. 64, 2; \textbf{H96}}

\vspace{-4mm}

\antiphona{VIII a}{temporalia/ant-tedecet.gtex}

\vspace{-2mm}

\scriptura{Psalmus 64.}

\vspace{-2mm}

\initiumpsalmi{temporalia/ps64-initium-viii-A-auto.gtex}

\input{temporalia/ps64-viii-A.tex} \Abardot{}

\vfill
\pagebreak
\fi
\ifx\laudc\undefined
\else
\pars{Psalmus 1.} \scriptura{Ps. 83, 5}

\vspace{-4mm}

\antiphona{VIII G}{temporalia/ant-beatiquihabitant.gtex}

\vspace{-2mm}

\scriptura{Psalmus 84.}

\vspace{-2mm}

\initiumpsalmi{temporalia/ps84-initium-viii-G-auto.gtex}

\input{temporalia/ps84-viii-G.tex} \Abardot{}

\vfill
\pagebreak

\pars{Psalmus 2.}

\vspace{-4mm}

\antiphona{VII d}{temporalia/ant-denoctespiritusmeus.gtex}

\vspace{-2mm}

\scriptura{Canticum Isaiæ, Is. 26, 1-12}

\vspace{-2mm}

\initiumpsalmi{temporalia/isaiae3-initium-vii-d.gtex}

\input{temporalia/isaiae3-vii-d.tex} \Abardot{}

\vfill
\pagebreak

\pars{Psalmus 3.} \scriptura{Ps. 66, 2}

\vspace{-4mm}

\antiphona{E}{temporalia/ant-illuminadomine.gtex}

%\vspace{-2mm}

\scriptura{Psalmus 66.}

%\vspace{-2mm}

\initiumpsalmi{temporalia/ps66-initium-e.gtex}

\input{temporalia/ps66-e.tex} \Abardot{}

\vfill
\pagebreak
\fi

\ifx\lectiobrevis\undefined
\ifx\laudb\undefined
\else
\pars{Lectio Brevis.} \scriptura{1 Th. 5, 4-5}

\noindent Vos, fratres, non estis in ténebris, ut vos dies ille tamquam fur comprehéndat; omnes enim vos fílii lucis estis et fílii diéi. Non sumus noctis neque tenebrárum.
\fi
\ifx\laudc\undefined
\else
\pars{Lectio Brevis.} \scriptura{1 Io. 4, 14-15}

\noindent Nos vídimus et testificámur quóniam Pater misit Fílium salvatórem mundi. Quisque conféssus fúerit: Iesus est Fílius Dei, Deus in ipso manet, et ipse in Deo.
\fi
\else
\lectiobrevis
\fi

\vfill

\ifx\responsoriumbreve\undefined
\ifx\laudac\undefined
\else
\pars{Responsorium breve.}

\cuminitiali{VI}{temporalia/resp-benedictusdominus.gtex}
\fi
\ifx\laudbd\undefined
\else
\pars{Responsorium breve.} \scriptura{Ps. 118, 149.147}

\cuminitiali{VI}{temporalia/resp-vocemmeamaudi.gtex}
\fi
\else
\responsoriumbreve
\fi

\vfill
\pagebreak

\ifx\benedictus\undefined
\ifx\laudbd\undefined
\else
\pars{Canticum Zachariæ.} \scriptura{Lc. 1, 71; \textbf{H423}}

\vspace{-5mm}

{
\grechangedim{interwordspacetext}{0.18 cm plus 0.15 cm minus 0.05 cm}{scalable}%
\antiphona{I g\textsuperscript{5}}{temporalia/ant-demanuomnium.gtex}
\grechangedim{interwordspacetext}{0.22 cm plus 0.15 cm minus 0.05 cm}{scalable}%
}

%\vspace{-3mm}

\scriptura{Lc. 1, 68-79}

%\vspace{-1mm}

\initiumpsalmi{temporalia/benedictus-initium-i-g5-auto.gtex}

\input{temporalia/benedictus-i-g5.tex} \Abardot{}
\fi
\else
\benedictus
\fi

\vspace{-1cm}

\vfill
\pagebreak

\pars{Preces.}

\sineinitiali{}{temporalia/tonusprecum.gtex}

\ifx\preces\undefined
\ifx\laudb\undefined
\else
\noindent Salvatóri nostro benedicámus, qui sua resurrectióne mundum clarificávit, \gredagger{} et humíliter invocémus eum dicéntes:

\Rbardot{} Salva nos, Dómine, in sémita tua.

\noindent Resurrectiónem tuam, Dómine Iesu, oratióne cólimus matutína, \gredagger{} spes glóriæ tuæ diem nostrum illúminet.

\Rbardot{} Salva nos, Dómine, in sémita tua.

\noindent Súscipe, Dómine, vota et propósita nostra, \gredagger{} tamquam diéi nostri primítias.

\Rbardot{} Salva nos, Dómine, in sémita tua.

\noindent Tríbue in dilectióne tua nos hódie profícere, \gredagger{} ut ómnia in nostrum omniúmque bonum cooperéntur.

\Rbardot{} Salva nos, Dómine, in sémita tua.

\noindent Da, Dómine, sic lucére lucem nostram coram homínibus, \gredagger{} ut vídeant ópera nostra bona et Patrem gloríficent.

\Rbardot{} Salva nos, Dómine, in sémita tua.
\fi
\else
\preces
\fi

\vfill

\pars{Oratio Dominica.}

\cuminitiali{}{temporalia/oratiodominicaalt.gtex}

\vfill
\pagebreak

\rubrica{vel:}

\pars{Supplicatio Litaniæ.}

\cuminitiali{}{temporalia/supplicatiolitaniae.gtex}

\vfill

\pars{Oratio Dominica.}

\cuminitiali{}{temporalia/oratiodominica.gtex}

\vfill
\pagebreak

% Oratio. %%%
\oratio

\vspace{-1mm}

\vfill

\rubrica{Hebdomadarius dicit Dominus vobiscum, vel, absente sacerdote vel diacono, sic concluditur:}

\vspace{2mm}

\antiphona{C}{temporalia/dominusnosbenedicat.gtex}

\rubrica{Postea cantatur a cantore:}

\vspace{2mm}

\cuminitiali{IV}{temporalia/benedicamus-feria-laudes.gtex}

\vspace{1mm}

\vfill
\pagebreak

\end{document}

