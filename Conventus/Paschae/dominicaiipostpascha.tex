\newcommand{\titulus}{\nomenFesti{Dominica III Paschæ (II post Pascha).}}
\newcommand{\tedeumsimplex}{Simplex}
\newcommand{\oratioLaudes}{\cuminitiali{}{temporalia/oratiod3.gtex}}
\newcommand{\oratio}{\pars{Oratio.}

\noindent Semper exsúltet pópulus tuus, Deus, renováta ánimæ iuventúte, ut, qui nunc lætátur in adoptiónis se glóriam restitútum, resurrectiónis diem spe certæ gratulatiónis exspéctet.

\pars{Pro pace in universo mundo.} \scriptura{Sir. 50, 25; 2 Esdr. 4, 20; \textbf{H416}}

\vspace{-4mm}

\antiphona{II D}{temporalia/ant-dapacemdomine.gtex}

\vfill

\noindent Deus, a quo sancta desidéria, recta consília et iusta sunt ópera: da servis tuis illam, quam mundus dare non potest, pacem; ut et corda nostra mandátis tuis dédita, et hóstium subláta formídine, témpora sint tua protectióne tranquílla.

\noindent Per Dóminum nostrum Iesum Christum, Fílium tuum, qui tecum vivit et regnat in unitáte Spíritus Sancti, Deus, per ómnia sǽcula sæculórum.

\noindent \Rbardot{} Amen.}
\newcommand{\invitatorium}{\pars{Invitatorium.} \scriptura{Lc. 24, 34; Psalmus 94; \textbf{H232}}

\vspace{-6mm}

\antiphona{VI}{temporalia/inv-surrexitdominusvere.gtex}}
\newcommand{\hymnusmatutinum}{\pars{Hymnus.}

\vspace{-5mm}

\antiphona{III}{temporalia/hym-HicEstDies.gtex}}
\newcommand{\nocturnoi}{\pars{Psalmus 1.}

\vspace{-4mm}

\antiphona{II D}{temporalia/ant-alleluia-bv21-n1.gtex}

%\vspace{-2mm}

\scriptura{Ps. 1}

%\vspace{-2mm}

%\initiumpsalmi{temporalia/ps1-initium-ii-D-auto.gtex}
\initiumpsalmi{temporalia/ps1-initium-ii-D.gtex}

%\input{temporalia/ps1-ii-D.tex} \Abardot{}
\input{temporalia/ps1-viii-G.tex} \Abardot{}

\vfill
\pagebreak

\pars{Psalmus 2.}

\vspace{-4mm}

\antiphona{VIII c}{temporalia/ant-alleluia-bv21-n2.gtex}

%\vspace{-2mm}

\scriptura{Ps. 2}

\initiumpsalmi{temporalia/ps2-initium-viii-c-auto.gtex}

\input{temporalia/ps2-viii-c.tex} \Abardot{}

\vfill
\pagebreak

\pars{Psalmus 3.}

%\vspace{-4mm}

\antiphona{VIII G\textsuperscript{3}}{temporalia/ant-alleluia-bv21-n3.gtex}

%\vspace{-2mm}

\scriptura{Ps. 3}

%\initiumpsalmi{temporalia/ps3-initium-viii-G2-auto.gtex}
\initiumpsalmi{temporalia/ps3-initium-viii-G3.gtex}

\input{temporalia/ps3-viii-G2.tex} \Abardot{}

\vfill
\pagebreak}
\newcommand{\nocturnoii}{\vspace{-4mm}

\pars{Psalmus 4.} \scriptura{Mt. 28, 2; Mc. 16, 4; \textbf{H230}}

\vspace{-4mm}

\antiphona{VII a}{temporalia/ant-alleluialapisrevolutus.gtex}

%\vspace{-2mm}

\scriptura{Ps. 144, 1-9}

%\vspace{-2mm}

\initiumpsalmi{temporalia/ps144i-initium-vii-a-auto.gtex}

\input{temporalia/ps144i-vii-a.tex} \Abardot{}

\vfill
\pagebreak

\pars{Psalmus 5.} \scriptura{Io. 20, 15.17}

\vspace{-4mm}

\antiphona{VIII G\textsuperscript{2}}{temporalia/ant-marianoliiamflere.gtex}

%\vspace{-2mm}

\scriptura{Ps. 144, 10-13}

\initiumpsalmi{temporalia/ps144x_xiii-initium-viii-G5-auto.gtex}

\input{temporalia/ps144x_xiii-viii-G5.tex} \Abardot{}

\vfill
\pagebreak

\pars{Psalmus 6.} \scriptura{Io. 20, 18; \textbf{H238}}

\vspace{-4mm}

\antiphona{VII a}{temporalia/ant-venitmarianuntians.gtex}

%\vspace{-4mm}

\scriptura{Ps. 144, 14-21}

%\vspace{-2mm}

\initiumpsalmi{temporalia/ps144xiv_xxi-initium-vii-a-auto.gtex}

%\vspace{-1.5mm}

\input{temporalia/ps144xiv_xxi-vii-a.tex} \Abardot{}

\vfill
\pagebreak}
\newcommand{\nocturnoiii}{\pars{Cantica.}

\vspace{-4mm}

\antiphona{IV E}{temporalia/ant-veniteomnesadoremus.gtex}

%\vspace{-2mm}

\scriptura{Canticum Isaiæ, Is. 63, 1-5}

%\vspace{-2mm}

\initiumpsalmi{temporalia/isaiae12-initium-iv-E-auto.gtex}

\input{temporalia/isaiae12-iv-E.tex} \hfill \rubrica{Hic non dicitur antiphona.}

\vfill
\pagebreak

\scriptura{Canticum Oseæ, Os. 6, 1-6}

%\vspace{-2mm}

\initiumpsalmi{temporalia/oseae-initium-iv-E-auto.gtex}

\input{temporalia/oseae-iv-E.tex}

\vfill
\pagebreak

\scriptura{Canticum Sophoniæ, Soph. 3, 8-13}

%\vspace{-2mm}

\initiumpsalmi{temporalia/sophoniae-initium-iv-E-auto.gtex}

\input{temporalia/sophoniae-iv-E.tex}

\vfill
\pagebreak

\antiphona{}{temporalia/ant-veniteomnesadoremus.gtex}

\vfill
\pagebreak}
\newcommand{\matversusi}{\pars{Versus.}

\noindent \Vbardot{} Surréxit Dóminus de sepúlcro, allelúia.

\noindent \Rbardot{} Qui pro nobis pepéndit in ligno, allelúia.}
\newcommand{\matversusii}{\pars{Versus.}

\noindent \Vbardot{} Surréxit Dóminus vere, allelúia.

\noindent \Rbardot{} Et appáruit Simóni, allelúia.}
\newcommand{\lectioi}{\pars{Lectio I.} \scriptura{Ap. 6, 1-4}

\noindent De libro Apocalýpsis beáti Ioánnis apóstoli.

\noindent Ego Ioánnes vidi, cum aperuísset Agnus unum de septem sigíllis, et audívi unum de quáttuor animálibus dicens tamquam vox tonítrui: «Veni». Et vidi: et ecce equus albus; et, qui sedébat super illum, habébat arcum, et data est ei coróna, et exívit vincens et ut vínceret.

\noindent Et cum aperuísset sigíllum secúndum, audívi secúndum ánimal dicens: «Veni». Et exívit álius equus rufus; et, qui sedébat super illum, datum est ei, ut súmeret pacem de terra, et ut ínvicem se interfíciant, et datus est illi gládius magnus.}
\newcommand{\responsoriumi}{\pars{Responsorium 1.} \scriptura{\Rbardot{} Ap. 5, 9 \Vbardot{} ibid. 5, 10; \textbf{H247}}

\vspace{-5mm}

\responsorium{VII}{temporalia/resp-dignusesdomine-CROCHU.gtex}{}}
\newcommand{\lectioii}{\pars{Lectio II.} \scriptura{Ap. 6, 5-8}

\noindent Et cum aperuísset sigíllum tértium, audívi tértium ánimal dicens: «Veni». Et vidi: et ecce equus niger; et, qui sedébat super eum, habébat statéram in manu sua. Et audívi tamquam vocem in médio quáttuor animálium dicéntem: «Bilíbris trítici denário, et tres bilíbres hórdei denário; et óleum et vinum ne lǽseris».

\noindent Et cum aperuísset sigíllum quartum, audívi vocem quarti animális dicéntis: «Veni» Et vidi: et ecce equus pállidus; et, qui sedébat désuper, nomen illi Mors, et Inférnus sequebátur eum, et data est illis potéstas super quartam partem terræ, interfícere gládio et fame et morte et a béstiis terræ.}
\newcommand{\responsoriumii}{\pars{Responsorium 2.} \scriptura{\Rbardot{} Eccli. 24, 23.26 \Vbardot{} ibid. 18; \textbf{H247}}

\vspace{-5mm}

\responsorium{III}{temporalia/resp-egosicutvitis-CROCHU.gtex}{}}
\newcommand{\lectioiii}{\pars{Lectio III.} \scriptura{Ap. 6, 9-12}

\noindent Et cum aperuísset quintum sigíllum, vidi subtus altáre ánimas interfectórum propter verbum Dei et propter testimónium, quod habébant. Et clamavérunt voce magna dicéntes: «Usquequo, Dómine, sanctus et verus, non iúdicas et víndicas sánguinem nostrum de his, qui hábitant in terra?».

\noindent Et datæ sunt illis síngulæ stolæ albæ et dictum est illis, ut requiéscant tempus adhuc módicum, donec impleántur et consérvi eórum et fratres eórum, qui interficiéndi sunt sicut et illi.

\noindent Et vidi, cum aperuísset sigíllum sextum, et terræmótus factus est magnus, et sol factus est niger tamquam saccus cilícinus, et luna tota facta est sicut sanguis, et stellæ cæli cecidérunt in terram, sicut ficus mittit grossos suos, cum vento magno movétur, et cælum recéssit sicut liber involútus, et omnis mons et ínsula de locis suis motæ sunt. Et reges terræ et magnátes et tribúni et dívites et fortes et omnis servus et liber abscondérunt se in spelúncis et in petris móntium; \emph{et dicunt móntibus et petris: «Cádite super nos et abscóndite nos} a fácie sedéntis super thronum et ab ira Agni, quóniam venit dies magnus iræ ipsórum, et quis póterit stare?».}
\newcommand{\responsoriumiii}{\pars{Responsorium 3.} \scriptura{\Rbardot{} Ap. 19, 6 \& Ap. 12, 10 \Vbardot{} ibid. 14, 6-7; \textbf{H247}}

\vspace{-5mm}

\responsorium{VII}{temporalia/resp-audivivocemincaelotamquam-CROCHU-cumdox.gtex}{}}
\newcommand{\lectioiv}{\pars{Lectio IV.} \scriptura{Cap. 66-67: PG 6, 427-431}

\noindent Ex Apológia prima sancti Iustíni mártyris pro Christiánis.

\noindent Eucharístiæ némini álii licet esse párticeps, nisi qui credat vera esse quæ docémus, atque illo ad remissiónem peccatórum et regeneratiónem lavácro ablútus fúerit, et ita vivat ut Christus trádidit.

\noindent Neque enim ut commúnem panem, neque ut commúnem potum ista súmimus; sed quemádmodum per Verbum Dei caro factus est Iesus Christus salvátor noster et carnem et sánguinem hábuit nostræ salútis causa; sic étiam illam, in qua per precem, ipsíus verba continéntem, grátiæ actæ sunt, alimóniam, ex qua sanguis et carnes nostræ per mutatiónem alúntur, incarnáti illíus Iesu et carnem et sánguinem esse edócti sumus.

\noindent Nam Apóstoli in commentáriis suis, quæ vocántur Evangélia, ita sibi mandásse Iesum tradidérunt: eum scílicet, accépto pane, cum grátias egísset, dixísse: Hoc fácite in meam commemoratiónem. Hoc est corpus meum; et póculo simíliter accépto, actísque grátiis, dixísse: Hic est sanguis meus, ipsísque solis tradidísse. Ex illo témpore hæc semper nobis ínvicem in memóriam revocámus; et qui habémus, indigéntibus ómnibus subvénimus, et semper una sumus. Atque in ómnibus oblatiónibus laudámus creatórem ómnium per Fílium eius Iesum Christum et per Spíritum Sanctum.}
\newcommand{\responsoriumiv}{\pars{Responsorium 4.} \scriptura{\Rbardot{} Ap. 21, 9-10; Is. 61, 10 \Vbardot{} Sir. 32, 18; \textbf{H247}}

\vspace{-5mm}

\responsorium{II}{temporalia/resp-locutusestadmeunus-CROCHU.gtex}{}}
\newcommand{\lectiov}{\pars{Lectio V.}

\noindent Ac Solis, ut dícitur, die, ómnium sive urbes sive agros incoléntium in eúndem locum fit convéntus, et commentária Apostolórum aut scripta prophetárum legúntur, quoad licet per tempus.

\noindent Deínde, ubi lector désiit, is qui præest admonitiónem verbis et adhortatiónem ad res tam præcláras imitándas súscipit.

\noindent Póstea omnes simul consúrgimus, et preces emíttimus; atque, ut iam díximus, ubi desíimus precári, panis affértur et vinum et aqua; et qui præest, preces et gratiárum actiónes totis víribus emíttit, et pópulus acclámat Amen, et eórum, in quibus grátiæ actæ sunt, distribútio fit et communicátio unicuíque præséntium, et abséntibus per diáconos míttitur.}
\newcommand{\responsoriumv}{\pars{Responsorium 5.} \scriptura{\Rbardot{} Ap. 14, 7; \Vbardot{} ibid., 6; \textbf{H248}}

\vspace{-5mm}

\responsorium{I}{temporalia/resp-audivivocemincaeloangelorum-CROCHU.gtex}{}}
\newcommand{\lectiovi}{\pars{Lectio VI.}

\noindent Qui abúndant et volunt, suo arbítrio, quod quisque vult, largiúntur, et quod collígitur apud eum, qui præest, depónitur, ac ipse súbvenit pupíllis et víduis, et iis qui vel ob morbum vel áliam ob causam egent, tum étiam iis qui in vínculis sunt et adveniéntibus péregre hospítibus; uno verbo ómnium indigéntium curam súscipit.

\noindent Die autem Solis omnes simul convenímus, tum quia prima hæc dies est, qua Deus, cum ténebras et matériam vertísset, mundum creávit, tum quia Iesus Christus salvátor noster eádem die ex mórtuis resurréxit. Prídie enim Satúrni eum crucifixérunt, et postrídie eiúsdem diéi, id est Solis die, Apóstolis suis et discípulis visus ea dócuit, quæ vobis quoque consideránda tradídimus.}
\newcommand{\responsoriumvi}{\pars{Responsorium 6.} \scriptura{\Rbardot{} Cf. 1 Chr. 13, 8 \Vbardot{} Ps. 98, 6; \textbf{H248}}

\vspace{-5mm}

\responsorium{VI}{temporalia/resp-decantabatpopulus-CROCHU-cumdox.gtex}{}}
\newcommand{\evangelium}{\pars{Versus.}

\noindent \Vbardot{} Gavísi sunt discípuli, allelúia.

\noindent \Rbardot{} Viso Dómino, allelúia.

\vspace{5mm}

\sineinitiali{temporalia/oratiodominica-mat.gtex}

\vspace{5mm}

\pars{Absolutio.}

\cuminitiali{}{temporalia/absolutio-avinculis.gtex}

\vfill
\pagebreak

\cuminitiali{}{temporalia/benedictio-solemn-evangelica.gtex}

\vspace{7mm}

\pars{Evangelium} \scriptura{Lc. 24, 35-38}

\noindent Léctio sancti Evangélii secúndum Lucam.

\noindent In illo témpore: Narrábant discípuli, quæ gesta erant in via, et quómodo cognovérunt Iesum in fractióne panis. Dum hæc autem loquúntur, ipse stetit in médio eórum et dicit eis: «Pax vobis!». Conturbáti vero et contérriti existimábant se spíritum vidére.

\noindent Et dixit eis: «Quid turbáti estis, et quare cogitatiónes ascéndunt in corda vestra? Vidéte manus meas et pedes meos, quia ipse ego sum! Palpáte me et vidéte, quia spíritus carnem et ossa non habet, sicut me vidétis habére». Et cum hoc dixísset, osténdit eis manus et pedes.

\noindent Adhuc autem illis non credéntibus præ gáudio et mirántibus, dixit eis: «Habétis hic áliquid, quod manducétur?». At illi obtulérunt ei partem piscis assi. Et sumens, coram eis manducávit.

\noindent Et dixit ad eos: «Hæc sunt verba, quæ locútus sum ad vos, cum adhuc essem vobíscum, quóniam necésse est impléri ómnia, quæ scripta sunt in Lege Móysis et Prophétis et Psalmis de me». Tunc apéruit illis sensum, ut intellégerent Scriptúras.

\noindent Et dixit eis: «Sic scriptum est, Christum pati et resúrgere a mórtuis die tértia, et prædicári in nómine eius pæniténtiam in remissiónem peccatórum in omnes gentes, incipiéntibus ab Ierúsalem. Vos estis testes horum».

%\rubrica{Hic non dicitur \textnormal{Tu autem Domine}.}

\vspace{5mm}

\scriptura{Sermo 242, 1}

\noindent Ex Sermónibus sancti Augustíni epíscopi.

\noindent Diébus his sanctis resurrectióni Dómini dedicátis, quantum donánte ipso póssumus, de carnis resurrectióne tractémus. Hæc enim fides est nostra, hoc donum in Dómini nostri Iesu Christi nobis carne promíssum est, et in ipso præcéssit exémplum. Vóluit enim nobis quod promísit in fine, non solum prænuntiáre, sed étiam demonstráre. Illi quidem qui tunc fuérunt cum illo, vidérunt, et cum expavéscerent, et spíritum se vidére créderent, soliditátem córporis tenuérunt.

\noindent Locútus est enim non solum verbis ad aures eórum, sed étiam spécie ad óculos eórum: parúmque erat se præbére cernéndum, nisi étiam offérret pertractándum atque palpándum. {\color{gray} Ait enim: \emph{Quid turbáti estis, et cogitatiónes ascéndunt in cor vestrum?} Putavérunt enim se spíritum vidére. \emph{Quid turbáti estis}, inquit, \emph{et cogitatiónes ascéndunt in cor vestrum? Vidéte manus meas et pedes meos: palpáte, et vidéte, quia spíritus ossa et carnem non habet, sicut me vidétis habére.}}

\noindent Resurréxit Christus: absolúta est res. Corpus erat, caro erat: pepéndit in cruce, emísit ánimam, pósita est caro in sepúlcro.

\noindent Exhíbuit illam vivam, qui vivébat in illa. Quare mirámur? Quare non crédimus? Deus est qui fecit: consídera auctórem et tolle dubitatiónem.

\vfill
\pagebreak

\pars{Responsorium 7.} \scriptura{\Rbardot{} Io. 20, 19; \textbf{H232}}

\vspace{-5mm}

\responsorium{VII}{temporalia/resp-surgensjesus-CROCHU-cumdox.gtex}{}

\rubrica{vel ad libitum:}

\vspace{3mm}

\pars{Responsorium 7.} \scriptura{\Rbardot{} Io. 10, 11; \textbf{H237}}

\vspace{-5mm}

\responsorium{I}{temporalia/resp-surrexitpastor-CROCHU-cumdox.gtex}{}

\vfill
\pagebreak}
\newcommand{\laudes}{\pars{Hymnus}

\cuminitiali{VIII}{temporalia/hym-AuroraLucis-alt.gtex}

\vfill
\pagebreak

\pars{Psalmus 1.} \scriptura{Cf. Ps. 92, 1.2}

\vspace{-4mm}

\antiphona{III a}{temporalia/ant-regnavitdominuspraecinctus.gtex}

%\vspace{-2mm}

\scriptura{Psalmus 92}

%\vspace{-2mm}

\initiumpsalmi{temporalia/ps92-initium-iii-a-auto.gtex}

%\vspace{-1.5mm}

\input{temporalia/ps92-iii-a.tex} \Abardot{}

\vfill
\pagebreak

\pars{Psalmus 2.} \scriptura{Cf. Dn. 3, 74.76; \textbf{H134}}

\vspace{-4mm}

\antiphona{III a trans.}{temporalia/ant-benedicatterradomino.gtex}

%\vspace{-2mm}

\scriptura{Canticum trium puerorum, Dan. 3, 57-88 et 56}

\initiumpsalmi{temporalia/dan3-initium-iii-a-trans2.gtex}

\input{temporalia/dan3-iii-a-sinedox.tex}

\rubrica{Hic non dicitur Gloria Patri, neque Amen.}

\vfill

\antiphona{}{temporalia/ant-benedicatterradomino.gtex}

\vfill
\pagebreak

\pars{Psalmus 3.}

\vspace{-4mm}

\antiphona{I d}{temporalia/ant-deresurrectionetuachriste.gtex}

\vspace{-2mm}

\scriptura{Psalmus 148.}

%\vspace{-2mm}

\initiumpsalmi{temporalia/ps148-initium-i-d-auto.gtex}

%\vspace{-1.5mm}

\input{temporalia/ps148-i-d.tex}

\vfill

\antiphona{}{temporalia/ant-deresurrectionetuachriste.gtex}

\vfill
\pagebreak}
\newcommand{\lectiobrevis}{\pars{Lectio brevis.} \scriptura{Ac. 10, 40-43}

\noindent Deus suscitávit Iesum tértia die et dedit eum maniféstum fíeri, non omni pópulo sed téstibus præordinátis a Deo, nobis, qui manducávimus et bíbimus cum illo postquam resurréxit a mórtuis; et præcépit nobis prædicáre pópulo et testificári quia ipse est, qui constitútus est a Deo iudex vivórum et mortuórum. Huic omnes prophétæ testimónium pérhibent remissiónem peccatórum accípere per nomen eius omnes, qui credunt in eum.}
\newcommand{\responsoriumbreve}{\pars{Responsorium breve.}

\cuminitiali{VI}{temporalia/resp-christefilideivivi-tp.gtex}}
\newcommand{\benedictus}{\pars{Canticum Zachariæ.} \scriptura{Lc. 24, 39; \textbf{H235}}

\vspace{-4mm}

{
\grechangedim{interwordspacetext}{0.18 cm plus 0.15 cm minus 0.05 cm}{scalable}%
\antiphona{VIII G\textsuperscript{2}}{temporalia/ant-spirituscarnemetossa.gtex}
\grechangedim{interwordspacetext}{0.32 cm plus 0.15 cm minus 0.05 cm}{scalable}%
}

%\vspace{-2mm}

\scriptura{Lc. 1, 68-79}

%\vspace{-2mm}

\cantusSineNeumas
\initiumpsalmi{temporalia/benedictus-initium-viiisoll-G5-auto.gtex}

%\vspace{-1.5mm}

\input{temporalia/benedictus-viiisoll-G5.tex} \Abardot{}

\vspace{-10mm}}
\newcommand{\preces}{\noindent Christum, auctórem vitæ,~\gredagger{} quem Deus suscitávit quique nos suscitábit per virtútem suam,~\grestar{} orémus, clamántes:

\Rbardot{} Christe, vita nostra, salva nos.

\noindent Christe, lux fúlgida in ténebris splendens,~\gredagger{} vitæ princeps et mortálium sanctificátor,~\grestar{} hanc diem ad laudem tuam fac nos transígere.

\Rbardot{} Christe, vita nostra, salva nos.

\noindent Dómine, qui ambulásti in via passiónis et crucis,~\grestar{} concéde nobis ut, tecum patiéntes et moriéntes, tecum étiam resuscitémur.

\Rbardot{} Christe, vita nostra, salva nos.

\noindent Fili Patris, magíster et frater noster,~\gredagger{} qui nos regnum et sacerdótes Deo nostro constituísti,~\grestar{} præsta ut tibi sacrifícium laudis offerámus in gáudio.

\Rbardot{} Christe, vita nostra, salva nos.

\noindent Rex glóriæ, præclárum exspectámus diem manifestatiónis tuæ,~\grestar{} ut vultum tuum contemplémur et símiles tui efficiámur.

\Rbardot{} Christe, vita nostra, salva nos.}
\newcommand{\magnificatii}{\pars{Canticum B. Mariæ V.} \scriptura{Io. 1, 41}

\vspace{-6mm}

{
\grechangedim{interwordspacetext}{0.18 cm plus 0.15 cm minus 0.05 cm}{scalable}%
\antiphona{I d\textsuperscript{3}}{temporalia/ant-ambulansiesus.gtex}
\grechangedim{interwordspacetext}{0.22 cm plus 0.15 cm minus 0.05 cm}{scalable}%
}

\vspace{-1.5mm}

\scriptura{Lc. 1, 46-55}

\vspace{-2.5mm}

\cantusSineNeumas
\initiumpsalmi{temporalia/magnificat-initium-isoll-d3.gtex}

\vspace{-1.5mm}

\input{temporalia/magnificat-isoll-d3.tex} \Abardot{}}
%\newcommand{\hebdomada}{infra Hebdom. III Adventus.}
\newcommand{\oratioLaudes}{\cuminitiali{}{temporalia/oratio3vo.gtex}}
\newcommand{\responsoriumbreve}{\pars{Responsorium breve.} \scriptura{Is. 60, 2; \textbf{H20}}

\cuminitiali{IV}{temporalia/resp-superte.gtex}}

% LuaLaTeX

\documentclass[a4paper, twoside, 12pt]{article}
\usepackage[latin]{babel}
%\usepackage[landscape, left=3cm, right=1.5cm, top=2cm, bottom=1cm]{geometry} % okraje stranky
%\usepackage[landscape, a4paper, mag=1166, truedimen, left=2cm, right=1.5cm, top=1.6cm, bottom=0.95cm]{geometry} % okraje stranky
\usepackage[landscape, a4paper, mag=1400, truedimen, left=0.5cm, right=0.5cm, top=0.5cm, bottom=0.5cm]{geometry} % okraje stranky

\usepackage{fontspec}
\setmainfont[FeatureFile={junicode.fea}, Ligatures={Common, TeX}, RawFeature=+fixi]{Junicode}
%\setmainfont{Junicode}

% shortcut for Junicode without ligatures (for the Czech texts)
\newfontfamily\nlfont[FeatureFile={junicode.fea}, Ligatures={Common, TeX}, RawFeature=+fixi]{Junicode}

\usepackage{multicol}
\usepackage{color}
\usepackage{lettrine}
\usepackage{fancyhdr}

% usual packages loading:
\usepackage{luatextra}
\usepackage{graphicx} % support the \includegraphics command and options
\usepackage{gregoriotex} % for gregorio score inclusion
\usepackage{gregoriosyms}
\usepackage{wrapfig} % figures wrapped by the text
\usepackage{parcolumns}
\usepackage[contents={},opacity=1,scale=1,color=black]{background}
\usepackage{tikzpagenodes}
\usepackage{calc}
\usepackage{longtable}
\usetikzlibrary{calc}

\setlength{\headheight}{14.5pt}

\input{conventuscommune.tex} % Often used macros
%%%% Preklady jednotlivych zpevu (nektere se opakuji, a je dobre mit je
% vsechny na jedne hromade)

% HOURS ---

\newcommand{\trAntI}{\translatioCantus{Muž boží měl kožený toulec, pečlivě
zavázaný, jenž mu visel na šíji a~často se ho dotýkal.}}

\newcommand{\trAntII}{\translatioCantus{Klíč od~něho tak dobře střežil, že
dokud žil v~těle, nikdo z~jeho žáků nezvěděl, co je uvnitř.}}

\newcommand{\trAntIII}{\translatioCantus{Ale když se odebral z~tohoto
života, schránku otevřeli a~objevili v~ní žíněné roucho a~měděný řetěz
potřísněný krví.}}

\newcommand{\trAntIV}{\translatioCantus{A když prohlédli mistrovo tělo,
nalezli jeho tělo na čtyřech místech hluboce zbrázděno ranami od řetězu.}}

\newcommand{\trAntV}{\translatioCantus{Krev vytékající z~těch ran, místy
prostoupila i~žíněným rouchem.}}

\newcommand{\trCapituli}{\translatioCantus{
Miláčkovi Boha a~lidí,
Mojžíšovi požehnané paměti,~\gredagger{}
dopřál slávu rovnou slávě svatých~\grestar{}
učinil ho mocným na postrach nepřátelům
a~jeho slovy zastavil divy.}}

\newcommand{\trLectioBrevis}{\translatioCantus{
Pamatujte na své představené,
kteří vám hlásali Boží slovo.
Uvažte, jak oni skončili život, a~napodobujte jejich víru.
Ježíš Kristus je stejný včera i~dnes i~navěky.
Nenechte se svést věelijakými cizími naukami.}}

\newcommand{\trRespLaud}{\translatioCantus{Spravedlivého vodil Hospodin~\grestar{}
po přímých stezkách. \Vbardot{} A~ukázal mu Boží království.}}

\newcommand{\trRespLaudB}{\translatioCantus{Na tvých hradbách, Jeruzaléme,
ustanovil jsem strážné;~\grestar{}
budou bdít nad mým lidem. \Vbardot{} Ani ve dne, ani v~noci nesmějí nikdy
mlčet.}}

\newcommand{\trVersus}{\translatioCantus{\Vbardot{} Ústa spravedlivého šeptají moudrost, aleluja.
\Rbardot{} A~jeho jazyk ohlašuje právo, aleluja.}}

\newcommand{\trAntBenedictus}{\translatioCantus{Když na bujné oře vložili
nosítka a~sňali jim uzdu, vydali se přímo k~cele božího muže.}}

\newcommand{\trPreces}{\translatioCantus{
\noindent S vděčností chvalme Krista, dobrého Pastýře, \gredagger{} který dal život za své ovce, \grestar{} a~pokorně ho prosme: \Rbardot{} Pane, buď pastýřem svého lidu.

\noindent Kriste, ty dáváš církvi pastýře, a~jejich službou se ujímáš svého lidu, \grestar{} dej, ať v~lásce těch, kteří nás vedou, poznáváme, jak nás miluješ. \Rbardot{} Pane, buď pastýřem svého lidu.

\noindent Ty stále konáš skrze své zástupce službu pastýře a~učitele, \grestar{} nepřestávej nás nikdy vést prostřednictvím svých služebníků. \Rbardot{} Pane, buď pastýřem svého lidu.

\noindent Ty prokazuješ svému lidu skrze jeho pastýře službu lékaře duše i~těla, \grestar{} ochraňuj náš život a~veď nás ke svatosti. \Rbardot{} Pane, buď pastýřem svého lidu.

\noindent Ty posíláš své svaté, aby slovem i~příkladem vedli tvůj lid k~tobě, \grestar{} na jejich přímluvu nás posiluj, abychom vytrvali na cestě, která vede k~věčnému životu. \Rbardot{} Pane, buď pastýřem svého lidu.}}

\newcommand{\trOrationis}{\translatioCantus{Bože, jenž nám dopřáváš radovat
se z~výroční slavnosti svatého tvého vyznavače Havla, uděl dobrotivě,
abychom když slavíme jeho narození, též se řídili podobou jeho skutků.
Skrze…}}
 % Czech translations of the proper texts

\newcommand{\annusEditionis}{2020}

%%%% Vicekrat opakovane kousky

\newcommand{\anteOrationem}{
  \rubrica{Ante Orationem, cantatur a Superiore:}

  \pars{Supplicatio Litaniæ.}

  \cuminitiali{}{temporalia/supplicatiolitaniae.gtex}

  \pars{Oratio Dominica.}

  \cuminitiali{}{temporalia/oratiodominica.gtex}

  \rubrica{Deinde dicitur ab Hebdomadario:}

  \cuminitiali{}{temporalia/dominusvobiscum-solemnis.gtex}

  \rubrica{In choro monialium loco Dominus vobiscum dicitur:}

  \sineinitiali{temporalia/domineexaudi.gtex}
}

\setlength{\columnsep}{30pt} % prostor mezi sloupci

%%%%%%%%%%%%%%%%%%%%%%%%%%%%%%%%%%%%%%%%%%%%%%%%%%%%%%%%%%%%%%%%%%%%%%%%%%%%%%%%%%%%%%%%%%%%%%%%%%%%%%%%%%%%%
\begin{document}

% Here we set the space around the initial.
% Please report to http://home.gna.org/gregorio/gregoriotex/details for more details and options
\grechangedim{afterinitialshift}{2.2mm}{scalable}
\grechangedim{beforeinitialshift}{2.2mm}{scalable}
\grechangedim{interwordspacetext}{0.22 cm plus 0.15 cm minus 0.05 cm}{scalable}%
\grechangedim{annotationraise}{-0.2cm}{scalable}

% Here we set the initial font. Change 38 if you want a bigger initial.
% Emit the initials in red.
\grechangestyle{initial}{\color{red}\fontsize{38}{38}\selectfont}

\pagestyle{empty}

%%%% Titulni stranka
\begin{titulusOfficii}
\titulus{}
\end{titulusOfficii}

% graphic
%\vspace{1.5cm}
%\begin{center}
%\includegraphics[width=8cm]{emmaus.jpg}
%\end{center}

\vfill

\begin{center}
%Ad usum et secundum consuetudines chori \guillemotright{}Conventus Choralis\guillemotleft.

%Editio Sancti Wolfgangi \annusEditionis
\end{center}

\pagebreak

\renewcommand{\headrulewidth}{0pt} % no horiz. rule at the header
\fancyhf{}
\pagestyle{fancy}

\pars{Oratio ante divinum Officium.}

\lettrine{{\color{red}A}}{peri,} Dómine, os meum ad benedicéndum nomen sanctum tuum:
munda quoque cor meum ab ómnibus vanis, pervérsis, et aliénis
cogitatiónibus:
intelléctum illúmina, afféctum inflámma,
ut digne, atténte ac devóte hoc Offícium recitáre váleam,
et exaudíri mérear ante conspéctum Divínæ Maiestátis tuæ.
Per Christum, Dóminum nostrum.
\Rbardot{} Amen.

Dómine, in unióne illíus divínæ intentiónis,
qua ipse in terris laudes Deo persolvísti,
has tibi Horas \rubricatum{(vel \textnormal{hanc tibi Horam})} persólvo.

%\trOratioAnteOfficium

\vfill

\pars{Oratio post divinum Officium.}

\rubrica{
  Orationem sequentem devote post Officium recitantibus
  Leo Papa X. defectus, et culpas in eo persolvendo ex humana
  fragilitate contractas, indulsit, et dicitur flexis genibus.
}

\lettrine{{\color{red}S}}{acrosánctæ} et indivíduæ Trinitáti,
crucifíxi Dómini nostri Iesu Christi humanitáti,
beatíssimæ et gloriosíssimæ sempérque Vírginis Maríæ
fecúndæ integritáti, 
et ómnium Sanctórum universitáti
sit sempitérna laus, honor, virtus et glória
ab omni creatúra,
nobísque remíssio ómnium peccatórum,
per infiníta sǽcula sæculórum.
\Rbardot{} Amen.

\noindent \Vbardot{} Beáta víscera Maríæ Virginis, quæ portavérunt
ætérni Patris Fílium.\\
\Rbardot{} Et beáta úbera, quæ lactavérunt Christum Dominum.

\rubrica{Et dicitur secreto \textnormal{Pater noster.} et \textnormal{Ave María.}}

%\trOratioPostOfficium

\vfill

\hora{Ad I. Vesperas.} %%%%%%%%%%%%%%%%%%%%%%%%%%%%%%%%%%%%%%%%%%%%%%%%%%%%%
%\sideThumbs{I. Vesperæ}

\cantusSineNeumas

\vspace{0.5cm}
\grechangedim{interwordspacetext}{0.18 cm plus 0.15 cm minus 0.05 cm}{scalable}%
\cuminitiali{}{temporalia/deusinadiutorium-solemnis.gtex}
\grechangedim{interwordspacetext}{0.22 cm plus 0.15 cm minus 0.05 cm}{scalable}%

\vfill
\pagebreak

\pars{Psalmus 1.} \scriptura{Ps. 144, 13; \textbf{H100}}

\vspace{-4mm}

\antiphona{VII c\textsuperscript{2}}{temporalia/ant-regnumtuum.gtex}

\scriptura{Psalmus 144, 10-21.}

\initiumpsalmi{temporalia/ps144ii-initium-vii-c2-auto.gtex}

%\psalmusEtTranslatioT{temporalia/ps144ii-VII-comb.tex}{10cm}
\input{temporalia/ps144ii-VII.tex} \Abardot{}

\vspace{-1cm}

\vfill
\pagebreak

\pars{Psalmus 2.} \scriptura{Ps. 145, 2; \textbf{H100}}

\vspace{-4mm}

\antiphona{IV E}{temporalia/ant-laudabodeum.gtex}

\scriptura{Psalmus 145.}

\initiumpsalmi{temporalia/ps145-initium-iv-E-auto.gtex}

%\psalmusEtTranslatioT{temporalia/ps145-VII-comb.tex}{10cm}
\input{temporalia/ps145-VII.tex} \Abardot{}

\vfill
\pagebreak

\pars{Psalmus 3.} \scriptura{Ps. 146, 1; \textbf{H101}}

\vspace{-4mm}

\antiphona{VIII a}{temporalia/ant-deonostro.gtex}

\scriptura{Psalmus 146.}

\initiumpsalmi{temporalia/ps146-initium-viii-A-auto.gtex}

%\psalmusEtTranslatioT{temporalia/ps146-VII-comb.tex}{10cm}
\input{temporalia/ps146-VII.tex} \Abardot{}

\vfill
\pagebreak

\pars{Psalmus 4.} \scriptura{Ps. 147, 1}

\vspace{-4mm}

\antiphona{E}{temporalia/ant-laudajerusalem.gtex}

\scriptura{Psalmus 147.}

\initiumpsalmi{temporalia/ps147-initium-e-auto.gtex}

%\psalmusEtTranslatioT{temporalia/ps147-VII-comb.tex}{10cm}
\input{temporalia/ps147-VII.tex} \Abardot{}

\vfill
\pagebreak

\pars{Capitulum.} \scriptura{Rom. 11, 33}

\grechangedim{interwordspacetext}{0.12 cm plus 0.15 cm minus 0.05 cm}{scalable}%
\cuminitiali{}{temporalia/capitulum-OAltitudo.gtex}
\grechangedim{interwordspacetext}{0.22 cm plus 0.15 cm minus 0.05 cm}{scalable}

% preklad Jeruz. bible
%\trCapituliI

\vfill

\pars{Responsorium breve.} \scriptura{Ps. 146, 5}

\cuminitiali{VI}{temporalia/resp-magnusdominusnoster.gtex}

%\trResp

\vfill
\pagebreak

\pars{Hymnus} \scriptura{Ambrosius (\olddag{} 397)}

\cuminitiali{I}{temporalia/hym-OLuxBeata-aestivalis.gtex}
\vspace{-3mm}
%\input{hym-OLuxBeata-bohtext.tex}

\vfill
%\pagebreak

\pars{Versus.}

% Versus. %%%
\sineinitiali{temporalia/versus-vespertina.gtex}

%\noindent \trVersus

\vfill
\pagebreak

\magnificati

\vfill
\pagebreak

%\sideThumbs{{\scriptsize{}Fine horarum}}

\anteOrationem

\pagebreak

% Oratio. %%%
\oratioLaudes

\vspace{-1mm}
%\trOrationisI

\vfill

\rubrica{Hebdomadarius dicit iterum Dominus vobiscum, vel cantor dicit:}

\vspace{2mm}

\sineinitiali{temporalia/domineexaudi.gtex}

\rubrica{Postea cantatur a cantore:}

\vspace{2mm}

\cuminitiali{I}{temporalia/benedicamus-dominica-perannum.gtex}

\vspace{1mm}

\vfill
\pagebreak

\hora{Ad Matutinum.} %%%%%%%%%%%%%%%%%%%%%%%%%%%%%%%%%%%%%%%%%%%%%%%%%%%%%
%\sideThumbs{Matutinum}

\vspace{2mm}

\cuminitiali{}{temporalia/dominelabiamea.gtex}

\vspace{2mm}

\pars{Invitatorium.} \scriptura{Ps. 94, 1; Psalmus 94}

\vspace{-6mm}

\antiphona{E}{temporalia/inv-veniteexsultemus.gtex}

\vfill
\pagebreak

\pars{Hymnus.} \scriptura{Adamus Sancti Victoris (\olddag 1146)}

\vspace{-5mm}

\antiphona{VII}{temporalia/hym-SalveDies.gtex}

\scriptura{Non dicitur \textnormal{Amen} in fine.}
%{
%\vspace{-5mm}
%\setlength{\columnsep}{0pt} % prostor mezi sloupci
%\input{hym-SalveDies-bohtext.tex}
%\setlength{\columnsep}{30pt} % prostor mezi sloupci
%}

\vfill
\pagebreak

\subhora{In I. Nocturno}

\pars{Psalmus 1.} \scriptura{Ps. 1, 1}

\vspace{-4mm}

\antiphona{VIII G}{temporalia/ant-beatusvir.gtex}

%\vspace{-5mm}

\scriptura{Ps. 1}

%\vspace{-2mm}

\initiumpsalmi{temporalia/ps1-initium-viii-G-auto.gtex}

%\psalmusEtTranslatioT{temporalia/ps1-I-comb.tex}{10cm}
\input{temporalia/ps1-I.tex} \Abardot{}

\vfill
\pagebreak

\pars{Psalmus 2.} \scriptura{Ps. 2, 11; \textbf{H93}}

\vspace{-4mm}

\antiphona{VII a}{temporalia/ant-servitedomino.gtex}

\vspace{-3mm}

\scriptura{Ps. 2}

\vspace{-2mm}

\initiumpsalmi{temporalia/ps2-initium-vii-a-auto.gtex}

%\psalmusEtTranslatioT{temporalia/ps2-I-comb.tex}{10cm}
\input{temporalia/ps2-I.tex} \Abardot{}

\vfill
\pagebreak

\pars{Psalmus 3.} \scriptura{Ps. 3, 7}

\vspace{-4mm}

\antiphona{VI F}{temporalia/ant-exsurgedominesalvum.gtex}

%\vspace{-5mm}

\scriptura{Ps. 3}

\initiumpsalmi{temporalia/ps3-initium-vi-F-auto.gtex}

%\psalmusEtTranslatioT{temporalia/ps3-I-comb.tex}{10cm}
\input{temporalia/ps3-I.tex} \Abardot{}

\vfill
\pagebreak

\pars{Versus.} \scriptura{Ps. 118, 55}

% Versus. %%%
\sineinitiali{temporalia/versus-memorfui.gtex}

\vspace{5mm}

\sineinitiali{temporalia/oratiodominica-mat.gtex}

\vspace{5mm}

\pars{Absolutio.}

\cuminitiali{}{temporalia/absolutio-exaudi.gtex}

\vfill
\pagebreak

\cuminitiali{}{temporalia/benedictio-solemn-benedictione.gtex}

\vspace{7mm}

\lectioi

\noindent \Vbardot{} Tu autem, Dómine, miserére nobis.
\noindent \Rbardot{} Deo grátias.

\vfill
\pagebreak

\responsoriumi

\vfill
\pagebreak

\cuminitiali{}{temporalia/benedictio-solemn-unigenitus.gtex}

\vspace{7mm}

\lectioii

\noindent \Vbardot{} Tu autem, Dómine, miserére nobis.
\noindent \Rbardot{} Deo grátias.

\vfill
\pagebreak

\responsoriumii

\vfill
\pagebreak

\cuminitiali{}{temporalia/benedictio-solemn-spiritus.gtex}

\vspace{7mm}

\lectioiii

\noindent \Vbardot{} Tu autem, Dómine, miserére nobis.
\noindent \Rbardot{} Deo grátias.

\vfill
\pagebreak

\responsoriumiii

\vfill
\pagebreak

\subhora{In II. Nocturno}

\pars{Psalmus 4.} \scriptura{Ps. 8, 2}

\vspace{-4mm}

\antiphona{I g}{temporalia/ant-quamadmirabileest.gtex}

%\vspace{-5mm}

\scriptura{Ps. 8}

%A\vspace{-2mm}

\initiumpsalmi{temporalia/ps8-initium-i-g-auto.gtex}

%\psalmusEtTranslatioT{temporalia/ps8-I-comb.tex}{10cm}
\input{temporalia/ps8-I.tex} \Abardot{}

\vfill
\pagebreak

\pars{Psalmus 5.} \scriptura{Ps. 9, 5}

\vspace{-4mm}

\antiphona{VIII G}{temporalia/ant-sedistisuperthronum.gtex}

%\vspace{-5mm}

\scriptura{Ps. 9, 2-11}

\initiumpsalmi{temporalia/ps9ii_xi-initium-viii-G-auto.gtex}

%\psalmusEtTranslatioT{temporalia/ps9ii_xi-I-comb.tex}{10cm}
\input{temporalia/ps9ii_xi-I.tex} \Abardot{}

\vfill
\pagebreak

\pars{Psalmus 6.} \scriptura{Ps. 9, 20}

\vspace{-4mm}

\antiphona{I g\textsuperscript{3}}{temporalia/ant-exsurgedominenon.gtex}

%\vspace{-5mm}

\scriptura{Ps. 9, 12-21}

\initiumpsalmi{temporalia/ps9xii_xxi-initium-i-g3-auto.gtex}

%\psalmusEtTranslatioT{temporalia/ps9xii_xxi-I-comb.tex}{10cm}
\input{temporalia/ps9xii_xxi-I.tex} \Abardot{}

\vfill
\pagebreak

\pars{Versus.} \scriptura{Ps. 118, 62}

% Versus. %%%
\sineinitiali{temporalia/versus-medianocte.gtex}

\vspace{5mm}

\sineinitiali{temporalia/oratiodominica-mat.gtex}

\vspace{5mm}

\pars{Absolutio.}

\cuminitiali{}{temporalia/absolutio-ipsius.gtex}

\vfill
\pagebreak

\cuminitiali{}{temporalia/benedictio-solemn-deus.gtex}

\vspace{7mm}

\lectioiv

\noindent \Vbardot{} Tu autem, Dómine, miserére nobis.
\noindent \Rbardot{} Deo grátias.

\vfill
\pagebreak

\responsoriumiv

\vfill
\pagebreak

\cuminitiali{}{temporalia/benedictio-solemn-christus.gtex}

\vspace{7mm}

\lectiov

\noindent \Vbardot{} Tu autem, Dómine, miserére nobis.
\noindent \Rbardot{} Deo grátias.

\vfill
\pagebreak

\responsoriumv

\vfill
\pagebreak

\cuminitiali{}{temporalia/benedictio-solemn-ignem.gtex}

\vspace{7mm}

\lectiovi

\noindent \Vbardot{} Tu autem, Dómine, miserére nobis.
\noindent \Rbardot{} Deo grátias.

\vfill
\pagebreak

\responsoriumvi

\vfill
\pagebreak

\subhora{In III. Nocturno}

\pars{Psalmus 7.} \scriptura{Ps. 9, 22}

\vspace{-4mm}

\antiphona{II D}{temporalia/ant-utquiddomine.gtex}

\vspace{-4mm}

\scriptura{Ps. 9, 22-32}

%\vspace{-2mm}

\initiumpsalmi{temporalia/ps9xxii_xxxii-initium-ii-D-auto.gtex}

%\psalmusEtTranslatioT{temporalia/ps9xxii_xxxii-I-comb.tex}{10cm}
\input{temporalia/ps9xxii_xxxii-I.tex} \Abardot{}

\vfill
\pagebreak

\pars{Psalmus 8.}\scriptura{Ex. 15, 18}

\vspace{-4mm}

\antiphona{IV* e}{temporalia/ant-inaeternum.gtex}

%\vspace{-4mm}

\scriptura{Ps. 9, 33-39}

\initiumpsalmi{temporalia/ps9xxxiii_xxxix-initium-iv_-e-auto.gtex}

%\psalmusEtTranslatioT{temporalia/ps9xxxiii_xxxix-I-comb.tex}{10cm}
\input{temporalia/ps9xxxiii_xxxix-I.tex} \Abardot{}

\vfill
\pagebreak

\pars{Psalmus 9.} \scriptura{Ps. 10, 8}

\vspace{-4mm}

\antiphona{II* f}{temporalia/ant-justusdominus.gtex}

%\vspace{-4mm}

\scriptura{Ps. 10}

%\initiumpsalmi{temporalia/ps10-initium-iv-c-auto.gtex}
\initiumpsalmi{temporalia/ps10-initium-ii_-f.gtex}

%\psalmusEtTranslatioT{temporalia/ps10-I-comb.tex}{10cm}
\input{temporalia/ps10-I.tex} \Abardot{}

\vfill
\pagebreak

\pars{Versus.} \scriptura{Ps. 118, 148}

% Versus. %%%
\sineinitiali{temporalia/versus-praevenerunt.gtex}

\vspace{5mm}

\sineinitiali{temporalia/oratiodominica-mat.gtex}

\vspace{5mm}

\pars{Absolutio.}

\cuminitiali{}{temporalia/absolutio-avinculis.gtex}

\vfill
\pagebreak

\cuminitiali{}{temporalia/benedictio-solemn-evangelica.gtex}

\vspace{7mm}

\lectiovii

\noindent \Vbardot{} Tu autem, Dómine, miserére nobis.
\noindent \Rbardot{} Deo grátias.

\vfill
\pagebreak

\responsoriumvii

\vfill
\pagebreak

\cuminitiali{}{temporalia/benedictio-solemn-divinum.gtex}

\vspace{7mm}

\lectioviii

\noindent \Vbardot{} Tu autem, Dómine, miserére nobis.
\noindent \Rbardot{} Deo grátias.

\vfill
\pagebreak

\responsoriumviii

\vfill
\pagebreak

\cuminitiali{}{temporalia/benedictio-solemn-adsocietatem.gtex}

\vspace{7mm}

\lectioix

\noindent \Vbardot{} Tu autem, Dómine, miserére nobis.
\noindent \Rbardot{} Deo grátias.

\vfill
\pagebreak

% Te Deum

{
\pars{Hymnus Ambrosianus} \scriptura{Tonus Solemnis}

\vspace{-2mm}

\grechangedim{interwordspacetext}{0.26 cm plus 0.15 cm minus 0.05 cm}{scalable}%
\cuminitiali{III}{temporalia/tedeum-solemnis-gn.gtex}
\grechangedim{interwordspacetext}{0.22 cm plus 0.15 cm minus 0.05 cm}{scalable}%
}

\vfill
\pagebreak

\rubrica{Reliqua omittuntur, nisi Laudes separandæ sint.}

\pars{Oratio}

\noindent \Vbardot{} Dómine, exáudi oratiónem meam.

\noindent \Rbardot{} Et clamor meus ad te véniat.

Orémus:

\oratioLaudes

\vspace{7mm}

\pars{Conclusio}

\noindent \Vbardot{} Dómine, exáudi oratiónem meam.

\noindent \Rbardot{} Et clamor meus ad te véniat.

\noindent \Vbardot{} Benedicámus Dómino, allelúia, allelúia.

\noindent \Rbardot{} Deo grátias, allelúia, allelúia.

\noindent \Vbardot{} Fidélium ánimæ per misericórdiam Dei requiéscant in pace.

\noindent \Rbardot{} Amen.

\vfill
\pagebreak

\hora{Ad Laudes.} %%%%%%%%%%%%%%%%%%%%%%%%%%%%%%%%%%%%%%%%%%%%%%%%%%%%%
%\sideThumbs{Laudes}

\cantusSineNeumas

\vspace{0.5cm}
\grechangedim{interwordspacetext}{0.18 cm plus 0.15 cm minus 0.05 cm}{scalable}%
\cuminitiali{}{temporalia/deusinadiutorium-alter.gtex}
\grechangedim{interwordspacetext}{0.22 cm plus 0.15 cm minus 0.05 cm}{scalable}%

\vfill
%\pagebreak

\pars{Psalmus 1.}

\vspace{-4mm}

\antiphona{VI F}{temporalia/ant-alleluia1.gtex}

\scriptura{Psalmus 50.}

\initiumpsalmi{temporalia/ps50-initium-vi-F-auto.gtex}

%\psalmusEtTranslatioT{temporalia/ps50-I-comb.tex}{10cm}
\input{temporalia/ps50-I.tex}

\vfill
\pagebreak

\pars{Psalmus 2.}

\scriptura{Psalmus 117.}

\initiumpsalmi{temporalia/ps117-initium-vi-F-auto.gtex}

%\psalmusEtTranslatioT{temporalia/ps117-I-comb.tex}{10cm}
\input{temporalia/ps117-I.tex}

\vfill
\pagebreak

\pars{Psalmus 3.}

\scriptura{Psalmus 62.}

\initiumpsalmi{temporalia/ps62-initium-vi-F-auto.gtex}

%\psalmusEtTranslatioT{temporalia/ps62-I-comb.tex}{10cm}
\input{temporalia/ps62-I.tex}

\vfill

\vspace{-6mm}

\antiphona{}{temporalia/ant-alleluia1.gtex} % repeat the antiphon - new page

\vfill
\pagebreak

\pars{Psalmus 4.} \scriptura{Dan. 3, 22-26; \textbf{H422}}

\vspace{-4mm}

\antiphona{VIII G}{temporalia/ant-trespueri.gtex}

\scriptura{Canticum trium puerorum, Dan. 3, 57-88 et 56}

\initiumpsalmi{temporalia/dan3-initium-viii-G-auto.gtex}

%\psalmusEtTranslatioT{temporalia/dan3-comb.tex}{10cm}
\input{temporalia/dan3.tex}

\rubrica{Hic non dicitur Gloria Patri, neque Amen.}

\vfill

\vspace{-6mm}

\antiphona{}{temporalia/ant-trespueri.gtex} % repeat the antiphon - new page

\vfill
\pagebreak

\pars{Psalmus 5.}

\vspace{-4mm}

\antiphona{VIII G}{temporalia/ant-alleluia2.gtex}

\scriptura{Psalmus 148.}

\initiumpsalmi{temporalia/ps148-initium-viii-G-auto.gtex}

%\psalmusEtTranslatioT{temporalia/ps148-I-comb.tex}{10cm}
\input{temporalia/ps148-I.tex}

\rubrica{Hic non dicitur Gloria Patri.}

\vfill
\pagebreak

%
\scriptura{Psalmus 149.}

\initiumpsalmi{temporalia/ps149-initium-viii-G-auto.gtex}

%\psalmusEtTranslatioT{temporalia/ps149-I-comb.tex}{10cm}
\input{temporalia/ps149-I.tex}

\rubrica{Hic non dicitur Gloria Patri.}

\vfill
\pagebreak

%
\scriptura{Psalmus 150.}

\initiumpsalmi{temporalia/ps150-initium-viii-G-auto.gtex}

%\psalmusEtTranslatioT{temporalia/ps150-I-comb.tex}{10cm}
\input{temporalia/ps150-I.tex}

\vfill

\vspace{-6mm}

\antiphona{}{temporalia/ant-alleluia2.gtex} % repeat the antiphon - new page

\vfill
\pagebreak

\pars{Capitulum.} \scriptura{Ac. 7, 12}

\grechangedim{interwordspacetext}{0.12 cm plus 0.15 cm minus 0.05 cm}{scalable}%
\cuminitiali{}{temporalia/capitulum-Benedictio.gtex}
\grechangedim{interwordspacetext}{0.22 cm plus 0.15 cm minus 0.05 cm}{scalable}

% preklad Jeruz. bible
%\trCapituliI

\vfill

\pars{Responsorium breve.} \scriptura{Ps. 118, 36-37}

\cuminitiali{IV}{temporalia/resp-inclinacormeum.gtex}

%\trResp

\vfill
\pagebreak

\pars{Hymnus} \scriptura{Gregorius Magnus (\olddag{} 604)}

\cuminitiali{IV}{temporalia/hym-EcceJamNoctis.gtex}
\vspace{-3mm}
%\input{hym-EcceJamNocis-bohtext.tex}

\vfill
%\pagebreak

\pars{Versus.} \scriptura{Ps. 92, 1}

% Versus. %%%
\sineinitiali{temporalia/versus-dominusregnavit.gtex}

%\noindent \trVersus

\vfill
\pagebreak

\benedictus

\vspace{-1cm}

\vfill
\pagebreak

%\sideThumbs{{\scriptsize{}Fine horarum}}

\anteOrationem

\pagebreak

% Oratio. %%%
\oratioLaudes

\vspace{-1mm}
%\trOrationisI

\vfill

\rubrica{Hebdomadarius dicit iterum Dominus vobiscum, vel cantor dicit:}

\vspace{2mm}

\sineinitiali{temporalia/domineexaudi.gtex}

\rubrica{Postea cantatur a cantore:}

\vspace{2mm}

\cuminitiali{I}{temporalia/benedicamus-dominica-perannum.gtex}

\vspace{1mm}

\vfill
\pagebreak

\hora{Ad II. Vesperas.} %%%%%%%%%%%%%%%%%%%%%%%%%%%%%%%%%%%%%%%%%%%%%%%%%%%%%
%\sideThumbs{II. Vesperæ}

\cantusSineNeumas

%\vspace{0.5cm}
\grechangedim{interwordspacetext}{0.18 cm plus 0.15 cm minus 0.05 cm}{scalable}%
\cuminitiali{}{temporalia/deusinadiutorium-solemnis.gtex}
\grechangedim{interwordspacetext}{0.22 cm plus 0.15 cm minus 0.05 cm}{scalable}%

\vfill
%\pagebreak

\vspace{-2mm}

\pars{Psalmus 1.} \scriptura{Ps. 109, 1; \textbf{H91}}

\vspace{-4mm}

\antiphona{VII c\textsuperscript{2}}{temporalia/ant-dixitdominus.gtex}

\vspace{-4mm}

\scriptura{Psalmus 109.}

\initiumpsalmi{temporalia/ps109-initium-vii-c2-auto.gtex}

%\psalmusEtTranslatioT{temporalia/ps109-I-comb.tex}{10cm}
\input{temporalia/ps109-I.tex} \Abardot{}

\vspace{-1cm}

\vfill
\pagebreak

\pars{Psalmus 2.} \scriptura{Ps. 110, 8; \textbf{H91}}

\vspace{-4mm}

\antiphona{IV g}{temporalia/ant-fideliaomnia.gtex}

\scriptura{Psalmus 110.}

\initiumpsalmi{temporalia/ps110-initium-iv-g-auto.gtex}

%\psalmusEtTranslatioT{temporalia/ps110-I-comb.tex}{10cm}
\input{temporalia/ps110-I.tex} \Abardot{}

\vfill
\pagebreak

\pars{Psalmus 3.} \scriptura{Ps. 111, 1; \textbf{H92}}

\vspace{-4mm}

\antiphona{IV a}{temporalia/ant-inmandatis.gtex}

\scriptura{Psalmus 111.}

\initiumpsalmi{temporalia/ps111-initium-iv-a-auto.gtex}

%\psalmusEtTranslatioT{temporalia/ps111-I-comb.tex}{10cm}
\input{temporalia/ps111-I.tex} \Abardot{}

\vfill
\pagebreak

\pars{Psalmus 4.} \scriptura{Ps. 112, 2; \textbf{H92}}

\vspace{-4mm}

\antiphona{VII c}{temporalia/ant-sitnomendomini.gtex}

\scriptura{Psalmus 112.}

\initiumpsalmi{temporalia/ps112-initium-vii-c-auto.gtex}

%\psalmusEtTranslatioT{temporalia/ps112-I-comb.tex}{10cm}
\input{temporalia/ps112-I.tex} \Abardot{}

\vfill
\pagebreak

\pars{Capitulum.} \scriptura{2 Cor. 1, 3-4}

\grechangedim{interwordspacetext}{0.12 cm plus 0.15 cm minus 0.05 cm}{scalable}%
\cuminitiali{}{temporalia/capitulum-BenedictusDeus.gtex}
\grechangedim{interwordspacetext}{0.22 cm plus 0.15 cm minus 0.05 cm}{scalable}

% preklad Jeruz. bible
%\trCapituliI

\vfill

\pars{Responsorium breve.} \scriptura{Ps. 103, 24}

\cuminitiali{VI}{temporalia/resp-quammagnificata.gtex}

%\trResp

\vfill
\pagebreak

\pars{Hymnus} \scriptura{Gregorius Magnus (\olddag{} 604)}

\cuminitiali{I}{temporalia/hym-LucisCreator-aestivalis.gtex}
\vspace{-3mm}
%\begin{translatioMulticol}{3}
Tvůrce světa předobrý,\\
tys ustanovil denní řád\\
a proudy světla rozhodil,\\
když světu základy jsi klad.\\
\\
A spojils ráno s večerem\\
a dnem tu dobu nazýváš;\\
hle padá temné noci stín -\\
slyš prosbu, vyslyš nářek náš.\columnbreak

Ach, nedej, by nás stihla smrt,\\
když svědomí nám tíží hřích,\\
když nemyslíme na věčnost\\
v té síti hříchů šalebných.\\
\\
Vzbuď naši touhu po nebi,\\
kde věčný život čeká nás,\\
a pomoz odložit vše zlé\\
a smýti z duše každý kaz.\columnbreak

To splň nám, dobrý Otče náš,\\
i ty, jenž rovné božství máš,\\
i Duchu, který těšíš nás\\
a vládneš, Bože, v každý čas.\\
Amen. 
\end{translatioMulticol}


\vfill
%\pagebreak

\pars{Versus.} \scriptura{Ps. 140, 2}

% Versus. %%%
\sineinitiali{temporalia/versus-dirigatur.gtex}

%\noindent \trVersus

\vfill
\pagebreak

\magnificatii

\vfill
\pagebreak

%\sideThumbs{{\scriptsize{}Fine horarum}}

\anteOrationem

\pagebreak

% Oratio. %%%
\oratioLaudes

\vspace{-1mm}
%\trOrationisI

\vfill

\rubrica{Hebdomadarius dicit iterum Dominus vobiscum, vel cantor dicit:}

\vspace{2mm}

\sineinitiali{temporalia/domineexaudi.gtex}

\rubrica{Postea cantatur a cantore:}

\vspace{2mm}

\cuminitiali{I}{temporalia/benedicamus-dominica-perannum.gtex}

\vspace{1mm}

\end{document}

