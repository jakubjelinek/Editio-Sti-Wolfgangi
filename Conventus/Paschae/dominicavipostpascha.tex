\newcommand{\titulus}{\nomenFesti{Dominica VII Paschæ (infra octavam Ascensionis).}}
\newcommand{\tedeumsimplex}{Simplex}
\newcommand{\oratioLaudes}{\cuminitiali{}{temporalia/oratiod7.gtex}}
\newcommand{\oratio}{\pars{Oratio.}

\noindent Supplicatiónibus nostris, Dómine, adésto propítius, ut, sicut humáni géneris Salvatórem tecum in tua crédimus maiestáte, ita usque ad consummatiónem sǽculi manére nobíscum, sicut ipse promísit, sentiámus.

\pars{Pro pace in Ucraina.} \scriptura{Sir. 50, 25; 2 Esdr. 4, 20; \textbf{H416}}

\vspace{-4mm}

\antiphona{II D}{temporalia/ant-dapacemdomine.gtex}

\vfill

\noindent Deus, a quo sancta desidéria, recta consília et iusta sunt ópera: da servis tuis illam, quam mundus dare non potest, pacem; ut et corda nostra mandátis tuis dédita, et hóstium subláta formídine, témpora sint tua protectióne tranquílla.

\noindent Per Dóminum nostrum Iesum Christum, Fílium tuum, qui tecum vivit et regnat in unitáte Spíritus Sancti, Deus, per ómnia sǽcula sæculórum.

\noindent \Rbardot{} Amen.}
\newcommand{\invitatorium}{\pars{Invitatorium.}

\vspace{-4mm}

\antiphona{VI**}{temporalia/inv-christumdominumquisanctum.gtex}}
\newcommand{\hymnusmatutinum}{\pars{Hymnus.} \scriptura{Anonymus X. sæculi; \textbf{AR488}}

\antiphona{IV}{temporalia/hym-AEterneRex.gtex}}
\newcommand{\nocturnoi}{\pars{Psalmus 1.}

\vspace{-4mm}

\antiphona{II D}{temporalia/ant-alleluia-bv21-n1.gtex}

%\vspace{-2mm}

\scriptura{Ps. 1}

%\vspace{-2mm}

%\initiumpsalmi{temporalia/ps1-initium-ii-D-auto.gtex}
\initiumpsalmi{temporalia/ps1-initium-ii-D.gtex}

%\input{temporalia/ps1-ii-D.tex} \Abardot{}
\input{temporalia/ps1-viii-G.tex} \Abardot{}

\vfill
\pagebreak

\pars{Psalmus 2.}

\vspace{-4mm}

\antiphona{VIII c}{temporalia/ant-alleluia-bv21-n2.gtex}

%\vspace{-2mm}

\scriptura{Ps. 2}

\initiumpsalmi{temporalia/ps2-initium-viii-c-auto.gtex}

\input{temporalia/ps2-viii-c.tex} \Abardot{}

\vfill
\pagebreak

\pars{Psalmus 3.}

%\vspace{-4mm}

\antiphona{VIII G\textsuperscript{3}}{temporalia/ant-alleluia-bv21-n3.gtex}

%\vspace{-2mm}

\scriptura{Ps. 3}

%\initiumpsalmi{temporalia/ps3-initium-viii-G2-auto.gtex}
\initiumpsalmi{temporalia/ps3-initium-viii-G3.gtex}

\input{temporalia/ps3-viii-G2.tex} \Abardot{}

\vfill
\pagebreak}
\newcommand{\nocturnoii}{\vspace{-6mm}

\pars{Psalmus 4.} \scriptura{Dn. 7, 14; \textbf{H78}}

\vspace{-4mm}

\antiphona{VIII G}{temporalia/ant-christodatusest.gtex}

%\vspace{-1mm}

\scriptura{Ps. 144, 1-9}

%\vspace{-2mm}

\initiumpsalmi{temporalia/ps144i-initium-viii-G-auto.gtex}

\input{temporalia/ps144i-viii-G.tex} \Abardot{}

\vfill
\pagebreak

\pars{Psalmus 5.} \scriptura{Ps. 5, 12}

\vspace{-4mm}

\antiphona{III a\textsuperscript{2}}{temporalia/ant-laetenturetexultentomnes.gtex}

%\vspace{-2mm}

\scriptura{Ps. 144, 10-13}

\initiumpsalmi{temporalia/ps144x_xiii-initium-iii-a4-auto.gtex}

\input{temporalia/ps144x_xiii-iii-a4.tex} \Abardot{}

\vfill
\pagebreak

\pars{Psalmus 6.} \scriptura{Mt. 28, 18; \textbf{H238}}

\vspace{-4mm}

\antiphona{VIII G\textsuperscript{2}}{temporalia/ant-dataestmihi.gtex}

\vspace{-2mm}

\scriptura{Ps. 144, 14-21}

%\vspace{-2mm}

\initiumpsalmi{temporalia/ps144xiv_xxi-initium-viii-G5-auto.gtex}

%\vspace{-1.5mm}

\input{temporalia/ps144xiv_xxi-viii-G5.tex} \Abardot{}

\vfill
\pagebreak}
\newcommand{\nocturnoiii}{\vspace{-4mm}

\pars{Cantica.} \scriptura{Io. 14, 2.3; 16, 22; \textbf{H265}}

\vspace{-4mm}

\antiphona{III b}{temporalia/ant-vadopararevobis.gtex}

\vspace{-2mm}

\scriptura{Canticum Isaiæ, Is. 63, 1-5}

\vspace{-2mm}

\initiumpsalmi{temporalia/isaiae12-initium-iii-b-auto.gtex}

\input{temporalia/isaiae12-iii-b.tex} \hfill \rubrica{Hic non dicitur antiphona.}

\vfill
\pagebreak

\scriptura{Canticum Oseæ, Os. 6, 1-6}

%\vspace{-2mm}

\initiumpsalmi{temporalia/oseae-initium-iii-b-auto.gtex}

\input{temporalia/oseae-iii-b.tex}

\vfill
\pagebreak

\scriptura{Canticum Sophoniæ, Soph. 3, 8-13}

%\vspace{-2mm}

\initiumpsalmi{temporalia/sophoniae-initium-iii-b-auto.gtex}

\input{temporalia/sophoniae-iii-b.tex}

\vfill
\pagebreak

\antiphona{}{temporalia/ant-vadopararevobis.gtex}

\vfill
\pagebreak}
\newcommand{\matversusi}{\pars{Versus.}

\noindent \Vbardot{} Ascéndit Deus in iubilatióne, allelúia.

\noindent \Rbardot{} Et Dóminus in voce tubæ, allelúia.}
\newcommand{\matversusii}{\pars{Versus.}

\noindent \Vbardot{} Ascéndens Christus in altum, allelúia.

\noindent \Rbardot{} Captívam duxit captivitátem, allelúia.}
\newcommand{\lectioi}{\pars{Lectio I.} \scriptura{1 Io. 3, 18-20}

\noindent De Epístola prima beáti Ioánnis apóstoli.

\noindent Filíoli, non diligámus verbo nec lingua sed in ópere et veritáte. In hoc cognoscémus quóniam ex veritáte sumus, et in conspéctu eius placábimus corda nostra, quóniam si reprehénderit nos cor, maior est Deus corde nostro et cognóscit ómnia.}
\newcommand{\responsoriumi}{\pars{Responsorium 1.} \scriptura{\Rbardot{} Ac. 1, 3.9 \Vbardot{} ibid., 4; \textbf{H262}}

\vspace{-5mm}

\responsorium{III}{temporalia/resp-postpassionemsuam-CROCHU.gtex}{}}
\newcommand{\lectioii}{\pars{Lectio II.} \scriptura{1 Io. 3, 21-22}

\noindent Caríssimi, si cor nostrum non reprehénderit nos, fidúciam habémus ad Deum et, quodcúmque petiérimus, accípimus ab eo, quóniam mandáta eius custodímus et ea, quæ sunt plácita coram eo, fácimus.}
\newcommand{\responsoriumii}{\pars{Responsorium 2.} \scriptura{\Rbardot{} Ps. 67, 35; \Vbardot{} Io. 16, 7; \textbf{H262}}

\vspace{-5mm}

\responsorium{II}{temporalia/resp-omnispulchritudodomini-CROCHU.gtex}{}}
\newcommand{\lectioiii}{\pars{Lectio III.} \scriptura{1 Io. 3, 23-24}

\noindent Et hoc est mandátum eius, ut credámus nómini Fílii eius Iesu Christi et diligámus altérutrum, sicut dedit mandátum nobis. Et, qui servat mandáta eius, in ipso manet, et ipse in eo; et in hoc cognóscimus quóniam manet in nobis, ex Spíritu, quem nobis dedit.}
\newcommand{\responsoriumiii}{\pars{Responsorium 3.} \scriptura{\Rbardot{} Io. 14, 18 \Vbardot{} ibid., 27; \textbf{H270} \& \textbf{E261}}

\vspace{-5mm}

\responsorium{V}{temporalia/resp-nonvosrelinquam-CROCHU-cumdox.gtex}{}}
\newcommand{\lectioiv}{\pars{Lectio IV.} \scriptura{Hom. 15: PG 44, 1115-1118}

\noindent Ex Homíliis sancti Gregórii Nysséni epíscopi in Cánticum canticórum.

\noindent Si diléctio plane tandem metum eíciat, mutatúsque metus diléctio fiat; tum vero únitas esse reperiétur id, quod salútem conséquitur, ómnibus inter se unítis in conglutinatióne cum uno ac solo bono, per illam quæ in colúmba est perfectiónem.

\noindent Nam huiúsmodi quiddam ex verbis sequéntibus intellégimus, cum ait: \emph{Una est colúmba mea, perfécta mea. Unica est matri suæ, elécta illi quæ péperit eam.}

\noindent Atque hoc in Evangélio per vocem Dómini nobis illústrius declarátur. Etenim ille, benedicéndo, vim omnem discípulis suis cónferens, cum ália bona verbis illis, quibus ad Patrem útitur, tríbuit dignis; tum étiam addit hoc, quod bonórum caput ac summa est, non ámplius eos, in diversitáte quadam electiónum, multiplíciter divísos fore in faciéndo de bono iudício, sed omnes unum futúros, uni illi ac soli bono agglutinátos; ita ut, per unitátem Sancti Spíritus (quemádmodum Apóstolus ait) pacis constrícti nexu, corpus unum fiant omnes et unus spíritus, per spem unam, in quam vocáti sint.}
\newcommand{\responsoriumiv}{\pars{Responsorium 4.} \scriptura{\Rbardot{} Io. 14, 1; 15, 26; 16, 28 \Vbardot{} ibid. 16, 7; \textbf{H263}}

\vspace{-5mm}

\responsorium{III}{temporalia/resp-nonconturbetur-CROCHU.gtex}{}}
\newcommand{\lectiov}{\pars{Lectio V.}

\noindent Enimvéro réctius fecérimus, si de verbo ad verbum ipsas Evangélii divínas sane voces apposuérimus. \emph{Ut omnes,} inquit, \emph{sint unum, quemádmodum tu, mi Pater, in me es, et ego in te; ut et ipsi unum sint in nobis.}

\noindent Huius autem unitátis nexus est glória. Spíritum autem Sanctum appellári glóriam, nemo prudentiórum negábit, si ipsa Dómini verba intuébitur, cum ait: \emph{Glóriam, quam dedísti mihi, dedi eis.}  Revéra enim eándem discípulis glóriam dedit, cum ad eos díceret: \emph{Accípite Spíritum Sanctum.}}
\newcommand{\responsoriumv}{\pars{Responsorium 5.} \scriptura{\Rbardot{} Eph. 4, 8 \Vbardot{} Ps. 46, 6; \textbf{H263}}

\vspace{-5mm}

\responsorium{IV}{temporalia/resp-ascendensinaltum-CROCHU.gtex}{}}
\newcommand{\lectiovi}{\pars{Lectio VI.}

\noindent Atque hanc glóriam, quam tamen semper hábuit, prius étiam quam mundus hic exsísteret, accépit id témporis, cum natúram humánam indúeret; qua quidem per Spíritum Sanctum glorificáta, quidquid huic cognátum est glóriæ, Spíritus fit párticeps, facto ab Apóstolis inítio.

\noindent Hanc ob causam ait: \emph{Glóriam, quam dedísti mihi, dedi eis; ut sint unum, quemádmodum nos unum sumus; ego in eis et tu in me, ut sint consummáti in unum.} Quámobrem is, qui de púero in virum perféctum augescéndo evásit, ad mensúram ætátis illíus pervénit, quam mente intellégimus; dénique capax glóriæ Spíritus Sancti per statum a vítiis ómnibus aliénum, perque puritátem; hic ígitur est perfécta illa colúmba, ad quam sponsus réspicit, cum ait: \emph{Una est colúmba mea, perfécta mea.}}
\newcommand{\responsoriumvi}{\pars{Responsorium 6.} \scriptura{\Rbar{} Io. 15, 15 \Vbar{} ibid. 15, 14; \textbf{H269}}

\vspace{-4mm}

\responsorium{VIII}{temporalia/resp-iamnondicam-cumdox.gtex}{}}
\newcommand{\evangelium}{\pars{Versus.}

\noindent \Vbardot{} Ascéndo ad Patrem meum, et Patrem vestrum, allelúia.

\noindent \Rbardot{} Deum meum, et Deum vestrum, allelúia.

\vspace{5mm}

\sineinitiali{temporalia/oratiodominica-mat.gtex}

\vspace{5mm}

\pars{Absolutio.}

\cuminitiali{}{temporalia/absolutio-avinculis.gtex}

\vfill
\pagebreak

\cuminitiali{}{temporalia/benedictio-solemn-evangelica.gtex}

\vspace{7mm}

\pars{Evangelium} \scriptura{Io. 17, 1-11}

\noindent Léctio sancti Evangélii secúndum Ioánnem.

\noindent In illo témpore: Sublevátis óculis suis in cælum, dixit Iesus:

\noindent «Pater, venit hora: clarífica Fílium tuum, ut Fílius claríficet te, sicut dedísti ei potestátem omnis carnis, ut omne, quod dedísti ei, det eis vitam ætérnam. Hæc est autem vita ætérna, ut cognóscant te solum verum Deum et, quem misísti, Iesum Christum.

\noindent Ego te clarificávi super terram; opus consummávi, quod dedísti mihi, ut fáciam; et nunc clarífica me tu, Pater, apud temetípsum claritáte, quam habébam, priúsquam mundus esset, apud te.

\noindent Manifestávi nomen tuum homínibus, quos dedísti mihi de mundo. Tui erant, et mihi eos dedísti, et sermónem tuum servavérunt. Nunc cognovérunt quia ómnia, quæ dedísti mihi, abs te sunt, quia verba, quæ dedísti mihi, dedi eis; et ipsi accepérunt et cognovérunt vere quia a te exívi et credidérunt quia tu me misísti.

\noindent Ego pro eis rogo; non pro mundo rogo, sed pro his, quos dedísti mihi, quia tui sunt; et mea ómnia tua sunt, et tua mea; et clarificátus sum in eis. Et iam non sum in mundo, et hi in mundo sunt, et ego ad te vénio».

\vspace{5mm}

\scriptura{Tract. 104,2-3: CCL 36,602-603}

\noindent Ex Tractátibus sancti Augustíni epíscopi in Ioánnem.
	
\noindent \emph{Iesus, sublevátis óculis in cælum dixit: «Pater, venit hora, clarífica Fílium tuum.»} Póterat Dóminus Unigénitus et coætérnus Patri in forma servi et ex forma servi, si hoc opus esset, oráre siléntio; sed ita se Patri exhibére vóluit precatórem, ut meminísset nostrum se esse doctórem.

\noindent Proínde eam quam fecit oratiónem pro nobis, notam fecit et nobis; quóniam tanti magístri non solum ad ipsos sermocinátio, sed étiam pro ipsis ad Patrem orátio, discipulórum est ædificátio.

\noindent Et si illórum qui hæc dicta áderant auditúri, profécto et nostra qui fuerámus conscrípta lectúri. Quaprópter hoc quod ait: \emph{Pater, venit hora, clarífica Fílium tuum,} osténdit omne tempus et quid quando fáceret vel fíeri síneret, ab illo esse dispósitum qui témpori súbditus non est; quóniam quæ futúra erant per síngula témpora, in Dei sapiéntia causas efficiéntes habent, in qua nulla sunt témpora.

\noindent {\color{gray} Non ítaque Christum tempus ut morerétur impégit, sed tempus Christus quo morerétur elégit, qui étiam tempus quo de vírgine natus, cum Patre constítuit, de quo sine témpore natus est. Secúndum quam veram sanámque doctrínam, étiam Paulus apóstolus: \emph{Cum autem venit,} inquit, \emph{plenitúdo témporis, misit Deus Fílium suum}; et Deus per Prophétam: \emph{Témpore,} ait, \emph{acceptábili exaudívi te, et in die salútis adiúvi te;} et rursus Apóstolus: \emph{Ecce nunc tempus acceptábile, ecce nunc dies salútis.} Dicat ergo: \emph{Pater, venit hora,} qui cum Patre dispósuit omnes horas, tamquam dicens: \emph{Pater,} quam propter hómines et apud hómines ad me clarificándum simul constitúimus, \emph{venit hora, clarífica Fílium tuum, ut et Fílius tuus claríficet te.}

\noindent Clarificátum a Patre Fílium nonnúlli accípiunt in hoc quod ei non pepércit, sed pro nobis ómnibus trádidit eum. Sed si passióne clarificátus dícitur, quanto magis resurrectióne? Nam in passióne magis eius humílitas quam cláritas commendátur, Apóstolo teste qui dicit: \emph{Humiliávit semetípsum;} deínde séquitur et de eius clarificatióne iam dicit: \emph{Propter quod et Deus exaltávit illum et donávit ei nomen quod est super omne nomen.} Hæc est clarificátio Dómini nostri Iesu Christi, quæ ab eius resurrectióne sumpsit exórdium.}

\vfill
\pagebreak

\pars{Responsorium 7.} \scriptura{\Rbardot{} Tob. 12, 20 \& Io. 14, 27; \Vbardot{} Io. 16, 7; \textbf{H263}}

\vspace{-5mm}

\responsorium{IV}{temporalia/resp-tempusest-CROCHU-cumdox.gtex}{}

\vfill
\pagebreak}
\newcommand{\laudes}{\vfill
\pagebreak

\pars{Hymnus}

\cuminitiali{I}{temporalia/hym-OptatusVotis.gtex}

\vfill
\pagebreak

\pars{Psalmus 1.} \scriptura{Cf. Ps. 92, 1.2}

\vspace{-4mm}

\antiphona{III a}{temporalia/ant-regnavitdominuspraecinctus.gtex}

%\vspace{-2mm}

\scriptura{Psalmus 92}

%\vspace{-2mm}

\initiumpsalmi{temporalia/ps92-initium-iii-a-auto.gtex}

%\vspace{-1.5mm}

\input{temporalia/ps92-iii-a.tex} \Abardot{}

\vfill
\pagebreak

\pars{Psalmus 2.} \scriptura{Cf. Dn. 3, 74.76; \textbf{H134}}

\vspace{-4mm}

\antiphona{III a trans.}{temporalia/ant-benedicatterradomino.gtex}

%\vspace{-2mm}

\scriptura{Canticum trium puerorum, Dan. 3, 57-88 et 56}

\initiumpsalmi{temporalia/dan3-initium-iii-a-trans2.gtex}

\input{temporalia/dan3-iii-a-sinedox.tex}

\rubrica{Hic non dicitur Gloria Patri, neque Amen.}

\vfill

\antiphona{}{temporalia/ant-benedicatterradomino.gtex}

\vfill
\pagebreak

\pars{Psalmus 3.}

\vspace{-4mm}

\antiphona{I d}{temporalia/ant-deresurrectionetuachriste.gtex}

\vspace{-2mm}

\scriptura{Psalmus 148.}

%\vspace{-2mm}

\initiumpsalmi{temporalia/ps148-initium-i-d-auto.gtex}

%\vspace{-1.5mm}

\input{temporalia/ps148-i-d.tex}

\vfill

\antiphona{}{temporalia/ant-deresurrectionetuachriste.gtex}

\vfill
\pagebreak}
\newcommand{\versuslaudes}{\pars{Versus.} \scriptura{Ps. 46, 6}

\sineinitiali{temporalia/versus-ascenditdeus-communis.gtex}}
\newcommand{\lectiobrevis}{\pars{Lectio brevis.} \scriptura{Ac. 10, 40-43}

\noindent Deus suscitávit Iesum tértia die et dedit eum maniféstum fíeri, non omni pópulo sed téstibus præordinátis a Deo, nobis, qui manducávimus et bíbimus cum illo postquam resurréxit a mórtuis; et præcépit nobis prædicáre pópulo et testificári quia ipse est, qui constitútus est a Deo iudex vivórum et mortuórum. Huic omnes prophétæ testimónium pérhibent remissiónem peccatórum accípere per nomen eius omnes, qui credunt in eum.}
%\newcommand{\responsoriumbreve}{\pars{Responsorium breve.} \scriptura{Cf. Ps. 67, 19}
%
%\cuminitiali{VI}{temporalia/resp-ascendenschristus.gtex}}
\newcommand{\responsoriumbreve}{\pars{Responsorium breve.}

\cuminitiali{VI}{temporalia/resp-christefilideivivi-tp.gtex}}
\newcommand{\benedictus}{\pars{Canticum Zachariæ.} \scriptura{Io. 17, 1-2}

\vspace{-4mm}

\antiphona{VII a}{temporalia/ant-sublevatisoculis.gtex}

\vspace{-2mm}

\scriptura{Lc. 1, 68-79}

%\vspace{-2mm}

\cantusSineNeumas
\initiumpsalmi{temporalia/benedictus-initium-viisoll-a-auto.gtex}

%\vspace{-1.5mm}

\input{temporalia/benedictus-viisoll-a.tex}

\antiphona{}{temporalia/ant-sublevatisoculis.gtex}}
\newcommand{\preces}{\noindent Cum ómnibus, qui in Spíritu Dei iustificáti sunt, \grestar{} nos in láudibus et oratióne coniungámus, dicéntes:

\Rbardot{} Spíritus tuus ádiuvet nos.

\noindent Dómine Iesu, \gredagger{} da, ut hódie a Spíritu Dei agámur, \grestar{} et semper ut fílii Dei ambulémus.

\Rbardot{} Spíritus tuus ádiuvet nos.

\noindent Per Spíritum tuum a Patre efflágita, \grestar{} ut digni efficiámur promissiónibus tuis.

\Rbardot{} Spíritus tuus ádiuvet nos.

\noindent Redde nos magnánimos, \gredagger{} ne quærámus quod nostrum est, \grestar{} sed aliórum bonum pótius intendámus.

\Rbardot{} Spíritus tuus ádiuvet nos.

\noindent Da nobis sciéntiam Dei, \grestar{} ut in cognitióne tui ac Patris per Spíritum Sanctum progrediámur.

\Rbardot{} Spíritus tuus ádiuvet nos.}
\newcommand{\magnificatii}{\pars{Canticum B. Mariæ V.} \scriptura{Io. 1, 41}

\vspace{-6mm}

{
\grechangedim{interwordspacetext}{0.18 cm plus 0.15 cm minus 0.05 cm}{scalable}%
\antiphona{I d\textsuperscript{3}}{temporalia/ant-ambulansiesus.gtex}
\grechangedim{interwordspacetext}{0.22 cm plus 0.15 cm minus 0.05 cm}{scalable}%
}

\vspace{-1.5mm}

\scriptura{Lc. 1, 46-55}

\vspace{-2.5mm}

\cantusSineNeumas
\initiumpsalmi{temporalia/magnificat-initium-isoll-d3.gtex}

\vspace{-1.5mm}

\input{temporalia/magnificat-isoll-d3.tex} \Abardot{}}
%\newcommand{\hebdomada}{infra Hebdom. VI post Pentecosten.}
\newcommand{\oratioLaudes}{\cuminitiali{}{temporalia/oratio6.gtex}}

% LuaLaTeX

\documentclass[a4paper, twoside, 12pt]{article}
\usepackage[latin]{babel}
%\usepackage[landscape, left=3cm, right=1.5cm, top=2cm, bottom=1cm]{geometry} % okraje stranky
%\usepackage[landscape, a4paper, mag=1166, truedimen, left=2cm, right=1.5cm, top=1.6cm, bottom=0.95cm]{geometry} % okraje stranky
\usepackage[landscape, a4paper, mag=1400, truedimen, left=0.5cm, right=0.5cm, top=0.5cm, bottom=0.5cm]{geometry} % okraje stranky

\usepackage{fontspec}
\setmainfont[FeatureFile={junicode.fea}, Ligatures={Common, TeX}, RawFeature=+fixi]{Junicode}
%\setmainfont{Junicode}

% shortcut for Junicode without ligatures (for the Czech texts)
\newfontfamily\nlfont[FeatureFile={junicode.fea}, Ligatures={Common, TeX}, RawFeature=+fixi]{Junicode}

\usepackage{multicol}
\usepackage{color}
\usepackage{lettrine}
\usepackage{fancyhdr}

% usual packages loading:
\usepackage{luatextra}
\usepackage{graphicx} % support the \includegraphics command and options
\usepackage{gregoriotex} % for gregorio score inclusion
\usepackage{gregoriosyms}
\usepackage{wrapfig} % figures wrapped by the text
\usepackage{parcolumns}
\usepackage[contents={},opacity=1,scale=1,color=black]{background}
\usepackage{tikzpagenodes}
\usepackage{calc}
\usepackage{longtable}
\usetikzlibrary{calc}

\setlength{\headheight}{14.5pt}

\input{conventuscommune.tex} % Often used macros
%%%% Preklady jednotlivych zpevu (nektere se opakuji, a je dobre mit je
% vsechny na jedne hromade)

% HOURS ---

\newcommand{\trAntI}{\translatioCantus{Muž boží měl kožený toulec, pečlivě
zavázaný, jenž mu visel na šíji a~často se ho dotýkal.}}

\newcommand{\trAntII}{\translatioCantus{Klíč od~něho tak dobře střežil, že
dokud žil v~těle, nikdo z~jeho žáků nezvěděl, co je uvnitř.}}

\newcommand{\trAntIII}{\translatioCantus{Ale když se odebral z~tohoto
života, schránku otevřeli a~objevili v~ní žíněné roucho a~měděný řetěz
potřísněný krví.}}

\newcommand{\trAntIV}{\translatioCantus{A když prohlédli mistrovo tělo,
nalezli jeho tělo na čtyřech místech hluboce zbrázděno ranami od řetězu.}}

\newcommand{\trAntV}{\translatioCantus{Krev vytékající z~těch ran, místy
prostoupila i~žíněným rouchem.}}

\newcommand{\trCapituli}{\translatioCantus{
Miláčkovi Boha a~lidí,
Mojžíšovi požehnané paměti,~\gredagger{}
dopřál slávu rovnou slávě svatých~\grestar{}
učinil ho mocným na postrach nepřátelům
a~jeho slovy zastavil divy.}}

\newcommand{\trLectioBrevis}{\translatioCantus{
Pamatujte na své představené,
kteří vám hlásali Boží slovo.
Uvažte, jak oni skončili život, a~napodobujte jejich víru.
Ježíš Kristus je stejný včera i~dnes i~navěky.
Nenechte se svést věelijakými cizími naukami.}}

\newcommand{\trRespLaud}{\translatioCantus{Spravedlivého vodil Hospodin~\grestar{}
po přímých stezkách. \Vbardot{} A~ukázal mu Boží království.}}

\newcommand{\trRespLaudB}{\translatioCantus{Na tvých hradbách, Jeruzaléme,
ustanovil jsem strážné;~\grestar{}
budou bdít nad mým lidem. \Vbardot{} Ani ve dne, ani v~noci nesmějí nikdy
mlčet.}}

\newcommand{\trVersus}{\translatioCantus{\Vbardot{} Ústa spravedlivého šeptají moudrost, aleluja.
\Rbardot{} A~jeho jazyk ohlašuje právo, aleluja.}}

\newcommand{\trAntBenedictus}{\translatioCantus{Když na bujné oře vložili
nosítka a~sňali jim uzdu, vydali se přímo k~cele božího muže.}}

\newcommand{\trPreces}{\translatioCantus{
\noindent S vděčností chvalme Krista, dobrého Pastýře, \gredagger{} který dal život za své ovce, \grestar{} a~pokorně ho prosme: \Rbardot{} Pane, buď pastýřem svého lidu.

\noindent Kriste, ty dáváš církvi pastýře, a~jejich službou se ujímáš svého lidu, \grestar{} dej, ať v~lásce těch, kteří nás vedou, poznáváme, jak nás miluješ. \Rbardot{} Pane, buď pastýřem svého lidu.

\noindent Ty stále konáš skrze své zástupce službu pastýře a~učitele, \grestar{} nepřestávej nás nikdy vést prostřednictvím svých služebníků. \Rbardot{} Pane, buď pastýřem svého lidu.

\noindent Ty prokazuješ svému lidu skrze jeho pastýře službu lékaře duše i~těla, \grestar{} ochraňuj náš život a~veď nás ke svatosti. \Rbardot{} Pane, buď pastýřem svého lidu.

\noindent Ty posíláš své svaté, aby slovem i~příkladem vedli tvůj lid k~tobě, \grestar{} na jejich přímluvu nás posiluj, abychom vytrvali na cestě, která vede k~věčnému životu. \Rbardot{} Pane, buď pastýřem svého lidu.}}

\newcommand{\trOrationis}{\translatioCantus{Bože, jenž nám dopřáváš radovat
se z~výroční slavnosti svatého tvého vyznavače Havla, uděl dobrotivě,
abychom když slavíme jeho narození, též se řídili podobou jeho skutků.
Skrze…}}
 % Czech translations of the proper texts

\newcommand{\annusEditionis}{2020}

%%%% Vicekrat opakovane kousky

\newcommand{\anteOrationem}{
  \rubrica{Ante Orationem, cantatur a Superiore:}

  \pars{Supplicatio Litaniæ.}

  \cuminitiali{}{temporalia/supplicatiolitaniae.gtex}

  \pars{Oratio Dominica.}

  \cuminitiali{}{temporalia/oratiodominica.gtex}

  \rubrica{Deinde dicitur ab Hebdomadario:}

  \cuminitiali{}{temporalia/dominusvobiscum-solemnis.gtex}

  \rubrica{In choro monialium loco Dominus vobiscum dicitur:}

  \sineinitiali{temporalia/domineexaudi.gtex}
}

\setlength{\columnsep}{30pt} % prostor mezi sloupci

%%%%%%%%%%%%%%%%%%%%%%%%%%%%%%%%%%%%%%%%%%%%%%%%%%%%%%%%%%%%%%%%%%%%%%%%%%%%%%%%%%%%%%%%%%%%%%%%%%%%%%%%%%%%%
\begin{document}

% Here we set the space around the initial.
% Please report to http://home.gna.org/gregorio/gregoriotex/details for more details and options
\grechangedim{afterinitialshift}{2.2mm}{scalable}
\grechangedim{beforeinitialshift}{2.2mm}{scalable}
\grechangedim{interwordspacetext}{0.22 cm plus 0.15 cm minus 0.05 cm}{scalable}%
\grechangedim{annotationraise}{-0.2cm}{scalable}

% Here we set the initial font. Change 38 if you want a bigger initial.
% Emit the initials in red.
\grechangestyle{initial}{\color{red}\fontsize{38}{38}\selectfont}

\pagestyle{empty}

%%%% Titulni stranka
\begin{titulusOfficii}
\titulus{}
\end{titulusOfficii}

% graphic
%\vspace{1.5cm}
%\begin{center}
%\includegraphics[width=8cm]{emmaus.jpg}
%\end{center}

\vfill

\begin{center}
%Ad usum et secundum consuetudines chori \guillemotright{}Conventus Choralis\guillemotleft.

%Editio Sancti Wolfgangi \annusEditionis
\end{center}

\pagebreak

\renewcommand{\headrulewidth}{0pt} % no horiz. rule at the header
\fancyhf{}
\pagestyle{fancy}

\pars{Oratio ante divinum Officium.}

\lettrine{{\color{red}A}}{peri,} Dómine, os meum ad benedicéndum nomen sanctum tuum:
munda quoque cor meum ab ómnibus vanis, pervérsis, et aliénis
cogitatiónibus:
intelléctum illúmina, afféctum inflámma,
ut digne, atténte ac devóte hoc Offícium recitáre váleam,
et exaudíri mérear ante conspéctum Divínæ Maiestátis tuæ.
Per Christum, Dóminum nostrum.
\Rbardot{} Amen.

Dómine, in unióne illíus divínæ intentiónis,
qua ipse in terris laudes Deo persolvísti,
has tibi Horas \rubricatum{(vel \textnormal{hanc tibi Horam})} persólvo.

%\trOratioAnteOfficium

\vfill

\pars{Oratio post divinum Officium.}

\rubrica{
  Orationem sequentem devote post Officium recitantibus
  Leo Papa X. defectus, et culpas in eo persolvendo ex humana
  fragilitate contractas, indulsit, et dicitur flexis genibus.
}

\lettrine{{\color{red}S}}{acrosánctæ} et indivíduæ Trinitáti,
crucifíxi Dómini nostri Iesu Christi humanitáti,
beatíssimæ et gloriosíssimæ sempérque Vírginis Maríæ
fecúndæ integritáti, 
et ómnium Sanctórum universitáti
sit sempitérna laus, honor, virtus et glória
ab omni creatúra,
nobísque remíssio ómnium peccatórum,
per infiníta sǽcula sæculórum.
\Rbardot{} Amen.

\noindent \Vbardot{} Beáta víscera Maríæ Virginis, quæ portavérunt
ætérni Patris Fílium.\\
\Rbardot{} Et beáta úbera, quæ lactavérunt Christum Dominum.

\rubrica{Et dicitur secreto \textnormal{Pater noster.} et \textnormal{Ave María.}}

%\trOratioPostOfficium

\vfill

\hora{Ad I. Vesperas.} %%%%%%%%%%%%%%%%%%%%%%%%%%%%%%%%%%%%%%%%%%%%%%%%%%%%%
%\sideThumbs{I. Vesperæ}

\cantusSineNeumas

\vspace{0.5cm}
\grechangedim{interwordspacetext}{0.18 cm plus 0.15 cm minus 0.05 cm}{scalable}%
\cuminitiali{}{temporalia/deusinadiutorium-solemnis.gtex}
\grechangedim{interwordspacetext}{0.22 cm plus 0.15 cm minus 0.05 cm}{scalable}%

\vfill
\pagebreak

\pars{Psalmus 1.} \scriptura{Ps. 144, 13; \textbf{H100}}

\vspace{-4mm}

\antiphona{VII c\textsuperscript{2}}{temporalia/ant-regnumtuum.gtex}

\scriptura{Psalmus 144, 10-21.}

\initiumpsalmi{temporalia/ps144ii-initium-vii-c2-auto.gtex}

%\psalmusEtTranslatioT{temporalia/ps144ii-VII-comb.tex}{10cm}
\input{temporalia/ps144ii-VII.tex} \Abardot{}

\vspace{-1cm}

\vfill
\pagebreak

\pars{Psalmus 2.} \scriptura{Ps. 145, 2; \textbf{H100}}

\vspace{-4mm}

\antiphona{IV E}{temporalia/ant-laudabodeum.gtex}

\scriptura{Psalmus 145.}

\initiumpsalmi{temporalia/ps145-initium-iv-E-auto.gtex}

%\psalmusEtTranslatioT{temporalia/ps145-VII-comb.tex}{10cm}
\input{temporalia/ps145-VII.tex} \Abardot{}

\vfill
\pagebreak

\pars{Psalmus 3.} \scriptura{Ps. 146, 1; \textbf{H101}}

\vspace{-4mm}

\antiphona{VIII a}{temporalia/ant-deonostro.gtex}

\scriptura{Psalmus 146.}

\initiumpsalmi{temporalia/ps146-initium-viii-A-auto.gtex}

%\psalmusEtTranslatioT{temporalia/ps146-VII-comb.tex}{10cm}
\input{temporalia/ps146-VII.tex} \Abardot{}

\vfill
\pagebreak

\pars{Psalmus 4.} \scriptura{Ps. 147, 1}

\vspace{-4mm}

\antiphona{E}{temporalia/ant-laudajerusalem.gtex}

\scriptura{Psalmus 147.}

\initiumpsalmi{temporalia/ps147-initium-e-auto.gtex}

%\psalmusEtTranslatioT{temporalia/ps147-VII-comb.tex}{10cm}
\input{temporalia/ps147-VII.tex} \Abardot{}

\vfill
\pagebreak

\pars{Capitulum.} \scriptura{Rom. 11, 33}

\grechangedim{interwordspacetext}{0.12 cm plus 0.15 cm minus 0.05 cm}{scalable}%
\cuminitiali{}{temporalia/capitulum-OAltitudo.gtex}
\grechangedim{interwordspacetext}{0.22 cm plus 0.15 cm minus 0.05 cm}{scalable}

% preklad Jeruz. bible
%\trCapituliI

\vfill

\pars{Responsorium breve.} \scriptura{Ps. 146, 5}

\cuminitiali{VI}{temporalia/resp-magnusdominusnoster.gtex}

%\trResp

\vfill
\pagebreak

\pars{Hymnus} \scriptura{Ambrosius (\olddag{} 397)}

\cuminitiali{I}{temporalia/hym-OLuxBeata-aestivalis.gtex}
\vspace{-3mm}
%\input{hym-OLuxBeata-bohtext.tex}

\vfill
%\pagebreak

\pars{Versus.}

% Versus. %%%
\sineinitiali{temporalia/versus-vespertina.gtex}

%\noindent \trVersus

\vfill
\pagebreak

\magnificati

\vfill
\pagebreak

%\sideThumbs{{\scriptsize{}Fine horarum}}

\anteOrationem

\pagebreak

% Oratio. %%%
\oratioLaudes

\vspace{-1mm}
%\trOrationisI

\vfill

\rubrica{Hebdomadarius dicit iterum Dominus vobiscum, vel cantor dicit:}

\vspace{2mm}

\sineinitiali{temporalia/domineexaudi.gtex}

\rubrica{Postea cantatur a cantore:}

\vspace{2mm}

\cuminitiali{I}{temporalia/benedicamus-dominica-perannum.gtex}

\vspace{1mm}

\vfill
\pagebreak

\hora{Ad Matutinum.} %%%%%%%%%%%%%%%%%%%%%%%%%%%%%%%%%%%%%%%%%%%%%%%%%%%%%
%\sideThumbs{Matutinum}

\vspace{2mm}

\cuminitiali{}{temporalia/dominelabiamea.gtex}

\vspace{2mm}

\pars{Invitatorium.} \scriptura{Ps. 94, 1; Psalmus 94}

\vspace{-6mm}

\antiphona{E}{temporalia/inv-veniteexsultemus.gtex}

\vfill
\pagebreak

\pars{Hymnus.} \scriptura{Adamus Sancti Victoris (\olddag 1146)}

\vspace{-5mm}

\antiphona{VII}{temporalia/hym-SalveDies.gtex}

\scriptura{Non dicitur \textnormal{Amen} in fine.}
%{
%\vspace{-5mm}
%\setlength{\columnsep}{0pt} % prostor mezi sloupci
%\input{hym-SalveDies-bohtext.tex}
%\setlength{\columnsep}{30pt} % prostor mezi sloupci
%}

\vfill
\pagebreak

\subhora{In I. Nocturno}

\pars{Psalmus 1.} \scriptura{Ps. 1, 1}

\vspace{-4mm}

\antiphona{VIII G}{temporalia/ant-beatusvir.gtex}

%\vspace{-5mm}

\scriptura{Ps. 1}

%\vspace{-2mm}

\initiumpsalmi{temporalia/ps1-initium-viii-G-auto.gtex}

%\psalmusEtTranslatioT{temporalia/ps1-I-comb.tex}{10cm}
\input{temporalia/ps1-I.tex} \Abardot{}

\vfill
\pagebreak

\pars{Psalmus 2.} \scriptura{Ps. 2, 11; \textbf{H93}}

\vspace{-4mm}

\antiphona{VII a}{temporalia/ant-servitedomino.gtex}

\vspace{-3mm}

\scriptura{Ps. 2}

\vspace{-2mm}

\initiumpsalmi{temporalia/ps2-initium-vii-a-auto.gtex}

%\psalmusEtTranslatioT{temporalia/ps2-I-comb.tex}{10cm}
\input{temporalia/ps2-I.tex} \Abardot{}

\vfill
\pagebreak

\pars{Psalmus 3.} \scriptura{Ps. 3, 7}

\vspace{-4mm}

\antiphona{VI F}{temporalia/ant-exsurgedominesalvum.gtex}

%\vspace{-5mm}

\scriptura{Ps. 3}

\initiumpsalmi{temporalia/ps3-initium-vi-F-auto.gtex}

%\psalmusEtTranslatioT{temporalia/ps3-I-comb.tex}{10cm}
\input{temporalia/ps3-I.tex} \Abardot{}

\vfill
\pagebreak

\pars{Versus.} \scriptura{Ps. 118, 55}

% Versus. %%%
\sineinitiali{temporalia/versus-memorfui.gtex}

\vspace{5mm}

\sineinitiali{temporalia/oratiodominica-mat.gtex}

\vspace{5mm}

\pars{Absolutio.}

\cuminitiali{}{temporalia/absolutio-exaudi.gtex}

\vfill
\pagebreak

\cuminitiali{}{temporalia/benedictio-solemn-benedictione.gtex}

\vspace{7mm}

\lectioi

\noindent \Vbardot{} Tu autem, Dómine, miserére nobis.
\noindent \Rbardot{} Deo grátias.

\vfill
\pagebreak

\responsoriumi

\vfill
\pagebreak

\cuminitiali{}{temporalia/benedictio-solemn-unigenitus.gtex}

\vspace{7mm}

\lectioii

\noindent \Vbardot{} Tu autem, Dómine, miserére nobis.
\noindent \Rbardot{} Deo grátias.

\vfill
\pagebreak

\responsoriumii

\vfill
\pagebreak

\cuminitiali{}{temporalia/benedictio-solemn-spiritus.gtex}

\vspace{7mm}

\lectioiii

\noindent \Vbardot{} Tu autem, Dómine, miserére nobis.
\noindent \Rbardot{} Deo grátias.

\vfill
\pagebreak

\responsoriumiii

\vfill
\pagebreak

\subhora{In II. Nocturno}

\pars{Psalmus 4.} \scriptura{Ps. 8, 2}

\vspace{-4mm}

\antiphona{I g}{temporalia/ant-quamadmirabileest.gtex}

%\vspace{-5mm}

\scriptura{Ps. 8}

%A\vspace{-2mm}

\initiumpsalmi{temporalia/ps8-initium-i-g-auto.gtex}

%\psalmusEtTranslatioT{temporalia/ps8-I-comb.tex}{10cm}
\input{temporalia/ps8-I.tex} \Abardot{}

\vfill
\pagebreak

\pars{Psalmus 5.} \scriptura{Ps. 9, 5}

\vspace{-4mm}

\antiphona{VIII G}{temporalia/ant-sedistisuperthronum.gtex}

%\vspace{-5mm}

\scriptura{Ps. 9, 2-11}

\initiumpsalmi{temporalia/ps9ii_xi-initium-viii-G-auto.gtex}

%\psalmusEtTranslatioT{temporalia/ps9ii_xi-I-comb.tex}{10cm}
\input{temporalia/ps9ii_xi-I.tex} \Abardot{}

\vfill
\pagebreak

\pars{Psalmus 6.} \scriptura{Ps. 9, 20}

\vspace{-4mm}

\antiphona{I g\textsuperscript{3}}{temporalia/ant-exsurgedominenon.gtex}

%\vspace{-5mm}

\scriptura{Ps. 9, 12-21}

\initiumpsalmi{temporalia/ps9xii_xxi-initium-i-g3-auto.gtex}

%\psalmusEtTranslatioT{temporalia/ps9xii_xxi-I-comb.tex}{10cm}
\input{temporalia/ps9xii_xxi-I.tex} \Abardot{}

\vfill
\pagebreak

\pars{Versus.} \scriptura{Ps. 118, 62}

% Versus. %%%
\sineinitiali{temporalia/versus-medianocte.gtex}

\vspace{5mm}

\sineinitiali{temporalia/oratiodominica-mat.gtex}

\vspace{5mm}

\pars{Absolutio.}

\cuminitiali{}{temporalia/absolutio-ipsius.gtex}

\vfill
\pagebreak

\cuminitiali{}{temporalia/benedictio-solemn-deus.gtex}

\vspace{7mm}

\lectioiv

\noindent \Vbardot{} Tu autem, Dómine, miserére nobis.
\noindent \Rbardot{} Deo grátias.

\vfill
\pagebreak

\responsoriumiv

\vfill
\pagebreak

\cuminitiali{}{temporalia/benedictio-solemn-christus.gtex}

\vspace{7mm}

\lectiov

\noindent \Vbardot{} Tu autem, Dómine, miserére nobis.
\noindent \Rbardot{} Deo grátias.

\vfill
\pagebreak

\responsoriumv

\vfill
\pagebreak

\cuminitiali{}{temporalia/benedictio-solemn-ignem.gtex}

\vspace{7mm}

\lectiovi

\noindent \Vbardot{} Tu autem, Dómine, miserére nobis.
\noindent \Rbardot{} Deo grátias.

\vfill
\pagebreak

\responsoriumvi

\vfill
\pagebreak

\subhora{In III. Nocturno}

\pars{Psalmus 7.} \scriptura{Ps. 9, 22}

\vspace{-4mm}

\antiphona{II D}{temporalia/ant-utquiddomine.gtex}

\vspace{-4mm}

\scriptura{Ps. 9, 22-32}

%\vspace{-2mm}

\initiumpsalmi{temporalia/ps9xxii_xxxii-initium-ii-D-auto.gtex}

%\psalmusEtTranslatioT{temporalia/ps9xxii_xxxii-I-comb.tex}{10cm}
\input{temporalia/ps9xxii_xxxii-I.tex} \Abardot{}

\vfill
\pagebreak

\pars{Psalmus 8.}\scriptura{Ex. 15, 18}

\vspace{-4mm}

\antiphona{IV* e}{temporalia/ant-inaeternum.gtex}

%\vspace{-4mm}

\scriptura{Ps. 9, 33-39}

\initiumpsalmi{temporalia/ps9xxxiii_xxxix-initium-iv_-e-auto.gtex}

%\psalmusEtTranslatioT{temporalia/ps9xxxiii_xxxix-I-comb.tex}{10cm}
\input{temporalia/ps9xxxiii_xxxix-I.tex} \Abardot{}

\vfill
\pagebreak

\pars{Psalmus 9.} \scriptura{Ps. 10, 8}

\vspace{-4mm}

\antiphona{II* f}{temporalia/ant-justusdominus.gtex}

%\vspace{-4mm}

\scriptura{Ps. 10}

%\initiumpsalmi{temporalia/ps10-initium-iv-c-auto.gtex}
\initiumpsalmi{temporalia/ps10-initium-ii_-f.gtex}

%\psalmusEtTranslatioT{temporalia/ps10-I-comb.tex}{10cm}
\input{temporalia/ps10-I.tex} \Abardot{}

\vfill
\pagebreak

\pars{Versus.} \scriptura{Ps. 118, 148}

% Versus. %%%
\sineinitiali{temporalia/versus-praevenerunt.gtex}

\vspace{5mm}

\sineinitiali{temporalia/oratiodominica-mat.gtex}

\vspace{5mm}

\pars{Absolutio.}

\cuminitiali{}{temporalia/absolutio-avinculis.gtex}

\vfill
\pagebreak

\cuminitiali{}{temporalia/benedictio-solemn-evangelica.gtex}

\vspace{7mm}

\lectiovii

\noindent \Vbardot{} Tu autem, Dómine, miserére nobis.
\noindent \Rbardot{} Deo grátias.

\vfill
\pagebreak

\responsoriumvii

\vfill
\pagebreak

\cuminitiali{}{temporalia/benedictio-solemn-divinum.gtex}

\vspace{7mm}

\lectioviii

\noindent \Vbardot{} Tu autem, Dómine, miserére nobis.
\noindent \Rbardot{} Deo grátias.

\vfill
\pagebreak

\responsoriumviii

\vfill
\pagebreak

\cuminitiali{}{temporalia/benedictio-solemn-adsocietatem.gtex}

\vspace{7mm}

\lectioix

\noindent \Vbardot{} Tu autem, Dómine, miserére nobis.
\noindent \Rbardot{} Deo grátias.

\vfill
\pagebreak

% Te Deum

{
\pars{Hymnus Ambrosianus} \scriptura{Tonus Solemnis}

\vspace{-2mm}

\grechangedim{interwordspacetext}{0.26 cm plus 0.15 cm minus 0.05 cm}{scalable}%
\cuminitiali{III}{temporalia/tedeum-solemnis-gn.gtex}
\grechangedim{interwordspacetext}{0.22 cm plus 0.15 cm minus 0.05 cm}{scalable}%
}

\vfill
\pagebreak

\rubrica{Reliqua omittuntur, nisi Laudes separandæ sint.}

\pars{Oratio}

\noindent \Vbardot{} Dómine, exáudi oratiónem meam.

\noindent \Rbardot{} Et clamor meus ad te véniat.

Orémus:

\oratioLaudes

\vspace{7mm}

\pars{Conclusio}

\noindent \Vbardot{} Dómine, exáudi oratiónem meam.

\noindent \Rbardot{} Et clamor meus ad te véniat.

\noindent \Vbardot{} Benedicámus Dómino, allelúia, allelúia.

\noindent \Rbardot{} Deo grátias, allelúia, allelúia.

\noindent \Vbardot{} Fidélium ánimæ per misericórdiam Dei requiéscant in pace.

\noindent \Rbardot{} Amen.

\vfill
\pagebreak

\hora{Ad Laudes.} %%%%%%%%%%%%%%%%%%%%%%%%%%%%%%%%%%%%%%%%%%%%%%%%%%%%%
%\sideThumbs{Laudes}

\cantusSineNeumas

\vspace{0.5cm}
\grechangedim{interwordspacetext}{0.18 cm plus 0.15 cm minus 0.05 cm}{scalable}%
\cuminitiali{}{temporalia/deusinadiutorium-alter.gtex}
\grechangedim{interwordspacetext}{0.22 cm plus 0.15 cm minus 0.05 cm}{scalable}%

\vfill
%\pagebreak

\pars{Psalmus 1.}

\vspace{-4mm}

\antiphona{VI F}{temporalia/ant-alleluia1.gtex}

\scriptura{Psalmus 50.}

\initiumpsalmi{temporalia/ps50-initium-vi-F-auto.gtex}

%\psalmusEtTranslatioT{temporalia/ps50-I-comb.tex}{10cm}
\input{temporalia/ps50-I.tex}

\vfill
\pagebreak

\pars{Psalmus 2.}

\scriptura{Psalmus 117.}

\initiumpsalmi{temporalia/ps117-initium-vi-F-auto.gtex}

%\psalmusEtTranslatioT{temporalia/ps117-I-comb.tex}{10cm}
\input{temporalia/ps117-I.tex}

\vfill
\pagebreak

\pars{Psalmus 3.}

\scriptura{Psalmus 62.}

\initiumpsalmi{temporalia/ps62-initium-vi-F-auto.gtex}

%\psalmusEtTranslatioT{temporalia/ps62-I-comb.tex}{10cm}
\input{temporalia/ps62-I.tex}

\vfill

\vspace{-6mm}

\antiphona{}{temporalia/ant-alleluia1.gtex} % repeat the antiphon - new page

\vfill
\pagebreak

\pars{Psalmus 4.} \scriptura{Dan. 3, 22-26; \textbf{H422}}

\vspace{-4mm}

\antiphona{VIII G}{temporalia/ant-trespueri.gtex}

\scriptura{Canticum trium puerorum, Dan. 3, 57-88 et 56}

\initiumpsalmi{temporalia/dan3-initium-viii-G-auto.gtex}

%\psalmusEtTranslatioT{temporalia/dan3-comb.tex}{10cm}
\input{temporalia/dan3.tex}

\rubrica{Hic non dicitur Gloria Patri, neque Amen.}

\vfill

\vspace{-6mm}

\antiphona{}{temporalia/ant-trespueri.gtex} % repeat the antiphon - new page

\vfill
\pagebreak

\pars{Psalmus 5.}

\vspace{-4mm}

\antiphona{VIII G}{temporalia/ant-alleluia2.gtex}

\scriptura{Psalmus 148.}

\initiumpsalmi{temporalia/ps148-initium-viii-G-auto.gtex}

%\psalmusEtTranslatioT{temporalia/ps148-I-comb.tex}{10cm}
\input{temporalia/ps148-I.tex}

\rubrica{Hic non dicitur Gloria Patri.}

\vfill
\pagebreak

%
\scriptura{Psalmus 149.}

\initiumpsalmi{temporalia/ps149-initium-viii-G-auto.gtex}

%\psalmusEtTranslatioT{temporalia/ps149-I-comb.tex}{10cm}
\input{temporalia/ps149-I.tex}

\rubrica{Hic non dicitur Gloria Patri.}

\vfill
\pagebreak

%
\scriptura{Psalmus 150.}

\initiumpsalmi{temporalia/ps150-initium-viii-G-auto.gtex}

%\psalmusEtTranslatioT{temporalia/ps150-I-comb.tex}{10cm}
\input{temporalia/ps150-I.tex}

\vfill

\vspace{-6mm}

\antiphona{}{temporalia/ant-alleluia2.gtex} % repeat the antiphon - new page

\vfill
\pagebreak

\pars{Capitulum.} \scriptura{Ac. 7, 12}

\grechangedim{interwordspacetext}{0.12 cm plus 0.15 cm minus 0.05 cm}{scalable}%
\cuminitiali{}{temporalia/capitulum-Benedictio.gtex}
\grechangedim{interwordspacetext}{0.22 cm plus 0.15 cm minus 0.05 cm}{scalable}

% preklad Jeruz. bible
%\trCapituliI

\vfill

\pars{Responsorium breve.} \scriptura{Ps. 118, 36-37}

\cuminitiali{IV}{temporalia/resp-inclinacormeum.gtex}

%\trResp

\vfill
\pagebreak

\pars{Hymnus} \scriptura{Gregorius Magnus (\olddag{} 604)}

\cuminitiali{IV}{temporalia/hym-EcceJamNoctis.gtex}
\vspace{-3mm}
%\input{hym-EcceJamNocis-bohtext.tex}

\vfill
%\pagebreak

\pars{Versus.} \scriptura{Ps. 92, 1}

% Versus. %%%
\sineinitiali{temporalia/versus-dominusregnavit.gtex}

%\noindent \trVersus

\vfill
\pagebreak

\benedictus

\vspace{-1cm}

\vfill
\pagebreak

%\sideThumbs{{\scriptsize{}Fine horarum}}

\anteOrationem

\pagebreak

% Oratio. %%%
\oratioLaudes

\vspace{-1mm}
%\trOrationisI

\vfill

\rubrica{Hebdomadarius dicit iterum Dominus vobiscum, vel cantor dicit:}

\vspace{2mm}

\sineinitiali{temporalia/domineexaudi.gtex}

\rubrica{Postea cantatur a cantore:}

\vspace{2mm}

\cuminitiali{I}{temporalia/benedicamus-dominica-perannum.gtex}

\vspace{1mm}

\vfill
\pagebreak

\hora{Ad II. Vesperas.} %%%%%%%%%%%%%%%%%%%%%%%%%%%%%%%%%%%%%%%%%%%%%%%%%%%%%
%\sideThumbs{II. Vesperæ}

\cantusSineNeumas

%\vspace{0.5cm}
\grechangedim{interwordspacetext}{0.18 cm plus 0.15 cm minus 0.05 cm}{scalable}%
\cuminitiali{}{temporalia/deusinadiutorium-solemnis.gtex}
\grechangedim{interwordspacetext}{0.22 cm plus 0.15 cm minus 0.05 cm}{scalable}%

\vfill
%\pagebreak

\vspace{-2mm}

\pars{Psalmus 1.} \scriptura{Ps. 109, 1; \textbf{H91}}

\vspace{-4mm}

\antiphona{VII c\textsuperscript{2}}{temporalia/ant-dixitdominus.gtex}

\vspace{-4mm}

\scriptura{Psalmus 109.}

\initiumpsalmi{temporalia/ps109-initium-vii-c2-auto.gtex}

%\psalmusEtTranslatioT{temporalia/ps109-I-comb.tex}{10cm}
\input{temporalia/ps109-I.tex} \Abardot{}

\vspace{-1cm}

\vfill
\pagebreak

\pars{Psalmus 2.} \scriptura{Ps. 110, 8; \textbf{H91}}

\vspace{-4mm}

\antiphona{IV g}{temporalia/ant-fideliaomnia.gtex}

\scriptura{Psalmus 110.}

\initiumpsalmi{temporalia/ps110-initium-iv-g-auto.gtex}

%\psalmusEtTranslatioT{temporalia/ps110-I-comb.tex}{10cm}
\input{temporalia/ps110-I.tex} \Abardot{}

\vfill
\pagebreak

\pars{Psalmus 3.} \scriptura{Ps. 111, 1; \textbf{H92}}

\vspace{-4mm}

\antiphona{IV a}{temporalia/ant-inmandatis.gtex}

\scriptura{Psalmus 111.}

\initiumpsalmi{temporalia/ps111-initium-iv-a-auto.gtex}

%\psalmusEtTranslatioT{temporalia/ps111-I-comb.tex}{10cm}
\input{temporalia/ps111-I.tex} \Abardot{}

\vfill
\pagebreak

\pars{Psalmus 4.} \scriptura{Ps. 112, 2; \textbf{H92}}

\vspace{-4mm}

\antiphona{VII c}{temporalia/ant-sitnomendomini.gtex}

\scriptura{Psalmus 112.}

\initiumpsalmi{temporalia/ps112-initium-vii-c-auto.gtex}

%\psalmusEtTranslatioT{temporalia/ps112-I-comb.tex}{10cm}
\input{temporalia/ps112-I.tex} \Abardot{}

\vfill
\pagebreak

\pars{Capitulum.} \scriptura{2 Cor. 1, 3-4}

\grechangedim{interwordspacetext}{0.12 cm plus 0.15 cm minus 0.05 cm}{scalable}%
\cuminitiali{}{temporalia/capitulum-BenedictusDeus.gtex}
\grechangedim{interwordspacetext}{0.22 cm plus 0.15 cm minus 0.05 cm}{scalable}

% preklad Jeruz. bible
%\trCapituliI

\vfill

\pars{Responsorium breve.} \scriptura{Ps. 103, 24}

\cuminitiali{VI}{temporalia/resp-quammagnificata.gtex}

%\trResp

\vfill
\pagebreak

\pars{Hymnus} \scriptura{Gregorius Magnus (\olddag{} 604)}

\cuminitiali{I}{temporalia/hym-LucisCreator-aestivalis.gtex}
\vspace{-3mm}
%\begin{translatioMulticol}{3}
Tvůrce světa předobrý,\\
tys ustanovil denní řád\\
a proudy světla rozhodil,\\
když světu základy jsi klad.\\
\\
A spojils ráno s večerem\\
a dnem tu dobu nazýváš;\\
hle padá temné noci stín -\\
slyš prosbu, vyslyš nářek náš.\columnbreak

Ach, nedej, by nás stihla smrt,\\
když svědomí nám tíží hřích,\\
když nemyslíme na věčnost\\
v té síti hříchů šalebných.\\
\\
Vzbuď naši touhu po nebi,\\
kde věčný život čeká nás,\\
a pomoz odložit vše zlé\\
a smýti z duše každý kaz.\columnbreak

To splň nám, dobrý Otče náš,\\
i ty, jenž rovné božství máš,\\
i Duchu, který těšíš nás\\
a vládneš, Bože, v každý čas.\\
Amen. 
\end{translatioMulticol}


\vfill
%\pagebreak

\pars{Versus.} \scriptura{Ps. 140, 2}

% Versus. %%%
\sineinitiali{temporalia/versus-dirigatur.gtex}

%\noindent \trVersus

\vfill
\pagebreak

\magnificatii

\vfill
\pagebreak

%\sideThumbs{{\scriptsize{}Fine horarum}}

\anteOrationem

\pagebreak

% Oratio. %%%
\oratioLaudes

\vspace{-1mm}
%\trOrationisI

\vfill

\rubrica{Hebdomadarius dicit iterum Dominus vobiscum, vel cantor dicit:}

\vspace{2mm}

\sineinitiali{temporalia/domineexaudi.gtex}

\rubrica{Postea cantatur a cantore:}

\vspace{2mm}

\cuminitiali{I}{temporalia/benedicamus-dominica-perannum.gtex}

\vspace{1mm}

\end{document}

