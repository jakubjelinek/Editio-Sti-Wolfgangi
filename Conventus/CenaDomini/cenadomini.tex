% LuaLaTeX

\documentclass[a4paper, twoside, 12pt]{article}
\usepackage[latin]{babel}
%\usepackage[landscape, left=3cm, right=1.5cm, top=2cm, bottom=1cm]{geometry} % okraje stranky
\usepackage[landscape, a4paper, mag=1166, truedimen, left=2cm, right=1.5cm, top=1.6cm, bottom=0.95cm]{geometry} % okraje stranky

\usepackage{fontspec}
\setmainfont[FeatureFile={junicode.fea}, Ligatures={Common, TeX}, RawFeature=+fixi]{Junicode}
%\setmainfont{Junicode}

% shortcut for Junicode without ligatures (for the Czech texts)
\newfontfamily\nlfont[FeatureFile={junicode.fea}, Ligatures={Common, TeX}, RawFeature=+fixi]{Junicode}

% Hebrew font: http://scripts.sil.org/cms/scripts/page.php?site_id=nrsi&id=SILHebrUnic2
\newfontfamily\hebfont[Scale=1]{Ezra SIL}

\usepackage{multicol}
\usepackage{color}
\usepackage{lettrine}
\usepackage{fancyhdr}

% usual packages loading:
\usepackage{luatextra}
\usepackage{graphicx} % support the \includegraphics command and options
\usepackage{gregoriotex} % for gregorio score inclusion
\usepackage{gregoriosyms}
\usepackage{wrapfig} % figures wrapped by the text
\usepackage{parcolumns}
\usepackage[contents={},opacity=1,scale=1,color=black]{background}
\usepackage{tikzpagenodes}
\usepackage{calc}
\usepackage{longtable}
\usetikzlibrary{calc}

\setlength{\headheight}{14.5pt}

\input{conventuscommune.tex} % Often used macros
%%%% Preklady jednotlivych zpevu (nektere se opakuji, a je dobre mit je
% vsechny na jedne hromade)

% HOURS ---

\newcommand{\trAntI}{\translatioCantus{Muž boží měl kožený toulec, pečlivě
zavázaný, jenž mu visel na šíji a~často se ho dotýkal.}}

\newcommand{\trAntII}{\translatioCantus{Klíč od~něho tak dobře střežil, že
dokud žil v~těle, nikdo z~jeho žáků nezvěděl, co je uvnitř.}}

\newcommand{\trAntIII}{\translatioCantus{Ale když se odebral z~tohoto
života, schránku otevřeli a~objevili v~ní žíněné roucho a~měděný řetěz
potřísněný krví.}}

\newcommand{\trAntIV}{\translatioCantus{A když prohlédli mistrovo tělo,
nalezli jeho tělo na čtyřech místech hluboce zbrázděno ranami od řetězu.}}

\newcommand{\trAntV}{\translatioCantus{Krev vytékající z~těch ran, místy
prostoupila i~žíněným rouchem.}}

\newcommand{\trCapituli}{\translatioCantus{
Miláčkovi Boha a~lidí,
Mojžíšovi požehnané paměti,~\gredagger{}
dopřál slávu rovnou slávě svatých~\grestar{}
učinil ho mocným na postrach nepřátelům
a~jeho slovy zastavil divy.}}

\newcommand{\trLectioBrevis}{\translatioCantus{
Pamatujte na své představené,
kteří vám hlásali Boží slovo.
Uvažte, jak oni skončili život, a~napodobujte jejich víru.
Ježíš Kristus je stejný včera i~dnes i~navěky.
Nenechte se svést věelijakými cizími naukami.}}

\newcommand{\trRespLaud}{\translatioCantus{Spravedlivého vodil Hospodin~\grestar{}
po přímých stezkách. \Vbardot{} A~ukázal mu Boží království.}}

\newcommand{\trRespLaudB}{\translatioCantus{Na tvých hradbách, Jeruzaléme,
ustanovil jsem strážné;~\grestar{}
budou bdít nad mým lidem. \Vbardot{} Ani ve dne, ani v~noci nesmějí nikdy
mlčet.}}

\newcommand{\trVersus}{\translatioCantus{\Vbardot{} Ústa spravedlivého šeptají moudrost, aleluja.
\Rbardot{} A~jeho jazyk ohlašuje právo, aleluja.}}

\newcommand{\trAntBenedictus}{\translatioCantus{Když na bujné oře vložili
nosítka a~sňali jim uzdu, vydali se přímo k~cele božího muže.}}

\newcommand{\trPreces}{\translatioCantus{
\noindent S vděčností chvalme Krista, dobrého Pastýře, \gredagger{} který dal život za své ovce, \grestar{} a~pokorně ho prosme: \Rbardot{} Pane, buď pastýřem svého lidu.

\noindent Kriste, ty dáváš církvi pastýře, a~jejich službou se ujímáš svého lidu, \grestar{} dej, ať v~lásce těch, kteří nás vedou, poznáváme, jak nás miluješ. \Rbardot{} Pane, buď pastýřem svého lidu.

\noindent Ty stále konáš skrze své zástupce službu pastýře a~učitele, \grestar{} nepřestávej nás nikdy vést prostřednictvím svých služebníků. \Rbardot{} Pane, buď pastýřem svého lidu.

\noindent Ty prokazuješ svému lidu skrze jeho pastýře službu lékaře duše i~těla, \grestar{} ochraňuj náš život a~veď nás ke svatosti. \Rbardot{} Pane, buď pastýřem svého lidu.

\noindent Ty posíláš své svaté, aby slovem i~příkladem vedli tvůj lid k~tobě, \grestar{} na jejich přímluvu nás posiluj, abychom vytrvali na cestě, která vede k~věčnému životu. \Rbardot{} Pane, buď pastýřem svého lidu.}}

\newcommand{\trOrationis}{\translatioCantus{Bože, jenž nám dopřáváš radovat
se z~výroční slavnosti svatého tvého vyznavače Havla, uděl dobrotivě,
abychom když slavíme jeho narození, též se řídili podobou jeho skutků.
Skrze…}}
 % Czech translations of the proper texts

\newcommand{\annusEditionis}{2016}

\def\hebinitial#1{%
\leavevmode{\newbox\hebbox\setbox\hebbox\hbox{\hebfont{#1}\hskip 1mm}\kern -\wd\hebbox\hbox{\hebfont{#1}\hskip 1mm}}%
}

%%%% Vicekrat opakovane kousky

\newcommand{\anteOrationem}{
  \rubrica{Ante Orationem, cantatur a Superiore:}

  \pars{Supplicatio Litaniæ.}

  \gregorioscore{temporalia/supplicatiolitaniae.gtex}

  \pars{Oratio Dominica.}

  \gregorioscore{temporalia/oratiodominica.gtex}

  \rubrica{Deinde dicitur ab Hebdomadario:}

  \gregorioscore{temporalia/dominusvobiscum-solemnis.gtex}

  \rubrica{In choro monialium loco Dominus vobiscum dicitur:}

  \gregorioscore{temporalia/domineexaudi.gtex}
}

\newcommand{\tuAutem}{
  \vfill

  \gregorioscore{temporalia/tuautem.gtex}
}

\setlength{\columnsep}{30pt} % prostor mezi sloupci

%%%%%%%%%%%%%%%%%%%%%%%%%%%%%%%%%%%%%%%%%%%%%%%%%%%%%%%%%%%%%%%%%%%%%%%%%%%%%%%%%%%%%%%%%%%%%%%%%%%%%%%%%%%%%
\begin{document}

% Here we set the space around the initial.
% Please report to http://home.gna.org/gregorio/gregoriotex/details for more details and options
\grechangedim{afterinitialshift}{2.2mm}{scalable}
\grechangedim{beforeinitialshift}{2.2mm}{scalable}
\grechangedim{interwordspacetext}{0.32 cm plus 0.15 cm minus 0.05 cm}{scalable}%
\grechangedim{annotationraise}{-0.2cm}{scalable}

% Here we set the initial font. Change 38 if you want a bigger initial.
% Emit the initials in red.
\grechangestyle{initial}{\color{red}\fontsize{38}{38}\selectfont}

\pagestyle{empty}

\begin{titulusOfficii}
\nomenFesti{Feria V in Cœna Domini.}
\celebratio{Duplex 1. classis.}
\end{titulusOfficii}

%\hora{Ad Matutinum.}
%%%%%%%%%%%%%%%%%%%%%%%%%%%%%%%%%%%%%%%%%%%%%%%%%%%%%%%%%%
\sideThumbs{Matutinum}

\vfill

% Incipit Lamentátio Ieremíæ Prophétæ.
\pars{Lectio I.} \scriptura{Lam. 1, 1-5}

\noindent Incípit Lamentátio Ieremíæ Prophétæ.

\textusEtTranslatio{
\hebinitial{א} Quómodo sedet sola cívitas plena pópulo! Facta est quasi vídua Dómina géntium; Princeps provinciárum facta est sub tribúto.\\
\hebinitial{ב} Plorans plorávit in nocte, et lácrimæ eius in maxíllis eius: non est qui consolétur eam ex ómnibus caris eius; Omnes amíci eius sprevérunt eam, et facti sunt ei inimíci.\\
\hebinitial{ג} Migrávit Iudas propter afflictiónem, et multitúdinem servitútis; habitávit inter gentes, nec invénit réquiem: Omnes persecutóres eius apprehendérunt eam inter angústias.\\
\hebinitial{ד} Viæ Sion lugent, eo quod non sint qui véniant ad solemnitátem: omnes portæ eius destrúctæ, sacerdótes eius geméntes; vírgines eius squálidæ, et ipsa oppréssa amaritúdine.\\
\hebinitial{ה} Facti sunt hostes eius in cápite; inimíci eius locupletáti sunt: quia Dóminus locútus est super eam propter multitúdinem iniquitátum eius. Párvuli eius ducti sunt in captivitátem ante fáciem tribulántis.
Ierúsalem, Ierúsalem, convértere ad Dóminum Deum tuum.
}{\trMatLecI}{10cm}

\vfill

\pars{Lectio II.} \scriptura{Lam. 1, 6-9}

\textusEtTranslatio{
\hebinitial{ו} Et egréssus est a fília Sion omnis decor eius; facti sunt princípes eius velut aríetes non inveniéntes páscua, et abiérunt absque fortitúdine ante fáciem subsequéntis.\\
\hebinitial{ז} Recordáta est Ierúsalem diérum afflictiónis suæ, et prævaricatiónis, ómnium desiderabílium suórum, quæ habúerat a diébus antíquis, cum cáderet pópulus eius in manu hostíli, et non esset auxiliátor: vidérunt eam hostes, et derisérunt sábbata eius.\\
\hebinitial{ח} Peccátum peccávit Ierúsalem, proptérea instábilis facta est; omnes qui glorificábant eam sprevérunt illam, quia vidérunt ignomíniam eius: ipsa autem gemens convérsa est retrórsum.\\
\hebinitial{ט} Sordes eius in pédibus eius, nec recordáta est finis sui; depósita est veheménter, non habens consolatórem. Vide, Dómine, afflictiónem meam, quóniam eréctus est inimícus.
Ierúsalem, Ierúsalem, convértere ad Dóminum Deum tuum.
}{\trMatLecII}{10cm}

\vfill

\pars{Lectio III.} \scriptura{Lam. 1, 10-14}

\textusEtTranslatio{
\hebinitial{י} Manum suam misit hostis ad ómnia desiderabília eius, quia vidit gentes ingréssas sanctuárium suum, de quibus præcéperas ne intrárent in ecclésiam tuam.\\
\hebinitial{כ} Omnis pópulus eius gemens, et quærens panem; dedérunt pretiósa quæque pro cibo ad refocillándam ánimam. Vide, Dómine, et consídera quóniam facta sum vilis!\\
\hebinitial{ל} O vos omnes qui transítis per viam, atténdite, et vidéte si est dolor sicut dolor meus! Quóniam vindemiávit me, ut locútus est Dóminus, in die iræ furóris sui.\\
\hebinitial{מ} De excelso misit ignem in óssibus meis, et erudívit me: expándit rete pédibus meis, convértit me retrórsum; pósuit me desolátam, tota die mœróre conféctam.\\
\hebinitial{נ} Vigilávit iugum iniquitátum meárum; in manu eius convolútæ sunt, et impósitæ collo meo. Infirmáta est virtus mea: dedit me Dóminus in manu de qua non pótero súrgere.
Ierúsalem, Ierúsalem, convértere ad Dóminum Deum tuum.
}{\trMatLecIII}{10cm}

\vfill

\sineinitiali{temporalia/tonus-lectionis-solemnis.gtex}

\vspace{-0.4cm}

\vfill

\pars{Lectio IV.} \scriptura{In Psalmum 54. ad 1. versum}

\noindent Ex Tractátu sancti Augustíni Epíscopi super Psalmos.

\textusEtTranslatio{
Exáudi, Deus, oratiónem meam, et ne despéxeris deprecatiónem meam:
inténde mihi, et exáudi me.
Satagéntis, sollíciti, in tribulatióne pósiti, verba sunt ista.
Orat multa pátiens, de malo liberári desíderans.
Súperest ut videámus in quo malo sit: et cum dícere cœperit:
agnoscámus ibi nos esse:
ut communicáta tribulatióne, coniungámus oratiónem.
Contristátus sum, inquit, in exercitatióne mea, et conturbátus sum.
Ubi contristátus?
Ubi conturbátus?
In exercitatióne mea, inquit.
Hómines malos, quos pátitur, commemorátus est:
eamdémque passiónem malórum hóminum, exercitatiónem suam dixit.
Ne putétis gratis esse malos in hoc mundo,
et nihil boni de illis ágere Deum.
Omnis malus aut ídeo vivit, ut corrigátur;
aut ídeo vivit, ut per illum bonus exerceátur.
}{\trMatLecIV}{10cm}

\pars{Lectio V.}

\textusEtTranslatio{
Utínam ergo qui nos modo exércent, convertántur, et nobíscum exerceántur:
tamen quámdiu ita sunt ut exérceant, non eos odérimus:
quia in eo quod malus est quis eórum, utrum usque in inem perseveratúrus sit, ignorámus.
Et plerúmque cum tibi vidéris odísse inimícum, fratrem odísti, et nescis.
Diábolus, et ángeli eius in Scriptúris sanctis manifestáti sunt nobis,
quod ad ignem ætérnum sint destináti.
Ipsórum tantum desperánda est corréctio, contra quos habémus occúltam luctam:
ad quam luctam nos armat Apóstolus, dicens:
Non est nobis colluctátio advérsus carnem et sánguinem:
id est, non advérsus hómines, quos vidétis,
sed advérsus príncipes, et potestátes, et rectóres mundi, tenebrárum harum.
Ne forte cum dixísset, mundi, intellígeres dǽmones esse rectóres cæli et terræ,
mundi dixit, tenebrárum harum: mundi dixit amatórum mundi:
mundi dixit, impiórum et iniquórum: mundi dixit, de quo dicit Evangélium:
Et mundus eum non cognóvit.
}{\trMatLecV}{10cm}

\pars{Lectio VI.}

\textusEtTranslatio{
Quóniam vidi iniquitátem, et contradictiónem in civitáte.
Atténde glóriam crucis ipsíus.
Iam in fronte regum crux illa ixa est,
cui inimíci insultavérunt.
Eféctus probávit virtútem: dómuit orbem non ferro, sed ligno.
Lignum crucis contuméliis dignum visum est inimícis,
et ante ipsum lignum stantes caput agitábant, et dicébant:
Si Fílius Dei est, descéndat de cruce.
Extendébat ille manus suas ad pópulum non credéntem et contradicéntem.
Si enim iustus est, qui ex ide vivit;
iníquus est, qui non habet idem.
Quod ergo hic ait, iniquitátem: perfídiam intéllige.
Vidébat ergo Dóminus in civitáte iniquitátem et contradictiónem,
et extendébat manus suas ad pópulum non credéntem, et contradicéntem:
et tamen et ipsos exspéctans dicébat:
Pater, ignósce illis, quia nésciunt quid fáciunt.
}{\trMatLecVI}{10cm}

\pars{Lectio VII.} \scriptura{I Cor. 11, 17-22}

\noindent De Epístola prima beáti Pauli Apóstoli ad Corínthios.

\textusEtTranslatio{
Hoc autem præcípio: non laudans quod non in mélius,
sed in detérius convenítis.
Primum quidem conveniéntibus vobis in Ecclésiam,
áudio scissúras esse inter vos, et ex parte credo.
Nam opórtet et hǽreses esse, ut et qui probáti sunt,
manifésti iant in vobis.
Conveniéntibus ergo vobis in unum,
iam non est Domínicam cenam manducáre.
Unusquísque enim suam cenam præsúmit ad manducándum,
Et álius quidem ésurit, álius autem ébrius est.
Numquid domos non habétis ad manducándum, et bibéndum?
Aut Ecclésiam Dei contémnitis, et confúnditis eos, qui non habent?
Quid dicam vobis?
Laudo vos?
In hoc non laudo.
}{\trMatLecVII}{10cm}

\pars{Lectio VIII.} \scriptura{I Cor. 11, 23-26}

\textusEtTranslatio{
Ego enim accépi a Dómino quod et trádidi vobis,
quóniam Dóminus Iesus, in qua nocte tradebátur, accépit panem,
et grátias agens fregit, et dixit:
«\textsc{Accípite et manducáte: hoc est corpus meum, quod pro vobis tradétur:
hoc fácite in meam commemoratiónem}».
Simíliter et cálicem, post\-quam cenávit, dicens:
«\textsc{Hic calix novum testaméntum est in meo sánguine:
hoc fácite, quotiescúmque bibétis, in meam commemoratiónem}».
Quotiescúmque enim manducábitis panem hunc, et cálicem bibétis:
mortem Dómini annuntiábitis, donec véniat.
}{\trMatLecVIII}{10cm}

\pars{Lectio IX.} \scriptura{I Cor. 11, 27-34}

\textusEtTranslatio{
Itaque quicúmque manducáverit panem hunc vel bíberit cálicem Dómini indígne,
reus erit córporis et sánguinis Dómini.
Probet autem seípsum homo: et sic de pane illo edat et de cálice bibat.
Qui enim mandúcat et bibit indígne, iudícium sibi mandúcat et bibit;
non diiúdicans corpus Dómini.
Ideo inter vos multi infírmi et imbecílles,
et dórmiunt multi.
Quod si nosmetípsos diiudicarémus, non útique iudicarémur.
Dum iudicámur autem, a Dómino corrípimur,
ut non cum hoc mundo damnémur.
Itaque, fratres mei, cum convenítis ad manducándum, ínvicem exspectáte.
Si quis ésurit, domi mandúcet: ut non in iudícium conveniátis.
Cétera autem, cum vénero dispónam.
}{\trMatLecIX}{10cm}

\end{document}
