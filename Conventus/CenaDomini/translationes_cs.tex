%%%% Preklady jednotlivych zpevu (nektere se opakuji, a je dobre mit je
% vsechny na jedne hromade)

\newcommand{\trMatLecI}{\translatioCantus{
\hebinitial{א} Jak samotno sedí město, které mělo tolik lidu! Je jako vdova, ač bylo mocné mezi pronárody. Bylo kněžnou mezi krajinami, a je podrobeno nuceným pracím.\\
\hebinitial{ב} Za noci pláče a pláče, po líci slzy jí kanou, ze všech jejích milovníků není nikdo, kdo by ji potěšil. Všichni její druhové se k ní zachovali věrolomně, stali se jejími nepřáteli.\\
\hebinitial{ג} Juda odešel do vyhnanství ponížen a nesmírně zotročen. Usadil se mezi pronárody, odpočinutí nenachází. Všichni jeho pronásledovatelé ho dostihli v jeho úzkostech.\\
\hebinitial{ד} Cesty na Sión truchlí, nikdo nepřichází ke slavnosti. Všechny jeho brány jsou zpustošené, jeho kněží vzdychají, jeho panny jsou zarmoucené, hořko je siónské dceři.\\
\hebinitial{ה} Její protivníci nabyli vrchu, její nepřátelé si žijí v klidu, neboť Hospodin ji zarmoutil pro množství jejích nevěrností. Její pacholátka odešla před tváří protivníka do zajetí.
Jeruzaléme, Jeruzaléme, obrať se k Pánu, Bohu svému.}}

\newcommand{\trMatLecII}{\translatioCantus{
\hebinitial{ו} Odešla od siónské dcery veškerá její důstojnost. Její velmožové jsou jako jeleni, když nenacházejí pastvu: táhnou vysíleni před tváří pronásledovatele.\\
\hebinitial{ז} Jeruzalém se rozpomíná ve dnech svého ponížení a zmateného toulání na všechno, co míval za žádoucí ode dnů dávnověkých. Když jeho lid padl do rukou protivníka, nebyl nikdo, kdo by mu pomohl. Protivníci to viděli a posmívali se jeho zániku.\\
\hebinitial{ח} Těžce zhřešila jeruzalémská dcera, proto se stala nečistou. Všichni, kdo si jí vážili, ji mají za bezectnou, neboť vidí její nahotu; a ona vzdychá a odvrací se.\\
\hebinitial{ט} Má na lemu roucha nečistotu, nepamatovala na svůj konec. Předivně sešla a není nikdo, kdo by ji potěšil. ,,Pohleď, Hospodine, na mé ponížení, jak se nepřítel vypíná.``
Jeruzaléme, Jeruzaléme, obrať se k Pánu, Bohu svému.}}

\newcommand{\trMatLecIII}{\translatioCantus{
\hebinitial{י} Protivník vztáhl ruku na všechno, co měla za žádoucí. Ba, vidí pronárody vcházet do své svatyně, ač jsi o nich přikázal: ,,Nevejdou do tvého shro\-máž\-dě\-ní.``\\
\hebinitial{כ} Veškerý její lid vzdychá a žebrá o chléb. Za pokrm dávají všechno, co měli za žádoucí, jen aby se udrželi při životě. ,,Pohleď, Hospodine, popatř, jak jsem zbavena cti.``\\
\hebinitial{ל} ,,Je vám to lhostejné, vy všichni, kteří jdete kolem? Popatřte a hleďte, je-li jaká bolest jako bolest moje, ta, která mi byla způsobena, jíž mě zarmoutil Hospodin v den svého planoucího hněvu.\\
\hebinitial{מ} Z výšiny seslal do mých kostí oheň a pošlapal je. Před nohy mi rozprostřel síť, obrátil mě nazpět. Učinil mě místem zpustošeným, nečistým po všechny dny.\\
\hebinitial{נ} Neustále mě tlačí moje nevěrnosti, jež jeho ruka spletla ve jho, dostaly se mi na šíji; podlomil mou sílu, Panovník mě vydal do rukou těch, před nimiž neobstojím.``
Jeruzaléme, Jeruzaléme, obrať se k Pánu, Bohu svému.}}

\newcommand{\trMatLecIV}{\translatioCantus{
,,Nakloň, Bože, sluch k mé modlitbě, nestraň se mých úpěnlivých proseb, 
slyš mě pozorně a vyslyš mě.``
To jsou slova člověka stísněného, ustaraného, v soužení.
Trpící se mnoho modlí v touze být vysvobozen od zla.
Je ale třeba vidět, v jakém zlu se nachází.
A když to začne říkat, seznáváme, že jsme tam i my,
abychom s ním sdíleli jeho utrpení a spojili se s ním v modlitbě.
,,Zarmoucen jsem,`` praví, ,,a rozrušen ve svém výcviku``.
Kde je zarmoucen?
Kde rozrušen?
Říká, že ve svém výcviku.
Připomíná špatné lidi, kteří jej trápí a toto trápení od špatných lidí nazývá svým výcvikem.
Nemyslete si, že špatní jsou na světě zbytečně a že Bůh z nich neučiní nic dobrého.
Každý špatný žije buď proto, aby byl napraven, nebo proto, aby byl skrze něj cvičen dobrý.}}

\newcommand{\trMatLecV}{\translatioCantus{
Kéž by se tedy ti, kteří nás nyní cvičí, obrátili, a byli cvičeni spolu s námi!
Přesto však, dokud jsou v takovém stavu, že nás cvičí, nemějme k nim nenávist.
Nevíme totiž, zda někdo z nich setrvá ve své špatnosti až do konce.
Velmi často, když se domníváš nenávidět nepřítele, nenávidíš bratra a nevíš o tom.
O ďáblovi a jeho andělech víme z Písma, že jsou určeni k věčnému ohni.
Jen u nich není naděje na nápravu.
Proti nim vedeme skrytý zápas, ke kterému nás vyzbrojuje apoštol slovy:
,,Vedeme přece zápas ne proti nějaké obyčejné lidské moci,
ale proti knížatům a mocnostem, proti těm, kdo mají svou říši tmy v tomto světě.``
Ne abys to snad chápal tak, že by jejich vláda znamenala,
že démonům náleží řízení nebe a země.
Svět, kterému vládnou, je říší tmy, svět milovníků tohoto světa,
svět bezbožných a nespravedlivých.
Vládnou světu, o kterém říká evangelium:
,,ale svět ho nepoznal``.
}}

\newcommand{\trMatLecVI}{\translatioCantus{
,,Ve městě vidím nespravedlnost a sváry.``
Buď pozorný k slávě kříže samého.
Onen kříž, jejž dříve nepřátelé uráželi, je již upevněn v čele králů.
Účinek prokázal moc: podřídil si svět nikoli železem, nýbrž dřevem.
Dřevo kříže se nepřátelům zdálo být hodné urážek;
když stáli před tím dřevem, pokyvovali hlavou a říkali:
,,Jsi-li Syn Boží, sestup z kříže.``
Ježíš vztahoval ruce k lidu nevěřícímu a vzpurnému.
Je-li totiž spravedlivý ten, kdo žije z víry,
nespravedlivý je ten, kdo nemá víru.
Když je zde řeč o nespravedlnosti, rozuměj nevěru.
Pán tedy viděl ve městě nespravedlnost a sváry,
vztahoval své ruce k lidu nevěřícímu a odporujícímu.
A přesto říkal:
,,Otče, odpusť jim, neboť nevědí, co činí.``}}

\newcommand{\trMatLecVII}{\translatioCantus{
Když už vás napomínám, nemohu také pochválit, že se shromažďujete spíše ke škodě než k prospěchu.
Předně slyším, že jsou mezi vámi roztržky, když se v církvi shromažďujete, a jsem nakloněn tomu věřit.
Neboť musí mezi vámi být i různé skupiny, aby se ukázalo, kdo z vás se osvědčí.
Když vy se však shromažďujete, není to už společenství večeře Páně:
každý se dá hned do své večeře, a jeden má hlad, druhý se opije.
Což nemáte své domácnosti, kde byste jedli a pili? Či snad pohrdáte církví Boží a chcete zahanbit ty, kteří nic nemají? Co vám mám říci? Mám vás snad pochválit? Za to vás nechválím! }}

\newcommand{\trMatLecVIII}{\translatioCantus{
Já jsem přijal od Pána, co jsem vám také odevzdal: Pán Ježíš v tu noc, kdy byl zrazen, vzal chléb,
vzdal díky, lámal jej a řekl: ,,\textsc{Toto jest mé tělo, které se za vás vydává; to čiňte na mou památku}.``
Stejně vzal po večeři i kalich a řekl: ,,\textsc{Tento kalich je nová smlouva, zpečetěná mou krví; to čiňte, kdykoli budete píti, na mou památku}.``
Kdykoli tedy jíte tento chléb a pijete tento kalich, zvěstujete smrt Páně, dokud on nepřijde.}}

\newcommand{\trMatLecIX}{\translatioCantus{
Kdo by tedy jedl tento chléb a pil kalich Páně nehodně, proviní se proti tělu a krvi Páně.
Nechť každý sám sebe zkoumá, než tento chléb jí a z tohoto kalicha pije.
Kdo jí a pije a nerozpoznává, že jde o tělo Páně, jí a pije sám sobě odsouzení.
Proto je mezi vámi tolik slabých a nemocných a mnozí umírají.
Kdybychom soudili sami sebe, nebyli bychom souzeni.
Když nás však soudí Pán, je to k naší nápravě, abychom nebyli odsouzeni spolu se světem.
A tak, bratří moji, když se shromažďujete k společnému stolu, čekejte jeden na druhého.
Kdo má hlad, ať se nají doma, abyste se neshromažďovali k odsouzení. Ostatní věci zařídím, až přijdu.}}
