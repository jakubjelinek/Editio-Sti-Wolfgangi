\newcommand{\titulus}{\nomenFesti{Feria IV infra octavam Corporis Christi.}
\celebratio{Semiduplex.}}
\newcommand{\aequus}{Feria IV}
\newcommand{\lectioi}{\pars{Lectio I.} \scriptura{1 Reg. 6, 19-21; 7:1}

\noindent De libro primo Regum.

\noindent Percússit autem de viris Bethsamítibus, eo quod vidíssent arcam Dómini; et percússit de pópulo septuagínta viros et quinquagínta míllia plebis. Luxítque pópulus, eo quod Dóminus percussísset plebem plaga magna. Et dixérunt viri Bethsamítæ: Quis póterit stare in conspéctu Dómini Dei sancti huius? et ad quem ascéndet a nobis? Miserúntque núntios ad habitatóres Cariathíarim, dicéntes: Reduxérunt Philísthiim arcam Dómini: descéndite, et redúcite eam ad vos. Venérunt ergo viri Cariathíarim, et reduxérunt arcam Dómini, et intulérunt eam in domum Abínadab in Gábaa: Eleázarum autem fílium eius sanctificavérunt, ut custodíret arcam Dómini.}
\newcommand{\lectioii}{\pars{Lectio II.} \scriptura{1 Reg. 7, 2-4}

\noindent Et factum est, ex qua die mansit arca Dómini in Cariathíarim, multiplicáti sunt dies (erat quippe iam annus vigésimus), et requiévit omnis domus Israël post Dóminum. Ait autem Sámuel ad univérsam domum Israël dicens: Si in toto corde vestro revertímini ad Dóminum, auférte deos aliénos de médio vestri Báalim et Astaroth et præparáte corda vestra Dómino et servíte ei soli, et éruet vos de manu Philísthiim. Abstulérunt ergo fílii Israël Báalim et Astaroth, et serviérunt Dómino soli.}
\newcommand{\lectioiii}{\pars{Lectio III.} \scriptura{1 Reg. 7, 5-8}

\noindent Dixit autem Sámuel: Congregáte univérsum Israël in Masphath, ut orem pro vobis Dóminum. Et convenérunt in Masphath hauserúntque aquam et effudérunt in conspéctu Dómini et ieiunavérunt in die illa atque dixérunt ibi: Peccávimus Dómino. Iudicavítque Sámuel fílios Israël in Masphath. Et audiérunt Philísthiim quod congregáti essent fílii Israël in Masphath, et ascendérunt sátrapæ Philisthinórum ad Israël. Quod cum audíssent fílii Israël, timuérunt a fácie Philisthinórum. Dixerúntque ad Samuélem: Ne cesses pro nobis clamáre ad Dóminum Deum nostrum, ut salvet nos de manu Philisthinórum.}
\newcommand{\lectioiv}{\pars{Lectio IV.} \scriptura{Lib. 4, Cap. 4}

\noindent Ex libro sancti Ambrósii Epíscopi de Sacraméntis.

\noindent Auctor sacramentórum quis est, nisi Dóminus Iesus? De cælo ista sacraménta venérunt. Consílium enim omne de cælo est. Vere autem magnum est et divínum miráculum, quod pópulo pluit Deus manna de cælo, et non laborábat pópulus, et manducábat. Tu forte dicis: Meus panis est usitátus. Sed panis iste, panis est ante verba sacramentórum: ubi accésserit consecrátio, de pane fit caro Christi. Hoc ígitur astruámus. Quómodo potest, qui panis est, corpus esse Christi? Consecratióne. Consecrátio ígitur quibus verbis est, et cuius sermónibus? Dómini Iesu. Nam réliqua ómnia quæ dicúntur, laudem Deo déferunt, orátio præmíttitur pro pópulo, pro régibus, pro céteris: ubi venítur ut conficiátur venerábile Sacraméntum, iam non suis sermónibus sacérdos, sed útitur sermónibus Christi.}
\newcommand{\lectiov}{\pars{Lectio V.}

\noindent Ergo sermo Christi hoc cónficit Sacraméntum. Quis sermo Christi? Nempe is, quo facta sunt ómnia. Iussit Dóminus, et factum est cælum: iussit Dóminus, et facta est terra: iussit Dóminus, et facta sunt mária: iussit Dóminus, et omnis creatúra generáta est. Vides ergo quam operatórius sit sermo Christi. Si ergo tanta vis est in sermóne Dómini Iesu, ut incíperent esse quæ non erant; quanto magis operatórius est, ut quæ erant, in áliud commuténtur? Cælum non erat, terra non erat. Sed audi dicéntem: Ipse dixit, et facta sunt: ipse mandávit, et creáta sunt. Ergo tibi ut respóndeam, non erat corpus Christi ante consecratiónem; sed post consecratiónem dico tibi quod iam corpus est Christi. Ipse dixit, et factum est: ipse mandávit, et creátum est.}
\newcommand{\lectiovi}{\pars{Lectio VI.}

\noindent Iam redi mecum ad propositiónem meam. Magnum quidem et venerábile, quod manna Iudǽis pluit e cælo. Sed intéllige quid est ámplius, manna de cælo, an corpus Christi? Corpus Christi útique, qui auctor est cæli. Deínde, manna qui manducávit, mórtuus est: qui manducáverit hoc corpus, fiet ei remíssio peccatórum, et non moriétur in ætérnum. Ergo non otióse, cum áccipis, tu dicis, Amen; iam in spíritu cónfitens quod accípias corpus Christi. Dicit tibi sacérdos, Corpus Christi; et tu dicis, Amen, hoc est, Verum. Quod confitétur lingua, téneat afféctus.}
\newcommand{\lectiovii}{\pars{Lectio VII.} \scriptura{Io. 6, 56-59}

\noindent Léctio sancti Evangélii secúndum Ioánnem.

\noindent In illo témpore: Dixit Iesus turbis Iudæórum: Caro mea vere est cibus, et sanguis meus vere est potus. Et réliqua.

\scriptura{Liber 8 de Trinitate, ante medium}

\noindent Homilía sancti Hilárii Epíscopi.

\noindent Non est humáno aut sǽculi sensu in Dei rebus loquéndum. Quæ scripta sunt, legámus, et quæ legérimus, intelligámus; et tunc perféctæ fídei offício fungémur. De naturáli enim in nobis Christi veritáte quæ dícimus, nisi ab eo díscimus, stulte atque ímpie dícimus. Ipse enim ait: Caro mea vere est esca, et sanguis meus vere est potus. Qui edit carnem meam et bibit sánguinem meum, in me manet, et ego in eo. De veritáte carnis et sánguinis non relíctus est ambigéndi locus.}
\newcommand{\lectioviii}{\pars{Lectio VIII.}

\noindent Nunc enim, et ipsíus Dómini professióne, et fide nostra, vere caro est, et vere sanguis est. Et hæc accépta atque hausta id effíciunt, ut et nos in Christo, et Christus in nobis sit. An ne hoc véritas non est? Contíngat plane his verum non esse, qui Christum Iesum verum esse Deum negant. Est ergo in nobis ipse per carnem, et sumus in eo, dum secum hoc quod nos sumus, in Deo est. Quod autem in eo per sacraméntum communicátæ carnis et sánguinis simus, ipse testátur, dicens: Et hic mundus iam me non videt, vos autem me vidébitis: quóniam ego vivo, et vos vivétis: quóniam ego in Patre meo, et vos in me, et ego in vobis.}
\newcommand{\lectioix}{\pars{Lectio IX.}

\noindent Quod autem in nobis naturális hæc únitas sit, ipse ita testátus est: Qui edit carnem meam et bibit sánguinem meum, in me manet, et ego in eo. Non enim quis in eo erit, nisi in quo ipse fúerit; eius tantum in se assúmptam habens carnem, qui suam súmpserit. Perféctæ autem huius unitátis sacraméntum supérius iam docúerat, dicens: Sicut me misit vivens Pater, et ego vivo per Patrem; et qui mandúcat meam carnem, et ipse vivet per me. Vivit ergo per Patrem: et quo modo per Patrem vivit, eódem modo nos per carnem eius vivémus.}
% LuaLaTeX

\documentclass[a4paper, twoside, 12pt]{article}
\usepackage[latin]{babel}
%\usepackage[landscape, left=3cm, right=1.5cm, top=2cm, bottom=1cm]{geometry} % okraje stranky
%\usepackage[landscape, a4paper, mag=1166, truedimen, left=2cm, right=1.5cm, top=1.6cm, bottom=0.95cm]{geometry} % okraje stranky
\usepackage[landscape, a4paper, mag=1400, truedimen, left=0.5cm, right=0.5cm, top=0.5cm, bottom=0.5cm]{geometry} % okraje stranky

\usepackage{fontspec}
\setmainfont[FeatureFile={junicode.fea}, Ligatures={Common, TeX}, RawFeature=+fixi]{Junicode}
%\setmainfont{Junicode}

% shortcut for Junicode without ligatures (for the Czech texts)
\newfontfamily\nlfont[FeatureFile={junicode.fea}, Ligatures={Common, TeX}, RawFeature=+fixi]{Junicode}

\usepackage{multicol}
\usepackage{color}
\usepackage{lettrine}
\usepackage{fancyhdr}

% usual packages loading:
\usepackage{luatextra}
\usepackage{graphicx} % support the \includegraphics command and options
\usepackage{gregoriotex} % for gregorio score inclusion
\usepackage{gregoriosyms}
\usepackage{wrapfig} % figures wrapped by the text
\usepackage{parcolumns}
\usepackage[contents={},opacity=1,scale=1,color=black]{background}
\usepackage{tikzpagenodes}
\usepackage{calc}
\usepackage{longtable}
\usetikzlibrary{calc}

\setlength{\headheight}{14.5pt}

\input{conventuscommune.tex} % Often used macros
%%%% Preklady jednotlivych zpevu (nektere se opakuji, a je dobre mit je
% vsechny na jedne hromade)

% HOURS ---

\newcommand{\trAntI}{\translatioCantus{Muž boží měl kožený toulec, pečlivě
zavázaný, jenž mu visel na šíji a~často se ho dotýkal.}}

\newcommand{\trAntII}{\translatioCantus{Klíč od~něho tak dobře střežil, že
dokud žil v~těle, nikdo z~jeho žáků nezvěděl, co je uvnitř.}}

\newcommand{\trAntIII}{\translatioCantus{Ale když se odebral z~tohoto
života, schránku otevřeli a~objevili v~ní žíněné roucho a~měděný řetěz
potřísněný krví.}}

\newcommand{\trAntIV}{\translatioCantus{A když prohlédli mistrovo tělo,
nalezli jeho tělo na čtyřech místech hluboce zbrázděno ranami od řetězu.}}

\newcommand{\trAntV}{\translatioCantus{Krev vytékající z~těch ran, místy
prostoupila i~žíněným rouchem.}}

\newcommand{\trCapituli}{\translatioCantus{
Miláčkovi Boha a~lidí,
Mojžíšovi požehnané paměti,~\gredagger{}
dopřál slávu rovnou slávě svatých~\grestar{}
učinil ho mocným na postrach nepřátelům
a~jeho slovy zastavil divy.}}

\newcommand{\trLectioBrevis}{\translatioCantus{
Pamatujte na své představené,
kteří vám hlásali Boží slovo.
Uvažte, jak oni skončili život, a~napodobujte jejich víru.
Ježíš Kristus je stejný včera i~dnes i~navěky.
Nenechte se svést věelijakými cizími naukami.}}

\newcommand{\trRespLaud}{\translatioCantus{Spravedlivého vodil Hospodin~\grestar{}
po přímých stezkách. \Vbardot{} A~ukázal mu Boží království.}}

\newcommand{\trRespLaudB}{\translatioCantus{Na tvých hradbách, Jeruzaléme,
ustanovil jsem strážné;~\grestar{}
budou bdít nad mým lidem. \Vbardot{} Ani ve dne, ani v~noci nesmějí nikdy
mlčet.}}

\newcommand{\trVersus}{\translatioCantus{\Vbardot{} Ústa spravedlivého šeptají moudrost, aleluja.
\Rbardot{} A~jeho jazyk ohlašuje právo, aleluja.}}

\newcommand{\trAntBenedictus}{\translatioCantus{Když na bujné oře vložili
nosítka a~sňali jim uzdu, vydali se přímo k~cele božího muže.}}

\newcommand{\trPreces}{\translatioCantus{
\noindent S vděčností chvalme Krista, dobrého Pastýře, \gredagger{} který dal život za své ovce, \grestar{} a~pokorně ho prosme: \Rbardot{} Pane, buď pastýřem svého lidu.

\noindent Kriste, ty dáváš církvi pastýře, a~jejich službou se ujímáš svého lidu, \grestar{} dej, ať v~lásce těch, kteří nás vedou, poznáváme, jak nás miluješ. \Rbardot{} Pane, buď pastýřem svého lidu.

\noindent Ty stále konáš skrze své zástupce službu pastýře a~učitele, \grestar{} nepřestávej nás nikdy vést prostřednictvím svých služebníků. \Rbardot{} Pane, buď pastýřem svého lidu.

\noindent Ty prokazuješ svému lidu skrze jeho pastýře službu lékaře duše i~těla, \grestar{} ochraňuj náš život a~veď nás ke svatosti. \Rbardot{} Pane, buď pastýřem svého lidu.

\noindent Ty posíláš své svaté, aby slovem i~příkladem vedli tvůj lid k~tobě, \grestar{} na jejich přímluvu nás posiluj, abychom vytrvali na cestě, která vede k~věčnému životu. \Rbardot{} Pane, buď pastýřem svého lidu.}}

\newcommand{\trOrationis}{\translatioCantus{Bože, jenž nám dopřáváš radovat
se z~výroční slavnosti svatého tvého vyznavače Havla, uděl dobrotivě,
abychom když slavíme jeho narození, též se řídili podobou jeho skutků.
Skrze…}}
 % Czech translations of the proper texts

\newcommand{\annusEditionis}{2020}

%%%% Vicekrat opakovane kousky

\newcommand{\anteOrationem}{
  \rubrica{Ante Orationem, cantatur a Superiore:}

  \pars{Supplicatio Litaniæ.}

  \cuminitiali{}{temporalia/supplicatiolitaniae.gtex}

  \pars{Oratio Dominica.}

  \cuminitiali{}{temporalia/oratiodominica.gtex}

  \rubrica{Deinde dicitur ab Hebdomadario:}

  \cuminitiali{}{temporalia/dominusvobiscum-solemnis.gtex}

  \rubrica{In choro monialium loco Dominus vobiscum dicitur:}

  \sineinitiali{temporalia/domineexaudi.gtex}
}

\ifx\dominica\undefined
\newcommand{\capitulumLaudes}{\pars{Capitulum.} \scriptura{1 Cor. 11, 23-24}

\grechangedim{interwordspacetext}{0.12 cm plus 0.15 cm minus 0.05 cm}{scalable}%
\cuminitiali{}{temporalia/capitulum-FratresEgo.gtex}
\grechangedim{interwordspacetext}{0.22 cm plus 0.15 cm minus 0.05 cm}{scalable}}
\else
\newcommand{\capitulumLaudes}{\pars{Capitulum.} \scriptura{1 Ptr. 4, 7-8}

\grechangedim{interwordspacetext}{0.12 cm plus 0.15 cm minus 0.05 cm}{scalable}%
\cuminitiali{}{temporalia/capitulum-CarissimiEstote.gtex}
\grechangedim{interwordspacetext}{0.22 cm plus 0.15 cm minus 0.05 cm}{scalable}}
\fi

\setlength{\columnsep}{30pt} % prostor mezi sloupci

%%%%%%%%%%%%%%%%%%%%%%%%%%%%%%%%%%%%%%%%%%%%%%%%%%%%%%%%%%%%%%%%%%%%%%%%%%%%%%%%%%%%%%%%%%%%%%%%%%%%%%%%%%%%%
\begin{document}

% Here we set the space around the initial.
% Please report to http://home.gna.org/gregorio/gregoriotex/details for more details and options
\grechangedim{afterinitialshift}{2.2mm}{scalable}
\grechangedim{beforeinitialshift}{2.2mm}{scalable}
\grechangedim{interwordspacetext}{0.22 cm plus 0.15 cm minus 0.05 cm}{scalable}%
\grechangedim{annotationraise}{-0.2cm}{scalable}

% Here we set the initial font. Change 38 if you want a bigger initial.
% Emit the initials in red.
\grechangestyle{initial}{\color{red}\fontsize{38}{38}\selectfont}

\pagestyle{empty}

\newcommand{\vesperasi}{
\pars{Psalmus 2.} \scriptura{Ps. 110, 4-5}

\vspace{-0.4cm}

\antiphona{II D}{temporalia/ant-miseratordominus.gtex}

\scriptura{Psalmus 110.}

\initiumpsalmi{temporalia/ps110-initium-ii-D-auto.gtex}

%\psalmusEtTranslatioT{temporalia/ps110-comb.tex}{10cm}
\input{temporalia/ps110.tex} \Abardot{}

\vfill
\pagebreak

\pars{Psalmus 3.} \scriptura{Ps. 115, 4.8}

\vspace{-0.4cm}

\antiphona{III a}{temporalia/ant-calicemsalutaris.gtex}

\scriptura{Psalmus 115.}

\initiumpsalmi{temporalia/ps115-initium-iii-a-auto.gtex}

%\psalmusEtTranslatioT{temporalia/ps115-comb.tex}{10cm}
\input{temporalia/ps115.tex} \Abardot{}

\vfill
\pagebreak
}

\newcommand{\vesperasii}{
\pars{Psalmus 4.} \scriptura{Ps. 127, 3}

\vspace{-0.4cm}

\antiphona{IV E}{temporalia/ant-sicutnovellae.gtex}

\scriptura{Psalmus 127.}

\initiumpsalmi{temporalia/ps127-initium-iv-E-auto.gtex}

%\psalmusEtTranslatioT{temporalia/ps127-comb.tex}{10cm}
\input{temporalia/ps127.tex} \Abardot{}

%\vfill

%\vspace{-6mm}

%\antiphona{}{temporalia/ant-sicutnovellae.gtex} % repeat the antiphon - new page

\vfill
\pagebreak
}

\newcommand{\vesperasiii}{
\pars{Psalmus 4.} \scriptura{Cf. Ps. 147, 3}

\vspace{-0.4cm}

\antiphona{V g}{temporalia/ant-quipacem.gtex}

\scriptura{Psalmus 147.}

\initiumpsalmi{temporalia/ps147-initium-v-g-auto.gtex}

%\psalmusEtTranslatioT{temporalia/ps147-comb.tex}{10cm}
\input{temporalia/ps147.tex} \Abardot{}

%\vfill

%\vspace{-6mm}

%\antiphona{}{temporalia/ant-sicutnovellae.gtex} % repeat the antiphon - new page

\vfill
\pagebreak
}

\newcommand{\vesperasiv}{
\pars{Hymnus} \scriptura{Thomas de Aquino?}

\cuminitiali{IV}{temporalia/hym-PangeLingua.gtex}
\vspace{-3mm}
%\input{hym-PangeLingua-bohtext.tex}

\vfill
\pagebreak

\pars{Versus.} \scriptura{Sap. 16, 20}

% Versus. %%%
\ifx\festum\undefined
\sineinitiali{temporalia/versus-panemdecaelo-communis.gtex}
\else
\sineinitiali{temporalia/versus-panemdecaelo.gtex}
\fi

%\noindent \trVersus

\vfill
\pagebreak
}

%%%% Titulni stranka
\begin{titulusOfficii}
\titulus
\end{titulusOfficii}

% graphic
%\vspace{1.5cm}
%\begin{center}
%\includegraphics[width=8cm]{emmaus.jpg}
%\end{center}

\vfill

\begin{center}
%Ad usum et secundum consuetudines chori \guillemotright{}Conventus Choralis\guillemotleft.

%Editio Sancti Wolfgangi \annusEditionis
\end{center}

\pagebreak

\renewcommand{\headrulewidth}{0pt} % no horiz. rule at the header
\fancyhf{}
\pagestyle{fancy}

\cantusSineNeumas

\ifx\festumveldominica\undefined
\else
\pars{Oratio ante divinum Officium.}

\lettrine{{\color{red}A}}{peri,} Dómine, os meum ad benedicéndum nomen sanctum tuum:
munda quoque cor meum ab ómnibus vanis, pervérsis, et aliénis
cogitatiónibus:
intelléctum illúmina, afféctum inflámma,
ut digne, atténte ac devóte hoc Offícium recitáre váleam,
et exaudíri mérear ante conspéctum Divínæ Maiestátis tuæ.
Per Christum, Dóminum nostrum.
\Rbardot{} Amen.

Dómine, in unióne illíus divínæ intentiónis,
qua ipse in terris laudes Deo persolvísti,
has tibi Horas \rubricatum{(vel \textnormal{hanc tibi Horam})} persólvo.

%\trOratioAnteOfficium

\vfill

\pars{Oratio post divinum Officium.}

\rubrica{
  Orationem sequentem devote post Officium recitantibus
  Leo Papa X. defectus, et culpas in eo persolvendo ex humana
  fragilitate contractas, indulsit, et dicitur flexis genibus.
}

\lettrine{{\color{red}S}}{acrosánctæ} et indivíduæ Trinitáti,
crucifíxi Dómini nostri Iesu Christi humanitáti,
beatíssimæ et gloriosíssimæ sempérque Vírginis Maríæ
fecúndæ integritáti, 
et ómnium Sanctórum universitáti
sit sempitérna laus, honor, virtus et glória
ab omni creatúra,
nobísque remíssio ómnium peccatórum,
per infiníta sǽcula sæculórum.
\Rbardot{} Amen.

\noindent \Vbardot{} Beáta víscera Maríæ Virginis, quæ portavérunt
ætérni Patris Fílium.\\
\Rbardot{} Et beáta úbera, quæ lactavérunt Christum Dominum.

\rubrica{Et dicitur secreto \textnormal{Pater noster.} et \textnormal{Ave María.}}

%\trOratioPostOfficium

\vfill

\hora{Ad I. Vesperas.} %%%%%%%%%%%%%%%%%%%%%%%%%%%%%%%%%%%%%%%%%%%%%%%%%%%%%
%\sideThumbs{I. Vesperæ}

\vspace{0.5cm}
\grechangedim{interwordspacetext}{0.18 cm plus 0.15 cm minus 0.05 cm}{scalable}%
\cuminitiali{}{temporalia/deusinadiutorium-solemnis.gtex}
\grechangedim{interwordspacetext}{0.22 cm plus 0.15 cm minus 0.05 cm}{scalable}%

\vfill
\pagebreak

\pars{Psalmus 1.} \scriptura{Ps. 109, 4; Gn 14, 18}

\vspace{-0.4cm}

\antiphona{I f}{temporalia/ant-sacerdosinaeternum.gtex}

\scriptura{Psalmus 109.}

\initiumpsalmi{temporalia/ps109-initium-i-f-auto.gtex}

%\psalmusEtTranslatioT{temporalia/ps109-comb.tex}{10cm}
\input{temporalia/ps109.tex} \Abardot{}

\vspace{-1cm}

\vfill
\pagebreak

\vesperasi

\ifx\festum\undefined
\vesperasiii
\else
\vesperasii
\fi

\capitulumLaudes

% preklad Jeruz. bible
%\trCapituliI

\vfill

\pars{Responsorium breve.} \scriptura{Ps. 30, 20; \textbf{H085}}

\cuminitiali{I}{temporalia/resp-quammagnamultitudo-cumdox.gtex}

%\trResp

\vfill
\pagebreak

\vesperasiv

\ifx\dominica\undefined
\pars{Canticum B. Mariæ V.} \scriptura{Cf. Sap. 16, 21,20}

\vspace{-3mm}

{
\grechangedim{interwordspacetext}{0.18 cm plus 0.15 cm minus 0.05 cm}{scalable}%
\antiphona{VI F}{temporalia/ant-oquamsuavisest.gtex}
\grechangedim{interwordspacetext}{0.22 cm plus 0.15 cm minus 0.05 cm}{scalable}%
}

%\trAntIMagnificat

\vspace{-2mm}

\scriptura{Lc. 1, 46-55}

\vspace{-2mm}

\cantusSineNeumas
\initiumpsalmi{temporalia/magnificat-initium-visoll-F.gtex}

%\psalmusEtTranslatioT{temporalia/magnificat-I-comb.tex}{10.2cm}
\input{temporalia/magnificat-I.tex} \Abardot{}
\else
\pars{Canticum B. Mariæ V.} \scriptura{Io. 15, 26; \textbf{H267}}

\vspace{-6mm}

{
\grechangedim{interwordspacetext}{0.18 cm plus 0.15 cm minus 0.05 cm}{scalable}%
\antiphona{VIII G}{temporalia/ant-XXXcumveneritparaclitus.gtex}
\grechangedim{interwordspacetext}{0.22 cm plus 0.15 cm minus 0.05 cm}{scalable}%
}

%\trAntIMagnificat

\vspace{-3mm}

\scriptura{Lc. 1, 46-55}

\vspace{-2mm}

\cantusSineNeumas
\initiumpsalmi{temporalia/magnificat-initium-viiisoll-G.gtex}

\vspace{-1.5mm}

%\psalmusEtTranslatioT{temporalia/magnificat-III-comb.tex}{10.2cm}
\input{temporalia/magnificat-III.tex} \Abardot{}
\fi

\vspace{-1cm}

\vfill
\pagebreak

%\sideThumbs{{\scriptsize{}Fine horarum}}

\anteOrationem

\pagebreak

% Oratio. %%%
\cuminitiali{}{temporalia/oratio.gtex}

\vspace{-1mm}
%\trOrationisI

\vfill

\rubrica{Hebdomadarius dicit iterum Dominus vobiscum, vel cantor dicit:}

\vspace{2mm}

\sineinitiali{temporalia/domineexaudi.gtex}

\rubrica{Postea cantatur a cantore:}

\vspace{2mm}

\ifx\festum\undefined
\cuminitiali{VII}{temporalia/benedicamus-tempore-paschali.gtex}
\else
\cuminitiali{II}{temporalia/benedicamus-solemnism-1vesp.gtex}
\fi

\vspace{1mm}

\vfill
\pagebreak
\fi

\ifx\festum\undefined
\else
\hora{Ad Completorium.} %%%%%%%%%%%%%%%%%%%%%%%%%%%%%%%%%%%%%%%%%%%%%%%%%%%%%%%%%%
%\sideThumbs{{\scriptsize{}Completorium}}

\rubrica{Lector petit benedictionem, dicens:}

\cuminitiali{}{temporalia/jubedomnebenedicere.gtex}

%\trJubeDomne

\vfill

\pars{Benedictio.}

\cuminitiali{}{temporalia/benedictio-noctemquietam.gtex}

%\trComplBenedictio

\vfill

\pars{Lectio brevis.} \scriptura{1Ptr. 5, 8-9}

\cuminitiali{}{temporalia/lectiobrevis-fratressobrii.gtex}

%\trComplLectioBr

\vfill

\noindent \Vbardot{} Adiutórium nostrum in nómine Dómini.

\noindent \Rbardot{} Qui fecit cælum, et terram.

\vfill
\pagebreak

\pars{Confessio.}

\noindent Confíteor Deo omnipoténti, beátæ Maríæ semper Vírgini, beáto
Michaéli Archángelo, beáto Ioánni Baptístæ, sanctis Apóstolis Petro
et Paulo, ómnibus Sanctis, et vobis fratres: quia peccávi nimis cogitatióne,
verbo et ópere: mea culpa, mea culpa, mea máxima culpa.
Ideo precor beátam Maríam semper Vírginem, beátum Michaélem
Archángelum, beátum Ioánnem Baptístam, sanctos Apóstolos Petrum
et Paulum, omnes Sanctos, et vos fratres, oráre pro me ad Dóminum
Deum nostrum.

\vfill

\noindent \Vbardot{} Misereátur nostri omnípotens Deus, et, dimíssis peccátis nostris, perdúcat
nos ad vitam ætérnam. \Rbardot{} Amen.

\vfill

\noindent \Vbardot{} Indulgéntiam, absolutiónem et remissiónem peccatórum nostrórum tríbuat nobis
omnípotens et miséricors Dóminus. \Rbardot{} Amen.

\vfill

\rubrica{Et facta absolutione dicitur:}

\sineinitiali{temporalia/convertenosdeus.gtex}

\vfill

\cuminitiali{}{temporalia/deusinadiutorium-communis.gtex}

\vfill
\pagebreak

\pars{Psalmus 1.}

\antiphona{VIII G}{temporalia/ant-miserere.gtex}

\scriptura{Ps. 4}

\initiumpsalmi{temporalia/ps4-initium-viii-G-auto.gtex}

%\psalmusEtTranslatioT{temporalia/ps4-comb.tex}{10cm}
\input{temporalia/ps4.tex}

\vfill
\pagebreak

\pars{Psalmus 2.} \scriptura{Ps. 90}

\initiumpsalmi{temporalia/ps90-initium-viii-G-auto.gtex}

%\psalmusEtTranslatioT{temporalia/ps90-comb.tex}{10cm}
\input{temporalia/ps90.tex}

\pagebreak

\pars{Psalmus 3.} \scriptura{Ps. 133}

\initiumpsalmi{temporalia/ps133-initium-viii-G-auto.gtex}

%\psalmusEtTranslatioT{temporalia/ps133-comb.tex}{10cm}
\input{temporalia/ps133.tex}

\vfill

\antiphona{}{temporalia/ant-miserere.gtex}

\vfill

\pars{Hymnus.}

\antiphona{II}{temporalia/hym-TeLucis.gtex}
%\input{hym-TeLucis-bohtext.tex}

\pagebreak

\pars{Capitulum.} \scriptura{Ier. 14, 9}

\cuminitiali{}{temporalia/capitulum-tuautem.gtex}

% preklad Jeruz. bible
%\trComplCapituli

\vfill

\pars{Responsorium breve.} \scriptura{Ps. 30, 6}

\cuminitiali{VI}{temporalia/resp-inmanus.gtex}

%\trRespCompl
\vfill

\pars{Versus.} \scriptura{Ps. 16, 8}

\sineinitiali{temporalia/versus-custodi.gtex}

%\noindent \trComplVersus

\vfill
\pagebreak

\cantusCumNeumis

\pars{Canticum Simeonis.}

\vspace{-3mm}

\antiphona{III a}{temporalia/ant-salvanos-antiquo.gtex}

%\trAntSalvaNos

%\vspace{-1mm}

\scriptura{Lc. 2, 29-32}

\vspace{-2mm}

\initiumpsalmi{temporalia/nuncdimittis-initium-iii-a-auto.gtex}

%\psalmusEtTranslatioT{temporalia/nuncdimittis-comb.tex}{10cm}
\input{temporalia/nuncdimittis.tex} \Abardot{}

\vfill

\rubrica{Ante Orationem, cantatur a Superiore:}

\vspace{3mm}

\pars{Supplicatio Litaniæ.}

\cuminitiali{}{temporalia/supplicatiolitaniae.gtex}

\vspace{7mm}

\pars{Oratio Dominica.}

\noindent Pater noster.

\vfill
\pagebreak

\sineinitiali{temporalia/domineexaudi-simplex.gtex}

\vspace{7mm}

\pars{Oratio.}

\cantusSineNeumas

\cuminitiali{}{temporalia/oratio-visita.gtex}

%\trComplOrationis

\vfill

%\sineinitiali{temporalia/domineexaudi-communis.gtex}

\noindent \Vbardot{} Dómine, exáudi oratiónem meam. \Rbardot{} Et clamor meus ad te véniat.

\vfill

%\vfill

\sineinitiali{temporalia/benedicamus-minor.gtex}

\vfill

\pars{Benedictio.}

\noindent Benedícat et custódiat nos omnípotens et miséricors Dóminus, \gredagger{}
Pater, et Fílius, et Spíritus Sanctus. \Rbardot{} Amen.

\vfill
\pagebreak

\pars{Antiphona finalis B. M. V.}

\vspace{-4mm}

\antiphona{I}{temporalia/an_salve_regina.gtex}

\rubrica{vel:}

\vspace{-4mm}

\antiphona{V}{temporalia/ant-salveregina-simplex.gtex}

\vfill
\pagebreak

\rubrica{vel:}

\antiphona{VII}{temporalia/ant-subtuum.gtex}

\vfill
\pagebreak
\fi

\hora{Ad Matutinum.} %%%%%%%%%%%%%%%%%%%%%%%%%%%%%%%%%%%%%%%%%%%%%%%%%%%%%
%\sideThumbs{Matutinum}

\vspace{2mm}

\cuminitiali{}{temporalia/dominelabiamea.gtex}

\vspace{2mm}

\pars{Invitatorium.} \scriptura{Cantor; Psalmus 94; \textbf{H261}}

\vspace{-6mm}

\antiphona{V}{temporalia/inv-alleluiachristumdominum.gtex}

\vfill
\pagebreak

\pars{Hymnus.}

\vspace{-5mm}

\scriptura{Anonymus X. sæculi; \textbf{AR488}}

\antiphona{IV}{temporalia/hym-AEterneRexAltissime.gtex}
%{
%\vspace{-5mm}
%\setlength{\columnsep}{0pt} % prostor mezi sloupci
%\input{hym-RexSempiterne-bohtext.tex}
%\setlength{\columnsep}{30pt} % prostor mezi sloupci
%}

\vfill
\pagebreak

\subhora{In I. Nocturno}

\pars{Psalmus 1.} \scriptura{Ps. 8, 2; \textbf{H262}}

%\vspace{-5mm}

\antiphona{IV A*}{temporalia/ant-elevataestmagnificentiatua.gtex}

%\vspace{-5mm}

\scriptura{Ps. 8}

%\vspace{-2mm}

\initiumpsalmi{temporalia/ps8-initium-iv-A_-auto.gtex}

%\psalmusEtTranslatioT{temporalia/ps8-comb.tex}{10cm}
\input{temporalia/ps8.tex} \Abardot{}

\vfill
\pagebreak

\pars{Psalmus 2.} \scriptura{Ps. 10, 5; \textbf{H262}}

%\vspace{-5mm}

\antiphona{VIII c}{temporalia/ant-dominusintemplosanctosuo.gtex}

%\vspace{-5mm}

\scriptura{Ps. 10}

\initiumpsalmi{temporalia/ps10-initium-viii-C-auto.gtex}

%\psalmusEtTranslatioT{temporalia/ps10-comb.tex}{10cm}
\input{temporalia/ps10.tex} \Abardot{}

\vfill
\pagebreak

\pars{Psalmus 3.} \scriptura{Ps. 18, 7; \textbf{H262}}

%\vspace{-5mm}

\antiphona{IV A*}{temporalia/ant-asummocoeloegressioejus.gtex}

%\vspace{-5mm}

\scriptura{Ps. 18}

\initiumpsalmi{temporalia/ps18-initium-iv-A_-auto.gtex}

%\psalmusEtTranslatioT{temporalia/ps18-comb.tex}{10cm}
\input{temporalia/ps18.tex}

\vfill

\antiphona{}{temporalia/ant-asummocoeloegressioejus.gtex}

\vfill
\pagebreak

\noindent \Vbardot{} Ascéndit Deus in iubilatióne, allelúia.
\noindent \Rbardot{} Et Dóminus in voce tubæ, allelúia.

\vspace{5mm}

\sineinitiali{temporalia/oratiodominica-mat.gtex}

\vspace{5mm}

\pars{Absolutio.}

\cuminitiali{}{temporalia/absolutio-exaudi.gtex}

\vfill
\pagebreak

\cuminitiali{}{temporalia/benedictio-solemn-benedictione.gtex}

\vspace{7mm}

\lectioi

\noindent \Vbardot{} Tu autem, Dómine, miserére nobis.
\noindent \Rbardot{} Deo grátias.

\vfill
\pagebreak

\pars{Responsorium 1.} \scriptura{\Rbardot{} Ac. 1, 3.9; \Vbardot{} ibid. 1, 4; \textbf{H262}}

\vspace{-5mm}

\responsorium{III transp.}{temporalia/resp-postpassionemsuam-sinedox.gtex}{}

\vfill
\pagebreak

\cuminitiali{}{temporalia/benedictio-solemn-unigenitus.gtex}

\vspace{7mm}

\lectioii

\noindent \Vbardot{} Tu autem, Dómine, miserére nobis.
\noindent \Rbardot{} Deo grátias.

\vfill
\pagebreak

\pars{Responsorium 2.} \scriptura{\Rbardot{} Cantor; \Vbardot{} Ps. 18, 7; \textbf{H262}}

\vspace{-5mm}

\responsorium{II}{temporalia/resp-omnispulchritudodomini-sinedox.gtex}{}

\vfill
\pagebreak

\cuminitiali{}{temporalia/benedictio-solemn-spiritus.gtex}

\vspace{7mm}

\lectioiii

\noindent \Vbardot{} Tu autem, Dómine, miserére nobis.
\noindent \Rbardot{} Deo grátias.

\vfill
\pagebreak

\pars{Responsorium 3.} \scriptura{\Rbardot{} Ps. 20, 14; \Vbardot{} Ps. 8, 2; \textbf{H262}}

\vspace{-5mm}

\responsorium{VII}{temporalia/resp-exaltaredomine-cumdox.gtex}{}

\vfill
\pagebreak

\subhora{In II. Nocturno}

\pars{Psalmus 4.} \scriptura{Ps. 20, 14; \textbf{H262}}

\vspace{-5mm}

\antiphona{IV A*}{temporalia/ant-exaltaredomine.gtex}

\vspace{-2mm}

\scriptura{Ps. 20}

\vspace{-1mm}

\initiumpsalmi{temporalia/ps20-initium-iv-A_-auto.gtex}

%\psalmusEtTranslatioT{temporalia/ps20-comb.tex}{10cm}
\input{temporalia/ps20.tex} \Abardot{}

\vfill
\pagebreak

\pars{Psalmus 5.} \scriptura{Ps. 29, 2; \textbf{H262}}

\vspace{-5.5mm}

\antiphona{VIII G}{temporalia/ant-exaltabotedomine.gtex}

\vspace{-3mm}

\scriptura{Ps. 29}

\vspace{-2mm}

\initiumpsalmi{temporalia/ps29-initium-viii-G-auto.gtex}

\vspace{-1.5mm}

%\psalmusEtTranslatioT{temporalia/ps29-comb.tex}{10cm}
\input{temporalia/ps29.tex} \Abardot{}

\vspace{-1cm}

\vfill
\pagebreak

\pars{Psalmus 6.} \scriptura{Ps. 46, 6; \textbf{H262}}

%\vspace{-5mm}

\antiphona{IV A*}{temporalia/ant-ascenditdeus.gtex}

%\vspace{-5mm}

\scriptura{Ps. 46}

\initiumpsalmi{temporalia/ps46-initium-iv-A_-auto.gtex}

%\psalmusEtTranslatioT{temporalia/ps46-comb.tex}{10cm}
\input{temporalia/ps46.tex} \Abardot{}

\vfill
\pagebreak

\noindent \Vbardot{} Ascéndens Christus in altum, allelúia.
\noindent \Rbardot{} Captívam duxit captivitátem, allelúia.

\vspace{5mm}

\sineinitiali{temporalia/oratiodominica-mat.gtex}

\vspace{5mm}

\pars{Absolutio.}

\cuminitiali{}{temporalia/absolutio-ipsius.gtex}

\vfill
\pagebreak

\cuminitiali{}{temporalia/benedictio-solemn-deus.gtex}

\vspace{7mm}

\lectioiv

\noindent \Vbardot{} Tu autem, Dómine, miserére nobis.
\noindent \Rbardot{} Deo grátias.

\vfill
\pagebreak

\pars{Responsorium 4.} \scriptura{\Rbardot{} Tob. 12, 20 \& Io. 14, 27; \Vbardot{} Io. 16, 7; \textbf{H263}}

\vspace{-5mm}

\responsorium{IV}{temporalia/resp-tempusest-sinedox.gtex}{}

\vfill
\pagebreak

\cuminitiali{}{temporalia/benedictio-solemn-christus.gtex}

\vspace{7mm}

\lectiov

\noindent \Vbardot{} Tu autem, Dómine, miserére nobis.
\noindent \Rbardot{} Deo grátias.

\vfill
\pagebreak

\pars{Responsorium 5.} \scriptura{\Rbardot{} Cantor super Ioannem; \Vbardot{} Io. 14, 16; \textbf{H263}}

\vspace{-5mm}

\responsorium{III}{temporalia/resp-nonconturbetur-sinedox.gtex}{}

\vfill
\pagebreak

\cuminitiali{}{temporalia/benedictio-solemn-ignem.gtex}

\vspace{7mm}

\lectiovi

\noindent \Vbardot{} Tu autem, Dómine, miserére nobis.
\noindent \Rbardot{} Deo grátias.

\vfill
\pagebreak

\pars{Responsorium 6.} \scriptura{\Rbardot{} Eph. 4, 8; \Vbardot{} Ps. 46, 6; \textbf{H263}}

\vspace{-5mm}

\responsorium{IV}{temporalia/resp-ascendensinaltum-cumdox.gtex}{}

\vfill
\pagebreak

\subhora{In III. Nocturno}

\pars{Psalmus 7.} \scriptura{Ps. 96, 9; \textbf{H263}}

\vspace{-5mm}

\antiphona{VI F}{temporalia/ant-nimisexaltatusest.gtex}

\vspace{-4mm}

\scriptura{Ps. 96}

%\vspace{-2mm}

\initiumpsalmi{temporalia/ps96-initium-vi-F-auto.gtex}

%\psalmusEtTranslatioT{temporalia/ps96-comb.tex}{10cm}
\input{temporalia/ps96.tex} \Abardot{}

\vfill
\pagebreak

\pars{Psalmus 8.} \scriptura{Ps. 98, 2; \textbf{H263}}

\vspace{-5mm}

\antiphona{VI F}{temporalia/ant-dominusinsion.gtex}

\vspace{-4mm}

\scriptura{Ps. 98}

\initiumpsalmi{temporalia/ps98-initium-vi-F-auto.gtex}

%\psalmusEtTranslatioT{temporalia/ps98-comb.tex}{10cm}
\input{temporalia/ps98.tex} \Abardot{}

\vfill
\pagebreak

\pars{Psalmus 9.} \scriptura{Ps. 102, 19; \textbf{H263}}

\vspace{-5mm}

\antiphona{VI F}{temporalia/ant-dominusincoelo.gtex}

\vspace{-4mm}

\scriptura{Ps. 102}

\initiumpsalmi{temporalia/ps102-initium-vi-F-auto.gtex}

%\psalmusEtTranslatioT{temporalia/ps102-comb.tex}{10cm}
\input{temporalia/ps102.tex}

\vfill

\antiphona{}{temporalia/ant-dominusincoelo.gtex}

\vfill
\pagebreak

\noindent \Vbardot{} Ascéndo ad Patrem meum, et Patrem vestrum, allelúia.
\noindent \Rbardot{} Deum meum, et Deum vestrum, allelúia.

\vspace{5mm}

\sineinitiali{temporalia/oratiodominica-mat.gtex}

\vspace{5mm}

\pars{Absolutio.}

\cuminitiali{}{temporalia/absolutio-avinculis.gtex}

\vfill
\pagebreak

\cuminitiali{}{temporalia/benedictio-solemn-evangelica.gtex}

\vspace{7mm}

\lectiovii

\noindent \Vbardot{} Tu autem, Dómine, miserére nobis.
\noindent \Rbardot{} Deo grátias.

\vfill
\pagebreak

\pars{Responsorium 7.} \scriptura{\Rbardot{} Io. 14, 16.17; \Vbardot{} ibid. 16, 7; \textbf{Sar.275}}

\vspace{-5mm}

\responsorium{III}{temporalia/resp-egorogabopatrem-sinedox.gtex}{}

\vfill
\pagebreak

\cuminitiali{}{temporalia/benedictio-solemn-divinum.gtex}

\vspace{7mm}

\lectioviii

\noindent \Vbardot{} Tu autem, Dómine, miserére nobis.
\noindent \Rbardot{} Deo grátias.

\vfill
\pagebreak

\ifx\dominica\undefined
\pars{Responsorium 8.} \scriptura{\Rbardot{} Ps. 103, 3; \Vbardot{} Ps. 103, 1.2; \textbf{H264}}

\vspace{-5mm}

\responsorium{II}{temporalia/resp-ponitnubem-cumdox.gtex}{}
\else
\pars{Responsorium 8.} \scriptura{\Rbardot{} Io. 16, 7; \Vbardot{} ibid. 16, 13; \textbf{Sar.272}}

\vspace{-5mm}

\responsorium{III}{temporalia/resp-sienimnonabiero-cumdox.gtex}{}
\fi

\vfill
\pagebreak

\cuminitiali{}{temporalia/benedictio-solemn-adsocietatem.gtex}

\vspace{7mm}

\lectioix

\noindent \Vbardot{} Tu autem, Dómine, miserére nobis.
\noindent \Rbardot{} Deo grátias.

\vfill
\pagebreak

% Te Deum

%\pars{Hymnus Ambrosianus}

\vspace{-5mm}

\cuminitiali{III}{temporalia/tedeum-solemnis.gtex}

\vfill
\pagebreak

\rubrica{Reliqua omittuntur, nisi Laudes separandæ sint.}

\pars{Oratio}

\noindent \Vbardot{} Dómine, exáudi oratiónem meam.

\noindent \Rbardot{} Et clamor meus ad te véniat.

Orémus:

\ifx\dominica\undefined
\noindent Concéde, quǽsumus, omnípotens Deus: \gredagger{} ut, qui hodiérna die Unigénitum tuum, Redemptórem nostrum, ad cælos ascendísse crédimus; \grestar{} ipsi quoque mente in cæléstibus habitémus. Per eúmdem Dóminum.
\else
\noindent Omnípotens sempitérne Deus: \gredagger{} fac nos tibi semper et devótam gérere voluntátem; \grestar{} et majestáti tuæ sincéro corde servíre. Per Dóminum.
\fi

\noindent \Rbardot{} Amen.

\vspace{7mm}

\pars{Conclusio}

\noindent \Vbardot{} Dómine, exáudi oratiónem meam.

\noindent \Rbardot{} Et clamor meus ad te véniat.

\noindent \Vbardot{} Benedicámus Dómino, allelúia, allelúia.

\noindent \Rbardot{} Deo grátias, allelúia, allelúia.

\noindent \Vbardot{} Fidélium ánimæ per misericórdiam Dei requiéscant in pace.

\noindent \Rbardot{} Amen.

\vfill
\pagebreak

\hora{Ad Laudes.} %%%%%%%%%%%%%%%%%%%%%%%%%%%%%%%%%%%%%%%%%%%%%%%%%%%%%
%\sideThumbs{Laudes}

\cantusSineNeumas

\ifx\postoctavam\undefined
\vspace{0.5cm}
\grechangedim{interwordspacetext}{0.18 cm plus 0.15 cm minus 0.05 cm}{scalable}%
\ifx\festumveldominica\undefined
\cuminitiali{}{temporalia/deusinadiutorium-communis.gtex}
\else
\cuminitiali{}{temporalia/deusinadiutorium-alter.gtex}
\fi
\grechangedim{interwordspacetext}{0.22 cm plus 0.15 cm minus 0.05 cm}{scalable}%

\vfill
%\pagebreak
\else
\rubrica{Absolute incipitur Officium ab Antiphona primi Psalmi.}

\vspace{7mm}
\fi

\pars{Psalmus 1.} \scriptura{Ac. 1, 11; \textbf{H265}}

\vspace{-0.4cm}

\antiphona{VII a}{temporalia/ant-virigalilaeiquidaspicitis.gtex}

\scriptura{Psalmus 92.}

\initiumpsalmi{temporalia/ps92-initium-vii-a-auto.gtex}

\ifx\postoctavam\undefined
%\psalmusEtTranslatioT{temporalia/ps92-comb.tex}{10cm}
\input{temporalia/ps92.tex}

\vfill

\vspace{-1cm}

\antiphona{}{temporalia/ant-virigalilaeiquidaspicitis.gtex}
\else
%\psalmusEtTranslatioT{temporalia/ps92-comb.tex}{10cm}
\input{temporalia/ps92.tex} \Abardot{}
\fi

\vfill
\pagebreak

\pars{Psalmus 2.} \scriptura{Ac. 1, 10; \textbf{H265}}

\vspace{-0.4cm}

\antiphona{VIII G\textsuperscript{2}}{temporalia/ant-cumqueintuerentur.gtex}

\scriptura{Psalmus 99.}

\initiumpsalmi{temporalia/ps99-initium-viii-G2-auto.gtex}

%\psalmusEtTranslatioT{temporalia/ps99-comb.tex}{10cm}
\input{temporalia/ps99.tex} \Abardot{}

\vfill
\pagebreak

\pars{Psalmus 3.} \scriptura{Lc. 24, 50.51; \textbf{H265}}

\vspace{-0.4cm}

\antiphona{IV A*}{temporalia/ant-elevatismanibus.gtex}

\scriptura{Psalmus 62.}

\initiumpsalmi{temporalia/ps62-initium-iv-A_-auto.gtex}

%\psalmusEtTranslatioT{temporalia/ps62-comb.tex}{10cm}
\input{temporalia/ps62.tex} \Abardot{}

%\vfill

%\vspace{-6mm}

%\antiphona{}{temporalia/ant-elevatismanibus.gtex} % repeat the antiphon - new page

\vfill
\pagebreak

\pars{Psalmus 4.} \scriptura{\textbf{H265}}

\vspace{-0.4cm}

\antiphona{VIII G\textsuperscript{2}}{temporalia/ant-exaltateregemregum.gtex}

\scriptura{Canticum trium puerorum, Dan. 3, 57-88 et 56}

\initiumpsalmi{temporalia/dan3-initium-viii-G2-auto.gtex}

%\psalmusEtTranslatioT{temporalia/dan3-comb.tex}{10cm}
\input{temporalia/dan3.tex}

\rubrica{Hic non dicitur Gloria Patri, neque Amen.}

\vfill

\vspace{-6mm}

\antiphona{}{temporalia/ant-exaltateregemregum.gtex} % repeat the antiphon - new page

\vfill
\pagebreak

\pars{Psalmus 5.} \scriptura{Ac. 1, 9; \textbf{H265}}

\vspace{-0.4cm}

\antiphona{VIII G}{temporalia/ant-videntibusillis.gtex}

\scriptura{Psalmus 148.}

\initiumpsalmi{temporalia/ps148-initium-viii-G-auto.gtex}

%\psalmusEtTranslatioT{temporalia/ps148-comb.tex}{10cm}
\input{temporalia/ps148.tex}

\rubrica{Hic non dicitur Gloria Patri.}

\vfill
\pagebreak

%
\scriptura{Psalmus 149.}

\initiumpsalmi{temporalia/ps149-initium-viii-G-auto.gtex}

%\psalmusEtTranslatioT{temporalia/ps149-comb.tex}{10cm}
\input{temporalia/ps149.tex}

\rubrica{Hic non dicitur Gloria Patri.}

\vfill
\pagebreak

%
\scriptura{Psalmus 150.}

\initiumpsalmi{temporalia/ps150-initium-viii-G-auto.gtex}

%\psalmusEtTranslatioT{temporalia/ps150-comb.tex}{10cm}
\input{temporalia/ps150.tex}

\vfill

\vspace{-6mm}

\antiphona{}{temporalia/ant-videntibusillis.gtex} % repeat the antiphon - new page

\vfill
\pagebreak

\capitulumLaudes

% preklad Jeruz. bible
%\trCapituliI

\vfill

\pars{Responsorium breve.} \scriptura{Ps. 46, 6}

\cuminitiali{VI}{temporalia/resp-ascenditdeus.gtex}

%\trResp

\vfill
\pagebreak

\pars{Hymnus}

\cuminitiali{VIII}{temporalia/hym-JesuNostraRedemptio.gtex}
\vspace{-3mm}
%\begin{translatioMulticol}{3}
Výkupné naše, Ježíši,\\
lásko a tužbo nejčistší,\\
tys Tvůrce věcí stvořených\\
a člověk věků posledních.\\
\\
Jaký tě musil soucit vést,\\
žes naše hříchy za své vzal,\\
že chtěl jsi muky smrti nést,\\
bys kletbu smrti z lidí sňal.\columnbreak

Pronikáš v žalář pekelný,\\
propouštíš z něho zajatce.\\
Vítězi, slávou oděný,\\
po boku trůníš u Otce.\\
\\
Kéž donutí té soucit týž,\\
že rány vin v nás zacelíš,\\
nás podle slibu ušetříš\\
a vlídnou tváří potěšíš.\columnbreak

Ty budiž naší radostí,\\
odměnou ve tvé věčnosti,\\
kéž naše sláva veškerá\\
jen z tebe věčně vyvěrá.\\
Amen.
\end{translatioMulticol}


\vfill
%\pagebreak

\pars{Versus.}

% Versus. %%%
\ifx\festum\undefined
\sineinitiali{temporalia/versus-dominusincaelo-communis.gtex}
\else
\sineinitiali{temporalia/versus-dominusincaelo.gtex}
\fi

%\noindent \trVersus

\vfill
\pagebreak

\ifx\dominica\undefined
\pars{Canticum Zachariæ.} \scriptura{Io. 20, 17; \textbf{H265}}

%\vspace{-6mm}

{
\grechangedim{interwordspacetext}{0.18 cm plus 0.15 cm minus 0.05 cm}{scalable}%
\antiphona{VII a}{temporalia/ant-ascendoadpatrem.gtex}
\grechangedim{interwordspacetext}{0.22 cm plus 0.15 cm minus 0.05 cm}{scalable}%
}

%\trAntIMagnificat

%\vspace{-3mm}

\scriptura{Lc. 1, 68-79}

%\vspace{-2.5mm}

\cantusSineNeumas
\initiumpsalmi{temporalia/benedictus-initium-viisoll-a-auto.gtex}

%\vspace{-1.5mm}

%\psalmusEtTranslatioT{temporalia/benedictus-I-comb.tex}{10.2cm}
\input{temporalia/benedictus-I.tex} \Abardot{}
\else
\pars{Canticum Zachariæ.} \scriptura{Io. 15, 26; \textbf{H267}}

\vspace{-3mm}

{
\grechangedim{interwordspacetext}{0.18 cm plus 0.15 cm minus 0.05 cm}{scalable}%
\antiphona{VIII G}{temporalia/ant-cumveneritparaclitus.gtex}
\grechangedim{interwordspacetext}{0.22 cm plus 0.15 cm minus 0.05 cm}{scalable}%
}

%\trAntIMagnificat

%\vspace{-3mm}

\scriptura{Lc. 1, 68-79}

%\vspace{-2.5mm}

\cantusSineNeumas
\initiumpsalmi{temporalia/benedictus-initium-viiisoll-G-auto.gtex}

%\vspace{-1.5mm}

%\psalmusEtTranslatioT{temporalia/benedictus-II-comb.tex}{10.2cm}
\input{temporalia/benedictus-II.tex}

\vfill

\antiphona{}{temporalia/ant-cumveneritparaclitus.gtex}
\fi

\vspace{-1cm}

\vfill
\pagebreak

%\sideThumbs{{\scriptsize{}Fine horarum}}

\anteOrationem

\pagebreak

% Oratio. %%%
\ifx\dominica\undefined
\cuminitiali{}{temporalia/oratio.gtex}
\else
\cuminitiali{}{temporalia/oratio2.gtex}
\fi

\vspace{-1mm}
%\trOrationisI

\vfill

\rubrica{Hebdomadarius dicit iterum Dominus vobiscum, vel cantor dicit:}

\vspace{2mm}

\sineinitiali{temporalia/domineexaudi.gtex}

\rubrica{Postea cantatur a cantore:}

\vspace{2mm}

\ifx\festum\undefined
\cuminitiali{VII}{temporalia/benedicamus-tempore-paschali.gtex}
\else
\cuminitiali{II}{temporalia/benedicamus-solemnism-laud.gtex}
\fi

\vspace{1mm}

\vfill
\pagebreak

\ifx\sabbato\undefined
\ifx\festumveldominica\undefined
\hora{Ad Vesperas.} %%%%%%%%%%%%%%%%%%%%%%%%%%%%%%%%%%%%%%%%%%%%%%%%%%%%%
%\sideThumbs{Vesperæ}
\else
\hora{Ad II. Vesperas.} %%%%%%%%%%%%%%%%%%%%%%%%%%%%%%%%%%%%%%%%%%%%%%%%%%%%%
%\sideThumbs{II. Vesperæ}
\fi

\cantusSineNeumas

%\vspace{-2mm}
\grechangedim{interwordspacetext}{0.18 cm plus 0.15 cm minus 0.05 cm}{scalable}%
\cuminitiali{}{temporalia/deusinadiutorium-solemnis.gtex}
\grechangedim{interwordspacetext}{0.22 cm plus 0.15 cm minus 0.05 cm}{scalable}%

\vfill
%\pagebreak

%\vspace{-2mm}

\pars{Psalmus 1.} \scriptura{Ps. 109, 4; Gn 14, 18}

\vspace{-0.4cm}

\antiphona{I f}{temporalia/ant-sacerdosinaeternum.gtex}

\scriptura{Psalmus 109.}

\initiumpsalmi{temporalia/ps109-initium-i-f-auto.gtex}

%\psalmusEtTranslatioT{temporalia/ps109-comb.tex}{10cm}
\input{temporalia/ps109.tex} \Abardot{}

%\vfill

%\antiphona{}{temporalia/ant-sacerdosinaeternum.gtex}

%\vspace{-1cm}

\vfill
\pagebreak

\vesperasi

\ifx\festum\undefined
\vesperasii
\else
\vesperasiii
\fi

\capitulumLaudes

% preklad Jeruz. bible
%\trCapituliI

\vfill

\pars{Responsorium breve.} \scriptura{Ps. 80, 17}

\ifx\festum\undefined
\cuminitiali{VI}{temporalia/resp-cibavit-simplex.gtex}
\else
\cuminitiali{VI}{temporalia/resp-cibavit.gtex}
\fi

%\trResp

\vfill
\pagebreak

\vesperasiv

\ifx\dominica\undefined
\pars{Canticum B. Mariæ V.} \scriptura{Thomas de Aquino?}

%\vspace{-5.5mm}

{
\grechangedim{interwordspacetext}{0.18 cm plus 0.15 cm minus 0.05 cm}{scalable}%
\antiphona{V a}{temporalia/ant-osacrumconvivium.gtex}
\grechangedim{interwordspacetext}{0.22 cm plus 0.15 cm minus 0.05 cm}{scalable}%
}

%\trAntIMagnificat

\vspace{-3mm}

\scriptura{Lc. 1, 46-55}

\vspace{-2.5mm}

\cantusSineNeumas
\initiumpsalmi{temporalia/magnificat-initium-vsoll-a_.gtex}

\vspace{-1.5mm}

%\psalmusEtTranslatioT{temporalia/magnificat-II-comb.tex}{10.2cm}
\input{temporalia/magnificat-II.tex} \Abardot{}
\else
\pars{Canticum B. Mariæ V.} \scriptura{Io. 16, 4}

{
\grechangedim{interwordspacetext}{0.18 cm plus 0.15 cm minus 0.05 cm}{scalable}%
\antiphona{VIII G}{temporalia/ant-xxx.gtex}
\grechangedim{interwordspacetext}{0.22 cm plus 0.15 cm minus 0.05 cm}{scalable}%
}

%\trAntIMagnificat

\scriptura{Lc. 1, 46-55}

\cantusSineNeumas
\initiumpsalmi{temporalia/magnificat-initium-viiisoll-G.gtex}

%\psalmusEtTranslatioT{temporalia/magnificat-III-comb.tex}{10.2cm}
\input{temporalia/magnificat-III.tex} \Abardot{}
\fi

\vspace{-1cm}

\vfill
\pagebreak

%\sideThumbs{{\scriptsize{}Fine horarum}}

\anteOrationem

\pagebreak

% Oratio. %%%
\ifx\dominica\undefined
\cuminitiali{}{temporalia/oratio.gtex}
\else
\cuminitiali{}{temporalia/oratio2.gtex}
\fi

\vspace{-1mm}
%\trOrationisI

\vfill

\rubrica{Hebdomadarius dicit iterum Dominus vobiscum, vel cantor dicit:}

\vspace{2mm}

\sineinitiali{temporalia/domineexaudi.gtex}

\rubrica{Postea cantatur a cantore:}

\vspace{2mm}

\ifx\festum\undefined
\cuminitiali{VII}{temporalia/benedicamus-tempore-paschali.gtex}
\else
\cuminitiali{II}{temporalia/benedicamus-solemnism-2vesp.gtex}
\fi

\vspace{1mm}
\fi

\end{document}

