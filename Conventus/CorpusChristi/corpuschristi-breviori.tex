\newcommand{\titulus}{\nomenFesti{In Festo Sanctissimi Corporis Christi.}}
\newcommand{\festum}{Corpus Christi}
\newcommand{\aequus}{Corpus Christi}
\newcommand{\festumveldominica}{Corpus Christi}
\newcommand{\solemnis}{Corpus Christi}
\newcommand{\oratio}{\noindent Orémus. 

\noindent Deus, qui nobis sub Sacraménto mirábili passiónis tuæ memóriam reliquísti:~\gredagger{} tríbue, quǽsumus, ita nos Córporis, et Sánguinis tui sacra mystéria venerári;~\grestar{} ut redemptiónis tuæ fructum in nobis iúgiter sentiámus.

\pars{Pro pace in universo mundo.} \scriptura{Sir. 50, 25; 2 Esdr. 4, 20; \textbf{H416}}

\vspace{-4mm}

\antiphona{II D}{temporalia/ant-dapacemdomine.gtex}

\vfill

\noindent Deus, a quo sancta desidéria, recta consília et iusta sunt ópera: da servis tuis illam, quam mundus dare non potest, pacem; ut et corda nostra mandátis tuis dédita, et hóstium subláta formídine, témpora sint tua protectióne tranquílla.

\noindent Per Dóminum nostrum Iesum Christum, Fílium tuum, qui tecum vivit et regnat in unitáte Spíritus Sancti, Deus, per ómnia sǽcula sæculórum.

\noindent \Rbardot{} Amen.}
\newcommand{\nocturnoi}{\pars{Psalmus 1.} \scriptura{Mt. 22, 4}

\vspace{-4mm}

\antiphona{VIII G\textsuperscript{2}}{temporalia/ant-diciteinvitatis.gtex}

%\vspace{-3mm}

\scriptura{Ps. 22}

%\vspace{-2mm}

\initiumpsalmi{temporalia/ps22-initium-viii-G5-auto.gtex}

%\vspace{-1.5mm}

\input{temporalia/ps22viiiG5.tex} \Abardot{}

\vfill
\pagebreak

\pars{Psalmus 2.} \scriptura{Io. 7, 37.38}

\vspace{-4mm}

\antiphona{II* a}{temporalia/ant-siquissititveniat.gtex}

%\vspace{-5mm}

\scriptura{Ps. 41}

\initiumpsalmi{temporalia/ps41-initium-ii_-a-auto.gtex}

\input{temporalia/ps41ii_a.tex}

\vfill

\antiphona{}{temporalia/ant-siquissititveniat.gtex}

\vfill
\pagebreak

\pars{Psalmus 3.} \scriptura{Ps. 80, 17; \textbf{LU\textsubscript{935}}}

\vspace{-5mm}

\antiphona{VIII G}{temporalia/ant-cibavitnos-FKP.gtex}

\vspace{-3mm}

\scriptura{Ps. 80}

\vspace{-2mm}

\initiumpsalmi{temporalia/ps80-initium-viii-G-auto.gtex}

\vspace{-1mm}

\input{temporalia/ps80.tex} \Abardot{}

\vfill
\pagebreak}
\newcommand{\nocturnoii}{\pars{Cantica.}

\vspace{-4mm}

\antiphona{VII d}{temporalia/ant-caromeavereest.gtex}

%\vspace{-2mm}

\scriptura{Canticum Proverbiæ, Prv. 9, 1-6.10-12}

%\vspace{-2mm}

\initiumpsalmi{temporalia/proverbiae-initium-vii-d-auto.gtex}

\input{temporalia/proverbiae.tex} \hfill \rubrica{Hic non dicitur antiphona.}

\vfill
\pagebreak

\scriptura{Canticum Ieremiæ, 1 Ier. 31, 10-14}

%\vspace{-3mm}

\initiumpsalmi{temporalia/jeremiae3-initium-vii-d-auto.gtex}

\input{temporalia/jeremiae3.tex}

\vfill
\pagebreak

\scriptura{Canticum Sapientiæ, Sap. 16, 20-21.26; 17, 1}

%\vspace{-2mm}

\initiumpsalmi{temporalia/sapientia2-initium-vii-d-auto.gtex}

\input{temporalia/sapientia2.tex}

\vfill

\antiphona{}{temporalia/ant-caromeavereest.gtex}

\vfill
\pagebreak}
\newcommand{\lectioi}{\pars{Lectio I.} \scriptura{Ex. 24, 1-11}

\noindent De libro Exodi.

\noindent In diébus illis: Dixit Dóminus Móysi: «Ascénde ad Dóminum, tu et Aaron, Nadab et Abiu et septuagínta senes ex Israel, et adorábitis procul. Solúsque Móyses ascéndet ad Dóminum, et illi non appropinquábunt, nec pópulus ascéndet cum eo». Venit ergo Móyses et narrávit plebi ómnia verba Dómini atque iudícia; respondítque omnis pópulus una voce: «Omnia verba Dómini, quæ locútus est, faciémus». Scripsit autem Móyses univérsos sermónes Dómini; et mane consúrgens ædificávit altáre ad radíces montis et duódecim lápides per duódecim tribus Israel.

\noindent Misítque iúvenes de fíliis Israel, et obtulérunt holocáusta; immolaverúntque víctimas pacíficas Dómino vítulos. Tulit ítaque Móyses dimídiam partem sánguinis et misit in cratéras; partem autem resíduam respérsit super altáre. Assuménsque volúmen fœ́deris legit, audiénte pópulo, qui dixérunt: «Omnia, quæ locútus est Dóminus, faciémus et érimus obœdiéntes». Ille vero sumptum sánguinem respérsit in pópulum et ait: «Hic est sanguis fœ́deris, quod pépigit Dóminus vobíscum super cunctis sermónibus his».

\noindent Ascenderúntque Móyses et Aaron, Nadab et Abiu et septuagínta de senióribus Israel. Et vidérunt Deum Israel, et sub pédibus eius quasi opus lápidis sapphiríni et quasi ipsum cælum, cum serénum est. Nec in eléctos filiórum Israel misit manum suam; viderúntque Deum et comedérunt ac bibérunt.}
\newcommand{\lectioii}{\pars{Lectio II.} \scriptura{Opusculum 57, in festo Corporis Christi, lect. 1-4}

\noindent Ex Opéribus sancti Thomæ de Aquíno presbýteri.

\noindent Unigénitus Dei Fílius, suæ divinitátis volens nos esse partícipes, natúram nostram assúmpsit, ut hómines deos fáceret factus homo.

\noindent Et hoc ínsuper, quod de nostro assúmpsit, totum nobis cóntulit ad salútem. Corpus namque suum pro nostra reconciliatióne in ara crucis hóstiam óbtulit Deo Patri, sánguinem suum fudit in prétium simul et lavácrum; ut redémpti a miserábili servitúte, a peccátis ómnibus mundarémur.

\noindent Ut autem tanti benefícii iugis in nobis manéret memória, corpus suum in cibum, et sánguinem suum in potum, sub spécie panis et vini suméndum fidélibus derelíquit.}
\newcommand{\lectioiii}{\pars{Lectio III.}

\noindent O pretiósum et admirándum convívium, salutíferum et omni suavitáte replétum! Quid enim hoc convívio pretiósius esse potest? in quo non carnes vitulórum et hircórum, ut olim in lege, sed nobis Christus suméndus propónitur verus Deus. Quid hoc sacraménto mirabílius?

\noindent Nullum étiam sacraméntum est isto salúbrius, quo purgántur peccáta, virtútes augéntur, et mens ómnium spiritálium charísmatum abundántia impinguátur.

\noindent Offértur in Ecclésia pro vivis et mórtuis, ut ómnibus prosit, quod est pro salúte ómnium institútum.

\noindent Suavitátem dénique huius sacraménti nullus exprímere súfficit, per quod spiritális dulcédo in suo fonte gustátur; et recólitur memória illíus, quam in sua passióne Christus monstrávit, excellentíssimæ caritátis.

\noindent Unde, ut árctius huius caritátis imménsitas fidélium córdibus infigerétur, in última cena, quando, Pascha cum discípulis celebráto, transitúrus erat de hoc mundo ad Patrem, hoc sacraméntum instítuit, tamquam passiónis suæ memoriále perénne, figurárum véterum impletívum, miraculórum ab ipso factórum máximum, et de sua contristátis abséntia solácium singuláre relíquit.}
\newcommand{\lectioiv}{\pars{Lectio IV.} \scriptura{Mc. 14, 12-16, 22-26}

\noindent Léctio sancti Evangélii secúndum Marcum.

\noindent Primo die Azymórum, quando Pascha immolábant, dicunt Iesu discípuli eius: «Quo vis eámus et parémus, ut mandúces Pascha?»

\noindent Et mittit duos ex discípulis suis et dicit eis: «Ite in civitátem, et occúrret vobis homo lagœ́nam aquæ báiulans; sequímini eum et, quocúmque introíerit, dícite dómino domus: Magíster dicit: Ubi est reféctio mea, ubi Pascha cum discípulis meis mandúcem? Et ipse vobis demonstrábit cenáculum grande stratum parátum; et illic paráte nobis». Et abiérunt discípuli et venérunt in civitátem et invenérunt, sicut díxerat illis, et paravérunt Pascha.

\noindent Et manducántibus illis, accépit panem et benedícens fregit et dedit eis et ait: «Súmite: hoc est corpus meum». Et accépto cálice, grátias agens dedit eis; et bibérunt ex illo omnes. Et ait illis: «Hic est sanguis meus novi testaménti, qui pro multis effúnditur. Amen dico vobis: Iam non bibam de genímine vitis usque in diem illum, cum illud bibam novum in regno Dei».

\noindent Et hymno dicto, exiérunt in montem Olivárum.}
\newcommand{\responsoriumi}{\pars{Responsorium 1.} \scriptura{\Rbardot{} Ex. 12, 5.6.8 \Vbardot{} 1 Cor. 5, 7.8; \textbf{LU\textsubscript{926}}}

\vspace{-2mm}

\responsorium{I}{temporalia/resp-immolabithaedum.gtex}{}

\vfill

\rubrica{vel ad libitum:}

\vspace{3mm}

\pars{Responsorium 1.} \scriptura{\Rbardot{} Ex. 16, 12.15 \Vbardot{} Io. 6, 21; \textbf{LU\textsubscript{927}}}

\vspace{-2mm}

\responsorium{II}{temporalia/resp-comedetiscarnes.gtex}{}}
\newcommand{\responsoriumii}{\pars{Responsorium 2.} \scriptura{\Rbardot{} Mt. 26, 26 \Vbardot{} Io. 31, 31; \textbf{LU\textsubscript{931}}}

\vspace{-2mm}

\responsorium{V}{temporalia/resp-coenantibus.gtex}{}

\vfill

\rubrica{vel ad libitum:}

\vspace{3mm}

\pars{Responsorium 2.} \scriptura{\Rbardot{} 1 Cor. 11, 25 \Vbardot{} Thren. 3, 20; \textbf{LU\textsubscript{932}}}

\vspace{-2mm}

\responsorium{VI}{temporalia/resp-accepitiesus.gtex}{}}
\newcommand{\responsoriumiii}{\pars{Responsorium 3.} \scriptura{\Rbardot{} Io. 6, 48 \Vbardot{} ibid. 6, 51.52}

\vspace{-5mm}

\responsorium{VII}{temporalia/resp-egosumpanisvitae-E611-cumdox.gtex}{}

\vfill

\rubrica{vel ad libitum:}

\vspace{3mm}

\pars{Responsorium 3.} \scriptura{\Rbardot{} 3 Reg. 19, 6.8 \Vbardot{} Io. 6, 52; \textbf{LU\textsubscript{927}}}

\vspace{-2mm}

\responsorium{III}{temporalia/resp-respexitelias.gtex}{}}
\newcommand{\responsoriumiv}{\pars{Responsorium 4.} \scriptura{\Rbardot{} Io. 6, 57 \Vbardot{} Dt. 4, 7}

\vspace{-2mm}

\responsorium{VII}{temporalia/resp-quimanducat-E611-cumdoxalt.gtex}{}}
\newcommand{\laudes}{\pars{Hymnus} \scriptura{Thomas de Aquino?}

\cuminitiali{VIII}{temporalia/hym-VerbumSupernum.gtex}

\vfill
\pagebreak

\pars{Psalmus 1.} \scriptura{Sap. 16, 20}

\vspace{-0.4cm}

\antiphona{II D}{temporalia/ant-angelorumesca.gtex}

\scriptura{Psalmus 62.}

\initiumpsalmi{temporalia/ps62-initium-ii-D-auto.gtex}

\input{temporalia/ps62iiD.tex} \Abardot{}

\vfill
\pagebreak

\pars{Psalmus 2.} \scriptura{Cf. Lv. 21, 6}

\vspace{-0.4cm}

\antiphona{IV E}{temporalia/ant-sacerdotessancti.gtex}

\scriptura{Canticum trium puerorum, Dan. 3, 57-88 et 56}

\initiumpsalmi{temporalia/dan3-initium-iv-E-auto.gtex}

\input{temporalia/dan3.tex}

\rubrica{Hic non dicitur Gloria Patri, neque Amen.}

\vfill

\vspace{-6mm}

\antiphona{}{temporalia/ant-sacerdotessancti.gtex} % repeat the antiphon - new page

\vfill
\pagebreak

\pars{Psalmus 3.} \scriptura{Ap. 2, 17}

\vspace{-0.4cm}

\antiphona{V a}{temporalia/ant-vincentidabo.gtex}

\scriptura{Psalmus 149.}

\initiumpsalmi{temporalia/ps149-initium-v-a-auto.gtex}

\input{temporalia/ps149.tex}

\begin{psalmus}

Glória Pa\-tri et \textbf{Fí}\-lio,~\grestar{} 
et Spi\textbf{rí}\-tui \textbf{Sanc}\-to.

Sicut erat in princípio, et nunc et \textbf{sem}\-per,~\grestar{} 
et in sǽcula sæcu\textbf{ló}\-rum. \textbf{A}\-men.
\end{psalmus} \Abardot{}

\vfill
\pagebreak}
\newcommand{\lectiobrevis}{\pars{Lectio Brevis.} \scriptura{Mal. 1, 11}

\noindent Ab ortu solis usque ad occásum magnum est nomen meum in géntibus, et in omni loco sacrificátur et offértur nómini meo oblátio munda, quia magnum nomen meum in géntibus, dicit Dóminus exercítuum.}
\newcommand{\responsoriumbreve}{\pars{Responsorium breve.} \scriptura{Ps. 103, 14-15}

\antiphona{VI}{temporalia/resp-educaspanem.gtex}}
\newcommand{\preces}{\noindent Fratres, Iesum Christum, panem vitæ deprecémur,~\grestar{} lætánter dicéntes:

\Rbardot{} Beátus qui manducábit panem in regno tuo, Dómine.

\noindent Christe, sacérdos novi et ætérni fœ́deris,~\gredagger{} qui Patri in ara crucis sacrifícium perféctum obtulísti,~\grestar{} doce nos tecum illud offérre.

\Rbardot{} Beátus qui manducábit panem in regno tuo, Dómine.

\noindent Christe, summe rex pacis et iustítiæ,~\gredagger{} qui panem et vinum in signum oblatiónis tui consecrásti,~\grestar{} tecum nos hóstias consócia.

\Rbardot{} Beátus qui manducábit panem in regno tuo, Dómine.

\noindent Christe, vere adorátor Patris,~\gredagger{} cuius oblátio munda ab ortu solis usque ad occásum ab Ecclésia offértur,~\grestar{} in tuo córpore únias quos uno pane sátias.

\Rbardot{} Beátus qui manducábit panem in regno tuo, Dómine.

\noindent Christe, manna de cælo descéndens,~\gredagger{} qui Ecclésiam pascis córpore et sánguine tuo,~\grestar{} in fortitúdine huius cibi fac nos ambuláre.

\Rbardot{} Beátus qui manducábit panem in regno tuo, Dómine.

\noindent Christe, invisíbilis hospes convívii nostri,~\gredagger{} qui stas ad óstium et pulsas,~\grestar{} ad nos veni, cena nobíscum et nos tecum.

\Rbardot{} Beátus qui manducábit panem in regno tuo, Dómine.}
\newcommand{\benedictioiv}{\cuminitiali{}{temporalia/benedictio-solemn-evangelica.gtex}}
% LuaLaTeX

\documentclass[a4paper, twoside, 12pt]{article}
\usepackage[latin]{babel}
%\usepackage[landscape, left=3cm, right=1.5cm, top=2cm, bottom=1cm]{geometry} % okraje stranky
%\usepackage[landscape, a4paper, mag=1166, truedimen, left=2cm, right=1.5cm, top=1.6cm, bottom=0.95cm]{geometry} % okraje stranky
\usepackage[landscape, a4paper, mag=1400, truedimen, left=0.5cm, right=0.5cm, top=0.5cm, bottom=0.5cm]{geometry} % okraje stranky

\usepackage{fontspec}
\setmainfont[FeatureFile={junicode.fea}, Ligatures={Common, TeX}, RawFeature=+fixi]{Junicode}
%\setmainfont{Junicode}

% shortcut for Junicode without ligatures (for the Czech texts)
\newfontfamily\nlfont[FeatureFile={junicode.fea}, Ligatures={Common, TeX}, RawFeature=+fixi]{Junicode}

\usepackage{multicol}
\usepackage{color}
\usepackage{lettrine}
\usepackage{fancyhdr}

% usual packages loading:
\usepackage{luatextra}
\usepackage{graphicx} % support the \includegraphics command and options
\usepackage{gregoriotex} % for gregorio score inclusion
\usepackage{gregoriosyms}
\usepackage{wrapfig} % figures wrapped by the text
\usepackage{parcolumns}
\usepackage[contents={},opacity=1,scale=1,color=black]{background}
\usepackage{tikzpagenodes}
\usepackage{calc}
\usepackage{longtable}
\usetikzlibrary{calc}

\setlength{\headheight}{14.5pt}

\input{conventuscommune.tex} % Often used macros
%%%% Preklady jednotlivych zpevu (nektere se opakuji, a je dobre mit je
% vsechny na jedne hromade)

% HOURS ---

\newcommand{\trAntI}{\translatioCantus{Muž boží měl kožený toulec, pečlivě
zavázaný, jenž mu visel na šíji a~často se ho dotýkal.}}

\newcommand{\trAntII}{\translatioCantus{Klíč od~něho tak dobře střežil, že
dokud žil v~těle, nikdo z~jeho žáků nezvěděl, co je uvnitř.}}

\newcommand{\trAntIII}{\translatioCantus{Ale když se odebral z~tohoto
života, schránku otevřeli a~objevili v~ní žíněné roucho a~měděný řetěz
potřísněný krví.}}

\newcommand{\trAntIV}{\translatioCantus{A když prohlédli mistrovo tělo,
nalezli jeho tělo na čtyřech místech hluboce zbrázděno ranami od řetězu.}}

\newcommand{\trAntV}{\translatioCantus{Krev vytékající z~těch ran, místy
prostoupila i~žíněným rouchem.}}

\newcommand{\trCapituli}{\translatioCantus{
Miláčkovi Boha a~lidí,
Mojžíšovi požehnané paměti,~\gredagger{}
dopřál slávu rovnou slávě svatých~\grestar{}
učinil ho mocným na postrach nepřátelům
a~jeho slovy zastavil divy.}}

\newcommand{\trLectioBrevis}{\translatioCantus{
Pamatujte na své představené,
kteří vám hlásali Boží slovo.
Uvažte, jak oni skončili život, a~napodobujte jejich víru.
Ježíš Kristus je stejný včera i~dnes i~navěky.
Nenechte se svést věelijakými cizími naukami.}}

\newcommand{\trRespLaud}{\translatioCantus{Spravedlivého vodil Hospodin~\grestar{}
po přímých stezkách. \Vbardot{} A~ukázal mu Boží království.}}

\newcommand{\trRespLaudB}{\translatioCantus{Na tvých hradbách, Jeruzaléme,
ustanovil jsem strážné;~\grestar{}
budou bdít nad mým lidem. \Vbardot{} Ani ve dne, ani v~noci nesmějí nikdy
mlčet.}}

\newcommand{\trVersus}{\translatioCantus{\Vbardot{} Ústa spravedlivého šeptají moudrost, aleluja.
\Rbardot{} A~jeho jazyk ohlašuje právo, aleluja.}}

\newcommand{\trAntBenedictus}{\translatioCantus{Když na bujné oře vložili
nosítka a~sňali jim uzdu, vydali se přímo k~cele božího muže.}}

\newcommand{\trPreces}{\translatioCantus{
\noindent S vděčností chvalme Krista, dobrého Pastýře, \gredagger{} který dal život za své ovce, \grestar{} a~pokorně ho prosme: \Rbardot{} Pane, buď pastýřem svého lidu.

\noindent Kriste, ty dáváš církvi pastýře, a~jejich službou se ujímáš svého lidu, \grestar{} dej, ať v~lásce těch, kteří nás vedou, poznáváme, jak nás miluješ. \Rbardot{} Pane, buď pastýřem svého lidu.

\noindent Ty stále konáš skrze své zástupce službu pastýře a~učitele, \grestar{} nepřestávej nás nikdy vést prostřednictvím svých služebníků. \Rbardot{} Pane, buď pastýřem svého lidu.

\noindent Ty prokazuješ svému lidu skrze jeho pastýře službu lékaře duše i~těla, \grestar{} ochraňuj náš život a~veď nás ke svatosti. \Rbardot{} Pane, buď pastýřem svého lidu.

\noindent Ty posíláš své svaté, aby slovem i~příkladem vedli tvůj lid k~tobě, \grestar{} na jejich přímluvu nás posiluj, abychom vytrvali na cestě, která vede k~věčnému životu. \Rbardot{} Pane, buď pastýřem svého lidu.}}

\newcommand{\trOrationis}{\translatioCantus{Bože, jenž nám dopřáváš radovat
se z~výroční slavnosti svatého tvého vyznavače Havla, uděl dobrotivě,
abychom když slavíme jeho narození, též se řídili podobou jeho skutků.
Skrze…}}
 % Czech translations of the proper texts

\newcommand{\annusEditionis}{2020}

%%%% Vicekrat opakovane kousky

\newcommand{\anteOrationem}{
  \rubrica{Ante Orationem, cantatur a Superiore:}

  \pars{Supplicatio Litaniæ.}

  \cuminitiali{}{temporalia/supplicatiolitaniae.gtex}

  \pars{Oratio Dominica.}

  \cuminitiali{}{temporalia/oratiodominica.gtex}

  \rubrica{Deinde dicitur ab Hebdomadario:}

  \cuminitiali{}{temporalia/dominusvobiscum-solemnis.gtex}

  \rubrica{In choro monialium loco Dominus vobiscum dicitur:}

  \sineinitiali{temporalia/domineexaudi.gtex}
}

\ifx\dominica\undefined
\newcommand{\capitulumLaudes}{\pars{Capitulum.} \scriptura{1 Cor. 11, 23-24}

\grechangedim{interwordspacetext}{0.12 cm plus 0.15 cm minus 0.05 cm}{scalable}%
\cuminitiali{}{temporalia/capitulum-FratresEgo.gtex}
\grechangedim{interwordspacetext}{0.22 cm plus 0.15 cm minus 0.05 cm}{scalable}}
\else
\newcommand{\capitulumLaudes}{\pars{Capitulum.} \scriptura{1 Ptr. 4, 7-8}

\grechangedim{interwordspacetext}{0.12 cm plus 0.15 cm minus 0.05 cm}{scalable}%
\cuminitiali{}{temporalia/capitulum-CarissimiEstote.gtex}
\grechangedim{interwordspacetext}{0.22 cm plus 0.15 cm minus 0.05 cm}{scalable}}
\fi

\setlength{\columnsep}{30pt} % prostor mezi sloupci

%%%%%%%%%%%%%%%%%%%%%%%%%%%%%%%%%%%%%%%%%%%%%%%%%%%%%%%%%%%%%%%%%%%%%%%%%%%%%%%%%%%%%%%%%%%%%%%%%%%%%%%%%%%%%
\begin{document}

% Here we set the space around the initial.
% Please report to http://home.gna.org/gregorio/gregoriotex/details for more details and options
\grechangedim{afterinitialshift}{2.2mm}{scalable}
\grechangedim{beforeinitialshift}{2.2mm}{scalable}
\grechangedim{interwordspacetext}{0.22 cm plus 0.15 cm minus 0.05 cm}{scalable}%
\grechangedim{annotationraise}{-0.2cm}{scalable}

% Here we set the initial font. Change 38 if you want a bigger initial.
% Emit the initials in red.
\grechangestyle{initial}{\color{red}\fontsize{38}{38}\selectfont}

\pagestyle{empty}

\newcommand{\vesperasi}{
\pars{Psalmus 2.} \scriptura{Ps. 110, 4-5}

\vspace{-0.4cm}

\antiphona{II D}{temporalia/ant-miseratordominus.gtex}

\scriptura{Psalmus 110.}

\initiumpsalmi{temporalia/ps110-initium-ii-D-auto.gtex}

%\psalmusEtTranslatioT{temporalia/ps110-comb.tex}{10cm}
\input{temporalia/ps110.tex} \Abardot{}

\vfill
\pagebreak

\pars{Psalmus 3.} \scriptura{Ps. 115, 4.8}

\vspace{-0.4cm}

\antiphona{III a}{temporalia/ant-calicemsalutaris.gtex}

\scriptura{Psalmus 115.}

\initiumpsalmi{temporalia/ps115-initium-iii-a-auto.gtex}

%\psalmusEtTranslatioT{temporalia/ps115-comb.tex}{10cm}
\input{temporalia/ps115.tex} \Abardot{}

\vfill
\pagebreak
}

\newcommand{\vesperasii}{
\pars{Psalmus 4.} \scriptura{Ps. 127, 3}

\vspace{-0.4cm}

\antiphona{IV E}{temporalia/ant-sicutnovellae.gtex}

\scriptura{Psalmus 127.}

\initiumpsalmi{temporalia/ps127-initium-iv-E-auto.gtex}

%\psalmusEtTranslatioT{temporalia/ps127-comb.tex}{10cm}
\input{temporalia/ps127.tex} \Abardot{}

%\vfill

%\vspace{-6mm}

%\antiphona{}{temporalia/ant-sicutnovellae.gtex} % repeat the antiphon - new page

\vfill
\pagebreak
}

\newcommand{\vesperasiii}{
\pars{Psalmus 4.} \scriptura{Cf. Ps. 147, 3}

\vspace{-0.4cm}

\antiphona{V g}{temporalia/ant-quipacem.gtex}

\scriptura{Psalmus 147.}

\initiumpsalmi{temporalia/ps147-initium-v-g-auto.gtex}

%\psalmusEtTranslatioT{temporalia/ps147-comb.tex}{10cm}
\input{temporalia/ps147.tex} \Abardot{}

%\vfill

%\vspace{-6mm}

%\antiphona{}{temporalia/ant-sicutnovellae.gtex} % repeat the antiphon - new page

\vfill
\pagebreak
}

\newcommand{\vesperasiv}{
\pars{Hymnus} \scriptura{Thomas de Aquino?}

\cuminitiali{IV}{temporalia/hym-PangeLingua.gtex}
\vspace{-3mm}
%\input{hym-PangeLingua-bohtext.tex}

\vfill
\pagebreak

\pars{Versus.} \scriptura{Sap. 16, 20}

% Versus. %%%
\ifx\festum\undefined
\sineinitiali{temporalia/versus-panemdecaelo-communis.gtex}
\else
\sineinitiali{temporalia/versus-panemdecaelo.gtex}
\fi

%\noindent \trVersus

\vfill
\pagebreak
}

%%%% Titulni stranka
\begin{titulusOfficii}
\titulus
\end{titulusOfficii}

% graphic
%\vspace{1.5cm}
%\begin{center}
%\includegraphics[width=8cm]{emmaus.jpg}
%\end{center}

\vfill

\begin{center}
%Ad usum et secundum consuetudines chori \guillemotright{}Conventus Choralis\guillemotleft.

%Editio Sancti Wolfgangi \annusEditionis
\end{center}

\pagebreak

\renewcommand{\headrulewidth}{0pt} % no horiz. rule at the header
\fancyhf{}
\pagestyle{fancy}

\cantusSineNeumas

\ifx\festumveldominica\undefined
\else
\pars{Oratio ante divinum Officium.}

\lettrine{{\color{red}A}}{peri,} Dómine, os meum ad benedicéndum nomen sanctum tuum:
munda quoque cor meum ab ómnibus vanis, pervérsis, et aliénis
cogitatiónibus:
intelléctum illúmina, afféctum inflámma,
ut digne, atténte ac devóte hoc Offícium recitáre váleam,
et exaudíri mérear ante conspéctum Divínæ Maiestátis tuæ.
Per Christum, Dóminum nostrum.
\Rbardot{} Amen.

Dómine, in unióne illíus divínæ intentiónis,
qua ipse in terris laudes Deo persolvísti,
has tibi Horas \rubricatum{(vel \textnormal{hanc tibi Horam})} persólvo.

%\trOratioAnteOfficium

\vfill

\pars{Oratio post divinum Officium.}

\rubrica{
  Orationem sequentem devote post Officium recitantibus
  Leo Papa X. defectus, et culpas in eo persolvendo ex humana
  fragilitate contractas, indulsit, et dicitur flexis genibus.
}

\lettrine{{\color{red}S}}{acrosánctæ} et indivíduæ Trinitáti,
crucifíxi Dómini nostri Iesu Christi humanitáti,
beatíssimæ et gloriosíssimæ sempérque Vírginis Maríæ
fecúndæ integritáti, 
et ómnium Sanctórum universitáti
sit sempitérna laus, honor, virtus et glória
ab omni creatúra,
nobísque remíssio ómnium peccatórum,
per infiníta sǽcula sæculórum.
\Rbardot{} Amen.

\noindent \Vbardot{} Beáta víscera Maríæ Virginis, quæ portavérunt
ætérni Patris Fílium.\\
\Rbardot{} Et beáta úbera, quæ lactavérunt Christum Dominum.

\rubrica{Et dicitur secreto \textnormal{Pater noster.} et \textnormal{Ave María.}}

%\trOratioPostOfficium

\vfill

\hora{Ad I. Vesperas.} %%%%%%%%%%%%%%%%%%%%%%%%%%%%%%%%%%%%%%%%%%%%%%%%%%%%%
%\sideThumbs{I. Vesperæ}

\vspace{0.5cm}
\grechangedim{interwordspacetext}{0.18 cm plus 0.15 cm minus 0.05 cm}{scalable}%
\cuminitiali{}{temporalia/deusinadiutorium-solemnis.gtex}
\grechangedim{interwordspacetext}{0.22 cm plus 0.15 cm minus 0.05 cm}{scalable}%

\vfill
\pagebreak

\pars{Psalmus 1.} \scriptura{Ps. 109, 4; Gn 14, 18}

\vspace{-0.4cm}

\antiphona{I f}{temporalia/ant-sacerdosinaeternum.gtex}

\scriptura{Psalmus 109.}

\initiumpsalmi{temporalia/ps109-initium-i-f-auto.gtex}

%\psalmusEtTranslatioT{temporalia/ps109-comb.tex}{10cm}
\input{temporalia/ps109.tex} \Abardot{}

\vspace{-1cm}

\vfill
\pagebreak

\vesperasi

\ifx\festum\undefined
\vesperasiii
\else
\vesperasii
\fi

\capitulumLaudes

% preklad Jeruz. bible
%\trCapituliI

\vfill

\pars{Responsorium breve.} \scriptura{Ps. 30, 20; \textbf{H085}}

\cuminitiali{I}{temporalia/resp-quammagnamultitudo-cumdox.gtex}

%\trResp

\vfill
\pagebreak

\vesperasiv

\ifx\dominica\undefined
\pars{Canticum B. Mariæ V.} \scriptura{Cf. Sap. 16, 21,20}

\vspace{-3mm}

{
\grechangedim{interwordspacetext}{0.18 cm plus 0.15 cm minus 0.05 cm}{scalable}%
\antiphona{VI F}{temporalia/ant-oquamsuavisest.gtex}
\grechangedim{interwordspacetext}{0.22 cm plus 0.15 cm minus 0.05 cm}{scalable}%
}

%\trAntIMagnificat

\vspace{-2mm}

\scriptura{Lc. 1, 46-55}

\vspace{-2mm}

\cantusSineNeumas
\initiumpsalmi{temporalia/magnificat-initium-visoll-F.gtex}

%\psalmusEtTranslatioT{temporalia/magnificat-I-comb.tex}{10.2cm}
\input{temporalia/magnificat-I.tex} \Abardot{}
\else
\pars{Canticum B. Mariæ V.} \scriptura{Io. 15, 26; \textbf{H267}}

\vspace{-6mm}

{
\grechangedim{interwordspacetext}{0.18 cm plus 0.15 cm minus 0.05 cm}{scalable}%
\antiphona{VIII G}{temporalia/ant-XXXcumveneritparaclitus.gtex}
\grechangedim{interwordspacetext}{0.22 cm plus 0.15 cm minus 0.05 cm}{scalable}%
}

%\trAntIMagnificat

\vspace{-3mm}

\scriptura{Lc. 1, 46-55}

\vspace{-2mm}

\cantusSineNeumas
\initiumpsalmi{temporalia/magnificat-initium-viiisoll-G.gtex}

\vspace{-1.5mm}

%\psalmusEtTranslatioT{temporalia/magnificat-III-comb.tex}{10.2cm}
\input{temporalia/magnificat-III.tex} \Abardot{}
\fi

\vspace{-1cm}

\vfill
\pagebreak

%\sideThumbs{{\scriptsize{}Fine horarum}}

\anteOrationem

\pagebreak

% Oratio. %%%
\cuminitiali{}{temporalia/oratio.gtex}

\vspace{-1mm}
%\trOrationisI

\vfill

\rubrica{Hebdomadarius dicit iterum Dominus vobiscum, vel cantor dicit:}

\vspace{2mm}

\sineinitiali{temporalia/domineexaudi.gtex}

\rubrica{Postea cantatur a cantore:}

\vspace{2mm}

\ifx\festum\undefined
\cuminitiali{VII}{temporalia/benedicamus-tempore-paschali.gtex}
\else
\cuminitiali{II}{temporalia/benedicamus-solemnism-1vesp.gtex}
\fi

\vspace{1mm}

\vfill
\pagebreak
\fi

\ifx\festum\undefined
\else
\hora{Ad Completorium.} %%%%%%%%%%%%%%%%%%%%%%%%%%%%%%%%%%%%%%%%%%%%%%%%%%%%%%%%%%
%\sideThumbs{{\scriptsize{}Completorium}}

\rubrica{Lector petit benedictionem, dicens:}

\cuminitiali{}{temporalia/jubedomnebenedicere.gtex}

%\trJubeDomne

\vfill

\pars{Benedictio.}

\cuminitiali{}{temporalia/benedictio-noctemquietam.gtex}

%\trComplBenedictio

\vfill

\pars{Lectio brevis.} \scriptura{1Ptr. 5, 8-9}

\cuminitiali{}{temporalia/lectiobrevis-fratressobrii.gtex}

%\trComplLectioBr

\vfill

\noindent \Vbardot{} Adiutórium nostrum in nómine Dómini.

\noindent \Rbardot{} Qui fecit cælum, et terram.

\vfill
\pagebreak

\pars{Confessio.}

\noindent Confíteor Deo omnipoténti, beátæ Maríæ semper Vírgini, beáto
Michaéli Archángelo, beáto Ioánni Baptístæ, sanctis Apóstolis Petro
et Paulo, ómnibus Sanctis, et vobis fratres: quia peccávi nimis cogitatióne,
verbo et ópere: mea culpa, mea culpa, mea máxima culpa.
Ideo precor beátam Maríam semper Vírginem, beátum Michaélem
Archángelum, beátum Ioánnem Baptístam, sanctos Apóstolos Petrum
et Paulum, omnes Sanctos, et vos fratres, oráre pro me ad Dóminum
Deum nostrum.

\vfill

\noindent \Vbardot{} Misereátur nostri omnípotens Deus, et, dimíssis peccátis nostris, perdúcat
nos ad vitam ætérnam. \Rbardot{} Amen.

\vfill

\noindent \Vbardot{} Indulgéntiam, absolutiónem et remissiónem peccatórum nostrórum tríbuat nobis
omnípotens et miséricors Dóminus. \Rbardot{} Amen.

\vfill

\rubrica{Et facta absolutione dicitur:}

\sineinitiali{temporalia/convertenosdeus.gtex}

\vfill

\cuminitiali{}{temporalia/deusinadiutorium-communis.gtex}

\vfill
\pagebreak

\pars{Psalmus 1.}

\antiphona{VIII G}{temporalia/ant-miserere.gtex}

\scriptura{Ps. 4}

\initiumpsalmi{temporalia/ps4-initium-viii-G-auto.gtex}

%\psalmusEtTranslatioT{temporalia/ps4-comb.tex}{10cm}
\input{temporalia/ps4.tex}

\vfill
\pagebreak

\pars{Psalmus 2.} \scriptura{Ps. 90}

\initiumpsalmi{temporalia/ps90-initium-viii-G-auto.gtex}

%\psalmusEtTranslatioT{temporalia/ps90-comb.tex}{10cm}
\input{temporalia/ps90.tex}

\pagebreak

\pars{Psalmus 3.} \scriptura{Ps. 133}

\initiumpsalmi{temporalia/ps133-initium-viii-G-auto.gtex}

%\psalmusEtTranslatioT{temporalia/ps133-comb.tex}{10cm}
\input{temporalia/ps133.tex}

\vfill

\antiphona{}{temporalia/ant-miserere.gtex}

\vfill

\pars{Hymnus.}

\antiphona{II}{temporalia/hym-TeLucis.gtex}
%\input{hym-TeLucis-bohtext.tex}

\pagebreak

\pars{Capitulum.} \scriptura{Ier. 14, 9}

\cuminitiali{}{temporalia/capitulum-tuautem.gtex}

% preklad Jeruz. bible
%\trComplCapituli

\vfill

\pars{Responsorium breve.} \scriptura{Ps. 30, 6}

\cuminitiali{VI}{temporalia/resp-inmanus.gtex}

%\trRespCompl
\vfill

\pars{Versus.} \scriptura{Ps. 16, 8}

\sineinitiali{temporalia/versus-custodi.gtex}

%\noindent \trComplVersus

\vfill
\pagebreak

\cantusCumNeumis

\pars{Canticum Simeonis.}

\vspace{-3mm}

\antiphona{III a}{temporalia/ant-salvanos-antiquo.gtex}

%\trAntSalvaNos

%\vspace{-1mm}

\scriptura{Lc. 2, 29-32}

\vspace{-2mm}

\initiumpsalmi{temporalia/nuncdimittis-initium-iii-a-auto.gtex}

%\psalmusEtTranslatioT{temporalia/nuncdimittis-comb.tex}{10cm}
\input{temporalia/nuncdimittis.tex} \Abardot{}

\vfill

\rubrica{Ante Orationem, cantatur a Superiore:}

\vspace{3mm}

\pars{Supplicatio Litaniæ.}

\cuminitiali{}{temporalia/supplicatiolitaniae.gtex}

\vspace{7mm}

\pars{Oratio Dominica.}

\noindent Pater noster.

\vfill
\pagebreak

\sineinitiali{temporalia/domineexaudi-simplex.gtex}

\vspace{7mm}

\pars{Oratio.}

\cantusSineNeumas

\cuminitiali{}{temporalia/oratio-visita.gtex}

%\trComplOrationis

\vfill

%\sineinitiali{temporalia/domineexaudi-communis.gtex}

\noindent \Vbardot{} Dómine, exáudi oratiónem meam. \Rbardot{} Et clamor meus ad te véniat.

\vfill

%\vfill

\sineinitiali{temporalia/benedicamus-minor.gtex}

\vfill

\pars{Benedictio.}

\noindent Benedícat et custódiat nos omnípotens et miséricors Dóminus, \gredagger{}
Pater, et Fílius, et Spíritus Sanctus. \Rbardot{} Amen.

\vfill
\pagebreak

\pars{Antiphona finalis B. M. V.}

\vspace{-4mm}

\antiphona{I}{temporalia/an_salve_regina.gtex}

\rubrica{vel:}

\vspace{-4mm}

\antiphona{V}{temporalia/ant-salveregina-simplex.gtex}

\vfill
\pagebreak

\rubrica{vel:}

\antiphona{VII}{temporalia/ant-subtuum.gtex}

\vfill
\pagebreak
\fi

\hora{Ad Matutinum.} %%%%%%%%%%%%%%%%%%%%%%%%%%%%%%%%%%%%%%%%%%%%%%%%%%%%%
%\sideThumbs{Matutinum}

\vspace{2mm}

\cuminitiali{}{temporalia/dominelabiamea.gtex}

\vspace{2mm}

\pars{Invitatorium.} \scriptura{Cantor; Psalmus 94; \textbf{H261}}

\vspace{-6mm}

\antiphona{V}{temporalia/inv-alleluiachristumdominum.gtex}

\vfill
\pagebreak

\pars{Hymnus.}

\vspace{-5mm}

\scriptura{Anonymus X. sæculi; \textbf{AR488}}

\antiphona{IV}{temporalia/hym-AEterneRexAltissime.gtex}
%{
%\vspace{-5mm}
%\setlength{\columnsep}{0pt} % prostor mezi sloupci
%\input{hym-RexSempiterne-bohtext.tex}
%\setlength{\columnsep}{30pt} % prostor mezi sloupci
%}

\vfill
\pagebreak

\subhora{In I. Nocturno}

\pars{Psalmus 1.} \scriptura{Ps. 8, 2; \textbf{H262}}

%\vspace{-5mm}

\antiphona{IV A*}{temporalia/ant-elevataestmagnificentiatua.gtex}

%\vspace{-5mm}

\scriptura{Ps. 8}

%\vspace{-2mm}

\initiumpsalmi{temporalia/ps8-initium-iv-A_-auto.gtex}

%\psalmusEtTranslatioT{temporalia/ps8-comb.tex}{10cm}
\input{temporalia/ps8.tex} \Abardot{}

\vfill
\pagebreak

\pars{Psalmus 2.} \scriptura{Ps. 10, 5; \textbf{H262}}

%\vspace{-5mm}

\antiphona{VIII c}{temporalia/ant-dominusintemplosanctosuo.gtex}

%\vspace{-5mm}

\scriptura{Ps. 10}

\initiumpsalmi{temporalia/ps10-initium-viii-C-auto.gtex}

%\psalmusEtTranslatioT{temporalia/ps10-comb.tex}{10cm}
\input{temporalia/ps10.tex} \Abardot{}

\vfill
\pagebreak

\pars{Psalmus 3.} \scriptura{Ps. 18, 7; \textbf{H262}}

%\vspace{-5mm}

\antiphona{IV A*}{temporalia/ant-asummocoeloegressioejus.gtex}

%\vspace{-5mm}

\scriptura{Ps. 18}

\initiumpsalmi{temporalia/ps18-initium-iv-A_-auto.gtex}

%\psalmusEtTranslatioT{temporalia/ps18-comb.tex}{10cm}
\input{temporalia/ps18.tex}

\vfill

\antiphona{}{temporalia/ant-asummocoeloegressioejus.gtex}

\vfill
\pagebreak

\noindent \Vbardot{} Ascéndit Deus in iubilatióne, allelúia.
\noindent \Rbardot{} Et Dóminus in voce tubæ, allelúia.

\vspace{5mm}

\sineinitiali{temporalia/oratiodominica-mat.gtex}

\vspace{5mm}

\pars{Absolutio.}

\cuminitiali{}{temporalia/absolutio-exaudi.gtex}

\vfill
\pagebreak

\cuminitiali{}{temporalia/benedictio-solemn-benedictione.gtex}

\vspace{7mm}

\lectioi

\noindent \Vbardot{} Tu autem, Dómine, miserére nobis.
\noindent \Rbardot{} Deo grátias.

\vfill
\pagebreak

\pars{Responsorium 1.} \scriptura{\Rbardot{} Ac. 1, 3.9; \Vbardot{} ibid. 1, 4; \textbf{H262}}

\vspace{-5mm}

\responsorium{III transp.}{temporalia/resp-postpassionemsuam-sinedox.gtex}{}

\vfill
\pagebreak

\cuminitiali{}{temporalia/benedictio-solemn-unigenitus.gtex}

\vspace{7mm}

\lectioii

\noindent \Vbardot{} Tu autem, Dómine, miserére nobis.
\noindent \Rbardot{} Deo grátias.

\vfill
\pagebreak

\pars{Responsorium 2.} \scriptura{\Rbardot{} Cantor; \Vbardot{} Ps. 18, 7; \textbf{H262}}

\vspace{-5mm}

\responsorium{II}{temporalia/resp-omnispulchritudodomini-sinedox.gtex}{}

\vfill
\pagebreak

\cuminitiali{}{temporalia/benedictio-solemn-spiritus.gtex}

\vspace{7mm}

\lectioiii

\noindent \Vbardot{} Tu autem, Dómine, miserére nobis.
\noindent \Rbardot{} Deo grátias.

\vfill
\pagebreak

\pars{Responsorium 3.} \scriptura{\Rbardot{} Ps. 20, 14; \Vbardot{} Ps. 8, 2; \textbf{H262}}

\vspace{-5mm}

\responsorium{VII}{temporalia/resp-exaltaredomine-cumdox.gtex}{}

\vfill
\pagebreak

\subhora{In II. Nocturno}

\pars{Psalmus 4.} \scriptura{Ps. 20, 14; \textbf{H262}}

\vspace{-5mm}

\antiphona{IV A*}{temporalia/ant-exaltaredomine.gtex}

\vspace{-2mm}

\scriptura{Ps. 20}

\vspace{-1mm}

\initiumpsalmi{temporalia/ps20-initium-iv-A_-auto.gtex}

%\psalmusEtTranslatioT{temporalia/ps20-comb.tex}{10cm}
\input{temporalia/ps20.tex} \Abardot{}

\vfill
\pagebreak

\pars{Psalmus 5.} \scriptura{Ps. 29, 2; \textbf{H262}}

\vspace{-5.5mm}

\antiphona{VIII G}{temporalia/ant-exaltabotedomine.gtex}

\vspace{-3mm}

\scriptura{Ps. 29}

\vspace{-2mm}

\initiumpsalmi{temporalia/ps29-initium-viii-G-auto.gtex}

\vspace{-1.5mm}

%\psalmusEtTranslatioT{temporalia/ps29-comb.tex}{10cm}
\input{temporalia/ps29.tex} \Abardot{}

\vspace{-1cm}

\vfill
\pagebreak

\pars{Psalmus 6.} \scriptura{Ps. 46, 6; \textbf{H262}}

%\vspace{-5mm}

\antiphona{IV A*}{temporalia/ant-ascenditdeus.gtex}

%\vspace{-5mm}

\scriptura{Ps. 46}

\initiumpsalmi{temporalia/ps46-initium-iv-A_-auto.gtex}

%\psalmusEtTranslatioT{temporalia/ps46-comb.tex}{10cm}
\input{temporalia/ps46.tex} \Abardot{}

\vfill
\pagebreak

\noindent \Vbardot{} Ascéndens Christus in altum, allelúia.
\noindent \Rbardot{} Captívam duxit captivitátem, allelúia.

\vspace{5mm}

\sineinitiali{temporalia/oratiodominica-mat.gtex}

\vspace{5mm}

\pars{Absolutio.}

\cuminitiali{}{temporalia/absolutio-ipsius.gtex}

\vfill
\pagebreak

\cuminitiali{}{temporalia/benedictio-solemn-deus.gtex}

\vspace{7mm}

\lectioiv

\noindent \Vbardot{} Tu autem, Dómine, miserére nobis.
\noindent \Rbardot{} Deo grátias.

\vfill
\pagebreak

\pars{Responsorium 4.} \scriptura{\Rbardot{} Tob. 12, 20 \& Io. 14, 27; \Vbardot{} Io. 16, 7; \textbf{H263}}

\vspace{-5mm}

\responsorium{IV}{temporalia/resp-tempusest-sinedox.gtex}{}

\vfill
\pagebreak

\cuminitiali{}{temporalia/benedictio-solemn-christus.gtex}

\vspace{7mm}

\lectiov

\noindent \Vbardot{} Tu autem, Dómine, miserére nobis.
\noindent \Rbardot{} Deo grátias.

\vfill
\pagebreak

\pars{Responsorium 5.} \scriptura{\Rbardot{} Cantor super Ioannem; \Vbardot{} Io. 14, 16; \textbf{H263}}

\vspace{-5mm}

\responsorium{III}{temporalia/resp-nonconturbetur-sinedox.gtex}{}

\vfill
\pagebreak

\cuminitiali{}{temporalia/benedictio-solemn-ignem.gtex}

\vspace{7mm}

\lectiovi

\noindent \Vbardot{} Tu autem, Dómine, miserére nobis.
\noindent \Rbardot{} Deo grátias.

\vfill
\pagebreak

\pars{Responsorium 6.} \scriptura{\Rbardot{} Eph. 4, 8; \Vbardot{} Ps. 46, 6; \textbf{H263}}

\vspace{-5mm}

\responsorium{IV}{temporalia/resp-ascendensinaltum-cumdox.gtex}{}

\vfill
\pagebreak

\subhora{In III. Nocturno}

\pars{Psalmus 7.} \scriptura{Ps. 96, 9; \textbf{H263}}

\vspace{-5mm}

\antiphona{VI F}{temporalia/ant-nimisexaltatusest.gtex}

\vspace{-4mm}

\scriptura{Ps. 96}

%\vspace{-2mm}

\initiumpsalmi{temporalia/ps96-initium-vi-F-auto.gtex}

%\psalmusEtTranslatioT{temporalia/ps96-comb.tex}{10cm}
\input{temporalia/ps96.tex} \Abardot{}

\vfill
\pagebreak

\pars{Psalmus 8.} \scriptura{Ps. 98, 2; \textbf{H263}}

\vspace{-5mm}

\antiphona{VI F}{temporalia/ant-dominusinsion.gtex}

\vspace{-4mm}

\scriptura{Ps. 98}

\initiumpsalmi{temporalia/ps98-initium-vi-F-auto.gtex}

%\psalmusEtTranslatioT{temporalia/ps98-comb.tex}{10cm}
\input{temporalia/ps98.tex} \Abardot{}

\vfill
\pagebreak

\pars{Psalmus 9.} \scriptura{Ps. 102, 19; \textbf{H263}}

\vspace{-5mm}

\antiphona{VI F}{temporalia/ant-dominusincoelo.gtex}

\vspace{-4mm}

\scriptura{Ps. 102}

\initiumpsalmi{temporalia/ps102-initium-vi-F-auto.gtex}

%\psalmusEtTranslatioT{temporalia/ps102-comb.tex}{10cm}
\input{temporalia/ps102.tex}

\vfill

\antiphona{}{temporalia/ant-dominusincoelo.gtex}

\vfill
\pagebreak

\noindent \Vbardot{} Ascéndo ad Patrem meum, et Patrem vestrum, allelúia.
\noindent \Rbardot{} Deum meum, et Deum vestrum, allelúia.

\vspace{5mm}

\sineinitiali{temporalia/oratiodominica-mat.gtex}

\vspace{5mm}

\pars{Absolutio.}

\cuminitiali{}{temporalia/absolutio-avinculis.gtex}

\vfill
\pagebreak

\cuminitiali{}{temporalia/benedictio-solemn-evangelica.gtex}

\vspace{7mm}

\lectiovii

\noindent \Vbardot{} Tu autem, Dómine, miserére nobis.
\noindent \Rbardot{} Deo grátias.

\vfill
\pagebreak

\pars{Responsorium 7.} \scriptura{\Rbardot{} Io. 14, 16.17; \Vbardot{} ibid. 16, 7; \textbf{Sar.275}}

\vspace{-5mm}

\responsorium{III}{temporalia/resp-egorogabopatrem-sinedox.gtex}{}

\vfill
\pagebreak

\cuminitiali{}{temporalia/benedictio-solemn-divinum.gtex}

\vspace{7mm}

\lectioviii

\noindent \Vbardot{} Tu autem, Dómine, miserére nobis.
\noindent \Rbardot{} Deo grátias.

\vfill
\pagebreak

\ifx\dominica\undefined
\pars{Responsorium 8.} \scriptura{\Rbardot{} Ps. 103, 3; \Vbardot{} Ps. 103, 1.2; \textbf{H264}}

\vspace{-5mm}

\responsorium{II}{temporalia/resp-ponitnubem-cumdox.gtex}{}
\else
\pars{Responsorium 8.} \scriptura{\Rbardot{} Io. 16, 7; \Vbardot{} ibid. 16, 13; \textbf{Sar.272}}

\vspace{-5mm}

\responsorium{III}{temporalia/resp-sienimnonabiero-cumdox.gtex}{}
\fi

\vfill
\pagebreak

\cuminitiali{}{temporalia/benedictio-solemn-adsocietatem.gtex}

\vspace{7mm}

\lectioix

\noindent \Vbardot{} Tu autem, Dómine, miserére nobis.
\noindent \Rbardot{} Deo grátias.

\vfill
\pagebreak

% Te Deum

%\pars{Hymnus Ambrosianus}

\vspace{-5mm}

\cuminitiali{III}{temporalia/tedeum-solemnis.gtex}

\vfill
\pagebreak

\rubrica{Reliqua omittuntur, nisi Laudes separandæ sint.}

\pars{Oratio}

\noindent \Vbardot{} Dómine, exáudi oratiónem meam.

\noindent \Rbardot{} Et clamor meus ad te véniat.

Orémus:

\ifx\dominica\undefined
\noindent Concéde, quǽsumus, omnípotens Deus: \gredagger{} ut, qui hodiérna die Unigénitum tuum, Redemptórem nostrum, ad cælos ascendísse crédimus; \grestar{} ipsi quoque mente in cæléstibus habitémus. Per eúmdem Dóminum.
\else
\noindent Omnípotens sempitérne Deus: \gredagger{} fac nos tibi semper et devótam gérere voluntátem; \grestar{} et majestáti tuæ sincéro corde servíre. Per Dóminum.
\fi

\noindent \Rbardot{} Amen.

\vspace{7mm}

\pars{Conclusio}

\noindent \Vbardot{} Dómine, exáudi oratiónem meam.

\noindent \Rbardot{} Et clamor meus ad te véniat.

\noindent \Vbardot{} Benedicámus Dómino, allelúia, allelúia.

\noindent \Rbardot{} Deo grátias, allelúia, allelúia.

\noindent \Vbardot{} Fidélium ánimæ per misericórdiam Dei requiéscant in pace.

\noindent \Rbardot{} Amen.

\vfill
\pagebreak

\hora{Ad Laudes.} %%%%%%%%%%%%%%%%%%%%%%%%%%%%%%%%%%%%%%%%%%%%%%%%%%%%%
%\sideThumbs{Laudes}

\cantusSineNeumas

\ifx\postoctavam\undefined
\vspace{0.5cm}
\grechangedim{interwordspacetext}{0.18 cm plus 0.15 cm minus 0.05 cm}{scalable}%
\ifx\festumveldominica\undefined
\cuminitiali{}{temporalia/deusinadiutorium-communis.gtex}
\else
\cuminitiali{}{temporalia/deusinadiutorium-alter.gtex}
\fi
\grechangedim{interwordspacetext}{0.22 cm plus 0.15 cm minus 0.05 cm}{scalable}%

\vfill
%\pagebreak
\else
\rubrica{Absolute incipitur Officium ab Antiphona primi Psalmi.}

\vspace{7mm}
\fi

\pars{Psalmus 1.} \scriptura{Ac. 1, 11; \textbf{H265}}

\vspace{-0.4cm}

\antiphona{VII a}{temporalia/ant-virigalilaeiquidaspicitis.gtex}

\scriptura{Psalmus 92.}

\initiumpsalmi{temporalia/ps92-initium-vii-a-auto.gtex}

\ifx\postoctavam\undefined
%\psalmusEtTranslatioT{temporalia/ps92-comb.tex}{10cm}
\input{temporalia/ps92.tex}

\vfill

\vspace{-1cm}

\antiphona{}{temporalia/ant-virigalilaeiquidaspicitis.gtex}
\else
%\psalmusEtTranslatioT{temporalia/ps92-comb.tex}{10cm}
\input{temporalia/ps92.tex} \Abardot{}
\fi

\vfill
\pagebreak

\pars{Psalmus 2.} \scriptura{Ac. 1, 10; \textbf{H265}}

\vspace{-0.4cm}

\antiphona{VIII G\textsuperscript{2}}{temporalia/ant-cumqueintuerentur.gtex}

\scriptura{Psalmus 99.}

\initiumpsalmi{temporalia/ps99-initium-viii-G2-auto.gtex}

%\psalmusEtTranslatioT{temporalia/ps99-comb.tex}{10cm}
\input{temporalia/ps99.tex} \Abardot{}

\vfill
\pagebreak

\pars{Psalmus 3.} \scriptura{Lc. 24, 50.51; \textbf{H265}}

\vspace{-0.4cm}

\antiphona{IV A*}{temporalia/ant-elevatismanibus.gtex}

\scriptura{Psalmus 62.}

\initiumpsalmi{temporalia/ps62-initium-iv-A_-auto.gtex}

%\psalmusEtTranslatioT{temporalia/ps62-comb.tex}{10cm}
\input{temporalia/ps62.tex} \Abardot{}

%\vfill

%\vspace{-6mm}

%\antiphona{}{temporalia/ant-elevatismanibus.gtex} % repeat the antiphon - new page

\vfill
\pagebreak

\pars{Psalmus 4.} \scriptura{\textbf{H265}}

\vspace{-0.4cm}

\antiphona{VIII G\textsuperscript{2}}{temporalia/ant-exaltateregemregum.gtex}

\scriptura{Canticum trium puerorum, Dan. 3, 57-88 et 56}

\initiumpsalmi{temporalia/dan3-initium-viii-G2-auto.gtex}

%\psalmusEtTranslatioT{temporalia/dan3-comb.tex}{10cm}
\input{temporalia/dan3.tex}

\rubrica{Hic non dicitur Gloria Patri, neque Amen.}

\vfill

\vspace{-6mm}

\antiphona{}{temporalia/ant-exaltateregemregum.gtex} % repeat the antiphon - new page

\vfill
\pagebreak

\pars{Psalmus 5.} \scriptura{Ac. 1, 9; \textbf{H265}}

\vspace{-0.4cm}

\antiphona{VIII G}{temporalia/ant-videntibusillis.gtex}

\scriptura{Psalmus 148.}

\initiumpsalmi{temporalia/ps148-initium-viii-G-auto.gtex}

%\psalmusEtTranslatioT{temporalia/ps148-comb.tex}{10cm}
\input{temporalia/ps148.tex}

\rubrica{Hic non dicitur Gloria Patri.}

\vfill
\pagebreak

%
\scriptura{Psalmus 149.}

\initiumpsalmi{temporalia/ps149-initium-viii-G-auto.gtex}

%\psalmusEtTranslatioT{temporalia/ps149-comb.tex}{10cm}
\input{temporalia/ps149.tex}

\rubrica{Hic non dicitur Gloria Patri.}

\vfill
\pagebreak

%
\scriptura{Psalmus 150.}

\initiumpsalmi{temporalia/ps150-initium-viii-G-auto.gtex}

%\psalmusEtTranslatioT{temporalia/ps150-comb.tex}{10cm}
\input{temporalia/ps150.tex}

\vfill

\vspace{-6mm}

\antiphona{}{temporalia/ant-videntibusillis.gtex} % repeat the antiphon - new page

\vfill
\pagebreak

\capitulumLaudes

% preklad Jeruz. bible
%\trCapituliI

\vfill

\pars{Responsorium breve.} \scriptura{Ps. 46, 6}

\cuminitiali{VI}{temporalia/resp-ascenditdeus.gtex}

%\trResp

\vfill
\pagebreak

\pars{Hymnus}

\cuminitiali{VIII}{temporalia/hym-JesuNostraRedemptio.gtex}
\vspace{-3mm}
%\begin{translatioMulticol}{3}
Výkupné naše, Ježíši,\\
lásko a tužbo nejčistší,\\
tys Tvůrce věcí stvořených\\
a člověk věků posledních.\\
\\
Jaký tě musil soucit vést,\\
žes naše hříchy za své vzal,\\
že chtěl jsi muky smrti nést,\\
bys kletbu smrti z lidí sňal.\columnbreak

Pronikáš v žalář pekelný,\\
propouštíš z něho zajatce.\\
Vítězi, slávou oděný,\\
po boku trůníš u Otce.\\
\\
Kéž donutí té soucit týž,\\
že rány vin v nás zacelíš,\\
nás podle slibu ušetříš\\
a vlídnou tváří potěšíš.\columnbreak

Ty budiž naší radostí,\\
odměnou ve tvé věčnosti,\\
kéž naše sláva veškerá\\
jen z tebe věčně vyvěrá.\\
Amen.
\end{translatioMulticol}


\vfill
%\pagebreak

\pars{Versus.}

% Versus. %%%
\ifx\festum\undefined
\sineinitiali{temporalia/versus-dominusincaelo-communis.gtex}
\else
\sineinitiali{temporalia/versus-dominusincaelo.gtex}
\fi

%\noindent \trVersus

\vfill
\pagebreak

\ifx\dominica\undefined
\pars{Canticum Zachariæ.} \scriptura{Io. 20, 17; \textbf{H265}}

%\vspace{-6mm}

{
\grechangedim{interwordspacetext}{0.18 cm plus 0.15 cm minus 0.05 cm}{scalable}%
\antiphona{VII a}{temporalia/ant-ascendoadpatrem.gtex}
\grechangedim{interwordspacetext}{0.22 cm plus 0.15 cm minus 0.05 cm}{scalable}%
}

%\trAntIMagnificat

%\vspace{-3mm}

\scriptura{Lc. 1, 68-79}

%\vspace{-2.5mm}

\cantusSineNeumas
\initiumpsalmi{temporalia/benedictus-initium-viisoll-a-auto.gtex}

%\vspace{-1.5mm}

%\psalmusEtTranslatioT{temporalia/benedictus-I-comb.tex}{10.2cm}
\input{temporalia/benedictus-I.tex} \Abardot{}
\else
\pars{Canticum Zachariæ.} \scriptura{Io. 15, 26; \textbf{H267}}

\vspace{-3mm}

{
\grechangedim{interwordspacetext}{0.18 cm plus 0.15 cm minus 0.05 cm}{scalable}%
\antiphona{VIII G}{temporalia/ant-cumveneritparaclitus.gtex}
\grechangedim{interwordspacetext}{0.22 cm plus 0.15 cm minus 0.05 cm}{scalable}%
}

%\trAntIMagnificat

%\vspace{-3mm}

\scriptura{Lc. 1, 68-79}

%\vspace{-2.5mm}

\cantusSineNeumas
\initiumpsalmi{temporalia/benedictus-initium-viiisoll-G-auto.gtex}

%\vspace{-1.5mm}

%\psalmusEtTranslatioT{temporalia/benedictus-II-comb.tex}{10.2cm}
\input{temporalia/benedictus-II.tex}

\vfill

\antiphona{}{temporalia/ant-cumveneritparaclitus.gtex}
\fi

\vspace{-1cm}

\vfill
\pagebreak

%\sideThumbs{{\scriptsize{}Fine horarum}}

\anteOrationem

\pagebreak

% Oratio. %%%
\ifx\dominica\undefined
\cuminitiali{}{temporalia/oratio.gtex}
\else
\cuminitiali{}{temporalia/oratio2.gtex}
\fi

\vspace{-1mm}
%\trOrationisI

\vfill

\rubrica{Hebdomadarius dicit iterum Dominus vobiscum, vel cantor dicit:}

\vspace{2mm}

\sineinitiali{temporalia/domineexaudi.gtex}

\rubrica{Postea cantatur a cantore:}

\vspace{2mm}

\ifx\festum\undefined
\cuminitiali{VII}{temporalia/benedicamus-tempore-paschali.gtex}
\else
\cuminitiali{II}{temporalia/benedicamus-solemnism-laud.gtex}
\fi

\vspace{1mm}

\vfill
\pagebreak

\ifx\sabbato\undefined
\ifx\festumveldominica\undefined
\hora{Ad Vesperas.} %%%%%%%%%%%%%%%%%%%%%%%%%%%%%%%%%%%%%%%%%%%%%%%%%%%%%
%\sideThumbs{Vesperæ}
\else
\hora{Ad II. Vesperas.} %%%%%%%%%%%%%%%%%%%%%%%%%%%%%%%%%%%%%%%%%%%%%%%%%%%%%
%\sideThumbs{II. Vesperæ}
\fi

\cantusSineNeumas

%\vspace{-2mm}
\grechangedim{interwordspacetext}{0.18 cm plus 0.15 cm minus 0.05 cm}{scalable}%
\cuminitiali{}{temporalia/deusinadiutorium-solemnis.gtex}
\grechangedim{interwordspacetext}{0.22 cm plus 0.15 cm minus 0.05 cm}{scalable}%

\vfill
%\pagebreak

%\vspace{-2mm}

\pars{Psalmus 1.} \scriptura{Ps. 109, 4; Gn 14, 18}

\vspace{-0.4cm}

\antiphona{I f}{temporalia/ant-sacerdosinaeternum.gtex}

\scriptura{Psalmus 109.}

\initiumpsalmi{temporalia/ps109-initium-i-f-auto.gtex}

%\psalmusEtTranslatioT{temporalia/ps109-comb.tex}{10cm}
\input{temporalia/ps109.tex} \Abardot{}

%\vfill

%\antiphona{}{temporalia/ant-sacerdosinaeternum.gtex}

%\vspace{-1cm}

\vfill
\pagebreak

\vesperasi

\ifx\festum\undefined
\vesperasii
\else
\vesperasiii
\fi

\capitulumLaudes

% preklad Jeruz. bible
%\trCapituliI

\vfill

\pars{Responsorium breve.} \scriptura{Ps. 80, 17}

\ifx\festum\undefined
\cuminitiali{VI}{temporalia/resp-cibavit-simplex.gtex}
\else
\cuminitiali{VI}{temporalia/resp-cibavit.gtex}
\fi

%\trResp

\vfill
\pagebreak

\vesperasiv

\ifx\dominica\undefined
\pars{Canticum B. Mariæ V.} \scriptura{Thomas de Aquino?}

%\vspace{-5.5mm}

{
\grechangedim{interwordspacetext}{0.18 cm plus 0.15 cm minus 0.05 cm}{scalable}%
\antiphona{V a}{temporalia/ant-osacrumconvivium.gtex}
\grechangedim{interwordspacetext}{0.22 cm plus 0.15 cm minus 0.05 cm}{scalable}%
}

%\trAntIMagnificat

\vspace{-3mm}

\scriptura{Lc. 1, 46-55}

\vspace{-2.5mm}

\cantusSineNeumas
\initiumpsalmi{temporalia/magnificat-initium-vsoll-a_.gtex}

\vspace{-1.5mm}

%\psalmusEtTranslatioT{temporalia/magnificat-II-comb.tex}{10.2cm}
\input{temporalia/magnificat-II.tex} \Abardot{}
\else
\pars{Canticum B. Mariæ V.} \scriptura{Io. 16, 4}

{
\grechangedim{interwordspacetext}{0.18 cm plus 0.15 cm minus 0.05 cm}{scalable}%
\antiphona{VIII G}{temporalia/ant-xxx.gtex}
\grechangedim{interwordspacetext}{0.22 cm plus 0.15 cm minus 0.05 cm}{scalable}%
}

%\trAntIMagnificat

\scriptura{Lc. 1, 46-55}

\cantusSineNeumas
\initiumpsalmi{temporalia/magnificat-initium-viiisoll-G.gtex}

%\psalmusEtTranslatioT{temporalia/magnificat-III-comb.tex}{10.2cm}
\input{temporalia/magnificat-III.tex} \Abardot{}
\fi

\vspace{-1cm}

\vfill
\pagebreak

%\sideThumbs{{\scriptsize{}Fine horarum}}

\anteOrationem

\pagebreak

% Oratio. %%%
\ifx\dominica\undefined
\cuminitiali{}{temporalia/oratio.gtex}
\else
\cuminitiali{}{temporalia/oratio2.gtex}
\fi

\vspace{-1mm}
%\trOrationisI

\vfill

\rubrica{Hebdomadarius dicit iterum Dominus vobiscum, vel cantor dicit:}

\vspace{2mm}

\sineinitiali{temporalia/domineexaudi.gtex}

\rubrica{Postea cantatur a cantore:}

\vspace{2mm}

\ifx\festum\undefined
\cuminitiali{VII}{temporalia/benedicamus-tempore-paschali.gtex}
\else
\cuminitiali{II}{temporalia/benedicamus-solemnism-2vesp.gtex}
\fi

\vspace{1mm}
\fi

\end{document}

