\documentclass[options]{article}
\begin{document}
	\textbf{(Tract. 31, 3-4 : CCL 36, 294)}
	Ex Tractátibus sancti Augustíni epíscopi in Ioánnem 
	Audíte verbum Dómini, fratres, vidéte quemádmodum confirmávit Iud\'{æ}is et quod dixérunt : 
	\textit{Istum nóvimus unde sit,}
 et quod dixérunt:
 \textit{Christus cum vénerit, nemo scit unde sit.}
  Clamábat ergo docens in templo Iesus: 
  \textit{Et me scitis et unde sim scitis, et a meípso non veni, sed est verus qui me misit, quem vos nescítis.}
  Hoc est dícere: "Et me scitis, et me nescítis"; hoc est dícere: " Et unde sim scitis, et unde sim nescítis. Unde sim scitis, Iesum a Názareth, cuius étiam paréntes nostis." Solus, enim in hac causa latébat Vírginis partus, cui tamen testis erat marítus; ipse enim hoc póterat fidéliter indicáre, qui possit maritáliter et zeláre.
	

Hoc ergo excépto Vírginis partu, totum nóverant in Iesu quod ad hóminem pértinet; fácies ipsíus nota erat, pátria ipsíus nota erat, genus ipsíus notum erat, ubi natus est sciebátur. Recte ergo dixit:
\textit{Et me nostis et unde sim scitis,}
secúndum carnem et effígiem hóminis quam gerébat; secúndum divinitátem autem: 
 \textit{Et a meípso non veni, sed est verus qui me misit, quem vos nescítis.}
 Sed ut eum sciátis, crédite in eum quem misit, et sciétis. 
 \textit{Deum enim nemo vidit umquam, nisi unigénitus Fílius qui est in sinu Patris, ipse enarrávit, et Patrem non cognóscit, nisi Fílius et cui volúerit Fílius reveláre.}
 
 Dénique cum dixísset : 
 \textit{Sed est verus qui misit me, quem vos nescítis,}
 ut osténderet eis unde possent scire quod nesciébant, subiécit : 
\textit{Ego scio eum.} 
Ergo a me qu\'{æ}rite, ut sciátis eum. Quare autem scio eum? 
\textit{Quia ab ipso sum, et ipse me misit.}
Magnífice utrúmque monstrávit. 
\textit{Ab ipso,}
inquit,
\textit{sum} ; quia Fílius de Patre, et quidquid est Fílius, de illo est cuius est Fílius.
	
Ideo Dóminum Iesum dícimus Deum de Deo ; Patrem non dícimus Deum de Deo, sed tantum Deum; et dícimus Dóminum Iesum Lumen de lúmine ; Patrem non dícimus Lumen de Lúmine, sed tantum Lumen. Ad hoc ergo pértinet quod dixit : 
\textit{Ab ipso sum.}
Quod autem vidétis me in carne, 
\textit{ipse me misit.}
Ubi audis : Ipse me misit, noli intellégere natúræ dissimilitúdinem, sed generántis auctoritátem.
	
	
\end{document}
