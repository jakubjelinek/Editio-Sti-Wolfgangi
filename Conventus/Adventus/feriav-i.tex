\newcommand{\titulus}{\nomenFesti{S. Ambrosii, Episcopi \& Ecclesiæ Doctoris.}
\dies{Die 7. Decembris.}}
\newcommand{\oratio}{\pars{Oratio.}

\noindent Deus, qui beátum Ambrósium, epíscopum, cathólicæ fídei doctórem et apostólicæ fortitúdinis exémplum effecísti, éxcita in Ecclésia tua viros secúndum cor tuum, qui eam fórtiter et sapiénter gubérnent.

\pars{Pro pace in universo mundo.} \scriptura{Sir. 50, 25; 2 Esdr. 4, 20; \textbf{H416}}

\vspace{-4mm}

\antiphona{II D}{temporalia/ant-dapacemdomine.gtex}

\vfill

\noindent Deus, a quo sancta desidéria, recta consília et iusta sunt ópera: da servis tuis illam, quam mundus dare non potest, pacem; ut et corda nostra mandátis tuis dédita, et hóstium subláta formídine, témpora sint tua protectióne tranquílla.

\noindent Per Dóminum nostrum Iesum Christum, Fílium tuum, qui tecum vivit et regnat in unitáte Spíritus Sancti, Deus, per ómnia sǽcula sæculórum.

\noindent \Rbardot{} Amen.}
\newcommand{\invitatorium}{\pars{Invitatorium.}

\vspace{-4mm}

\antiphona{IV}{temporalia/inv-fontemsapientiae.gtex}}
\newcommand{\hymnusmatutinum}{\pars{Hymnus}

\cuminitiali{III}{temporalia/hym-AmbrosiDoctor.gtex}}
\newcommand{\matversus}{\noindent \Vbardot{} Dómine Deus noster, convérte nos.

\noindent \Rbardot{} Et osténde fáciem tuam et salvi érimus.}
\newcommand{\lectioi}{\pars{Lectio I.} \scriptura{Is. 16, 1-5; 17, 4-8}

\noindent De libro Isaíæ prophétæ.

\noindent Emíttite agnum dominatóri terræ, de Petra desérti ad montem fíliæ Sion. Et erit: sicut avis fúgiens et pulli de nido avolántes, sic erunt fíliæ Moab ad vada Arnon.

\noindent Affer consílium fac iudícium; pone quasi noctem umbram tuam in merídie, abscónde fugiéntes et vagos ne prodas.

\noindent Hábitent apud te prófugi Moab; esto latíbulum eórum a fácie vastatóris; finítus est enim exáctor, consummáta est devastátio, defécit calcátor a terra.

\noindent Et firmábitur in misericórdia sólium; et sedébit super illud in veritáte, in tabernáculo David, iúdicans et quærens iudícium et velóciter reddens, quod iustum est.

\noindent Et erit in die illa: attenuábitur glória Iacob et pinguédo carnis eius marcéscet; et erit, sicut cum messor árripit culmos, et bráchium eius spicas legit; et erit, sicut cum quis quærit spicas in valle Ráphaim.

\noindent Et relinquétur in eo racémus, et sicut cum excútitur ólea: duæ vel tres olívæ in summitáte rami sive quáttuor aut quinque in cacumínibus árboris fructíferæ, dicit Dóminus Deus Israel.

\noindent In die illa atténdet homo ad factórem suum, et óculi eius ad Sanctum Israel respícient; et non atténdet ad altária, quæ fecérunt manus eius, et quæ operáti sunt dígiti eius; non respíciet lucos et thymiatéria.}
\newcommand{\responsoriumi}{\pars{Responsorium 1.} \scriptura{\Rbardot{} Ex. 4, 13 \Vbardot{} Ps. 79, 2; \textbf{H417}}

\vspace{-5mm}

\responsorium{V}{temporalia/resp-obsecrodomine-CROCHU.gtex}{}}
\newcommand{\lectioii}{\pars{Lectio II.} \scriptura{Epist. 2, 1-2. 4-5. 7: PL 16 [edit. 1845], 847-881}

\noindent Ex Epístolis sancti Ambrósii epíscopi.

\noindent Suscepísti munus sacerdótii et, in puppe Ecclésiæ sedens, navim advérsus fluctus gubérnas. Tene clavum fídei, ut et graves huius sǽculi turbáre non possint procéllæ. Mare quidem magnum et spatiósum, sed noli veréri; quia \emph{ipse super mária fundávit eam, et super flúmina præparávit eam.}

\noindent Itaque non immérito inter tot mundi freta Ecclésia Dómini tamquam supra apostólicam ædificáta petram immóbilis manet, et inconcússo advérsum ímpetus sæviéntis salis persevérat fundámine. Ablúitur undis, non quátitur, et licet frequénter eleménta mundi huius magno illísa resúltent fragóre, habet tamen, quo laborántes excípiat, tutíssimum portum salútis. Sed tamen, etsi in mari flúctuat, currit in flumínibus; et vide ne in illis flumínibus magis, de quibus dictum est: \emph{Elevavérunt flúmina vocem suam.} Sunt enim flúmina, quæ de ventre eius fluent, qui potum a Christo accéperit, et de Spíritu Dei súmpserit. Hæc ergo flúmina, cum redúndant spiritáli grátia, \emph{élevant vocem suam.}

\noindent Est et flúvius, qui decúrrit in sanctos suos sicut torrens. Est et flúminis ímpetus, qui lætíficat ánimam pacíficam atque tranquíllam. Ex huius flúminis plenitúdine quicúmque accéperit, sicut Ioánnes evangelísta, sicut Petrus et Paulus, élevat vocem suam: et sicut Apóstoli evangélicæ prædicatiónis vocem usque in totos fines orbis terrárum canóro diffudérunt præcónio, ita et iste íncipit evangelizáre Dóminum Iesum.}
\newcommand{\responsoriumii}{\pars{Responsorium 2.} \scriptura{\Rbar{} Cf. Iac. 1, 12 \Vbar{} Ps. 131, 18; \textbf{H63}}

\vspace{-5mm}

\responsorium{VII}{temporalia/resp-famulomeo-CROCHU.gtex}{}}
\newcommand{\lectioiii}{\pars{Lectio III.}

\noindent Accipe ergo a Christo, ut et tuus sonus éxeat. Cóllige aquam Christi, illam quæ laudat Dóminum. Cóllige aquam de plúribus locis, quam effúndunt nubes prophéticæ.

\noindent Quicúmque cólligit de móntibus aquam, atque ad se trahit vel haurit e fóntibus, et ipse rorat sicut nubes. Imple ergo grémium mentis tuæ, ut terra tua huméscat, et domésticis irrigétur fóntibus.

\noindent Ergo qui multa legit et intéllegit, implétur: qui fúerit implétus, álios rigat; ideóque Scriptúra dicit: \emph{Si implétæ fúerint nubes, plúviam in terram effúndent.}

\noindent Sint ergo sermónes tui próflui, sint puri et dilúcidi, ut moráli disputatióne suavitátem infúndas populórum áuribus et grátia verbórum tuórum plebem demúlceas, ut volens, quo ducis, sequátur.

\noindent Allóquia tua plena intelléctus sint. Unde et Sálomon ait: \emph{Arma intelléctus lábia sapiéntis;} et álibi: \emph{Lábia tua alligáta sint sensu,} id est, fúlgeat sermónum tuórum manifestátio, intelléctus corúscet, et allóquium tuum atque tractátus aliéna non indígeat assertióne; sed sermo tuus velut armis suis sese ipse tueátur, nec ullum verbum tuum in vanum éxeat, et sine sensu pródeat.}
\newcommand{\responsoriumiii}{\pars{Responsorium 3.} \scriptura{\textbf{H378}}

\vspace{-5mm}

\responsorium{VIII}{temporalia/resp-isteestquiantedeum-CROCHU-cumdox.gtex}{}}
\newcommand{\hymnuslaudes}{\pars{Hymnus}

\cuminitiali{III}{temporalia/hym-FortemPiumque.gtex}}
\newcommand{\lectiobrevis}{\pars{Lectio Brevis.} \scriptura{Sap. 7, 13-14}

\noindent Sapiéntiam sine fictióne dídici et sine invídia commúnico; divítias illíus non abscóndo. Infinítus enim thesáurus est homínibus; quem qui acquisiérunt, ad amicítiam in Deum se paravérunt propter disciplínæ dona commendáti.}
\newcommand{\responsoriumbreve}{\pars{Responsorium breve.} \scriptura{Eccli. 44, 15}

\cuminitiali{VI}{temporalia/resp-sapientiamsanctorum.gtex}}
\newcommand{\benedictus}{\pars{Canticum Zachariæ.}

\vspace{-4mm}

{
\grechangedim{interwordspacetext}{0.18 cm plus 0.15 cm minus 0.05 cm}{scalable}%
\antiphona{II D}{temporalia/ant-odoctoroptime.gtex}
\grechangedim{interwordspacetext}{0.22 cm plus 0.15 cm minus 0.05 cm}{scalable}%
}

\vspace{-2mm}

\scriptura{Lc. 1, 68-79}

%\vspace{-2mm}

\cantusSineNeumas
\initiumpsalmi{temporalia/benedictus-initium-ii-D-auto.gtex}

%\vspace{-1.5mm}

\input{temporalia/benedictus-ii-D.tex} \Abardot{}}
\newcommand{\preces}{\noindent Christo, bono pastóri, qui pro suis óvibus ánimam pósuit,~\gredagger{} laudes grati exsolvámus et supplicémus, dicéntes:\grestar{}

\Rbardot{} Pasce pópulum tuum, Dómine.

\noindent Christe, qui in sanctis pastóribus misericórdiam et dilectiónem tuam dignátus es osténdere,~\grestar{} numquam désinas per eos nobíscum misericórditer ágere.

\Rbardot{} Pasce pópulum tuum, Dómine.

\noindent Qui múnere pastóris animárum fungi per tuos vicários pergis,~\grestar{} ne destíteris nos ipse per rectóres nostros dirígere.

\Rbardot{} Pasce pópulum tuum, Dómine.

\noindent Qui in sanctis tuis, populórum dúcibus,~\gredagger{} córporum animarúmque médicus exstitísti,~\grestar{} numquam cesses ministérium in nos vitæ et sanctitátis perágere.

\Rbardot{} Pasce pópulum tuum, Dómine.

\noindent Qui, prudéntia et caritáte sanctórum, tuum gregem erudísti,~\grestar{} nos in sanctitáte iúgiter per pastóres nostros ædífica.

\Rbardot{} Pasce pópulum tuum, Dómine.}
\newcommand{\magnificat}{\pars{Canticum B. Mariæ V.} \scriptura{Lc. 1, 28.42; \textbf{H21}}

\vspace{-4mm}

{
\grechangedim{interwordspacetext}{0.18 cm plus 0.15 cm minus 0.05 cm}{scalable}%
\antiphona{II* B}{temporalia/ant-benedictatu.gtex}
\grechangedim{interwordspacetext}{0.22 cm plus 0.15 cm minus 0.05 cm}{scalable}%
}

%\vspace{-3mm}

\scriptura{Lc. 1, 46-55}

%\vspace{-2mm}

\cantusSineNeumas

\initiumpsalmi{temporalia/magnificat-initium-ii_-B.gtex}

%\vspace{-2mm}

\input{temporalia/magnificat-ii_-B.tex} \Abardot{}

\vspace{-1cm}}
\newcommand{\benedicamuslaudes}{\cuminitiali{}{temporalia/benedicamus-memoria-laudes.gtex}}
\newcommand{\hebdomada}{I}
\newcommand{\oratioMatutinum}{\noindent Præsta, quǽsumus, omnípotens Deus: \gredagger{} ut qui paschália festa perégimus, \grestar{} hæc, te largiénte, móribus et vita teneámus. Per Dóminum.}
\newcommand{\oratioLaudes}{\cuminitiali{}{temporalia/oratio.gtex}}


% LuaLaTeX

\documentclass[a4paper, twoside, 12pt]{article}
\usepackage[latin]{babel}
%\usepackage[landscape, left=3cm, right=1.5cm, top=2cm, bottom=1cm]{geometry} % okraje stranky
%\usepackage[landscape, a4paper, mag=1166, truedimen, left=2cm, right=1.5cm, top=1.6cm, bottom=0.95cm]{geometry} % okraje stranky
\usepackage[landscape, a4paper, mag=1400, truedimen, left=0.5cm, right=0.5cm, top=0.5cm, bottom=0.5cm]{geometry} % okraje stranky

\usepackage{fontspec}
\setmainfont[FeatureFile={junicode.fea}, Ligatures={Common, TeX}, RawFeature=+fixi]{Junicode}
%\setmainfont{Junicode}

% shortcut for Junicode without ligatures (for the Czech texts)
\newfontfamily\nlfont[FeatureFile={junicode.fea}, Ligatures={Common, TeX}, RawFeature=+fixi]{Junicode}

\usepackage{multicol}
\usepackage{color}
\usepackage{lettrine}
\usepackage{fancyhdr}

% usual packages loading:
\usepackage{luatextra}
\usepackage{graphicx} % support the \includegraphics command and options
\usepackage{gregoriotex} % for gregorio score inclusion
\usepackage{gregoriosyms}
\usepackage{wrapfig} % figures wrapped by the text
\usepackage{parcolumns}
\usepackage[contents={},opacity=1,scale=1,color=black]{background}
\usepackage{tikzpagenodes}
\usepackage{calc}
\usepackage{longtable}
\usetikzlibrary{calc}

\setlength{\headheight}{14.5pt}

\input{conventuscommune.tex} % Often used macros

\newcommand{\annusEditionis}{2021}

%%%% Vicekrat opakovane kousky

\newcommand{\anteOrationem}{
  \rubrica{Ante Orationem, cantatur a Superiore:}

  \pars{Supplicatio Litaniæ.}

  \cuminitiali{}{temporalia/supplicatiolitaniae.gtex}

  \pars{Oratio Dominica.}

  \cuminitiali{}{temporalia/oratiodominica.gtex}

  \rubrica{Deinde dicitur ab Hebdomadario:}

  \cuminitiali{}{temporalia/dominusvobiscum-solemnis.gtex}

  \rubrica{In choro monialium loco Dominus vobiscum dicitur:}

  \sineinitiali{temporalia/domineexaudi.gtex}
}

\setlength{\columnsep}{30pt} % prostor mezi sloupci

%%%%%%%%%%%%%%%%%%%%%%%%%%%%%%%%%%%%%%%%%%%%%%%%%%%%%%%%%%%%%%%%%%%%%%%%%%%%%%%%%%%%%%%%%%%%%%%%%%%%%%%%%%%%%
\begin{document}

% Here we set the space around the initial.
% Please report to http://home.gna.org/gregorio/gregoriotex/details for more details and options
\grechangedim{afterinitialshift}{2.2mm}{scalable}
\grechangedim{beforeinitialshift}{2.2mm}{scalable}
\grechangedim{interwordspacetext}{0.22 cm plus 0.15 cm minus 0.05 cm}{scalable}%
\grechangedim{annotationraise}{-0.2cm}{scalable}

% Here we set the initial font. Change 38 if you want a bigger initial.
% Emit the initials in red.
\grechangestyle{initial}{\color{red}\fontsize{38}{38}\selectfont}

\pagestyle{empty}

%%%% Titulni stranka
\begin{titulusOfficii}
\ifx\titulus\undefined
\nomenFesti{Feria V \hebdomada{}}
\else
\titulus
\fi
\end{titulusOfficii}

\vfill

\begin{center}
%Ad usum et secundum consuetudines chori \guillemotright{}Conventus Choralis\guillemotleft.

%Editio Sancti Wolfgangi \annusEditionis
\end{center}

\scriptura{}

\pars{}

\pagebreak

\renewcommand{\headrulewidth}{0pt} % no horiz. rule at the header
\fancyhf{}
\pagestyle{fancy}

\cantusSineNeumas

\ifx\oratio\undefined
\ifx\lauda\undefined
\else
\newcommand{\oratio}{\pars{Oratio.}

\noindent Omnípotens sempitérne Deus, véspere, mane et merídie maiestátem tuam supplíciter deprecámur, ut, expúlsis de córdibus nostris peccatórum ténebris, ad veram lucem, quæ Christus est, nos fácias perveníre.

\noindent Qui tecum vivit et regnat in unitáte Spíritus Sancti, Deus, per ómnia sǽcula sæculórum.

\noindent \Rbardot{} Amen.}
\fi
\fi

\hora{Ad Matutinum.} %%%%%%%%%%%%%%%%%%%%%%%%%%%%%%%%%%%%%%%%%%%%%%%%%%%%%
%\sideThumbs{Matutinum}

\vspace{2mm}

\cuminitiali{}{temporalia/dominelabiamea.gtex}

\vfill
%\pagebreak

\vspace{2mm}

\ifx\invitatorium\undefined
\pars{Invitatorium.} \scriptura{Ps. 94, 6; Psalmus 94; \textbf{H136}}

\vspace{-6mm}

\antiphona{E}{temporalia/inv-adoremusdominum.gtex}
\else
\invitatorium
\fi

\vfill
\pagebreak

\ifx\hymnusmatutinum\undefined
\ifx\matuac\undefined
\else
\pars{Hymnus.} \scriptura{Gregorius Magnus (+604)}

{
\grechangedim{interwordspacetext}{0.10 cm plus 0.15 cm minus 0.05 cm}{scalable}%
\antiphona{IV}{temporalia/hym-NoxAtra.gtex}
\grechangedim{interwordspacetext}{0.22 cm plus 0.15 cm minus 0.05 cm}{scalable}%
}
\fi
\else
\hymnusmatutinum
\fi

\vspace{-3mm}

\vfill
\pagebreak

\ifx\matua\undefined
\else
% MAT A
\pars{Psalmus 1.} \scriptura{Ps. 17, 3; \textbf{H99}}

\vspace{-4mm}

\antiphona{VIII G}{temporalia/ant-dominusfirmamentum.gtex}

%\vspace{-2mm}

\scriptura{Ps. 17, 31-35}

%\vspace{-2mm}

\initiumpsalmi{temporalia/ps17xxxi_xxxv-initium-viii-G-auto.gtex}

\input{temporalia/ps17xxxi_xxxv-viii-G.tex} \Abardot{}

\vfill
\pagebreak

\pars{Psalmus 2.} \scriptura{Ps. 62, 9; \textbf{H393}}

\vspace{-4mm}

\antiphona{VII c trans.}{temporalia/ant-mesuscepit.gtex}

%\vspace{-2mm}

\scriptura{Ps. 17, 36-46}

%\vspace{-2mm}

\initiumpsalmi{temporalia/ps17xxxvi_xlvi-initium-vii-c-trans.gtex}

\input{temporalia/ps17xxxvi_xlvi-vii-c.tex} \Abardot{}

\vfill
\pagebreak

\pars{Psalmus 3.} \scriptura{Ps. 17, 47; \textbf{H100}}

\vspace{-4mm}

\antiphona{VII c\textsuperscript{2}}{temporalia/ant-vivitdominus.gtex}

%\vspace{-2mm}

\scriptura{Ps. 17, 47-51}

%\vspace{-2mm}

\initiumpsalmi{temporalia/ps17xlvii_li-initium-vii-c2-auto.gtex}

\input{temporalia/ps17xlvii_li-vii-c2.tex} \Abardot{}

\vfill
\pagebreak
\fi
\ifx\matuc\undefined
\else
% MAT C
\pars{Psalmus 1.} \scriptura{Lam. 1, 21; \textbf{H177}}

\vspace{-4mm}

\antiphona{VII a}{temporalia/ant-omnesinimici.gtex}

%\vspace{-2mm}

\scriptura{Ps. 88, 39-46}

%\vspace{-2mm}

\initiumpsalmi{temporalia/ps88xxxix_xlvi-initium-vii-a-auto.gtex}

\input{temporalia/ps88xxxix_xlvi-vii-a.tex} \Abardot{}

\vfill
\pagebreak

\pars{Psalmus 2.} \scriptura{Ps. 88, 53; \textbf{H98}}

\vspace{-4mm}

\antiphona{VI F}{temporalia/ant-benedictusdominusinaeternum.gtex}

%\vspace{-2mm}

\scriptura{Ps. 88, 47-53}

%\vspace{-2mm}

\initiumpsalmi{temporalia/ps88xlvii_liii-initium-vi-F-auto.gtex}

\input{temporalia/ps88xlvii_liii-vi-F.tex} \Abardot{}

\vfill
\pagebreak

\pars{Psalmus 3.} \scriptura{Ps. 89, 13}

\vspace{-4mm}

\antiphona{I g}{temporalia/ant-converteredomine.gtex}

%\vspace{-2mm}

\scriptura{Ps. 89}

%\vspace{-2mm}

\initiumpsalmi{temporalia/ps89-initium-i-g-auto.gtex}

\input{temporalia/ps89-i-g.tex}

\vfill

\antiphona{}{temporalia/ant-converteredomine.gtex}

\vfill
\pagebreak
\fi

\pars{Versus.}

\ifx\matversus\undefined
\ifx\matua\undefined
\else
\noindent \Vbardot{} Révela, Dómine, óculos meos.

\noindent \Rbardot{} Et considerábo mirabília de lege tua.
\fi
\ifx\matuc\undefined
\else
\noindent \Vbardot{} Audies de ore meo verbum.

\noindent \Rbardot{} Et annuntiábis eis ex me.
\fi
\else
\matversus
\fi

\vspace{5mm}

\sineinitiali{temporalia/oratiodominica-mat.gtex}

\vspace{5mm}

\pars{Absolutio.}

\cuminitiali{}{temporalia/absolutio-exaudi.gtex}

\vfill
\pagebreak

\cuminitiali{}{temporalia/benedictio-solemn-benedictione.gtex}

\vspace{7mm}

\lectioi

\noindent \Vbardot{} Tu autem, Dómine, miserére nobis.
\noindent \Rbardot{} Deo grátias.

\vfill
\pagebreak

\responsoriumi

\vfill
\pagebreak

\cuminitiali{}{temporalia/benedictio-solemn-unigenitus.gtex}

\vspace{7mm}

\lectioii

\noindent \Vbardot{} Tu autem, Dómine, miserére nobis.
\noindent \Rbardot{} Deo grátias.

\vfill
\pagebreak

\responsoriumii

\vfill
\pagebreak

\cuminitiali{}{temporalia/benedictio-solemn-spiritus.gtex}

\vspace{7mm}

\lectioiii

\noindent \Vbardot{} Tu autem, Dómine, miserére nobis.
\noindent \Rbardot{} Deo grátias.

\vfill
\pagebreak

\responsoriumiii

\vfill
\pagebreak

\rubrica{Reliqua omittuntur, nisi Laudes separandæ sint.}

\sineinitiali{temporalia/domineexaudi.gtex}

\vfill

\oratio

\vfill

\noindent \Vbardot{} Dómine, exáudi oratiónem meam.
\Rbardot{} Et clamor meus ad te véniat.

\vfill

\noindent \Vbardot{} Benedicámus Dómino.
\noindent \Rbardot{} Deo grátias.

\vfill

\noindent \Vbardot{} Fidélium ánimæ per misericórdiam Dei requiéscant in pace.
\Rbardot{} Amen.

\vfill
\pagebreak

\hora{Ad Laudes.} %%%%%%%%%%%%%%%%%%%%%%%%%%%%%%%%%%%%%%%%%%%%%%%%%%%%%
%\sideThumbs{Laudes}

\cantusSineNeumas

\vspace{0.5cm}
\grechangedim{interwordspacetext}{0.18 cm plus 0.15 cm minus 0.05 cm}{scalable}%
\cuminitiali{}{temporalia/deusinadiutorium-communis.gtex}
\grechangedim{interwordspacetext}{0.22 cm plus 0.15 cm minus 0.05 cm}{scalable}%

\vfill
\pagebreak

\ifx\hymnuslaudes\undefined
\ifx\laudac\undefined
\else
\pars{Hymnus}

\grechangedim{interwordspacetext}{0.16 cm plus 0.15 cm minus 0.05 cm}{scalable}%
\cuminitiali{I}{temporalia/hym-SolEcce.gtex}
\grechangedim{interwordspacetext}{0.22 cm plus 0.15 cm minus 0.05 cm}{scalable}%
\vspace{-3mm}
\fi
\else
\hymnuslaudes
\fi

\vfill
\pagebreak

\ifx\lauda\undefined
\else
\pars{Psalmus 1.}

\vspace{-4mm}

\antiphona{VIII G}{temporalia/ant-exsurgamdiluculo.gtex}

%\vspace{-2mm}

\scriptura{Psalmus 56}

%\vspace{-2mm}

\initiumpsalmi{temporalia/ps56-initium-viii-g-auto.gtex}

%\vspace{-1.5mm}

\input{temporalia/ps56-viii-g.tex} \Abardot{}

\vfill
\pagebreak

\pars{Psalmus 2.} \scriptura{Ier. 31, 14}

\vspace{-4mm}

\antiphona{IV* e}{temporalia/ant-populusmeusait.gtex}

%\vspace{-2mm}

\scriptura{Canticum Ieremiæ, 1 Ier. 31, 10-14}

%\vspace{-3mm}

\initiumpsalmi{temporalia/jeremiae3-initium-iv_-e-auto.gtex}

\input{temporalia/jeremiae3-iv_-e.tex} \Abardot{}

\vfill
\pagebreak

\pars{Psalmus 3.} \scriptura{Ps. 95, 4; \textbf{H94}}

\vspace{-4mm}

\antiphona{IV a}{temporalia/ant-magnusdominus.gtex}

\scriptura{Psalmus 47}

\initiumpsalmi{temporalia/ps47-initium-iv-a-auto.gtex}

\input{temporalia/ps47-iv-a.tex} \Abardot{}

\vfill
\pagebreak
\fi
\ifx\laudc\undefined
\else
\pars{Psalmus 1.} \scriptura{Ps. 86, 1; \textbf{H98}}

\vspace{-4mm}

\antiphona{I g}{temporalia/ant-fundamentaeius.gtex}

%\vspace{-2mm}

\scriptura{Psalmus 86}

%\vspace{-2mm}

\initiumpsalmi{temporalia/ps86-initium-i-g-auto.gtex}

%\vspace{-1.5mm}

\input{temporalia/ps86-i-g.tex} \Abardot{}

\vfill
\pagebreak

\pars{Psalmus 2.}

\vspace{-4mm}

\antiphona{II D}{temporalia/ant-eccedominusnosterbrachio.gtex}

%\vspace{-2mm}

\scriptura{Canticum Isaiæ, Is. 40, 10-17}

%\vspace{-3mm}

\initiumpsalmi{temporalia/isaiae9-initium-ii-D-auto.gtex}

\input{temporalia/isaiae9-ii-D.tex} \Abardot{}

\vfill
\pagebreak

\pars{Psalmus 3.} \scriptura{Ps. 144, 17}

\vspace{-4mm}

\antiphona{E}{temporalia/ant-iustusetsanctus.gtex}

\scriptura{Psalmus 98}

\initiumpsalmi{temporalia/ps98-initium-e.gtex}

\input{temporalia/ps98-e.tex} \Abardot{}

\vfill
\pagebreak
\fi

\ifx\lectiobrevis\undefined
\ifx\lauda\undefined
\else
\pars{Lectio Brevis.} \scriptura{Is. 66, 1-2}

\noindent Hæc dicit Dóminus: Cælum thronus meus, terra autem scabéllum pedum meórum. Quæ ista domus, quam ædificábitis mihi, et quis iste locus quiétis meæ? Omnia hæc manus mea fecit et mea sunt univérsa ista, dicit Dóminus. Ad hunc autem respíciam, ad paupérculum et contrítum spíritu et treméntem sermónes meos.
\fi
\else
\lectiobrevis
\fi

\vfill

\ifx\responsoriumbreve\undefined
\ifx\laudac\undefined
\else
\pars{Responsorium breve.} \scriptura{Ps. 118, 145}

\cuminitiali{VI}{temporalia/resp-clamaviintotocorde.gtex}
\fi
\else
\responsoriumbreve
\fi

\vfill
\pagebreak

\ifx\benedictus\undefined
\ifx\laudac\undefined
\else
\pars{Canticum Zachariæ.} \scriptura{Lc. 1, 74.75; \textbf{H423}}

%\vspace{-4mm}

{
\grechangedim{interwordspacetext}{0.18 cm plus 0.15 cm minus 0.05 cm}{scalable}%
\antiphona{VII a}{temporalia/ant-insanctitate.gtex}
\grechangedim{interwordspacetext}{0.22 cm plus 0.15 cm minus 0.05 cm}{scalable}%
}

%\vspace{-3mm}

\scriptura{Lc. 1, 68-79}

%\vspace{-2mm}

\cantusSineNeumas
\initiumpsalmi{temporalia/benedictus-initium-vii-a-auto.gtex}

%\vspace{-1.5mm}

\input{temporalia/benedictus-vii-a.tex} \Abardot{}
\fi
\else
\benedictus
\fi

\vspace{-1cm}

\vfill
\pagebreak

%\sideThumbs{{\scriptsize{}Fine horarum}}

\pars{Preces.}

\sineinitiali{}{temporalia/tonusprecum.gtex}

\ifx\preces\undefined
\ifx\lauda\undefined
\else
\noindent Grátias agámus Christo, qui lumen huius diéi nobis concédit, \gredagger{} et ad eum clamémus:

\Rbardot{} Bénedic et sanctífica nos, Dómine.

\noindent Qui te pro peccátis nostris hóstiam obtulísti, \gredagger{} incépta et propósita suscípias hodiérna.

\Rbardot{} Bénedic et sanctífica nos, Dómine.

\noindent Qui óculos nostros lucis dono lætíficas novæ, \gredagger{} lúcifer oriáris in córdibus nostris.

\Rbardot{} Bénedic et sanctífica nos, Dómine.

\noindent Tríbue hódie nos esse ómnibus longánimes, \gredagger{} ut imitatóres tui fíeri possímus.

\Rbardot{} Bénedic et sanctífica nos, Dómine.

\noindent Audítam, Dómine, fac nobis mane misericórdiam tuam. \gredagger{} Sit hódie gáudium tuum fortitúdo nostra.

\Rbardot{} Bénedic et sanctífica nos, Dómine.
\fi
\ifx\laudc\undefined
\else
\noindent Christo, bono pastóri, qui pro suis óvibus ánimam pósuit, \gredagger{} laudes grati exsolvámus et supplicémus, dicéntes:

\Rbardot{} Pasce pópulum tuum, Dómine.

\noindent Christe, qui in sanctis pastóribus misericórdiam et dilectiónem tuam dignátus es osténdere, \gredagger{} numquam désinas per eos nobíscum misericórditer ágere.

\Rbardot{} Pasce pópulum tuum, Dómine.

\noindent Qui múnere pastóris animárum fungi per tuos vicários pergis, \gredagger{} ne destíteris nos ipse per rectóres nostros dirígere.

\Rbardot{} Pasce pópulum tuum, Dómine.

\noindent Qui in sanctis tuis, populórum dúcibus, córporum animarúmque médicus exstitísti, \gredagger{} numquam cesses ministérium in nos vitæ et sanctitátis perágere.

\Rbardot{} Pasce pópulum tuum, Dómine.

\noindent Qui, prudéntia et caritáte sanctórum, tuum gregem erudísti, \gredagger{} nos in sanctitáte iúgiter per pastóres nostros ædífica.

\Rbardot{} Pasce pópulum tuum, Dómine.
\fi
\else
\preces
\fi

\vfill

\pars{Oratio Dominica.}

\cuminitiali{}{temporalia/oratiodominicaalt.gtex}

\vfill
\pagebreak

\rubrica{vel:}

\pars{Supplicatio Litaniæ.}

\cuminitiali{}{temporalia/supplicatiolitaniae.gtex}

\vfill

\pars{Oratio Dominica.}

\cuminitiali{}{temporalia/oratiodominica.gtex}

\vfill
\pagebreak

% Oratio. %%%
\oratio

\vspace{-1mm}

\vfill

\rubrica{Hebdomadarius dicit Dominus vobiscum, vel, absente sacerdote vel diacono, sic concluditur:}

\vspace{2mm}

\antiphona{C}{temporalia/dominusnosbenedicat.gtex}

\rubrica{Postea cantatur a cantore:}

\vspace{2mm}

\cuminitiali{IV}{temporalia/benedicamus-feria-laudes.gtex}

\vspace{1mm}

\vfill
\pagebreak

\end{document}

