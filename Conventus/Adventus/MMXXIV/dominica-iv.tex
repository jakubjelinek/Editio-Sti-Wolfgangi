\newcommand{\titulus}{\nomenFesti{Dominica IV Adventus.}}
\newcommand{\oratio}{\pars{Oratio.}

\noindent Grátiam tuam, quǽsumus, Dómine, méntibus nostris infúnde, ut, qui, Angelo nuntiánte, Christi Fílii tui incarnatiónem cognóvimus, per passiónem eius et crucem ad resurrectiónis glóriam perducámur.

\pars{Pro pace in universo mundo.} \scriptura{Sir. 50, 25; 2 Esdr. 4, 20; \textbf{H416}}

\vspace{-4mm}

\antiphona{II D}{temporalia/ant-dapacemdomine.gtex}

\vfill

\noindent Deus, a quo sancta desidéria, recta consília et iusta sunt ópera: da servis tuis illam, quam mundus dare non potest, pacem; ut et corda nostra mandátis tuis dédita, et hóstium subláta formídine, témpora sint tua protectióne tranquílla.

\noindent Per Dóminum nostrum Iesum Christum, Fílium tuum, qui tecum vivit et regnat in unitáte Spíritus Sancti, Deus, per ómnia sǽcula sæculórum.

\noindent \Rbardot{} Amen.}
\newcommand{\vesperasi}{\cantusSineNeumas

\vspace{0.5cm}
\grechangedim{interwordspacetext}{0.18 cm plus 0.15 cm minus 0.05 cm}{scalable}%
\cuminitiali{}{temporalia/deusinadiutorium-communis.gtex}
\grechangedim{interwordspacetext}{0.22 cm plus 0.15 cm minus 0.05 cm}{scalable}%

\vfill
\pagebreak

\pars{Psalmus 1.} \scriptura{Ioel 2, 1; \textbf{H33}}

\vspace{-4mm}

\antiphona{I g}{temporalia/ant-canitetuba.gtex}

\scriptura{Psalmus 144, 10-21.}

\initiumpsalmi{temporalia/ps144ii-initium-i-g-auto.gtex}

\input{temporalia/ps144ii-i-g.tex}

\vfill

\antiphona{}{temporalia/ant-canitetuba.gtex}

\vspace{-1cm}

\vfill
\pagebreak

\pars{Psalmus 2.} \scriptura{Ag. 2, 7; \textbf{H33}}

\vspace{-4mm}

\antiphona{I f}{temporalia/ant-eccevenietdesideratus.gtex}

\scriptura{Psalmus 145.}

\initiumpsalmi{temporalia/ps145-initium-i-f-auto.gtex}

\input{temporalia/ps145-i-f.tex} \Abardot{}

\vfill
\pagebreak

\pars{Psalmus 3.} \scriptura{Is. 40, 4; \textbf{H33}}

\vspace{-4mm}

\antiphona{I g}{temporalia/ant-eruntprava.gtex}

%\vspace{-2mm}

\scriptura{Psalmus 146.}

%\vspace{-2mm}

\initiumpsalmi{temporalia/ps146-initium-i-g-auto.gtex}

%\vspace{-1.5mm}

\input{temporalia/ps146-i-g.tex} \Abardot{}

\vfill
\pagebreak

\pars{Psalmus 4.} \scriptura{Sap. 18, 15; \textbf{H33}}

\vspace{-4mm}

\antiphona{II D}{temporalia/ant-omnipotenssermotuus.gtex}

\scriptura{Psalmus 147.}

\initiumpsalmi{temporalia/ps147-initium-ii-D-auto.gtex}

\input{temporalia/ps147-ii-D.tex} \Abardot{}

\vfill
\pagebreak

\pars{Capitulum.} \scriptura{1 Cor. 4, 1-2}

\cuminitiali{}{temporalia/capitulum-FratresSic.gtex}}
\newcommand{\hymnusvesperas}{\pars{Hymnus.} \scriptura{Ambrosius (\olddag{} 397)}

\vspace{-5mm}

\antiphona{IV}{temporalia/hym-VerbumSalutis.gtex}}
\newcommand{\magnificati}{\pars{Canticum B. Mariæ V.} \scriptura{Sap. 7, 26; Mal. 4, 2; Lc. 1, 78-79; \textbf{H40}}

\vspace{-5mm}

{
\grechangedim{interwordspacetext}{0.18 cm plus 0.15 cm minus 0.05 cm}{scalable}%
\antiphona{II D}{temporalia/ant-ooriens.gtex}
\grechangedim{interwordspacetext}{0.22 cm plus 0.15 cm minus 0.05 cm}{scalable}%
}

\vspace{-2mm}

\scriptura{Lc. 1, 46-55}

\vspace{-2mm}

\cantusSineNeumas

\initiumpsalmi{temporalia/magnificat-initium-iisoll-D.gtex}

\vspace{-1.5mm}

\input{temporalia/magnificat-iisoll-D.tex} \Abardot{}

\vspace{-1cm}}
\newcommand{\invitatorium}{\pars{Invitatorium.}

\vspace{-6mm}

\antiphona{IV *}{temporalia/inv-propeestiam.gtex}}
\newcommand{\hymnusmatutinum}{\pars{Hymnus.}

\vspace{-5mm}

\antiphona{II}{temporalia/hym-VeniRedemptor.gtex}}
\newcommand{\nocturnoii}{\vspace{-4mm}

\pars{Psalmus 4.} \scriptura{Ag. 2, 7; \textbf{H33}}

\vspace{-4mm}

\antiphona{I f}{temporalia/ant-eccevenietdesideratus.gtex}

%\vspace{-2mm}

\scriptura{Ps. 23}

%\vspace{-2mm}

\initiumpsalmi{temporalia/ps23-initium-i-f-auto.gtex}

\input{temporalia/ps23-i-f.tex} \Abardot{}

\vfill
\pagebreak

\pars{Psalmus 5.} \scriptura{Is. 40, 4; \textbf{H33}}

\vspace{-4mm}

\antiphona{I g}{temporalia/ant-eruntprava.gtex}

%\vspace{-2mm}

\scriptura{Ps. 65, 1-12}

\initiumpsalmi{temporalia/ps65i-initium-i-a4-auto.gtex}

\input{temporalia/ps65i-i-a4.tex} \Abardot{}

\vfill
\pagebreak

\pars{Psalmus 6.} \scriptura{Is. 24, 16}

\vspace{-4mm}

\antiphona{I f}{temporalia/ant-afinibusterrae.gtex}

%\vspace{-4mm}

\scriptura{Ps. 65, 13-20}

%\vspace{-2mm}

\initiumpsalmi{temporalia/ps65ii-initium-i-f-auto.gtex}

%\vspace{-1.5mm}

\input{temporalia/ps65ii-i-f.tex} \Abardot{}

\vfill
\pagebreak}
\newcommand{\lectioi}{\pars{Lectio I.} \scriptura{Is. 49, 14-21}

\noindent De libro Isaíæ prophétæ.

\noindent Dixit Sion: «Derelíquit me Dóminus et Dóminus oblítus est mei».

\noindent Numquid oblivísci potest múlier infántem suum, ut non misereátur fílio úteri sui?

\noindent Et si illa oblíta fúerit, ego tamen non oblivíscar tui.

\noindent Ecce in mánibus meis descrípsi te; muri tui coram me semper.

\noindent Festínant structóres tui; destruéntes te et dissipántes a te exíbunt.

\noindent Leva in circúitu óculos tuos et vide: omnes isti congregáti sunt, venérunt tibi.

\noindent «Vivo ego, dicit Dóminus, quia ómnibus his velut ornaménto vestiéris et circúmdabis tibi eos quasi sponsa».

\noindent Quia ruínæ tuæ et solitúdines tuæ et terra evérsa: nunc angústa eris præ habitatóribus; et longe erunt, qui devorábant te.

\noindent Adhuc dicent in áuribus tuis fílii orbitátis tuæ: «Angústus est mihi locus; fac spátium mihi, ut hábitem».

\noindent Et dices in corde tuo: «Quis génuit mihi istos?

\noindent Ego orbáta et non páriens, transmigráta et captíva; et istos quis enutrívit?

\noindent Ecce ego relícta eram sola; et isti ubi erant?».}
\newcommand{\responsoriumi}{\pars{Responsorium 1.} \scriptura{\Rbardot{} Cf. Ioel 2, 1 \Vbardot{} Ps. 106, 3; \textbf{H31}}

\vspace{-5mm}

\responsorium{I}{temporalia/resp-canitetuba-CROCHU.gtex}{}}
\newcommand{\lectioii}{\pars{Lectio II.} \scriptura{Is. 49, 22-26}

\noindent Hæc dicit Dóminus Deus:

\noindent «Ecce levábo ad gentes manum meam et ad pópulos exaltábo signum meum;

\noindent et áfferent fílios tuos in ulnis, et fíliæ tuæ super úmeros portabúntur.

\noindent Et erunt reges nutrícii tui, et regínæ nutríces tuæ; vultu in terram demísso adorábunt te et púlverem pedum tuórum lingent.

\noindent Et scies quia ego Dóminus: non confundéntur, qui sperant in me».

\noindent Numquid tollétur a forti præda, aut, quod captum fúerit, a robústo salvári póterit?

\noindent Quia hæc dicit Dóminus: «Equidem et captívus a forti tollétur, et, quod ablátum fúerit a robústo, salvábitur;

\noindent cum his, qui contendébant tecum, ego conténdam et fílios tuos ego salvábo.

\noindent Et cibábo hostes tuos cárnibus suis, et quasi musto sánguine suo inebriabúntur;

\noindent et sciet omnis caro quia ego Dóminus salvátor tuus et redémptor tuus Fortis Iacob».}
\newcommand{\responsoriumii}{\pars{Responsorium 2.} \scriptura{\Rbar{} Cf. E. 33, 1 \Vbar{} Lev. 11, 44; \textbf{H31}}

\vspace{-5mm}

\responsorium{VII}{temporalia/resp-vigesimaquarta-CROCHU.gtex}{}}
\newcommand{\lectioiii}{\pars{Lectio III.} \scriptura{Is. 50,1}

\noindent Hæc dicit Dóminus:

\noindent «Ubinam est liber repúdii matris vestræ, quo dimísi eam?

\noindent Aut quis est créditor meus, cui véndidi vos?

\noindent Ecce in iniquitátibus vestris vénditi estis, et in sceléribus vestris dimíssa est mater vestra».}
\newcommand{\responsoriumiii}{\pars{Responsorium 3.} \scriptura{\Rbar{} Gen. 49, 10 \Vbar{} Gen. 49, 13; \textbf{H32}}

\vspace{-5mm}

\responsorium{II}{temporalia/resp-nonauferetursceptrumdeiuda-CROCHU-cumdox.gtex}{}}
\newcommand{\lectioiv}{\pars{Lectio IV.} \scriptura{Lib.1, 46-55: CCL 120, 37-39}

\noindent Ex Expositióne sancti Bedæ Venerábilis presbýteri in Lucam.

\noindent \emph{Et ait María: Magníficat ánima mea Dóminum et exsultávit spíritus meus in Deo salutári meo.}

\noindent Tanto, inquit, me Dóminus tamque inaudíto múnere sublimávit,

\noindent quod non ullo linguæ offício explicári,

\noindent sed ipso vix íntimi péctoris afféctu váleat comprehéndi,

\noindent et ídeo totas ánimæ vires in agéndis gratiárum láudibus óffero,

\noindent totum in contemplánda magnitúdine eius, cui non est finis,

\noindent quicquid vivo, séntio, discérno, gratulánter impéndo,

\noindent quia et eiúsdem Iesu, id est salutáris, spíritus meus ætérna divinitáte lætátur,

\noindent cuius mea caro temporáli conceptióne fetátur.

\noindent \emph{Quia fecit mihi magna qui potens est et sanctum nomen eius.}

\noindent Ad inítium cárminis réspicit ubi dictum est: \emph{Magníficat ánima mea Dóminum.}

\noindent Sola quippe ánima illa, cui Dóminus magna fácere dignátur, dignis eum præcóniis magnificáre et ad consórtes eiúsdem voti ac propósiti potest cohortándo dícere: \emph{Magnificáte Dóminum mecum et exaltémus nomen eius in ínvicem.}

\noindent Nam qui Dóminum, quem cognóvit, quantum in se est magnificáre et nomen eius sanctificáre contémpserit, \textit{mínimus vocábitur in regno cælórum.}

\noindent Sanctum autem nomen eius vocátur, quia singuláris cúlmine poténtiæ transcéndit omnem creatúram atque ab univérsis quæ fecit longe segregátur.}
\newcommand{\responsoriumiv}{\pars{Responsorium 4.} \scriptura{\Rbar{} Io. 3, 30; 1, 27 \Vbar{} Cf. Io. 3, 32; \textbf{H32}}

\vspace{-5mm}

\responsorium{III}{temporalia/resp-meoportetminui-CROCHU.gtex}{}}
\newcommand{\lectiov}{\pars{Lectio V.}

\noindent \emph{Suscépit Israel púerum suum, memorátus misericórdiæ suæ.}

\noindent Pulchre púerum Dómini appéllat Israel, qui ab eo sit ad salvándum suscéptus, obœdiéntem vidélicet et húmilem, iuxta quod Osée dicit: \emph{Quia puer Israel et diléxi eum.}

\noindent Nam qui contémnit humiliári, non potest útique salvári nec dícere cum Prophéta: \emph{Ecce enim Deus ádiuvat me et Dóminus suscéptor est ánimæ meæ. Quicúmque autem humiliáverit se sicut párvulus, hic est maior in regno cælórum.}}
\newcommand{\responsoriumv}{\pars{Responsorium 5.} \scriptura{\Rbar{} Gal. 4, 4; Rom. 8, 3 \Vbar{} Is. 14, 1; \textbf{H32}}

\vspace{-5mm}

\responsorium{V}{temporalia/resp-ecceiamveniet-CROCHU.gtex}{}}
\newcommand{\lectiovi}{\pars{Lectio VI.}

\noindent \emph{Sicut locútus est ad patres nostros, Abraham et sémini eius in sǽcula.}

\noindent Semen Abrahæ non carnále sed spiritále signíficat, id est non eius tantum carne progénitos, sed sive in circumcisióne seu in præpútio fídei illíus vestígia secútos. Nam et ipse in præpútio pósitus crédidit, reputatúmque est ei ad iustítiam.

\noindent Advéntus ergo Salvatóris Abrahæ est et sémini eius in sǽcula promíssus, hoc est fíliis promissiónis, quibus dícitur: \emph{Si autem vos Christi, ergo Abrahæ semen estis secúndum promissiónem herédes.}

\noindent Bene autem vel Dómini vel Ioánnis exórtum matres prophetándo prævéniunt, ut sicut peccátum a muliéribus cœpit, ita étiam bona a muliéribus incípiant, et quæ per uníus deceptiónem périit, duábus certátim præconántibus mundo vita reddátur.}
\newcommand{\responsoriumvi}{\pars{Responsorium 6.} \scriptura{\Rbar{} Ier. 31, 21 \Vbar{} Ps. 106, 3; \textbf{H32}}

\vspace{-5mm}

\responsorium{VIII}{temporalia/resp-virgoisrael-CROCHU-cumdox.gtex}{}

\rubrica{vel ad libitum:}

\vspace{3mm}

\pars{Responsorium 6.} \scriptura{\Rbardot{} Lc. 1, 48-49 \Vbardot{} ibid. 50; \textbf{H297}}

\vspace{-5mm}

\responsorium{VIII}{temporalia/resp-beatammedicent.gtex}{}}
\newcommand{\evangelium}{
\pars{Versus.} \scriptura{Cf. Mch. 1, 3}

% Versus. %%%
\sineinitiali{temporalia/versus-egredietur.gtex}

\vspace{5mm}

\sineinitiali{temporalia/oratiodominica-mat.gtex}

\vspace{5mm}

\pars{Absolutio.}

\cuminitiali{}{temporalia/absolutio-avinculis.gtex}

\vfill
\pagebreak

\cuminitiali{}{temporalia/benedictio-solemn-evangelica.gtex}

\vspace{7mm}

\pars{Evangelium} \scriptura{Lc. 1, 39-45}

\noindent Léctio sancti Evangélii secúndum Lucam.

\noindent Exsúrgens María in diébus illis ábiit in montána cum festinatióne in civitátem Iudæ et intrávit in domum Zacharíæ et salutávit Elísabeth. Et factum est, ut audívit salutatiónem Maríæ Elísabeth, exsultávit infans in útero eius, et repléta est Spíritu Sancto Elísabeth et exclamávit voce magna et dixit: «Benedícta tu inter mulíeres, et benedíctus fructus ventris tui. Et unde hoc mihi, ut véniat mater Dómini mei ad me? Ecce enim ut facta est vox salutatiónis tuæ in áuribus meis, exsultávit in gáudio infans in útero meo. Et beáta, quæ crédidit, quóniam perficiéntur ea, quæ dicta sunt ei a Dómino».

\scriptura{Sermo 2 de Advéntus, 1. 2-3 : SC 166,104. 108}

\noindent Ex Sermónibus beáti Guerríci abbátis.

\noindent \emph{Ecce venit Rex: occurrámus óbviam Salvatóri nostro.} Pulchre Sálomon ait: \emph{Aqua frígida ánimæ sitiénti núntius bonus de terra longínqua.} Bonus útique núntius, qui advéntum Salvatóris núntiat, reconciliatiónem mundi, bona superventúri sǽculi. \emph{Quam pulchri pedes annuntiántium pacem, annuntiántium bona.} Multi síquidem, non unus, multi, inquam, sed in uno spíritu longa série ab inítio sǽculi nobis supervenére núntii; et ómnium vox símilis, una senténtia: \emph{Venit, ecce venit.}

\noindent Veni ergo, Dómine, \emph{salvum me fac et salvus ero;} veni et \emph{osténde fáciem tuam et salvi érimus. Te enim exspectávimus; esto salus nostra in témpore tribulatiónis.} Sic prophétæ et iusti desidério et afféctu tanto ante Christo ventúro occurrébant, desiderántes si fíeri posset óculis vidére quod spíritu prævidébant.

\noindent Unde Dóminus discípulis dicébat: \emph{Beáti óculi, qui vident quæ vos vidétis. Dico enim vobis, quod multi prophétæ et iusti voluérunt vidére quæ vos vidétis et non vidérunt.} Abraham quoque pater noster exultávit ut vidéret diem Christi. \emph{Vidit,} sed apud ínferos, \emph{et gavísus est.} In quo útique tepor et durítia cordis nostri sugillátur: si non vidélicet cum gáudio spiritáli Christi nascéntis diem anniversárium exspectámus, qui nobis in próximo vidéndus, Dómino annuénte, promíttitur.

\noindent Hoc sane gáudium nostrum tale vidétur exígere Scriptúra, ut spíritus noster levans se super se Christo veniénti quodámmodo occúrrere géstiat, desidério se exténdens in anterióra, impatiénsque morárum iam vidére conténdat futúra. Ego namque non solum ad secúndum advéntum sed étiam ad primum árbitror pertinére, quod tot locis Scripturárum ei monémur occúrrere. Quómodo? inquis. Quia vidélicet, sicut secúndo advéntui occurrémus motu et exsultatióne córporis, sic et primo occurréndum est afféctu et exsultatióne cordis.

\vfill
\pagebreak

\pars{Responsorium 7.} \scriptura{\Rbar{} Hebr. 7, 4 \Vbar{} Ps. 71, 8; \textbf{H33}}

\vspace{-5mm}

\responsorium{III}{temporalia/resp-intueminiquantus-CROCHU-cumdox.gtex}{}

\vfill
\pagebreak

{
\pars{Hymnus Ambrosianus} \scriptura{Tonus Monasticus}

\vspace{-2mm}

\grechangedim{interwordspacetext}{0.26 cm plus 0.15 cm minus 0.05 cm}{scalable}%
\cuminitiali{III}{temporalia/tedeum-monasticum-am34.gtex}
\grechangedim{interwordspacetext}{0.22 cm plus 0.15 cm minus 0.05 cm}{scalable}%
}

\vfill
\pagebreak}
\newcommand{\hymnuslaudes}{\pars{Hymnus}

\cuminitiali{D}{temporalia/hym-MagnisProphetae.gtex}}
\newcommand{\laudes}{\pars{Psalmus 1.} \scriptura{Ioel 2, 1; \textbf{H33}}

\vspace{-4mm}

\antiphona{I g}{temporalia/ant-canitetubainsion.gtex}

%\vspace{-2mm}

\scriptura{Psalmus 117}

%\vspace{-2mm}

\initiumpsalmi{temporalia/ps117-initium-i-g-auto.gtex}

%\vspace{-1.5mm}

\input{temporalia/ps117-i-g.tex}

\vfill

\antiphona{}{temporalia/ant-canitetubainsion.gtex}

\vfill
\pagebreak

\pars{Psalmus 2.} \scriptura{Cf. Is. 9, 5-6; \textbf{H33}}

\vspace{-4mm}

\antiphona{I f}{temporalia/ant-dominusvenietoccurite.gtex}

%\vspace{-2mm}

\scriptura{Canticum Danielis, Dan. 3, 52-57}

%\vspace{-3mm}

\initiumpsalmi{temporalia/dan33-initium-i-f-auto.gtex}

\input{temporalia/dan33-i-f.tex} \Abardot{}

\vfill
\pagebreak

\pars{Psalmus 3.} \scriptura{Sap. 18, 15; \textbf{H33}}

\vspace{-4mm}

\antiphona{II D}{temporalia/ant-omnipotenssermotuus.gtex}

\scriptura{Psalmus 150}

\initiumpsalmi{temporalia/ps150-initium-ii-D-auto.gtex}

\input{temporalia/ps150-ii-D.tex} \Abardot{}

\vfill
\pagebreak}
\newcommand{\lectiobrevis}{\pars{Lectio Brevis.} \scriptura{Rom. 13, 11-12}

\noindent Hora est iam vos de somno súrgere, nunc enim própior est nobis salus quam cum credídimus. Nox procéssit, dies autem appropiávit. Abiciámus ergo ópera tenebrárum et induámur arma lucis.}
\newcommand{\benedictus}{\pars{Canticum Zachariæ.} \scriptura{Lc. 1, 44; \textbf{H39}}

\vspace{-4mm}

\antiphona{II* f}{temporalia/ant-exquofactaestvox.gtex}

\vspace{-3mm}

\scriptura{Lc. 1, 68-79}

\vspace{-2mm}

\cantusSineNeumas
\initiumpsalmi{temporalia/benedictus-initium-ii_soll-f.gtex}

\vspace{-1.5mm}

\input{temporalia/benedictus-ii_soll-f.tex} \Abardot{}}
\newcommand{\preces}{\noindent Dóminum Iesum Christum, fratres caríssimi, deprecémur,~\gredagger{} qui est iudex vivórum et mortuórum,~\grestar{} ipsi dicéntes:

\Rbardot{} Veni, Dómine Iesu.

\noindent Christe Dómine,~\gredagger{} qui peccatóres salváre venísti,~\grestar{} nos ab omni tentatiónum adversitáte defénde.

\Rbardot{} Veni, Dómine Iesu.

\noindent Qui ad iudícium maniféste ventúrus esse créderis,~\grestar{} poténtiam tuæ salvatiónis in nobis osténde.

\Rbardot{} Veni, Dómine Iesu.

\noindent Da nobis legis tuæ præcépta virtúte spíritus custodíre,~\grestar{} ut advéntum tuum in caritáte præstolári possímus.

\Rbardot{} Veni, Dómine Iesu.

\noindent Tu, qui es benedíctus in sǽcula,~\gredagger{} fac ut per misericórdiam tuam pie et sóbrie in hoc sǽculo vivámus,~\grestar{} exspectántes beátam spem et advéntum magnificéntiæ tuæ.

\Rbardot{} Veni, Dómine Iesu.}
\newcommand{\vesperasii}{
\cantusSineNeumas

%\vspace{0.5cm}
\grechangedim{interwordspacetext}{0.18 cm plus 0.15 cm minus 0.05 cm}{scalable}%
\cuminitiali{}{temporalia/deusinadiutorium-communis.gtex}
\grechangedim{interwordspacetext}{0.22 cm plus 0.15 cm minus 0.05 cm}{scalable}%

\vfill
%\pagebreak

\vspace{-2mm}

\pars{Psalmus 1.} \scriptura{Ioel 2, 1; \textbf{H33}}

\vspace{-4mm}

\antiphona{I g}{temporalia/ant-canitetuba.gtex}

\vspace{-2mm}

\scriptura{Psalmus 109.}

\vspace{-1mm}

\initiumpsalmi{temporalia/ps109-initium-i-g-auto.gtex}

\input{temporalia/ps109-i-g.tex}

\vfill

\antiphona{}{temporalia/ant-canitetuba.gtex}

\vspace{-1cm}

\vfill
\pagebreak

\pars{Psalmus 2.} \scriptura{Ag. 2, 7; \textbf{H33}}

\vspace{-4mm}

\antiphona{I f}{temporalia/ant-eccevenietdesideratus.gtex}

\scriptura{Psalmus 110.}

\initiumpsalmi{temporalia/ps110-initium-i-f-auto.gtex}

\input{temporalia/ps110-i-f.tex} \Abardot{}

\vfill
\pagebreak

\pars{Psalmus 3.} \scriptura{Is. 40, 4; \textbf{H33}}

\vspace{-4mm}

\antiphona{I g}{temporalia/ant-eruntprava.gtex}

\scriptura{Psalmus 111.}

\initiumpsalmi{temporalia/ps111-initium-i-g-auto.gtex}

\input{temporalia/ps111-i-g.tex} \Abardot{}

\vfill
\pagebreak

\pars{Psalmus 4.} \scriptura{Sap. 18, 15; \textbf{H33}}

\vspace{-4mm}

\antiphona{II D}{temporalia/ant-omnipotenssermotuus.gtex}

\scriptura{Psalmus 112.}

\initiumpsalmi{temporalia/ps112-initium-ii-D-auto.gtex}

\input{temporalia/ps112-ii-D.tex} \Abardot{}

\vfill
\pagebreak

\pars{Capitulum.} \scriptura{1 Cor. 4, 1-2}

\cuminitiali{}{temporalia/capitulum-FratresSic.gtex}}
\newcommand{\magnificatii}{\pars{Canticum B. Mariæ V.} \scriptura{Ier. 10, 7; Ag. 2, 8; Eph. 2, 20.14; \textbf{H40}}

\vspace{-6.5mm}

{
\grechangedim{interwordspacetext}{0.18 cm plus 0.15 cm minus 0.05 cm}{scalable}%
\antiphona{II D}{temporalia/ant-orex.gtex}
\grechangedim{interwordspacetext}{0.22 cm plus 0.15 cm minus 0.05 cm}{scalable}%
}

\vspace{-3mm}

\scriptura{Lc. 1, 46-55}

\vspace{-2mm}

\cantusSineNeumas

\initiumpsalmi{temporalia/magnificat-initium-iisoll-D.gtex}

\vspace{-1.5mm}

\input{temporalia/magnificat-iisoll-D.tex} \Abardot{}

\vspace{-1cm}}
\newcommand{\hebdomada}{infra Hebdom. IV post Pentecosten.}
\newcommand{\oratioLaudes}{\cuminitiali{}{temporalia/oratio4.gtex}}

% LuaLaTeX

\documentclass[a4paper, twoside, 12pt]{article}
\usepackage[latin]{babel}
%\usepackage[landscape, left=3cm, right=1.5cm, top=2cm, bottom=1cm]{geometry} % okraje stranky
%\usepackage[landscape, a4paper, mag=1166, truedimen, left=2cm, right=1.5cm, top=1.6cm, bottom=0.95cm]{geometry} % okraje stranky
\usepackage[landscape, a4paper, mag=1400, truedimen, left=0.5cm, right=0.5cm, top=0.5cm, bottom=0.5cm]{geometry} % okraje stranky

\usepackage{fontspec}
\setmainfont[FeatureFile={junicode.fea}, Ligatures={Common, TeX}, RawFeature=+fixi]{Junicode}
%\setmainfont{Junicode}

% shortcut for Junicode without ligatures (for the Czech texts)
\newfontfamily\nlfont[FeatureFile={junicode.fea}, Ligatures={Common, TeX}, RawFeature=+fixi]{Junicode}

\usepackage{multicol}
\usepackage{color}
\usepackage{lettrine}
\usepackage{fancyhdr}

% usual packages loading:
\usepackage{luatextra}
\usepackage{graphicx} % support the \includegraphics command and options
\usepackage{gregoriotex} % for gregorio score inclusion
\usepackage{gregoriosyms}
\usepackage{wrapfig} % figures wrapped by the text
\usepackage{parcolumns}
\usepackage[contents={},opacity=1,scale=1,color=black]{background}
\usepackage{tikzpagenodes}
\usepackage{calc}
\usepackage{longtable}
\usetikzlibrary{calc}

\setlength{\headheight}{14.5pt}

\input{conventuscommune.tex} % Often used macros
%%%% Preklady jednotlivych zpevu (nektere se opakuji, a je dobre mit je
% vsechny na jedne hromade)

% HOURS ---

\newcommand{\trAntI}{\translatioCantus{Muž boží měl kožený toulec, pečlivě
zavázaný, jenž mu visel na šíji a~často se ho dotýkal.}}

\newcommand{\trAntII}{\translatioCantus{Klíč od~něho tak dobře střežil, že
dokud žil v~těle, nikdo z~jeho žáků nezvěděl, co je uvnitř.}}

\newcommand{\trAntIII}{\translatioCantus{Ale když se odebral z~tohoto
života, schránku otevřeli a~objevili v~ní žíněné roucho a~měděný řetěz
potřísněný krví.}}

\newcommand{\trAntIV}{\translatioCantus{A když prohlédli mistrovo tělo,
nalezli jeho tělo na čtyřech místech hluboce zbrázděno ranami od řetězu.}}

\newcommand{\trAntV}{\translatioCantus{Krev vytékající z~těch ran, místy
prostoupila i~žíněným rouchem.}}

\newcommand{\trCapituli}{\translatioCantus{
Miláčkovi Boha a~lidí,
Mojžíšovi požehnané paměti,~\gredagger{}
dopřál slávu rovnou slávě svatých~\grestar{}
učinil ho mocným na postrach nepřátelům
a~jeho slovy zastavil divy.}}

\newcommand{\trLectioBrevis}{\translatioCantus{
Pamatujte na své představené,
kteří vám hlásali Boží slovo.
Uvažte, jak oni skončili život, a~napodobujte jejich víru.
Ježíš Kristus je stejný včera i~dnes i~navěky.
Nenechte se svést věelijakými cizími naukami.}}

\newcommand{\trRespLaud}{\translatioCantus{Spravedlivého vodil Hospodin~\grestar{}
po přímých stezkách. \Vbardot{} A~ukázal mu Boží království.}}

\newcommand{\trRespLaudB}{\translatioCantus{Na tvých hradbách, Jeruzaléme,
ustanovil jsem strážné;~\grestar{}
budou bdít nad mým lidem. \Vbardot{} Ani ve dne, ani v~noci nesmějí nikdy
mlčet.}}

\newcommand{\trVersus}{\translatioCantus{\Vbardot{} Ústa spravedlivého šeptají moudrost, aleluja.
\Rbardot{} A~jeho jazyk ohlašuje právo, aleluja.}}

\newcommand{\trAntBenedictus}{\translatioCantus{Když na bujné oře vložili
nosítka a~sňali jim uzdu, vydali se přímo k~cele božího muže.}}

\newcommand{\trPreces}{\translatioCantus{
\noindent S vděčností chvalme Krista, dobrého Pastýře, \gredagger{} který dal život za své ovce, \grestar{} a~pokorně ho prosme: \Rbardot{} Pane, buď pastýřem svého lidu.

\noindent Kriste, ty dáváš církvi pastýře, a~jejich službou se ujímáš svého lidu, \grestar{} dej, ať v~lásce těch, kteří nás vedou, poznáváme, jak nás miluješ. \Rbardot{} Pane, buď pastýřem svého lidu.

\noindent Ty stále konáš skrze své zástupce službu pastýře a~učitele, \grestar{} nepřestávej nás nikdy vést prostřednictvím svých služebníků. \Rbardot{} Pane, buď pastýřem svého lidu.

\noindent Ty prokazuješ svému lidu skrze jeho pastýře službu lékaře duše i~těla, \grestar{} ochraňuj náš život a~veď nás ke svatosti. \Rbardot{} Pane, buď pastýřem svého lidu.

\noindent Ty posíláš své svaté, aby slovem i~příkladem vedli tvůj lid k~tobě, \grestar{} na jejich přímluvu nás posiluj, abychom vytrvali na cestě, která vede k~věčnému životu. \Rbardot{} Pane, buď pastýřem svého lidu.}}

\newcommand{\trOrationis}{\translatioCantus{Bože, jenž nám dopřáváš radovat
se z~výroční slavnosti svatého tvého vyznavače Havla, uděl dobrotivě,
abychom když slavíme jeho narození, též se řídili podobou jeho skutků.
Skrze…}}
 % Czech translations of the proper texts

\newcommand{\annusEditionis}{2020}

%%%% Vicekrat opakovane kousky

\newcommand{\anteOrationem}{
  \rubrica{Ante Orationem, cantatur a Superiore:}

  \pars{Supplicatio Litaniæ.}

  \cuminitiali{}{temporalia/supplicatiolitaniae.gtex}

  \pars{Oratio Dominica.}

  \cuminitiali{}{temporalia/oratiodominica.gtex}

  \rubrica{Deinde dicitur ab Hebdomadario:}

  \cuminitiali{}{temporalia/dominusvobiscum-solemnis.gtex}

  \rubrica{In choro monialium loco Dominus vobiscum dicitur:}

  \sineinitiali{temporalia/domineexaudi.gtex}
}

\setlength{\columnsep}{30pt} % prostor mezi sloupci

%%%%%%%%%%%%%%%%%%%%%%%%%%%%%%%%%%%%%%%%%%%%%%%%%%%%%%%%%%%%%%%%%%%%%%%%%%%%%%%%%%%%%%%%%%%%%%%%%%%%%%%%%%%%%
\begin{document}

% Here we set the space around the initial.
% Please report to http://home.gna.org/gregorio/gregoriotex/details for more details and options
\grechangedim{afterinitialshift}{2.2mm}{scalable}
\grechangedim{beforeinitialshift}{2.2mm}{scalable}
\grechangedim{interwordspacetext}{0.22 cm plus 0.15 cm minus 0.05 cm}{scalable}%
\grechangedim{annotationraise}{-0.2cm}{scalable}

% Here we set the initial font. Change 38 if you want a bigger initial.
% Emit the initials in red.
\grechangestyle{initial}{\color{red}\fontsize{38}{38}\selectfont}

\pagestyle{empty}

%%%% Titulni stranka
\begin{titulusOfficii}
\titulus{}
\end{titulusOfficii}

% graphic
%\vspace{1.5cm}
%\begin{center}
%\includegraphics[width=8cm]{emmaus.jpg}
%\end{center}

\vfill

\begin{center}
%Ad usum et secundum consuetudines chori \guillemotright{}Conventus Choralis\guillemotleft.

%Editio Sancti Wolfgangi \annusEditionis
\end{center}

\pagebreak

\renewcommand{\headrulewidth}{0pt} % no horiz. rule at the header
\fancyhf{}
\pagestyle{fancy}

\pars{Oratio ante divinum Officium.}

\lettrine{{\color{red}A}}{peri,} Dómine, os meum ad benedicéndum nomen sanctum tuum:
munda quoque cor meum ab ómnibus vanis, pervérsis, et aliénis
cogitatiónibus:
intelléctum illúmina, afféctum inflámma,
ut digne, atténte ac devóte hoc Offícium recitáre váleam,
et exaudíri mérear ante conspéctum Divínæ Maiestátis tuæ.
Per Christum, Dóminum nostrum.
\Rbardot{} Amen.

Dómine, in unióne illíus divínæ intentiónis,
qua ipse in terris laudes Deo persolvísti,
has tibi Horas \rubricatum{(vel \textnormal{hanc tibi Horam})} persólvo.

%\trOratioAnteOfficium

\vfill

\pars{Oratio post divinum Officium.}

\rubrica{
  Orationem sequentem devote post Officium recitantibus
  Leo Papa X. defectus, et culpas in eo persolvendo ex humana
  fragilitate contractas, indulsit, et dicitur flexis genibus.
}

\lettrine{{\color{red}S}}{acrosánctæ} et indivíduæ Trinitáti,
crucifíxi Dómini nostri Iesu Christi humanitáti,
beatíssimæ et gloriosíssimæ sempérque Vírginis Maríæ
fecúndæ integritáti, 
et ómnium Sanctórum universitáti
sit sempitérna laus, honor, virtus et glória
ab omni creatúra,
nobísque remíssio ómnium peccatórum,
per infiníta sǽcula sæculórum.
\Rbardot{} Amen.

\noindent \Vbardot{} Beáta víscera Maríæ Virginis, quæ portavérunt
ætérni Patris Fílium.\\
\Rbardot{} Et beáta úbera, quæ lactavérunt Christum Dominum.

\rubrica{Et dicitur secreto \textnormal{Pater noster.} et \textnormal{Ave María.}}

%\trOratioPostOfficium

\vfill

\hora{Ad I. Vesperas.} %%%%%%%%%%%%%%%%%%%%%%%%%%%%%%%%%%%%%%%%%%%%%%%%%%%%%
%\sideThumbs{I. Vesperæ}

\cantusSineNeumas

\vspace{0.5cm}
\grechangedim{interwordspacetext}{0.18 cm plus 0.15 cm minus 0.05 cm}{scalable}%
\cuminitiali{}{temporalia/deusinadiutorium-solemnis.gtex}
\grechangedim{interwordspacetext}{0.22 cm plus 0.15 cm minus 0.05 cm}{scalable}%

\vfill
\pagebreak

\pars{Psalmus 1.} \scriptura{Ps. 144, 13; \textbf{H100}}

\vspace{-4mm}

\antiphona{VII c\textsuperscript{2}}{temporalia/ant-regnumtuum.gtex}

\scriptura{Psalmus 144, 10-21.}

\initiumpsalmi{temporalia/ps144ii-initium-vii-c2-auto.gtex}

%\psalmusEtTranslatioT{temporalia/ps144ii-VII-comb.tex}{10cm}
\input{temporalia/ps144ii-VII.tex} \Abardot{}

\vspace{-1cm}

\vfill
\pagebreak

\pars{Psalmus 2.} \scriptura{Ps. 145, 2; \textbf{H100}}

\vspace{-4mm}

\antiphona{IV E}{temporalia/ant-laudabodeum.gtex}

\scriptura{Psalmus 145.}

\initiumpsalmi{temporalia/ps145-initium-iv-E-auto.gtex}

%\psalmusEtTranslatioT{temporalia/ps145-VII-comb.tex}{10cm}
\input{temporalia/ps145-VII.tex} \Abardot{}

\vfill
\pagebreak

\pars{Psalmus 3.} \scriptura{Ps. 146, 1; \textbf{H101}}

\vspace{-4mm}

\antiphona{VIII a}{temporalia/ant-deonostro.gtex}

\scriptura{Psalmus 146.}

\initiumpsalmi{temporalia/ps146-initium-viii-A-auto.gtex}

%\psalmusEtTranslatioT{temporalia/ps146-VII-comb.tex}{10cm}
\input{temporalia/ps146-VII.tex} \Abardot{}

\vfill
\pagebreak

\pars{Psalmus 4.} \scriptura{Ps. 147, 1}

\vspace{-4mm}

\antiphona{E}{temporalia/ant-laudajerusalem.gtex}

\scriptura{Psalmus 147.}

\initiumpsalmi{temporalia/ps147-initium-e-auto.gtex}

%\psalmusEtTranslatioT{temporalia/ps147-VII-comb.tex}{10cm}
\input{temporalia/ps147-VII.tex} \Abardot{}

\vfill
\pagebreak

\pars{Capitulum.} \scriptura{Rom. 11, 33}

\grechangedim{interwordspacetext}{0.12 cm plus 0.15 cm minus 0.05 cm}{scalable}%
\cuminitiali{}{temporalia/capitulum-OAltitudo.gtex}
\grechangedim{interwordspacetext}{0.22 cm plus 0.15 cm minus 0.05 cm}{scalable}

% preklad Jeruz. bible
%\trCapituliI

\vfill

\pars{Responsorium breve.} \scriptura{Ps. 146, 5}

\cuminitiali{VI}{temporalia/resp-magnusdominusnoster.gtex}

%\trResp

\vfill
\pagebreak

\pars{Hymnus} \scriptura{Ambrosius (\olddag{} 397)}

\cuminitiali{I}{temporalia/hym-OLuxBeata-aestivalis.gtex}
\vspace{-3mm}
%\input{hym-OLuxBeata-bohtext.tex}

\vfill
%\pagebreak

\pars{Versus.}

% Versus. %%%
\sineinitiali{temporalia/versus-vespertina.gtex}

%\noindent \trVersus

\vfill
\pagebreak

\magnificati

\vfill
\pagebreak

%\sideThumbs{{\scriptsize{}Fine horarum}}

\anteOrationem

\pagebreak

% Oratio. %%%
\oratioLaudes

\vspace{-1mm}
%\trOrationisI

\vfill

\rubrica{Hebdomadarius dicit iterum Dominus vobiscum, vel cantor dicit:}

\vspace{2mm}

\sineinitiali{temporalia/domineexaudi.gtex}

\rubrica{Postea cantatur a cantore:}

\vspace{2mm}

\cuminitiali{I}{temporalia/benedicamus-dominica-perannum.gtex}

\vspace{1mm}

\vfill
\pagebreak

\hora{Ad Matutinum.} %%%%%%%%%%%%%%%%%%%%%%%%%%%%%%%%%%%%%%%%%%%%%%%%%%%%%
%\sideThumbs{Matutinum}

\vspace{2mm}

\cuminitiali{}{temporalia/dominelabiamea.gtex}

\vspace{2mm}

\pars{Invitatorium.} \scriptura{Ps. 94, 1; Psalmus 94}

\vspace{-6mm}

\antiphona{E}{temporalia/inv-veniteexsultemus.gtex}

\vfill
\pagebreak

\pars{Hymnus.} \scriptura{Adamus Sancti Victoris (\olddag 1146)}

\vspace{-5mm}

\antiphona{VII}{temporalia/hym-SalveDies.gtex}

\scriptura{Non dicitur \textnormal{Amen} in fine.}
%{
%\vspace{-5mm}
%\setlength{\columnsep}{0pt} % prostor mezi sloupci
%\input{hym-SalveDies-bohtext.tex}
%\setlength{\columnsep}{30pt} % prostor mezi sloupci
%}

\vfill
\pagebreak

\subhora{In I. Nocturno}

\pars{Psalmus 1.} \scriptura{Ps. 1, 1}

\vspace{-4mm}

\antiphona{VIII G}{temporalia/ant-beatusvir.gtex}

%\vspace{-5mm}

\scriptura{Ps. 1}

%\vspace{-2mm}

\initiumpsalmi{temporalia/ps1-initium-viii-G-auto.gtex}

%\psalmusEtTranslatioT{temporalia/ps1-I-comb.tex}{10cm}
\input{temporalia/ps1-I.tex} \Abardot{}

\vfill
\pagebreak

\pars{Psalmus 2.} \scriptura{Ps. 2, 11; \textbf{H93}}

\vspace{-4mm}

\antiphona{VII a}{temporalia/ant-servitedomino.gtex}

\vspace{-3mm}

\scriptura{Ps. 2}

\vspace{-2mm}

\initiumpsalmi{temporalia/ps2-initium-vii-a-auto.gtex}

%\psalmusEtTranslatioT{temporalia/ps2-I-comb.tex}{10cm}
\input{temporalia/ps2-I.tex} \Abardot{}

\vfill
\pagebreak

\pars{Psalmus 3.} \scriptura{Ps. 3, 7}

\vspace{-4mm}

\antiphona{VI F}{temporalia/ant-exsurgedominesalvum.gtex}

%\vspace{-5mm}

\scriptura{Ps. 3}

\initiumpsalmi{temporalia/ps3-initium-vi-F-auto.gtex}

%\psalmusEtTranslatioT{temporalia/ps3-I-comb.tex}{10cm}
\input{temporalia/ps3-I.tex} \Abardot{}

\vfill
\pagebreak

\pars{Versus.} \scriptura{Ps. 118, 55}

% Versus. %%%
\sineinitiali{temporalia/versus-memorfui.gtex}

\vspace{5mm}

\sineinitiali{temporalia/oratiodominica-mat.gtex}

\vspace{5mm}

\pars{Absolutio.}

\cuminitiali{}{temporalia/absolutio-exaudi.gtex}

\vfill
\pagebreak

\cuminitiali{}{temporalia/benedictio-solemn-benedictione.gtex}

\vspace{7mm}

\lectioi

\noindent \Vbardot{} Tu autem, Dómine, miserére nobis.
\noindent \Rbardot{} Deo grátias.

\vfill
\pagebreak

\responsoriumi

\vfill
\pagebreak

\cuminitiali{}{temporalia/benedictio-solemn-unigenitus.gtex}

\vspace{7mm}

\lectioii

\noindent \Vbardot{} Tu autem, Dómine, miserére nobis.
\noindent \Rbardot{} Deo grátias.

\vfill
\pagebreak

\responsoriumii

\vfill
\pagebreak

\cuminitiali{}{temporalia/benedictio-solemn-spiritus.gtex}

\vspace{7mm}

\lectioiii

\noindent \Vbardot{} Tu autem, Dómine, miserére nobis.
\noindent \Rbardot{} Deo grátias.

\vfill
\pagebreak

\responsoriumiii

\vfill
\pagebreak

\subhora{In II. Nocturno}

\pars{Psalmus 4.} \scriptura{Ps. 8, 2}

\vspace{-4mm}

\antiphona{I g}{temporalia/ant-quamadmirabileest.gtex}

%\vspace{-5mm}

\scriptura{Ps. 8}

%A\vspace{-2mm}

\initiumpsalmi{temporalia/ps8-initium-i-g-auto.gtex}

%\psalmusEtTranslatioT{temporalia/ps8-I-comb.tex}{10cm}
\input{temporalia/ps8-I.tex} \Abardot{}

\vfill
\pagebreak

\pars{Psalmus 5.} \scriptura{Ps. 9, 5}

\vspace{-4mm}

\antiphona{VIII G}{temporalia/ant-sedistisuperthronum.gtex}

%\vspace{-5mm}

\scriptura{Ps. 9, 2-11}

\initiumpsalmi{temporalia/ps9ii_xi-initium-viii-G-auto.gtex}

%\psalmusEtTranslatioT{temporalia/ps9ii_xi-I-comb.tex}{10cm}
\input{temporalia/ps9ii_xi-I.tex} \Abardot{}

\vfill
\pagebreak

\pars{Psalmus 6.} \scriptura{Ps. 9, 20}

\vspace{-4mm}

\antiphona{I g\textsuperscript{3}}{temporalia/ant-exsurgedominenon.gtex}

%\vspace{-5mm}

\scriptura{Ps. 9, 12-21}

\initiumpsalmi{temporalia/ps9xii_xxi-initium-i-g3-auto.gtex}

%\psalmusEtTranslatioT{temporalia/ps9xii_xxi-I-comb.tex}{10cm}
\input{temporalia/ps9xii_xxi-I.tex} \Abardot{}

\vfill
\pagebreak

\pars{Versus.} \scriptura{Ps. 118, 62}

% Versus. %%%
\sineinitiali{temporalia/versus-medianocte.gtex}

\vspace{5mm}

\sineinitiali{temporalia/oratiodominica-mat.gtex}

\vspace{5mm}

\pars{Absolutio.}

\cuminitiali{}{temporalia/absolutio-ipsius.gtex}

\vfill
\pagebreak

\cuminitiali{}{temporalia/benedictio-solemn-deus.gtex}

\vspace{7mm}

\lectioiv

\noindent \Vbardot{} Tu autem, Dómine, miserére nobis.
\noindent \Rbardot{} Deo grátias.

\vfill
\pagebreak

\responsoriumiv

\vfill
\pagebreak

\cuminitiali{}{temporalia/benedictio-solemn-christus.gtex}

\vspace{7mm}

\lectiov

\noindent \Vbardot{} Tu autem, Dómine, miserére nobis.
\noindent \Rbardot{} Deo grátias.

\vfill
\pagebreak

\responsoriumv

\vfill
\pagebreak

\cuminitiali{}{temporalia/benedictio-solemn-ignem.gtex}

\vspace{7mm}

\lectiovi

\noindent \Vbardot{} Tu autem, Dómine, miserére nobis.
\noindent \Rbardot{} Deo grátias.

\vfill
\pagebreak

\responsoriumvi

\vfill
\pagebreak

\subhora{In III. Nocturno}

\pars{Psalmus 7.} \scriptura{Ps. 9, 22}

\vspace{-4mm}

\antiphona{II D}{temporalia/ant-utquiddomine.gtex}

\vspace{-4mm}

\scriptura{Ps. 9, 22-32}

%\vspace{-2mm}

\initiumpsalmi{temporalia/ps9xxii_xxxii-initium-ii-D-auto.gtex}

%\psalmusEtTranslatioT{temporalia/ps9xxii_xxxii-I-comb.tex}{10cm}
\input{temporalia/ps9xxii_xxxii-I.tex} \Abardot{}

\vfill
\pagebreak

\pars{Psalmus 8.}\scriptura{Ex. 15, 18}

\vspace{-4mm}

\antiphona{IV* e}{temporalia/ant-inaeternum.gtex}

%\vspace{-4mm}

\scriptura{Ps. 9, 33-39}

\initiumpsalmi{temporalia/ps9xxxiii_xxxix-initium-iv_-e-auto.gtex}

%\psalmusEtTranslatioT{temporalia/ps9xxxiii_xxxix-I-comb.tex}{10cm}
\input{temporalia/ps9xxxiii_xxxix-I.tex} \Abardot{}

\vfill
\pagebreak

\pars{Psalmus 9.} \scriptura{Ps. 10, 8}

\vspace{-4mm}

\antiphona{II* f}{temporalia/ant-justusdominus.gtex}

%\vspace{-4mm}

\scriptura{Ps. 10}

%\initiumpsalmi{temporalia/ps10-initium-iv-c-auto.gtex}
\initiumpsalmi{temporalia/ps10-initium-ii_-f.gtex}

%\psalmusEtTranslatioT{temporalia/ps10-I-comb.tex}{10cm}
\input{temporalia/ps10-I.tex} \Abardot{}

\vfill
\pagebreak

\pars{Versus.} \scriptura{Ps. 118, 148}

% Versus. %%%
\sineinitiali{temporalia/versus-praevenerunt.gtex}

\vspace{5mm}

\sineinitiali{temporalia/oratiodominica-mat.gtex}

\vspace{5mm}

\pars{Absolutio.}

\cuminitiali{}{temporalia/absolutio-avinculis.gtex}

\vfill
\pagebreak

\cuminitiali{}{temporalia/benedictio-solemn-evangelica.gtex}

\vspace{7mm}

\lectiovii

\noindent \Vbardot{} Tu autem, Dómine, miserére nobis.
\noindent \Rbardot{} Deo grátias.

\vfill
\pagebreak

\responsoriumvii

\vfill
\pagebreak

\cuminitiali{}{temporalia/benedictio-solemn-divinum.gtex}

\vspace{7mm}

\lectioviii

\noindent \Vbardot{} Tu autem, Dómine, miserére nobis.
\noindent \Rbardot{} Deo grátias.

\vfill
\pagebreak

\responsoriumviii

\vfill
\pagebreak

\cuminitiali{}{temporalia/benedictio-solemn-adsocietatem.gtex}

\vspace{7mm}

\lectioix

\noindent \Vbardot{} Tu autem, Dómine, miserére nobis.
\noindent \Rbardot{} Deo grátias.

\vfill
\pagebreak

% Te Deum

{
\pars{Hymnus Ambrosianus} \scriptura{Tonus Solemnis}

\vspace{-2mm}

\grechangedim{interwordspacetext}{0.26 cm plus 0.15 cm minus 0.05 cm}{scalable}%
\cuminitiali{III}{temporalia/tedeum-solemnis-gn.gtex}
\grechangedim{interwordspacetext}{0.22 cm plus 0.15 cm minus 0.05 cm}{scalable}%
}

\vfill
\pagebreak

\rubrica{Reliqua omittuntur, nisi Laudes separandæ sint.}

\pars{Oratio}

\noindent \Vbardot{} Dómine, exáudi oratiónem meam.

\noindent \Rbardot{} Et clamor meus ad te véniat.

Orémus:

\oratioLaudes

\vspace{7mm}

\pars{Conclusio}

\noindent \Vbardot{} Dómine, exáudi oratiónem meam.

\noindent \Rbardot{} Et clamor meus ad te véniat.

\noindent \Vbardot{} Benedicámus Dómino, allelúia, allelúia.

\noindent \Rbardot{} Deo grátias, allelúia, allelúia.

\noindent \Vbardot{} Fidélium ánimæ per misericórdiam Dei requiéscant in pace.

\noindent \Rbardot{} Amen.

\vfill
\pagebreak

\hora{Ad Laudes.} %%%%%%%%%%%%%%%%%%%%%%%%%%%%%%%%%%%%%%%%%%%%%%%%%%%%%
%\sideThumbs{Laudes}

\cantusSineNeumas

\vspace{0.5cm}
\grechangedim{interwordspacetext}{0.18 cm plus 0.15 cm minus 0.05 cm}{scalable}%
\cuminitiali{}{temporalia/deusinadiutorium-alter.gtex}
\grechangedim{interwordspacetext}{0.22 cm plus 0.15 cm minus 0.05 cm}{scalable}%

\vfill
%\pagebreak

\pars{Psalmus 1.}

\vspace{-4mm}

\antiphona{VI F}{temporalia/ant-alleluia1.gtex}

\scriptura{Psalmus 50.}

\initiumpsalmi{temporalia/ps50-initium-vi-F-auto.gtex}

%\psalmusEtTranslatioT{temporalia/ps50-I-comb.tex}{10cm}
\input{temporalia/ps50-I.tex}

\vfill
\pagebreak

\pars{Psalmus 2.}

\scriptura{Psalmus 117.}

\initiumpsalmi{temporalia/ps117-initium-vi-F-auto.gtex}

%\psalmusEtTranslatioT{temporalia/ps117-I-comb.tex}{10cm}
\input{temporalia/ps117-I.tex}

\vfill
\pagebreak

\pars{Psalmus 3.}

\scriptura{Psalmus 62.}

\initiumpsalmi{temporalia/ps62-initium-vi-F-auto.gtex}

%\psalmusEtTranslatioT{temporalia/ps62-I-comb.tex}{10cm}
\input{temporalia/ps62-I.tex}

\vfill

\vspace{-6mm}

\antiphona{}{temporalia/ant-alleluia1.gtex} % repeat the antiphon - new page

\vfill
\pagebreak

\pars{Psalmus 4.} \scriptura{Dan. 3, 22-26; \textbf{H422}}

\vspace{-4mm}

\antiphona{VIII G}{temporalia/ant-trespueri.gtex}

\scriptura{Canticum trium puerorum, Dan. 3, 57-88 et 56}

\initiumpsalmi{temporalia/dan3-initium-viii-G-auto.gtex}

%\psalmusEtTranslatioT{temporalia/dan3-comb.tex}{10cm}
\input{temporalia/dan3.tex}

\rubrica{Hic non dicitur Gloria Patri, neque Amen.}

\vfill

\vspace{-6mm}

\antiphona{}{temporalia/ant-trespueri.gtex} % repeat the antiphon - new page

\vfill
\pagebreak

\pars{Psalmus 5.}

\vspace{-4mm}

\antiphona{VIII G}{temporalia/ant-alleluia2.gtex}

\scriptura{Psalmus 148.}

\initiumpsalmi{temporalia/ps148-initium-viii-G-auto.gtex}

%\psalmusEtTranslatioT{temporalia/ps148-I-comb.tex}{10cm}
\input{temporalia/ps148-I.tex}

\rubrica{Hic non dicitur Gloria Patri.}

\vfill
\pagebreak

%
\scriptura{Psalmus 149.}

\initiumpsalmi{temporalia/ps149-initium-viii-G-auto.gtex}

%\psalmusEtTranslatioT{temporalia/ps149-I-comb.tex}{10cm}
\input{temporalia/ps149-I.tex}

\rubrica{Hic non dicitur Gloria Patri.}

\vfill
\pagebreak

%
\scriptura{Psalmus 150.}

\initiumpsalmi{temporalia/ps150-initium-viii-G-auto.gtex}

%\psalmusEtTranslatioT{temporalia/ps150-I-comb.tex}{10cm}
\input{temporalia/ps150-I.tex}

\vfill

\vspace{-6mm}

\antiphona{}{temporalia/ant-alleluia2.gtex} % repeat the antiphon - new page

\vfill
\pagebreak

\pars{Capitulum.} \scriptura{Ac. 7, 12}

\grechangedim{interwordspacetext}{0.12 cm plus 0.15 cm minus 0.05 cm}{scalable}%
\cuminitiali{}{temporalia/capitulum-Benedictio.gtex}
\grechangedim{interwordspacetext}{0.22 cm plus 0.15 cm minus 0.05 cm}{scalable}

% preklad Jeruz. bible
%\trCapituliI

\vfill

\pars{Responsorium breve.} \scriptura{Ps. 118, 36-37}

\cuminitiali{IV}{temporalia/resp-inclinacormeum.gtex}

%\trResp

\vfill
\pagebreak

\pars{Hymnus} \scriptura{Gregorius Magnus (\olddag{} 604)}

\cuminitiali{IV}{temporalia/hym-EcceJamNoctis.gtex}
\vspace{-3mm}
%\input{hym-EcceJamNocis-bohtext.tex}

\vfill
%\pagebreak

\pars{Versus.} \scriptura{Ps. 92, 1}

% Versus. %%%
\sineinitiali{temporalia/versus-dominusregnavit.gtex}

%\noindent \trVersus

\vfill
\pagebreak

\benedictus

\vspace{-1cm}

\vfill
\pagebreak

%\sideThumbs{{\scriptsize{}Fine horarum}}

\anteOrationem

\pagebreak

% Oratio. %%%
\oratioLaudes

\vspace{-1mm}
%\trOrationisI

\vfill

\rubrica{Hebdomadarius dicit iterum Dominus vobiscum, vel cantor dicit:}

\vspace{2mm}

\sineinitiali{temporalia/domineexaudi.gtex}

\rubrica{Postea cantatur a cantore:}

\vspace{2mm}

\cuminitiali{I}{temporalia/benedicamus-dominica-perannum.gtex}

\vspace{1mm}

\vfill
\pagebreak

\hora{Ad II. Vesperas.} %%%%%%%%%%%%%%%%%%%%%%%%%%%%%%%%%%%%%%%%%%%%%%%%%%%%%
%\sideThumbs{II. Vesperæ}

\cantusSineNeumas

%\vspace{0.5cm}
\grechangedim{interwordspacetext}{0.18 cm plus 0.15 cm minus 0.05 cm}{scalable}%
\cuminitiali{}{temporalia/deusinadiutorium-solemnis.gtex}
\grechangedim{interwordspacetext}{0.22 cm plus 0.15 cm minus 0.05 cm}{scalable}%

\vfill
%\pagebreak

\vspace{-2mm}

\pars{Psalmus 1.} \scriptura{Ps. 109, 1; \textbf{H91}}

\vspace{-4mm}

\antiphona{VII c\textsuperscript{2}}{temporalia/ant-dixitdominus.gtex}

\vspace{-4mm}

\scriptura{Psalmus 109.}

\initiumpsalmi{temporalia/ps109-initium-vii-c2-auto.gtex}

%\psalmusEtTranslatioT{temporalia/ps109-I-comb.tex}{10cm}
\input{temporalia/ps109-I.tex} \Abardot{}

\vspace{-1cm}

\vfill
\pagebreak

\pars{Psalmus 2.} \scriptura{Ps. 110, 8; \textbf{H91}}

\vspace{-4mm}

\antiphona{IV g}{temporalia/ant-fideliaomnia.gtex}

\scriptura{Psalmus 110.}

\initiumpsalmi{temporalia/ps110-initium-iv-g-auto.gtex}

%\psalmusEtTranslatioT{temporalia/ps110-I-comb.tex}{10cm}
\input{temporalia/ps110-I.tex} \Abardot{}

\vfill
\pagebreak

\pars{Psalmus 3.} \scriptura{Ps. 111, 1; \textbf{H92}}

\vspace{-4mm}

\antiphona{IV a}{temporalia/ant-inmandatis.gtex}

\scriptura{Psalmus 111.}

\initiumpsalmi{temporalia/ps111-initium-iv-a-auto.gtex}

%\psalmusEtTranslatioT{temporalia/ps111-I-comb.tex}{10cm}
\input{temporalia/ps111-I.tex} \Abardot{}

\vfill
\pagebreak

\pars{Psalmus 4.} \scriptura{Ps. 112, 2; \textbf{H92}}

\vspace{-4mm}

\antiphona{VII c}{temporalia/ant-sitnomendomini.gtex}

\scriptura{Psalmus 112.}

\initiumpsalmi{temporalia/ps112-initium-vii-c-auto.gtex}

%\psalmusEtTranslatioT{temporalia/ps112-I-comb.tex}{10cm}
\input{temporalia/ps112-I.tex} \Abardot{}

\vfill
\pagebreak

\pars{Capitulum.} \scriptura{2 Cor. 1, 3-4}

\grechangedim{interwordspacetext}{0.12 cm plus 0.15 cm minus 0.05 cm}{scalable}%
\cuminitiali{}{temporalia/capitulum-BenedictusDeus.gtex}
\grechangedim{interwordspacetext}{0.22 cm plus 0.15 cm minus 0.05 cm}{scalable}

% preklad Jeruz. bible
%\trCapituliI

\vfill

\pars{Responsorium breve.} \scriptura{Ps. 103, 24}

\cuminitiali{VI}{temporalia/resp-quammagnificata.gtex}

%\trResp

\vfill
\pagebreak

\pars{Hymnus} \scriptura{Gregorius Magnus (\olddag{} 604)}

\cuminitiali{I}{temporalia/hym-LucisCreator-aestivalis.gtex}
\vspace{-3mm}
%\begin{translatioMulticol}{3}
Tvůrce světa předobrý,\\
tys ustanovil denní řád\\
a proudy světla rozhodil,\\
když světu základy jsi klad.\\
\\
A spojils ráno s večerem\\
a dnem tu dobu nazýváš;\\
hle padá temné noci stín -\\
slyš prosbu, vyslyš nářek náš.\columnbreak

Ach, nedej, by nás stihla smrt,\\
když svědomí nám tíží hřích,\\
když nemyslíme na věčnost\\
v té síti hříchů šalebných.\\
\\
Vzbuď naši touhu po nebi,\\
kde věčný život čeká nás,\\
a pomoz odložit vše zlé\\
a smýti z duše každý kaz.\columnbreak

To splň nám, dobrý Otče náš,\\
i ty, jenž rovné božství máš,\\
i Duchu, který těšíš nás\\
a vládneš, Bože, v každý čas.\\
Amen. 
\end{translatioMulticol}


\vfill
%\pagebreak

\pars{Versus.} \scriptura{Ps. 140, 2}

% Versus. %%%
\sineinitiali{temporalia/versus-dirigatur.gtex}

%\noindent \trVersus

\vfill
\pagebreak

\magnificatii

\vfill
\pagebreak

%\sideThumbs{{\scriptsize{}Fine horarum}}

\anteOrationem

\pagebreak

% Oratio. %%%
\oratioLaudes

\vspace{-1mm}
%\trOrationisI

\vfill

\rubrica{Hebdomadarius dicit iterum Dominus vobiscum, vel cantor dicit:}

\vspace{2mm}

\sineinitiali{temporalia/domineexaudi.gtex}

\rubrica{Postea cantatur a cantore:}

\vspace{2mm}

\cuminitiali{I}{temporalia/benedicamus-dominica-perannum.gtex}

\vspace{1mm}

\end{document}

