\newcommand{\oratio}{\pars{Oratio.}

\noindent Dirigátur, quǽsumus, Dómine, in conspéctu tuo nostræ petitiónis orátio, ut ad magnum incarnatiónis Unigéniti tui mystérium nostræ vota servitútis illibáta puritáte pervéniant.

\pars{Pro pace in Ucraina.} \scriptura{Sir. 50, 25; 2 Esdr. 4, 20; \textbf{H416}}

\vspace{-4mm}

\antiphona{II D}{temporalia/ant-dapacemdomine.gtex}

\vfill

\noindent Deus, a quo sancta desidéria, recta consília et iusta sunt ópera: da servis tuis illam, quam mundus dare non potest, pacem; ut et corda nostra mandátis tuis dédita, et hóstium subláta formídine, témpora sint tua protectióne tranquílla.

\noindent Per Dóminum nostrum Iesum Christum, Fílium tuum, qui tecum vivit et regnat in unitáte Spíritus Sancti, Deus, per ómnia sǽcula sæculórum.

\noindent \Rbardot{} Amen.}
\newcommand{\matversus}{\noindent \Vbardot{} Osténde nobis, Dómine, misericórdiam tuam.

\noindent \Rbardot{} Et salutáre tuum da nobis.}
\newcommand{\lectioi}{\pars{Lectio I.} \scriptura{Is. 24, 1-18}

\noindent De libro Isaíæ prophétæ.

\noindent Ecce Dóminus díssipat terram et frangit eam et contúrbat fáciem eius et dispérgit habitatóres eius. Et erit sicut pópulus sic sacérdos, et sicut servus sic dóminus eius, sicut ancílla sic dómina eius, sicut emens sic ille qui vendit, sicut fenerátor sic is qui mútuum áccipit, sicut qui répetit sic qui debet. Dissipatióne dissipábitur terra et direptióne prædábitur: Dóminus enim locútus est verbum hoc. Luget, languet terra, marcéscit, languet orbis, marcéscit altitúdo simul cum terra. Et terra infécta est sub habitatóribus suis, quia transgréssi sunt leges, violavérunt mandátum, dissipavérunt fœdus sempitérnum. Propter hoc maledíctio vorávit terram et pœnas exsolvérunt habitatóres eius; ideóque imminúti sunt cultóres eius, et relícti sunt hómines pauci.

\noindent Luget mustum, emárcuit vitis, ingemíscunt omnes, qui lætabántur corde. Cessávit gáudium tympanórum, quiévit sónitus lætántium, cessávit gáudium cítharæ; cum cántico non bibent vinum, amára erit pótio bibéntibus illam. Attríta est cívitas inanitátis, clausa est omnis domus, ut nemo intróeat; clamor est super vino in platéis, óccidit omnis lætítia, translátum est gáudium terræ. Relícta est in urbe solitúdo, et in ruínam confrácta est porta; quia hæc erunt in médio terræ, in médio populórum, quómodo si olívæ excutiántur, et finíta vindémia colligántur racémi.

\noindent Hi levábunt vocem suam, laudábunt maiestátem Dómini, hínnient de mari. Propter hoc in regiónibus lucis glorificáte Dóminum, in ínsulis maris nomen Dómini, Dei Israel. A fínibus terræ laudes audívimus: «Glória iusto». Et dixi: «Secrétum meum mihi, secrétum meum mihi. Væ mihi!». Prævaricántes prævaricáti sunt et prævaricatióne prævaricántium prævaricáti sunt. Formído et fóvea et láqueus super te, habitátor terræ. Et erit: qui fúgerit a voce formídinis, cadet in fóveam; et, qui ascénderit de fóvea, tenébitur láqueo, quia cataráctæ de excélsis apértæ sunt, et concússa sunt fundaménta terræ.}
\newcommand{\responsoriumi}{\pars{Responsorium 1.} \scriptura{\Rbar{} Mich. 4, 8.9 \Vbar{} Ps. 80, 9-11; \textbf{H21}}

\vspace{-5mm}

\responsorium{IV}{temporalia/resp-jerusalemcito-CROCHU.gtex}{}}
\newcommand{\lectioii}{\pars{Lectio II.} \scriptura{Lib. 2, cap. 22}

\noindent Ex Tractátu sancti Ioánnis a Cruce presbýteri De ascénsu montis Carméli.

\noindent Præcípua causa ob quam in antíqua lege lícitæ erant quæ fiébant Deo interrogatiónes, et cur necessárium erat ut prophétæ et sacerdótes visiónes ac revelatiónes a Deo requírerent, hæc erat, quia nondum tunc témporis ádeo fundáta erat fides, neque lex evangélica stabilíta; unde necessárium erat ut a Deo de rebus sciscitaréntur, et ut Deus modo verbis, modo visiónibus et revelatiónibus, tum figúris et similitudínibus, tum dénique multis áliis significatiónum modis eis loquerétur. Omne enim illud quod respondébat, loquebátur et revelábat, nostræ sacræ fídei mystéria erant, vel certe res ad eam spectántes vel ad eam diréctæ.

\noindent At nunc quando fides in Christum est fundáta, et lex evangélica in statu grátiæ promulgáta, non necésse est eum illo modo interrogáre, nec est cur ipse sicut tunc loquátur. Dando quippe nobis, sicut dedit, Fílium suum, qui est únicum solúmque ipsíus Verbum, ómnia nobis simul unáque vice hoc suo único Verbo locútus est et revelávit, neque illi quidquam dicéndum manet.

\noindent Et hic est genuínus auctoritátis illíus sensus, qua nítitur sanctus Paulus persuadére Hebrǽis, ut a prióribus illis antíquæ legis Móysis cum Deo agéndi tractandíque modis abscédant, et óculos ad Christum solum convértant, dicens: \emph{Multifáriam multísque modis olim Deus loquens pátribus in prophétis, novíssime diébus istis locútus est nobis in Fílio.} Quibus verbis docet Apóstolus tot tántaque dixísse Deum per hoc suum Verbum, ut nihil iam ámplius optándum supérsit; id enim quod ántea per partes loquebátur prophétis, iam nobis totum in ipso dixit, ipsum nobis totum dando, id est, Fílium suum.}
\newcommand{\responsoriumii}{\pars{Responsorium 2.} \scriptura{\Rbar{} Cantor \Vbar{} Is. 16, 5; \textbf{H23}}

\vspace{-5mm}

\responsorium{III}{temporalia/resp-egredieturdominus-CROCHU.gtex}{}}
\newcommand{\lectioiii}{\pars{Lectio III.}

\noindent Quámobrem ille qui nunc vellet áliquid a Deo sciscitári, vel visiónem áliquam aut revelatiónem ab eo postuláre, viderétur iniúriam Deo inférre, non defigéndo omníno suos óculos in Christum, vel áliam rem aut novitátem extra illum requiréndo.

\noindent Posset enim illi Deus hoc modo respondére: \emph{Hic est Fílius meus diléctus, in quo mihi bene complácui; ipsum audíte.} Iam ómnia locútus sum per Verbum meum: ad illum solum tuos cónice óculos; in ipso síquidem univérsa iam tibi dixi cúnctaque revelávi; immo multo plura in eo repéries quam desideráre aut pétere possis.

\noindent Iam ego una cum Sancto meo Spíritu in illum descéndi in monte Thabor, dixíque: \emph{Hic est Fílius meus diléctus in quo mihi complácui; ipsum audíte.} Non est cur novos doctrínæ ac responsiónum modos requíras; si enim loquébar ántea, id erat Christum promitténdo; si vero áliquid sciscitabántur ex me, dirigebántur interrogatiónes illæ ad postulándum et sperándum Christum, in quo univérsa bona repertúri erant, quemádmodum nunc omnis evangelistárum apostolorúmque doctrína hoc declárat.}
\newcommand{\responsoriumiii}{\pars{Responsorium 3.} \scriptura{\Rbar{} Ir. 31, 5-7 \Vbar{} Cantor; \textbf{H23}}

\vspace{-5mm}

\responsorium{VIII}{temporalia/resp-jerusalemplantabis-CROCHU-cumdox.gtex}{}}
\newcommand{\benedictus}{\pars{Canticum Zachariæ.} \scriptura{Mt. 4, 17; \textbf{H30}}

\vspace{-4mm}

{
\grechangedim{interwordspacetext}{0.18 cm plus 0.15 cm minus 0.05 cm}{scalable}%
\antiphona{VIII G}{temporalia/ant-dicitdominuspaenitentiam.gtex}
\grechangedim{interwordspacetext}{0.22 cm plus 0.15 cm minus 0.05 cm}{scalable}%
}

\vspace{-2mm}

\scriptura{Lc. 1, 68-79}

%\vspace{-2mm}

\cantusSineNeumas
\initiumpsalmi{temporalia/benedictus-initium-viii-G-auto.gtex}

%\vspace{-1.5mm}

\input{temporalia/benedictus-viii-G.tex} \Abardot{}}
\newcommand{\preces}{\noindent Christum redemptórem, fratres caríssimi, deprecémur, \gredagger{} qui véniet, ut redeúntes ad se a mortis potestáte líberet, \grestar{} et súpplices implorémus:

\Rbardot{} Veni, Dómine Iesu.

\noindent Tuum, Dómine, dum annuntiámus advéntum, \grestar{} munda cor nostrum ab omni spíritu vanitátis.

\Rbardot{} Veni, Dómine Iesu.

\noindent Ecclésia tua, Dómine, quam fundásti, \grestar{} te per omnes gentes magníficet.

\Rbardot{} Veni, Dómine Iesu.

\noindent Lex tua, Dómine, illúminans óculos, \grestar{} prótegat pópulos tíbimet confiténtes.

\Rbardot{} Veni, Dómine Iesu.

\noindent Qui gáudia advéntus tui nobis ab Ecclésia prænuntiári concédis, \grestar{} fac ut promptíssima devotióne te excipiámus.

\Rbardot{} Veni, Dómine Iesu.}
\newcommand{\magnificat}{\pars{Canticum B. Mariæ V.} \scriptura{Cf. Zach. 9, 9; Mt. 21, 5; Io. 12, 15; \textbf{H24}}

\vspace{-4mm}

{
\grechangedim{interwordspacetext}{0.18 cm plus 0.15 cm minus 0.05 cm}{scalable}%
\antiphona{II* B}{temporalia/ant-eccerexveniet.gtex}
\grechangedim{interwordspacetext}{0.22 cm plus 0.15 cm minus 0.05 cm}{scalable}%
}

%\vspace{-3mm}

\scriptura{Lc. 1, 46-55}

%\vspace{-2mm}

\cantusSineNeumas

\initiumpsalmi{temporalia/magnificat-initium-ii_-B.gtex}

%\vspace{-2mm}

\input{temporalia/magnificat-ii_-B.tex} \Abardot{}

\vspace{-1cm}}
\newcommand{\hebdomada}{infra Hebdom. II per Annum.}
\newcommand{\matub}{Matutinum Hebdomadae B}
\newcommand{\laudb}{Laudes Hebdomadae B}
\newcommand{\laudbd}{Laudes Hebdomadae B vel D}

% LuaLaTeX

\documentclass[a4paper, twoside, 12pt]{article}
\usepackage[latin]{babel}
%\usepackage[landscape, left=3cm, right=1.5cm, top=2cm, bottom=1cm]{geometry} % okraje stranky
%\usepackage[landscape, a4paper, mag=1166, truedimen, left=2cm, right=1.5cm, top=1.6cm, bottom=0.95cm]{geometry} % okraje stranky
\usepackage[landscape, a4paper, mag=1400, truedimen, left=0.5cm, right=0.5cm, top=0.5cm, bottom=0.5cm]{geometry} % okraje stranky

\usepackage{fontspec}
\setmainfont[FeatureFile={junicode.fea}, Ligatures={Common, TeX}, RawFeature=+fixi]{Junicode}
%\setmainfont{Junicode}

% shortcut for Junicode without ligatures (for the Czech texts)
\newfontfamily\nlfont[FeatureFile={junicode.fea}, Ligatures={Common, TeX}, RawFeature=+fixi]{Junicode}

\usepackage{multicol}
\usepackage{color}
\usepackage{lettrine}
\usepackage{fancyhdr}

% usual packages loading:
\usepackage{luatextra}
\usepackage{graphicx} % support the \includegraphics command and options
\usepackage{gregoriotex} % for gregorio score inclusion
\usepackage{gregoriosyms}
\usepackage{wrapfig} % figures wrapped by the text
\usepackage{parcolumns}
\usepackage[contents={},opacity=1,scale=1,color=black]{background}
\usepackage{tikzpagenodes}
\usepackage{calc}
\usepackage{longtable}
\usetikzlibrary{calc}

\setlength{\headheight}{14.5pt}

\input{conventuscommune.tex} % Often used macros

\newcommand{\annusEditionis}{2021}

%%%% Vicekrat opakovane kousky

\newcommand{\anteOrationem}{
  \rubrica{Ante Orationem, cantatur a Superiore:}

  \pars{Supplicatio Litaniæ.}

  \cuminitiali{}{temporalia/supplicatiolitaniae.gtex}

  \pars{Oratio Dominica.}

  \cuminitiali{}{temporalia/oratiodominica.gtex}

  \rubrica{Deinde dicitur ab Hebdomadario:}

  \cuminitiali{}{temporalia/dominusvobiscum-solemnis.gtex}

  \rubrica{In choro monialium loco Dominus vobiscum dicitur:}

  \sineinitiali{temporalia/domineexaudi.gtex}
}

\setlength{\columnsep}{30pt} % prostor mezi sloupci

%%%%%%%%%%%%%%%%%%%%%%%%%%%%%%%%%%%%%%%%%%%%%%%%%%%%%%%%%%%%%%%%%%%%%%%%%%%%%%%%%%%%%%%%%%%%%%%%%%%%%%%%%%%%%
\begin{document}

% Here we set the space around the initial.
% Please report to http://home.gna.org/gregorio/gregoriotex/details for more details and options
\grechangedim{afterinitialshift}{2.2mm}{scalable}
\grechangedim{beforeinitialshift}{2.2mm}{scalable}
\grechangedim{interwordspacetext}{0.22 cm plus 0.15 cm minus 0.05 cm}{scalable}%
\grechangedim{annotationraise}{-0.2cm}{scalable}

% Here we set the initial font. Change 38 if you want a bigger initial.
% Emit the initials in red.
\grechangestyle{initial}{\color{red}\fontsize{38}{38}\selectfont}

\pagestyle{empty}

%%%% Titulni stranka
\begin{titulusOfficii}
\ifx\titulus\undefined
\nomenFesti{Feria II \hebdomada{}}
\else
\titulus
\fi
\end{titulusOfficii}

\vfill

\begin{center}
%Ad usum et secundum consuetudines chori \guillemotright{}Conventus Choralis\guillemotleft.

%Editio Sancti Wolfgangi \annusEditionis
\end{center}

\scriptura{}

\pars{}

\pagebreak

\renewcommand{\headrulewidth}{0pt} % no horiz. rule at the header
\fancyhf{}
\pagestyle{fancy}

\cantusSineNeumas

\ifx\oratio\undefined
\ifx\laudb\undefined
\else
\newcommand{\oratio}{\pars{Oratio.}

\noindent Dómine Deus omnípotens, qui ad princípium huius diéi nos perveníre fecísti, tua nos hódie salva virtúte, ut in hac die ad nullum declinémus peccátum, sed semper ad tuam iustítiam faciéndam nostra procédant elóquia, dirigántur cogitatiónes et ópera.

\noindent Per Dóminum nostrum Iesum Christum, Fílium tuum, qui tecum vivit et regnat in unitáte Spíritus Sancti, Deus, per ómnia sǽcula sæculórum.

\noindent \Rbardot{} Amen.}
\fi
\fi

\hora{Ad Matutinum.} %%%%%%%%%%%%%%%%%%%%%%%%%%%%%%%%%%%%%%%%%%%%%%%%%%%%%
%\sideThumbs{Matutinum}

\vspace{2mm}

\cuminitiali{}{temporalia/dominelabiamea.gtex}

\vfill
%\pagebreak

\vspace{2mm}

\ifx\invitatorium\undefined
\pars{Invitatorium.} \scriptura{Ps. 94, 1; Psalmus 94; \textbf{H451}}

\vspace{-6mm}

\antiphona{VI}{temporalia/inv-jubilemusdeo.gtex}\else
\invitatorium
\fi

\vfill
\pagebreak

\ifx\hymnusmatutinum\undefined
\ifx\matua\undefined
\else
\pars{Hymnus.}

{
\grechangedim{interwordspacetext}{0.10 cm plus 0.15 cm minus 0.05 cm}{scalable}%
\antiphona{II}{temporalia/hym-IpsumNunc.gtex}
\grechangedim{interwordspacetext}{0.22 cm plus 0.15 cm minus 0.05 cm}{scalable}%
}
\fi
\else
\hymnusmatutinum
\fi

\vspace{-3mm}

\vfill
\pagebreak

\ifx\matub\undefined
\else
% MAT B
\pars{Psalmus 1.} \scriptura{Ps. 30, 2; \textbf{H90}}

\vspace{-4mm}

\antiphona{VIII G}{temporalia/ant-intuaiustitia.gtex}

%\vspace{-2mm}

\scriptura{Ps. 30, 2-9}

%\vspace{-2mm}

\initiumpsalmi{temporalia/ps30i-initium-viii-G-auto.gtex}

\vspace{-1.5mm}

\input{temporalia/ps30i-viii-G.tex} \Abardot{}

\vfill
\pagebreak

\pars{Psalmus 2.} \scriptura{Ps. 66, 2}

\vspace{-4mm}

\antiphona{E}{temporalia/ant-illuminadomine.gtex}

%\vspace{-2mm}

\scriptura{Ps. 30, 10-17}

%\vspace{-2mm}

\initiumpsalmi{temporalia/ps30ii-initium-e-a-auto.gtex}

\input{temporalia/ps30ii-e-a.tex} \Abardot{}

\vfill
\pagebreak

\pars{Psalmus 3.} \scriptura{Ps. 30, 24}

\vspace{-4mm}

\antiphona{II D}{temporalia/ant-diligitedominum.gtex}

%\vspace{-5mm}

\scriptura{Ps. 30, 20-25}

%\vspace{-2mm}

\initiumpsalmi{temporalia/ps30iii-initium-ii-D-auto.gtex}

\input{temporalia/ps30iii-ii-D.tex} \Abardot{}

\vfill
\pagebreak
\fi

\pars{Versus.}

\ifx\matversus\undefined
\ifx\matub\undefined
\else
\noindent \Vbardot{} Dírige me, Dómine, in veritáte tua, et doce me.

\noindent \Rbardot{} Quia tu es Deus salútis meæ.
\fi
\else
\matversus
\fi

\vspace{5mm}

\sineinitiali{temporalia/oratiodominica-mat.gtex}

\vspace{5mm}

\pars{Absolutio.}

\cuminitiali{}{temporalia/absolutio-exaudi.gtex}

\vfill
\pagebreak

\cuminitiali{}{temporalia/benedictio-solemn-benedictione.gtex}

\vspace{7mm}

\lectioi

\noindent \Vbardot{} Tu autem, Dómine, miserére nobis.
\noindent \Rbardot{} Deo grátias.

\vfill
\pagebreak

\responsoriumi

\vfill
\pagebreak

\cuminitiali{}{temporalia/benedictio-solemn-unigenitus.gtex}

\vspace{7mm}

\lectioii

\noindent \Vbardot{} Tu autem, Dómine, miserére nobis.
\noindent \Rbardot{} Deo grátias.

\vfill
\pagebreak

\responsoriumii

\vfill
\pagebreak

\cuminitiali{}{temporalia/benedictio-solemn-spiritus.gtex}

\vspace{7mm}

\lectioiii

\noindent \Vbardot{} Tu autem, Dómine, miserére nobis.
\noindent \Rbardot{} Deo grátias.

\vfill
\pagebreak

\responsoriumiii

\vfill
\pagebreak

\rubrica{Reliqua omittuntur, nisi Laudes separandæ sint.}

\sineinitiali{temporalia/domineexaudi.gtex}

\vfill

\oratio

\vfill

\noindent \Vbardot{} Dómine, exáudi oratiónem meam.
\Rbardot{} Et clamor meus ad te véniat.

\vfill

\noindent \Vbardot{} Benedicámus Dómino.
\noindent \Rbardot{} Deo grátias.

\vfill

\noindent \Vbardot{} Fidélium ánimæ per misericórdiam Dei requiéscant in pace.
\Rbardot{} Amen.

\vfill
\pagebreak

\hora{Ad Laudes.} %%%%%%%%%%%%%%%%%%%%%%%%%%%%%%%%%%%%%%%%%%%%%%%%%%%%%
%\sideThumbs{Laudes}

\cantusSineNeumas

\vspace{0.5cm}
\grechangedim{interwordspacetext}{0.18 cm plus 0.15 cm minus 0.05 cm}{scalable}%
\cuminitiali{}{temporalia/deusinadiutorium-communis.gtex}
\grechangedim{interwordspacetext}{0.22 cm plus 0.15 cm minus 0.05 cm}{scalable}%

\vfill
\pagebreak

\ifx\hymnuslaudes\undefined
\ifx\laudbd\undefined
\else
\pars{Hymnus} \scriptura{Hilarius (\olddag{} 367)}

\grechangedim{interwordspacetext}{0.16 cm plus 0.15 cm minus 0.05 cm}{scalable}%
\cuminitiali{IV}{temporalia/hym-LucisLargitor.gtex}
\grechangedim{interwordspacetext}{0.22 cm plus 0.15 cm minus 0.05 cm}{scalable}%
\vspace{-3mm}
\fi
\else
\hymnuslaudes
\fi

\vfill
\pagebreak

\ifx\laudb\undefined
\else
\pars{Psalmus 1.} \scriptura{Ps. 41, 3; \textbf{H391}}

\vspace{-4mm}

\antiphona{II D}{temporalia/ant-sitivitanima.gtex}

%\vspace{-2mm}

\scriptura{Psalmus 41}

%\vspace{-2mm}

\initiumpsalmi{temporalia/ps41-initium-ii-D-auto.gtex}

%\vspace{-1.5mm}

\input{temporalia/ps41-ii-D.tex}

\vfill

\antiphona{}{temporalia/ant-sitivitanima.gtex}

\vfill
\pagebreak

\pars{Psalmus 2.}

\vspace{-4mm}

\antiphona{III a}{temporalia/ant-ostendenobisdomine.gtex}

%\vspace{-2mm}

\scriptura{Canticum Ecclesiastici, Sir. 36, 1-7.13-16}

%\vspace{-3mm}

\initiumpsalmi{temporalia/ecclesiastici-initium-iii-a-auto.gtex}

\input{temporalia/ecclesiastici-iii-a.tex} \Abardot{}

\vfill
\pagebreak

\pars{Psalmus 3.}

\vspace{-4mm}

\antiphona{II D}{temporalia/ant-operamanuumeius.gtex}

\scriptura{Psalmus 18, 1-7}

\initiumpsalmi{temporalia/ps18i-initium-ii-D-auto.gtex}

\input{temporalia/ps18i-ii-D.tex} \Abardot{}

\vfill
\pagebreak
\fi

\ifx\lectiobrevis\undefined
\ifx\laudb\undefined
\else
\pars{Lectio Brevis.} \scriptura{Ier. 15, 16}

\noindent Invénti sunt sermónes tui, et comédi eos, et factum est mihi verbum tuum in gáudium et in lætítiam cordis mei, quóniam invocátum est nomen tuum super me, Dómine Deus exercítuum.
\fi
\else
\lectiobrevis
\fi

\vfill

\ifx\responsoriumbreve\undefined
\ifx\laudbd\undefined
\else
\pars{Responsorium breve.} \scriptura{Ps. 32, 1.3}

\cuminitiali{VI}{temporalia/resp-exsultateiusti.gtex}
\fi
\else
\responsoriumbreve
\fi

\vfill
\pagebreak

\ifx\benedictus\undefined
\ifx\laudbd\undefined
\else
\pars{Canticum Zachariæ.} \scriptura{Lc. 1, 68; \textbf{H422}}

\vspace{-4mm}

{
\grechangedim{interwordspacetext}{0.18 cm plus 0.15 cm minus 0.05 cm}{scalable}%
\antiphona{IV E}{temporalia/ant-benedictusdominus.gtex}
\grechangedim{interwordspacetext}{0.22 cm plus 0.15 cm minus 0.05 cm}{scalable}%
}

%\vspace{-3mm}

\scriptura{Lc. 1, 68-79}

%\vspace{-2mm}

\cantusSineNeumas
\initiumpsalmi{temporalia/benedictus-initium-iv-E-auto.gtex}

%\vspace{-1.5mm}

\input{temporalia/benedictus-iv-E.tex} \Abardot{}
\fi
\else
\benedictus
\fi

\vspace{-1cm}

\vfill
\pagebreak

%\sideThumbs{{\scriptsize{}Fine horarum}}

\pars{Preces.}

\sineinitiali{}{temporalia/tonusprecum.gtex}

\ifx\preces\undefined
\ifx\laudb\undefined
\else
\noindent Salvátor noster fecit nos regnum et sacerdótium, ut hóstias Deo acceptábiles offerámus. \gredagger{} Grati ígitur eum invocémus:

\Rbardot{} Serva nos in tuo ministério, Dómine.

\noindent Christe, sacérdos ætérne, qui sanctum pópulo tuo sacerdótium concessísti, \gredagger{} concéde, ut spiritáles hóstias Deo acceptábiles iúgiter offerámus.

\Rbardot{} Serva nos in tuo ministério, Dómine.

\noindent Spíritus tui fructus nobis largíre propítius, \gredagger{} patiéntiam, benignitátem et mansuetúdinem.

\Rbardot{} Serva nos in tuo ministério, Dómine.

\noindent Da nobis te amáre, ut te, qui es cáritas, possideámus, \gredagger{} et bene ágere, ut per vitam étiam nostram te laudémus.

\Rbardot{} Serva nos in tuo ministério, Dómine.

\noindent Quæ frátribus nostris sunt utília, nos quǽrere concéde, \gredagger{} ut salútem facílius consequántur.

\Rbardot{} Serva nos in tuo ministério, Dómine.
\fi
\else
\preces
\fi

\vfill

\pars{Oratio Dominica.}

\cuminitiali{}{temporalia/oratiodominicaalt.gtex}

\vfill
\pagebreak

\rubrica{vel:}

\pars{Supplicatio Litaniæ.}

\cuminitiali{}{temporalia/supplicatiolitaniae.gtex}

\vfill

\pars{Oratio Dominica.}

\cuminitiali{}{temporalia/oratiodominica.gtex}

\vfill
\pagebreak

% Oratio. %%%
\oratio

\vspace{-1mm}

\vfill

\rubrica{Hebdomadarius dicit Dominus vobiscum, vel, absente sacerdote vel diacono, sic concluditur:}

\vspace{2mm}

\antiphona{C}{temporalia/dominusnosbenedicat.gtex}

\rubrica{Postea cantatur a cantore:}

\vspace{2mm}

\cuminitiali{IV}{temporalia/benedicamus-feria-laudes.gtex}

\vspace{1mm}

\vfill
\pagebreak

\end{document}

