\newcommand{\titulus}{\nomenFesti{Beatæ Mariæ Virginis de Guadalupe.}
\dies{Die 12. Decembris.}}
\newcommand{\oratio}{\pars{Oratio.}

\noindent Deus, Pater misericordiárum, qui sub Sanctíssimæ Matris Fílii tui singulári patrocínio plebem tuam constituísti, tríbue cunctis, qui Beátam Vírginem Guadalupénsem ínvocant, ut, alacrióri fide, populórum progressiónem in viis iustítiæ quǽreant et pacis.

\vfill

\pars{Pro pace in universo mundo.} \scriptura{Sir. 50, 25; 2 Esdr. 4, 20; \textbf{H416}}

\vspace{-4mm}

\antiphona{II D}{temporalia/ant-dapacemdomine.gtex}

\vfill

\noindent Deus, a quo sancta desidéria, recta consília et iusta sunt ópera: da servis tuis illam, quam mundus dare non potest, pacem; ut et corda nostra mandátis tuis dédita, et hóstium subláta formídine, témpora sint tua protectióne tranquílla.

\noindent Per Dóminum nostrum Iesum Christum, Fílium tuum, qui tecum vivit et regnat in unitáte Spíritus Sancti, Deus, per ómnia sǽcula sæculórum.

\noindent \Rbardot{} Amen.}
\newcommand{\invitatorium}{\pars{Invitatorium.}

\vspace{-4mm}

\antiphona{V}{temporalia/inv-christummariaefilium.gtex}}
\newcommand{\hymnusmatutinum}{\pars{Hymnus}

\cuminitiali{VIII}{temporalia/hym-QuemTerra-simplex.gtex}}
\newcommand{\matversus}{\noindent \Vbardot{} Osténde nobis, Dómine, misericórdiam tuam.

\noindent \Rbardot{} Et salutáre tuum da nobis.}
\newcommand{\absolutio}{\cuminitiali{}{temporalia/absolutio-precibus.gtex}}
\newcommand{\benedictioi}{\cuminitiali{}{temporalia/benedictio-solemn-noscum.gtex}}
\newcommand{\benedictioii}{\cuminitiali{}{temporalia/benedictio-solemn-ipsavirgo.gtex}}
\newcommand{\benedictioiii}{\cuminitiali{}{temporalia/benedictio-solemn-pervirginem.gtex}}
\newcommand{\lectioi}{\pars{Lectio I.} \scriptura{Is. 30, 18-26}

\noindent De libro Isaíæ prophétæ.

\noindent Exspéctat Dóminus, ut misereátur vestri, et ídeo exaltábitur parcens vobis, quia Deus iudícii Dóminus; beáti omnes, qui exspéctant eum.

\noindent Nam, pópule Sion, qui hábitas in Ierúsalem, plorans nequáquam plorábis: míserans miserébitur tui ad vocem clamóris tui; statim ut audíerit, respondébit tibi. Et dabit vobis Dóminus panem angústiæ et aquam afflictiónis, sed non ámplius avolábit a te doctor tuus; et erunt óculi tui vidéntes præceptórem tuum, et aures tuæ áudient verbum post tergum monéntis: «Hæc via, ambuláte in ea», si declinavéritis ad déxteram vel ad sinístram. Et contaminábis láminas sculptílium argentórum tuórum et vestiméntum conflátilis áurei tui; dispérges ea sicut immundítiam menstruátæ: «Egrédere» dices ei.

\noindent Et dabit plúviam sémini tuo, quod semináveris in terra, et panis frugum terræ erit ubérrimus et pinguis; pascétur pecus tuum in die illo, agnus in páscuis spatiósis, et boves tui et ásini, qui operántur terram, commíxtum migma cómedent ventilátum in pala et ventilábro. Et erunt super omnem montem excélsum et super omnem collem elevátum rivi curréntium aquárum in die interfectiónis multórum, cum cecíderint turres.

\noindent Et erit lux lunæ sicut lux solis et lux solis erit septemplíciter sicut lux septem diérum in die, qua alligáverit Dóminus vulnus pópuli sui et percussúram plagæ eius sanáverit.}
\newcommand{\responsoriumi}{\pars{Responsorium 1.} \scriptura{\Rbar{} Ps. 71, 5-6 \Vbar{} Ps. 106, 3; \textbf{H29}}

\vspace{-5mm}

\responsorium{IV}{temporalia/resp-descendetdominus-CROCHU.gtex}{}}
\newcommand{\lectioii}{\pars{Lectio II.} \scriptura{sæc. XVI, ex archivo Archidiœcesis Mexicopolitanæ}

\noindent Ex trádita relatióne, quæ «Nican Mopohua» nuncupátur.

\noindent {\color{gray} Anno MDXXXI, post dies áliquot mensis decémbris, cum esset quidam indus pauper et affábilis, cui nomen Ioánnes Dídacus, ut fertur, ex \emph{Cuauhtitlan,} cuius cura, quoad spirituálem administratiónem, ad religiósos in \emph{Tlatilolco} residéntes pertinébat, die sábbato, valde mane, \emph{Tlatilolco} rem divínam ille adíbat. Ut autem ad collem \emph{Tepeyac} dictum advénit, iam illucescébat. Cantum ergo supra collem audívit. Ut vero cantus cessávit, nec iam fuit ámplius áditus, vocátum se audívit e superióre parte collis: «Dilécte, Ioánnes Dídace», dictum est ei. Statim illuc ausus est ascéndere, unde se vocári cognóvit.}

\noindent Ut autem [Ioánnes Dídacus] supra collem advénit, dóminam vidit stantem, quæ illum, ut ipse adíret, vocávit. Cum ante illam pervénit, valde mirátus est quantum esset decóra: vestis eius sicut sol effulgébat. Illico voluntátem suam illi Virgo declarávit. Ait illi: «Scito, dilectíssime fili, Sanctam Maríam me esse, perféctam semper Vírginem, Matrem veríssimi Dei, vitæ Auctóris, qui ómnia creávit et sústinet, Dómini cæli et terræ. Magnópere volo, ardénter desídero, ut isto in loco templum meum ædificétur, ubi eum osténdam, eum maniféstans laudábo, meum amórem ac pietátem, auxílium et defensiónem impértiam, quóniam revéra ego clemens Mater vestra sum, et tua et ómnium qui hac in terra in unum consistétis et aliórum quorumcúmque qui díligunt me, qui me quærunt, qui devóte et confidénter me invocáverint. Ibi lácrimas ac mæstítiam eórum exáudiam, in angústiis benefáciam et in omni tribulatióne remédium áfferam. Ut autem meum desidérium adimpleátur, Mexicópolim adi in palátium epíscopi. Te a me missum dices illi, ut ipsum scire fácias quómodo mihi domum hic volo ædificári, templum hic in valle mihi érigi».}
\newcommand{\responsoriumii}{\pars{Responsorium 2.} \scriptura{\Vbardot{} Ct. 3, 6; \textbf{H296}}

%\vspace{-5mm}

\responsorium{III}{temporalia/resp-vidispeciosam-sinedox.gtex}{}}
\newcommand{\lectioiii}{\pars{Lectio III.}

\noindent Ut pervénit intra civitátem, statim domum adívit epíscopi, cui nomen Ioánnes de Zumárraga, ordinis sancti Francísci. Ut autem antístes Ioánnem Dídacum audívit, quasi non omníno credens, illis respóndit: «Fili, íterum vénies et adhuc áudiam te. Ego autem mihi cogitábo quid fácere opórteat de tua voluntáte et desidério».

\noindent Altera die, vidit ergo Regínam de colle descendéntem unde ipsum aspiciébat. Quæ venit óbviam illi prope collem, eum detínuit dixítque: «Audi, dilécte fili: nullátenus tímeas neque corde dóleas, nec áliquid fácias tui avúnculi infirmitátem aut quámlibet angústiam. Numquid hic non adsum ego Mater tua? Numquid non sub umbra et protectióne mea tu es constitútus? Numquid ego non sum fons tua vitæ et felicitátis? Numquid tu non in meo grémio, in bráchiis meis subsístis? Numquid áliud quodcúmque tibi necésse est? Nihil dóleas, nec turbéris. Ascénde, inquit, dilécte fili, supra collem atque in eo loco, ubi me vidísti et tibi locúta sum, flores ibi divérsas vidébis. Accipe et cóllige illas atque inde descéndens affer illas coram me».

\noindent Descéndit ergo Ioánnes atque cæli regínæ détulit, quas collégerat flores. Illa autem, ut eos vidit, suis venerabílibus mánibus illos accépit rursúmque in Ioánnis pallíolo collocávit dixítque illi: «Fili dilectíssime, hi flores signum, quod déferes ad epíscopum, sunt. Eh, tu meus núntius es, cuius fidelitáti hæc commítto. Te rigoróse præcípio: cáveas ne pallíolum tuum, nisi coram epíscopo, éxplices et, quæ defers, illi osténdas. Narrábis quoque quómodo, ut collem ascénderes et inde flores accípere, tibi præcépi et quidquid vidísti et admirátus es, ut credat et agat de templo erigéndo quod volo».

\noindent {\color{gray} Ut ergo hæc præcépit cæli regína, iter arrípuit mexicópolim versus. Lætus ibat, quia ómnia próspere fient. Ingréssus autem Ioánnes, coram epíscopo se prostrávit atque illi narrávit quæcúmque víderat et ad quid ad ipsum missus erat. Dixit illi: «Dómine, mihi quæ præcepísti adimplévi. Dictúrus adívi Dóminam meam, cæli regínam, Sanctam Maríam Dei Genetrícem, te signum pétere ad mihi credéndum atque ut templum ibi éxstruas ubi ipsa Virgo desíderat. Dixi ergo illi me signum áliquod eius voluntátis ad te afférre promisísse. Audívit ergo quæ tu expéteres: benígne tulit te signum pétere ad impléndam voluntátem eius atque hódie, valde mane, me rursus ad te veníre præcépit».

\noindent Occúrrit ergo univérsa cívitas: venerábilem imáginem vidébant, mirabántur, ut opus divínum eam mirábant, deprecabántur. Et die illa dixit avúnculus Ioánnis Dídaci necnon quæ sit Vírginis advocátio et quod eius imágo nuncupétur Sanctæ Maríæ semper Vírginis de Guadalúpe.}}
\newcommand{\responsoriumiii}{\pars{Responsorium 3.} \scriptura{\Rbardot{} Ct. 6, 9 \Vbardot{} ibid. 3, 6; \textbf{H297}}

\vspace{-5mm}

\responsorium{IV}{temporalia/resp-quaeestista-CROCHU-cumdox.gtex}{}}
\newcommand{\hymnuslaudes}{\pars{Hymnus.}

\cuminitiali{II}{temporalia/hym-OGloriosaDomina-praglia-lh.gtex}}
\newcommand{\lectiobrevis}{\pars{Lectio brevis.} \scriptura{Cf. Is. 61, 10}

\noindent Gaudens gaudébo in Dómino, et exsultábit ánima mea in Deo meo, quia índuit me vestiméntis salútis et induménto iustítiæ circúmdedit me, quasi sponsam ornátam monílibus suis.}
\newcommand{\responsoriumbreve}{\pars{Responsorium breve.} \scriptura{Ps. 44, 3}

\antiphona{VI}{temporalia/resp-diffusaest.gtex}}
\newcommand{\benedictus}{\pars{Canticum Zachariæ.} \scriptura{Ct. 7, 4; \textbf{H299}}

\vspace{-4mm}

\antiphona{VII b}{temporalia/ant-oculituisanctadeigenetrix.gtex}

\vspace{-2mm}

\scriptura{Lc. 1, 68-79}

%\vspace{-2mm}

\cantusSineNeumas
\initiumpsalmi{temporalia/benedictus-initium-vii-b-auto.gtex}

%\vspace{-1.5mm}

\input{temporalia/benedictus-vii-b.tex} \Abardot{}}
\newcommand{\preces}{\noindent Salvatórem nostrum celebrántes,~\gredagger{} qui ex María Vírgine nasci dignátus est,~\grestar{} exorémus dicéntes:

\Rbardot{} Intercédat pro nobis mater tua, Dómine.

\noindent Salvátor mundi,~\gredagger{} qui redemptiónis tuæ virtúte ab omni peccáti labe matrem tuam præservásti,~\grestar{} serva nos mundos a peccáto.

\Rbardot{} Intercédat pro nobis mater tua, Dómine.

\noindent Redémptor noster,~\gredagger{} qui Vírginem Maríam thálamum puríssimum habitatiónis tuæ et Spíritus Sancti fecísti sacrárium,~\grestar{} nos templum fac perénne tui Spíritus.

\Rbardot{} Intercédat pro nobis mater tua, Dómine.

\noindent Verbum ætérnum,~\gredagger{} qui matrem tuam docuísti óptimam sibi partem elígere,~\grestar{} tríbue nobis eam imitári, cibum quæréntes, qui permáneat in vitam ætérnam.

\Rbardot{} Intercédat pro nobis mater tua, Dómine.

\noindent Rex regum,~\gredagger{} qui matrem tuam córpore et ánima tecum voluísti in cælum assúmptam,~\grestar{} fac ut quæ sursum sunt semper cogitémus.

\Rbardot{} Intercédat pro nobis mater tua, Dómine.

\noindent Dómine cæli et terræ,~\gredagger{} qui Maríam regínam a dextris tuis adstáre fecísti,~\grestar{} tríbue nos eiúsdem glóriæ meréri consórtium.

\Rbardot{} Intercédat pro nobis mater tua, Dómine.}
\newcommand{\magnificat}{\pars{Canticum B. Mariæ V.} \scriptura{Lc. 1, 48; \textbf{H30}}

\vspace{-4mm}

{
\grechangedim{interwordspacetext}{0.18 cm plus 0.15 cm minus 0.05 cm}{scalable}%
\antiphona{VIII G}{temporalia/ant-beatammedicent.gtex}
\grechangedim{interwordspacetext}{0.22 cm plus 0.15 cm minus 0.05 cm}{scalable}%
}

%\vspace{-3mm}

\scriptura{Lc. 1, 46-55}

%\vspace{-2mm}

\cantusSineNeumas

\initiumpsalmi{temporalia/magnificat-initium-viii-G.gtex}

%\vspace{-2mm}

\input{temporalia/magnificat-viii-G.tex} \Abardot{}

\vspace{-1cm}}
\newcommand{\benedicamuslaudes}{\cuminitiali{I}{temporalia/benedicamus-festis-bmv.gtex}}
\newcommand{\hebdomada}{infra Hebdom. III Adventus.}
\newcommand{\oratioLaudes}{\cuminitiali{}{temporalia/oratio3vo.gtex}}
\newcommand{\responsoriumbreve}{\pars{Responsorium breve.} \scriptura{Is. 60, 2; \textbf{H20}}

\cuminitiali{IV}{temporalia/resp-superte.gtex}}

% LuaLaTeX

\documentclass[a4paper, twoside, 12pt]{article}
\usepackage[latin]{babel}
%\usepackage[landscape, left=3cm, right=1.5cm, top=2cm, bottom=1cm]{geometry} % okraje stranky
%\usepackage[landscape, a4paper, mag=1166, truedimen, left=2cm, right=1.5cm, top=1.6cm, bottom=0.95cm]{geometry} % okraje stranky
\usepackage[landscape, a4paper, mag=1400, truedimen, left=0.5cm, right=0.5cm, top=0.5cm, bottom=0.5cm]{geometry} % okraje stranky

\usepackage{fontspec}
\setmainfont[FeatureFile={junicode.fea}, Ligatures={Common, TeX}, RawFeature=+fixi]{Junicode}
%\setmainfont{Junicode}

% shortcut for Junicode without ligatures (for the Czech texts)
\newfontfamily\nlfont[FeatureFile={junicode.fea}, Ligatures={Common, TeX}, RawFeature=+fixi]{Junicode}

\usepackage{multicol}
\usepackage{color}
\usepackage{lettrine}
\usepackage{fancyhdr}

% usual packages loading:
\usepackage{luatextra}
\usepackage{graphicx} % support the \includegraphics command and options
\usepackage{gregoriotex} % for gregorio score inclusion
\usepackage{gregoriosyms}
\usepackage{wrapfig} % figures wrapped by the text
\usepackage{parcolumns}
\usepackage[contents={},opacity=1,scale=1,color=black]{background}
\usepackage{tikzpagenodes}
\usepackage{calc}
\usepackage{longtable}
\usetikzlibrary{calc}

\setlength{\headheight}{14.5pt}

\input{conventuscommune.tex} % Often used macros

\newcommand{\annusEditionis}{2021}

%%%% Vicekrat opakovane kousky

\newcommand{\anteOrationem}{
  \rubrica{Ante Orationem, cantatur a Superiore:}

  \pars{Supplicatio Litaniæ.}

  \cuminitiali{}{temporalia/supplicatiolitaniae.gtex}

  \pars{Oratio Dominica.}

  \cuminitiali{}{temporalia/oratiodominica.gtex}

  \rubrica{Deinde dicitur ab Hebdomadario:}

  \cuminitiali{}{temporalia/dominusvobiscum-solemnis.gtex}

  \rubrica{In choro monialium loco Dominus vobiscum dicitur:}

  \sineinitiali{temporalia/domineexaudi.gtex}
}

\setlength{\columnsep}{30pt} % prostor mezi sloupci

%%%%%%%%%%%%%%%%%%%%%%%%%%%%%%%%%%%%%%%%%%%%%%%%%%%%%%%%%%%%%%%%%%%%%%%%%%%%%%%%%%%%%%%%%%%%%%%%%%%%%%%%%%%%%
\begin{document}

% Here we set the space around the initial.
% Please report to http://home.gna.org/gregorio/gregoriotex/details for more details and options
\grechangedim{afterinitialshift}{2.2mm}{scalable}
\grechangedim{beforeinitialshift}{2.2mm}{scalable}
\grechangedim{interwordspacetext}{0.22 cm plus 0.15 cm minus 0.05 cm}{scalable}%
\grechangedim{annotationraise}{-0.2cm}{scalable}

% Here we set the initial font. Change 38 if you want a bigger initial.
% Emit the initials in red.
\grechangestyle{initial}{\color{red}\fontsize{38}{38}\selectfont}

\pagestyle{empty}

%%%% Titulni stranka
\begin{titulusOfficii}
\ifx\titulus\undefined
\nomenFesti{Feria II \hebdomada{}}
\else
\titulus
\fi
\end{titulusOfficii}

\vfill

\begin{center}
%Ad usum et secundum consuetudines chori \guillemotright{}Conventus Choralis\guillemotleft.

%Editio Sancti Wolfgangi \annusEditionis
\end{center}

\scriptura{}

\pars{}

\pagebreak

\renewcommand{\headrulewidth}{0pt} % no horiz. rule at the header
\fancyhf{}
\pagestyle{fancy}

\cantusSineNeumas

\ifx\oratio\undefined
\ifx\laudb\undefined
\else
\newcommand{\oratio}{\pars{Oratio.}

\noindent Dómine Deus omnípotens, qui ad princípium huius diéi nos perveníre fecísti, tua nos hódie salva virtúte, ut in hac die ad nullum declinémus peccátum, sed semper ad tuam iustítiam faciéndam nostra procédant elóquia, dirigántur cogitatiónes et ópera.

\noindent Per Dóminum nostrum Iesum Christum, Fílium tuum, qui tecum vivit et regnat in unitáte Spíritus Sancti, Deus, per ómnia sǽcula sæculórum.

\noindent \Rbardot{} Amen.}
\fi
\fi

\hora{Ad Matutinum.} %%%%%%%%%%%%%%%%%%%%%%%%%%%%%%%%%%%%%%%%%%%%%%%%%%%%%
%\sideThumbs{Matutinum}

\vspace{2mm}

\cuminitiali{}{temporalia/dominelabiamea.gtex}

\vfill
%\pagebreak

\vspace{2mm}

\ifx\invitatorium\undefined
\pars{Invitatorium.} \scriptura{Ps. 94, 1; Psalmus 94; \textbf{H451}}

\vspace{-6mm}

\antiphona{VI}{temporalia/inv-jubilemusdeo.gtex}\else
\invitatorium
\fi

\vfill
\pagebreak

\ifx\hymnusmatutinum\undefined
\ifx\matua\undefined
\else
\pars{Hymnus.}

{
\grechangedim{interwordspacetext}{0.10 cm plus 0.15 cm minus 0.05 cm}{scalable}%
\antiphona{II}{temporalia/hym-IpsumNunc.gtex}
\grechangedim{interwordspacetext}{0.22 cm plus 0.15 cm minus 0.05 cm}{scalable}%
}
\fi
\else
\hymnusmatutinum
\fi

\vspace{-3mm}

\vfill
\pagebreak

\ifx\matub\undefined
\else
% MAT B
\pars{Psalmus 1.} \scriptura{Ps. 30, 2; \textbf{H90}}

\vspace{-4mm}

\antiphona{VIII G}{temporalia/ant-intuaiustitia.gtex}

%\vspace{-2mm}

\scriptura{Ps. 30, 2-9}

%\vspace{-2mm}

\initiumpsalmi{temporalia/ps30i-initium-viii-G-auto.gtex}

\vspace{-1.5mm}

\input{temporalia/ps30i-viii-G.tex} \Abardot{}

\vfill
\pagebreak

\pars{Psalmus 2.} \scriptura{Ps. 66, 2}

\vspace{-4mm}

\antiphona{E}{temporalia/ant-illuminadomine.gtex}

%\vspace{-2mm}

\scriptura{Ps. 30, 10-17}

%\vspace{-2mm}

\initiumpsalmi{temporalia/ps30ii-initium-e-a-auto.gtex}

\input{temporalia/ps30ii-e-a.tex} \Abardot{}

\vfill
\pagebreak

\pars{Psalmus 3.} \scriptura{Ps. 30, 24}

\vspace{-4mm}

\antiphona{II D}{temporalia/ant-diligitedominum.gtex}

%\vspace{-5mm}

\scriptura{Ps. 30, 20-25}

%\vspace{-2mm}

\initiumpsalmi{temporalia/ps30iii-initium-ii-D-auto.gtex}

\input{temporalia/ps30iii-ii-D.tex} \Abardot{}

\vfill
\pagebreak
\fi

\pars{Versus.}

\ifx\matversus\undefined
\ifx\matub\undefined
\else
\noindent \Vbardot{} Dírige me, Dómine, in veritáte tua, et doce me.

\noindent \Rbardot{} Quia tu es Deus salútis meæ.
\fi
\else
\matversus
\fi

\vspace{5mm}

\sineinitiali{temporalia/oratiodominica-mat.gtex}

\vspace{5mm}

\pars{Absolutio.}

\cuminitiali{}{temporalia/absolutio-exaudi.gtex}

\vfill
\pagebreak

\cuminitiali{}{temporalia/benedictio-solemn-benedictione.gtex}

\vspace{7mm}

\lectioi

\noindent \Vbardot{} Tu autem, Dómine, miserére nobis.
\noindent \Rbardot{} Deo grátias.

\vfill
\pagebreak

\responsoriumi

\vfill
\pagebreak

\cuminitiali{}{temporalia/benedictio-solemn-unigenitus.gtex}

\vspace{7mm}

\lectioii

\noindent \Vbardot{} Tu autem, Dómine, miserére nobis.
\noindent \Rbardot{} Deo grátias.

\vfill
\pagebreak

\responsoriumii

\vfill
\pagebreak

\cuminitiali{}{temporalia/benedictio-solemn-spiritus.gtex}

\vspace{7mm}

\lectioiii

\noindent \Vbardot{} Tu autem, Dómine, miserére nobis.
\noindent \Rbardot{} Deo grátias.

\vfill
\pagebreak

\responsoriumiii

\vfill
\pagebreak

\rubrica{Reliqua omittuntur, nisi Laudes separandæ sint.}

\sineinitiali{temporalia/domineexaudi.gtex}

\vfill

\oratio

\vfill

\noindent \Vbardot{} Dómine, exáudi oratiónem meam.
\Rbardot{} Et clamor meus ad te véniat.

\vfill

\noindent \Vbardot{} Benedicámus Dómino.
\noindent \Rbardot{} Deo grátias.

\vfill

\noindent \Vbardot{} Fidélium ánimæ per misericórdiam Dei requiéscant in pace.
\Rbardot{} Amen.

\vfill
\pagebreak

\hora{Ad Laudes.} %%%%%%%%%%%%%%%%%%%%%%%%%%%%%%%%%%%%%%%%%%%%%%%%%%%%%
%\sideThumbs{Laudes}

\cantusSineNeumas

\vspace{0.5cm}
\grechangedim{interwordspacetext}{0.18 cm plus 0.15 cm minus 0.05 cm}{scalable}%
\cuminitiali{}{temporalia/deusinadiutorium-communis.gtex}
\grechangedim{interwordspacetext}{0.22 cm plus 0.15 cm minus 0.05 cm}{scalable}%

\vfill
\pagebreak

\ifx\hymnuslaudes\undefined
\ifx\laudbd\undefined
\else
\pars{Hymnus} \scriptura{Hilarius (\olddag{} 367)}

\grechangedim{interwordspacetext}{0.16 cm plus 0.15 cm minus 0.05 cm}{scalable}%
\cuminitiali{IV}{temporalia/hym-LucisLargitor.gtex}
\grechangedim{interwordspacetext}{0.22 cm plus 0.15 cm minus 0.05 cm}{scalable}%
\vspace{-3mm}
\fi
\else
\hymnuslaudes
\fi

\vfill
\pagebreak

\ifx\laudb\undefined
\else
\pars{Psalmus 1.} \scriptura{Ps. 41, 3; \textbf{H391}}

\vspace{-4mm}

\antiphona{II D}{temporalia/ant-sitivitanima.gtex}

%\vspace{-2mm}

\scriptura{Psalmus 41}

%\vspace{-2mm}

\initiumpsalmi{temporalia/ps41-initium-ii-D-auto.gtex}

%\vspace{-1.5mm}

\input{temporalia/ps41-ii-D.tex}

\vfill

\antiphona{}{temporalia/ant-sitivitanima.gtex}

\vfill
\pagebreak

\pars{Psalmus 2.}

\vspace{-4mm}

\antiphona{III a}{temporalia/ant-ostendenobisdomine.gtex}

%\vspace{-2mm}

\scriptura{Canticum Ecclesiastici, Sir. 36, 1-7.13-16}

%\vspace{-3mm}

\initiumpsalmi{temporalia/ecclesiastici-initium-iii-a-auto.gtex}

\input{temporalia/ecclesiastici-iii-a.tex} \Abardot{}

\vfill
\pagebreak

\pars{Psalmus 3.}

\vspace{-4mm}

\antiphona{II D}{temporalia/ant-operamanuumeius.gtex}

\scriptura{Psalmus 18, 1-7}

\initiumpsalmi{temporalia/ps18i-initium-ii-D-auto.gtex}

\input{temporalia/ps18i-ii-D.tex} \Abardot{}

\vfill
\pagebreak
\fi

\ifx\lectiobrevis\undefined
\ifx\laudb\undefined
\else
\pars{Lectio Brevis.} \scriptura{Ier. 15, 16}

\noindent Invénti sunt sermónes tui, et comédi eos, et factum est mihi verbum tuum in gáudium et in lætítiam cordis mei, quóniam invocátum est nomen tuum super me, Dómine Deus exercítuum.
\fi
\else
\lectiobrevis
\fi

\vfill

\ifx\responsoriumbreve\undefined
\ifx\laudbd\undefined
\else
\pars{Responsorium breve.} \scriptura{Ps. 32, 1.3}

\cuminitiali{VI}{temporalia/resp-exsultateiusti.gtex}
\fi
\else
\responsoriumbreve
\fi

\vfill
\pagebreak

\ifx\benedictus\undefined
\ifx\laudbd\undefined
\else
\pars{Canticum Zachariæ.} \scriptura{Lc. 1, 68; \textbf{H422}}

\vspace{-4mm}

{
\grechangedim{interwordspacetext}{0.18 cm plus 0.15 cm minus 0.05 cm}{scalable}%
\antiphona{IV E}{temporalia/ant-benedictusdominus.gtex}
\grechangedim{interwordspacetext}{0.22 cm plus 0.15 cm minus 0.05 cm}{scalable}%
}

%\vspace{-3mm}

\scriptura{Lc. 1, 68-79}

%\vspace{-2mm}

\cantusSineNeumas
\initiumpsalmi{temporalia/benedictus-initium-iv-E-auto.gtex}

%\vspace{-1.5mm}

\input{temporalia/benedictus-iv-E.tex} \Abardot{}
\fi
\else
\benedictus
\fi

\vspace{-1cm}

\vfill
\pagebreak

%\sideThumbs{{\scriptsize{}Fine horarum}}

\pars{Preces.}

\sineinitiali{}{temporalia/tonusprecum.gtex}

\ifx\preces\undefined
\ifx\laudb\undefined
\else
\noindent Salvátor noster fecit nos regnum et sacerdótium, ut hóstias Deo acceptábiles offerámus. \gredagger{} Grati ígitur eum invocémus:

\Rbardot{} Serva nos in tuo ministério, Dómine.

\noindent Christe, sacérdos ætérne, qui sanctum pópulo tuo sacerdótium concessísti, \gredagger{} concéde, ut spiritáles hóstias Deo acceptábiles iúgiter offerámus.

\Rbardot{} Serva nos in tuo ministério, Dómine.

\noindent Spíritus tui fructus nobis largíre propítius, \gredagger{} patiéntiam, benignitátem et mansuetúdinem.

\Rbardot{} Serva nos in tuo ministério, Dómine.

\noindent Da nobis te amáre, ut te, qui es cáritas, possideámus, \gredagger{} et bene ágere, ut per vitam étiam nostram te laudémus.

\Rbardot{} Serva nos in tuo ministério, Dómine.

\noindent Quæ frátribus nostris sunt utília, nos quǽrere concéde, \gredagger{} ut salútem facílius consequántur.

\Rbardot{} Serva nos in tuo ministério, Dómine.
\fi
\else
\preces
\fi

\vfill

\pars{Oratio Dominica.}

\cuminitiali{}{temporalia/oratiodominicaalt.gtex}

\vfill
\pagebreak

\rubrica{vel:}

\pars{Supplicatio Litaniæ.}

\cuminitiali{}{temporalia/supplicatiolitaniae.gtex}

\vfill

\pars{Oratio Dominica.}

\cuminitiali{}{temporalia/oratiodominica.gtex}

\vfill
\pagebreak

% Oratio. %%%
\oratio

\vspace{-1mm}

\vfill

\rubrica{Hebdomadarius dicit Dominus vobiscum, vel, absente sacerdote vel diacono, sic concluditur:}

\vspace{2mm}

\antiphona{C}{temporalia/dominusnosbenedicat.gtex}

\rubrica{Postea cantatur a cantore:}

\vspace{2mm}

\cuminitiali{IV}{temporalia/benedicamus-feria-laudes.gtex}

\vspace{1mm}

\vfill
\pagebreak

\end{document}

