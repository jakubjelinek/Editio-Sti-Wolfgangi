\newcommand{\lectioi}{\pars{Lectio I.} \scriptura{Is. 49, 14-20}

\noindent De libro Isaíæ prophétæ.

\noindent Dixit Sion: «Derelíquit me Dóminus, et Dóminus oblítus est mei». Numquid oblivísci potest múlier infántem suum, ut non misereátur fílio úteri sui? Et si illa oblíta fúerit, ego tamen non oblivíscar tui. Ecce in mánibus meis descrípsi te; muri tui coram me semper. Festínant structóres tui; destruéntes te et dissipántes a te exíbunt. Leva in circúitu óculos tuos et vide: omnes isti congregáti sunt, venérunt tibi. «Vivo ego, dicit Dóminus, quia ómnibus his velut ornaménto vestiéris et circúmdabis tibi eos quasi sponsa». Quia ruínæ tuæ et solitúdines tuæ et terra evérsa: nunc angústa eris præ habitatóribus; et longe erunt, qui devorábant te. Adhuc dicent in áuribus tuis fílii orbitátis tuæ: «Angústus est mihi locus; fac spátium mihi, ut hábitem».}
\newcommand{\responsoriumi}{\pars{Responsorium 1.} \scriptura{\textbf{H34}}

\vspace{-5mm}

\responsorium{II}{temporalia/resp-festina-CROCHU.gtex}{}}
\newcommand{\lectioii}{\pars{Lectio II.} \scriptura{Is. 49, 21-23}

\noindent Et dices in corde tuo: «Quis génuit mihi istos? Ego orbáta et non páriens, transmigráta et captíva; et istos quis enutrívit? Ecce ego relícta eram sola; et isti ubi erant?». Hæc dicit Dóminus Deus: «Ecce levábo ad gentes manum meam et ad pópulos exaltábo signum meum; et áfferent fílios tuos in ulnis, et fíliæ tuæ super úmeros portabúntur. Et erunt reges nutrícii tui, et regínæ nutríces tuæ; vultu in terram demísso adorábunt te et púlverem pedum tuórum lingent.}
\newcommand{\responsoriumii}{\pars{Responsorium 2.} \scriptura{\Rbar{} Cf. Is. 35, 2 \Vbar{} Is. 40, 10; \textbf{H35}}

\vspace{-5mm}

\responsorium{I}{temporalia/resp-germinaveruntcampi-CROCHU.gtex}{}}
\newcommand{\lectioiii}{\pars{Lectio III.} \scriptura{Is. 49, 23-26; 50, 1}

\noindent Et scies quia ego Dóminus: non confundéntur, qui sperant in me». Numquid tollétur a forti præda, aut, quod captum fúerit, a robústo salvári póterit? Quia hæc dicit Dóminus: «Equidem et captívus a forti tollétur, et, quod ablátum fúerit a robústo, salvábitur; cum his, qui contendébant tecum, ego conténdam et fílios tuos ego salvábo. Et cibábo hostes tuos cárnibus suis, et quasi musto sánguine suo inebriabúntur; et sciet omnis caro quia ego Dóminus salvátor tuus et redémptor tuus Fortis Iacob». Hæc dicit Dóminus: «Ubinam est liber repúdii matris vestræ, quo dimísi eam? Aut quis est créditor meus, cui véndidi vos? Ecce in iniquitátibus vestris vénditi estis, et in sceléribus vestris dimíssa est mater vestra».}
\newcommand{\responsoriumiii}{\pars{Responsorium 3.} \scriptura{\Rbar{} Cantor \Vbar{} Lc. 1, 28; \textbf{H36}}

\vspace{-5mm}

\responsorium{I}{temporalia/resp-annuntiatumest-CROCHU-cumdox.gtex}{}}
\newcommand{\lectiobrevis}{\pars{Lectio Brevis.} \scriptura{Is. 45, 8}

\noindent Roráte, cæli, désuper, et nubes pluant iustítiam; aperiátur terra et gérminet salvatiónem; et iustítia oriátur simul.}
\newcommand{\benedictus}{\pars{Canticum Zachariæ.} \scriptura{Lc. 21, 31-32; \textbf{H43}}

\vspace{-5mm}

{
\grechangedim{interwordspacetext}{0.18 cm plus 0.15 cm minus 0.05 cm}{scalable}%
\antiphona{VIII G}{temporalia/ant-scitotequiapropeest.gtex}
\grechangedim{interwordspacetext}{0.22 cm plus 0.15 cm minus 0.05 cm}{scalable}%
}

%\trAntIMagnificat

%\vspace{-3mm}

\scriptura{Lc. 1, 68-79}

%\vspace{-1mm}

\cantusSineNeumas
\initiumpsalmi{temporalia/benedictus-initium-viii-G-auto.gtex}

\input{temporalia/benedictus-viii-G.tex} \Abardot{}}
\newcommand{\preces}{\noindent Christum redemptórem exorémus, fratres dilectíssimi, qui venit ut nos grátia advéntus sui iustificáret,~\gredagger{} vocémque cum iúbilo innovémus:

\Rbardot{} Veni, Dómine Iesu.

\noindent Qui olim prophetárum vaticínio in carne prædíctus es nascitúrus,~\gredagger{} nascéntia virtútum in nos corróbora.

\Rbardot{} Veni, Dómine Iesu.

\noindent Præsta nobis ut, qui tuam prædicámus salútem,~\gredagger{} in te salvatiónem habeámus.

\Rbardot{} Veni, Dómine Iesu.

\noindent Qui venísti contrítis corde medéri,~\gredagger{} pópuli tui sana languóres.

\Rbardot{} Veni, Dómine Iesu.

\noindent Qui cum venísti reconciliáre dignátus es mundum,~\gredagger{} ad iudícium véniens ab omni nos pœnárum líbera cruciátu.

\Rbardot{} Veni, Dómine Iesu.}
\newcommand{\oratio}{\pars{Oratio.}

\noindent Deus, qui hóminem delápsum in mortem conspíciens, Unigéniti tui advéntu redímere voluísti, præsta, quǽsumus,~\gredagger{} ut qui húmili eius incarnatiónem devotióne faténtur,~\grestar{} ipsíus étiam Redemptóris consórtia mereántur.

\noindent Per Dóminum nostrum Iesum Christum, Fílium tuum, qui tecum vivit et regnat in unitáte Spíritus Sancti, Deus, per ómnia sǽcula sæculórum.

\noindent \Rbardot{} Amen.}
\newcommand{\magnificat}{\pars{Canticum B. Mariæ V.} \scriptura{Ier. 10, 7; Ag. 2, 8; Eph. 2, 20.14; \textbf{H40}}

\vspace{-6.5mm}

{
\grechangedim{interwordspacetext}{0.18 cm plus 0.15 cm minus 0.05 cm}{scalable}%
\antiphona{II D}{temporalia/ant-orex.gtex}
\grechangedim{interwordspacetext}{0.22 cm plus 0.15 cm minus 0.05 cm}{scalable}%
}

%\trAntIMagnificat

\vspace{-3mm}

\scriptura{Lc. 1, 46-55}

\vspace{-2mm}

\cantusSineNeumas

\initiumpsalmi{temporalia/magnificat-initium-iisoll-D.gtex}

\vspace{-1.5mm}

\input{temporalia/magnificat-iisoll-D.tex} \Abardot{}

\vspace{-1cm}}
\newcommand{\hebdomada}{infra Hebdom. IV post Pentecosten.}
\newcommand{\oratioLaudes}{\cuminitiali{}{temporalia/oratio4.gtex}}

% LuaLaTeX

\documentclass[a4paper, twoside, 12pt]{article}
\usepackage[latin]{babel} 
%\usepackage[landscape, left=3cm, right=1.5cm, top=2cm, bottom=1cm]{geometry} % okraje stranky
%\usepackage[landscape, a4paper, mag=1166, truedimen, left=2cm, right=1.5cm, top=1.6cm, bottom=0.95cm]{geometry} % okraje stranky
\usepackage[landscape, a4paper, mag=1400, truedimen, left=0.5cm, right=0.5cm, top=0.5cm, bottom=0.5cm]{geometry} % okraje stranky

\usepackage{fontspec}
\setmainfont[FeatureFile={junicode.fea}, Ligatures={Common, TeX}, RawFeature=+fixi]{Junicode}
%\setmainfont{Junicode}

% shortcut for Junicode without ligatures (for the Czech texts)
\newfontfamily\nlfont[FeatureFile={junicode.fea}, Ligatures={Common, TeX}, RawFeature=+fixi]{Junicode}

% Hebrew font:
% http://scripts.sil.org/cms/scripts/page.php?site_id=nrsi&id=SILHebrUnic2
\newfontfamily\hebfont[Scale=1]{Ezra SIL}

\usepackage{multicol}
\usepackage{color}
\usepackage{lettrine}
\usepackage{fancyhdr}

% usual packages loading:
\usepackage{luatextra}
\usepackage{graphicx} % support the \includegraphics command and options
\usepackage{gregoriotex} % for gregorio score inclusion
\usepackage{gregoriosyms}
\usepackage{wrapfig} % figures wrapped by the text
\usepackage{parcolumns}
\usepackage[contents={},opacity=1,scale=1,color=black]{background}
\usepackage{tikzpagenodes}
\usepackage{calc}
\usepackage{longtable}
\usetikzlibrary{calc}

\setlength{\headheight}{14.5pt}

\input{conventuscommune.tex} % Often used macros
%%%% Preklady jednotlivych zpevu (nektere se opakuji, a je dobre mit je
% vsechny na jedne hromade)

% HOURS ---

\newcommand{\trAntI}{\translatioCantus{Muž boží měl kožený toulec, pečlivě
zavázaný, jenž mu visel na šíji a~často se ho dotýkal.}}

\newcommand{\trAntII}{\translatioCantus{Klíč od~něho tak dobře střežil, že
dokud žil v~těle, nikdo z~jeho žáků nezvěděl, co je uvnitř.}}

\newcommand{\trAntIII}{\translatioCantus{Ale když se odebral z~tohoto
života, schránku otevřeli a~objevili v~ní žíněné roucho a~měděný řetěz
potřísněný krví.}}

\newcommand{\trAntIV}{\translatioCantus{A když prohlédli mistrovo tělo,
nalezli jeho tělo na čtyřech místech hluboce zbrázděno ranami od řetězu.}}

\newcommand{\trAntV}{\translatioCantus{Krev vytékající z~těch ran, místy
prostoupila i~žíněným rouchem.}}

\newcommand{\trCapituli}{\translatioCantus{
Miláčkovi Boha a~lidí,
Mojžíšovi požehnané paměti,~\gredagger{}
dopřál slávu rovnou slávě svatých~\grestar{}
učinil ho mocným na postrach nepřátelům
a~jeho slovy zastavil divy.}}

\newcommand{\trLectioBrevis}{\translatioCantus{
Pamatujte na své představené,
kteří vám hlásali Boží slovo.
Uvažte, jak oni skončili život, a~napodobujte jejich víru.
Ježíš Kristus je stejný včera i~dnes i~navěky.
Nenechte se svést věelijakými cizími naukami.}}

\newcommand{\trRespLaud}{\translatioCantus{Spravedlivého vodil Hospodin~\grestar{}
po přímých stezkách. \Vbardot{} A~ukázal mu Boží království.}}

\newcommand{\trRespLaudB}{\translatioCantus{Na tvých hradbách, Jeruzaléme,
ustanovil jsem strážné;~\grestar{}
budou bdít nad mým lidem. \Vbardot{} Ani ve dne, ani v~noci nesmějí nikdy
mlčet.}}

\newcommand{\trVersus}{\translatioCantus{\Vbardot{} Ústa spravedlivého šeptají moudrost, aleluja.
\Rbardot{} A~jeho jazyk ohlašuje právo, aleluja.}}

\newcommand{\trAntBenedictus}{\translatioCantus{Když na bujné oře vložili
nosítka a~sňali jim uzdu, vydali se přímo k~cele božího muže.}}

\newcommand{\trPreces}{\translatioCantus{
\noindent S vděčností chvalme Krista, dobrého Pastýře, \gredagger{} který dal život za své ovce, \grestar{} a~pokorně ho prosme: \Rbardot{} Pane, buď pastýřem svého lidu.

\noindent Kriste, ty dáváš církvi pastýře, a~jejich službou se ujímáš svého lidu, \grestar{} dej, ať v~lásce těch, kteří nás vedou, poznáváme, jak nás miluješ. \Rbardot{} Pane, buď pastýřem svého lidu.

\noindent Ty stále konáš skrze své zástupce službu pastýře a~učitele, \grestar{} nepřestávej nás nikdy vést prostřednictvím svých služebníků. \Rbardot{} Pane, buď pastýřem svého lidu.

\noindent Ty prokazuješ svému lidu skrze jeho pastýře službu lékaře duše i~těla, \grestar{} ochraňuj náš život a~veď nás ke svatosti. \Rbardot{} Pane, buď pastýřem svého lidu.

\noindent Ty posíláš své svaté, aby slovem i~příkladem vedli tvůj lid k~tobě, \grestar{} na jejich přímluvu nás posiluj, abychom vytrvali na cestě, která vede k~věčnému životu. \Rbardot{} Pane, buď pastýřem svého lidu.}}

\newcommand{\trOrationis}{\translatioCantus{Bože, jenž nám dopřáváš radovat
se z~výroční slavnosti svatého tvého vyznavače Havla, uděl dobrotivě,
abychom když slavíme jeho narození, též se řídili podobou jeho skutků.
Skrze…}}
 % Czech translations of the proper texts

\newcommand{\annusEditionis}{2020}

\def\hebinitial#1{%
\leavevmode{\newbox\hebbox\setbox\hebbox\hbox{\hebfont{#1}\hskip 1mm}\kern -\wd\hebbox\hbox{\hebfont{#1}\hskip 1mm}}%
}

%%%% Vicekrat opakovane kousky

\newcommand{\anteOrationem}{
  \rubrica{Ante Orationem, cantatur a Superiore:}

  \pars{Supplicatio Litaniæ.}

  \cuminitiali{}{temporalia/supplicatiolitaniae.gtex}

  \pars{Oratio Dominica.}

  \cuminitiali{}{temporalia/oratiodominica.gtex}

  \rubrica{Deinde dicitur ab Hebdomadario:}

  \cuminitiali{}{temporalia/dominusvobiscum-solemnis.gtex}

  \rubrica{In choro monialium loco Dominus vobiscum dicitur:}

  \sineinitiali{temporalia/domineexaudi.gtex}
}

\setlength{\columnsep}{30pt} % prostor mezi sloupci

%%%%%%%%%%%%%%%%%%%%%%%%%%%%%%%%%%%%%%%%%%%%%%%%%%%%%%%%%%%%%%%%%%%%%%%%%%%%%%%%%%%%%%%%%%%%%%%%%%%%%%%%%%%%%
\begin{document}

% Here we set the space around the initial.
% Please report to http://home.gna.org/gregorio/gregoriotex/details for more details and options
\grechangedim{afterinitialshift}{2.2mm}{scalable}
\grechangedim{beforeinitialshift}{2.2mm}{scalable}

\grechangedim{interwordspacetext}{0.22 cm plus 0.15 cm minus 0.05 cm}{scalable}%
\grechangedim{annotationraise}{-0.2cm}{scalable}

% Here we set the initial font. Change 38 if you want a bigger initial.
% Emit the initials in red.
\grechangestyle{initial}{\color{red}\fontsize{38}{38}\selectfont}

\pagestyle{empty}

%%%% Titulni stranka
\begin{titulusOfficii}
\nomenFesti{Feria III infra Hebdom. Ultima Adventus.}
\end{titulusOfficii}

\pars{}

\scriptura{}

\pagebreak

% graphic
\renewcommand{\headrulewidth}{0pt} % no horiz. rule at the header
\fancyhf{}
\pagestyle{fancy}

\cantusSineNeumas

\hora{Ad Matutinum.}

\vspace{2mm}

\cuminitiali{}{temporalia/dominelabiamea.gtex}

\vspace{2mm}

\pars{Invitatorium.} \scriptura{Phil. 4, 4.5}

\vspace{-6mm}

\antiphona{VI}{temporalia/inv-propeestiamsimplex.gtex}

\vfill
\pagebreak

\pars{Hymnus.}

\vspace{-5mm}

\antiphona{II}{temporalia/hym-VeniRedemptor.gtex}
%{
%\vspace{-5mm}
%\setlength{\columnsep}{0pt} % prostor mezi sloupci
%\begin{translatioMulticol}{5}
Nebeské Slovo, Syn ty jsi\\
a světlo z Otce prýštící,\\
jenž zrozen, jdeš zachránit svět,\\
když naplnil se časů věk.\columnbreak

Už teď nám duši rozjasni\\
a plamen lásky zapal v ní,\\
ať srdce vlády zbavené\\
nebeská rozkoš naplní.\columnbreak

Aby, až soudce na trůnu\\
ohni zlé bude vydávat\\
a ty, kdo toho hodni jsou,\\
milý hlas v nebe povolá,\columnbreak

zmatku temnot ať vydáni\\
nejsme, ni v stravu plamenů,\\
leč na tvář Boha pohledět\\
smíme a těšit se v nebesích.\columnbreak

Otci společně se Synem,\\
a Tobě, Duše přesvatý,\\
sláva, jež vždycky trvala,\\
stejná budiž i na věky.\\
Amen.
\end{translatioMulticol}

%\setlength{\columnsep}{30pt} % prostor mezi sloupci
%}

\vfill
\pagebreak

\iffalse
% MA
\pars{Psalmus 1.} \scriptura{Ps. 9, 22}

\vspace{-4mm}

\antiphona{II D}{temporalia/ant-utquiddomine.gtex}

%\vspace{-5mm}

\scriptura{Ps. 9, 22-32}

%\vspace{-2mm}

\initiumpsalmi{temporalia/ps9xxii_xxxii-initium-ii-D-auto.gtex}

%\psalmusEtTranslatioT{temporalia/ps9xxii_xxxii-III-comb.tex}{10cm}
\input{temporalia/ps9xxii_xxxii-III.tex} \Abardot{}

\vfill
\pagebreak

\pars{Psalmus 2.} \scriptura{Ex. 15, 18}

\vspace{-4mm}

\antiphona{IV* e}{temporalia/ant-inaeternum.gtex}

%\vspace{-5mm}

\scriptura{Ps. 9, 33-39}

%\vspace{-2mm}

\initiumpsalmi{temporalia/ps9xxxiii_xxxix-initium-iv_-e-auto.gtex}

%\psalmusEtTranslatioT{temporalia/ps9xxxiii_xxxix-III-comb.tex}{10cm}
\input{temporalia/ps9xxxiii_xxxix-III.tex} \Abardot{}

\vfill
\pagebreak

\pars{Psalmus 3.} \scriptura{Ps. 11, 8}

\vspace{-4mm}

\antiphona{VIII G}{temporalia/ant-tudomine.gtex}

%\vspace{-5mm}

\scriptura{Ps. 11}

%\vspace{-2mm}

\initiumpsalmi{temporalia/ps11-initium-viii-G-auto.gtex}

%\psalmusEtTranslatioT{temporalia/ps11-III-comb.tex}{10cm}
\input{temporalia/ps11-III.tex} \Abardot{}

\vfill
\pagebreak

\pars{Psalmus 4.} \scriptura{Ps. 67, 2}

\vspace{-4mm}

\antiphona{VII a}{temporalia/ant-exsurgatdeus.gtex}

%\vspace{-5mm}

\scriptura{Ps. 67, 2-11}

\initiumpsalmi{temporalia/ps67i-initium-vii-a-auto.gtex}

%\psalmusEtTranslatioT{temporalia/ps67i-III-comb.tex}{10cm}
\input{temporalia/ps67i-III.tex} \Abardot{}

\vfill
\pagebreak

\pars{Psalmus 5.}

\vspace{-4mm}

\antiphona{I f}{temporalia/ant-deusnosterdeussalvos.gtex}

%\vspace{-3mm}

\scriptura{Ps. 67, 12-24}

%\vspace{-2mm}

\initiumpsalmi{temporalia/ps67ii-initium-i-f-auto.gtex}

%\psalmusEtTranslatioT{temporalia/ps67ii-III-comb.tex}{10cm}
\input{temporalia/ps67ii-III.tex} \Abardot{}

\vfill
\pagebreak

\pars{Psalmus 6.} \scriptura{Ps. 67, 27; \textbf{H96}}

\vspace{-4mm}

\antiphona{D}{temporalia/ant-inecclesiis.gtex}

%\vspace{-5mm}

\scriptura{Ps. 67, 25-36}

\initiumpsalmi{temporalia/ps67iii-initium-d-g2-auto.gtex}

%\psalmusEtTranslatioT{temporalia/ps67iii-III-comb.tex}{10cm}
\input{temporalia/ps67iii-III.tex} \Abardot{}
\else
% MB
\pars{Psalmus 1.} \scriptura{Ps. 36, 5; \textbf{H93}}

\vspace{-4mm}

\antiphona{VI F}{temporalia/ant-reveladomino.gtex}

%\vspace{-5mm}

\scriptura{Ps. 36, 1-11}

%\vspace{-2mm}

\initiumpsalmi{temporalia/ps36i_xi-initium-vi-F-auto.gtex}

%\psalmusEtTranslatioT{temporalia/ps36i_xi-III-comb.tex}{10cm}
\input{temporalia/ps36i_xi-III.tex} \Abardot{}

\vfill
\pagebreak

\pars{Psalmus 2.}

\vspace{-4mm}

\antiphona{II D}{temporalia/ant-iuniorfui.gtex}

\vspace{-2mm}

\scriptura{Ps. 36, 12-29}

\vspace{-2mm}

\initiumpsalmi{temporalia/ps36xii_xxix-initium-ii-D-auto.gtex}

%\psalmusEtTranslatioT{temporalia/ps36xii_xxix-III-comb.tex}{10cm}
\input{temporalia/ps36xii_xxix-III.tex}

\vfill

\antiphona{}{temporalia/ant-iuniorfui.gtex}

\vfill
\pagebreak

\pars{Psalmus 3.} \scriptura{Ps. 36, 3}

\vspace{-4mm}

\antiphona{VI F}{temporalia/ant-speraindomino.gtex}

%\vspace{-5mm}

\scriptura{Ps. 36, 30-40}

%\vspace{-2mm}

\initiumpsalmi{temporalia/ps36iii-initium-vi-F-auto.gtex}

%\psalmusEtTranslatioT{temporalia/ps36iii-III-comb.tex}{10cm}
\input{temporalia/ps36iii-III.tex} \Abardot{}

\vfill
\pagebreak

\pars{Psalmus 4.} \scriptura{Ps. 101, 2; \textbf{H99}}

\vspace{-4mm}

\antiphona{I a}{temporalia/ant-clamormeus1.gtex}

%\vspace{-5mm}

\scriptura{Ps. 101, 2-12}

%\vspace{-2mm}

\initiumpsalmi{temporalia/ps101ii_xii-initium-i-a.gtex}

%\psalmusEtTranslatioT{temporalia/ps101ii_xii-III-comb.tex}{10cm}
\input{temporalia/ps101ii_xii-III.tex} \Abardot{}

\vfill
\pagebreak

\pars{Psalmus 5.}

\vspace{-4mm}

\antiphona{VIII G}{temporalia/ant-respicehumilitatem.gtex}

\vspace{-2mm}

\scriptura{Ps. 101, 13-23}

\vspace{-2mm}

\initiumpsalmi{temporalia/ps101xiii_xxiii-initium-viii-G-auto.gtex}

%\psalmusEtTranslatioT{temporalia/ps101xiii_xxiii-III-comb.tex}{10cm}
\input{temporalia/ps101xiii_xxiii-III.tex} \Abardot{}

\vfill
\pagebreak

\pars{Psalmus 6.} \scriptura{Ps. 139, 14; \textbf{H371}}

\vspace{-4mm}

\antiphona{I g}{temporalia/ant-iusticonfitebuntur1.gtex}

%\vspace{-5mm}

\scriptura{Ps. 101, 24-29}

%\vspace{-2mm}

\initiumpsalmi{temporalia/ps101iii-initium-i-g-auto.gtex}

%\psalmusEtTranslatioT{temporalia/ps101iii-III-comb.tex}{10cm}
\input{temporalia/ps101iii-III.tex} \Abardot{}
\fi

\vfill
\pagebreak

\pars{Versus.} \scriptura{Mc. 1, 3; Is. 40, 3}

% Versus. %%%
\sineinitiali{temporalia/versus-voxclamantis-simplex.gtex}

\vspace{5mm}

\sineinitiali{temporalia/oratiodominica-mat.gtex}

\vspace{5mm}

\pars{Absolutio.}

\cuminitiali{}{temporalia/absolutio-ipsius.gtex}

\vfill
\pagebreak

\cuminitiali{}{temporalia/benedictio-solemn-deus.gtex}

\vspace{7mm}

\lectioi

\noindent \Vbardot{} Tu autem, Dómine, miserére nobis.
\noindent \Rbardot{} Deo grátias.

\vfill
\pagebreak

\responsoriumi

\vfill
\pagebreak

\cuminitiali{}{temporalia/benedictio-solemn-christus.gtex}

\vspace{7mm}

\lectioii

\noindent \Vbardot{} Tu autem, Dómine, miserére nobis.
\noindent \Rbardot{} Deo grátias.

\vfill
\pagebreak

\responsoriumii

\vfill
\pagebreak

\cuminitiali{}{temporalia/benedictio-solemn-ignem.gtex}

\vspace{7mm}

\lectioiii

\noindent \Vbardot{} Tu autem, Dómine, miserére nobis.
\noindent \Rbardot{} Deo grátias.

\vfill
\pagebreak

\responsoriumiii

\vfill
\pagebreak

\rubrica{Reliqua omittuntur, nisi Laudes separandæ sint.}

\pars{Oratio}

\noindent \Vbardot{} Dómine, exáudi oratiónem meam.

\noindent \Rbardot{} Et clamor meus ad te véniat.

\oratio

\vspace{7mm}

\pars{Conclusio}

\noindent \Vbardot{} Dómine, exáudi oratiónem meam.

\noindent \Rbardot{} Et clamor meus ad te véniat.

\noindent \Vbardot{} Benedicámus Dómino.

\noindent \Rbardot{} Deo grátias.

\noindent \Vbardot{} Fidélium ánimæ per misericórdiam Dei requiéscant in pace.

\noindent \Rbardot{} Amen.

\vfill
\pagebreak

\hora{Ad Laudes.} %%%%%%%%%%%%%%%%%%%%%%%%%%%%%%%%%%%%%%%%%%%%%%%%%%%%%
%\sideThumbs{Laudes}

\cantusSineNeumas

\vspace{0.5cm}
\grechangedim{interwordspacetext}{0.18 cm plus 0.15 cm minus 0.05 cm}{scalable}%
\cuminitiali{}{temporalia/deusinadiutorium-communis.gtex}
\grechangedim{interwordspacetext}{0.22 cm plus 0.15 cm minus 0.05 cm}{scalable}%

\vfill
%\pagebreak

\pars{Hymnus}

\cuminitiali{D}{temporalia/hym-MagnisProphetae.gtex}
\vspace{-3mm}
%\input{hym-MagnisProphetae-bohtext.tex}

\vfill
\pagebreak

\iffalse
% LA
\pars{Psalmus 1.} \scriptura{Ps. 23, 4.3}

\vspace{-4mm}

\antiphona{IV* e}{temporalia/ant-innocensmanibus.gtex}

\scriptura{Psalmus 23.}

\initiumpsalmi{temporalia/ps23-initium-iv_-e-auto.gtex}

%\psalmusEtTranslatioT{temporalia/ps23-III-comb.tex}{10cm}
\input{temporalia/ps23-III.tex} \Abardot{}

\vfill
\pagebreak

\pars{Psalmus 2.} \scriptura{Tob. 13, 6}

\vspace{-4mm}

\antiphona{VII a}{temporalia/ant-exaltateregem.gtex}

\scriptura{Canticum Tobiæ, Tob. 13, 2-8}

\initiumpsalmi{temporalia/tobiae-initium-vii-a-auto.gtex}

%\psalmusEtTranslatioT{temporalia/tobiae-comb.tex}{10cm}
\input{temporalia/tobiae.tex} \Abardot{}

\vfill
\pagebreak

\pars{Psalmus 3.} \scriptura{Ps. 32, 1; \textbf{H93}}

\vspace{-4mm}

\antiphona{IV E}{temporalia/ant-rectosdecet.gtex}

\vspace{-4mm}

\scriptura{Psalmus 32.}

\vspace{-2mm}

\initiumpsalmi{temporalia/ps32-initium-iv-E-auto.gtex}

%\psalmusEtTranslatioT{temporalia/ps32-III-comb.tex}{10cm}
\input{temporalia/ps32-III.tex}

\vfill

\antiphona{}{temporalia/ant-rectosdecet.gtex}
\else
\iffalse
% LC
\pars{Psalmus 1.} \scriptura{Ps. 83, 5}

\vspace{-4mm}

\antiphona{VIII G}{temporalia/ant-beatiquihabitant.gtex}

\vspace{-2mm}

\scriptura{Psalmus 84.}

\vspace{-2mm}

\initiumpsalmi{temporalia/ps84-initium-viii-G-auto.gtex}

%\psalmusEtTranslatioT{temporalia/ps84-III-comb.tex}{10cm}
\input{temporalia/ps84-III.tex} \Abardot{}

\vfill
\pagebreak

\pars{Psalmus 2.}

\vspace{-4mm}

\antiphona{VII d}{temporalia/ant-denoctespiritusmeus.gtex}

\vspace{-2mm}

\scriptura{Canticum Isaiæ, Is. 26, 1-12}

\vspace{-2mm}

%\initiumpsalmi{temporalia/isaiae3-initium-vii-d-auto.gtex}
\initiumpsalmi{temporalia/isaiae3-initium-vii-d.gtex}

%\psalmusEtTranslatioT{temporalia/isaiae3-comb.tex}{10cm}
\input{temporalia/isaiae3.tex} \Abardot{}

\vfill
\pagebreak

\pars{Psalmus 3.} \scriptura{Ps. 66, 2}

\vspace{-4mm}

\antiphona{E}{temporalia/ant-illuminadomine.gtex}

%\vspace{-4mm}

\scriptura{Psalmus 66.}

%\vspace{-2mm}

\initiumpsalmi{temporalia/ps66-initium-e.gtex}

%\psalmusEtTranslatioT{temporalia/ps66-III-comb.tex}{10cm}
\input{temporalia/ps66-III.tex} \Abardot{}
\else
% LD
\pars{Psalmus 1.} \scriptura{Is. 16, 1; \textbf{H37}}

\vspace{-4mm}

\antiphona{II* a}{temporalia/ant-emitteagnum.gtex}

\vspace{-2mm}

\scriptura{Psalmus 100.}

\vspace{-2mm}

\initiumpsalmi{temporalia/ps100-initium-ii_-a-auto.gtex}

%\psalmusEtTranslatioT{temporalia/ps100-III-comb.tex}{10cm}
\input{temporalia/ps100-III.tex} \Abardot{}

\vfill
\pagebreak

\pars{Psalmus 2.} \scriptura{Is. 26, 1.2; \textbf{H24}}

\vspace{-4mm}

\antiphona{VII d}{temporalia/ant-urbsfortitudinis.gtex}

%\vspace{-2mm}

\scriptura{Canticum Danielis, Dan. 3, 26.27.29.34-41}

%\vspace{-2mm}

\initiumpsalmi{temporalia/dan32-initium-vii-d-auto.gtex}

%\psalmusEtTranslatioT{temporalia/dan32-comb.tex}{10cm}
\input{temporalia/dan32.tex}

\vfill

\antiphona{}{temporalia/ant-urbsfortitudinis.gtex}

\vfill
\pagebreak

\pars{Psalmus 3.} \scriptura{Ps. 66, 3; \textbf{H37}}

\vspace{-4mm}

\antiphona{II* a}{temporalia/ant-utcognoscamus.gtex}

%\vspace{-4mm}

\scriptura{Psalmus 143, 1-10.}

%\vspace{-2mm}

\initiumpsalmi{temporalia/ps143i_x-initium-ii_-a-auto.gtex}

%\psalmusEtTranslatioT{temporalia/ps143i_x-III-comb.tex}{10cm}
\input{temporalia/ps143i_x-III.tex} \Abardot{}
\fi
\fi

\vfill
\pagebreak

\lectiobrevis

% preklad Jeruz. bible
%\trCapituliI

\vfill

\responsoriumbreve

%\trResp

\vfill
\pagebreak

\benedictus

\vfill
\pagebreak

%\sideThumbs{{\scriptsize{}Fine horarum}}

\ifx\preces\undefined
\rubrica{Ante Orationem, cantatur a Superiore:}

\pars{Supplicatio Litaniæ.}

\cuminitiali{}{temporalia/supplicatiolitaniae.gtex}

\pars{Oratio Dominica.}

\cuminitiali{}{temporalia/oratiodominica.gtex}
\else
\pars{Preces.}

\sineinitiali{}{temporalia/tonusprecum.gtex}

\preces

\vfill

\pars{Oratio Dominica.}

\cuminitiali{}{temporalia/oratiodominicaalt.gtex}

\vfill
\pagebreak

\rubrica{vel:}

\pars{Supplicatio Litaniæ.}

\cuminitiali{}{temporalia/supplicatiolitaniae.gtex}

\vfill

\pars{Oratio Dominica.}

\cuminitiali{}{temporalia/oratiodominica.gtex}
\fi

\vfill
\pagebreak

% Oratio. %%%
\oratio

\vspace{-1mm}
%\trOrationisI

\vfill

\rubrica{Hebdomadarius dicit Dominus vobiscum, vel, absente sacerdote vel diacono, sic concluditur:}

\vspace{2mm}

\antiphona{C}{temporalia/dominusnosbenedicat.gtex}

\rubrica{Postea cantatur a cantore:}

\vspace{2mm}

\cuminitiali{IV}{temporalia/benedicamus-dominica-advequad.gtex}

\vspace{1mm}

\vfill
\pagebreak

\hora{Ad Vesperas.} %%%%%%%%%%%%%%%%%%%%%%%%%%%%%%%%%%%%%%%%%%%%%%%%%%%%%
%\sideThumbs{Vesperæ}

\cantusSineNeumas

%\vspace{0.5cm}
\grechangedim{interwordspacetext}{0.18 cm plus 0.15 cm minus 0.05 cm}{scalable}%
\cuminitiali{}{temporalia/deusinadiutorium-communis.gtex}
\grechangedim{interwordspacetext}{0.22 cm plus 0.15 cm minus 0.05 cm}{scalable}%

\vfill
%\pagebreak

%\vspace{4mm}

\pars{Psalmus 1.} \scriptura{Is. 16, 1; \textbf{H37}}

\vspace{-4mm}

\antiphona{II* a}{temporalia/ant-emitteagnum.gtex}

\vspace{-4mm}

\scriptura{Psalmus 129.}

\initiumpsalmi{temporalia/ps129-initium-ii_-a-auto.gtex}

%\psalmusEtTranslatioT{temporalia/ps129-IV-comb.tex}{10cm}
\input{temporalia/ps129-IV.tex} \Abardot{}

\vspace{-1cm}

\vfill
\pagebreak

\pars{Psalmus 2.} \scriptura{\textbf{H38}}

\vspace{-4mm}

\antiphona{VIII G}{temporalia/ant-convertere.gtex}

\vspace{-4mm}

\scriptura{Psalmus 130.}

\initiumpsalmi{temporalia/ps130-initium-viii-g-auto.gtex}

%\psalmusEtTranslatioT{temporalia/ps130-IV-comb.tex}{10cm}
\input{temporalia/ps130-IV.tex}

\vfill
\pagebreak

\pars{Psalmus 3.} \scriptura{Ps. 66, 3; \textbf{H37}}

\vspace{-4mm}

\antiphona{II* a}{temporalia/ant-utcognoscamus.gtex}

\vspace{-4mm}

\scriptura{Psalmus 131.}

\initiumpsalmi{temporalia/ps131-initium-ii_-a-auto.gtex}

%\psalmusEtTranslatioT{temporalia/ps131-IV-comb.tex}{10cm}
\input{temporalia/ps131-IV.tex}

\vfill

\antiphona{}{temporalia/ant-utcognoscamus.gtex}

\vfill
\pagebreak

\pars{Psalmus 4.} \scriptura{Is. 30, 18}

\vspace{-4mm}

\antiphona{I g}{temporalia/ant-deusiudicii.gtex}

\vspace{-4mm}

\scriptura{Psalmus 132.}

\initiumpsalmi{temporalia/ps132-initium-i-g-auto.gtex}

%\psalmusEtTranslatioT{temporalia/ps132-IV-comb.tex}{10cm}
\input{temporalia/ps132-IV.tex} \Abardot{}

\vfill
\pagebreak

\pars{Capitulum.} \scriptura{Gen. 49, 10}

\grechangedim{interwordspacetext}{0.12 cm plus 0.15 cm minus 0.05 cm}{scalable}%
\cuminitiali{}{temporalia/capitulum-NosAuferetur.gtex}
\grechangedim{interwordspacetext}{0.22 cm plus 0.15 cm minus 0.05 cm}{scalable}%

% preklad Jeruz. bible
%\trCapituliI

\vfill

\pars{Responsorium breve.} \scriptura{Ps. 84, 8; \textbf{H20}}

\cuminitiali{IV}{temporalia/resp-ostendenobis.gtex}

%\trResp

\vfill
\pagebreak

\pars{Hymnus} \scriptura{Ambrosius (\olddag{} 397)}

\cuminitiali{IV}{temporalia/hym-VerbumSalutis.gtex}
\vspace{-3mm}
%\input{hym-VerbumSalutis-bohtext.tex}

\vfill
%\pagebreak

\pars{Versus.} \scriptura{Is. 45, 8}

% Versus. %%%
\sineinitiali{temporalia/versus-rorate.gtex}

%\noindent \trVersus

\vfill
\pagebreak

\magnificat

\vfill
\pagebreak

%\sideThumbs{{\scriptsize{}Fine horarum}}

\anteOrationem

\pagebreak

% Oratio. %%%
\oratioLaudes

\vspace{-1mm}
%\trOrationisI

\vfill

\rubrica{Hebdomadarius dicit iterum Dominus vobiscum, vel cantor dicit:}

\vspace{2mm}

\sineinitiali{temporalia/domineexaudi.gtex}

\rubrica{Postea cantatur a cantore:}

\vspace{2mm}

\cuminitiali{IV}{temporalia/benedicamus-feria-advequad.gtex}

\vspace{1mm}

\end{document}

