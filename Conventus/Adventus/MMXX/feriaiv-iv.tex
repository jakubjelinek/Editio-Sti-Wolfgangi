\newcommand{\lectioi}{\pars{Lectio I.} \scriptura{Is. 51, 1-5.7-8.11}

\noindent De libro Isaíæ prophétæ.

\noindent Audíte me, qui sequímini iustítiam, qui quǽritis Dóminum; atténdite ad petram, unde excísi estis, et ad cavérnam laci, de qua præcísi estis. Atténdite ad Abraham patrem vestrum et ad Saram, quæ péperit vos; quia unum vocávi eum et benedíxi ei et multiplicávi eum. Consolátur enim Dóminus Sion, consolátur omnes ruínas eius; et ponit desértum eius quasi Eden et solitúdinem eius quasi hortum Dómini. Gáudium et lætítia inveniétur in ea, gratiárum áctio et vox laudis. Atténdite ad me, pópule meus, et, natiónes, me audíte, quia lex a me éxiet, et iudícium meum in lucem populórum státuam. Prope est iustítia mea, egréssa est salus mea, et bráchia mea pópulos iudicábunt; in me ínsulæ sperábunt et ad bráchium meum atténdent. Audíte me, qui scitis iustítiam, pópule, in cuius corde est lex mea: nolíte timére oppróbrium hóminum et blasphémias eórum ne metuátis. Sicut enim vestiméntum sic cómedet eos vermis, et sicut lanam sic devorábit eos tínea; iustítia autem mea in sempitérnum erit et salus mea in generatiónes generatiónum. Et redémpti a Dómino reverténtur et vénient in Sion laudántes; et lætítia sempitérna super cápita eórum, gáudium et lætítiam obtinébunt; fúgiet dolor et gémitus.}
\newcommand{\responsoriumi}{\pars{Responsorium 1.} \scriptura{\Rbar{} Is. 40, 9.10 \Vbar{} ibid. 40, 9; \textbf{H34}}

\vspace{-5mm}

\responsorium{VI}{temporalia/resp-clamainfortitudine-CROCHU.gtex}{}}
\newcommand{\lectioii}{\pars{Lectio II.} \scriptura{Cap. 9-12: PG 10, 815-819}

\noindent Ex Tractátu sancti Hippólyti presbýteri Contra hǽresim Noéti.

\noindent Deus, solus cum esset, nihílque sibi coǽvum habéret, vóluit mundum effícere. Et mundum cógitans ac volens et dicens effécit; continuóque éxstitit ei factus sicut vóluit, et sicut vóluit, perfécit. Satis ígitur nobis est scire solum, nihil esse Deo coǽvum. Nihil erat præter ipsum; ipse solus, multus erat. Nec enim erat sine ratióne, sine sapiéntia, sine poténtia, sine consílio. Omnia erant in eo: ipse erat ómnia. Quando vóluit, et quómodo vóluit, osténdit Verbum suum tempóribus apud eum definítis; per quod ómnia fecit.

\noindent Quod Verbum cum in se habéret, essétque mundo creáto inaspectábile, fecit aspectábile, emíttens priórem vocem; et lumen ex lúmine génerans, deprómpsit ipsi creatúræ Dóminum, sensum suum; et qui prius ipsi tantum erat visíbilis, mundo autem invisíbilis, hunc visíbilem facit, ut mundus, cum eum qui appáruit vidéret, salvus fíeri posset.}
\newcommand{\responsoriumii}{\pars{Responsorium 2.} \scriptura{\Rbar{} Is. 1, 11 \Vbar{} Is. 45, 8; \textbf{H36}}

\vspace{-5mm}

\responsorium{I}{temporalia/resp-egredieturvirga-CROCHU.gtex}{}}
\newcommand{\lectioiii}{\pars{Lectio III.}

\noindent Hoc vero mens est, quod pródiens in mundum, osténsum est puer Dei. Omnia ígitur per ipsum, ipse autem solus ex Patre.

\noindent Hic autem dedit legem et prophétas; et dando coégit hos per Spíritum Sanctum loqui, ut accipiéntes virtútis patérnæ inspiratiónem, consílium et voluntátem Patris nuntiárent.

\noindent Factum est ígitur maniféstum Verbum, sicut ait beátus Ioánnes. Répetit enim summátim quæ a prophétis dicta sunt, demónstrans hoc esse Verbum per quod ómnia facta sunt. Sic enim ait: In princípio erat Verbum et Verbum erat apud Deum et Deus erat Verbum. Omnia per ipsum facta sunt et sine ipso factum est nihil. Et infra ait: Mundus per ipsum factus est et mundus eum non cognóvit. In própria venit et sui eum non recepérunt.}
\newcommand{\responsoriumiii}{\pars{Responsorium 3.} \scriptura{\Rbar{} Cantor \Vbar{} Lc. 1, 28; \textbf{H36}}

\vspace{-5mm}

\responsorium{I}{temporalia/resp-annuntiatumest-CROCHU-cumdox.gtex}{}}
\newcommand{\lectiobrevis}{\pars{Lectio Brevis.} \scriptura{Ier. 30, 21,22}

\noindent Hæc dicit Dóminus: Erit dux eius ex Iacob, et princeps de médio eius procédet; et applicábo eum, et accédet ad me. Et éritis mihi in pópulum, et ego ero vobis in Deum.}
\newcommand{\benedictus}{\pars{Canticum Zachariæ.} \scriptura{\textbf{H81}}

\vspace{-4mm}

{
\grechangedim{interwordspacetext}{0.18 cm plus 0.15 cm minus 0.05 cm}{scalable}%
\antiphona{VIII C}{temporalia/ant-eccecompletasunt.gtex}
\grechangedim{interwordspacetext}{0.22 cm plus 0.15 cm minus 0.05 cm}{scalable}%
}

%\trAntIMagnificat

%\vspace{-2mm}

\scriptura{Lc. 1, 68-79}

%\vspace{-2mm}

\cantusSineNeumas
\initiumpsalmi{temporalia/benedictus-initium-viii-c-auto.gtex}

%\vspace{-1.5mm}

\input{temporalia/benedictus-viii-c.tex} \Abardot{}}
\newcommand{\preces}{\noindent Deum Patrem, fratres caríssimi, implorántes, qui misit Fílium suum ad salvándos hómines,~\gredagger{} súpplices acclamémus:

\Rbardot{} Osténde nobis, Dómine, misericórdiam tuam.

\noindent Christum tuum, Pater clementíssime, quem plena fide os nostrum annúntiat,~\gredagger{} conversátio nostra ópere ne despíciat.

\Rbardot{} Osténde nobis, Dómine, misericórdiam tuam.

\noindent Qui Fílium tuum misísti ad salútem,~\gredagger{} univérsum aufer a fácie terræ et a civitáte ista dolórem.

\Rbardot{} Osténde nobis, Dómine, misericórdiam tuam.

\noindent Terra nostra advéntu Fílii tui iucunditáte perfúsa,~\gredagger{} tuæ plenitúdinis gáudium ubérius experiátur.

\Rbardot{} Osténde nobis, Dómine, misericórdiam tuam.

\noindent Per misericórdiam tuam, fac ut nos pie et sóbrie in hoc sǽculo vivámus,~\gredagger{} exspectántes beátam spem et advéntum glóriæ Christi.

\Rbardot{} Osténde nobis, Dómine, misericórdiam tuam.}
\newcommand{\oratio}{\pars{Oratio.}

\noindent Omnípotens sempitérne Deus, nativitátem Fílii tui secúndum carnem propinquáre cernéntes, quǽsumus,~\gredagger{} ut nobis indígnis fámulis tuis misericórdiam præstet Verbum,~\grestar{} quod ex Vírgine María dignátum est caro fíeri et habitáre in nobis.

\noindent Qui tecum vivit et regnat in unitáte Spíritus Sancti, Deus, per ómnia sǽcula sæculórum.

\noindent \Rbardot{} Amen.}
\newcommand{\magnificat}{\pars{Canticum B. Mariæ V.} \scriptura{Is. 7, 14; 33, 22; Gn. 49, 10; \textbf{H41}}

\vspace{-6.5mm}

{
\grechangedim{interwordspacetext}{0.18 cm plus 0.15 cm minus 0.05 cm}{scalable}%
\antiphona{II D}{temporalia/ant-oemmanuel.gtex}
\grechangedim{interwordspacetext}{0.22 cm plus 0.15 cm minus 0.05 cm}{scalable}%
}

%\trAntIMagnificat

\vspace{-3mm}

\scriptura{Lc. 1, 46-55}

\vspace{-2mm}

\cantusSineNeumas

\initiumpsalmi{temporalia/magnificat-initium-iisoll-D.gtex}

\vspace{-1.5mm}

\input{temporalia/magnificat-iisoll-D.tex} \Abardot{}

\vspace{-1cm}}
\newcommand{\hebdomada}{infra Hebdom. IV post Pentecosten.}
\newcommand{\oratioLaudes}{\cuminitiali{}{temporalia/oratio4.gtex}}

% LuaLaTeX

\documentclass[a4paper, twoside, 12pt]{article}
\usepackage[latin]{babel} 
%\usepackage[landscape, left=3cm, right=1.5cm, top=2cm, bottom=1cm]{geometry} % okraje stranky
%\usepackage[landscape, a4paper, mag=1166, truedimen, left=2cm, right=1.5cm, top=1.6cm, bottom=0.95cm]{geometry} % okraje stranky
\usepackage[landscape, a4paper, mag=1400, truedimen, left=0.5cm, right=0.5cm, top=0.5cm, bottom=0.5cm]{geometry} % okraje stranky

\usepackage{fontspec}
\setmainfont[FeatureFile={junicode.fea}, Ligatures={Common, TeX}, RawFeature=+fixi]{Junicode}
%\setmainfont{Junicode}

% shortcut for Junicode without ligatures (for the Czech texts)
\newfontfamily\nlfont[FeatureFile={junicode.fea}, Ligatures={Common, TeX}, RawFeature=+fixi]{Junicode}

% Hebrew font:
% http://scripts.sil.org/cms/scripts/page.php?site_id=nrsi&id=SILHebrUnic2
\newfontfamily\hebfont[Scale=1]{Ezra SIL}

\usepackage{multicol}
\usepackage{color}
\usepackage{lettrine}
\usepackage{fancyhdr}

% usual packages loading:
\usepackage{luatextra}
\usepackage{graphicx} % support the \includegraphics command and options
\usepackage{gregoriotex} % for gregorio score inclusion
\usepackage{gregoriosyms}
\usepackage{wrapfig} % figures wrapped by the text
\usepackage{parcolumns}
\usepackage[contents={},opacity=1,scale=1,color=black]{background}
\usepackage{tikzpagenodes}
\usepackage{calc}
\usepackage{longtable}
\usetikzlibrary{calc}

\setlength{\headheight}{14.5pt}

\input{conventuscommune.tex} % Often used macros
%%%% Preklady jednotlivych zpevu (nektere se opakuji, a je dobre mit je
% vsechny na jedne hromade)

% HOURS ---

\newcommand{\trAntI}{\translatioCantus{Muž boží měl kožený toulec, pečlivě
zavázaný, jenž mu visel na šíji a~často se ho dotýkal.}}

\newcommand{\trAntII}{\translatioCantus{Klíč od~něho tak dobře střežil, že
dokud žil v~těle, nikdo z~jeho žáků nezvěděl, co je uvnitř.}}

\newcommand{\trAntIII}{\translatioCantus{Ale když se odebral z~tohoto
života, schránku otevřeli a~objevili v~ní žíněné roucho a~měděný řetěz
potřísněný krví.}}

\newcommand{\trAntIV}{\translatioCantus{A když prohlédli mistrovo tělo,
nalezli jeho tělo na čtyřech místech hluboce zbrázděno ranami od řetězu.}}

\newcommand{\trAntV}{\translatioCantus{Krev vytékající z~těch ran, místy
prostoupila i~žíněným rouchem.}}

\newcommand{\trCapituli}{\translatioCantus{
Miláčkovi Boha a~lidí,
Mojžíšovi požehnané paměti,~\gredagger{}
dopřál slávu rovnou slávě svatých~\grestar{}
učinil ho mocným na postrach nepřátelům
a~jeho slovy zastavil divy.}}

\newcommand{\trLectioBrevis}{\translatioCantus{
Pamatujte na své představené,
kteří vám hlásali Boží slovo.
Uvažte, jak oni skončili život, a~napodobujte jejich víru.
Ježíš Kristus je stejný včera i~dnes i~navěky.
Nenechte se svést věelijakými cizími naukami.}}

\newcommand{\trRespLaud}{\translatioCantus{Spravedlivého vodil Hospodin~\grestar{}
po přímých stezkách. \Vbardot{} A~ukázal mu Boží království.}}

\newcommand{\trRespLaudB}{\translatioCantus{Na tvých hradbách, Jeruzaléme,
ustanovil jsem strážné;~\grestar{}
budou bdít nad mým lidem. \Vbardot{} Ani ve dne, ani v~noci nesmějí nikdy
mlčet.}}

\newcommand{\trVersus}{\translatioCantus{\Vbardot{} Ústa spravedlivého šeptají moudrost, aleluja.
\Rbardot{} A~jeho jazyk ohlašuje právo, aleluja.}}

\newcommand{\trAntBenedictus}{\translatioCantus{Když na bujné oře vložili
nosítka a~sňali jim uzdu, vydali se přímo k~cele božího muže.}}

\newcommand{\trPreces}{\translatioCantus{
\noindent S vděčností chvalme Krista, dobrého Pastýře, \gredagger{} který dal život za své ovce, \grestar{} a~pokorně ho prosme: \Rbardot{} Pane, buď pastýřem svého lidu.

\noindent Kriste, ty dáváš církvi pastýře, a~jejich službou se ujímáš svého lidu, \grestar{} dej, ať v~lásce těch, kteří nás vedou, poznáváme, jak nás miluješ. \Rbardot{} Pane, buď pastýřem svého lidu.

\noindent Ty stále konáš skrze své zástupce službu pastýře a~učitele, \grestar{} nepřestávej nás nikdy vést prostřednictvím svých služebníků. \Rbardot{} Pane, buď pastýřem svého lidu.

\noindent Ty prokazuješ svému lidu skrze jeho pastýře službu lékaře duše i~těla, \grestar{} ochraňuj náš život a~veď nás ke svatosti. \Rbardot{} Pane, buď pastýřem svého lidu.

\noindent Ty posíláš své svaté, aby slovem i~příkladem vedli tvůj lid k~tobě, \grestar{} na jejich přímluvu nás posiluj, abychom vytrvali na cestě, která vede k~věčnému životu. \Rbardot{} Pane, buď pastýřem svého lidu.}}

\newcommand{\trOrationis}{\translatioCantus{Bože, jenž nám dopřáváš radovat
se z~výroční slavnosti svatého tvého vyznavače Havla, uděl dobrotivě,
abychom když slavíme jeho narození, též se řídili podobou jeho skutků.
Skrze…}}
 % Czech translations of the proper texts

\newcommand{\annusEditionis}{2020}

\def\hebinitial#1{%
\leavevmode{\newbox\hebbox\setbox\hebbox\hbox{\hebfont{#1}\hskip 1mm}\kern -\wd\hebbox\hbox{\hebfont{#1}\hskip 1mm}}%
}

%%%% Vicekrat opakovane kousky

\newcommand{\anteOrationem}{
  \rubrica{Ante Orationem, cantatur a Superiore:}

  \pars{Supplicatio Litaniæ.}

  \cuminitiali{}{temporalia/supplicatiolitaniae.gtex}

  \pars{Oratio Dominica.}

  \cuminitiali{}{temporalia/oratiodominica.gtex}

  \rubrica{Deinde dicitur ab Hebdomadario:}

  \cuminitiali{}{temporalia/dominusvobiscum-solemnis.gtex}

  \rubrica{In choro monialium loco Dominus vobiscum dicitur:}

  \sineinitiali{temporalia/domineexaudi.gtex}
}

\setlength{\columnsep}{30pt} % prostor mezi sloupci

%%%%%%%%%%%%%%%%%%%%%%%%%%%%%%%%%%%%%%%%%%%%%%%%%%%%%%%%%%%%%%%%%%%%%%%%%%%%%%%%%%%%%%%%%%%%%%%%%%%%%%%%%%%%%
\begin{document}

% Here we set the space around the initial.
% Please report to http://home.gna.org/gregorio/gregoriotex/details for more details and options
\grechangedim{afterinitialshift}{2.2mm}{scalable}
\grechangedim{beforeinitialshift}{2.2mm}{scalable}

\grechangedim{interwordspacetext}{0.22 cm plus 0.15 cm minus 0.05 cm}{scalable}%
\grechangedim{annotationraise}{-0.2cm}{scalable}

% Here we set the initial font. Change 38 if you want a bigger initial.
% Emit the initials in red.
\grechangestyle{initial}{\color{red}\fontsize{38}{38}\selectfont}

\pagestyle{empty}

%%%% Titulni stranka
\begin{titulusOfficii}
\nomenFesti{Feria IV infra Hebdom. Ultima Adventus.}
\end{titulusOfficii}

\pars{}

\scriptura{}

\pagebreak

% graphic
\renewcommand{\headrulewidth}{0pt} % no horiz. rule at the header
\fancyhf{}
\pagestyle{fancy}

\cantusSineNeumas

\hora{Ad Matutinum.}

\vspace{2mm}

\cuminitiali{}{temporalia/dominelabiamea.gtex}

\vspace{2mm}

\pars{Invitatorium.} \scriptura{Phil. 4, 4.5}

\vspace{-6mm}

\antiphona{VI}{temporalia/inv-propeestiamsimplex.gtex}

\vfill
\pagebreak

\pars{Hymnus.}

\vspace{-5mm}

\antiphona{II}{temporalia/hym-VeniRedemptor.gtex}
%{
%\vspace{-5mm}
%\setlength{\columnsep}{0pt} % prostor mezi sloupci
%\input{hym-VeniRedemptor-bohtext.tex}
%\setlength{\columnsep}{30pt} % prostor mezi sloupci
%}

\vfill
\pagebreak

% MB

\pars{Psalmus 1.} \scriptura{Ps. 38, 2; \textbf{H93}}

\vspace{-4mm}

\antiphona{IV E}{temporalia/ant-utnondelinquam.gtex}

%\vspace{-5mm}

\scriptura{Ps. 38, 2-7}

%\vspace{-2mm}

\initiumpsalmi{temporalia/ps38i-initium-iv-E-auto.gtex}

%\psalmusEtTranslatioT{temporalia/ps38i-IV-comb.tex}{10cm}
\input{temporalia/ps38i-IV.tex} \Abardot{}

\vfill
\pagebreak

\pars{Psalmus 2.} \scriptura{Ier. 17, 17; \textbf{H177}}

\vspace{-4mm}

\antiphona{VII c}{temporalia/ant-nonsismihi.gtex}

%\vspace{-5mm}

\scriptura{Ps. 38, 8-14}

%\vspace{-2mm}

\initiumpsalmi{temporalia/ps38ii-initium-vii-c-auto.gtex}

%\psalmusEtTranslatioT{temporalia/ps38ii-IV-comb.tex}{10cm}
\input{temporalia/ps38ii-IV.tex} \Abardot{}

\vfill
\pagebreak

\pars{Psalmus 3.}

\vspace{-4mm}

\antiphona{IV e}{temporalia/ant-exspectabonomentuum.gtex}

%\vspace{-5mm}

\scriptura{Ps. 51}

%\vspace{-2mm}

\initiumpsalmi{temporalia/ps51-initium-iv-e-auto.gtex}

%\psalmusEtTranslatioT{temporalia/ps51-IV-comb.tex}{10cm}
\input{temporalia/ps51-IV.tex} \Abardot{}

\vfill
\pagebreak

\pars{Psalmus 4.} \scriptura{Ps. 102, 1; \textbf{H99}}

\vspace{-4mm}

\antiphona{VIII c}{temporalia/ant-benedicanimamea.gtex}

%\vspace{-5mm}

\scriptura{Ps. 102, 1-7}

%\vspace{-2mm}

\initiumpsalmi{temporalia/ps102i-initium-viii-C-auto.gtex}

%\psalmusEtTranslatioT{temporalia/ps102i-IV-comb.tex}{10cm}
\input{temporalia/ps102i-IV.tex} \Abardot{}

\vfill
\pagebreak

\pars{Psalmus 5.} \scriptura{Ps. 102, 11}

\vspace{-4mm}

\antiphona{I d}{temporalia/ant-supertimentesdominum.gtex}

%\vspace{-5mm}

\scriptura{Ps. 102, 8-16}

%\vspace{-2mm}

\initiumpsalmi{temporalia/ps102ii-initium-i-d-auto.gtex}

%\psalmusEtTranslatioT{temporalia/ps102ii-IV-comb.tex}{10cm}
\input{temporalia/ps102ii-IV.tex} \Abardot{}

\vfill
\pagebreak

\pars{Psalmus 6.} \scriptura{Ps. 102, 20; \textbf{H332}}

\vspace{-4mm}

\antiphona{III g}{temporalia/ant-benedicitedomino.gtex}

%\vspace{-5mm}

\scriptura{Ps. 102, 17-22}

%\vspace{-2mm}

\initiumpsalmi{temporalia/ps102iii-initium-iii-g.gtex}

%\psalmusEtTranslatioT{temporalia/ps102iii-IV-comb.tex}{10cm}
\input{temporalia/ps102iii-IV.tex} \Abardot{}

\vfill
\pagebreak

\pars{Versus.} \scriptura{Mc. 1, 3; Is. 40, 3}

% Versus. %%%
\sineinitiali{temporalia/versus-voxclamantis-simplex.gtex}

\vspace{5mm}

\sineinitiali{temporalia/oratiodominica-mat.gtex}

\vspace{5mm}

\pars{Absolutio.}

\cuminitiali{}{temporalia/absolutio-avinculis.gtex}

\vfill
\pagebreak

\cuminitiali{}{temporalia/benedictio-solemn-ille.gtex}

\vspace{7mm}

\lectioi

\noindent \Vbardot{} Tu autem, Dómine, miserére nobis.
\noindent \Rbardot{} Deo grátias.

\vfill
\pagebreak

\responsoriumi

\vfill
\pagebreak

\cuminitiali{}{temporalia/benedictio-solemn-divinum.gtex}

\vspace{7mm}

\lectioii

\noindent \Vbardot{} Tu autem, Dómine, miserére nobis.
\noindent \Rbardot{} Deo grátias.

\vfill
\pagebreak

\responsoriumii

\vfill
\pagebreak

\cuminitiali{}{temporalia/benedictio-solemn-ignem.gtex}

\vspace{7mm}

\lectioiii

\noindent \Vbardot{} Tu autem, Dómine, miserére nobis.
\noindent \Rbardot{} Deo grátias.

\vfill
\pagebreak

\responsoriumiii

\vfill
\pagebreak

\rubrica{Reliqua omittuntur, nisi Laudes separandæ sint.}

\pars{Oratio}

\noindent \Vbardot{} Dómine, exáudi oratiónem meam.

\noindent \Rbardot{} Et clamor meus ad te véniat.

\oratio

\vspace{7mm}

\pars{Conclusio}

\noindent \Vbardot{} Dómine, exáudi oratiónem meam.

\noindent \Rbardot{} Et clamor meus ad te véniat.

\noindent \Vbardot{} Benedicámus Dómino.

\noindent \Rbardot{} Deo grátias.

\noindent \Vbardot{} Fidélium ánimæ per misericórdiam Dei requiéscant in pace.

\noindent \Rbardot{} Amen.

\vfill
\pagebreak

\hora{Ad Laudes.} %%%%%%%%%%%%%%%%%%%%%%%%%%%%%%%%%%%%%%%%%%%%%%%%%%%%%
%\sideThumbs{Laudes}

\cantusSineNeumas

\vspace{0.5cm}
\grechangedim{interwordspacetext}{0.18 cm plus 0.15 cm minus 0.05 cm}{scalable}%
\cuminitiali{}{temporalia/deusinadiutorium-communis.gtex}
\grechangedim{interwordspacetext}{0.22 cm plus 0.15 cm minus 0.05 cm}{scalable}%

\vfill
%\pagebreak

\pars{Hymnus}

\cuminitiali{D}{temporalia/hym-MagnisProphetae.gtex}
\vspace{-3mm}
%\input{hym-MagnisProphetae-bohtext.tex}

\vfill
\pagebreak

\pars{Psalmus 1.} \scriptura{\textbf{H38}}

\vspace{-5mm}

\antiphona{I g\textsuperscript{3}}{temporalia/ant-desionvenietdominus.gtex}

\vspace{-2mm}

\scriptura{Psalmus 107.}

\vspace{-2mm}

\initiumpsalmi{temporalia/ps107-initium-i-g3-auto.gtex}

%\vspace{-1.5mm}

%\psalmusEtTranslatioT{temporalia/ps107-IV-comb.tex}{10cm}
\input{temporalia/ps107-IV.tex} \Abardot{}

\vfill
\pagebreak

\pars{Psalmus 2.} \scriptura{Is. 62, 1; \textbf{H38}}

\vspace{-4mm}

\antiphona{II* b}{temporalia/ant-proptersion.gtex}

%\vspace{-4mm}

\scriptura{Canticum Isaiæ, Is. 61, 10-11; Is. 62, 1-5}

%\vspace{-3mm}

\initiumpsalmi{temporalia/isaiae4-initium-ii_-B-auto.gtex}

%\psalmusEtTranslatioT{temporalia/isaiae4-comb.tex}{10cm}
\input{temporalia/isaiae4.tex}

%\vfill

\antiphona{}{temporalia/ant-proptersion.gtex}

\vfill
\pagebreak

\pars{Psalmus 3.} \scriptura{Lc. 4, 18; \textbf{H38}}

\vspace{-4mm}

\antiphona{II D}{temporalia/ant-spiritusdomini.gtex}

\scriptura{Psalmus 145.}

\initiumpsalmi{temporalia/ps145-initium-ii-D-auto.gtex}

%\psalmusEtTranslatioT{temporalia/ps145-IV-comb.tex}{10cm}
\input{temporalia/ps145-IV.tex} \Abardot{}

\vfill
\pagebreak

\lectiobrevis

% preklad Jeruz. bible
%\trCapituliI

\vfill

\responsoriumbreve

%\trResp

\vfill
\pagebreak

\benedictus

\vfill
\pagebreak

%\sideThumbs{{\scriptsize{}Fine horarum}}

\ifx\preces\undefined
\rubrica{Ante Orationem, cantatur a Superiore:}

\pars{Supplicatio Litaniæ.}

\cuminitiali{}{temporalia/supplicatiolitaniae.gtex}

\pars{Oratio Dominica.}

\cuminitiali{}{temporalia/oratiodominica.gtex}
\else
\pars{Preces.}

\sineinitiali{}{temporalia/tonusprecum.gtex}

\preces

\vfill

\pars{Oratio Dominica.}

\cuminitiali{}{temporalia/oratiodominicaalt.gtex}

\vfill
\pagebreak

\rubrica{vel:}

\pars{Supplicatio Litaniæ.}

\cuminitiali{}{temporalia/supplicatiolitaniae.gtex}

\vfill

\pars{Oratio Dominica.}

\cuminitiali{}{temporalia/oratiodominica.gtex}
\fi

\vfill
\pagebreak

% Oratio. %%%
\oratio

\vspace{-1mm}
%\trOrationisI

\vfill

\rubrica{Hebdomadarius dicit Dominus vobiscum, vel, absente sacerdote vel diacono, sic concluditur:}

\vspace{2mm}

\antiphona{C}{temporalia/dominusnosbenedicat.gtex}

\rubrica{Postea cantatur a cantore:}

\vspace{2mm}

\cuminitiali{IV}{temporalia/benedicamus-feria-advequad.gtex}

\vspace{1mm}

\vfill
\pagebreak

\hora{Ad Vesperas.} %%%%%%%%%%%%%%%%%%%%%%%%%%%%%%%%%%%%%%%%%%%%%%%%%%%%%
%\sideThumbs{Vesperæ}

\cantusSineNeumas

%\vspace{0.5cm}
\grechangedim{interwordspacetext}{0.18 cm plus 0.15 cm minus 0.05 cm}{scalable}%
\cuminitiali{}{temporalia/deusinadiutorium-communis.gtex}
\grechangedim{interwordspacetext}{0.22 cm plus 0.15 cm minus 0.05 cm}{scalable}%

\vfill
%\pagebreak

\vspace{4mm}

\pars{Psalmus 1.} \scriptura{\textbf{H38}}

\vspace{-4mm}

\antiphona{I g\textsuperscript{3}}{temporalia/ant-desionvenietdominus.gtex}

\vspace{-4mm}

\scriptura{Psalmus 134.}

\initiumpsalmi{temporalia/ps134-initium-i-g3-auto.gtex}

%\psalmusEtTranslatioT{temporalia/ps134-V-comb.tex}{10cm}
\input{temporalia/ps134-V.tex}

\vfill

\antiphona{}{temporalia/ant-desionvenietdominus.gtex}

\vfill
\pagebreak

\pars{Psalmus 2.} \scriptura{Lc. 4, 18; \textbf{H38}}

\vspace{-4mm}

\antiphona{II D}{temporalia/ant-spiritusdomini.gtex}

\scriptura{Psalmus 135.}

\initiumpsalmi{temporalia/ps135-initium-ii-D-auto.gtex}

%\psalmusEtTranslatioT{temporalia/ps135-V-comb.tex}{10cm}
\input{temporalia/ps135-V.tex}

\vfill

\antiphona{}{temporalia/ant-spiritusdomini.gtex}

\vfill
\pagebreak

\pars{Psalmus 3.} \scriptura{Is. 62, 1; \textbf{H38}}

\vspace{-4mm}

\antiphona{II* b}{temporalia/ant-proptersion.gtex}

\scriptura{Psalmus 136.}

\initiumpsalmi{temporalia/ps136-initium-ii_-B-auto.gtex}

%\psalmusEtTranslatioT{temporalia/ps136-V-comb.tex}{10cm}
\input{temporalia/ps136-V.tex} \Abardot{}

\vfill
\pagebreak

\pars{Psalmus 4.} \scriptura{Is. 42, 12; Hab. 2, 3; \textbf{H31}}

\vspace{-4mm}

\antiphona{V a}{temporalia/ant-ponentdomino.gtex}

\scriptura{Psalmus 137.}

\initiumpsalmi{temporalia/ps137-initium-v-a-auto.gtex}

%\psalmusEtTranslatioT{temporalia/ps137-V-comb.tex}{10cm}
\input{temporalia/ps137-V.tex} \Abardot{}

\vfill
\pagebreak

\pars{Capitulum.} \scriptura{Gen. 49, 10}

\grechangedim{interwordspacetext}{0.12 cm plus 0.15 cm minus 0.05 cm}{scalable}%
\cuminitiali{}{temporalia/capitulum-NosAuferetur.gtex}
\grechangedim{interwordspacetext}{0.22 cm plus 0.15 cm minus 0.05 cm}{scalable}%

% preklad Jeruz. bible
%\trCapituliI

\vfill

\pars{Responsorium breve.} \scriptura{Ps. 84, 8; \textbf{H20}}

\cuminitiali{IV}{temporalia/resp-ostendenobis.gtex}

%\trResp

\vfill
\pagebreak

\pars{Hymnus} \scriptura{Ambrosius (\olddag{} 397)}

\cuminitiali{IV}{temporalia/hym-VerbumSalutis.gtex}
\vspace{-3mm}
%\input{hym-VerbumSalutis-bohtext.tex}

\vfill
%\pagebreak

\pars{Versus.} \scriptura{Is. 45, 8}

% Versus. %%%
\sineinitiali{temporalia/versus-rorate.gtex}

%\noindent \trVersus

\vfill
\pagebreak

\magnificat

\vfill
\pagebreak

%\sideThumbs{{\scriptsize{}Fine horarum}}

\anteOrationem

\pagebreak

% Oratio. %%%
\oratioLaudes

\vspace{-1mm}
%\trOrationisI

\vfill

\rubrica{Hebdomadarius dicit iterum Dominus vobiscum, vel cantor dicit:}

\vspace{2mm}

\sineinitiali{temporalia/domineexaudi.gtex}

\rubrica{Postea cantatur a cantore:}

\vspace{2mm}

\cuminitiali{IV}{temporalia/benedicamus-feria-advequad.gtex}

\vspace{1mm}

\end{document}

