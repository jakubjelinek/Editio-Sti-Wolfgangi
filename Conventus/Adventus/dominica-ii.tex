\newcommand{\titulus}{\nomenFesti{Dominica II Adventus.}}
\newcommand{\oratio}{\pars{Oratio.}

\noindent Omnípotens et miséricors Deus, in tui occúrsum Fílii festinántes nulla ópera terréni actus impédiant, sed sapiéntiæ cæléstis erudítio nos fáciat eius esse consórtes.

\pars{Pro pace in universo mundo.} \scriptura{Sir. 50, 25; 2 Esdr. 4, 20; \textbf{H416}}

\vspace{-4mm}

\antiphona{II D}{temporalia/ant-dapacemdomine.gtex}

\vfill

\noindent Deus, a quo sancta desidéria, recta consília et iusta sunt ópera: da servis tuis illam, quam mundus dare non potest, pacem; ut et corda nostra mandátis tuis dédita, et hóstium subláta formídine, témpora sint tua protectióne tranquílla.

\noindent Per Dóminum nostrum Iesum Christum, Fílium tuum, qui tecum vivit et regnat in unitáte Spíritus Sancti, Deus, per ómnia sǽcula sæculórum.

%\noindent Qui tecum vivit et regnat in unitáte Spíritus Sancti, Deus, per ómnia sǽcula sæculórum.

\noindent \Rbardot{} Amen.}
\newcommand{\vesperasi}{\cantusSineNeumas

\vspace{0.5cm}
\grechangedim{interwordspacetext}{0.18 cm plus 0.15 cm minus 0.05 cm}{scalable}%
\cuminitiali{}{temporalia/deusinadiutorium-alter.gtex}
\grechangedim{interwordspacetext}{0.22 cm plus 0.15 cm minus 0.05 cm}{scalable}%

\vfill
\pagebreak

\pars{Psalmus 1.} \scriptura{Cf. Dn. 3, 17.18; Mc. 13, 26; \textbf{H23}}

\vspace{-4mm}

\antiphona{I g}{temporalia/ant-ecceinnubibus.gtex}

\scriptura{Psalmus 144, 10-21.}

\initiumpsalmi{temporalia/ps144ii-initium-i-g-auto.gtex}

\input{temporalia/ps144ii-i-g.tex} \Abardot{}

\vspace{-1cm}

\vfill
\pagebreak

\pars{Psalmus 2.} \scriptura{Is. 26, 1.2; \textbf{H24}}

\vspace{-4mm}

\antiphona{VII d}{temporalia/ant-urbsfortitudinis.gtex}

\scriptura{Psalmus 145.}

\initiumpsalmi{temporalia/ps145-initium-vii-d-auto.gtex}

\input{temporalia/ps145-vii-d.tex} \Abardot{}

\vfill
\pagebreak

\pars{Psalmus 3.} \scriptura{Hab. 2, 3; \textbf{H24}}

\vspace{-4mm}

\antiphona{VII a}{temporalia/ant-ecceapparebit.gtex}

%\vspace{-2mm}

\scriptura{Psalmus 146.}

%\vspace{-2mm}

\initiumpsalmi{temporalia/ps146-initium-vii-a-auto.gtex}

%\vspace{-1.5mm}

\input{temporalia/ps146-vii-a.tex}

\vfill

\antiphona{}{temporalia/ant-ecceapparebit.gtex}

\vfill
\pagebreak

\pars{Psalmus 4.} \scriptura{\textbf{H24}}

\vspace{-4mm}

\antiphona{III a}{temporalia/ant-eccedominusnoster.gtex}

\scriptura{Psalmus 147.}

\initiumpsalmi{temporalia/ps147-initium-iii-a-auto.gtex}

\input{temporalia/ps147-iii-a.tex} \Abardot{}

\vfill
\pagebreak

\pars{Capitulum.} \scriptura{Rom. 15, 4}

\cuminitiali{}{temporalia/capitulum-FratresQuaecumque.gtex}}
\newcommand{\magnificati}{\pars{Canticum B. Mariæ V.} \scriptura{\textbf{H21}}

\vspace{-4mm}

{
\grechangedim{interwordspacetext}{0.18 cm plus 0.15 cm minus 0.05 cm}{scalable}%
\antiphona{VII a}{temporalia/ant-venidominevisitare.gtex}
\grechangedim{interwordspacetext}{0.22 cm plus 0.15 cm minus 0.05 cm}{scalable}%
}

%\vspace{-3mm}

\scriptura{Lc. 1, 46-55}

%\vspace{-2mm}

\cantusSineNeumas
\initiumpsalmi{temporalia/magnificat-initium-viisoll-a.gtex}

%\vspace{-1.5mm}

\input{temporalia/magnificat-viisoll-a.tex} \Abardot{}}
\newcommand{\nocturnoii}{\vspace{-4mm}

\pars{Psalmus 4.} \scriptura{Cf. Dn. 3, 17.18; Mc. 13, 26; \textbf{H23}}

\vspace{-4mm}

\antiphona{I g}{temporalia/ant-ecceinnubibuscaeli.gtex}

%\vspace{-2mm}

\scriptura{Ps. 103, 1-12}

%\vspace{-2mm}

\initiumpsalmi{temporalia/ps103i-initium-i-g-auto.gtex}

\input{temporalia/ps103i-i-g.tex} \Abardot{}

\vfill
\pagebreak

\pars{Psalmus 5.} \scriptura{Cf. Is. 55, 12; \textbf{H24}}

\vspace{-4mm}

\antiphona{I f}{temporalia/ant-montesetcollescantabunt.gtex}

\vspace{-2mm}

\scriptura{Ps. 103, 13-23}

\vspace{-2mm}

\initiumpsalmi{temporalia/ps103ii-initium-i-f-auto.gtex}

\input{temporalia/ps103ii-i-f.tex} \Abardot{}

\vfill
\pagebreak

\pars{Psalmus 6.} \scriptura{Hab. 2, 3; \textbf{H24}}

\vspace{-4mm}

\antiphona{VII a}{temporalia/ant-ecceapparebit.gtex}

\scriptura{Ps. 103, 24-35}

\vspace{-2mm}

\initiumpsalmi{temporalia/ps103iii-initium-vii-a-auto.gtex}

%\vspace{-1.5mm}

\input{temporalia/ps103iii-vii-a.tex}

\vfill

\antiphona{}{temporalia/ant-ecceapparebit.gtex}

\vfill
\pagebreak}
\newcommand{\lectioi}{\pars{Lectio I.} \scriptura{Is. 22, 8-14}

\noindent De libro Isaíæ prophétæ.

\noindent Revelátum est operiméntum Iudæ et respexísti in die illa armamentárium domus Saltus; 

\noindent et scissúras civitátis David vidístis, quia multiplicátæ sunt; 

\noindent et congregástis aquas piscínæ inferióris. 

\noindent Et domos Ierúsalem numerástis et destruxístis domos ad muniéndum murum; 

\noindent et lacum fecístis inter duos muros pro aqua piscínæ véteris; 

\noindent sed non suspexístis ad eum, qui fecit hæc, et eum, qui hæc de longe formávit, non vidístis. 

\noindent Et vocávit Dóminus, Deus exercítuum, in die illa ad fletum et ad planctum, ad calvítium et ad cingéndum saccum; 

\noindent et ecce gáudium et lætítia occídere boves et iuguláre pecus, 

\noindent comédere carnes et bíbere vinum: «Comedámus et bibámus, cras enim moriémur». 

\noindent Et revelátum est in áuribus meis a Dómino exercítuum: 

\noindent «Certe non dimittétur iníquitas hæc vobis, donec moriámini!», dicit Dóminus, Deus exercítuum.}
\newcommand{\responsoriumi}{\pars{Responsorium 1.} \scriptura{\Rbar{} Mich. 4, 8.9 \Vbar{} Ps. 80, 9-11; \textbf{H21}}

\vspace{-5mm}

\responsorium{IV}{temporalia/resp-jerusalemcito-CROCHU.gtex}{}}
\newcommand{\lectioii}{\pars{Lectio II.} \scriptura{Is. 22, 15-19}

\noindent Hæc dicit Dóminus, Deus exercítuum: 

\noindent  «Vade, ingrédere ad procuratórem istum, ad Sobnam præpósitum palátii: 

\noindent  “Quid tibi hic? Aut quis tibi hic, quia excidísti tibi hic sepúlcrum?”. 

\noindent Effódiens in excélso sepúlcrum suum, excavábat in petra tabernáculum sibi. 

\noindent Ecce Dóminus veheménter proíciet te, homo, violénter te apprehéndens. 

\noindent In globum te convólvet glómerans; 

\noindent quasi pilam mittet te in terram latam et spatiósam: 

\noindent ibi moriéris, et ibi erunt currus glóriæ tuæ, ignomínia domus dómini tui. 

\noindent Et expéllam te de statióne tua et de ministério tuo depónam te.}
\newcommand{\responsoriumii}{\pars{Responsorium 2.} \scriptura{\Rbar{} Zach. 14, 5.7-9 \Vbar{} Ps. 106, 3; \textbf{H22}}

\vspace{-5mm}

\responsorium{V}{temporalia/resp-eccedominusveniet-CROCHU.gtex}{}}
\newcommand{\lectioiii}{\pars{Lectio III.} \scriptura{Is. 22, 20-23}

\noindent Et erit in die illa: 

\noindent Vocábo servum meum Elíachim fílium Helcíæ 

\noindent et índuam illum túnicam tuam 

\noindent et cíngulo tuo cingam eum 

\noindent et potestátem tuam dabo in manu eius; 

\noindent et erit in patrem habitántibus Ierúsalem et dómui Iudæ. 

\noindent Et dabo clavem domus David super úmerum eius; 

\noindent et apériet, et non erit qui claudat; 

\noindent et claudet, et non erit qui apériat. 

\noindent Et figam illum paxíllum in loco secúro, 

\noindent et erit in sólium glóriæ dómui patris sui».}
\newcommand{\responsoriumiii}{\pars{Responsorium 3.} \scriptura{\Rbar{} Bar. 5, 5; 4, 36 \Vbar{} Cf. Is. 49, 18; 60, 4; \textbf{H22}}

\vspace{-5mm}

\responsorium{I}{temporalia/resp-jerusalemsurge-CROCHU-cumdox.gtex}{}}
\newcommand{\lectioiv}{\pars{Lectio IV.} \scriptura{Cap. 40: PG 24, 3766-367}

\noindent Ex Commentáriis Eusébii Cæsariensis epíscopi in Isaíam.

\noindent \emph{Vox clamántis in desérto, paráte viam Dómini, rectas fácite sémitas Dei nostri.} 

\noindent Apérte declárat ea, quæ in vaticínio ferúntur, non Hierosólymæ, sed in desérto gerénda esse; 

\noindent nempe quod futúrum sit, ut glória Dómini appáreat, et salutáre Dei in omnis carnis notítiam véniat.

\noindent Et hæc quidem secúndum históriam et ad verbum, tunc impléta sunt, cum Ioánnes Baptísta salutárem Dei advéntum prædicávit in desérto Iordánis, ubi salutáre Dei visum fuit. 

\noindent Nam tunc Christus eiúsque glória ómnibus innótuit, cum, ipso baptizáto, apérti sunt cæli, et Spíritus Sanctus, in colúmbæ spécie descéndens, super eo insédit, patérnaque vox deláta est, Fílio testimónium reddens, \emph{Hic est Fílius meus diléctus, ipsum audíte.}

\noindent Hæc quippe dicebántur, quia Deus in desértum, a sǽculo impérvium et inaccéssum, adventúrus erat. 

\noindent Erant porro gentes omnes Dei cognitióne vácuæ, a quarum áditu omnes iusti Dei ac prophétæ arcebántur.}
\newcommand{\responsoriumiv}{\pars{Responsorium 4.} \scriptura{\Rbar{} Cantor \Vbar{} Is. 40, 10; \textbf{H22}}

\vspace{-5mm}

\responsorium{I}{temporalia/resp-civitasjerusalem-CROCHU.gtex}{}}
\newcommand{\lectiov}{\pars{Lectio V.}

\noindent Quámobrem iubet vox illa viam paráre Dei Verbo, et ínviam asperámque complanáre, ut en advéniens Deus noster prógredi váleat. 

\noindent \emph{Paráte viam Dómini:} ea est evangélica prædicátio nóvaque consolátio, quæ salutáre Dei in ómnium hóminum notítiam veníre exóptat.

\noindent \emph{Super montem excélsum ascénde, qui evangelízas Sion. Exálta in fortitúdine vocem tuam, qui evangelízas Ierúsalem.} 

\noindent Hæc præmissórum senténtiæ appríme convéniunt, atque opportúne evangelistárum mentiónem fáciunt, et advéntum Dei ad hómines annúntiant, postquam de voce in desérto clamánte sermo hábitus est. 

\noindent Etenim prophetíam de Ioánne Baptísta evangelistárum méntio congruénter sequebátur.}
\newcommand{\responsoriumv}{\pars{Responsorium 5.} \scriptura{\Rbar{} Is. 43, 14.15 \Vbar{} Cantor; \textbf{H22}}

\vspace{-5mm}

\responsorium{V}{temporalia/resp-eccevenietdominus-CROCHU.gtex}{}}
\newcommand{\lectiovi}{\pars{Lectio VI.}

\noindent Quænam ígitur hæc Sion est, nisi quæ ántea Ierúsalem vocabátur? 

\noindent Nam et ipsa mons erat, quod declárat Scriptúra illa quæ dicit: \emph{Mons Sion hic, in quo habitásti}; et Apóstolus: \emph{Accessístis ad Sion montem.} 

\noindent Num forte chorus apostólicus, ex prisco pópulo ex circumcisióne deléctus, hac ratióne significátur?

\noindent Hæc enim Sion et Ierúsalem est, quæ salutáre Dei accépit, quæ et ipsa monti Dei, vidélicet unigénito Verbo eius, sublímis impónitur: quam iubet, conscénso monte sublími, salutáre verbum annuntiáre. 

\noindent Quis autem ille est, qui evangelízat, nisi evangélicus chorus? 

\noindent Quid est evangelizáre? univérsis homínibus, et ante omnes, civitátibus Iuda, Christi in terram advéntum prædicáre.}
\newcommand{\responsoriumvi}{\pars{Responsorium 6.} \scriptura{\Rbar{} Is. 66, 13.14 \Vbar{} Cantor; \textbf{H22}}

\vspace{-5mm}

\responsorium{VIII}{temporalia/resp-sicutmater-CROCHU-cumdox.gtex}{}}
\newcommand{\evangelium}{
\pars{Versus.} \scriptura{Cf. Mch. 1, 3}

% Versus. %%%
\sineinitiali{temporalia/versus-egredietur.gtex}

\vspace{5mm}

\sineinitiali{temporalia/oratiodominica-mat.gtex}

\vspace{5mm}

\pars{Absolutio.}

\cuminitiali{}{temporalia/absolutio-avinculis.gtex}

\vfill
\pagebreak

\cuminitiali{}{temporalia/benedictio-solemn-evangelica.gtex}

\vspace{7mm}

\pars{Evangelium} \scriptura{Lc. 3, 1-6}

\noindent Léctio sancti Evangélii secúndum Lucam.

\noindent Anno quinto décimo impérii Tibérii Cǽsaris, procuránte Póntio Piláto Iudǽam, tetrárcha autem Galilǽæ Heróde, Philíppo autem fratre eius tetrárcha Iturǽæ et Trachonítidis regiónis, et Lysánia Abilínæ tetrárcha, sub príncipe sacerdótum Anna et Cáipha, factum est verbum Dei super Ioánnem Zacharíæ fílium in desérto.

\noindent Et venit in omnem regiónem circa Iordánem prǽdicans baptísmum pæniténtiæ in remissiónem peccatórum, sicut scriptum est in libro sermónum Isaíæ prophétæ: «Vox clamántis in desérto: “Paráte viam Dómini, rectas fácite sémitas eius. Omnis vallis implébitur, et omnis mons et collis humiliábitur; et erunt prava in dirécta, et áspera in vias planas: et vidébit omnis caro salutáre Dei”».

\scriptura{Hom. 22,1-3: SC 87, 300-302}

\noindent Ex Homíliis Orígenis presbýteri in Lucam.

\noindent Videámus quæ in Christi prædicéntur advéntu, inter quæ primum de Ioánne scríbitur: \emph{Vox clamántis in desérto: Paráte viam Dómini, rectas fácite sémitas eius.} Et quod séquitur, próprie de Dómino Salvatóre est. Neque enim a Ioánne \emph{omnis vallis impléta est,} sed a Dómino Salvatóre. Seípsum unusquísque consíderet quis erat, ántequam créderet, et tunc animadvértet vallem húmilem, vallem se fuísse præcípitem et in ima demérsam. Quando vero venit Dóminus Iesus et misit Spíritum Sanctum vicárium suum, \emph{vallis omnis expléta est.} Expléta est autem opéribus bonis et frúctibus Spíritus Sancti. Cáritas non sinit permanére in te vallem, quod, si pacem habúeris et patiéntiam et bonitátem, non solum vallis esse desístes, sed étiam mons esse incípies Dei.

\noindent Tam de géntibus cotídie fíeri vidémus atque compléri: \emph{Omnis vallis implébitur,} quam de pópulo Israel, qui de excélso depósitus est: \emph{Omnis mons et collis humiliábitur.} Illórum delícto salus géntibus data est, ad æmulándum eos. Quod si et contrárias fortitúdines, quæ advérsus mortáles erigebántur, díxeris montes et colles esse depósitos, non peccábis. Ut enim impleántur huiuscémodi valles, contráriæ fortitúdines, montes et colles, humiliándæ sunt.

\noindent Sed et hoc quod in advéntu Christi prophetátum est, utrum explétum sit, contemplémur. Séquitur enim: \emph{et ómnia prava erunt in dirécta.} Unusquísque nostrum pravus erat, si tamen erat et non usque hódie persevérat, et per advéntum Christi, qui factus est ad ánimam nostram, prava quæque dirécta sunt. Quid enim tibi prodest, si Christum quondam venit in carne, nisi ad tuam quoque ánimam vénerit?

\noindent Orémus ut illíus cotídie nobis advéntus fiat et possímus dícere: \emph{Vivo autem, iam non ego, vivit autem in me Christus.} Si enim Christus vivit in Paulo et non vivit in me, quid mihi próderit? Cum autem et ad me vénerit et frúitus illo fúero, sicut frúitus est Paulus, tunc et ego possum Paulo simíliter loqui: \emph{Vivo, iam non ego, vivit autem in me Christus.}

\vfill
\pagebreak

\pars{Responsorium 7.} \scriptura{\Rbar{} Cantor \Vbar{} Io. 1, 29; \textbf{H23}}

\vspace{-5mm}

\responsorium{IV}{temporalia/resp-rexnosteradveniet-CROCHU-cumdox.gtex}{}

\vfill
\pagebreak

{
\pars{Hymnus Ambrosianus} \scriptura{Tonus Monasticus}

\vspace{-2mm}

\grechangedim{interwordspacetext}{0.26 cm plus 0.15 cm minus 0.05 cm}{scalable}%
\cuminitiali{III}{temporalia/tedeum-monasticum-am34.gtex}
\grechangedim{interwordspacetext}{0.22 cm plus 0.15 cm minus 0.05 cm}{scalable}%
}

\vfill
\pagebreak}
\newcommand{\laudes}{\pars{Psalmus 1.} \scriptura{Is. 26, 1.2; \textbf{H24}}

\vspace{-4mm}

\antiphona{VII d}{temporalia/ant-urbsfortitudinisnostrae.gtex}

\scriptura{Psalmus 117}

%\vspace{-2mm}

\initiumpsalmi{temporalia/ps117-initium-vii-d-auto.gtex}

%\vspace{-1.5mm}

\input{temporalia/ps117-vii-d.tex}

\vfill

\antiphona{}{temporalia/ant-urbsfortitudinisnostrae.gtex}

\vfill
\pagebreak

\pars{Psalmus 2.} \scriptura{Is. 55, 1.6; \textbf{H18}}

\vspace{-4mm}

\antiphona{VII c}{temporalia/ant-omnessitientesvenite.gtex}

\scriptura{Canticum Danielis, Dan. 3, 52-57}

%\vspace{-3mm}

\initiumpsalmi{temporalia/dan33-initium-vii-c-auto.gtex}

\input{temporalia/dan33-vii-c.tex} \Abardot{}

\vfill
\pagebreak

\pars{Psalmus 3.} \scriptura{\textbf{H24}}

\vspace{-4mm}

\antiphona{III a}{temporalia/ant-eccedominusnostercum.gtex}

\scriptura{Psalmus 150}

\initiumpsalmi{temporalia/ps150-initium-iii-a-auto.gtex}

\input{temporalia/ps150-iii-a.tex} \Abardot{}

\vfill
\pagebreak}
\newcommand{\lectiobrevis}{\pars{Lectio Brevis.} \scriptura{Rom. 13, 11-12}

\noindent Hora est iam vos de somno súrgere, nunc enim própior est nobis salus quam cum credídimus. Nox procéssit, dies autem appropiávit. Abiciámus ergo ópera tenebrárum et induámur arma lucis.}
\newcommand{\benedictus}{\pars{Canticum Zachariæ.} \scriptura{Lc. 3, 2-3}

\vspace{-4mm}

{
\grechangedim{interwordspacetext}{0.18 cm plus 0.15 cm minus 0.05 cm}{scalable}%
\antiphona{VIII G\textsuperscript{2}}{temporalia/ant-factumestverbum.gtex}
\grechangedim{interwordspacetext}{0.22 cm plus 0.15 cm minus 0.05 cm}{scalable}%
}

\vspace{-2mm}

\scriptura{Lc. 1, 68-79}

\vspace{-2mm}

\cantusSineNeumas
\initiumpsalmi{temporalia/benedictus-initium-viiisoll-g5-auto.gtex}

%\vspace{-1.5mm}

\input{temporalia/benedictus-viiisoll-g5.tex} \Abardot{}}
\newcommand{\preces}{\noindent Dóminum Iesum Christum, fratres caríssimi, deprecémur,~\gredagger{} qui est iudex vivórum et mortuórum,~\grestar{} ipsi dicéntes:

\Rbardot{} Veni, Dómine Iesu.

\noindent Christe Dómine, qui peccatóres salváre venísti,~\grestar{} nos ab omni tentatiónum adversitáte defénde.

\Rbardot{} Veni, Dómine Iesu.

\noindent Qui ad iudícium maniféste ventúrus esse créderis,~\grestar{} poténtiam tuæ salvatiónis in nobis osténde.

\Rbardot{} Veni, Dómine Iesu.

\noindent Da nobis legis tuæ præcépta virtúte spíritus custodíre,~\grestar{} ut advéntum tuum in caritáte præstolári possímus.

\Rbardot{} Veni, Dómine Iesu.

\noindent Tu, qui es benedíctus in sǽcula,~\gredagger{} fac ut per misericórdiam tuam pie et sóbrie in hoc sǽculo vivámus,~\grestar{} exspectántes beátam spem et advéntum magnificéntiæ tuæ.

\Rbardot{} Veni, Dómine Iesu.}
\newcommand{\vesperasii}{
\cantusSineNeumas

%\vspace{0.5cm}
\grechangedim{interwordspacetext}{0.18 cm plus 0.15 cm minus 0.05 cm}{scalable}%
\cuminitiali{}{temporalia/deusinadiutorium-alter.gtex}
\grechangedim{interwordspacetext}{0.22 cm plus 0.15 cm minus 0.05 cm}{scalable}%

\vfill
%\pagebreak

\vspace{-2mm}

\pars{Psalmus 1.} \scriptura{Cf. Dn. 3, 17.18; Mc. 13, 26; \textbf{H23}}

\vspace{-4mm}

\antiphona{I g}{temporalia/ant-ecceinnubibus.gtex}

\vspace{-2mm}

\scriptura{Psalmus 109.}

\vspace{-1mm}

\initiumpsalmi{temporalia/ps109-initium-i-g-auto.gtex}

\input{temporalia/ps109-i-g.tex} \Abardot{}

\vspace{-1cm}

\vfill
\pagebreak

\pars{Psalmus 2.} \scriptura{Is. 26, 1.2; \textbf{H24}}

\vspace{-4mm}

\antiphona{VII d}{temporalia/ant-urbsfortitudinis.gtex}

\scriptura{Psalmus 110.}

\initiumpsalmi{temporalia/ps110-initium-vii-d-auto.gtex}

\input{temporalia/ps110-vii-d.tex} \Abardot{}

\vfill
\pagebreak

\pars{Psalmus 3.} \scriptura{Hab. 2, 3; \textbf{H24}}

\vspace{-4mm}

\antiphona{VII a}{temporalia/ant-ecceapparebit.gtex}

\scriptura{Psalmus 111.}

\initiumpsalmi{temporalia/ps111-initium-vii-a-auto.gtex}

\input{temporalia/ps111-vii-a.tex} \Abardot{}

\vfill
\pagebreak

\pars{Psalmus 4.} \scriptura{\textbf{H24}}

\vspace{-4mm}

\antiphona{III a}{temporalia/ant-eccedominusnoster.gtex}

\scriptura{Psalmus 112.}

\initiumpsalmi{temporalia/ps112-initium-iii-a-auto.gtex}

\input{temporalia/ps112-iii-a.tex} \Abardot{}

\vfill
\pagebreak

\pars{Capitulum.} \scriptura{Rom. 15, 4}

\cuminitiali{}{temporalia/capitulum-FratresQuaecumque.gtex}}
\newcommand{\magnificatii}{\pars{Canticum B. Mariæ V.} \scriptura{Cf. Mt. 11, 3.5; \textbf{H30}}

\vspace{-6.5mm}

{
\grechangedim{interwordspacetext}{0.18 cm plus 0.15 cm minus 0.05 cm}{scalable}%
\antiphona{VIII G\textsuperscript{2}}{temporalia/ant-tuesquiventurus.gtex}
\grechangedim{interwordspacetext}{0.22 cm plus 0.15 cm minus 0.05 cm}{scalable}%
}

%\trAntIMagnificat

\vspace{-3mm}

\scriptura{Lc. 1, 46-55}

\vspace{-2mm}

\cantusSineNeumas
\initiumpsalmi{temporalia/magnificat-initium-viiisoll-G2.gtex}

\vspace{-1.5mm}

\input{temporalia/magnificat-viiisoll-G2.tex} \Abardot{}}
\newcommand{\hebdomada}{infra Hebdom. II per Annum.}
\newcommand{\matub}{Matutinum Hebdomadae B}
\newcommand{\laudb}{Laudes Hebdomadae B}
\newcommand{\laudbd}{Laudes Hebdomadae B vel D}

% LuaLaTeX

\documentclass[a4paper, twoside, 12pt]{article}
\usepackage[latin]{babel}
%\usepackage[landscape, left=3cm, right=1.5cm, top=2cm, bottom=1cm]{geometry} % okraje stranky
%\usepackage[landscape, a4paper, mag=1166, truedimen, left=2cm, right=1.5cm, top=1.6cm, bottom=0.95cm]{geometry} % okraje stranky
\usepackage[landscape, a4paper, mag=1400, truedimen, left=0.5cm, right=0.5cm, top=0.5cm, bottom=0.5cm]{geometry} % okraje stranky

\usepackage{fontspec}
\setmainfont[FeatureFile={junicode.fea}, Ligatures={Common, TeX}, RawFeature=+fixi]{Junicode}
%\setmainfont{Junicode}

% shortcut for Junicode without ligatures (for the Czech texts)
\newfontfamily\nlfont[FeatureFile={junicode.fea}, Ligatures={Common, TeX}, RawFeature=+fixi]{Junicode}

\usepackage{multicol}
\usepackage{color}
\usepackage{lettrine}
\usepackage{fancyhdr}

% usual packages loading:
\usepackage{luatextra}
\usepackage{graphicx} % support the \includegraphics command and options
\usepackage{gregoriotex} % for gregorio score inclusion
\usepackage{gregoriosyms}
\usepackage{wrapfig} % figures wrapped by the text
\usepackage{parcolumns}
\usepackage[contents={},opacity=1,scale=1,color=black]{background}
\usepackage{tikzpagenodes}
\usepackage{calc}
\usepackage{longtable}
\usetikzlibrary{calc}

\setlength{\headheight}{14.5pt}

\input{conventuscommune.tex} % Often used macros
%%%% Preklady jednotlivych zpevu (nektere se opakuji, a je dobre mit je
% vsechny na jedne hromade)

% HOURS ---

\newcommand{\trAntI}{\translatioCantus{Muž boží měl kožený toulec, pečlivě
zavázaný, jenž mu visel na šíji a~často se ho dotýkal.}}

\newcommand{\trAntII}{\translatioCantus{Klíč od~něho tak dobře střežil, že
dokud žil v~těle, nikdo z~jeho žáků nezvěděl, co je uvnitř.}}

\newcommand{\trAntIII}{\translatioCantus{Ale když se odebral z~tohoto
života, schránku otevřeli a~objevili v~ní žíněné roucho a~měděný řetěz
potřísněný krví.}}

\newcommand{\trAntIV}{\translatioCantus{A když prohlédli mistrovo tělo,
nalezli jeho tělo na čtyřech místech hluboce zbrázděno ranami od řetězu.}}

\newcommand{\trAntV}{\translatioCantus{Krev vytékající z~těch ran, místy
prostoupila i~žíněným rouchem.}}

\newcommand{\trCapituli}{\translatioCantus{
Miláčkovi Boha a~lidí,
Mojžíšovi požehnané paměti,~\gredagger{}
dopřál slávu rovnou slávě svatých~\grestar{}
učinil ho mocným na postrach nepřátelům
a~jeho slovy zastavil divy.}}

\newcommand{\trLectioBrevis}{\translatioCantus{
Pamatujte na své představené,
kteří vám hlásali Boží slovo.
Uvažte, jak oni skončili život, a~napodobujte jejich víru.
Ježíš Kristus je stejný včera i~dnes i~navěky.
Nenechte se svést věelijakými cizími naukami.}}

\newcommand{\trRespLaud}{\translatioCantus{Spravedlivého vodil Hospodin~\grestar{}
po přímých stezkách. \Vbardot{} A~ukázal mu Boží království.}}

\newcommand{\trRespLaudB}{\translatioCantus{Na tvých hradbách, Jeruzaléme,
ustanovil jsem strážné;~\grestar{}
budou bdít nad mým lidem. \Vbardot{} Ani ve dne, ani v~noci nesmějí nikdy
mlčet.}}

\newcommand{\trVersus}{\translatioCantus{\Vbardot{} Ústa spravedlivého šeptají moudrost, aleluja.
\Rbardot{} A~jeho jazyk ohlašuje právo, aleluja.}}

\newcommand{\trAntBenedictus}{\translatioCantus{Když na bujné oře vložili
nosítka a~sňali jim uzdu, vydali se přímo k~cele božího muže.}}

\newcommand{\trPreces}{\translatioCantus{
\noindent S vděčností chvalme Krista, dobrého Pastýře, \gredagger{} který dal život za své ovce, \grestar{} a~pokorně ho prosme: \Rbardot{} Pane, buď pastýřem svého lidu.

\noindent Kriste, ty dáváš církvi pastýře, a~jejich službou se ujímáš svého lidu, \grestar{} dej, ať v~lásce těch, kteří nás vedou, poznáváme, jak nás miluješ. \Rbardot{} Pane, buď pastýřem svého lidu.

\noindent Ty stále konáš skrze své zástupce službu pastýře a~učitele, \grestar{} nepřestávej nás nikdy vést prostřednictvím svých služebníků. \Rbardot{} Pane, buď pastýřem svého lidu.

\noindent Ty prokazuješ svému lidu skrze jeho pastýře službu lékaře duše i~těla, \grestar{} ochraňuj náš život a~veď nás ke svatosti. \Rbardot{} Pane, buď pastýřem svého lidu.

\noindent Ty posíláš své svaté, aby slovem i~příkladem vedli tvůj lid k~tobě, \grestar{} na jejich přímluvu nás posiluj, abychom vytrvali na cestě, která vede k~věčnému životu. \Rbardot{} Pane, buď pastýřem svého lidu.}}

\newcommand{\trOrationis}{\translatioCantus{Bože, jenž nám dopřáváš radovat
se z~výroční slavnosti svatého tvého vyznavače Havla, uděl dobrotivě,
abychom když slavíme jeho narození, též se řídili podobou jeho skutků.
Skrze…}}
 % Czech translations of the proper texts

\newcommand{\annusEditionis}{2020}

%%%% Vicekrat opakovane kousky

\newcommand{\anteOrationem}{
  \rubrica{Ante Orationem, cantatur a Superiore:}

  \pars{Supplicatio Litaniæ.}

  \cuminitiali{}{temporalia/supplicatiolitaniae.gtex}

  \pars{Oratio Dominica.}

  \cuminitiali{}{temporalia/oratiodominica.gtex}

  \rubrica{Deinde dicitur ab Hebdomadario:}

  \cuminitiali{}{temporalia/dominusvobiscum-solemnis.gtex}

  \rubrica{In choro monialium loco Dominus vobiscum dicitur:}

  \sineinitiali{temporalia/domineexaudi.gtex}
}

\setlength{\columnsep}{30pt} % prostor mezi sloupci

%%%%%%%%%%%%%%%%%%%%%%%%%%%%%%%%%%%%%%%%%%%%%%%%%%%%%%%%%%%%%%%%%%%%%%%%%%%%%%%%%%%%%%%%%%%%%%%%%%%%%%%%%%%%%
\begin{document}

% Here we set the space around the initial.
% Please report to http://home.gna.org/gregorio/gregoriotex/details for more details and options
\grechangedim{afterinitialshift}{2.2mm}{scalable}
\grechangedim{beforeinitialshift}{2.2mm}{scalable}
\grechangedim{interwordspacetext}{0.22 cm plus 0.15 cm minus 0.05 cm}{scalable}%
\grechangedim{annotationraise}{-0.2cm}{scalable}

% Here we set the initial font. Change 38 if you want a bigger initial.
% Emit the initials in red.
\grechangestyle{initial}{\color{red}\fontsize{38}{38}\selectfont}

\pagestyle{empty}

%%%% Titulni stranka
\begin{titulusOfficii}
\titulus{}
\end{titulusOfficii}

% graphic
%\vspace{1.5cm}
%\begin{center}
%\includegraphics[width=8cm]{emmaus.jpg}
%\end{center}

\vfill

\begin{center}
%Ad usum et secundum consuetudines chori \guillemotright{}Conventus Choralis\guillemotleft.

%Editio Sancti Wolfgangi \annusEditionis
\end{center}

\pagebreak

\renewcommand{\headrulewidth}{0pt} % no horiz. rule at the header
\fancyhf{}
\pagestyle{fancy}

\pars{Oratio ante divinum Officium.}

\lettrine{{\color{red}A}}{peri,} Dómine, os meum ad benedicéndum nomen sanctum tuum:
munda quoque cor meum ab ómnibus vanis, pervérsis, et aliénis
cogitatiónibus:
intelléctum illúmina, afféctum inflámma,
ut digne, atténte ac devóte hoc Offícium recitáre váleam,
et exaudíri mérear ante conspéctum Divínæ Maiestátis tuæ.
Per Christum, Dóminum nostrum.
\Rbardot{} Amen.

Dómine, in unióne illíus divínæ intentiónis,
qua ipse in terris laudes Deo persolvísti,
has tibi Horas \rubricatum{(vel \textnormal{hanc tibi Horam})} persólvo.

%\trOratioAnteOfficium

\vfill

\pars{Oratio post divinum Officium.}

\rubrica{
  Orationem sequentem devote post Officium recitantibus
  Leo Papa X. defectus, et culpas in eo persolvendo ex humana
  fragilitate contractas, indulsit, et dicitur flexis genibus.
}

\lettrine{{\color{red}S}}{acrosánctæ} et indivíduæ Trinitáti,
crucifíxi Dómini nostri Iesu Christi humanitáti,
beatíssimæ et gloriosíssimæ sempérque Vírginis Maríæ
fecúndæ integritáti, 
et ómnium Sanctórum universitáti
sit sempitérna laus, honor, virtus et glória
ab omni creatúra,
nobísque remíssio ómnium peccatórum,
per infiníta sǽcula sæculórum.
\Rbardot{} Amen.

\noindent \Vbardot{} Beáta víscera Maríæ Virginis, quæ portavérunt
ætérni Patris Fílium.\\
\Rbardot{} Et beáta úbera, quæ lactavérunt Christum Dominum.

\rubrica{Et dicitur secreto \textnormal{Pater noster.} et \textnormal{Ave María.}}

%\trOratioPostOfficium

\vfill

\hora{Ad I. Vesperas.} %%%%%%%%%%%%%%%%%%%%%%%%%%%%%%%%%%%%%%%%%%%%%%%%%%%%%
%\sideThumbs{I. Vesperæ}

\cantusSineNeumas

\vspace{0.5cm}
\grechangedim{interwordspacetext}{0.18 cm plus 0.15 cm minus 0.05 cm}{scalable}%
\cuminitiali{}{temporalia/deusinadiutorium-solemnis.gtex}
\grechangedim{interwordspacetext}{0.22 cm plus 0.15 cm minus 0.05 cm}{scalable}%

\vfill
\pagebreak

\pars{Psalmus 1.} \scriptura{Ps. 144, 13; \textbf{H100}}

\vspace{-4mm}

\antiphona{VII c\textsuperscript{2}}{temporalia/ant-regnumtuum.gtex}

\scriptura{Psalmus 144, 10-21.}

\initiumpsalmi{temporalia/ps144ii-initium-vii-c2-auto.gtex}

%\psalmusEtTranslatioT{temporalia/ps144ii-VII-comb.tex}{10cm}
\input{temporalia/ps144ii-VII.tex} \Abardot{}

\vspace{-1cm}

\vfill
\pagebreak

\pars{Psalmus 2.} \scriptura{Ps. 145, 2; \textbf{H100}}

\vspace{-4mm}

\antiphona{IV E}{temporalia/ant-laudabodeum.gtex}

\scriptura{Psalmus 145.}

\initiumpsalmi{temporalia/ps145-initium-iv-E-auto.gtex}

%\psalmusEtTranslatioT{temporalia/ps145-VII-comb.tex}{10cm}
\input{temporalia/ps145-VII.tex} \Abardot{}

\vfill
\pagebreak

\pars{Psalmus 3.} \scriptura{Ps. 146, 1; \textbf{H101}}

\vspace{-4mm}

\antiphona{VIII a}{temporalia/ant-deonostro.gtex}

\scriptura{Psalmus 146.}

\initiumpsalmi{temporalia/ps146-initium-viii-A-auto.gtex}

%\psalmusEtTranslatioT{temporalia/ps146-VII-comb.tex}{10cm}
\input{temporalia/ps146-VII.tex} \Abardot{}

\vfill
\pagebreak

\pars{Psalmus 4.} \scriptura{Ps. 147, 1}

\vspace{-4mm}

\antiphona{E}{temporalia/ant-laudajerusalem.gtex}

\scriptura{Psalmus 147.}

\initiumpsalmi{temporalia/ps147-initium-e-auto.gtex}

%\psalmusEtTranslatioT{temporalia/ps147-VII-comb.tex}{10cm}
\input{temporalia/ps147-VII.tex} \Abardot{}

\vfill
\pagebreak

\pars{Capitulum.} \scriptura{Rom. 11, 33}

\grechangedim{interwordspacetext}{0.12 cm plus 0.15 cm minus 0.05 cm}{scalable}%
\cuminitiali{}{temporalia/capitulum-OAltitudo.gtex}
\grechangedim{interwordspacetext}{0.22 cm plus 0.15 cm minus 0.05 cm}{scalable}

% preklad Jeruz. bible
%\trCapituliI

\vfill

\pars{Responsorium breve.} \scriptura{Ps. 146, 5}

\cuminitiali{VI}{temporalia/resp-magnusdominusnoster.gtex}

%\trResp

\vfill
\pagebreak

\pars{Hymnus} \scriptura{Ambrosius (\olddag{} 397)}

\cuminitiali{I}{temporalia/hym-OLuxBeata-aestivalis.gtex}
\vspace{-3mm}
%\input{hym-OLuxBeata-bohtext.tex}

\vfill
%\pagebreak

\pars{Versus.}

% Versus. %%%
\sineinitiali{temporalia/versus-vespertina.gtex}

%\noindent \trVersus

\vfill
\pagebreak

\magnificati

\vfill
\pagebreak

%\sideThumbs{{\scriptsize{}Fine horarum}}

\anteOrationem

\pagebreak

% Oratio. %%%
\oratioLaudes

\vspace{-1mm}
%\trOrationisI

\vfill

\rubrica{Hebdomadarius dicit iterum Dominus vobiscum, vel cantor dicit:}

\vspace{2mm}

\sineinitiali{temporalia/domineexaudi.gtex}

\rubrica{Postea cantatur a cantore:}

\vspace{2mm}

\cuminitiali{I}{temporalia/benedicamus-dominica-perannum.gtex}

\vspace{1mm}

\vfill
\pagebreak

\hora{Ad Matutinum.} %%%%%%%%%%%%%%%%%%%%%%%%%%%%%%%%%%%%%%%%%%%%%%%%%%%%%
%\sideThumbs{Matutinum}

\vspace{2mm}

\cuminitiali{}{temporalia/dominelabiamea.gtex}

\vspace{2mm}

\pars{Invitatorium.} \scriptura{Ps. 94, 1; Psalmus 94}

\vspace{-6mm}

\antiphona{E}{temporalia/inv-veniteexsultemus.gtex}

\vfill
\pagebreak

\pars{Hymnus.} \scriptura{Adamus Sancti Victoris (\olddag 1146)}

\vspace{-5mm}

\antiphona{VII}{temporalia/hym-SalveDies.gtex}

\scriptura{Non dicitur \textnormal{Amen} in fine.}
%{
%\vspace{-5mm}
%\setlength{\columnsep}{0pt} % prostor mezi sloupci
%\input{hym-SalveDies-bohtext.tex}
%\setlength{\columnsep}{30pt} % prostor mezi sloupci
%}

\vfill
\pagebreak

\subhora{In I. Nocturno}

\pars{Psalmus 1.} \scriptura{Ps. 1, 1}

\vspace{-4mm}

\antiphona{VIII G}{temporalia/ant-beatusvir.gtex}

%\vspace{-5mm}

\scriptura{Ps. 1}

%\vspace{-2mm}

\initiumpsalmi{temporalia/ps1-initium-viii-G-auto.gtex}

%\psalmusEtTranslatioT{temporalia/ps1-I-comb.tex}{10cm}
\input{temporalia/ps1-I.tex} \Abardot{}

\vfill
\pagebreak

\pars{Psalmus 2.} \scriptura{Ps. 2, 11; \textbf{H93}}

\vspace{-4mm}

\antiphona{VII a}{temporalia/ant-servitedomino.gtex}

\vspace{-3mm}

\scriptura{Ps. 2}

\vspace{-2mm}

\initiumpsalmi{temporalia/ps2-initium-vii-a-auto.gtex}

%\psalmusEtTranslatioT{temporalia/ps2-I-comb.tex}{10cm}
\input{temporalia/ps2-I.tex} \Abardot{}

\vfill
\pagebreak

\pars{Psalmus 3.} \scriptura{Ps. 3, 7}

\vspace{-4mm}

\antiphona{VI F}{temporalia/ant-exsurgedominesalvum.gtex}

%\vspace{-5mm}

\scriptura{Ps. 3}

\initiumpsalmi{temporalia/ps3-initium-vi-F-auto.gtex}

%\psalmusEtTranslatioT{temporalia/ps3-I-comb.tex}{10cm}
\input{temporalia/ps3-I.tex} \Abardot{}

\vfill
\pagebreak

\pars{Versus.} \scriptura{Ps. 118, 55}

% Versus. %%%
\sineinitiali{temporalia/versus-memorfui.gtex}

\vspace{5mm}

\sineinitiali{temporalia/oratiodominica-mat.gtex}

\vspace{5mm}

\pars{Absolutio.}

\cuminitiali{}{temporalia/absolutio-exaudi.gtex}

\vfill
\pagebreak

\cuminitiali{}{temporalia/benedictio-solemn-benedictione.gtex}

\vspace{7mm}

\lectioi

\noindent \Vbardot{} Tu autem, Dómine, miserére nobis.
\noindent \Rbardot{} Deo grátias.

\vfill
\pagebreak

\responsoriumi

\vfill
\pagebreak

\cuminitiali{}{temporalia/benedictio-solemn-unigenitus.gtex}

\vspace{7mm}

\lectioii

\noindent \Vbardot{} Tu autem, Dómine, miserére nobis.
\noindent \Rbardot{} Deo grátias.

\vfill
\pagebreak

\responsoriumii

\vfill
\pagebreak

\cuminitiali{}{temporalia/benedictio-solemn-spiritus.gtex}

\vspace{7mm}

\lectioiii

\noindent \Vbardot{} Tu autem, Dómine, miserére nobis.
\noindent \Rbardot{} Deo grátias.

\vfill
\pagebreak

\responsoriumiii

\vfill
\pagebreak

\subhora{In II. Nocturno}

\pars{Psalmus 4.} \scriptura{Ps. 8, 2}

\vspace{-4mm}

\antiphona{I g}{temporalia/ant-quamadmirabileest.gtex}

%\vspace{-5mm}

\scriptura{Ps. 8}

%A\vspace{-2mm}

\initiumpsalmi{temporalia/ps8-initium-i-g-auto.gtex}

%\psalmusEtTranslatioT{temporalia/ps8-I-comb.tex}{10cm}
\input{temporalia/ps8-I.tex} \Abardot{}

\vfill
\pagebreak

\pars{Psalmus 5.} \scriptura{Ps. 9, 5}

\vspace{-4mm}

\antiphona{VIII G}{temporalia/ant-sedistisuperthronum.gtex}

%\vspace{-5mm}

\scriptura{Ps. 9, 2-11}

\initiumpsalmi{temporalia/ps9ii_xi-initium-viii-G-auto.gtex}

%\psalmusEtTranslatioT{temporalia/ps9ii_xi-I-comb.tex}{10cm}
\input{temporalia/ps9ii_xi-I.tex} \Abardot{}

\vfill
\pagebreak

\pars{Psalmus 6.} \scriptura{Ps. 9, 20}

\vspace{-4mm}

\antiphona{I g\textsuperscript{3}}{temporalia/ant-exsurgedominenon.gtex}

%\vspace{-5mm}

\scriptura{Ps. 9, 12-21}

\initiumpsalmi{temporalia/ps9xii_xxi-initium-i-g3-auto.gtex}

%\psalmusEtTranslatioT{temporalia/ps9xii_xxi-I-comb.tex}{10cm}
\input{temporalia/ps9xii_xxi-I.tex} \Abardot{}

\vfill
\pagebreak

\pars{Versus.} \scriptura{Ps. 118, 62}

% Versus. %%%
\sineinitiali{temporalia/versus-medianocte.gtex}

\vspace{5mm}

\sineinitiali{temporalia/oratiodominica-mat.gtex}

\vspace{5mm}

\pars{Absolutio.}

\cuminitiali{}{temporalia/absolutio-ipsius.gtex}

\vfill
\pagebreak

\cuminitiali{}{temporalia/benedictio-solemn-deus.gtex}

\vspace{7mm}

\lectioiv

\noindent \Vbardot{} Tu autem, Dómine, miserére nobis.
\noindent \Rbardot{} Deo grátias.

\vfill
\pagebreak

\responsoriumiv

\vfill
\pagebreak

\cuminitiali{}{temporalia/benedictio-solemn-christus.gtex}

\vspace{7mm}

\lectiov

\noindent \Vbardot{} Tu autem, Dómine, miserére nobis.
\noindent \Rbardot{} Deo grátias.

\vfill
\pagebreak

\responsoriumv

\vfill
\pagebreak

\cuminitiali{}{temporalia/benedictio-solemn-ignem.gtex}

\vspace{7mm}

\lectiovi

\noindent \Vbardot{} Tu autem, Dómine, miserére nobis.
\noindent \Rbardot{} Deo grátias.

\vfill
\pagebreak

\responsoriumvi

\vfill
\pagebreak

\subhora{In III. Nocturno}

\pars{Psalmus 7.} \scriptura{Ps. 9, 22}

\vspace{-4mm}

\antiphona{II D}{temporalia/ant-utquiddomine.gtex}

\vspace{-4mm}

\scriptura{Ps. 9, 22-32}

%\vspace{-2mm}

\initiumpsalmi{temporalia/ps9xxii_xxxii-initium-ii-D-auto.gtex}

%\psalmusEtTranslatioT{temporalia/ps9xxii_xxxii-I-comb.tex}{10cm}
\input{temporalia/ps9xxii_xxxii-I.tex} \Abardot{}

\vfill
\pagebreak

\pars{Psalmus 8.}\scriptura{Ex. 15, 18}

\vspace{-4mm}

\antiphona{IV* e}{temporalia/ant-inaeternum.gtex}

%\vspace{-4mm}

\scriptura{Ps. 9, 33-39}

\initiumpsalmi{temporalia/ps9xxxiii_xxxix-initium-iv_-e-auto.gtex}

%\psalmusEtTranslatioT{temporalia/ps9xxxiii_xxxix-I-comb.tex}{10cm}
\input{temporalia/ps9xxxiii_xxxix-I.tex} \Abardot{}

\vfill
\pagebreak

\pars{Psalmus 9.} \scriptura{Ps. 10, 8}

\vspace{-4mm}

\antiphona{II* f}{temporalia/ant-justusdominus.gtex}

%\vspace{-4mm}

\scriptura{Ps. 10}

%\initiumpsalmi{temporalia/ps10-initium-iv-c-auto.gtex}
\initiumpsalmi{temporalia/ps10-initium-ii_-f.gtex}

%\psalmusEtTranslatioT{temporalia/ps10-I-comb.tex}{10cm}
\input{temporalia/ps10-I.tex} \Abardot{}

\vfill
\pagebreak

\pars{Versus.} \scriptura{Ps. 118, 148}

% Versus. %%%
\sineinitiali{temporalia/versus-praevenerunt.gtex}

\vspace{5mm}

\sineinitiali{temporalia/oratiodominica-mat.gtex}

\vspace{5mm}

\pars{Absolutio.}

\cuminitiali{}{temporalia/absolutio-avinculis.gtex}

\vfill
\pagebreak

\cuminitiali{}{temporalia/benedictio-solemn-evangelica.gtex}

\vspace{7mm}

\lectiovii

\noindent \Vbardot{} Tu autem, Dómine, miserére nobis.
\noindent \Rbardot{} Deo grátias.

\vfill
\pagebreak

\responsoriumvii

\vfill
\pagebreak

\cuminitiali{}{temporalia/benedictio-solemn-divinum.gtex}

\vspace{7mm}

\lectioviii

\noindent \Vbardot{} Tu autem, Dómine, miserére nobis.
\noindent \Rbardot{} Deo grátias.

\vfill
\pagebreak

\responsoriumviii

\vfill
\pagebreak

\cuminitiali{}{temporalia/benedictio-solemn-adsocietatem.gtex}

\vspace{7mm}

\lectioix

\noindent \Vbardot{} Tu autem, Dómine, miserére nobis.
\noindent \Rbardot{} Deo grátias.

\vfill
\pagebreak

% Te Deum

{
\pars{Hymnus Ambrosianus} \scriptura{Tonus Solemnis}

\vspace{-2mm}

\grechangedim{interwordspacetext}{0.26 cm plus 0.15 cm minus 0.05 cm}{scalable}%
\cuminitiali{III}{temporalia/tedeum-solemnis-gn.gtex}
\grechangedim{interwordspacetext}{0.22 cm plus 0.15 cm minus 0.05 cm}{scalable}%
}

\vfill
\pagebreak

\rubrica{Reliqua omittuntur, nisi Laudes separandæ sint.}

\pars{Oratio}

\noindent \Vbardot{} Dómine, exáudi oratiónem meam.

\noindent \Rbardot{} Et clamor meus ad te véniat.

Orémus:

\oratioLaudes

\vspace{7mm}

\pars{Conclusio}

\noindent \Vbardot{} Dómine, exáudi oratiónem meam.

\noindent \Rbardot{} Et clamor meus ad te véniat.

\noindent \Vbardot{} Benedicámus Dómino, allelúia, allelúia.

\noindent \Rbardot{} Deo grátias, allelúia, allelúia.

\noindent \Vbardot{} Fidélium ánimæ per misericórdiam Dei requiéscant in pace.

\noindent \Rbardot{} Amen.

\vfill
\pagebreak

\hora{Ad Laudes.} %%%%%%%%%%%%%%%%%%%%%%%%%%%%%%%%%%%%%%%%%%%%%%%%%%%%%
%\sideThumbs{Laudes}

\cantusSineNeumas

\vspace{0.5cm}
\grechangedim{interwordspacetext}{0.18 cm plus 0.15 cm minus 0.05 cm}{scalable}%
\cuminitiali{}{temporalia/deusinadiutorium-alter.gtex}
\grechangedim{interwordspacetext}{0.22 cm plus 0.15 cm minus 0.05 cm}{scalable}%

\vfill
%\pagebreak

\pars{Psalmus 1.}

\vspace{-4mm}

\antiphona{VI F}{temporalia/ant-alleluia1.gtex}

\scriptura{Psalmus 50.}

\initiumpsalmi{temporalia/ps50-initium-vi-F-auto.gtex}

%\psalmusEtTranslatioT{temporalia/ps50-I-comb.tex}{10cm}
\input{temporalia/ps50-I.tex}

\vfill
\pagebreak

\pars{Psalmus 2.}

\scriptura{Psalmus 117.}

\initiumpsalmi{temporalia/ps117-initium-vi-F-auto.gtex}

%\psalmusEtTranslatioT{temporalia/ps117-I-comb.tex}{10cm}
\input{temporalia/ps117-I.tex}

\vfill
\pagebreak

\pars{Psalmus 3.}

\scriptura{Psalmus 62.}

\initiumpsalmi{temporalia/ps62-initium-vi-F-auto.gtex}

%\psalmusEtTranslatioT{temporalia/ps62-I-comb.tex}{10cm}
\input{temporalia/ps62-I.tex}

\vfill

\vspace{-6mm}

\antiphona{}{temporalia/ant-alleluia1.gtex} % repeat the antiphon - new page

\vfill
\pagebreak

\pars{Psalmus 4.} \scriptura{Dan. 3, 22-26; \textbf{H422}}

\vspace{-4mm}

\antiphona{VIII G}{temporalia/ant-trespueri.gtex}

\scriptura{Canticum trium puerorum, Dan. 3, 57-88 et 56}

\initiumpsalmi{temporalia/dan3-initium-viii-G-auto.gtex}

%\psalmusEtTranslatioT{temporalia/dan3-comb.tex}{10cm}
\input{temporalia/dan3.tex}

\rubrica{Hic non dicitur Gloria Patri, neque Amen.}

\vfill

\vspace{-6mm}

\antiphona{}{temporalia/ant-trespueri.gtex} % repeat the antiphon - new page

\vfill
\pagebreak

\pars{Psalmus 5.}

\vspace{-4mm}

\antiphona{VIII G}{temporalia/ant-alleluia2.gtex}

\scriptura{Psalmus 148.}

\initiumpsalmi{temporalia/ps148-initium-viii-G-auto.gtex}

%\psalmusEtTranslatioT{temporalia/ps148-I-comb.tex}{10cm}
\input{temporalia/ps148-I.tex}

\rubrica{Hic non dicitur Gloria Patri.}

\vfill
\pagebreak

%
\scriptura{Psalmus 149.}

\initiumpsalmi{temporalia/ps149-initium-viii-G-auto.gtex}

%\psalmusEtTranslatioT{temporalia/ps149-I-comb.tex}{10cm}
\input{temporalia/ps149-I.tex}

\rubrica{Hic non dicitur Gloria Patri.}

\vfill
\pagebreak

%
\scriptura{Psalmus 150.}

\initiumpsalmi{temporalia/ps150-initium-viii-G-auto.gtex}

%\psalmusEtTranslatioT{temporalia/ps150-I-comb.tex}{10cm}
\input{temporalia/ps150-I.tex}

\vfill

\vspace{-6mm}

\antiphona{}{temporalia/ant-alleluia2.gtex} % repeat the antiphon - new page

\vfill
\pagebreak

\pars{Capitulum.} \scriptura{Ac. 7, 12}

\grechangedim{interwordspacetext}{0.12 cm plus 0.15 cm minus 0.05 cm}{scalable}%
\cuminitiali{}{temporalia/capitulum-Benedictio.gtex}
\grechangedim{interwordspacetext}{0.22 cm plus 0.15 cm minus 0.05 cm}{scalable}

% preklad Jeruz. bible
%\trCapituliI

\vfill

\pars{Responsorium breve.} \scriptura{Ps. 118, 36-37}

\cuminitiali{IV}{temporalia/resp-inclinacormeum.gtex}

%\trResp

\vfill
\pagebreak

\pars{Hymnus} \scriptura{Gregorius Magnus (\olddag{} 604)}

\cuminitiali{IV}{temporalia/hym-EcceJamNoctis.gtex}
\vspace{-3mm}
%\input{hym-EcceJamNocis-bohtext.tex}

\vfill
%\pagebreak

\pars{Versus.} \scriptura{Ps. 92, 1}

% Versus. %%%
\sineinitiali{temporalia/versus-dominusregnavit.gtex}

%\noindent \trVersus

\vfill
\pagebreak

\benedictus

\vspace{-1cm}

\vfill
\pagebreak

%\sideThumbs{{\scriptsize{}Fine horarum}}

\anteOrationem

\pagebreak

% Oratio. %%%
\oratioLaudes

\vspace{-1mm}
%\trOrationisI

\vfill

\rubrica{Hebdomadarius dicit iterum Dominus vobiscum, vel cantor dicit:}

\vspace{2mm}

\sineinitiali{temporalia/domineexaudi.gtex}

\rubrica{Postea cantatur a cantore:}

\vspace{2mm}

\cuminitiali{I}{temporalia/benedicamus-dominica-perannum.gtex}

\vspace{1mm}

\vfill
\pagebreak

\hora{Ad II. Vesperas.} %%%%%%%%%%%%%%%%%%%%%%%%%%%%%%%%%%%%%%%%%%%%%%%%%%%%%
%\sideThumbs{II. Vesperæ}

\cantusSineNeumas

%\vspace{0.5cm}
\grechangedim{interwordspacetext}{0.18 cm plus 0.15 cm minus 0.05 cm}{scalable}%
\cuminitiali{}{temporalia/deusinadiutorium-solemnis.gtex}
\grechangedim{interwordspacetext}{0.22 cm plus 0.15 cm minus 0.05 cm}{scalable}%

\vfill
%\pagebreak

\vspace{-2mm}

\pars{Psalmus 1.} \scriptura{Ps. 109, 1; \textbf{H91}}

\vspace{-4mm}

\antiphona{VII c\textsuperscript{2}}{temporalia/ant-dixitdominus.gtex}

\vspace{-4mm}

\scriptura{Psalmus 109.}

\initiumpsalmi{temporalia/ps109-initium-vii-c2-auto.gtex}

%\psalmusEtTranslatioT{temporalia/ps109-I-comb.tex}{10cm}
\input{temporalia/ps109-I.tex} \Abardot{}

\vspace{-1cm}

\vfill
\pagebreak

\pars{Psalmus 2.} \scriptura{Ps. 110, 8; \textbf{H91}}

\vspace{-4mm}

\antiphona{IV g}{temporalia/ant-fideliaomnia.gtex}

\scriptura{Psalmus 110.}

\initiumpsalmi{temporalia/ps110-initium-iv-g-auto.gtex}

%\psalmusEtTranslatioT{temporalia/ps110-I-comb.tex}{10cm}
\input{temporalia/ps110-I.tex} \Abardot{}

\vfill
\pagebreak

\pars{Psalmus 3.} \scriptura{Ps. 111, 1; \textbf{H92}}

\vspace{-4mm}

\antiphona{IV a}{temporalia/ant-inmandatis.gtex}

\scriptura{Psalmus 111.}

\initiumpsalmi{temporalia/ps111-initium-iv-a-auto.gtex}

%\psalmusEtTranslatioT{temporalia/ps111-I-comb.tex}{10cm}
\input{temporalia/ps111-I.tex} \Abardot{}

\vfill
\pagebreak

\pars{Psalmus 4.} \scriptura{Ps. 112, 2; \textbf{H92}}

\vspace{-4mm}

\antiphona{VII c}{temporalia/ant-sitnomendomini.gtex}

\scriptura{Psalmus 112.}

\initiumpsalmi{temporalia/ps112-initium-vii-c-auto.gtex}

%\psalmusEtTranslatioT{temporalia/ps112-I-comb.tex}{10cm}
\input{temporalia/ps112-I.tex} \Abardot{}

\vfill
\pagebreak

\pars{Capitulum.} \scriptura{2 Cor. 1, 3-4}

\grechangedim{interwordspacetext}{0.12 cm plus 0.15 cm minus 0.05 cm}{scalable}%
\cuminitiali{}{temporalia/capitulum-BenedictusDeus.gtex}
\grechangedim{interwordspacetext}{0.22 cm plus 0.15 cm minus 0.05 cm}{scalable}

% preklad Jeruz. bible
%\trCapituliI

\vfill

\pars{Responsorium breve.} \scriptura{Ps. 103, 24}

\cuminitiali{VI}{temporalia/resp-quammagnificata.gtex}

%\trResp

\vfill
\pagebreak

\pars{Hymnus} \scriptura{Gregorius Magnus (\olddag{} 604)}

\cuminitiali{I}{temporalia/hym-LucisCreator-aestivalis.gtex}
\vspace{-3mm}
%\begin{translatioMulticol}{3}
Tvůrce světa předobrý,\\
tys ustanovil denní řád\\
a proudy světla rozhodil,\\
když světu základy jsi klad.\\
\\
A spojils ráno s večerem\\
a dnem tu dobu nazýváš;\\
hle padá temné noci stín -\\
slyš prosbu, vyslyš nářek náš.\columnbreak

Ach, nedej, by nás stihla smrt,\\
když svědomí nám tíží hřích,\\
když nemyslíme na věčnost\\
v té síti hříchů šalebných.\\
\\
Vzbuď naši touhu po nebi,\\
kde věčný život čeká nás,\\
a pomoz odložit vše zlé\\
a smýti z duše každý kaz.\columnbreak

To splň nám, dobrý Otče náš,\\
i ty, jenž rovné božství máš,\\
i Duchu, který těšíš nás\\
a vládneš, Bože, v každý čas.\\
Amen. 
\end{translatioMulticol}


\vfill
%\pagebreak

\pars{Versus.} \scriptura{Ps. 140, 2}

% Versus. %%%
\sineinitiali{temporalia/versus-dirigatur.gtex}

%\noindent \trVersus

\vfill
\pagebreak

\magnificatii

\vfill
\pagebreak

%\sideThumbs{{\scriptsize{}Fine horarum}}

\anteOrationem

\pagebreak

% Oratio. %%%
\oratioLaudes

\vspace{-1mm}
%\trOrationisI

\vfill

\rubrica{Hebdomadarius dicit iterum Dominus vobiscum, vel cantor dicit:}

\vspace{2mm}

\sineinitiali{temporalia/domineexaudi.gtex}

\rubrica{Postea cantatur a cantore:}

\vspace{2mm}

\cuminitiali{I}{temporalia/benedicamus-dominica-perannum.gtex}

\vspace{1mm}

\end{document}

