% LuaLaTeX

\documentclass[a4paper, twoside, 12pt]{article}
\usepackage[latin]{babel}
%\usepackage[landscape, left=3cm, right=1.5cm, top=2cm, bottom=1cm]{geometry} % okraje stranky
%\usepackage[landscape, a4paper, mag=1166, truedimen, left=2cm, right=1.5cm, top=1.6cm, bottom=0.95cm]{geometry} % okraje stranky
\usepackage[landscape, a4paper, mag=1400, truedimen, left=0.5cm, right=0.5cm, top=0.5cm, bottom=0.5cm]{geometry} % okraje stranky

\usepackage{fontspec}
\setmainfont[FeatureFile={junicode.fea}, Ligatures={Common, TeX}, RawFeature=+fixi]{Junicode}
%\setmainfont{Junicode}

% shortcut for Junicode without ligatures (for the Czech texts)
\newfontfamily\nlfont[FeatureFile={junicode.fea}, Ligatures={Common, TeX}, RawFeature=+fixi]{Junicode}

% Hebrew font:
% http://scripts.sil.org/cms/scripts/page.php?site_id=nrsi&id=SILHebrUnic2
\newfontfamily\hebfont[Scale=1]{Ezra SIL}

\usepackage{multicol}
\usepackage{color}
\usepackage{lettrine}
\usepackage{fancyhdr}

% usual packages loading:
\usepackage{luatextra}
\usepackage{graphicx} % support the \includegraphics command and options
\usepackage{gregoriotex} % for gregorio score inclusion
\usepackage{gregoriosyms}
\usepackage{wrapfig} % figures wrapped by the text
\usepackage{parcolumns}
\usepackage[contents={},opacity=1,scale=1,color=black]{background}
\usepackage{tikzpagenodes}
\usepackage{calc}
\usepackage{longtable}
\usetikzlibrary{calc}

\setlength{\headheight}{14.5pt}

\input{conventuscommune.tex} % Often used macros
%%%% Preklady jednotlivych zpevu (nektere se opakuji, a je dobre mit je
% vsechny na jedne hromade)

% HOURS ---

\newcommand{\trAntI}{\translatioCantus{Muž boží měl kožený toulec, pečlivě
zavázaný, jenž mu visel na šíji a~často se ho dotýkal.}}

\newcommand{\trAntII}{\translatioCantus{Klíč od~něho tak dobře střežil, že
dokud žil v~těle, nikdo z~jeho žáků nezvěděl, co je uvnitř.}}

\newcommand{\trAntIII}{\translatioCantus{Ale když se odebral z~tohoto
života, schránku otevřeli a~objevili v~ní žíněné roucho a~měděný řetěz
potřísněný krví.}}

\newcommand{\trAntIV}{\translatioCantus{A když prohlédli mistrovo tělo,
nalezli jeho tělo na čtyřech místech hluboce zbrázděno ranami od řetězu.}}

\newcommand{\trAntV}{\translatioCantus{Krev vytékající z~těch ran, místy
prostoupila i~žíněným rouchem.}}

\newcommand{\trCapituli}{\translatioCantus{
Miláčkovi Boha a~lidí,
Mojžíšovi požehnané paměti,~\gredagger{}
dopřál slávu rovnou slávě svatých~\grestar{}
učinil ho mocným na postrach nepřátelům
a~jeho slovy zastavil divy.}}

\newcommand{\trLectioBrevis}{\translatioCantus{
Pamatujte na své představené,
kteří vám hlásali Boží slovo.
Uvažte, jak oni skončili život, a~napodobujte jejich víru.
Ježíš Kristus je stejný včera i~dnes i~navěky.
Nenechte se svést věelijakými cizími naukami.}}

\newcommand{\trRespLaud}{\translatioCantus{Spravedlivého vodil Hospodin~\grestar{}
po přímých stezkách. \Vbardot{} A~ukázal mu Boží království.}}

\newcommand{\trRespLaudB}{\translatioCantus{Na tvých hradbách, Jeruzaléme,
ustanovil jsem strážné;~\grestar{}
budou bdít nad mým lidem. \Vbardot{} Ani ve dne, ani v~noci nesmějí nikdy
mlčet.}}

\newcommand{\trVersus}{\translatioCantus{\Vbardot{} Ústa spravedlivého šeptají moudrost, aleluja.
\Rbardot{} A~jeho jazyk ohlašuje právo, aleluja.}}

\newcommand{\trAntBenedictus}{\translatioCantus{Když na bujné oře vložili
nosítka a~sňali jim uzdu, vydali se přímo k~cele božího muže.}}

\newcommand{\trPreces}{\translatioCantus{
\noindent S vděčností chvalme Krista, dobrého Pastýře, \gredagger{} který dal život za své ovce, \grestar{} a~pokorně ho prosme: \Rbardot{} Pane, buď pastýřem svého lidu.

\noindent Kriste, ty dáváš církvi pastýře, a~jejich službou se ujímáš svého lidu, \grestar{} dej, ať v~lásce těch, kteří nás vedou, poznáváme, jak nás miluješ. \Rbardot{} Pane, buď pastýřem svého lidu.

\noindent Ty stále konáš skrze své zástupce službu pastýře a~učitele, \grestar{} nepřestávej nás nikdy vést prostřednictvím svých služebníků. \Rbardot{} Pane, buď pastýřem svého lidu.

\noindent Ty prokazuješ svému lidu skrze jeho pastýře službu lékaře duše i~těla, \grestar{} ochraňuj náš život a~veď nás ke svatosti. \Rbardot{} Pane, buď pastýřem svého lidu.

\noindent Ty posíláš své svaté, aby slovem i~příkladem vedli tvůj lid k~tobě, \grestar{} na jejich přímluvu nás posiluj, abychom vytrvali na cestě, která vede k~věčnému životu. \Rbardot{} Pane, buď pastýřem svého lidu.}}

\newcommand{\trOrationis}{\translatioCantus{Bože, jenž nám dopřáváš radovat
se z~výroční slavnosti svatého tvého vyznavače Havla, uděl dobrotivě,
abychom když slavíme jeho narození, též se řídili podobou jeho skutků.
Skrze…}}
 % Czech translations of the proper texts

\newcommand{\annusEditionis}{2020}

\def\hebinitial#1{%
\leavevmode{\newbox\hebbox\setbox\hebbox\hbox{\hebfont{#1}\hskip 1mm}\kern -\wd\hebbox\hbox{\hebfont{#1}\hskip 1mm}}%
}

%%%% Vicekrat opakovane kousky

\newcommand{\anteOrationem}{
  \rubrica{Ante Orationem, cantatur a Superiore:}

  \pars{Supplicatio Litaniæ.}

  \cuminitiali{}{temporalia/supplicatiolitaniae.gtex}

  \pars{Oratio Dominica.}

  \cuminitiali{}{temporalia/oratiodominica.gtex}
}

\setlength{\columnsep}{30pt} % prostor mezi sloupci

\newcommand{\lectioi}{\pars{Lectio I.} \scriptura{Is. 51, 17-23; 52, 1-2.7-10}

\noindent De libro Isaíæ prophétæ.

\noindent Eleváre, eleváre, consúrge, Ierúsalem, quæ bibísti de manu Dómini cálicem iræ eius; póculum sopóris bibísti, epotásti.

\noindent Non est qui susténtet eam ex ómnibus fíliis, quos génuit;et non est qui apprehéndat manum eius ex ómnibus fíliis, quos enutrívit.

\noindent Duo sunt quæ occurrérunt tibi; quis contristábitur super te? Vástitas et contrítio et fames et gládius; quis consolábitur te?

\noindent Fílii tui defecérunt, iácent in cápite ómnium viárum sicut oryx illaqueátus, pleni indignatióne Dómini, increpatióne Dei tui.

\noindent Idcírco audi hoc, paupércula et ébria, sed non a vino.

\noindent Hæc dicit dominátor tuus, Dóminus et Deus tuus, qui conténdit pro pópulo suo:

\noindent «Ecce tuli de manu tua cálicem sopóris, póculum indignatiónis meæ; non adícies, ut bibas illum ultra.

\noindent Et ponam illum in manu eórum, qui te humiliavérunt et dixérunt tibi: “Incurváre, ut transeámus”;

\noindent et ponébas ut terram dorsum tuum et quasi viam transeúntibus».

\noindent Consúrge, consúrge, indúere fortitúdine tua, Sion;

\noindent indúere vestiméntis glóriæ tuæ, Ierúsalem, cívitas sanctitátis, quia non adíciet ultra ut pertránseat per te incircumcísus et immúndus.

\noindent Excútere de púlvere, consúrge, captíva Ierúsalem; solve víncula colli tui, captíva fília Sion.

\noindent Quam pulchri super montes pedes annuntiántis, prædicántis pacem, annuntiántis bonum, prædicántis salútem, dicéntis Sion: «Regnávit Deus tuus!».

\noindent Vox speculatórum tuórum: levavérunt vocem, simul exsultábunt, quia óculo ad óculum vidébunt, cum redíerit Dóminus ad Sion.

\noindent Gaudéte et exsultáte simul, desérta Ierúsalem, quia consolátus est Dóminus pópulum suum, redémit Ierúsalem.

\noindent Nudávit Dóminus bráchium sanctum suum in óculis ómnium géntium; et vidébunt omnes fines terræ salutáre Dei nostri.}
\newcommand{\responsoriumi}{\pars{Responsorium 1.} \scriptura{\Rbar{} Cf. Ios. 3, 5 \Vbar{} Cf. Ex. 16, 7; \textbf{H42}}

\vspace{-5mm}

\responsorium{VII}{temporalia/resp-sanctificaminihodie-CROCHU.gtex}{}

\rubrica{vel ad libitum:}

\vspace{3mm}

\pars{Responsorium 1.} \scriptura{\textbf{H42}}

\vspace{-5mm}

\responsorium{VIII}{temporalia/resp-sanctificaminifilii-CROCHU.gtex}{}}
\newcommand{\lectioii}{\pars{Lectio II.} \scriptura{Sermo 185: PL 38, 997-999}

\noindent Ex Sermónibus sancti Augustíni epíscopi.

\noindent Expergíscere, homo: pro te Deus factus est homo. \textit{Surge, qui dormis, et exsúrge a mórtuis, et illuminábit te Christus.}

\noindent Pro te, inquam, Deus factus est homo.

\noindent In ætérnum mórtuus esses, nisi in témpore natus esset.

\noindent Numquam liberaréris a carne peccáti, nisi suscepísset similitúdinem carnis peccáti.

\noindent Perpétua te possidéret miséria, nisi fíeret hæc misericórdia.

\noindent Non revixísses, nisi tuæ morti convenísset. Defecísses, nisi subvenísset. Perísses, nisi venísset.

\noindent Celebrémus læti nostræ salútis et redemptiónis advéntum.

\noindent Celebrémus festum diem, quo magnus et ætérnus dies ex magno et ætérno die venit in hunc nostrum tam brevem temporálem diem.

\noindent Hic \textit{est nobis factus iustítia et sanctificátio et redémptio: ut, quemádmodum scriptum est: Qui gloriátur, in Dómino gloriétur.}

\noindent \textit{Véritas ergo de terra orta est:} Christus qui dixit: \textit{Ego sum véritas}, de Vírgine natus est.

\noindent \textit{Et iustítia de cælo prospéxit:} quóniam credens in eum qui natus est, non homo a se ipso, sed a Deo iustificátus est.

\noindent \textit{Véritas de terra orta est:} quia \textit{Verbum caro factum est.} Et iustítia de cælo prospéxit: quia \textit{omne datum óptimum et omne donum perféctum desúrsum est.}}
\newcommand{\responsoriumii}{\pars{Responsorium 2.} \scriptura{\Rbardot{} 2 Chr. 20, 17 \Vbardot{} Is. 26, 19; \textbf{H42}}

\vspace{-5mm}

\responsorium{VIII}{temporalia/resp-constantesestote-CROCHU.gtex}{}}
\newcommand{\lectioiii}{\pars{Lectio III.}

\noindent \textit{Véritas de terra orta est,} caro de María.

\noindent \textit{Et iustítia de cælo prospéxit:} quia \textit{non potest homo accípere quidquam, nisi fúerit ei datum de cælo.}

\noindent \textit{Iustificáti ex fide, pacem habeámus ad Deum:} quia \textit{iustítia et pax osculátæ sunt ínvicem. Per Dóminum nostrum Iesum Christum:} quia \textit{Véritas de terra orta est. Per quem et accéssum habémus in grátiam istam, in qua stamus, et gloriámur in spe glóriæ Dei.}

\noindent Non ait: «glóriæ nostræ»; sed, \textit{glóriæ Dei:} quia \textit{iustítia} non de nobis procéssit, sed \textit{de cælo prospéxit.} Ergo \textit{qui gloriátur,} non in se, sed \textit{in Dómino gloriétur.}

\noindent Hinc enim et nato ex Vírgine Dómino præcónium vocis angélicæ factum est: \textit{Glória in excélsis Deo et in terra pax homínibus bonæ voluntátis.}

\noindent In terra enim pax unde, nisi quia \textit{Véritas de terra orta est,} id est, Christus de carne natus est?

\noindent Et \textit{ipse est pax nostra qui fecit útraque unum:} ut essémus hómines bonæ voluntátis, suáviter conéxi vínculo unitátis.

\noindent In hac ígitur grátia gaudeámus, ut sit glória nostra testimónium consciéntiæ nostræ: ubi non in nobis, sed in Dómino gloriémur.

\noindent Hinc enim dictum est, \textit{Glória mea et exáltans caput meum.}

\noindent Nam quæ maior grátia Dei nobis pótuit illucéscere, quam ut, habens unigénitum Fílium, fáceret eum hóminis Fílium, atque ita vicíssim hóminis fílium fáceret Dei fílium?

\noindent Quære méritum, quære causam, quære iustítiam; et vide utrum invénias nisi grátiam.}
\newcommand{\responsoriumiii}{\pars{Responsorium 3.} \scriptura{\Vbar{} Ps. 49, 2; \textbf{H42}}

\vspace{-5mm}

\responsorium{VIII}{temporalia/resp-deillaocculta-CROCHU-cumdox.gtex}{}

\rubrica{vel ad libitum:}

\vspace{3mm}

\pars{Responsorium 3.} \scriptura{\Rbar{} Is. 1, 11 \Vbar{} Is. 45, 8; \textbf{H36}}

\vspace{-5mm}

\responsorium{I}{temporalia/resp-egredieturvirga-CROCHU-cumdox.gtex}{}}
\newcommand{\lectiobrevis}{\pars{Lectio Brevis.} \scriptura{Is. 11, 1-3}

\noindent Egrediétur virga de stirpe Iesse, et flos de radíce eius ascéndet; et requiéscet super eum spíritus Dómini: spíritus sapiéntiæ et intelléctus, spíritus consílii et fortitúdinis, spíritus sciéntiæ et timóris Dómini; et delíciæ eius in timóre Dómini.}
\newcommand{\benedictus}{\pars{Canticum Zachariæ.} \scriptura{Lc. 2, 6-7; \textbf{H71}}

\vspace{-4mm}

{
\grechangedim{interwordspacetext}{0.18 cm plus 0.15 cm minus 0.05 cm}{scalable}%
\antiphona{VIII G}{temporalia/ant-completisunt.gtex}
\grechangedim{interwordspacetext}{0.22 cm plus 0.15 cm minus 0.05 cm}{scalable}%
}

%\vspace{-2mm}

\scriptura{Lc. 1, 68-79}

%\vspace{-2mm}

\cantusSineNeumas
\initiumpsalmi{temporalia/benedictus-initium-viii-G-auto.gtex}

%\vspace{-1.5mm}

\input{temporalia/benedictus-viii-G.tex} \Abardot{}}
\newcommand{\preces}{\noindent Christum redemptórem, fratres caríssimi, devotióne mentis orémus,~\gredagger{} qui véniet cum glória et potestáte magna,~\grestar{} atque súpplices invocémus:

\Rbardot{} Veni, Dómine Iesu.

\noindent Christe Dómine, qui excélsus in fortitúdine vénies,~\grestar{} réspice humília nostra, ut dignos nos fácias munéribus tuis.

\Rbardot{} Veni, Dómine Iesu.

\noindent Qui Evangélium homínibus manifestáre venísti,~\grestar{} da tuam nos semper prædicáre salútem.

\Rbardot{} Veni, Dómine Iesu.

\noindent Tu, qui es benedíctus et vivis et ómnia regis,~\grestar{} da nos gaudéntes exspectáre beátam spem et advéntum magnificéntiæ tuæ.

\Rbardot{} Veni, Dómine Iesu.

\noindent Et nos, qui advéntus tui grátiæ anhelámus,~\grestar{} divinitátis tuæ múnere consoláre.

\Rbardot{} Veni, Dómine Iesu.}
\newcommand{\oratio}{\pars{Oratio.}

\noindent Festína, quǽsumus, ne tardáveris, Dómine Iesu,~\gredagger{} ut advéntus tui consolatiónibus sublevéntur,~\grestar{} qui in tua pietáte confídunt.

\pars{Pro pace in Ucraina.} \scriptura{Sir. 50, 25; 2 Esdr. 4, 20; \textbf{H416}}

\vspace{-4mm}

\antiphona{II D}{temporalia/ant-dapacemdomine.gtex}

\vfill

\noindent Deus, a quo sancta desidéria, recta consília et iusta sunt ópera: da servis tuis illam, quam mundus dare non potest, pacem; ut et corda nostra mandátis tuis dédita, et hóstium subláta formídine, témpora sint tua protectióne tranquílla.

\noindent Per Dóminum nostrum Iesum Christum, Fílium tuum, qui tecum vivit et regnat in unitáte Spíritus Sancti, Deus, per ómnia sǽcula sæculórum.

%\noindent Qui vivis et regnas cum Deo Patre in unitáte Spíritus Sancti, Deus, per ómnia sǽcula sæculórum.

\noindent \Rbardot{} Amen.}
\newcommand{\responsoriumbreve}{\pars{Responsorium breve.}

\cuminitiali{VI}{temporalia/resp-crastinadie.gtex}}

%%%%%%%%%%%%%%%%%%%%%%%%%%%%%%%%%%%%%%%%%%%%%%%%%%%%%%%%%%%%%%%%%%%%%%%%%%%%%%%%%%%%%%%%%%%%%%%%%%%%%%%%%%%%%
\begin{document}

% Here we set the space around the initial.
% Please report to http://home.gna.org/gregorio/gregoriotex/details for more details and options
\grechangedim{afterinitialshift}{2.2mm}{scalable}
\grechangedim{beforeinitialshift}{2.2mm}{scalable}

\grechangedim{interwordspacetext}{0.22 cm plus 0.15 cm minus 0.05 cm}{scalable}%
\grechangedim{annotationraise}{-0.2cm}{scalable}

% Here we set the initial font. Change 38 if you want a bigger initial.
% Emit the initials in red.
\grechangestyle{initial}{\color{red}\fontsize{38}{38}\selectfont}

\pagestyle{empty}

%%%% Titulni stranka
\begin{titulusOfficii}
\dies{Die 24. Decembris.}
\nomenFesti{In Vigilia Nativitatis Domini.}
\end{titulusOfficii}

\pars{}

\scriptura{}

\pagebreak

% graphic
\renewcommand{\headrulewidth}{0pt} % no horiz. rule at the header
\fancyhf{}
\pagestyle{fancy}

\cantusSineNeumas

\hora{Ad Matutinum.}

\vspace{2mm}

\cuminitiali{}{temporalia/dominelabiamea.gtex}

\vspace{2mm}

\pars{Invitatorium.} \scriptura{\textbf{H41}}

\vspace{-6mm}

\antiphona{E}{temporalia/inv-hodiescietis.gtex}

\vfill
\pagebreak

\pars{Hymnus.}

\vspace{-5mm}

{
\grechangedim{interwordspacetext}{0.30 cm plus 0.15 cm minus 0.05 cm}{scalable}%
\antiphona{II}{temporalia/hym-VeniRedemptor.gtex}
\grechangedim{interwordspacetext}{0.22 cm plus 0.15 cm minus 0.05 cm}{scalable}%
}

\vfill
\pagebreak

\pars{Psalmus 1.} \scriptura{2. Par. 20, 17}

\vspace{-4mm}

\antiphona{VIII G}{temporalia/ant-iudaeaetierusalem.gtex}

\vspace{-2mm}

\scriptura{Ps. 102, 1-7}

%\vspace{-2mm}

\initiumpsalmi{temporalia/ps102i-initium-viii-G-auto.gtex}

\input{temporalia/ps102i-viii-G.tex} \Abardot{}

\vfill
\pagebreak

\pars{Psalmus 2.} \scriptura{Cf. Ex. 16, 6-7; \textbf{H42}}

\vspace{-4mm}

\antiphona{VIII a}{temporalia/ant-hodiescietis.gtex}

\vspace{-3mm}

\scriptura{Ps. 102, 8-16}

\vspace{-1mm}

\initiumpsalmi{temporalia/ps102ii-initium-viii-a-auto.gtex}

\vspace{-1.5mm}

\input{temporalia/ps102ii-viii-a.tex} \Abardot{}

\vspace{-5mm}

\vfill
\pagebreak

\pars{Psalmus 3.} \scriptura{Cf. Dt. 32, 2; \textbf{H43}}

\vspace{-4mm}

\antiphona{II* b}{temporalia/ant-exspectetur.gtex}

%\vspace{-5mm}

\scriptura{Ps. 102, 17-22}

%\vspace{-2mm}

\initiumpsalmi{temporalia/ps102iii-initium-ii_-B-auto.gtex}

\input{temporalia/ps102iii-ii_-B.tex} \Abardot{}

\vfill
\pagebreak

%\pars{Versus} \scriptura{Mc. 1, 3; Is. 40, 3}

% Versus. %%%
%\sineinitiali{temporalia/versus-voxclamantis-simplex.gtex}

\pars{Versus}

\noindent \Vbardot{} Annúntiat Dóminus verbum suum Iacob.

\noindent \Rbardot{} Iustítias et iudícia sua Israel.

\vspace{5mm}

\sineinitiali{temporalia/oratiodominica-mat.gtex}

\vspace{5mm}

\pars{Absolutio.}

\cuminitiali{}{temporalia/absolutio-exaudi.gtex}

\vfill
\pagebreak

\cuminitiali{}{temporalia/benedictio-solemn-benedictione.gtex}

\vspace{7mm}

\lectioi

\noindent \Vbardot{} Tu autem, Dómine, miserére nobis.
\noindent \Rbardot{} Deo grátias.

\vfill
\pagebreak

\responsoriumi

\vfill
\pagebreak

\cuminitiali{}{temporalia/benedictio-solemn-unigenitus.gtex}

\vspace{7mm}

\lectioii

\noindent \Vbardot{} Tu autem, Dómine, miserére nobis.
\noindent \Rbardot{} Deo grátias.

\vfill
\pagebreak

\responsoriumii

\vfill
\pagebreak

\cuminitiali{}{temporalia/benedictio-solemn-spiritus.gtex}

\vspace{7mm}

\lectioiii

\noindent \Vbardot{} Tu autem, Dómine, miserére nobis.
\noindent \Rbardot{} Deo grátias.

\vfill
\pagebreak

\responsoriumiii

\vfill
\pagebreak

\rubrica{Reliqua omittuntur, nisi Laudes separandæ sint.}

\pars{Oratio}

\noindent \Vbardot{} Dómine, exáudi oratiónem meam.

\noindent \Rbardot{} Et clamor meus ad te véniat.

\oratio

\vspace{7mm}

\pars{Conclusio}

\noindent \Vbardot{} Dómine, exáudi oratiónem meam.

\noindent \Rbardot{} Et clamor meus ad te véniat.

\noindent \Vbardot{} Benedicámus Dómino, allelúia, allelúia.

\noindent \Rbardot{} Deo grátias, allelúia, allelúia.

\noindent \Vbardot{} Fidélium ánimæ per misericórdiam Dei requiéscant in pace.

\noindent \Rbardot{} Amen.

\vfill
\pagebreak

\hora{Ad Laudes.} %%%%%%%%%%%%%%%%%%%%%%%%%%%%%%%%%%%%%%%%%%%%%%%%%%%%%
%\sideThumbs{Laudes}

\cantusSineNeumas

\vspace{0.5cm}
\grechangedim{interwordspacetext}{0.18 cm plus 0.15 cm minus 0.05 cm}{scalable}%
\cuminitiali{}{temporalia/deusinadiutorium-communis.gtex}
\grechangedim{interwordspacetext}{0.22 cm plus 0.15 cm minus 0.05 cm}{scalable}%

\vfill

\pars{Hymnus}

\cuminitiali{I}{temporalia/hym-MagnisProphetae.gtex}
\vspace{-3mm}

\vfill
\pagebreak

\pars{Psalmus 1.} \scriptura{Mt. 2, 6; \textbf{H37}}

\vspace{-4.5mm}

\antiphona{III a}{temporalia/ant-tubethlehem.gtex}

\vspace{-1mm}

\scriptura{Psalmus 107.}

\vspace{-2mm}

\initiumpsalmi{temporalia/ps107-initium-iii-a-auto.gtex}

\vspace{-1.5mm}

\input{temporalia/ps107-iii-a.tex} \Abardot{}

\vspace{-3mm}

\vfill
\pagebreak

\pars{Psalmus 2.} \scriptura{Lc. 21, 28; \textbf{H43}}

\vspace{-4mm}

\antiphona{I g}{temporalia/ant-levatecapita.gtex}

\scriptura{Canticum Isaiaæ, Is. 61, 10-11; 62, 1-7}

%\vspace{-2mm}

\initiumpsalmi{temporalia/isaiae4-initium-i-g-auto.gtex}

\input{temporalia/isaiae4-i-g.tex}

\antiphona{}{temporalia/ant-levatecapita.gtex}

\vfill
\pagebreak

\pars{Psalmus 3.} \scriptura{1 Sam. 11, 9; \textbf{H43}}

\vspace{-4mm}

\antiphona{VIII c}{temporalia/ant-crastinaerit.gtex}

\scriptura{Psalmus 145.}

%\vspace{-2mm}

\initiumpsalmi{temporalia/ps145-initium-viii-c-auto.gtex}

\input{temporalia/ps145-viii-c.tex} \Abardot{}

\vfill
\pagebreak

\lectiobrevis

\vfill

\responsoriumbreve

\vfill
\pagebreak

\benedictus

\vfill
\pagebreak

%\sideThumbs{{\scriptsize{}Fine horarum}}

\ifx\preces\undefined
\rubrica{Ante Orationem, cantatur a Superiore:}

\pars{Supplicatio Litaniæ.}

\cuminitiali{}{temporalia/supplicatiolitaniae.gtex}

\pars{Oratio Dominica.}

\cuminitiali{}{temporalia/oratiodominica.gtex}
\else
\pars{Preces.}

\sineinitiali{}{temporalia/tonusprecum.gtex}

\preces

\vfill

\pars{Oratio Dominica.}

\cuminitiali{}{temporalia/oratiodominicaalt.gtex}

\vfill
\pagebreak

\rubrica{vel:}

\pars{Deprecatio Gelasii}

\vspace{-5mm}

\grechangedim{interwordspacetext}{0.16 cm plus 0.15 cm minus 0.05 cm}{scalable}%
\antiphona{D\textsuperscript{1}}{temporalia/deprecatio4-propace.gtex}
\grechangedim{interwordspacetext}{0.22 cm plus 0.15 cm minus 0.05 cm}{scalable}%

\vfill

\pars{Oratio Dominica.}

\cuminitiali{D}{temporalia/oratiodominica-d.gtex}
\fi

\vfill
\pagebreak

% Oratio. %%%
\oratio

\vspace{-1mm}

\vfill

\rubrica{Hebdomadarius dicit Dominus vobiscum, vel, absente sacerdote vel diacono, sic concluditur:}

\vspace{2mm}

\antiphona{C}{temporalia/dominusnosbenedicat.gtex}

\rubrica{Postea cantatur a cantore:}

\vspace{2mm}

\cuminitiali{IV}{temporalia/benedicamus-feria-advequad.gtex}

\vfill

\vspace{1mm}

\end{document}
