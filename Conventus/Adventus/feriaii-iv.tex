\newcommand{\oratio}{\pars{Oratio.}

\noindent Deus, qui hóminem delápsum in mortem conspíciens, Unigéniti tui advéntu redímere voluísti, præsta, quǽsumus, ut qui húmili eius incarnatiónem devotióne faténtur, ipsíus étiam Redemptóris consórtia mereántur.

\pars{Pro pace in universo mundo.} \scriptura{Sir. 50, 25; 2 Esdr. 4, 20; \textbf{H416}}

\vspace{-4mm}

\antiphona{II D}{temporalia/ant-dapacemdomine.gtex}

\vfill

\noindent Deus, a quo sancta desidéria, recta consília et iusta sunt ópera: da servis tuis illam, quam mundus dare non potest, pacem; ut et corda nostra mandátis tuis dédita, et hóstium subláta formídine, témpora sint tua protectióne tranquílla.

\noindent Per Dóminum nostrum Iesum Christum, Fílium tuum, qui tecum vivit et regnat in unitáte Spíritus Sancti, Deus, per ómnia sǽcula sæculórum.

\noindent \Rbardot{} Amen.}
\newcommand{\matversus}{\noindent \Vbardot{} Véniat super me misericórdia tua, Dómine.

\noindent \Rbardot{} Salutáre tuum, secúndum elóquium tuum.}
\newcommand{\lectioi}{\pars{Lectio I.} \scriptura{Is. 49, 14-21}

\noindent De libro Isaíæ prophétæ.

\noindent Dixit Sion: «Derelíquit me Dóminus et Dóminus oblítus est mei».

\noindent Numquid oblivísci potest múlier infántem suum, ut non misereátur fílio úteri sui?

\noindent Et si illa oblíta fúerit, ego tamen non oblivíscar tui.

\noindent Ecce in mánibus meis descrípsi te; muri tui coram me semper.

\noindent Festínant structóres tui; destruéntes te et dissipántes a te exíbunt.

\noindent Leva in circúitu óculos tuos et vide: omnes isti congregáti sunt, venérunt tibi.

\noindent «Vivo ego, dicit Dóminus, quia ómnibus his velut ornaménto vestiéris et circúmdabis tibi eos quasi sponsa».

\noindent Quia ruínæ tuæ et solitúdines tuæ et terra evérsa: nunc angústa eris præ habitatóribus; et longe erunt, qui devorábant te.

\noindent Adhuc dicent in áuribus tuis fílii orbitátis tuæ: «Angústus est mihi locus; fac spátium mihi, ut hábitem».

\noindent Et dices in corde tuo: «Quis génuit mihi istos?

\noindent Ego orbáta et non páriens, transmigráta et captíva; et istos quis enutrívit?

\noindent Ecce ego relícta eram sola; et isti ubi erant?».

\noindent Hæc dicit Dóminus Deus:

\noindent «Ecce levábo ad gentes manum meam et ad pópulos exaltábo signum meum;

\noindent et áfferent fílios tuos in ulnis, et fíliæ tuæ super úmeros portabúntur.

\noindent Et erunt reges nutrícii tui, et regínæ nutríces tuæ; vultu in terram demísso adorábunt te et púlverem pedum tuórum lingent.

\noindent Et scies quia ego Dóminus: non confundéntur, qui sperant in me».

\noindent Numquid tollétur a forti præda, aut, quod captum fúerit, a robústo salvári póterit?

\noindent Quia hæc dicit Dóminus: «Equidem et captívus a forti tollétur, et, quod ablátum fúerit a robústo, salvábitur;

\noindent cum his, qui contendébant tecum, ego conténdam et fílios tuos ego salvábo.

\noindent Et cibábo hostes tuos cárnibus suis, et quasi musto sánguine suo inebriabúntur;

\noindent et sciet omnis caro quia ego Dóminus salvátor tuus et redémptor tuus Fortis Iacob».

\noindent Hæc dicit Dóminus:

\noindent «Ubinam est liber repúdii matris vestræ, quo dimísi eam?

\noindent Aut quis est créditor meus, cui véndidi vos?}
\newcommand{\responsoriumi}{\pars{Responsorium 1.} \scriptura{\Rbar{} Malach. 3, 1 \& Is. 7, 14 \textbf{H34}}

\vspace{-5mm}

\responsorium{VI}{temporalia/resp-modoveniet-CROCHU.gtex}{}

\rubrica{vel ad libitum:}

\vspace{3mm}

\pars{Responsorium 1.} \scriptura{\Rbar{} Rom. 15, 12; Cf. Ps. 71, 17 \Vbar{} Ps. 49, 2-3; \textbf{H36}}

\vspace{-5mm}

\responsorium{VIII}{temporalia/resp-radixjesse-CROCHU.gtex}{}}
\newcommand{\lectioii}{\pars{Lectio II.} \scriptura{Lib.1, 46-55: CCL 120, 37-39}

\noindent Ex Expositióne sancti Bedæ Venerábilis presbýteri in Lucam.

\noindent \emph{Et ait María: Magníficat ánima mea Dóminum et exsultávit spíritus meus in Deo salutári meo.}

\noindent Tanto, inquit, me Dóminus tamque inaudíto múnere sublimávit,

\noindent quod non ullo linguæ offício explicári,

\noindent sed ipso vix íntimi péctoris afféctu váleat comprehéndi,

\noindent et ídeo totas ánimæ vires in agéndis gratiárum láudibus óffero,

\noindent totum in contemplánda magnitúdine eius, cui non est finis,

\noindent quicquid vivo, séntio, discérno, gratulánter impéndo,

\noindent quia et eiúsdem Iesu, id est salutáris, spíritus meus ætérna divinitáte lætátur,

\noindent cuius mea caro temporáli conceptióne fetátur.

\noindent \emph{Quia fecit mihi magna qui potens est et sanctum nomen eius.}

\noindent Ad inítium cárminis réspicit ubi dictum est: \emph{Magníficat ánima mea Dóminum.}

\noindent Sola quippe ánima illa, cui Dóminus magna fácere dignátur, dignis eum præcóniis magnificáre et ad consórtes eiúsdem voti ac propósiti potest cohortándo dícere: \emph{Magnificáte Dóminum mecum et exaltémus nomen eius in ínvicem.}

\noindent Nam qui Dóminum, quem cognóvit, quantum in se est magnificáre et nomen eius sanctificáre contémpserit, \textit{mínimus vocábitur in regno cælórum.}

\noindent Sanctum autem nomen eius vocátur, quia singuláris cúlmine poténtiæ transcéndit omnem creatúram atque ab univérsis quæ fecit longe segregátur.}
\newcommand{\responsoriumii}{\pars{Responsorium 2.} \scriptura{\Rbar{} Is. 45, 8 \Vbar{} Is. 16, 1; \textbf{H35}}

\vspace{-5mm}

\responsorium{II}{temporalia/resp-roratecaeli-CROCHU.gtex}{}

\rubrica{vel ad libitum:}

\vspace{3mm}

\pars{Responsorium 2.} \scriptura{\Rbardot{} Lc. 1, 48 \Vbardot{} ibid. 1, 50; \textbf{H297}}

\vspace{-5mm}

\responsorium{VIII}{temporalia/resp-beatammedicent-sinedox.gtex}{}}
\newcommand{\lectioiii}{\pars{Lectio III.}

\noindent \emph{Suscépit Israel púerum suum, memorátus misericórdiæ suæ.}

\noindent Pulchre púerum Dómini appéllat Israel, qui ab eo sit ad salvándum suscéptus, obœdiéntem vidélicet et húmilem, iuxta quod Osée dicit: \emph{Quia puer Israel et diléxi eum.}

\noindent Nam qui contémnit humiliári, non potest útique salvári nec dícere cum Prophéta: \emph{Ecce enim Deus ádiuvat me et Dóminus suscéptor est ánimæ meæ. Quicúmque autem humiliáverit se sicut párvulus, hic est maior in regno cælórum.}

\noindent \emph{Sicut locútus est ad patres nostros, Abraham et sémini eius in sǽcula.}

\noindent Semen Abrahæ non carnále sed spiritále signíficat, id est non eius tantum carne progénitos, sed sive in circumcisióne seu in præpútio fídei illíus vestígia secútos. Nam et ipse in præpútio pósitus crédidit, reputatúmque est ei ad iustítiam.

\noindent Advéntus ergo Salvatóris Abrahæ est et sémini eius in sǽcula promíssus, hoc est fíliis promissiónis, quibus dícitur: \emph{Si autem vos Christi, ergo Abrahæ semen estis secúndum promissiónem herédes.}

\noindent Bene autem vel Dómini vel Ioánnis exórtum matres prophetándo prævéniunt, ut sicut peccátum a muliéribus cœpit, ita étiam bona a muliéribus incípiant, et quæ per uníus deceptiónem périit, duábus certátim præconántibus mundo vita reddátur.}
\newcommand{\responsoriumiii}{\pars{Responsorium 3.} \scriptura{\Rbar{} Cantor \Vbar{} Lc. 1, 28; \textbf{H36}}

\vspace{-5mm}

\responsorium{I}{temporalia/resp-annuntiatumest-CROCHU-cumdox.gtex}{}}
\newcommand{\laudes}{\pars{Psalmus 1.} \scriptura{\textbf{H36}}

\vspace{-4mm}

\antiphona{II* b}{temporalia/ant-eccevenietdominus.gtex}

%\vspace{-2mm}

\scriptura{Psalmus 89}

%\vspace{-2mm}

\initiumpsalmi{temporalia/ps89-initium-ii_-B-auto.gtex}

%\vspace{-1.5mm}

\input{temporalia/ps89-ii_-B.tex}

\vfill

\antiphona{}{temporalia/ant-eccevenietdominus.gtex}

\vfill
\pagebreak

\pars{Psalmus 2.} \scriptura{Is. 42, 10; \textbf{H25}}

\vspace{-4mm}

\antiphona{VII b}{temporalia/ant-cantatedomino.gtex}

%\vspace{-2mm}

\scriptura{Canticum Isaiæ, Is. 42, 10-16}

%\vspace{-3mm}

\initiumpsalmi{temporalia/isaiae10-initium-vii-b-auto.gtex}

\input{temporalia/isaiae10-vii-b.tex} \Abardot{}

\vfill
\pagebreak

\pars{Psalmus 3.} \scriptura{Lc. 18, 8; \textbf{H36}}

\vspace{-6mm}

\antiphona{VIII C\textsuperscript{2}}{temporalia/ant-dumvenerit.gtex}

\scriptura{Psalmus 134, 1-12}

\initiumpsalmi{temporalia/ps134i-initium-viii-C2-auto.gtex}

\input{temporalia/ps134i-viii-C2.tex} \Abardot{}

\vfill
\pagebreak}
\newcommand{\lectiobrevis}{\pars{Lectio Brevis.} \scriptura{Is. 45, 8}

\noindent Roráte, cæli, désuper, et nubes pluant iustítiam; aperiátur terra et gérminet salvatiónem; et iustítia oriátur simul.}
\newcommand{\benedictus}{\pars{Canticum Zachariæ.} \scriptura{Gn. 49, 10}

\vspace{-4mm}

{
\grechangedim{interwordspacetext}{0.18 cm plus 0.15 cm minus 0.05 cm}{scalable}%
\antiphona{I D\textsuperscript{2}}{temporalia/ant-nonauferetursceptrum.gtex}
\grechangedim{interwordspacetext}{0.22 cm plus 0.15 cm minus 0.05 cm}{scalable}%
}

%\trAntIMagnificat

\vspace{-2mm}

\scriptura{Lc. 1, 68-79}

\vspace{-2mm}

\cantusSineNeumas
\initiumpsalmi{temporalia/benedictus-initium-i-D2-auto.gtex}

%\vspace{-1.5mm}

\input{temporalia/benedictus-i-D2.tex} \Abardot{}}
\newcommand{\preces}{\noindent Christum redemptórem exorémus, fratres dilectíssimi,~\gredagger{} qui venit ut nos grátia advéntus sui iustificáret,~\grestar{} vocémque cum iúbilo innovémus:

\Rbardot{} Veni, Dómine Iesu.

\noindent Qui olim prophetárum vaticínio in carne prædíctus es nascitúrus,~\grestar{} nascéntia virtútum in nos corróbora.

\Rbardot{} Veni, Dómine Iesu.

\noindent Præsta nobis ut, qui tuam prædicámus salútem,~\grestar{} in te salvatiónem habeámus.

\Rbardot{} Veni, Dómine Iesu.

\noindent Qui venísti contrítis corde medéri,~\grestar{} pópuli tui sana languóres.

\Rbardot{} Veni, Dómine Iesu.

\noindent Qui cum venísti reconciliáre dignátus es mundum,~\grestar{} ad iudícium véniens ab omni nos pœnárum líbera cruciátu.

\Rbardot{} Veni, Dómine Iesu.}
\newcommand{\vesperas}{\pars{Psalmus 1.} \scriptura{\textbf{H36}}

\vspace{-6mm}

\antiphona{II* b}{temporalia/ant-eccevenietdominus.gtex}

\vspace{-4mm}

\scriptura{Psalmus 113.}

\initiumpsalmi{temporalia/ps113-initium-ii_-B-auto.gtex}

\vspace{-1.5mm}

\input{temporalia/ps113-ii_-B.tex}

\vfill

\vspace{-6mm}

\antiphona{}{temporalia/ant-eccevenietdominus.gtex}

\vspace{-1cm}

\vfill
\pagebreak

\pars{Psalmus 2.} \scriptura{Lc. 18, 8; \textbf{H36}}

\vspace{-4mm}

\antiphona{VIII C}{temporalia/ant-dumvenerit.gtex}

\vspace{-4mm}

\scriptura{Psalmus 114.}

\initiumpsalmi{temporalia/ps114-initium-viii-C-auto.gtex}

\input{temporalia/ps114-viii-C.tex} \Abardot{}

\vfill
\pagebreak

\pars{Psalmus 3.} \scriptura{Is. 25, 9; \textbf{H34}}

\vspace{-4mm}

\antiphona{VIII G}{temporalia/ant-eccedeusnoster.gtex}

\vspace{-4mm}

\scriptura{Psalmus 115.}

\initiumpsalmi{temporalia/ps115-initium-viii-G-auto.gtex}

\input{temporalia/ps115-viii-G-sinedox.tex}

\rubrica{Hic non dicitur Gloria Patri.}

\vfill

\scriptura{Psalmus 116.}

\initiumpsalmi{temporalia/ps116-initium-viii-G-auto.gtex}

\input{temporalia/ps116-viii-G.tex} \Abardot{}

\vfill
\pagebreak

\pars{Psalmus 4.} \scriptura{\textbf{H37}}

\vspace{-4mm}

\antiphona{II* a}{temporalia/ant-egredieturdominus.gtex}

\vspace{-4mm}

\scriptura{Psalmus 128.}

\initiumpsalmi{temporalia/ps128-initium-ii_-a.gtex}

\input{temporalia/ps128-ii_-a.tex} \Abardot{}

\vfill
\pagebreak}
\newcommand{\magnificat}{\pars{Canticum B. Mariæ V.} \scriptura{Ier. 10, 7; Ag. 2, 8; Eph. 2, 20.14; \textbf{H40}}

\vspace{-6.5mm}

{
\grechangedim{interwordspacetext}{0.18 cm plus 0.15 cm minus 0.05 cm}{scalable}%
\antiphona{II D}{temporalia/ant-orex.gtex}
\grechangedim{interwordspacetext}{0.22 cm plus 0.15 cm minus 0.05 cm}{scalable}%
}

\vspace{-3mm}

\scriptura{Lc. 1, 46-55}

\vspace{-2mm}

\cantusSineNeumas

\initiumpsalmi{temporalia/magnificat-initium-iisoll-D.gtex}

\vspace{-1.5mm}

\input{temporalia/magnificat-iisoll-D.tex} \Abardot{}

\vspace{-1cm}}
\newcommand{\hebdomada}{infra Hebdom. IV post Pentecosten.}
\newcommand{\oratioLaudes}{\cuminitiali{}{temporalia/oratio4.gtex}}

\renewcommand{\hebdomada}{infra Hebdom. Ultima Adventus.}
\ifx\invitatorium\undefined
\newcommand{\invitatorium}{\pars{Invitatorium.} \scriptura{Phil. 4, 4.5}

\vspace{-6mm}

\antiphona{VI}{temporalia/inv-propeestiamsimplex.gtex}}
\fi
\ifx\hymnusmatutinum\undefined
\newcommand{\hymnusmatutinum}{\pars{Hymnus.}

\vspace{-5mm}

\antiphona{II}{temporalia/hym-VeniRedemptor.gtex}}
\fi
\ifx\hymnuslaudes\undefined
\newcommand{\hymnuslaudes}{\pars{Hymnus}

\cuminitiali{D}{temporalia/hym-MagnisProphetae.gtex}}
\fi
\ifx\hymnusvesperas\undefined
\newcommand{\hymnusvesperas}{\pars{Hymnus}

\cuminitiali{IV}{temporalia/hym-VerbumSalutis.gtex}}
\fi

% LuaLaTeX

\documentclass[a4paper, twoside, 12pt]{article}
\usepackage[latin]{babel}
%\usepackage[landscape, left=3cm, right=1.5cm, top=2cm, bottom=1cm]{geometry} % okraje stranky
%\usepackage[landscape, a4paper, mag=1166, truedimen, left=2cm, right=1.5cm, top=1.6cm, bottom=0.95cm]{geometry} % okraje stranky
\usepackage[landscape, a4paper, mag=1400, truedimen, left=0.5cm, right=0.5cm, top=0.5cm, bottom=0.5cm]{geometry} % okraje stranky

\usepackage{fontspec}
\setmainfont[FeatureFile={junicode.fea}, Ligatures={Common, TeX}, RawFeature=+fixi]{Junicode}
%\setmainfont{Junicode}

% shortcut for Junicode without ligatures (for the Czech texts)
\newfontfamily\nlfont[FeatureFile={junicode.fea}, Ligatures={Common, TeX}, RawFeature=+fixi]{Junicode}

\usepackage{multicol}
\usepackage{color}
\usepackage{lettrine}
\usepackage{fancyhdr}

% usual packages loading:
\usepackage{luatextra}
\usepackage{graphicx} % support the \includegraphics command and options
\usepackage{gregoriotex} % for gregorio score inclusion
\usepackage{gregoriosyms}
\usepackage{wrapfig} % figures wrapped by the text
\usepackage{parcolumns}
\usepackage[contents={},opacity=1,scale=1,color=black]{background}
\usepackage{tikzpagenodes}
\usepackage{calc}
\usepackage{longtable}
\usetikzlibrary{calc}

\setlength{\headheight}{14.5pt}

\input{conventuscommune.tex} % Often used macros

\newcommand{\annusEditionis}{2021}

%%%% Vicekrat opakovane kousky

\newcommand{\anteOrationem}{
  \rubrica{Ante Orationem, cantatur a Superiore:}

  \pars{Supplicatio Litaniæ.}

  \cuminitiali{}{temporalia/supplicatiolitaniae.gtex}

  \pars{Oratio Dominica.}

  \cuminitiali{}{temporalia/oratiodominica.gtex}

  \rubrica{Deinde dicitur ab Hebdomadario:}

  \cuminitiali{}{temporalia/dominusvobiscum-solemnis.gtex}

  \rubrica{In choro monialium loco Dominus vobiscum dicitur:}

  \sineinitiali{temporalia/domineexaudi.gtex}
}

\setlength{\columnsep}{30pt} % prostor mezi sloupci

%%%%%%%%%%%%%%%%%%%%%%%%%%%%%%%%%%%%%%%%%%%%%%%%%%%%%%%%%%%%%%%%%%%%%%%%%%%%%%%%%%%%%%%%%%%%%%%%%%%%%%%%%%%%%
\begin{document}

% Here we set the space around the initial.
% Please report to http://home.gna.org/gregorio/gregoriotex/details for more details and options
\grechangedim{afterinitialshift}{2.2mm}{scalable}
\grechangedim{beforeinitialshift}{2.2mm}{scalable}
\grechangedim{interwordspacetext}{0.22 cm plus 0.15 cm minus 0.05 cm}{scalable}%
\grechangedim{annotationraise}{-0.2cm}{scalable}

% Here we set the initial font. Change 38 if you want a bigger initial.
% Emit the initials in red.
\grechangestyle{initial}{\color{red}\fontsize{38}{38}\selectfont}

\pagestyle{empty}

%%%% Titulni stranka
\begin{titulusOfficii}
\ifx\titulus\undefined
\nomenFesti{Feria II \hebdomada{}}
\else
\titulus
\fi
\end{titulusOfficii}

\vfill

\begin{center}
%Ad usum et secundum consuetudines chori \guillemotright{}Conventus Choralis\guillemotleft.

%Editio Sancti Wolfgangi \annusEditionis
\end{center}

\scriptura{}

\pars{}

\pagebreak

\renewcommand{\headrulewidth}{0pt} % no horiz. rule at the header
\fancyhf{}
\pagestyle{fancy}

\cantusSineNeumas

\ifx\oratio\undefined
\ifx\laudb\undefined
\else
\newcommand{\oratio}{\pars{Oratio.}

\noindent Dómine Deus omnípotens, qui ad princípium huius diéi nos perveníre fecísti, tua nos hódie salva virtúte, ut in hac die ad nullum declinémus peccátum, sed semper ad tuam iustítiam faciéndam nostra procédant elóquia, dirigántur cogitatiónes et ópera.

\noindent Per Dóminum nostrum Iesum Christum, Fílium tuum, qui tecum vivit et regnat in unitáte Spíritus Sancti, Deus, per ómnia sǽcula sæculórum.

\noindent \Rbardot{} Amen.}
\fi
\fi

\hora{Ad Matutinum.} %%%%%%%%%%%%%%%%%%%%%%%%%%%%%%%%%%%%%%%%%%%%%%%%%%%%%
%\sideThumbs{Matutinum}

\vspace{2mm}

\cuminitiali{}{temporalia/dominelabiamea.gtex}

\vfill
%\pagebreak

\vspace{2mm}

\ifx\invitatorium\undefined
\pars{Invitatorium.} \scriptura{Ps. 94, 1; Psalmus 94; \textbf{H451}}

\vspace{-6mm}

\antiphona{VI}{temporalia/inv-jubilemusdeo.gtex}\else
\invitatorium
\fi

\vfill
\pagebreak

\ifx\hymnusmatutinum\undefined
\ifx\matua\undefined
\else
\pars{Hymnus.}

{
\grechangedim{interwordspacetext}{0.10 cm plus 0.15 cm minus 0.05 cm}{scalable}%
\antiphona{II}{temporalia/hym-IpsumNunc.gtex}
\grechangedim{interwordspacetext}{0.22 cm plus 0.15 cm minus 0.05 cm}{scalable}%
}
\fi
\else
\hymnusmatutinum
\fi

\vspace{-3mm}

\vfill
\pagebreak

\ifx\matub\undefined
\else
% MAT B
\pars{Psalmus 1.} \scriptura{Ps. 30, 2; \textbf{H90}}

\vspace{-4mm}

\antiphona{VIII G}{temporalia/ant-intuaiustitia.gtex}

%\vspace{-2mm}

\scriptura{Ps. 30, 2-9}

%\vspace{-2mm}

\initiumpsalmi{temporalia/ps30i-initium-viii-G-auto.gtex}

\vspace{-1.5mm}

\input{temporalia/ps30i-viii-G.tex} \Abardot{}

\vfill
\pagebreak

\pars{Psalmus 2.} \scriptura{Ps. 66, 2}

\vspace{-4mm}

\antiphona{E}{temporalia/ant-illuminadomine.gtex}

%\vspace{-2mm}

\scriptura{Ps. 30, 10-17}

%\vspace{-2mm}

\initiumpsalmi{temporalia/ps30ii-initium-e-a-auto.gtex}

\input{temporalia/ps30ii-e-a.tex} \Abardot{}

\vfill
\pagebreak

\pars{Psalmus 3.} \scriptura{Ps. 30, 24}

\vspace{-4mm}

\antiphona{II D}{temporalia/ant-diligitedominum.gtex}

%\vspace{-5mm}

\scriptura{Ps. 30, 20-25}

%\vspace{-2mm}

\initiumpsalmi{temporalia/ps30iii-initium-ii-D-auto.gtex}

\input{temporalia/ps30iii-ii-D.tex} \Abardot{}

\vfill
\pagebreak
\fi

\pars{Versus.}

\ifx\matversus\undefined
\ifx\matub\undefined
\else
\noindent \Vbardot{} Dírige me, Dómine, in veritáte tua, et doce me.

\noindent \Rbardot{} Quia tu es Deus salútis meæ.
\fi
\else
\matversus
\fi

\vspace{5mm}

\sineinitiali{temporalia/oratiodominica-mat.gtex}

\vspace{5mm}

\pars{Absolutio.}

\cuminitiali{}{temporalia/absolutio-exaudi.gtex}

\vfill
\pagebreak

\cuminitiali{}{temporalia/benedictio-solemn-benedictione.gtex}

\vspace{7mm}

\lectioi

\noindent \Vbardot{} Tu autem, Dómine, miserére nobis.
\noindent \Rbardot{} Deo grátias.

\vfill
\pagebreak

\responsoriumi

\vfill
\pagebreak

\cuminitiali{}{temporalia/benedictio-solemn-unigenitus.gtex}

\vspace{7mm}

\lectioii

\noindent \Vbardot{} Tu autem, Dómine, miserére nobis.
\noindent \Rbardot{} Deo grátias.

\vfill
\pagebreak

\responsoriumii

\vfill
\pagebreak

\cuminitiali{}{temporalia/benedictio-solemn-spiritus.gtex}

\vspace{7mm}

\lectioiii

\noindent \Vbardot{} Tu autem, Dómine, miserére nobis.
\noindent \Rbardot{} Deo grátias.

\vfill
\pagebreak

\responsoriumiii

\vfill
\pagebreak

\rubrica{Reliqua omittuntur, nisi Laudes separandæ sint.}

\sineinitiali{temporalia/domineexaudi.gtex}

\vfill

\oratio

\vfill

\noindent \Vbardot{} Dómine, exáudi oratiónem meam.
\Rbardot{} Et clamor meus ad te véniat.

\vfill

\noindent \Vbardot{} Benedicámus Dómino.
\noindent \Rbardot{} Deo grátias.

\vfill

\noindent \Vbardot{} Fidélium ánimæ per misericórdiam Dei requiéscant in pace.
\Rbardot{} Amen.

\vfill
\pagebreak

\hora{Ad Laudes.} %%%%%%%%%%%%%%%%%%%%%%%%%%%%%%%%%%%%%%%%%%%%%%%%%%%%%
%\sideThumbs{Laudes}

\cantusSineNeumas

\vspace{0.5cm}
\grechangedim{interwordspacetext}{0.18 cm plus 0.15 cm minus 0.05 cm}{scalable}%
\cuminitiali{}{temporalia/deusinadiutorium-communis.gtex}
\grechangedim{interwordspacetext}{0.22 cm plus 0.15 cm minus 0.05 cm}{scalable}%

\vfill
\pagebreak

\ifx\hymnuslaudes\undefined
\ifx\laudbd\undefined
\else
\pars{Hymnus} \scriptura{Hilarius (\olddag{} 367)}

\grechangedim{interwordspacetext}{0.16 cm plus 0.15 cm minus 0.05 cm}{scalable}%
\cuminitiali{IV}{temporalia/hym-LucisLargitor.gtex}
\grechangedim{interwordspacetext}{0.22 cm plus 0.15 cm minus 0.05 cm}{scalable}%
\vspace{-3mm}
\fi
\else
\hymnuslaudes
\fi

\vfill
\pagebreak

\ifx\laudb\undefined
\else
\pars{Psalmus 1.} \scriptura{Ps. 41, 3; \textbf{H391}}

\vspace{-4mm}

\antiphona{II D}{temporalia/ant-sitivitanima.gtex}

%\vspace{-2mm}

\scriptura{Psalmus 41}

%\vspace{-2mm}

\initiumpsalmi{temporalia/ps41-initium-ii-D-auto.gtex}

%\vspace{-1.5mm}

\input{temporalia/ps41-ii-D.tex}

\vfill

\antiphona{}{temporalia/ant-sitivitanima.gtex}

\vfill
\pagebreak

\pars{Psalmus 2.}

\vspace{-4mm}

\antiphona{III a}{temporalia/ant-ostendenobisdomine.gtex}

%\vspace{-2mm}

\scriptura{Canticum Ecclesiastici, Sir. 36, 1-7.13-16}

%\vspace{-3mm}

\initiumpsalmi{temporalia/ecclesiastici-initium-iii-a-auto.gtex}

\input{temporalia/ecclesiastici-iii-a.tex} \Abardot{}

\vfill
\pagebreak

\pars{Psalmus 3.}

\vspace{-4mm}

\antiphona{II D}{temporalia/ant-operamanuumeius.gtex}

\scriptura{Psalmus 18, 1-7}

\initiumpsalmi{temporalia/ps18i-initium-ii-D-auto.gtex}

\input{temporalia/ps18i-ii-D.tex} \Abardot{}

\vfill
\pagebreak
\fi

\ifx\lectiobrevis\undefined
\ifx\laudb\undefined
\else
\pars{Lectio Brevis.} \scriptura{Ier. 15, 16}

\noindent Invénti sunt sermónes tui, et comédi eos, et factum est mihi verbum tuum in gáudium et in lætítiam cordis mei, quóniam invocátum est nomen tuum super me, Dómine Deus exercítuum.
\fi
\else
\lectiobrevis
\fi

\vfill

\ifx\responsoriumbreve\undefined
\ifx\laudbd\undefined
\else
\pars{Responsorium breve.} \scriptura{Ps. 32, 1.3}

\cuminitiali{VI}{temporalia/resp-exsultateiusti.gtex}
\fi
\else
\responsoriumbreve
\fi

\vfill
\pagebreak

\ifx\benedictus\undefined
\ifx\laudbd\undefined
\else
\pars{Canticum Zachariæ.} \scriptura{Lc. 1, 68; \textbf{H422}}

\vspace{-4mm}

{
\grechangedim{interwordspacetext}{0.18 cm plus 0.15 cm minus 0.05 cm}{scalable}%
\antiphona{IV E}{temporalia/ant-benedictusdominus.gtex}
\grechangedim{interwordspacetext}{0.22 cm plus 0.15 cm minus 0.05 cm}{scalable}%
}

%\vspace{-3mm}

\scriptura{Lc. 1, 68-79}

%\vspace{-2mm}

\cantusSineNeumas
\initiumpsalmi{temporalia/benedictus-initium-iv-E-auto.gtex}

%\vspace{-1.5mm}

\input{temporalia/benedictus-iv-E.tex} \Abardot{}
\fi
\else
\benedictus
\fi

\vspace{-1cm}

\vfill
\pagebreak

%\sideThumbs{{\scriptsize{}Fine horarum}}

\pars{Preces.}

\sineinitiali{}{temporalia/tonusprecum.gtex}

\ifx\preces\undefined
\ifx\laudb\undefined
\else
\noindent Salvátor noster fecit nos regnum et sacerdótium, ut hóstias Deo acceptábiles offerámus. \gredagger{} Grati ígitur eum invocémus:

\Rbardot{} Serva nos in tuo ministério, Dómine.

\noindent Christe, sacérdos ætérne, qui sanctum pópulo tuo sacerdótium concessísti, \gredagger{} concéde, ut spiritáles hóstias Deo acceptábiles iúgiter offerámus.

\Rbardot{} Serva nos in tuo ministério, Dómine.

\noindent Spíritus tui fructus nobis largíre propítius, \gredagger{} patiéntiam, benignitátem et mansuetúdinem.

\Rbardot{} Serva nos in tuo ministério, Dómine.

\noindent Da nobis te amáre, ut te, qui es cáritas, possideámus, \gredagger{} et bene ágere, ut per vitam étiam nostram te laudémus.

\Rbardot{} Serva nos in tuo ministério, Dómine.

\noindent Quæ frátribus nostris sunt utília, nos quǽrere concéde, \gredagger{} ut salútem facílius consequántur.

\Rbardot{} Serva nos in tuo ministério, Dómine.
\fi
\else
\preces
\fi

\vfill

\pars{Oratio Dominica.}

\cuminitiali{}{temporalia/oratiodominicaalt.gtex}

\vfill
\pagebreak

\rubrica{vel:}

\pars{Supplicatio Litaniæ.}

\cuminitiali{}{temporalia/supplicatiolitaniae.gtex}

\vfill

\pars{Oratio Dominica.}

\cuminitiali{}{temporalia/oratiodominica.gtex}

\vfill
\pagebreak

% Oratio. %%%
\oratio

\vspace{-1mm}

\vfill

\rubrica{Hebdomadarius dicit Dominus vobiscum, vel, absente sacerdote vel diacono, sic concluditur:}

\vspace{2mm}

\antiphona{C}{temporalia/dominusnosbenedicat.gtex}

\rubrica{Postea cantatur a cantore:}

\vspace{2mm}

\cuminitiali{IV}{temporalia/benedicamus-feria-laudes.gtex}

\vspace{1mm}

\vfill
\pagebreak

\end{document}

