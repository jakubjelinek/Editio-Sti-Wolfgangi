\newcommand{\lectioi}{\pars{Lectio I.} \scriptura{Is. 48, 12-16}

\noindent De libro Isaíæ prophétæ.

\noindent Hæc dicit Dóminus: «Audi me, Iacob, et, Israel, quem ego vocávi; ego, ego primus et ego novíssimus. Manus mea fundávit terram, et déxtera mea expándit cælos; ego voco eos, et stant simul. Congregámini, omnes vos et audíte: Quis de eis annuntiávit hæc? Dóminus diléxit eum; fáciet voluntátem suam in Babylóne et bráchium suum in Chaldǽis. Ego, ego locútus sum et vocávi eum; addúxi eum, et próspera fuit via eius. Accédite ad me et audíte hoc: Non a princípio in abscóndito locútus sum; ex témpore, ántequam fíeret, ibi eram; et nunc Dóminus Deus misit me cum spíritu suo».}
\newcommand{\responsoriumi}{\pars{Responsorium 1.} \scriptura{\Rbardot{} Num. 24, 17 \Vbardot{} Ps. 71, 10; \textbf{H34}}

\vspace{-5mm}

\responsorium{VIII}{temporalia/resp-orieturstella-CROCHU.gtex}{}}
\newcommand{\lectioii}{\pars{Lectio II.} \scriptura{Is. 48, 17-21; 49, 9b-13}

\noindent Hæc dicit Dóminus, redémptor tuus, Sanctus Israel: «Ego Dóminus Deus tuus docens te utília, gubérnans te in via, qua ámbulas. Utinam attendísses mandáta mea! Facta fuísset sicut flumen pax tua, et iustítia tua sicut gúrgites maris; et fuísset quasi aréna semen tuum, et stirps úteri tui ut lapílli eius; non interísset et non fuísset attrítum nomen eius a fácie mea. Egredímini de Babylóne, fúgite a Chaldǽis, in voce exsultatiónis annuntiáte: “Redémit Dóminus servum suum Iacob”. Non sitiérunt, cum per desértum dúceret eos; aquam de petra prodúxit eis et scidit petram, et fluxérunt aquæ. Super vias pascéntur, et in ómnibus cóllibus decalvátis páscua eórum; non esúrient neque sítient, et non percútiet eos æstus vel sol, quia miserátor eórum reget eos et ad fontes aquárum addúcet eos. Et ponam omnes montes meos in viam, et sémitæ meæ exaltabúntur. Ecce isti de longe vénient, et ecce illi ab aquilóne et mari, et isti de terra Sinim». Laudáte, cæli, et exsúlta, terra; iubiláte, montes, laudem, quia consolátur Dóminus pópulum suum et páuperum suórum miserétur.}
\newcommand{\responsoriumii}{\pars{Responsorium 2.} \scriptura{\Rbar{} Heb. 6, 20 \Vbar{} Is. 40, 10; \textbf{H35}}

\vspace{-5mm}

\responsorium{VII}{temporalia/resp-praecursorpronobis-CROCHU.gtex}{}}
\newcommand{\lectioiii}{\pars{Lectio III.} \scriptura{Lib. 2, 19. 22-23. 26-27: CCL 14, 39-42}

\noindent Ex Expositióne sancti Ambrósii epíscopi in Lucam.

\noindent Angelus, cum abscóndita nuntiáret, ut fides astruerétur exémplo, senióris féminæ sterilísque concéptum Vírgini Maríæ nuntiávit, ut possíbile Deo omne quod ei placúerit asséreret.

\noindent Ubi audívit hoc María, non quasi incrédula de oráculo nec quasi incérta de núntio nec quasi dúbitans de exémplo, sed quasi læta pro voto, religiósa pro offício, festína pro gáudio in montána perréxit.

\noindent Ut audívit salutatiónem Maríæ Elísabeth, exsultávit infans in útero eius et repléta est Spíritu Sancto. Beáta, inquit, quæ credidísti.

\noindent Sed et vos beáti, qui audístis et credidístis; quæcúmque enim credíderit ánima, et cóncipit et génerat Dei Verbum et ópera eius agnóscit.}
\newcommand{\responsoriumiii}{\pars{Responsorium 3.} \scriptura{\Rbar{} Is. 62, 2 \Vbar{} ibid. 62, 3; \textbf{H35}}

\vspace{-5mm}

\responsorium{IV}{temporalia/resp-videbuntgentesiustumtuum-CROCHU-cumdox.gtex}{}}
\newcommand{\lectiobrevis}{\pars{Lectio Brevis.} \scriptura{Is. 7, 14-15}

\noindent Ecce, Virgo concípiet et páriet fílium et vocábit nomen eius Emmánuel; butýrum et mel cómedet, ut ipse sciat reprobáre malum et elígere bonum.}
\newcommand{\benedictus}{\pars{Canticum Zachariæ.} \scriptura{\textbf{H37}}

\vspace{-4mm}

{
\grechangedim{interwordspacetext}{0.18 cm plus 0.15 cm minus 0.05 cm}{scalable}%
\antiphona{VIII g}{temporalia/ant-nolitetimere.gtex}
\grechangedim{interwordspacetext}{0.22 cm plus 0.15 cm minus 0.05 cm}{scalable}%
}

%\trAntIMagnificat

%\vspace{-2mm}

\scriptura{Lc. 1, 68-79}

%\vspace{-2mm}

\cantusSineNeumas
\initiumpsalmi{temporalia/benedictus-initium-viii-g-auto.gtex}

%\vspace{-1.5mm}

%\psalmusEtTranslatioT{temporalia/benedictus-XIX-comb.tex}{10.2cm}
\input{temporalia/benedictus-XIX.tex} \Abardot{}}
\newcommand{\oratio}{Orémus:

\noindent Preces pópuli tui, quǽsumus, Dómine, cleménter exáudi, ut qui de Unigéniti tui in nostra carne advéntu lætántur, cum vénerit in sua maiestáte, ætérnæ vitæ prǽmium consequántur.

\noindent Per Dóminum nostrum Iesum Christum, Fílium tuum, qui tecum vivit et regnat in unitáte Spíritus Sancti, Deus, per ómnia sǽcula sæculórum.

\noindent \Rbardot{} Amen.}
\newcommand{\magnificat}{\pars{Canticum B. Mariæ V.} \scriptura{Sap. 7, 26; Mal. 4, 2; Lc. 1, 78-79; \textbf{H40}}

\vspace{-5mm}

{
\grechangedim{interwordspacetext}{0.18 cm plus 0.15 cm minus 0.05 cm}{scalable}%
\antiphona{II D}{temporalia/ant-ooriens.gtex}
\grechangedim{interwordspacetext}{0.22 cm plus 0.15 cm minus 0.05 cm}{scalable}%
}

%\trAntIMagnificat

\vspace{-2mm}

\scriptura{Lc. 1, 46-55}

\vspace{-2mm}

\cantusSineNeumas

\initiumpsalmi{temporalia/magnificat-initium-iisoll-D.gtex}

\vspace{-1.5mm}

%\psalmusEtTranslatioT{temporalia/magnificat-XXXII-comb.tex}{10.2cm}
\input{temporalia/magnificat-XXXII.tex} \Abardot{}

\vspace{-1cm}}
\newcommand{\hebdomada}{infra Hebdom. IV post Pentecosten.}
\newcommand{\oratioLaudes}{\cuminitiali{}{temporalia/oratio4.gtex}}

% LuaLaTeX

\documentclass[a4paper, twoside, 12pt]{article}
\usepackage[latin]{babel} 
%\usepackage[landscape, left=3cm, right=1.5cm, top=2cm, bottom=1cm]{geometry} % okraje stranky
%\usepackage[landscape, a4paper, mag=1166, truedimen, left=2cm, right=1.5cm, top=1.6cm, bottom=0.95cm]{geometry} % okraje stranky
\usepackage[landscape, a4paper, mag=1400, truedimen, left=0.5cm, right=0.5cm, top=0.5cm, bottom=0.5cm]{geometry} % okraje stranky

\usepackage{fontspec}
\setmainfont[FeatureFile={junicode.fea}, Ligatures={Common, TeX}, RawFeature=+fixi]{Junicode}
%\setmainfont{Junicode}

% shortcut for Junicode without ligatures (for the Czech texts)
\newfontfamily\nlfont[FeatureFile={junicode.fea}, Ligatures={Common, TeX}, RawFeature=+fixi]{Junicode}

% Hebrew font:
% http://scripts.sil.org/cms/scripts/page.php?site_id=nrsi&id=SILHebrUnic2
\newfontfamily\hebfont[Scale=1]{Ezra SIL}

\usepackage{multicol}
\usepackage{color}
\usepackage{lettrine}
\usepackage{fancyhdr}

% usual packages loading:
\usepackage{luatextra}
\usepackage{graphicx} % support the \includegraphics command and options
\usepackage{gregoriotex} % for gregorio score inclusion
\usepackage{gregoriosyms}
\usepackage{wrapfig} % figures wrapped by the text
\usepackage{parcolumns}
\usepackage[contents={},opacity=1,scale=1,color=black]{background}
\usepackage{tikzpagenodes}
\usepackage{calc}
\usepackage{longtable}
\usetikzlibrary{calc}

\setlength{\headheight}{14.5pt}

\input{conventuscommune.tex} % Often used macros
%%%% Preklady jednotlivych zpevu (nektere se opakuji, a je dobre mit je
% vsechny na jedne hromade)

% HOURS ---

\newcommand{\trAntI}{\translatioCantus{Muž boží měl kožený toulec, pečlivě
zavázaný, jenž mu visel na šíji a~často se ho dotýkal.}}

\newcommand{\trAntII}{\translatioCantus{Klíč od~něho tak dobře střežil, že
dokud žil v~těle, nikdo z~jeho žáků nezvěděl, co je uvnitř.}}

\newcommand{\trAntIII}{\translatioCantus{Ale když se odebral z~tohoto
života, schránku otevřeli a~objevili v~ní žíněné roucho a~měděný řetěz
potřísněný krví.}}

\newcommand{\trAntIV}{\translatioCantus{A když prohlédli mistrovo tělo,
nalezli jeho tělo na čtyřech místech hluboce zbrázděno ranami od řetězu.}}

\newcommand{\trAntV}{\translatioCantus{Krev vytékající z~těch ran, místy
prostoupila i~žíněným rouchem.}}

\newcommand{\trCapituli}{\translatioCantus{
Miláčkovi Boha a~lidí,
Mojžíšovi požehnané paměti,~\gredagger{}
dopřál slávu rovnou slávě svatých~\grestar{}
učinil ho mocným na postrach nepřátelům
a~jeho slovy zastavil divy.}}

\newcommand{\trLectioBrevis}{\translatioCantus{
Pamatujte na své představené,
kteří vám hlásali Boží slovo.
Uvažte, jak oni skončili život, a~napodobujte jejich víru.
Ježíš Kristus je stejný včera i~dnes i~navěky.
Nenechte se svést věelijakými cizími naukami.}}

\newcommand{\trRespLaud}{\translatioCantus{Spravedlivého vodil Hospodin~\grestar{}
po přímých stezkách. \Vbardot{} A~ukázal mu Boží království.}}

\newcommand{\trRespLaudB}{\translatioCantus{Na tvých hradbách, Jeruzaléme,
ustanovil jsem strážné;~\grestar{}
budou bdít nad mým lidem. \Vbardot{} Ani ve dne, ani v~noci nesmějí nikdy
mlčet.}}

\newcommand{\trVersus}{\translatioCantus{\Vbardot{} Ústa spravedlivého šeptají moudrost, aleluja.
\Rbardot{} A~jeho jazyk ohlašuje právo, aleluja.}}

\newcommand{\trAntBenedictus}{\translatioCantus{Když na bujné oře vložili
nosítka a~sňali jim uzdu, vydali se přímo k~cele božího muže.}}

\newcommand{\trPreces}{\translatioCantus{
\noindent S vděčností chvalme Krista, dobrého Pastýře, \gredagger{} který dal život za své ovce, \grestar{} a~pokorně ho prosme: \Rbardot{} Pane, buď pastýřem svého lidu.

\noindent Kriste, ty dáváš církvi pastýře, a~jejich službou se ujímáš svého lidu, \grestar{} dej, ať v~lásce těch, kteří nás vedou, poznáváme, jak nás miluješ. \Rbardot{} Pane, buď pastýřem svého lidu.

\noindent Ty stále konáš skrze své zástupce službu pastýře a~učitele, \grestar{} nepřestávej nás nikdy vést prostřednictvím svých služebníků. \Rbardot{} Pane, buď pastýřem svého lidu.

\noindent Ty prokazuješ svému lidu skrze jeho pastýře službu lékaře duše i~těla, \grestar{} ochraňuj náš život a~veď nás ke svatosti. \Rbardot{} Pane, buď pastýřem svého lidu.

\noindent Ty posíláš své svaté, aby slovem i~příkladem vedli tvůj lid k~tobě, \grestar{} na jejich přímluvu nás posiluj, abychom vytrvali na cestě, která vede k~věčnému životu. \Rbardot{} Pane, buď pastýřem svého lidu.}}

\newcommand{\trOrationis}{\translatioCantus{Bože, jenž nám dopřáváš radovat
se z~výroční slavnosti svatého tvého vyznavače Havla, uděl dobrotivě,
abychom když slavíme jeho narození, též se řídili podobou jeho skutků.
Skrze…}}
 % Czech translations of the proper texts

\newcommand{\annusEditionis}{2020}

\def\hebinitial#1{%
\leavevmode{\newbox\hebbox\setbox\hebbox\hbox{\hebfont{#1}\hskip 1mm}\kern -\wd\hebbox\hbox{\hebfont{#1}\hskip 1mm}}%
}

%%%% Vicekrat opakovane kousky

\newcommand{\anteOrationem}{
  \rubrica{Ante Orationem, cantatur a Superiore:}

  \pars{Supplicatio Litaniæ.}

  \cuminitiali{}{temporalia/supplicatiolitaniae.gtex}

  \pars{Oratio Dominica.}

  \cuminitiali{}{temporalia/oratiodominica.gtex}

  \rubrica{Deinde dicitur ab Hebdomadario:}

  \cuminitiali{}{temporalia/dominusvobiscum-solemnis.gtex}

  \rubrica{In choro monialium loco Dominus vobiscum dicitur:}

  \sineinitiali{temporalia/domineexaudi.gtex}
}

\setlength{\columnsep}{30pt} % prostor mezi sloupci

%%%%%%%%%%%%%%%%%%%%%%%%%%%%%%%%%%%%%%%%%%%%%%%%%%%%%%%%%%%%%%%%%%%%%%%%%%%%%%%%%%%%%%%%%%%%%%%%%%%%%%%%%%%%%
\begin{document}

% Here we set the space around the initial.
% Please report to http://home.gna.org/gregorio/gregoriotex/details for more details and options
\grechangedim{afterinitialshift}{2.2mm}{scalable}
\grechangedim{beforeinitialshift}{2.2mm}{scalable}

\grechangedim{interwordspacetext}{0.22 cm plus 0.15 cm minus 0.05 cm}{scalable}%
\grechangedim{annotationraise}{-0.2cm}{scalable}

% Here we set the initial font. Change 38 if you want a bigger initial.
% Emit the initials in red.
\grechangestyle{initial}{\color{red}\fontsize{38}{38}\selectfont}

\pagestyle{empty}

%%%% Titulni stranka
\begin{titulusOfficii}
\nomenFesti{Feria II infra Hebdom. Ultima Adventus.}
\end{titulusOfficii}

\pars{}

\scriptura{}

\pagebreak

% graphic
\renewcommand{\headrulewidth}{0pt} % no horiz. rule at the header
\fancyhf{}
\pagestyle{fancy}

\cantusSineNeumas

\hora{Ad Matutinum.}

\vspace{2mm}

\cuminitiali{}{temporalia/dominelabiamea.gtex}

\vspace{2mm}

\pars{Invitatorium.} \scriptura{Phil. 4, 4.5}

\vspace{-6mm}

\antiphona{VI}{temporalia/inv-propeestiamsimplex.gtex}

\vfill
\pagebreak

\pars{Hymnus.}

\vspace{-5mm}

{
\grechangedim{interwordspacetext}{0.30 cm plus 0.15 cm minus 0.05 cm}{scalable}%
\antiphona{II}{temporalia/hym-VeniRedemptor.gtex}
\grechangedim{interwordspacetext}{0.22 cm plus 0.15 cm minus 0.05 cm}{scalable}%
}
%{
%\vspace{-5mm}
%\setlength{\columnsep}{0pt} % prostor mezi sloupci
%\input{hym-VeniRedemptor-bohtext.tex}
%\setlength{\columnsep}{30pt} % prostor mezi sloupci
%}

\vfill
\pagebreak

\pars{Psalmus 1.} \scriptura{Ps. 13, 2}

\vspace{-4mm}

\antiphona{II D}{temporalia/ant-dominusdecaelo.gtex}

%\vspace{-5mm}

\scriptura{Ps. 13}

%\vspace{-2mm}

\initiumpsalmi{temporalia/ps13-initium-ii-D-auto.gtex}

\input{temporalia/ps13-ii-D.tex} \Abardot{}

\vfill
\pagebreak

\pars{Psalmus 2.} \scriptura{Ps. 14, 1; \textbf{H222}}

\vspace{-4mm}

\antiphona{IV* e}{temporalia/ant-habitabit.gtex}

%\vspace{-5mm}

\scriptura{Ps. 14}

\initiumpsalmi{temporalia/ps14-initium-iv_-e-auto.gtex}

\input{temporalia/ps14-iv_-e.tex} \Abardot{}

\vfill
\pagebreak

\pars{Psalmus 3.} \scriptura{Ps. 16, 6; \textbf{H99}}

\vspace{-4mm}

\antiphona{VII c}{temporalia/ant-inclinadomine.gtex}

%\vspace{-5mm}

\scriptura{Ps. 16}

\vspace{-1mm}

\initiumpsalmi{temporalia/ps16-initium-vii-c-auto.gtex}

\input{temporalia/ps16-I.tex}

\vfill

\antiphona{}{temporalia/ant-inclinadomine.gtex}

\vfill
\pagebreak

\pars{Psalmus 4.} \scriptura{Ps. 17, 2}

\vspace{-4mm}

\antiphona{VI F}{temporalia/ant-diligamtedomine.gtex}

%\vspace{-5mm}

\scriptura{Ps. 17, 2-16}

\vspace{-1mm}

\initiumpsalmi{temporalia/ps17i-initium-vi-F-auto.gtex}

\input{temporalia/ps17i-vi-F.tex}

\vfill

\antiphona{}{temporalia/ant-diligamtedomine.gtex}

\vfill
\pagebreak

\pars{Psalmus 5.} \scriptura{Ps. 17, 21}

\vspace{-4mm}

\antiphona{IV* e}{temporalia/ant-retribuetmihi.gtex}

%\vspace{-5mm}

\scriptura{Ps. 17, 17-35}

\vspace{-1mm}

\initiumpsalmi{temporalia/ps17ii-initium-iv_-e-auto.gtex}

\input{temporalia/ps17ii-iv_-e.tex}

\vfill

\antiphona{}{temporalia/ant-retribuetmihi.gtex}

\vfill
\pagebreak

\pars{Psalmus 6.} \scriptura{Ps. 17, 47; \textbf{H100}}

\vspace{-4mm}

\antiphona{VII c\textsuperscript{2}}{temporalia/ant-vivitdominus.gtex}

%\vspace{-5mm}

\scriptura{Ps. 17, 36-51}

\vspace{-1mm}

\initiumpsalmi{temporalia/ps17iii-initium-vii-c2-auto.gtex}

\input{temporalia/ps17iii-vii-c2.tex}

\vfill

\antiphona{}{temporalia/ant-vivitdominus.gtex}

\vfill
\pagebreak

\pars{Psalmus 7.} \scriptura{Ps. 19, 2; \textbf{H100}}

\vspace{-4mm}

\antiphona{VIII G}{temporalia/ant-exaudiatte.gtex}

%\vspace{-5mm}

\scriptura{Ps. 19}

\vspace{-1mm}

\initiumpsalmi{temporalia/ps19-initium-viii-G-auto.gtex}

\input{temporalia/ps19-viii-G.tex} \Abardot{}

\vfill
\pagebreak

\pars{Psalmus 8.} \scriptura{Ps. 20, 1; \textbf{H89}}

\vspace{-4mm}

\antiphona{VIII G}{temporalia/ant-domineinvirtute.gtex}

%\vspace{-5mm}

\scriptura{Ps. 20}

\vspace{-1mm}

\initiumpsalmi{temporalia/ps20-initium-viii-G-auto.gtex}

\input{temporalia/ps20-viii-G.tex} \Abardot{}

\vfill
\pagebreak

\pars{Psalmus 9.} \scriptura{Ps. 29, 2; \textbf{H262}}

\vspace{-7mm}

\antiphona{VIII G}{temporalia/ant-exaltabote.gtex}

\vspace{-3mm}

\scriptura{Ps. 29}

\vspace{-2mm}

\initiumpsalmi{temporalia/ps29-initium-viii-G-auto.gtex}

\vspace{-1.5mm}

\input{temporalia/ps29-viii-G.tex} \Abardot{}

\vfill
\pagebreak

\pars{Versus.} \scriptura{Mc. 1, 3; Is. 40, 3 }

% Versus. %%%
\sineinitiali{temporalia/versus-voxclamantis-simplex.gtex}

\vspace{5mm}

\sineinitiali{temporalia/oratiodominica-mat.gtex}

\vspace{5mm}

\pars{Absolutio.}

\cuminitiali{}{temporalia/absolutio-exaudi.gtex}

\vfill
\pagebreak

\cuminitiali{}{temporalia/benedictio-solemn-benedictione.gtex}

\vspace{7mm}

\lectioi

\noindent \Vbardot{} Tu autem, Dómine, miserére nobis.
\noindent \Rbardot{} Deo grátias.

\vfill
\pagebreak

\responsoriumi

\vfill
\pagebreak

\cuminitiali{}{temporalia/benedictio-solemn-unigenitus.gtex}

\vspace{7mm}

\lectioii

\noindent \Vbardot{} Tu autem, Dómine, miserére nobis.
\noindent \Rbardot{} Deo grátias.

\vfill
\pagebreak

\responsoriumii

\vfill
\pagebreak

\cuminitiali{}{temporalia/benedictio-solemn-spiritus.gtex}

\vspace{7mm}

\lectioiii

\noindent \Vbardot{} Tu autem, Dómine, miserére nobis.
\noindent \Rbardot{} Deo grátias.

\vfill
\pagebreak

\responsoriumiii

\vfill
\pagebreak

\rubrica{Reliqua omittuntur, nisi Laudes separandæ sint.}

\pars{Oratio}

\noindent \Vbardot{} Dómine, exáudi oratiónem meam.

\noindent \Rbardot{} Et clamor meus ad te véniat.

\oratio

\vspace{7mm}

\pars{Conclusio}

\noindent \Vbardot{} Dómine, exáudi oratiónem meam.

\noindent \Rbardot{} Et clamor meus ad te véniat.

\noindent \Vbardot{} Benedicámus Dómino.

\noindent \Rbardot{} Deo grátias.

\noindent \Vbardot{} Fidélium ánimæ per misericórdiam Dei requiéscant in pace.

\noindent \Rbardot{} Amen.

\vfill
\pagebreak

\hora{Ad Laudes.} %%%%%%%%%%%%%%%%%%%%%%%%%%%%%%%%%%%%%%%%%%%%%%%%%%%%%
%\sideThumbs{Laudes}

\cantusSineNeumas

\vspace{0.5cm}
\grechangedim{interwordspacetext}{0.18 cm plus 0.15 cm minus 0.05 cm}{scalable}%
\cuminitiali{}{temporalia/deusinadiutorium-communis.gtex}
\grechangedim{interwordspacetext}{0.22 cm plus 0.15 cm minus 0.05 cm}{scalable}%

\vfill
%\pagebreak

\vspace{5mm}

\pars{Psalmus 1.} \scriptura{\textbf{H36}}

\vspace{-4mm}

\antiphona{II* b}{temporalia/ant-eccevenietdominus.gtex}

\scriptura{Psalmus 50.}

\initiumpsalmi{temporalia/ps50-initium-ii_-B-auto.gtex}

\input{temporalia/ps50-ii_-B.tex}

\vfill

\antiphona{}{temporalia/ant-eccevenietdominus.gtex}

\vfill
\pagebreak

\pars{Psalmus 2.} \scriptura{Lc. 18, 8; \textbf{H36}}

\vspace{-6mm}

\antiphona{VIII C}{temporalia/ant-dumvenerit.gtex}

\vspace{-3mm}

\scriptura{Psalmus 5.}

\vspace{-2mm}

\initiumpsalmi{temporalia/ps5-initium-viii-C-auto.gtex}

\vspace{-1.5mm}

\input{temporalia/ps5-viii-C.tex} \Abardot{}

\vfill
\pagebreak

\pars{Psalmus 3.} \scriptura{Is. 25, 9; \textbf{H34}}

\vspace{-4mm}

\antiphona{VIII G}{temporalia/ant-eccedeusnoster.gtex}

\scriptura{Psalmus 35.}

\initiumpsalmi{temporalia/ps35-initium-viii-G-auto.gtex}

\input{temporalia/ps35-viii-A.tex} \Abardot{}

\vfill
\pagebreak

\pars{Psalmus 4.} \scriptura{Is. 46, 13; \textbf{H31}}

\vspace{-4mm}

\antiphona{VIII G}{temporalia/ant-ponaminsion.gtex}

\scriptura{Canticum Isaiæ Prophetæ, Is. 12, 1-7}

\initiumpsalmi{temporalia/isaiae-initium-viii-G-auto.gtex}

\input{temporalia/isaiae-viii-G.tex} \Abardot{}

\vfill
\pagebreak

\pars{Psalmus 5.}\scriptura{\textbf{H37}}

\vspace{-4mm}

\antiphona{II* a}{temporalia/ant-egredieturdominus.gtex}

\scriptura{Psalmus 148.}

\initiumpsalmi{temporalia/ps148-initium-ii_-a-auto.gtex}

\input{temporalia/ps148-ii_-a-sinedox.tex}

\rubrica{Hic non dicitur Gloria Patri.}

\vfill
\pagebreak

%
\scriptura{Psalmus 149.}

\initiumpsalmi{temporalia/ps149-initium-ii_-a-auto.gtex}

\input{temporalia/ps149-ii_-a-sinedox.tex}

\rubrica{Hic non dicitur Gloria Patri.}

\vfill
\pagebreak

%
\scriptura{Psalmus 150.}

\initiumpsalmi{temporalia/ps150-initium-ii_-a-auto.gtex}

\input{temporalia/ps150-ii_-a.tex}

\vfill

\vspace{-6mm}

\antiphona{}{temporalia/ant-egredieturdominus.gtex} % repeat the antiphon - new page

\vfill
\pagebreak

\lectiobrevis

% preklad Jeruz. bible
%\trCapituliI

\vfill

\responsoriumbreve

%\trResp

\vfill
\pagebreak

\pars{Hymnus}

\cuminitiali{D}{temporalia/hym-MagnisProphetae.gtex}
\vspace{-3mm}
%\input{hym-MagnisProphetae-bohtext.tex}

\vfill

\pars{Versus.} \scriptura{Mc. 1, 3; Is. 40, 3}

% Versus. %%%
\sineinitiali{temporalia/versus-voxclamantis.gtex}

%\noindent \trVersus

\vfill
\pagebreak

\benedictus

\vfill
\pagebreak

%\sideThumbs{{\scriptsize{}Fine horarum}}

\rubrica{Ante Orationem, cantatur a Superiore:}

\pars{Supplicatio Litaniæ.}

\cuminitiali{}{temporalia/supplicatiolitaniae.gtex}

\pars{Oratio Dominica.}

\cuminitiali{}{temporalia/oratiodominica.gtex}

\vfill
\pagebreak

% Oratio. %%%
\oratio

\vspace{-1mm}
%\trOrationisI

\vfill

\rubrica{Hebdomadarius dicit Dominus vobiscum, vel, absente sacerdote vel diacono, sic concluditur:}

\vspace{2mm}

\antiphona{C}{temporalia/dominusnosbenedicat.gtex}

\rubrica{Postea cantatur a cantore:}

\vspace{2mm}

\cuminitiali{IV}{temporalia/benedicamus-dominica-advequad.gtex}

\vspace{1mm}

\vfill
\pagebreak

\hora{Ad Vesperas.} %%%%%%%%%%%%%%%%%%%%%%%%%%%%%%%%%%%%%%%%%%%%%%%%%%%%%
%\sideThumbs{Vesperæ}

\cantusSineNeumas

%\vspace{0.5cm}
\grechangedim{interwordspacetext}{0.18 cm plus 0.15 cm minus 0.05 cm}{scalable}%
\cuminitiali{}{temporalia/deusinadiutorium-communis.gtex}
\grechangedim{interwordspacetext}{0.22 cm plus 0.15 cm minus 0.05 cm}{scalable}%

\vfill
%\pagebreak

%\vspace{2mm}

\pars{Psalmus 1.} \scriptura{\textbf{H36}}

\vspace{-6mm}

\antiphona{II* b}{temporalia/ant-eccevenietdominus.gtex}

\vspace{-4mm}

\scriptura{Psalmus 113.}

\initiumpsalmi{temporalia/ps113-initium-ii_-B-auto.gtex}

\vspace{-1.5mm}

\input{temporalia/ps113-ii_-B.tex}

\vfill

\vspace{-6mm}

\antiphona{}{temporalia/ant-eccevenietdominus.gtex}

\vspace{-1cm}

\vfill
\pagebreak

\pars{Psalmus 2.} \scriptura{Lc. 18, 8; \textbf{H36}}

\vspace{-4mm}

\antiphona{VIII C}{temporalia/ant-dumvenerit.gtex}

\vspace{-4mm}

\scriptura{Psalmus 114.}

\initiumpsalmi{temporalia/ps114-initium-viii-C-auto.gtex}

\input{temporalia/ps114-viii-C.tex} \Abardot{}

\vfill
\pagebreak

\pars{Psalmus 3.} \scriptura{Is. 25, 9; \textbf{H34}}

\vspace{-4mm}

\antiphona{VIII G}{temporalia/ant-eccedeusnoster.gtex}

\vspace{-4mm}

\scriptura{Psalmus 115.}

\initiumpsalmi{temporalia/ps115-initium-viii-G-auto.gtex}

\input{temporalia/ps115-viii-G-sinedox.tex}

\rubrica{Hic non dicitur Gloria Patri.}

\vfill

\scriptura{Psalmus 116.}

\initiumpsalmi{temporalia/ps116-initium-viii-G-auto.gtex}

\input{temporalia/ps116-viii-G.tex} \Abardot{}

\vfill
\pagebreak

\pars{Psalmus 4.} \scriptura{\textbf{H37}}

\vspace{-4mm}

\antiphona{II* a}{temporalia/ant-egredieturdominus.gtex}

\vspace{-4mm}

\scriptura{Psalmus 128.}

\initiumpsalmi{temporalia/ps128-initium-ii_-a.gtex}

\input{temporalia/ps128-ii_-a.tex} \Abardot{}

\vfill
\pagebreak

\pars{Capitulum.} \scriptura{Gen. 49, 10}

\grechangedim{interwordspacetext}{0.12 cm plus 0.15 cm minus 0.05 cm}{scalable}%
\cuminitiali{}{temporalia/capitulum-NosAuferetur.gtex}
\grechangedim{interwordspacetext}{0.22 cm plus 0.15 cm minus 0.05 cm}{scalable}%

% preklad Jeruz. bible
%\trCapituliI

\vfill

\pars{Responsorium breve.} \scriptura{Ps. 84, 8; \textbf{H20}}

\cuminitiali{IV}{temporalia/resp-ostendenobis.gtex}

%\trResp

\vfill
\pagebreak

\pars{Hymnus} \scriptura{Ambrosius (\olddag{} 397)}

\cuminitiali{IV}{temporalia/hym-VerbumSalutis.gtex}
\vspace{-3mm}
%\input{hym-VerbumSalutis-bohtext.tex}

\vfill
%\pagebreak

\pars{Versus.} \scriptura{Is. 45, 8}

% Versus. %%%
\sineinitiali{temporalia/versus-rorate.gtex}

%\noindent \trVersus

\vfill
\pagebreak

\magnificat

\vfill
\pagebreak

%\sideThumbs{{\scriptsize{}Fine horarum}}

\anteOrationem

\pagebreak

% Oratio. %%%
\oratioLaudes

\vspace{-1mm}
%\trOrationisI

\vfill

\rubrica{Hebdomadarius dicit iterum Dominus vobiscum, vel cantor dicit:}

\vspace{2mm}

\sineinitiali{temporalia/domineexaudi.gtex}

\rubrica{Postea cantatur a cantore:}

\vspace{2mm}

\cuminitiali{IV}{temporalia/benedicamus-dominica-advequad.gtex}

\vspace{1mm}

\end{document}

