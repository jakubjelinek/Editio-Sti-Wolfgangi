\newcommand{\oratio}{\pars{Oratio.}

\noindent Omnípotens sempitérne Deus, nativitátem Fílii tui secúndum carnem propinquáre cernéntes, quǽsumus, ut nobis indígnis fámulis tuis misericórdiam præstet Verbum, quod ex Vírgine María dignátum est caro fíeri et habitáre in nobis.

\pars{Pro commemoratione Sancti Ioannis de Kęty, Presbyteri.} \scriptura{Io. 13, 35; \textbf{H185}}

\vspace{-4mm}

\antiphona{VII c}{temporalia/ant-inhoccognoscentomnes.gtex}

\vfill

\noindent Da, quǽsumus, omnípotens Deus, ut, exémplo sancti Ioánnis presbýteri, in sanctórum sciéntia procedámus atque, misericórdiam ómnibus exhibéntes, apud te indulgéntiam consequámur.

\pars{Pro pace in universo mundo.} \scriptura{Sir. 50, 25; 2 Esdr. 4, 20; \textbf{H416}}

\vspace{-4mm}

\antiphona{II D}{temporalia/ant-dapacemdomine.gtex}

\vfill

\noindent Deus, a quo sancta desidéria, recta consília et iusta sunt ópera: da servis tuis illam, quam mundus dare non potest, pacem; ut et corda nostra mandátis tuis dédita, et hóstium subláta formídine, témpora sint tua protectióne tranquílla.

\noindent Per Dóminum nostrum Iesum Christum, Fílium tuum, qui tecum vivit et regnat in unitáte Spíritus Sancti, Deus, per ómnia sǽcula sæculórum.

\noindent \Rbardot{} Amen.}
\newcommand{\matversus}{\noindent \Vbardot{} Véniat super me misericórdia tua, Dómine.

\noindent \Rbardot{} Salutáre tuum, secúndum elóquium tuum.}
\newcommand{\lectioi}{\pars{Lectio I.} \scriptura{Is. 51, 1-11}

\noindent De libro Isaíæ prophétæ.

\noindent Audíte me, qui sequímini iustítiam, qui quǽritis Dóminum; atténdite ad petram, unde excísi estis, et ad cavérnam laci, de qua præcísi estis. Atténdite ad Abraham patrem vestrum et ad Saram, quæ péperit vos; quia unum vocávi eum et benedíxi ei et multiplicávi eum. Consolátur enim Dóminus Sion, consolátur omnes ruínas eius; et ponit desértum eius quasi Eden et solitúdinem eius quasi hortum Dómini. Gáudium et lætítia inveniétur in ea, gratiárum áctio et vox laudis.

\noindent Atténdite ad me, pópule meus, et, natiónes, me audíte, quia lex a me éxiet, et iudícium meum in lucem populórum státuam. Prope est iustítia mea, egréssa est salus mea, et bráchia mea pópulos iudicábunt; in me ínsulæ sperábunt et ad bráchium meum atténdent.

\noindent Leváte in cælum óculos vestros et inspícite in terram deórsum, quia cæli sicut fumus liquéscent, et terra sicut vestiméntum atterétur, et habitatóres eius sicut hæc interíbunt. Salus autem mea in sempitérnum erit, et iustítia mea non defíciet.

\noindent Audíte me, qui scitis iustítiam, pópule, in cuius corde est lex mea: nolíte timére oppróbrium hóminum et blasphémias eórum ne metuátis. Sicut enim vestiméntum sic cómedet eos vermis, et sicut lanam sic devorábit eos tínea; iustítia autem mea in sempitérnum erit et salus mea in generatiónes generatiónum.

\noindent Consúrge, consúrge, indúere fortitúdinem, bráchium Dómini; consúrge sicut in diébus antíquis, in generatiónibus sæculórum. Numquid non tu percussísti Rahab, vulnerásti dracónem? Numquid non tu siccásti mare, aquam abýssi veheméntis, qui posuísti profúndum maris viam, ut transírent liberáti?

\noindent Et redémpti a Dómino reverténtur et vénient in Sion laudántes; et lætítia sempitérna super cápita eórum, gáudium et lætítiam obtinébunt; fúgiet dolor et gémitus.}
\newcommand{\responsoriumi}{\pars{Responsorium 1.} \scriptura{\Rbar{} Is. 62, 2 \Vbar{} ibid. 62, 3; \textbf{H35}}

\vspace{-5mm}

\responsorium{IV}{temporalia/resp-videbuntgentesiustumtuum-CROCHU.gtex}{}}
\newcommand{\lectioii}{\pars{Lectio II.} \scriptura{Cap. 9-12: PG 10, 815-819}

\noindent Ex Tractátu sancti Hippólyti presbýteri \emph{Contra hǽresim Noéti}.

\noindent Unus Deus est, quem non aliúnde, fratres, agnóscimus, quam ex sanctis Scriptúris. Quæcúmque ergo divínæ Scriptúræ prǽdicant, sciámus; et quæcúmque docent, cognoscámus: et sicut Pater vult credi, sic credámus; et sicut vult Fílium glorificári, sic glorificémus; et sicut vult donári Spíritum Sanctum, sic accipiámus. Non secúndum própriam voluntátem, neque secúndum próprium sensum, neque vim inferéntes iis quæ a Deo data sunt; sed quómodo ipse per sanctas Scriptúras docére vóluit, sic intellegámus.

\noindent Deus, solus cum esset, nihílque sibi coǽvum habéret, vóluit mundum effícere. Et mundum cógitans ac volens et dicens effécit; continuóque éxstitit ei factus sicut vóluit, et sicut vóluit, perfécit. Satis ígitur nobis est scire solum, nihil esse Deo coǽvum. Nihil erat præter ipsum; ipse solus, multus erat. Nec enim erat sine ratióne, sine sapiéntia, sine poténtia, sine consílio. Omnia erant in eo: ipse erat ómnia. Quando vóluit, et quómodo vóluit, osténdit Verbum suum tempóribus apud eum definítis; per quod ómnia fecit.}
\newcommand{\responsoriumii}{\pars{Responsorium 2.} \scriptura{\Rbar{} Cantor \Vbar{} Lc. 1, 28; \textbf{H36}}

\vspace{-5mm}

\responsorium{I}{temporalia/resp-annuntiatumest-CROCHU.gtex}{}}
\newcommand{\lectioiii}{\pars{Lectio III.}

\noindent Quod Verbum cum in se habéret, essétque mundo creáto inaspectábile, fecit aspectábile, emíttens priórem vocem; et lumen ex lúmine génerans, deprómpsit ipsi creatúræ Dóminum, sensum suum; et qui prius ipsi tantum erat visíbilis, mundo autem invisíbilis, hunc visíbilem facit, ut mundus, cum eum qui appáruit vidéret, salvus fíeri posset.

\noindent Hoc vero mens est, quod pródiens in mundum, osténsum est puer Dei. Omnia ígitur per ipsum, ipse autem solus ex Patre.

\noindent Hic autem dedit legem et prophétas; et dando coégit hos per Spíritum Sanctum loqui, ut accipiéntes virtútis patérnæ inspiratiónem, consílium et voluntátem Patris nuntiárent.

\noindent Factum est ígitur maniféstum Verbum, sicut ait beátus Ioánnes. Répetit enim summátim quæ a prophétis dicta sunt, demónstrans hoc esse Verbum per quod ómnia facta sunt. Sic enim ait: \emph{In princípio erat Verbum et Verbum erat apud Deum et Deus erat Verbum. Omnia per ipsum facta sunt et sine ipso factum est nihil.} Et infra ait: \emph{Mundus per ipsum factus est et mundus eum non cognóvit. In própria venit et sui eum non recepérunt.}}
\newcommand{\responsoriumiii}{\pars{Responsorium 3.} \scriptura{\Rbar{} Is. 9, 6 \Vbar{} Cf. Mal. 3, 1; \textbf{H32}}

\vspace{-5mm}

\responsorium{VII}{temporalia/resp-nasceturnobisparvulus-CROCHU-cumdox.gtex}{}}
\newcommand{\laudes}{\pars{Psalmus 1.} \scriptura{\textbf{H36}}

\vspace{-4mm}

\antiphona{II* b}{temporalia/ant-eccevenietdominus.gtex}

%\vspace{-2mm}

\scriptura{Psalmus 89}

%\vspace{-2mm}

\initiumpsalmi{temporalia/ps89-initium-ii_-B-auto.gtex}

%\vspace{-1.5mm}

\input{temporalia/ps89-ii_-B.tex}

\vfill

\antiphona{}{temporalia/ant-eccevenietdominus.gtex}

\vfill
\pagebreak

\pars{Psalmus 2.} \scriptura{Is. 42, 10; \textbf{H25}}

\vspace{-4mm}

\antiphona{VII b}{temporalia/ant-cantatedomino.gtex}

%\vspace{-2mm}

\scriptura{Canticum Isaiæ, Is. 42, 10-16}

%\vspace{-3mm}

\initiumpsalmi{temporalia/isaiae10-initium-vii-b-auto.gtex}

\input{temporalia/isaiae10-vii-b.tex} \Abardot{}

\vfill
\pagebreak

\pars{Psalmus 3.} \scriptura{Lc. 18, 8; \textbf{H36}}

\vspace{-6mm}

\antiphona{VIII C\textsuperscript{2}}{temporalia/ant-dumvenerit.gtex}

\scriptura{Psalmus 134, 1-12}

\initiumpsalmi{temporalia/ps134i-initium-viii-C2-auto.gtex}

\input{temporalia/ps134i-viii-C2.tex} \Abardot{}

\vfill
\pagebreak}
\newcommand{\lectiobrevis}{\pars{Lectio Brevis.} \scriptura{Ier. 30, 21.22}

\noindent Hæc dicit Dóminus: Erit dux eius ex Iacob, et princeps de médio eius procédet; et applicábo eum, et accédet ad me. Et éritis mihi in pópulum, et ego ero vobis in Deum.}
\newcommand{\benedictus}{\pars{Canticum Zachariæ.} \scriptura{\textbf{H81}}

\vspace{-4mm}

{
\grechangedim{interwordspacetext}{0.18 cm plus 0.15 cm minus 0.05 cm}{scalable}%
\antiphona{VIII C}{temporalia/ant-eccecompletasunt.gtex}
\grechangedim{interwordspacetext}{0.22 cm plus 0.15 cm minus 0.05 cm}{scalable}%
}

%\vspace{-2mm}

\scriptura{Lc. 1, 68-79}

%\vspace{-2mm}

\cantusSineNeumas
\initiumpsalmi{temporalia/benedictus-initium-viii-c-auto.gtex}

%\vspace{-1.5mm}

\input{temporalia/benedictus-viii-c.tex} \Abardot{}}
\newcommand{\preces}{\noindent Deum Patrem, fratres caríssimi, implorántes,~\gredagger{} qui misit Fílium suum ad salvándos hómines,~\grestar{} súpplices acclamémus:

\Rbardot{} Osténde nobis, Dómine, misericórdiam tuam.

\noindent Christum tuum, Pater clementíssime,~\gredagger{} quem plena fide os nostrum annúntiat,~\grestar{} conversátio nostra ópere ne despíciat.

\Rbardot{} Osténde nobis, Dómine, misericórdiam tuam.

\noindent Qui Fílium tuum misísti ad salútem,~\grestar{} univérsum aufer a fácie terræ et a civitáte ista dolórem.

\Rbardot{} Osténde nobis, Dómine, misericórdiam tuam.

\noindent Terra nostra advéntu Fílii tui iucunditáte perfúsa,~\grestar{} tuæ plenitúdinis gáudium ubérius experiátur.

\Rbardot{} Osténde nobis, Dómine, misericórdiam tuam.

\noindent Per misericórdiam tuam, fac ut nos pie et sóbrie in hoc sǽculo vivámus,~\gredagger{} exspectántes beátam spem~\grestar{} et advéntum glóriæ Christi.

\Rbardot{} Osténde nobis, Dómine, misericórdiam tuam.}
\newcommand{\vesperas}{\pars{Psalmus 1.} \scriptura{\textbf{H36}}

\vspace{-6mm}

\antiphona{II* b}{temporalia/ant-eccevenietdominus.gtex}

\vspace{-4mm}

\scriptura{Psalmus 113.}

\initiumpsalmi{temporalia/ps113-initium-ii_-B-auto.gtex}

\vspace{-1.5mm}

\input{temporalia/ps113-ii_-B.tex}

\vfill

\vspace{-6mm}

\antiphona{}{temporalia/ant-eccevenietdominus.gtex}

\vspace{-1cm}

\vfill
\pagebreak

\pars{Psalmus 2.} \scriptura{Lc. 18, 8; \textbf{H36}}

\vspace{-4mm}

\antiphona{VIII C}{temporalia/ant-dumvenerit.gtex}

\vspace{-4mm}

\scriptura{Psalmus 114.}

\initiumpsalmi{temporalia/ps114-initium-viii-C-auto.gtex}

\input{temporalia/ps114-viii-C.tex} \Abardot{}

\vfill
\pagebreak

\pars{Psalmus 3.} \scriptura{Is. 25, 9; \textbf{H34}}

\vspace{-4mm}

\antiphona{VIII G}{temporalia/ant-eccedeusnoster.gtex}

\vspace{-4mm}

\scriptura{Psalmus 115.}

\initiumpsalmi{temporalia/ps115-initium-viii-G-auto.gtex}

\input{temporalia/ps115-viii-G-sinedox.tex}

\rubrica{Hic non dicitur Gloria Patri.}

\vfill

\scriptura{Psalmus 116.}

\initiumpsalmi{temporalia/ps116-initium-viii-G-auto.gtex}

\input{temporalia/ps116-viii-G.tex} \Abardot{}

\vfill
\pagebreak

\pars{Psalmus 4.} \scriptura{\textbf{H37}}

\vspace{-4mm}

\antiphona{II* a}{temporalia/ant-egredieturdominus.gtex}

\vspace{-4mm}

\scriptura{Psalmus 128.}

\initiumpsalmi{temporalia/ps128-initium-ii_-a.gtex}

\input{temporalia/ps128-ii_-a.tex} \Abardot{}

\vfill
\pagebreak}
\newcommand{\magnificat}{\pars{Canticum B. Mariæ V.} \scriptura{Is. 7, 14; 33, 22; Gn. 49, 10; \textbf{H41}}

\vspace{-6.5mm}

{
\grechangedim{interwordspacetext}{0.18 cm plus 0.15 cm minus 0.05 cm}{scalable}%
\antiphona{II D}{temporalia/ant-oemmanuel.gtex}
\grechangedim{interwordspacetext}{0.22 cm plus 0.15 cm minus 0.05 cm}{scalable}%
}

\vspace{-3mm}

\scriptura{Lc. 1, 46-55}

\vspace{-2mm}

\cantusSineNeumas

\initiumpsalmi{temporalia/magnificat-initium-iisoll-D.gtex}

\vspace{-1.5mm}

\input{temporalia/magnificat-iisoll-D.tex} \Abardot{}

\vspace{-1cm}}
\newcommand{\hebdomada}{infra Hebdom. IV post Pentecosten.}
\newcommand{\oratioLaudes}{\cuminitiali{}{temporalia/oratio4.gtex}}

\renewcommand{\hebdomada}{infra Hebdom. Ultima Adventus.}
\ifx\invitatorium\undefined
\newcommand{\invitatorium}{\pars{Invitatorium.} \scriptura{Phil. 4, 4.5}

\vspace{-6mm}

\antiphona{VI}{temporalia/inv-propeestiamsimplex.gtex}}
\fi
\ifx\hymnusmatutinum\undefined
\newcommand{\hymnusmatutinum}{\pars{Hymnus.}

\vspace{-5mm}

\antiphona{II}{temporalia/hym-VeniRedemptor.gtex}}
\fi
\ifx\hymnuslaudes\undefined
\newcommand{\hymnuslaudes}{\pars{Hymnus}

\cuminitiali{D}{temporalia/hym-MagnisProphetae.gtex}}
\fi
\ifx\hymnusvesperas\undefined
\newcommand{\hymnusvesperas}{\pars{Hymnus}

\cuminitiali{IV}{temporalia/hym-VerbumSalutis.gtex}}
\fi

% LuaLaTeX

\documentclass[a4paper, twoside, 12pt]{article}
\usepackage[latin]{babel}
%\usepackage[landscape, left=3cm, right=1.5cm, top=2cm, bottom=1cm]{geometry} % okraje stranky
%\usepackage[landscape, a4paper, mag=1166, truedimen, left=2cm, right=1.5cm, top=1.6cm, bottom=0.95cm]{geometry} % okraje stranky
\usepackage[landscape, a4paper, mag=1400, truedimen, left=0.5cm, right=0.5cm, top=0.5cm, bottom=0.5cm]{geometry} % okraje stranky

\usepackage{fontspec}
\setmainfont[FeatureFile={junicode.fea}, Ligatures={Common, TeX}, RawFeature=+fixi]{Junicode}
%\setmainfont{Junicode}

% shortcut for Junicode without ligatures (for the Czech texts)
\newfontfamily\nlfont[FeatureFile={junicode.fea}, Ligatures={Common, TeX}, RawFeature=+fixi]{Junicode}

\usepackage{multicol}
\usepackage{color}
\usepackage{lettrine}
\usepackage{fancyhdr}

% usual packages loading:
\usepackage{luatextra}
\usepackage{graphicx} % support the \includegraphics command and options
\usepackage{gregoriotex} % for gregorio score inclusion
\usepackage{gregoriosyms}
\usepackage{wrapfig} % figures wrapped by the text
\usepackage{parcolumns}
\usepackage[contents={},opacity=1,scale=1,color=black]{background}
\usepackage{tikzpagenodes}
\usepackage{calc}
\usepackage{longtable}
\usetikzlibrary{calc}

\setlength{\headheight}{14.5pt}

\input{conventuscommune.tex} % Often used macros

\newcommand{\annusEditionis}{2021}

%%%% Vicekrat opakovane kousky

\newcommand{\anteOrationem}{
  \rubrica{Ante Orationem, cantatur a Superiore:}

  \pars{Supplicatio Litaniæ.}

  \cuminitiali{}{temporalia/supplicatiolitaniae.gtex}

  \pars{Oratio Dominica.}

  \cuminitiali{}{temporalia/oratiodominica.gtex}

  \rubrica{Deinde dicitur ab Hebdomadario:}

  \cuminitiali{}{temporalia/dominusvobiscum-solemnis.gtex}

  \rubrica{In choro monialium loco Dominus vobiscum dicitur:}

  \sineinitiali{temporalia/domineexaudi.gtex}
}

\setlength{\columnsep}{30pt} % prostor mezi sloupci

%%%%%%%%%%%%%%%%%%%%%%%%%%%%%%%%%%%%%%%%%%%%%%%%%%%%%%%%%%%%%%%%%%%%%%%%%%%%%%%%%%%%%%%%%%%%%%%%%%%%%%%%%%%%%
\begin{document}

% Here we set the space around the initial.
% Please report to http://home.gna.org/gregorio/gregoriotex/details for more details and options
\grechangedim{afterinitialshift}{2.2mm}{scalable}
\grechangedim{beforeinitialshift}{2.2mm}{scalable}
\grechangedim{interwordspacetext}{0.22 cm plus 0.15 cm minus 0.05 cm}{scalable}%
\grechangedim{annotationraise}{-0.2cm}{scalable}

% Here we set the initial font. Change 38 if you want a bigger initial.
% Emit the initials in red.
\grechangestyle{initial}{\color{red}\fontsize{38}{38}\selectfont}

\pagestyle{empty}

%%%% Titulni stranka
\begin{titulusOfficii}
\ifx\titulus\undefined
\nomenFesti{Feria II \hebdomada{}}
\else
\titulus
\fi
\end{titulusOfficii}

\vfill

\begin{center}
%Ad usum et secundum consuetudines chori \guillemotright{}Conventus Choralis\guillemotleft.

%Editio Sancti Wolfgangi \annusEditionis
\end{center}

\scriptura{}

\pars{}

\pagebreak

\renewcommand{\headrulewidth}{0pt} % no horiz. rule at the header
\fancyhf{}
\pagestyle{fancy}

\cantusSineNeumas

\ifx\oratio\undefined
\ifx\laudb\undefined
\else
\newcommand{\oratio}{\pars{Oratio.}

\noindent Dómine Deus omnípotens, qui ad princípium huius diéi nos perveníre fecísti, tua nos hódie salva virtúte, ut in hac die ad nullum declinémus peccátum, sed semper ad tuam iustítiam faciéndam nostra procédant elóquia, dirigántur cogitatiónes et ópera.

\noindent Per Dóminum nostrum Iesum Christum, Fílium tuum, qui tecum vivit et regnat in unitáte Spíritus Sancti, Deus, per ómnia sǽcula sæculórum.

\noindent \Rbardot{} Amen.}
\fi
\fi

\hora{Ad Matutinum.} %%%%%%%%%%%%%%%%%%%%%%%%%%%%%%%%%%%%%%%%%%%%%%%%%%%%%
%\sideThumbs{Matutinum}

\vspace{2mm}

\cuminitiali{}{temporalia/dominelabiamea.gtex}

\vfill
%\pagebreak

\vspace{2mm}

\ifx\invitatorium\undefined
\pars{Invitatorium.} \scriptura{Ps. 94, 1; Psalmus 94; \textbf{H451}}

\vspace{-6mm}

\antiphona{VI}{temporalia/inv-jubilemusdeo.gtex}\else
\invitatorium
\fi

\vfill
\pagebreak

\ifx\hymnusmatutinum\undefined
\ifx\matua\undefined
\else
\pars{Hymnus.}

{
\grechangedim{interwordspacetext}{0.10 cm plus 0.15 cm minus 0.05 cm}{scalable}%
\antiphona{II}{temporalia/hym-IpsumNunc.gtex}
\grechangedim{interwordspacetext}{0.22 cm plus 0.15 cm minus 0.05 cm}{scalable}%
}
\fi
\else
\hymnusmatutinum
\fi

\vspace{-3mm}

\vfill
\pagebreak

\ifx\matub\undefined
\else
% MAT B
\pars{Psalmus 1.} \scriptura{Ps. 30, 2; \textbf{H90}}

\vspace{-4mm}

\antiphona{VIII G}{temporalia/ant-intuaiustitia.gtex}

%\vspace{-2mm}

\scriptura{Ps. 30, 2-9}

%\vspace{-2mm}

\initiumpsalmi{temporalia/ps30i-initium-viii-G-auto.gtex}

\vspace{-1.5mm}

\input{temporalia/ps30i-viii-G.tex} \Abardot{}

\vfill
\pagebreak

\pars{Psalmus 2.} \scriptura{Ps. 66, 2}

\vspace{-4mm}

\antiphona{E}{temporalia/ant-illuminadomine.gtex}

%\vspace{-2mm}

\scriptura{Ps. 30, 10-17}

%\vspace{-2mm}

\initiumpsalmi{temporalia/ps30ii-initium-e-a-auto.gtex}

\input{temporalia/ps30ii-e-a.tex} \Abardot{}

\vfill
\pagebreak

\pars{Psalmus 3.} \scriptura{Ps. 30, 24}

\vspace{-4mm}

\antiphona{II D}{temporalia/ant-diligitedominum.gtex}

%\vspace{-5mm}

\scriptura{Ps. 30, 20-25}

%\vspace{-2mm}

\initiumpsalmi{temporalia/ps30iii-initium-ii-D-auto.gtex}

\input{temporalia/ps30iii-ii-D.tex} \Abardot{}

\vfill
\pagebreak
\fi

\pars{Versus.}

\ifx\matversus\undefined
\ifx\matub\undefined
\else
\noindent \Vbardot{} Dírige me, Dómine, in veritáte tua, et doce me.

\noindent \Rbardot{} Quia tu es Deus salútis meæ.
\fi
\else
\matversus
\fi

\vspace{5mm}

\sineinitiali{temporalia/oratiodominica-mat.gtex}

\vspace{5mm}

\pars{Absolutio.}

\cuminitiali{}{temporalia/absolutio-exaudi.gtex}

\vfill
\pagebreak

\cuminitiali{}{temporalia/benedictio-solemn-benedictione.gtex}

\vspace{7mm}

\lectioi

\noindent \Vbardot{} Tu autem, Dómine, miserére nobis.
\noindent \Rbardot{} Deo grátias.

\vfill
\pagebreak

\responsoriumi

\vfill
\pagebreak

\cuminitiali{}{temporalia/benedictio-solemn-unigenitus.gtex}

\vspace{7mm}

\lectioii

\noindent \Vbardot{} Tu autem, Dómine, miserére nobis.
\noindent \Rbardot{} Deo grátias.

\vfill
\pagebreak

\responsoriumii

\vfill
\pagebreak

\cuminitiali{}{temporalia/benedictio-solemn-spiritus.gtex}

\vspace{7mm}

\lectioiii

\noindent \Vbardot{} Tu autem, Dómine, miserére nobis.
\noindent \Rbardot{} Deo grátias.

\vfill
\pagebreak

\responsoriumiii

\vfill
\pagebreak

\rubrica{Reliqua omittuntur, nisi Laudes separandæ sint.}

\sineinitiali{temporalia/domineexaudi.gtex}

\vfill

\oratio

\vfill

\noindent \Vbardot{} Dómine, exáudi oratiónem meam.
\Rbardot{} Et clamor meus ad te véniat.

\vfill

\noindent \Vbardot{} Benedicámus Dómino.
\noindent \Rbardot{} Deo grátias.

\vfill

\noindent \Vbardot{} Fidélium ánimæ per misericórdiam Dei requiéscant in pace.
\Rbardot{} Amen.

\vfill
\pagebreak

\hora{Ad Laudes.} %%%%%%%%%%%%%%%%%%%%%%%%%%%%%%%%%%%%%%%%%%%%%%%%%%%%%
%\sideThumbs{Laudes}

\cantusSineNeumas

\vspace{0.5cm}
\grechangedim{interwordspacetext}{0.18 cm plus 0.15 cm minus 0.05 cm}{scalable}%
\cuminitiali{}{temporalia/deusinadiutorium-communis.gtex}
\grechangedim{interwordspacetext}{0.22 cm plus 0.15 cm minus 0.05 cm}{scalable}%

\vfill
\pagebreak

\ifx\hymnuslaudes\undefined
\ifx\laudbd\undefined
\else
\pars{Hymnus} \scriptura{Hilarius (\olddag{} 367)}

\grechangedim{interwordspacetext}{0.16 cm plus 0.15 cm minus 0.05 cm}{scalable}%
\cuminitiali{IV}{temporalia/hym-LucisLargitor.gtex}
\grechangedim{interwordspacetext}{0.22 cm plus 0.15 cm minus 0.05 cm}{scalable}%
\vspace{-3mm}
\fi
\else
\hymnuslaudes
\fi

\vfill
\pagebreak

\ifx\laudb\undefined
\else
\pars{Psalmus 1.} \scriptura{Ps. 41, 3; \textbf{H391}}

\vspace{-4mm}

\antiphona{II D}{temporalia/ant-sitivitanima.gtex}

%\vspace{-2mm}

\scriptura{Psalmus 41}

%\vspace{-2mm}

\initiumpsalmi{temporalia/ps41-initium-ii-D-auto.gtex}

%\vspace{-1.5mm}

\input{temporalia/ps41-ii-D.tex}

\vfill

\antiphona{}{temporalia/ant-sitivitanima.gtex}

\vfill
\pagebreak

\pars{Psalmus 2.}

\vspace{-4mm}

\antiphona{III a}{temporalia/ant-ostendenobisdomine.gtex}

%\vspace{-2mm}

\scriptura{Canticum Ecclesiastici, Sir. 36, 1-7.13-16}

%\vspace{-3mm}

\initiumpsalmi{temporalia/ecclesiastici-initium-iii-a-auto.gtex}

\input{temporalia/ecclesiastici-iii-a.tex} \Abardot{}

\vfill
\pagebreak

\pars{Psalmus 3.}

\vspace{-4mm}

\antiphona{II D}{temporalia/ant-operamanuumeius.gtex}

\scriptura{Psalmus 18, 1-7}

\initiumpsalmi{temporalia/ps18i-initium-ii-D-auto.gtex}

\input{temporalia/ps18i-ii-D.tex} \Abardot{}

\vfill
\pagebreak
\fi

\ifx\lectiobrevis\undefined
\ifx\laudb\undefined
\else
\pars{Lectio Brevis.} \scriptura{Ier. 15, 16}

\noindent Invénti sunt sermónes tui, et comédi eos, et factum est mihi verbum tuum in gáudium et in lætítiam cordis mei, quóniam invocátum est nomen tuum super me, Dómine Deus exercítuum.
\fi
\else
\lectiobrevis
\fi

\vfill

\ifx\responsoriumbreve\undefined
\ifx\laudbd\undefined
\else
\pars{Responsorium breve.} \scriptura{Ps. 32, 1.3}

\cuminitiali{VI}{temporalia/resp-exsultateiusti.gtex}
\fi
\else
\responsoriumbreve
\fi

\vfill
\pagebreak

\ifx\benedictus\undefined
\ifx\laudbd\undefined
\else
\pars{Canticum Zachariæ.} \scriptura{Lc. 1, 68; \textbf{H422}}

\vspace{-4mm}

{
\grechangedim{interwordspacetext}{0.18 cm plus 0.15 cm minus 0.05 cm}{scalable}%
\antiphona{IV E}{temporalia/ant-benedictusdominus.gtex}
\grechangedim{interwordspacetext}{0.22 cm plus 0.15 cm minus 0.05 cm}{scalable}%
}

%\vspace{-3mm}

\scriptura{Lc. 1, 68-79}

%\vspace{-2mm}

\cantusSineNeumas
\initiumpsalmi{temporalia/benedictus-initium-iv-E-auto.gtex}

%\vspace{-1.5mm}

\input{temporalia/benedictus-iv-E.tex} \Abardot{}
\fi
\else
\benedictus
\fi

\vspace{-1cm}

\vfill
\pagebreak

%\sideThumbs{{\scriptsize{}Fine horarum}}

\pars{Preces.}

\sineinitiali{}{temporalia/tonusprecum.gtex}

\ifx\preces\undefined
\ifx\laudb\undefined
\else
\noindent Salvátor noster fecit nos regnum et sacerdótium, ut hóstias Deo acceptábiles offerámus. \gredagger{} Grati ígitur eum invocémus:

\Rbardot{} Serva nos in tuo ministério, Dómine.

\noindent Christe, sacérdos ætérne, qui sanctum pópulo tuo sacerdótium concessísti, \gredagger{} concéde, ut spiritáles hóstias Deo acceptábiles iúgiter offerámus.

\Rbardot{} Serva nos in tuo ministério, Dómine.

\noindent Spíritus tui fructus nobis largíre propítius, \gredagger{} patiéntiam, benignitátem et mansuetúdinem.

\Rbardot{} Serva nos in tuo ministério, Dómine.

\noindent Da nobis te amáre, ut te, qui es cáritas, possideámus, \gredagger{} et bene ágere, ut per vitam étiam nostram te laudémus.

\Rbardot{} Serva nos in tuo ministério, Dómine.

\noindent Quæ frátribus nostris sunt utília, nos quǽrere concéde, \gredagger{} ut salútem facílius consequántur.

\Rbardot{} Serva nos in tuo ministério, Dómine.
\fi
\else
\preces
\fi

\vfill

\pars{Oratio Dominica.}

\cuminitiali{}{temporalia/oratiodominicaalt.gtex}

\vfill
\pagebreak

\rubrica{vel:}

\pars{Supplicatio Litaniæ.}

\cuminitiali{}{temporalia/supplicatiolitaniae.gtex}

\vfill

\pars{Oratio Dominica.}

\cuminitiali{}{temporalia/oratiodominica.gtex}

\vfill
\pagebreak

% Oratio. %%%
\oratio

\vspace{-1mm}

\vfill

\rubrica{Hebdomadarius dicit Dominus vobiscum, vel, absente sacerdote vel diacono, sic concluditur:}

\vspace{2mm}

\antiphona{C}{temporalia/dominusnosbenedicat.gtex}

\rubrica{Postea cantatur a cantore:}

\vspace{2mm}

\cuminitiali{IV}{temporalia/benedicamus-feria-laudes.gtex}

\vspace{1mm}

\vfill
\pagebreak

\end{document}

