\newcommand{\lectioi}{\pars{Lectio I.} \scriptura{Is. 30, 27-33; 31, 4-6}

\noindent De libro Isaíæ prophétæ.

\noindent Ecce nomen Dómini venit de longínquo, ardens furor eius et gravis eius fragor;

\noindent lábia eius repléta sunt indignatióne et lingua eius quasi ignis dévorans.

\noindent Spíritus eius velut torrens inúndans, usque ad collum pertíngens,

\noindent ad cribrándas gentes in cribro funésto, et frenum dolósum in maxíllis populórum.

\noindent Cánticum erit vobis sicut nox sanctificátæ sollemnitátis

\noindent et lætítia cordis sicut eius, qui ad sonum tíbiæ pergit

\noindent Hæc dicit Dóminus de uncto suo Cyro: 

\noindent «Apprehéndi déxteram eius, ut subíciam ante fáciem eius gentes et dorsa regum vertam et apériam coram eo iánuas; et portæ non claudéntur.

\noindent Ego ante te ibo et montes humiliábo; portas ǽreas cónteram et vectes férreos confríngam. 

\noindent Et dabo tibi thesáuros abscónditos et divítias occúltas, ut scias quia ego Dóminus, qui vocávi te nómine tuo, Deus Israel. 

\noindent Propter servum meum Iacob et Israel eléctum meum, et vocávi te nómine tuo; designávi te, et non cognovísti me.

\noindent Ego Dóminus, et non est ámplius: extra me non est Deus. 

\noindent Accínxi te, et non cognovísti me, ut sciant ab ortu solis et ab occidénte quóniam absque me nullus est.

\noindent in túrbine et in imbre et in lápide grándinis.

\noindent A voce enim Dómini pavébit Assýrius virga percússus.

\noindent Væ, qui contradícit fictóri suo, testa de vasis fictílibus terræ! 

\noindent Numquid dicet lutum fígulo suo: «Quid facis?» et «Opus tuum absque mánibus est?». 

\noindent Væ, qui dicit patri: «Quid géneras?» et mulíeri: «Quid párturis?».

\noindent Hæc dicit Dóminus, Sanctus Israel, plastes eius: 

\noindent «Numquid ventúra interrogátis me super fílios meos et super opus mánuum meárum mandátis mihi?

\noindent et in bellis agitátis expugnábit eos.

\noindent Ego suscitávi eum in iustítia et omnes vias eius dírigam; 

\noindent ipse ædificábit civitátem meam et captivitátem meam dimíttet non in prétio neque in munéribus», dicit Dóminus exercítuum.}
\newcommand{\responsoriumi}{\pars{Responsorium 1.} \scriptura{\Rbar{} Is. 45, 8 \Vbar{} Is. 16, 1; \textbf{H35}}

\noindent præparáta, profúnda et dilatáta, in pyra eius ignis et ligna multa;

\noindent flatus Dómini sicut torrens súlphuris succéndit eam.

\noindent Quia hæc dicit Dóminus ad me:

\noindent Nihil prodest Dóminum nostrum, Beátæ Maríæ Vírginis fílium, verum perfectúmque hóminem dícere, si non illíus géneris homo créditur, cuius in Evangélio prædicátur. 

\noindent Dicit enim Matthǽus: \emph{Liber generatiónis Iesu Christi fílii David, fílii Abraham;} 

\noindent et ita humánæ oríginis órdinem séquitur, ut generatiónum líneas usque ad Ioseph, cui mater Dómini erat desponsáta, dedúcat. 

\noindent Lucas vero, retrórsum successiónum gradus rélegens, ad ipsum humáni géneris príncipem redit, ut Adam primum et Adam novíssimum eiúsdem osténdat esse natúræ. 

\noindent Potúerat quippe omnipoténtia Fílii Dei sic ad docéndos iustificandósque hómines apparére, 

\noindent quómodo et patriárchis et prophétis in spécie carnis appáruit, 

\noindent cum aut luctámen íniit, aut sermónem conséruit, cumve offícia hospitalitátis non ábnuit, vel étiam appósitum cibum sumpsit. 

\noindent Sed illæ imágines huius hóminis erant índices, cuius veritátem ex præcedéntium patrum stirpe suméndam significatiónes mýsticæ nuntiábant.}
\newcommand{\responsoriumii}{\pars{Responsorium 2.} \scriptura{\Rbardot{} Num. 24, 17 \Vbardot{} Ps. 71, 10; \textbf{H34}}

\noindent cum occúrrerit ei multitúdo pastórum, a voce eórum non formidábit et a multitúdine eórum non pavébit,

\noindent sic descéndet Dóminus exercítuum, ut prœliétur super montem Sion et super collem eius.

\noindent Et ídeo sacraméntum reconciliatiónis nostræ, ante témpora ætérna dispósitum, nullæ implébant figúræ, 

\noindent quia nondum \emph{supervénerat Spíritus Sanctus} in Vírginem, \emph{nec virtus Altíssimi obumbráverat ei}; 

\noindent ut intra intemeráta víscera, \emph{ædificánte sibi Sapiéntia domum,} Verbum caro fíeret; 

\noindent et, forma Dei ac forma servi in unam conveniénte persónam, Creátor témporum nascerétur in témpore; 

\noindent et, \emph{per quem facta sunt ómnia,} ipse inter ómnia gignerétur.

\noindent Nisi enim novus homo, factus \emph{in similitúdinem carnis peccáti,} nostram suscíperet vetustátem, 

\noindent et, consubstantiális Patri, consubstantiális esse dignarétur et matri, 

\noindent naturámque sibi nostram solus a peccáto liber uníret, sub iugo diáboli generáliter tenerétur humána captívitas; 

\noindent nec uti possémus triumphántis victória, si extra nostram esset consérta natúram.

\noindent De hac autem participatióne mirábili sacraméntum nobis regeneratiónis illúxit, 

\noindent ut per ipsum Spíritum, per quem Christum et concéptus est et natus, étiam nos spiritáli rursus orígine nascerémur. 

\noindent Propter quod ab Evangelísta de credéntibus dícitur: 

\noindent \emph{Qui non ex sanguínibus, neque ex voluntáte carnis, neque ex voluntáte viri, sed ex Deo nati sunt.}}
\newcommand{\responsoriumiii}{\pars{Responsorium 3.} \scriptura{\Rbardot{} Zach. 14, 3.4 \Vbardot{} Ps. 49, 2-3; \textbf{H34}}

\vspace{-5mm}

\responsorium{I}{temporalia/resp-ecceapparebit-CROCHU.gtex}{}

\rubrica{vel ad libitum:}

\vspace{3mm}

\pars{Responsorium 1.} \scriptura{\Rbar{} Cantor \Vbar{} Is. 40, 10; \textbf{H28}}

\vspace{-5mm}

\responsorium{III}{temporalia/resp-aegyptenoliflere-CROCHU.gtex}{}}
\newcommand{\lectioii}{\pars{Lectio II.} \scriptura{Lib. 2, cap. 2-3}

\noindent E Libro De imitatióne Christi.

\noindent Non magni pendas, quis pro te, vel contra te sit: sed hoc age et cura, ut Deus tecum sit in omni re quam facis.

\noindent Hábeas consciéntiam bonam et Deus bene te defensábit.

\noindent Quem enim Deus adiuváre volúerit, nullíus pervérsitas nocére póterit.

\noindent Si tu scis tacére et pati, vidébis procul dúbio auxílium Dómini.

\noindent Ipse novit tempus et modum liberándi te, et ídeo te debes illi resignáre.

\noindent Dei est adiuváre, et ab omni confusióne liberáre.

\noindent Sæpe valde prodest, ad maiórem humilitátem servándam, quod deféctus nostros álii sciunt et redárguunt.

\noindent Quando homo pro deféctibus suis se humíliat, tunc facíliter álios placat, et léviter satísfacit sibi irascéntibus.

\noindent Húmilem Deus prótegit et líberat, húmilem díligit et consolátur:

\noindent húmili hómini se inclínat, húmili largítur grátiam magnam, et post eius depressiónem, levat ad glóriam.

\noindent Húmili sua secréta revélat, et ad se dúlciter trahit et invítat.

\noindent Húmilis, accépta confusióne, satis bene est in pace: quia stat in Deo, et non in mundo.

\noindent Non réputes te áliquid profecísse, nisi ómnibus inferiórem te esse séntias.}
\newcommand{\responsoriumii}{\pars{Responsorium 2.} \scriptura{\Rbar{} Is. 11, 10 \Vbar{} Cf. Hab. 3, 3; \textbf{H29}}

\vspace{-5mm}

\responsorium{VIII}{temporalia/resp-ecceradixjesse-CROCHU.gtex}{}

\rubrica{vel ad libitum:}

\vspace{3mm}

\pars{Responsorium 2.} \scriptura{\Rbar{} Is. 14, 1 \Vbar{} Hebr. 10, 37; \textbf{H28}}

\vspace{-5mm}

\responsorium{III}{temporalia/resp-propeestutveniat-CROCHU.gtex}{}}
\newcommand{\lectioiii}{\pars{Lectio III.} \scriptura{Is. 4, 2-6}

\noindent Tene te primo in pace, et tunc póteris álios pacificáre.

\noindent Homo pacíficus magis prodest, quam bene doctus.

\noindent Homo passionátus étiam bonum in malum trahit, et facíliter malum credit.

\noindent Qui bene in pace est, de nullo suspicátur.

\noindent Qui autem male conténtus est et commótus, váriis suspiciónibus agitátur: nec ipse quiéscit, nec álios quiéscere permíttit.

\noindent Dicit sæpe quod dícere non debéret; et omíttit quod sibi magis fácere expedíret.

\noindent Consíderat quod álii fácere tenéntur: et néglegit quod ipse tenétur.

\noindent Habe ergo primo zelum super teípsum, et tunc iuste zeláre póteris étiam próximum tuum.

\noindent Tu bene scis facta tua excusáre et coloráre, et aliórum excusatiónes non vis recípere.

\noindent Iústius esset, ut te accusáres, et fratrem tuum excusáres.

\noindent Si portári vis, porta et álium.}
\newcommand{\responsoriumiii}{\pars{Responsorium 3.} \scriptura{\Rbar{} Is. 2, 3 \Vbar{} Ps. 49, 2; \textbf{H29}}

\vspace{-5mm}

\responsorium{II}{temporalia/resp-docebitnos-CROCHU-cumdox.gtex}{}

\rubrica{vel ad libitum:}

\vspace{3mm}

\pars{Responsorium 3.} \scriptura{\Rbar{} Ps. 71, 5-6 \Vbar{} Ps. 106, 3; \textbf{H29}}

\vspace{-5mm}

\responsorium{IV}{temporalia/resp-descendetdominus-CROCHU-cumdox.gtex}{}}
\newcommand{\lectiobrevis}{\pars{Lectio Brevis.} \scriptura{Gen. 49, 10}

\noindent Non auferétur sceptrum de Iuda et báculus ducis de pédibus eius, donec véniat ille, cuius est, et cui erit obœdiéntia géntium.}
\newcommand{\benedictus}{\pars{Canticum Zachariæ.} \scriptura{Is. 51, 17; 52, 2; \textbf{H30}}

\vspace{-5mm}

{
\grechangedim{interwordspacetext}{0.18 cm plus 0.15 cm minus 0.05 cm}{scalable}%
\antiphona{VIII G}{temporalia/ant-elevareelevare.gtex}
\grechangedim{interwordspacetext}{0.22 cm plus 0.15 cm minus 0.05 cm}{scalable}%
}

%\trAntIMagnificat

%\vspace{-3mm}

\scriptura{Lc. 1, 68-79}

%\vspace{-1mm}

\cantusSineNeumas
\initiumpsalmi{temporalia/benedictus-initium-viii-G-auto.gtex}

\input{temporalia/benedictus-viii-G.tex} \Abardot{}}
\newcommand{\preces}{\noindent Deus Pater omnípotens íterum manum suam exténdet ad possidéndum resíduum pópuli sui.~\gredagger{} Proínde eum rogémus:

\Rbardot{} Advéniat regnum tuum, Dómine.

\noindent Concéde, Dómine, ut faciámus fructus dignos pæniténtiæ,~\gredagger{} ad accipiéndum regnum tuum, quod prope est.

\Rbardot{} Advéniat regnum tuum, Dómine.

\noindent Para, Dómine, viam in córdibus nostris Verbo tuo ventúro,~\gredagger{} ut eius glória in nobis revelétur.

\Rbardot{} Advéniat regnum tuum, Dómine.

\noindent Humília montes supérbiæ nostræ,~\gredagger{} exálta valles infirmitátis nostræ.

\Rbardot{} Advéniat regnum tuum, Dómine.

\noindent Murum ódii evérte, natiónes dividéntem,~\gredagger{} et vias concórdiæ fac homínibus planas.

\Rbardot{} Advéniat regnum tuum, Dómine.}
\newcommand{\oratio}{\pars{Oratio.}

\noindent Deus, qui novam creatúram per Unigénitum tuum nos esse fecísti, in ópera misericórdiæ tuæ propítius intuére et in advéntu Fílii tui ab ómnibus nos máculis vetustátis emúnda.

\noindent Per Dóminum nostrum Iesum Christum, Fílium tuum, qui tecum vivit et regnat in unitáte Spíritus Sancti, Deus, per ómnia sǽcula sæculórum.

\noindent \Rbardot{} Amen.}
\newcommand{\magnificat}{\pars{Canticum B. Mariæ V.} \scriptura{Mt. 1, 18; \textbf{H20}}

\vspace{-4mm}

{
\grechangedim{interwordspacetext}{0.18 cm plus 0.15 cm minus 0.05 cm}{scalable}%
\antiphona{I f}{temporalia/ant-antequamconvenirent.gtex}
\grechangedim{interwordspacetext}{0.22 cm plus 0.15 cm minus 0.05 cm}{scalable}%
}

%\trAntIMagnificat

%\vspace{-3mm}

\scriptura{Lc. 1, 46-55}

%\vspace{-2mm}

\cantusSineNeumas

\initiumpsalmi{temporalia/magnificat-initium-i-f.gtex}

%\vspace{-2mm}

\input{temporalia/magnificat-i-f.tex} \Abardot{}

\vspace{-1cm}}
\newcommand{\hebdomada}{infra Hebdom. III Adventus.}
\newcommand{\oratioLaudes}{\cuminitiali{}{temporalia/oratio3vo.gtex}}
\newcommand{\responsoriumbreve}{\pars{Responsorium breve.} \scriptura{Is. 60, 2; \textbf{H20}}

\cuminitiali{IV}{temporalia/resp-superte.gtex}}

% LuaLaTeX

\documentclass[a4paper, twoside, 12pt]{article}
\usepackage[latin]{babel}
%\usepackage[landscape, left=3cm, right=1.5cm, top=2cm, bottom=1cm]{geometry} % okraje stranky
%\usepackage[landscape, a4paper, mag=1166, truedimen, left=2cm, right=1.5cm, top=1.6cm, bottom=0.95cm]{geometry} % okraje stranky
\usepackage[landscape, a4paper, mag=1400, truedimen, left=0.5cm, right=0.5cm, top=0.5cm, bottom=0.5cm]{geometry} % okraje stranky

\usepackage{fontspec}
\setmainfont[FeatureFile={junicode.fea}, Ligatures={Common, TeX}, RawFeature=+fixi]{Junicode}
%\setmainfont{Junicode}

% shortcut for Junicode without ligatures (for the Czech texts)
\newfontfamily\nlfont[FeatureFile={junicode.fea}, Ligatures={Common, TeX}, RawFeature=+fixi]{Junicode}

\usepackage{multicol}
\usepackage{color}
\usepackage{lettrine}
\usepackage{fancyhdr}

% usual packages loading:
\usepackage{luatextra}
\usepackage{graphicx} % support the \includegraphics command and options
\usepackage{gregoriotex} % for gregorio score inclusion
\usepackage{gregoriosyms}
\usepackage{wrapfig} % figures wrapped by the text
\usepackage{parcolumns}
\usepackage[contents={},opacity=1,scale=1,color=black]{background}
\usepackage{tikzpagenodes}
\usepackage{calc}
\usepackage{longtable}
\usetikzlibrary{calc}

\setlength{\headheight}{14.5pt}

\input{conventuscommune.tex} % Often used macros

\newcommand{\annusEditionis}{2021}

%%%% Vicekrat opakovane kousky

\newcommand{\anteOrationem}{
  \rubrica{Ante Orationem, cantatur a Superiore:}

  \pars{Supplicatio Litaniæ.}

  \cuminitiali{}{temporalia/supplicatiolitaniae.gtex}

  \pars{Oratio Dominica.}

  \cuminitiali{}{temporalia/oratiodominica.gtex}

  \rubrica{Deinde dicitur ab Hebdomadario:}

  \cuminitiali{}{temporalia/dominusvobiscum-solemnis.gtex}

  \rubrica{In choro monialium loco Dominus vobiscum dicitur:}

  \sineinitiali{temporalia/domineexaudi.gtex}
}

\setlength{\columnsep}{30pt} % prostor mezi sloupci

%%%%%%%%%%%%%%%%%%%%%%%%%%%%%%%%%%%%%%%%%%%%%%%%%%%%%%%%%%%%%%%%%%%%%%%%%%%%%%%%%%%%%%%%%%%%%%%%%%%%%%%%%%%%%
\begin{document}

% Here we set the space around the initial.
% Please report to http://home.gna.org/gregorio/gregoriotex/details for more details and options
\grechangedim{afterinitialshift}{2.2mm}{scalable}
\grechangedim{beforeinitialshift}{2.2mm}{scalable}
\grechangedim{interwordspacetext}{0.22 cm plus 0.15 cm minus 0.05 cm}{scalable}%
\grechangedim{annotationraise}{-0.2cm}{scalable}

% Here we set the initial font. Change 38 if you want a bigger initial.
% Emit the initials in red.
\grechangestyle{initial}{\color{red}\fontsize{38}{38}\selectfont}

\pagestyle{empty}

%%%% Titulni stranka
\begin{titulusOfficii}
\ifx\titulus\undefined
\nomenFesti{Feria III \hebdomada{}}
\else
\titulus
\fi
\end{titulusOfficii}

\vfill

\begin{center}
%Ad usum et secundum consuetudines chori \guillemotright{}Conventus Choralis\guillemotleft.

%Editio Sancti Wolfgangi \annusEditionis
\end{center}

\scriptura{}

\pars{}

\pagebreak

\renewcommand{\headrulewidth}{0pt} % no horiz. rule at the header
\fancyhf{}
\pagestyle{fancy}

\cantusSineNeumas

\ifx\oratio\undefined
\ifx\laudb\undefined
\else
\newcommand{\oratio}{\pars{Oratio.}

\noindent Dómine Iesu Christe, lux vera, qui omnes hómines illúminas ad salútem, nobis, quǽsumus, concéde virtútem, ut ante te vias pacis et iustítiæ præparémus.

\noindent Qui vivis et regnas cum Deo Patre in unitáte Spíritus Sancti, Deus, per ómnia sǽcula sæculórum.

\noindent \Rbardot{} Amen.}
\fi
\fi

\hora{Ad Matutinum.} %%%%%%%%%%%%%%%%%%%%%%%%%%%%%%%%%%%%%%%%%%%%%%%%%%%%%

\vspace{2mm}

\cuminitiali{}{temporalia/dominelabiamea.gtex}

\vfill
%\pagebreak

\vspace{2mm}

\ifx\invitatorium\undefined
\ifx\matuac\undefined
\else
\pars{Invitatorium.} \scriptura{Ps. 94, 1; Psalmus 94; \textbf{H451}}

\vspace{-6mm}

\antiphona{VI}{temporalia/inv-jubilemusdeo.gtex}
\fi
\ifx\matubd\undefined
\else
\pars{Invitatorium.} \scriptura{Cantor; Psalmus 94; \textbf{H449}}

\vspace{-6mm}

\antiphona{E}{temporalia/inv-regemmagnum.gtex}
\fi
\else
\invitatorium
\fi

\vfill
\pagebreak

\ifx\hymnusmatutinum\undefined
\ifx\matuac\undefined
\else
\pars{Hymnus}

\cuminitiali{IV}{temporalia/hym-SomnoRefectis.gtex}
\fi
\ifx\matubd\undefined
\else
\pars{Hymnus.} \scriptura{Gregorius Magnus (\olddag{} 604)}

{
\grechangedim{interwordspacetext}{0.10 cm plus 0.15 cm minus 0.05 cm}{scalable}%
\antiphona{I}{temporalia/hym-NocteSurgentes.gtex}
\grechangedim{interwordspacetext}{0.22 cm plus 0.15 cm minus 0.05 cm}{scalable}%
}
\fi
\else
\hymnusmatutinum
\fi

\vspace{-3mm}

\vfill
\pagebreak

\ifx\matub\undefined
\else
% MAT B
\pars{Psalmus 1.} \scriptura{Ps. 36, 5; \textbf{H93}}

\vspace{-4mm}

\antiphona{VI F}{temporalia/ant-reveladomino.gtex}

%\vspace{-2mm}

\scriptura{Ps. 36, 1-11}

%\vspace{-2mm}

\initiumpsalmi{temporalia/ps36i_xi-initium-vi-F-auto.gtex}

\input{temporalia/ps36i_xi-vi-F.tex} \Abardot{}

\vfill
\pagebreak

\pars{Psalmus 2.}

\vspace{-4mm}

\antiphona{II D}{temporalia/ant-iuniorfui.gtex}

\vspace{-2mm}

\scriptura{Ps. 36, 12-29}

\vspace{-2mm}

\initiumpsalmi{temporalia/ps36xii_xxix-initium-ii-D-auto.gtex}

\input{temporalia/ps36xii_xxix-ii-D.tex}

\vfill

\antiphona{}{temporalia/ant-iuniorfui.gtex}

\vfill
\pagebreak

\pars{Psalmus 3.} \scriptura{Ps. 36, 3}

\vspace{-4mm}

\antiphona{VI F}{temporalia/ant-speraindomino.gtex}

%\vspace{-2mm}

\scriptura{Ps. 36, 30-40}

%\vspace{-2mm}

\initiumpsalmi{temporalia/ps36iii-initium-vi-F-auto.gtex}

\input{temporalia/ps36iii-vi-F.tex} \Abardot{}

\vfill
\pagebreak
\fi
\ifx\matuc\undefined
\else
% MAT C
\pars{Psalmus 1.} \scriptura{Ps. 67, 2}

\vspace{-4mm}

\antiphona{VII a}{temporalia/ant-exsurgatdeus.gtex}

%\vspace{-2mm}

\scriptura{Ps. 67, 2-11}

\initiumpsalmi{temporalia/ps67i-initium-vii-a-auto.gtex}

\input{temporalia/ps67i-vii-a.tex} \Abardot{}

\vfill
\pagebreak

\pars{Psalmus 2.}

\vspace{-4mm}

\antiphona{I f}{temporalia/ant-deusnosterdeussalvos.gtex}

%\vspace{-2mm}

\scriptura{Ps. 67, 12-24}

%\vspace{-2mm}

\initiumpsalmi{temporalia/ps67ii-initium-i-f-auto.gtex}

\input{temporalia/ps67ii-i-f.tex} \Abardot{}

\vfill
\pagebreak

\pars{Psalmus 3.} \scriptura{Ps. 67, 27; \textbf{H96}}

\vspace{-4mm}

\antiphona{D}{temporalia/ant-inecclesiis.gtex}

%\vspace{-2mm}

\scriptura{Ps. 67, 25-36}

\initiumpsalmi{temporalia/ps67iii-initium-d-g2-auto.gtex}

\input{temporalia/ps67iii-d-g2.tex} \Abardot{}

\vfill
\pagebreak
\fi

\pars{Versus.}

\ifx\matversus\undefined
\ifx\matub\undefined
\else
\noindent \Vbardot{} Bonitátem et prudéntiam et sciéntiam doce me.

\noindent \Rbardot{} Quia præcéptis tuis crédidi.
\fi
\ifx\matuc\undefined
\else
\noindent \Vbardot{} Audiam quid loquátur Dóminus Deus.

\noindent \Rbardot{} Loquétur pacem ad plebem suam.
\fi
\else
\matversus
\fi

\vspace{5mm}

\sineinitiali{temporalia/oratiodominica-mat.gtex}

\vspace{5mm}

\pars{Absolutio.}

\cuminitiali{}{temporalia/absolutio-ipsius.gtex}

\vfill
\pagebreak

\cuminitiali{}{temporalia/benedictio-solemn-deus.gtex}

\vspace{7mm}

\lectioi

\noindent \Vbardot{} Tu autem, Dómine, miserére nobis.
\noindent \Rbardot{} Deo grátias.

\vfill
\pagebreak

\responsoriumi

\vfill
\pagebreak

\cuminitiali{}{temporalia/benedictio-solemn-christus.gtex}

\vspace{7mm}

\lectioii

\noindent \Vbardot{} Tu autem, Dómine, miserére nobis.
\noindent \Rbardot{} Deo grátias.

\vfill
\pagebreak

\responsoriumii

\vfill
\pagebreak

\cuminitiali{}{temporalia/benedictio-solemn-ignem.gtex}

\vspace{7mm}

\lectioiii

\noindent \Vbardot{} Tu autem, Dómine, miserére nobis.
\noindent \Rbardot{} Deo grátias.

\vfill
\pagebreak

\responsoriumiii

\vfill
\pagebreak

\rubrica{Reliqua omittuntur, nisi Laudes separandæ sint.}

\sineinitiali{temporalia/domineexaudi.gtex}

\vfill

\oratio

\vfill

\noindent \Vbardot{} Dómine, exáudi oratiónem meam.
\Rbardot{} Et clamor meus ad te véniat.

\vfill

\noindent \Vbardot{} Benedicámus Dómino.
\noindent \Rbardot{} Deo grátias.

\vfill

\noindent \Vbardot{} Fidélium ánimæ per misericórdiam Dei requiéscant in pace.
\Rbardot{} Amen.

\vfill
\pagebreak

\hora{Ad Laudes.} %%%%%%%%%%%%%%%%%%%%%%%%%%%%%%%%%%%%%%%%%%%%%%%%%%%%%

\cantusSineNeumas

\vspace{0.5cm}
\grechangedim{interwordspacetext}{0.18 cm plus 0.15 cm minus 0.05 cm}{scalable}%
\cuminitiali{}{temporalia/deusinadiutorium-communis.gtex}
\grechangedim{interwordspacetext}{0.22 cm plus 0.15 cm minus 0.05 cm}{scalable}%

\vfill
\pagebreak

\ifx\hymnuslaudes\undefined
\ifx\laudac\undefined
\else
\pars{Hymnus} \scriptura{Ambrosius (\olddag{} 397)}

\cuminitiali{I}{temporalia/hym-SplendorPaternae-hiemalis.gtex}
\fi
\ifx\laudbd\undefined
\else
\pars{Hymnus}

\grechangedim{interwordspacetext}{0.16 cm plus 0.15 cm minus 0.05 cm}{scalable}%
\cuminitiali{IV}{temporalia/hym-AEterneLucis.gtex}
\grechangedim{interwordspacetext}{0.22 cm plus 0.15 cm minus 0.05 cm}{scalable}%
\vspace{-3mm}
\fi
\else
\hymnuslaudes
\fi

\vfill
\pagebreak

\ifx\laudb\undefined
\else
\pars{Psalmus 1.} \scriptura{Ps. 42, 5; \textbf{H95}}

\vspace{-4mm}

\antiphona{VI F}{temporalia/ant-salutarevultusmei.gtex}

\scriptura{Psalmus 42.}

\initiumpsalmi{temporalia/ps42-initium-vi-F-auto.gtex}

\input{temporalia/ps42-vi-F.tex} \Abardot{}

\vfill
\pagebreak

\pars{Psalmus 2.} \scriptura{Is. 38, 20; \textbf{H95}}

\vspace{-7mm}

\antiphona{E}{temporalia/ant-cunctisdiebus.gtex}

\vspace{-4mm}

\scriptura{Canticum Ezechiæ, Is. 38, 10-20}

\vspace{-3mm}

\initiumpsalmi{temporalia/ezechiae-initium-e-auto.gtex}

\input{temporalia/ezechiae-e.tex} \Abardot{}

\vfill
\pagebreak

\pars{Psalmus 3.} \scriptura{Ps. 64, 2; \textbf{H96}}

\vspace{-4mm}

\antiphona{VIII a}{temporalia/ant-tedecet.gtex}

\vspace{-2mm}

\scriptura{Psalmus 64.}

\vspace{-2mm}

\initiumpsalmi{temporalia/ps64-initium-viii-A-auto.gtex}

\input{temporalia/ps64-viii-A.tex} \Abardot{}

\vfill
\pagebreak
\fi
\ifx\laudc\undefined
\else
\pars{Psalmus 1.} \scriptura{Ps. 83, 5}

\vspace{-4mm}

\antiphona{VIII G}{temporalia/ant-beatiquihabitant.gtex}

\vspace{-2mm}

\scriptura{Psalmus 84.}

\vspace{-2mm}

\initiumpsalmi{temporalia/ps84-initium-viii-G-auto.gtex}

\input{temporalia/ps84-viii-G.tex} \Abardot{}

\vfill
\pagebreak

\pars{Psalmus 2.}

\vspace{-4mm}

\antiphona{VII d}{temporalia/ant-denoctespiritusmeus.gtex}

\vspace{-2mm}

\scriptura{Canticum Isaiæ, Is. 26, 1-12}

\vspace{-2mm}

\initiumpsalmi{temporalia/isaiae3-initium-vii-d.gtex}

\input{temporalia/isaiae3-vii-d.tex} \Abardot{}

\vfill
\pagebreak

\pars{Psalmus 3.} \scriptura{Ps. 66, 2}

\vspace{-4mm}

\antiphona{E}{temporalia/ant-illuminadomine.gtex}

%\vspace{-2mm}

\scriptura{Psalmus 66.}

%\vspace{-2mm}

\initiumpsalmi{temporalia/ps66-initium-e.gtex}

\input{temporalia/ps66-e.tex} \Abardot{}

\vfill
\pagebreak
\fi

\ifx\lectiobrevis\undefined
\ifx\laudb\undefined
\else
\pars{Lectio Brevis.} \scriptura{1 Th. 5, 4-5}

\noindent Vos, fratres, non estis in ténebris, ut vos dies ille tamquam fur comprehéndat; omnes enim vos fílii lucis estis et fílii diéi. Non sumus noctis neque tenebrárum.
\fi
\ifx\laudc\undefined
\else
\pars{Lectio Brevis.} \scriptura{1 Io. 4, 14-15}

\noindent Nos vídimus et testificámur quóniam Pater misit Fílium salvatórem mundi. Quisque conféssus fúerit: Iesus est Fílius Dei, Deus in ipso manet, et ipse in Deo.
\fi
\else
\lectiobrevis
\fi

\vfill

\ifx\responsoriumbreve\undefined
\ifx\laudac\undefined
\else
\pars{Responsorium breve.}

\cuminitiali{VI}{temporalia/resp-benedictusdominus.gtex}
\fi
\ifx\laudbd\undefined
\else
\pars{Responsorium breve.} \scriptura{Ps. 118, 149.147}

\cuminitiali{VI}{temporalia/resp-vocemmeamaudi.gtex}
\fi
\else
\responsoriumbreve
\fi

\vfill
\pagebreak

\ifx\benedictus\undefined
\ifx\laudbd\undefined
\else
\pars{Canticum Zachariæ.} \scriptura{Lc. 1, 71; \textbf{H423}}

\vspace{-5mm}

{
\grechangedim{interwordspacetext}{0.18 cm plus 0.15 cm minus 0.05 cm}{scalable}%
\antiphona{I g\textsuperscript{5}}{temporalia/ant-demanuomnium.gtex}
\grechangedim{interwordspacetext}{0.22 cm plus 0.15 cm minus 0.05 cm}{scalable}%
}

%\vspace{-3mm}

\scriptura{Lc. 1, 68-79}

%\vspace{-1mm}

\initiumpsalmi{temporalia/benedictus-initium-i-g5-auto.gtex}

\input{temporalia/benedictus-i-g5.tex} \Abardot{}
\fi
\else
\benedictus
\fi

\vspace{-1cm}

\vfill
\pagebreak

\pars{Preces.}

\sineinitiali{}{temporalia/tonusprecum.gtex}

\ifx\preces\undefined
\ifx\laudb\undefined
\else
\noindent Salvatóri nostro benedicámus, qui sua resurrectióne mundum clarificávit, \gredagger{} et humíliter invocémus eum dicéntes:

\Rbardot{} Salva nos, Dómine, in sémita tua.

\noindent Resurrectiónem tuam, Dómine Iesu, oratióne cólimus matutína, \gredagger{} spes glóriæ tuæ diem nostrum illúminet.

\Rbardot{} Salva nos, Dómine, in sémita tua.

\noindent Súscipe, Dómine, vota et propósita nostra, \gredagger{} tamquam diéi nostri primítias.

\Rbardot{} Salva nos, Dómine, in sémita tua.

\noindent Tríbue in dilectióne tua nos hódie profícere, \gredagger{} ut ómnia in nostrum omniúmque bonum cooperéntur.

\Rbardot{} Salva nos, Dómine, in sémita tua.

\noindent Da, Dómine, sic lucére lucem nostram coram homínibus, \gredagger{} ut vídeant ópera nostra bona et Patrem gloríficent.

\Rbardot{} Salva nos, Dómine, in sémita tua.
\fi
\else
\preces
\fi

\vfill

\pars{Oratio Dominica.}

\cuminitiali{}{temporalia/oratiodominicaalt.gtex}

\vfill
\pagebreak

\rubrica{vel:}

\pars{Supplicatio Litaniæ.}

\cuminitiali{}{temporalia/supplicatiolitaniae.gtex}

\vfill

\pars{Oratio Dominica.}

\cuminitiali{}{temporalia/oratiodominica.gtex}

\vfill
\pagebreak

% Oratio. %%%
\oratio

\vspace{-1mm}

\vfill

\rubrica{Hebdomadarius dicit Dominus vobiscum, vel, absente sacerdote vel diacono, sic concluditur:}

\vspace{2mm}

\antiphona{C}{temporalia/dominusnosbenedicat.gtex}

\rubrica{Postea cantatur a cantore:}

\vspace{2mm}

\cuminitiali{IV}{temporalia/benedicamus-feria-laudes.gtex}

\vspace{1mm}

\vfill
\pagebreak

\end{document}

