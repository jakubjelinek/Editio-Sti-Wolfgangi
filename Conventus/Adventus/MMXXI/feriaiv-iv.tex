\newcommand{\oratio}{\pars{Oratio.}

\noindent Deus, qui hóminem delápsum in mortem conspíciens, Unigéniti tui advéntu redímere voluísti, præsta, quǽsumus, ut qui húmili eius incarnatiónem devotióne faténtur, ipsíus étiam Redemptóris consórtia mereántur.

\noindent Per Dóminum nostrum Iesum Christum, Fílium tuum, qui tecum vivit et regnat in unitáte Spíritus Sancti, Deus, per ómnia sǽcula sæculórum.

\noindent \Rbardot{} Amen.}
\newcommand{\matversus}{\noindent \Vbardot{} Audíte verbum Dómini, gentes.

\noindent \Rbardot{} Et annuntiáte illud in fínibus terræ.}
\newcommand{\lectioi}{\pars{Lectio I.} \scriptura{Is. 49, 14-26; 50, 1}

\noindent De libro Isaíæ prophétæ.

\noindent Dixit Sion: «Derelíquit me Dóminus et Dóminus oblítus est mei». Numquid oblivísci potest múlier infántem suum, ut non misereátur fílio úteri sui? Et si illa oblíta fúerit, ego tamen non oblivíscar tui. Ecce in mánibus meis descrípsi te; muri tui coram me semper. Festínant structóres tui; destruéntes te et dissipántes a te exíbunt. Leva in circúitu óculos tuos et vide: omnes isti congregáti sunt, venérunt tibi. «Vivo ego, dicit Dóminus, quia ómnibus his velut ornaménto vestiéris et circúmdabis tibi eos quasi sponsa». Quia ruínæ tuæ et solitúdines tuæ et terra evérsa: nunc angústa eris præ habitatóribus; et longe erunt, qui devorábant te. Adhuc dicent in áuribus tuis fílii orbitátis tuæ: «Angústus est mihi locus; fac spátium mihi, ut hábitem». Et dices in corde tuo: «Quis génuit mihi istos? Ego orbáta et non páriens, transmigráta et captíva; et istos quis enutrívit? Ecce ego relícta eram sola; et isti ubi erant?».

\noindent Hæc dicit Dóminus Deus: «Ecce levábo ad gentes manum meam et ad pópulos exaltábo signum meum; et áfferent fílios tuos in ulnis, et fíliæ tuæ super úmeros portabúntur. Et erunt reges nutrícii tui, et regínæ nutríces tuæ; vultu in terram demísso adorábunt te et púlverem pedum tuórum lingent. Et scies quia ego Dóminus: non confundéntur, qui sperant in me». Numquid tollétur a forti præda, aut, quod captum fúerit, a robústo salvári póterit? Quia hæc dicit Dóminus: «Equidem et captívus a forti tollétur, et, quod ablátum fúerit a robústo, salvábitur; cum his, qui contendébant tecum, ego conténdam et fílios tuos ego salvábo. Et cibábo hostes tuos cárnibus suis, et quasi musto sánguine suo inebriabúntur; et sciet omnis caro quia ego Dóminus salvátor tuus et redémptor tuus Fortis Iacob».

\noindent Hæc dicit Dóminus: «Ubinam est liber repúdii matris vestræ, quo dimísi eam? Aut quis est créditor meus, cui véndidi vos? Ecce in iniquitátibus vestris vénditi estis, et in sceléribus vestris dimíssa est mater vestra».}
\newcommand{\responsoriumi}{\pars{Responsorium 1.} \scriptura{\Rbar{} Malach. 3, 1 \& Is. 7, 14 \textbf{H34}}

\vspace{-5mm}

\responsorium{VI}{temporalia/resp-modoveniet-CROCHU.gtex}{}}
\newcommand{\lectioii}{\pars{Lectio II.} \scriptura{Lib.1, 46-55: CCL 120, 37-39}

\noindent Ex Expositióne sancti Bedæ Venerábilis presbýteri in Lucam.

\noindent \emph{Et ait María: Magníficat ánima mea Dóminum et exsultávit spíritus meus in Deo salutári meo.} Tanto, inquit, me Dóminus tamque inaudíto múnere sublimávit, quod non ullo linguæ offício explicári, sed ipso vix íntimi péctoris afféctu váleat comprehéndi, et ídeo totas ánimæ vires in agéndis gratiárum láudibus óffero, totum in contemplánda magnitúdine eius, cui non est finis, quicquid vivo, séntio, discérno, gratulánter impéndo, quia et eiúsdem Iesu, id est salutáris, spíritus meus ætérna divinitáte lætátur, cuius mea caro temporáli conceptióne fetátur.

\noindent \emph{Quia fecit mihi magna qui potens est et sanctum nomen eius.}

\noindent Ad inítium cárminis réspicit ubi dictum est: \emph{Magníficat ánima mea Dóminum. } Sola quippe ánima illa, cui Dóminus magna fácere dignátur, dignis eum præcóniis magnificáre et ad consórtes eiúsdem voti ac propósiti potest cohortándo dícere: \emph{Magnificáte Dóminum mecum et exaltémus nomen eius in ínvicem.}

\noindent Nam qui Dóminum, quem cognóvit, quantum in se est magnificáre et nomen eius sanctificáre contémpserit, \textit{mínimus vocábitur in regno cælórum.} Sanctum autem nomen eius vocátur, quia singuláris cúlmine poténtiæ transcéndit omnem creatúram atque ab univérsis quæ fecit longe segregátur.}
\newcommand{\responsoriumii}{\pars{Responsorium 2.} \scriptura{\Rbar{} Is. 45, 8 \Vbar{} Is. 16, 1; \textbf{H35}}

\vspace{-5mm}

\responsorium{II}{temporalia/resp-roratecaeli-CROCHU.gtex}{}}
\newcommand{\lectioiii}{\pars{Lectio III.}

\noindent \emph{Suscépit Israel púerum suum, memorátus misericórdiæ suæ.}

\noindent Pulchre púerum Dómini appéllat Israel, qui ab eo sit ad salvándum suscéptus, obœdiéntem vidélicet et húmilem, iuxta quod Osée dicit: \emph{Quia puer Israel et diléxi eum.}

\noindent Nam qui contémnit humiliári, non potest útique salvári nec dícere cum Prophéta: \emph{Ecce enim Deus ádiuvat me et Dóminus suscéptor est ánimæ meæ. Quicúmque autem humiliáverit se sicut párvulus, hic est maior in regno cælórum.}

\noindent \emph{Sicut locútus est ad patres nostros, Abraham et sémini eius in sǽcula.}

\noindent Semen Abrahæ non carnále sed spiritále signíficat, id est non eius tantum carne progénitos, sed sive in circumcisióne seu in præpútio fídei illíus vestígia secútos. Nam et ipse in præpútio pósitus crédidit, reputatúmque est ei ad iustítiam.

\noindent Advéntus ergo Salvatóris Abrahæ est et sémini eius in sǽcula promíssus, hoc est fíliis promissiónis, quibus dícitur: \emph{Si autem vos Christi, ergo Abrahæ semen estis secúndum promissiónem herédes.}

\noindent Bene autem vel Dómini vel Ioánnis exórtum matres prophetándo prævéniunt, ut sicut peccátum a muliéribus cœpit, ita étiam bona a muliéribus incípiant, et quæ per uníus deceptiónem périit, duábus certátim præconántibus mundo vita reddátur.}
\newcommand{\responsoriumiii}{\pars{Responsorium 3.} \scriptura{\Rbar{} Cantor \Vbar{} Lc. 1, 28; \textbf{H36}}

\vspace{-5mm}

\responsorium{I}{temporalia/resp-annuntiatumest-CROCHU-cumdox.gtex}{}}
\newcommand{\laudes}{\pars{Psalmus 1.} \scriptura{\textbf{H38}}

\vspace{-5mm}

\antiphona{I g\textsuperscript{3}}{temporalia/ant-desionvenietdominus.gtex}

\vspace{-2mm}

\scriptura{Psalmus 107.}

\vspace{-2mm}

\initiumpsalmi{temporalia/ps107-initium-i-g3-auto.gtex}

%\vspace{-1.5mm}

\input{temporalia/ps107-i-g3.tex} \Abardot{}

\vfill
\pagebreak

\pars{Psalmus 2.} \scriptura{Is. 62, 1; \textbf{H38}}

\vspace{-4mm}

\antiphona{II* b}{temporalia/ant-proptersion.gtex}

%\vspace{-4mm}

\scriptura{Canticum Isaiæ, Is. 61, 10-11; Is. 62, 1-5}

%\vspace{-3mm}

\initiumpsalmi{temporalia/isaiae4-initium-ii_-B-auto.gtex}

\input{temporalia/isaiae4-ii_-B.tex}

%\vfill

\antiphona{}{temporalia/ant-proptersion.gtex}

\vfill
\pagebreak

\pars{Psalmus 3.} \scriptura{Lc. 4, 18; \textbf{H38}}

\vspace{-4mm}

\antiphona{II D}{temporalia/ant-spiritusdomini.gtex}

\scriptura{Psalmus 145.}

\initiumpsalmi{temporalia/ps145-initium-ii-D-auto.gtex}

\input{temporalia/ps145-ii-D.tex} \Abardot{}

\vfill
\pagebreak}
\newcommand{\lectiobrevis}{\pars{Lectio Brevis.} \scriptura{Is. 45, 8}

\noindent Roráte, cæli, désuper, et nubes pluant iustítiam; aperiátur terra et gérminet salvatiónem; et iustítia oriátur simul.}
\newcommand{\benedictus}{\pars{Canticum Zachariæ.} \scriptura{Gn. 49, 10}

\vspace{-4mm}

{
\grechangedim{interwordspacetext}{0.18 cm plus 0.15 cm minus 0.05 cm}{scalable}%
\antiphona{I D\textsuperscript{2}}{temporalia/ant-nonauferetursceptrum.gtex}
\grechangedim{interwordspacetext}{0.22 cm plus 0.15 cm minus 0.05 cm}{scalable}%
}

%\trAntIMagnificat

\vspace{-2mm}

\scriptura{Lc. 1, 68-79}

\vspace{-2mm}

\cantusSineNeumas
\initiumpsalmi{temporalia/benedictus-initium-i-D2-auto.gtex}

%\vspace{-1.5mm}

\input{temporalia/benedictus-i-D2.tex} \Abardot{}}
\newcommand{\preces}{\noindent Christum redemptórem exorémus, fratres dilectíssimi, \gredagger{} qui venit ut nos grátia advéntus sui iustificáret, \grestar{} vocémque cum iúbilo innovémus:

\Rbardot{} Veni, Dómine Iesu.

\noindent Qui olim prophetárum vaticínio in carne prædíctus es nascitúrus, \grestar{} nascéntia virtútum in nos corróbora.

\Rbardot{} Veni, Dómine Iesu.

\noindent Præsta nobis ut, qui tuam prædicámus salútem, \grestar{} in te salvatiónem habeámus.

\Rbardot{} Veni, Dómine Iesu.

\noindent Qui venísti contrítis corde medéri, \grestar{} pópuli tui sana languóres.

\Rbardot{} Veni, Dómine Iesu.

\noindent Qui cum venísti reconciliáre dignátus es mundum, \grestar{} ad iudícium véniens ab omni nos pœnárum líbera cruciátu.

\Rbardot{} Veni, Dómine Iesu.}
\newcommand{\vesperas}{\vspace{4mm}

\pars{Psalmus 1.} \scriptura{\textbf{H38}}

\vspace{-4mm}

\antiphona{I g\textsuperscript{3}}{temporalia/ant-desionvenietdominus.gtex}

\vspace{-4mm}

\scriptura{Psalmus 134.}

\initiumpsalmi{temporalia/ps134-initium-i-g3-auto.gtex}

\input{temporalia/ps134-i-g3.tex}

\vfill

\antiphona{}{temporalia/ant-desionvenietdominus.gtex}

\vfill
\pagebreak

\pars{Psalmus 2.} \scriptura{Lc. 4, 18; \textbf{H38}}

\vspace{-4mm}

\antiphona{II D}{temporalia/ant-spiritusdomini.gtex}

\scriptura{Psalmus 135.}

\initiumpsalmi{temporalia/ps135-initium-ii-D-auto.gtex}

\input{temporalia/ps135-ii-D.tex}

\vfill

\antiphona{}{temporalia/ant-spiritusdomini.gtex}

\vfill
\pagebreak

\pars{Psalmus 3.} \scriptura{Is. 62, 1; \textbf{H38}}

\vspace{-4mm}

\antiphona{II* b}{temporalia/ant-proptersion.gtex}

\scriptura{Psalmus 136.}

\initiumpsalmi{temporalia/ps136-initium-ii_-B-auto.gtex}

\input{temporalia/ps136-ii_-B.tex} \Abardot{}

\vfill
\pagebreak

\pars{Psalmus 4.} \scriptura{Is. 42, 12; Hab. 2, 3; \textbf{H31}}

\vspace{-4mm}

\antiphona{V a}{temporalia/ant-ponentdomino.gtex}

\scriptura{Psalmus 137.}

\initiumpsalmi{temporalia/ps137-initium-v-a-auto.gtex}

\input{temporalia/ps137-v-a.tex} \Abardot{}

\vfill
\pagebreak}
\newcommand{\magnificat}{\pars{Canticum B. Mariæ V.} \scriptura{Ier. 10, 7; Ag. 2, 8; Eph. 2, 20.14; \textbf{H40}}

\vspace{-6.5mm}

{
\grechangedim{interwordspacetext}{0.18 cm plus 0.15 cm minus 0.05 cm}{scalable}%
\antiphona{II D}{temporalia/ant-orex.gtex}
\grechangedim{interwordspacetext}{0.22 cm plus 0.15 cm minus 0.05 cm}{scalable}%
}

\vspace{-3mm}

\scriptura{Lc. 1, 46-55}

\vspace{-2mm}

\cantusSineNeumas

\initiumpsalmi{temporalia/magnificat-initium-iisoll-D.gtex}

\vspace{-1.5mm}

\input{temporalia/magnificat-iisoll-D.tex} \Abardot{}

\vspace{-1cm}}
\newcommand{\hebdomada}{infra Hebdom. IV post Pentecosten.}
\newcommand{\oratioLaudes}{\cuminitiali{}{temporalia/oratio4.gtex}}

\renewcommand{\hebdomada}{infra Hebdom. Ultima Adventus.}
\ifx\invitatorium\undefined
\newcommand{\invitatorium}{\pars{Invitatorium.} \scriptura{Phil. 4, 4.5}

\vspace{-6mm}

\antiphona{VI}{temporalia/inv-propeestiamsimplex.gtex}}
\fi
\ifx\hymnusmatutinum\undefined
\newcommand{\hymnusmatutinum}{\pars{Hymnus.}

\vspace{-5mm}

\antiphona{II}{temporalia/hym-VeniRedemptor.gtex}}
\fi
\ifx\hymnuslaudes\undefined
\newcommand{\hymnuslaudes}{\pars{Hymnus}

\cuminitiali{D}{temporalia/hym-MagnisProphetae.gtex}}
\fi
\ifx\hymnusvesperas\undefined
\newcommand{\hymnusvesperas}{\pars{Hymnus}

\cuminitiali{IV}{temporalia/hym-VerbumSalutis.gtex}}
\fi

% LuaLaTeX

\documentclass[a4paper, twoside, 12pt]{article}
\usepackage[latin]{babel} 
%\usepackage[landscape, left=3cm, right=1.5cm, top=2cm, bottom=1cm]{geometry} % okraje stranky
%\usepackage[landscape, a4paper, mag=1166, truedimen, left=2cm, right=1.5cm, top=1.6cm, bottom=0.95cm]{geometry} % okraje stranky
\usepackage[landscape, a4paper, mag=1400, truedimen, left=0.5cm, right=0.5cm, top=0.5cm, bottom=0.5cm]{geometry} % okraje stranky

\usepackage{fontspec}
\setmainfont[FeatureFile={junicode.fea}, Ligatures={Common, TeX}, RawFeature=+fixi]{Junicode}
%\setmainfont{Junicode}

% shortcut for Junicode without ligatures (for the Czech texts)
\newfontfamily\nlfont[FeatureFile={junicode.fea}, Ligatures={Common, TeX}, RawFeature=+fixi]{Junicode}

% Hebrew font:
% http://scripts.sil.org/cms/scripts/page.php?site_id=nrsi&id=SILHebrUnic2
\newfontfamily\hebfont[Scale=1]{Ezra SIL}

\usepackage{multicol}
\usepackage{color}
\usepackage{lettrine}
\usepackage{fancyhdr}

% usual packages loading:
\usepackage{luatextra}
\usepackage{graphicx} % support the \includegraphics command and options
\usepackage{gregoriotex} % for gregorio score inclusion
\usepackage{gregoriosyms}
\usepackage{wrapfig} % figures wrapped by the text
\usepackage{parcolumns}
\usepackage[contents={},opacity=1,scale=1,color=black]{background}
\usepackage{tikzpagenodes}
\usepackage{calc}
\usepackage{longtable}
\usetikzlibrary{calc}

\setlength{\headheight}{14.5pt}

\input{conventuscommune.tex} % Often used macros
%%%% Preklady jednotlivych zpevu (nektere se opakuji, a je dobre mit je
% vsechny na jedne hromade)

% HOURS ---

\newcommand{\trAntI}{\translatioCantus{Muž boží měl kožený toulec, pečlivě
zavázaný, jenž mu visel na šíji a~často se ho dotýkal.}}

\newcommand{\trAntII}{\translatioCantus{Klíč od~něho tak dobře střežil, že
dokud žil v~těle, nikdo z~jeho žáků nezvěděl, co je uvnitř.}}

\newcommand{\trAntIII}{\translatioCantus{Ale když se odebral z~tohoto
života, schránku otevřeli a~objevili v~ní žíněné roucho a~měděný řetěz
potřísněný krví.}}

\newcommand{\trAntIV}{\translatioCantus{A když prohlédli mistrovo tělo,
nalezli jeho tělo na čtyřech místech hluboce zbrázděno ranami od řetězu.}}

\newcommand{\trAntV}{\translatioCantus{Krev vytékající z~těch ran, místy
prostoupila i~žíněným rouchem.}}

\newcommand{\trCapituli}{\translatioCantus{
Miláčkovi Boha a~lidí,
Mojžíšovi požehnané paměti,~\gredagger{}
dopřál slávu rovnou slávě svatých~\grestar{}
učinil ho mocným na postrach nepřátelům
a~jeho slovy zastavil divy.}}

\newcommand{\trLectioBrevis}{\translatioCantus{
Pamatujte na své představené,
kteří vám hlásali Boží slovo.
Uvažte, jak oni skončili život, a~napodobujte jejich víru.
Ježíš Kristus je stejný včera i~dnes i~navěky.
Nenechte se svést věelijakými cizími naukami.}}

\newcommand{\trRespLaud}{\translatioCantus{Spravedlivého vodil Hospodin~\grestar{}
po přímých stezkách. \Vbardot{} A~ukázal mu Boží království.}}

\newcommand{\trRespLaudB}{\translatioCantus{Na tvých hradbách, Jeruzaléme,
ustanovil jsem strážné;~\grestar{}
budou bdít nad mým lidem. \Vbardot{} Ani ve dne, ani v~noci nesmějí nikdy
mlčet.}}

\newcommand{\trVersus}{\translatioCantus{\Vbardot{} Ústa spravedlivého šeptají moudrost, aleluja.
\Rbardot{} A~jeho jazyk ohlašuje právo, aleluja.}}

\newcommand{\trAntBenedictus}{\translatioCantus{Když na bujné oře vložili
nosítka a~sňali jim uzdu, vydali se přímo k~cele božího muže.}}

\newcommand{\trPreces}{\translatioCantus{
\noindent S vděčností chvalme Krista, dobrého Pastýře, \gredagger{} který dal život za své ovce, \grestar{} a~pokorně ho prosme: \Rbardot{} Pane, buď pastýřem svého lidu.

\noindent Kriste, ty dáváš církvi pastýře, a~jejich službou se ujímáš svého lidu, \grestar{} dej, ať v~lásce těch, kteří nás vedou, poznáváme, jak nás miluješ. \Rbardot{} Pane, buď pastýřem svého lidu.

\noindent Ty stále konáš skrze své zástupce službu pastýře a~učitele, \grestar{} nepřestávej nás nikdy vést prostřednictvím svých služebníků. \Rbardot{} Pane, buď pastýřem svého lidu.

\noindent Ty prokazuješ svému lidu skrze jeho pastýře službu lékaře duše i~těla, \grestar{} ochraňuj náš život a~veď nás ke svatosti. \Rbardot{} Pane, buď pastýřem svého lidu.

\noindent Ty posíláš své svaté, aby slovem i~příkladem vedli tvůj lid k~tobě, \grestar{} na jejich přímluvu nás posiluj, abychom vytrvali na cestě, která vede k~věčnému životu. \Rbardot{} Pane, buď pastýřem svého lidu.}}

\newcommand{\trOrationis}{\translatioCantus{Bože, jenž nám dopřáváš radovat
se z~výroční slavnosti svatého tvého vyznavače Havla, uděl dobrotivě,
abychom když slavíme jeho narození, též se řídili podobou jeho skutků.
Skrze…}}
 % Czech translations of the proper texts

\newcommand{\annusEditionis}{2020}

\def\hebinitial#1{%
\leavevmode{\newbox\hebbox\setbox\hebbox\hbox{\hebfont{#1}\hskip 1mm}\kern -\wd\hebbox\hbox{\hebfont{#1}\hskip 1mm}}%
}

%%%% Vicekrat opakovane kousky

\newcommand{\anteOrationem}{
  \rubrica{Ante Orationem, cantatur a Superiore:}

  \pars{Supplicatio Litaniæ.}

  \cuminitiali{}{temporalia/supplicatiolitaniae.gtex}

  \pars{Oratio Dominica.}

  \cuminitiali{}{temporalia/oratiodominica.gtex}

  \rubrica{Deinde dicitur ab Hebdomadario:}

  \cuminitiali{}{temporalia/dominusvobiscum-solemnis.gtex}

  \rubrica{In choro monialium loco Dominus vobiscum dicitur:}

  \sineinitiali{temporalia/domineexaudi.gtex}
}

\setlength{\columnsep}{30pt} % prostor mezi sloupci

%%%%%%%%%%%%%%%%%%%%%%%%%%%%%%%%%%%%%%%%%%%%%%%%%%%%%%%%%%%%%%%%%%%%%%%%%%%%%%%%%%%%%%%%%%%%%%%%%%%%%%%%%%%%%
\begin{document}

% Here we set the space around the initial.
% Please report to http://home.gna.org/gregorio/gregoriotex/details for more details and options
\grechangedim{afterinitialshift}{2.2mm}{scalable}
\grechangedim{beforeinitialshift}{2.2mm}{scalable}

\grechangedim{interwordspacetext}{0.32 cm plus 0.15 cm minus 0.05 cm}{scalable}%
\grechangedim{annotationraise}{-0.2cm}{scalable}

% Here we set the initial font. Change 38 if you want a bigger initial.
% Emit the initials in red.
\grechangestyle{initial}{\color{red}\fontsize{38}{38}\selectfont}

\pagestyle{empty}

%%%% Titulni stranka
\begin{titulusOfficii}
\nomenFesti{Feria IV \hebdomada{}}
\end{titulusOfficii}

\pagebreak

% graphic
\renewcommand{\headrulewidth}{0pt} % no horiz. rule at the header
\fancyhf{}
\pagestyle{fancy}

\cantusSineNeumas

\hora{Ad Matutinum.}

\vspace{2mm}

\cuminitiali{}{temporalia/dominelabiamea.gtex}

\vspace{2mm}

\pars{Invitatorium.} \scriptura{Lc. 24, 34; Psalmus 94; \textbf{H232}}

\vspace{-6mm}

\antiphona{VI}{temporalia/inv-surrexitdominusvere.gtex}

\vfill
\pagebreak

\pars{Hymnus.}

\vspace{-5mm}

\scriptura{\textbf{AR454}}

{
\grechangedim{interwordspacetext}{0.30 cm plus 0.15 cm minus 0.05 cm}{scalable}%
\antiphona{IV}{temporalia/hym-RexSempiterne.gtex}
\grechangedim{interwordspacetext}{0.32 cm plus 0.15 cm minus 0.05 cm}{scalable}%
}
%{
%\vspace{-5mm}
%\setlength{\columnsep}{0pt} % prostor mezi sloupci
%\input{hym-RexSempiterne-bohtext.tex}
%\setlength{\columnsep}{30pt} % prostor mezi sloupci
%}

\vfill
\pagebreak

\pars{Psalmus 1.}

%\vspace{-5mm}

\antiphona{I g}{temporalia/ant-alleluia-fiv-matutinum.gtex}

%\vspace{-5mm}

\scriptura{Ps. 44, 2-10}

%\vspace{-2mm}

\initiumpsalmi{temporalia/ps44i-initium-i-g-auto.gtex}

%\psalmusEtTranslatioT{temporalia/ps44i-III-comb.tex}{10cm}

\input{temporalia/ps44i-III.tex}

\vfill
\pagebreak

\pars{Psalmus 2.} \scriptura{Ps. 44, 11-18}

%\vspace{-2mm}

\initiumpsalmi{temporalia/ps44ii-initium-i-g-auto.gtex}

%\psalmusEtTranslatioT{temporalia/ps44i-III-comb.tex}{10cm}

\input{temporalia/ps44ii-III.tex}

\vfill
\pagebreak

\pars{Psalmus 3.} \scriptura{Ps. 45}

%\vspace{-2mm}

\initiumpsalmi{temporalia/ps45-initium-i-g-auto.gtex}

%\psalmusEtTranslatioT{temporalia/ps45-III-comb.tex}{10cm}

\input{temporalia/ps45-III.tex}

\vfill
\pagebreak

\pars{Psalmus 4.} \scriptura{Ps. 47}

%\vspace{-2mm}

\initiumpsalmi{temporalia/ps47-initium-i-g-auto.gtex}

%\psalmusEtTranslatioT{temporalia/ps47-III-comb.tex}{10cm}

\input{temporalia/ps47-III.tex}

\vfill
\pagebreak

\pars{Psalmus 5.} \scriptura{Ps. 48, 2-13}

%\vspace{-2mm}

\initiumpsalmi{temporalia/ps48i-initium-i-g-auto.gtex}

%\psalmusEtTranslatioT{temporalia/ps48i-III-comb.tex}{10cm}

\input{temporalia/ps48i-III.tex}

\vfill
\pagebreak

\pars{Psalmus 6.} \scriptura{Ps. 48, 14-21}

%\vspace{-2mm}

\initiumpsalmi{temporalia/ps48ii-initium-i-g-auto.gtex}

%\psalmusEtTranslatioT{temporalia/ps48ii-III-comb.tex}{10cm}

\input{temporalia/ps48ii-III.tex}

\vfill
\pagebreak

\pars{Psalmus 7.} \scriptura{Ps. 49, 1-15}

%\vspace{-2mm}

\initiumpsalmi{temporalia/ps49i-initium-i-g-auto.gtex}

%\psalmusEtTranslatioT{temporalia/ps49i-III-comb.tex}{10cm}

\input{temporalia/ps49i-III.tex}

\vfill
\pagebreak

\pars{Psalmus 8.} \scriptura{Ps. 49, 16-23}

%\vspace{-2mm}

\initiumpsalmi{temporalia/ps49ii-initium-i-g-auto.gtex}

%\psalmusEtTranslatioT{temporalia/ps49ii-III-comb.tex}{10cm}

\input{temporalia/ps49ii-III.tex}

\vfill
\pagebreak

\pars{Psalmus 9.} \scriptura{Ps. 50}

%\vspace{-2mm}

\initiumpsalmi{temporalia/ps50-initium-i-g-auto.gtex}

%\psalmusEtTranslatioT{temporalia/ps50-VI-comb.tex}{10cm}

\input{temporalia/ps50-VI.tex}

\vfill
%\pagebreak

\antiphona{}{temporalia/ant-alleluia-fiv-matutinum.gtex}

\vfill
\pagebreak

\noindent \Vbardot{} Gavísi sunt discípuli, allelúia.
\noindent \Rbardot{} Viso Dómino, allelúia.

\noindent Pater noster.

\pars{Absolutio.}

\cuminitiali{}{temporalia/absolutio-avinculis.gtex}

\vfill
\pagebreak

\ifx\magnificat\undefined
\cuminitiali{}{temporalia/benedictio-solemn-evangelica.gtex}
\else
\cuminitiali{}{temporalia/benedictio-solemn-ille.gtex}
\fi

\vspace{7mm}

\lectioi

\noindent \Vbardot{} Tu autem, Dómine, miserére nobis.
\noindent \Rbardot{} Deo grátias.

\vfill
\pagebreak

\responsoriumi

\vfill
\pagebreak

\cuminitiali{}{temporalia/benedictio-solemn-divinum.gtex}

\vspace{7mm}

\lectioii

\noindent \Vbardot{} Tu autem, Dómine, miserére nobis.
\noindent \Rbardot{} Deo grátias.

\vfill
\pagebreak

\responsoriumii

\vfill
\pagebreak

\ifx\magnificat\undefined
\cuminitiali{}{temporalia/benedictio-solemn-adsocietatem.gtex}
\else
\cuminitiali{}{temporalia/benedictio-solemn-ignem.gtex}
\fi

\vspace{7mm}

\lectioiii

\noindent \Vbardot{} Tu autem, Dómine, miserére nobis.
\noindent \Rbardot{} Deo grátias.

\vfill
\pagebreak

% Te Deum

%\pars{Hymnus Ambrosianus}

\vspace{-5mm}

{
\grechangedim{interwordspacetext}{0.22 cm plus 0.15 cm minus 0.05 cm}{scalable}%
\cuminitiali{III}{temporalia/tedeum-solemnis.gtex}
\grechangedim{interwordspacetext}{0.32 cm plus 0.15 cm minus 0.05 cm}{scalable}%
}

\vfill
\pagebreak

\rubrica{Reliqua omittuntur, nisi Laudes separandæ sint.}

\pars{Oratio}

\noindent \Vbardot{} Dómine, exáudi oratiónem meam.

\noindent \Rbardot{} Et clamor meus ad te véniat.

Orémus:

\oratioMatutinum

\noindent \Rbardot{} Amen.

\vspace{7mm}

\pars{Conclusio}

\noindent \Vbardot{} Dómine, exáudi oratiónem meam.

\noindent \Rbardot{} Et clamor meus ad te véniat.

\noindent \Vbardot{} Benedicámus Dómino, allelúia, allelúia.

\noindent \Rbardot{} Deo grátias, allelúia, allelúia.

\noindent \Vbardot{} Fidélium ánimæ per misericórdiam Dei requiéscant in pace.

\noindent \Rbardot{} Amen.

\vfill
\pagebreak

\hora{Ad Laudes.} %%%%%%%%%%%%%%%%%%%%%%%%%%%%%%%%%%%%%%%%%%%%%%%%%%%%%
%\sideThumbs{Laudes}

\cantusSineNeumas

\vspace{0.5cm}
\grechangedim{interwordspacetext}{0.18 cm plus 0.15 cm minus 0.05 cm}{scalable}%
\cuminitiali{}{temporalia/deusinadiutorium-communis.gtex}
\grechangedim{interwordspacetext}{0.32 cm plus 0.15 cm minus 0.05 cm}{scalable}%

\vfill
%\pagebreak

\pars{Psalmus 1.}

\vspace{-0.4cm}

\antiphona{VII a}{temporalia/ant-alleluia-fiv-laudes-1.gtex}

\scriptura{Psalmus 50.}

\initiumpsalmi{temporalia/ps50-initium-vii-a-auto.gtex}

%\psalmusEtTranslatioT{temporalia/ps50-III-comb.tex}{10cm}
\input{temporalia/ps50-III.tex}

\vspace{-1cm}

\vfill
\pagebreak

\pars{Psalmus 2.} \scriptura{Psalmus 63.}

\initiumpsalmi{temporalia/ps63-initium-vii-a-auto.gtex}

%\psalmusEtTranslatioT{temporalia/ps63-III-comb.tex}{10cm}
\input{temporalia/ps63-III.tex}

\vfill
\pagebreak

\pars{Psalmus 3.} \scriptura{Psalmus 64.}

\initiumpsalmi{temporalia/ps64-initium-vii-a-auto.gtex}

%\psalmusEtTranslatioT{temporalia/ps64-III-comb.tex}{10cm}
\input{temporalia/ps64-III.tex}

\vfill

\vspace{-6mm}

\antiphona{}{temporalia/ant-alleluia-fiv-laudes-1.gtex} % repeat the antiphon - new page

\vfill
\pagebreak

\pars{Psalmus 4.} \scriptura{1 Sam. 2, 10; \textbf{H96}}

\vspace{-7mm}

\antiphona{I g\textsuperscript{2}}{temporalia/ant-dominusjudicabit-tp.gtex}

%\vspace{-4mm}

\scriptura{Canticum Annæ, 1 Reg. 2, 1-10}

%\vspace{-3mm}

\initiumpsalmi{temporalia/anna-initium-i-g2-auto.gtex}

%\psalmusEtTranslatioT{temporalia/anna-comb.tex}{10cm}
\input{temporalia/anna.tex}

%\vfill

\antiphona{}{temporalia/ant-dominusjudicabit-tp.gtex}

\vfill
\pagebreak

\pars{Psalmus 5.}

\vspace{-0.4cm}

\antiphona{II D}{temporalia/ant-alleluia-fiv-laudes-2.gtex}

\scriptura{Psalmus 148.}

\initiumpsalmi{temporalia/ps148-initium-ii-D-auto.gtex}

%\psalmusEtTranslatioT{temporalia/ps148-III-comb.tex}{10cm}
\input{temporalia/ps148-III.tex}

\rubrica{Hic non dicitur Gloria Patri.}

\vfill
\pagebreak

%
\scriptura{Psalmus 149.}

\initiumpsalmi{temporalia/ps149-initium-ii-D-auto.gtex}

%\psalmusEtTranslatioT{temporalia/ps149-III-comb.tex}{10cm}
\input{temporalia/ps149-III.tex}

\rubrica{Hic non dicitur Gloria Patri.}

\vfill
\pagebreak

%
\scriptura{Psalmus 150.}

\initiumpsalmi{temporalia/ps150-initium-ii-D-auto.gtex}

%\psalmusEtTranslatioT{temporalia/ps150-III-comb.tex}{10cm}
\input{temporalia/ps150-III.tex}

\vfill

\vspace{-6mm}

\antiphona{}{temporalia/ant-alleluia-fiv-laudes-2.gtex} % repeat the antiphon - new page

\vfill
\pagebreak

\pars{Capitulum.} \scriptura{Rom. 6, 9-10}

\grechangedim{interwordspacetext}{0.12 cm plus 0.15 cm minus 0.05 cm}{scalable}%
\cuminitiali{}{temporalia/capitulum-ChristusResurgens.gtex}
\grechangedim{interwordspacetext}{0.32 cm plus 0.15 cm minus 0.05 cm}{scalable}%

% preklad Jeruz. bible
%\trCapituliI

\vfill

\pars{Responsorium breve.} \scriptura{Cf. Mt. 28, 6; Cf. Gal. 3, 13}

\cuminitiali{VI}{temporalia/respbr-laud.gtex}

%\trResp

\vfill
\pagebreak

\pars{Hymnus}

\cuminitiali{VIII}{temporalia/hym-AuroraLucis.gtex}
\vspace{-3mm}
%\input{hym-AuroraLucis-bohtext.tex}

\vfill
%\pagebreak

\pars{Versus.}

% Versus. %%%
\sineinitiali{temporalia/versus-inresurrectione.gtex}

%\noindent \trVersus

\vfill
\pagebreak

\benedictus

\vspace{-1cm}

\vfill
\pagebreak

%\sideThumbs{{\scriptsize{}Fine horarum}}

\anteOrationem

\pagebreak

% Oratio. %%%
\oratioLaudes

\vspace{-1mm}
%\trOrationisI

\vfill

\rubrica{Hebdomadarius dicit iterum Dominus vobiscum. Postea cantatur a cantore:}
\vspace{2mm}

\cuminitiali{VII}{temporalia/benedicamus-tempore-paschali.gtex}

\vspace{1mm}

\ifx\magnificat\undefined
\else
\vfill
\pagebreak

\hora{Ad Vesperas.} %%%%%%%%%%%%%%%%%%%%%%%%%%%%%%%%%%%%%%%%%%%%%%%%%%%%%
%\sideThumbs{Vesperæ}

\cantusSineNeumas

%\vspace{0.5cm}
\grechangedim{interwordspacetext}{0.18 cm plus 0.15 cm minus 0.05 cm}{scalable}%
\cuminitiali{}{temporalia/deusinadiutorium-communis.gtex}
\grechangedim{interwordspacetext}{0.32 cm plus 0.15 cm minus 0.05 cm}{scalable}%

\vfill
%\pagebreak

\vspace{4mm}

\pars{Psalmus 1.}

\vspace{-0.4cm}

\antiphona{III g}{temporalia/ant-alleluia-fiv-vesperas.gtex}

\vspace{-4mm}

\scriptura{Psalmus 134.}

\initiumpsalmi{temporalia/ps134-initium-iii-g-auto.gtex}

%\psalmusEtTranslatioT{temporalia/ps134-III-comb.tex}{10cm}
\input{temporalia/ps134-III.tex}

\vspace{-1cm}

\vfill
\pagebreak

\pars{Psalmus 2.} \scriptura{Psalmus 135.}

\initiumpsalmi{temporalia/ps135-initium-iii-g-auto.gtex}

%\psalmusEtTranslatioT{temporalia/ps135-III-comb.tex}{10cm}
\input{temporalia/ps135-III.tex}

\vfill
\pagebreak

\pars{Psalmus 3.} \scriptura{Psalmus 136.}

\initiumpsalmi{temporalia/ps136-initium-iii-g-auto.gtex}

%\psalmusEtTranslatioT{temporalia/ps136-III-comb.tex}{10cm}
\input{temporalia/ps136-III.tex}

\vfill
\pagebreak

\pars{Psalmus 4.} \scriptura{Psalmus 137.}

\initiumpsalmi{temporalia/ps137-initium-iii-g-auto.gtex}

%\psalmusEtTranslatioT{temporalia/ps137-III-comb.tex}{10cm}
\input{temporalia/ps137-III.tex}

\vfill

\vspace{-6mm}

\antiphona{}{temporalia/ant-alleluia-fiv-vesperas.gtex} % repeat the antiphon - new page

\vfill
\pagebreak

\pars{Capitulum.} \scriptura{Rom. 6, 9-10}

\grechangedim{interwordspacetext}{0.12 cm plus 0.15 cm minus 0.05 cm}{scalable}%
\cuminitiali{}{temporalia/capitulum-ChristusResurgens.gtex}
\grechangedim{interwordspacetext}{0.32 cm plus 0.15 cm minus 0.05 cm}{scalable}%

% preklad Jeruz. bible
%\trCapituliI

\vfill

\pars{Responsorium breve.} \scriptura{Lc. 24, 34}

\cuminitiali{VI}{temporalia/respbr-vesp.gtex}

%\trResp

\vfill
\pagebreak

\pars{Hymnus}

\cuminitiali{VIII}{temporalia/hym-AdCoenam.gtex}
\vspace{-3mm}
%\begin{translatioMulticol}{4}
U~Beránkovy hostiny\\
oděni rouchy bílými,\\
když Rudým mořem prošli jsme,\\
Vladaři Kristu zpívejme.\\
\\
Když jeho tělem posvátným,\\
na kříži obětovaným,\\
se sytíme a~pijeme\\
jeho krev, v~Bohu žijeme.\columnbreak

Chráněni tímto pokrmem\\
před smrtonosným andělem,\\
svrhli jsme z~beder kruté jho\\
tyrana bezohledného.\\
\\
Kristus je naší paschou teď,\\
on sám se vydal za oběť\\
a~místo přesnic našim rtům\\
své tělo dává za pokrm.\columnbreak

Tys, nejčistější Oběti,\\
zlomila vládu podsvětí.\\
Z~otroctví lid je vykoupen,\\
odměna žití kyne všem.\\
\\
Hle, Kristus, když vstal ze hrobu,\\
jde z~pekel v~slavném průvodu\\
a~brány nebes otevřev,\\
vládce tmy vleče v~okovech.\columnbreak

Buď věčně, Kriste, věrným svým\\
plesáním velikonočním.\\
Nás, milostí tvou vzkříšené,\\
vem k~oslavě své vítězné. \\
\\
Sláva tobě, Pane,\\
jenž jsi vstal z~mrtvých,\\
s~Otcem i~Svatým Duchem\\
na věčné věky.\\
Amen.
\end{translatioMulticol}


\vfill
\pagebreak

\pars{Versus.} \scriptura{Lc. 24, 29}

% Versus. %%%
\sineinitiali{temporalia/versus-mane.gtex}

%\noindent \trVersus

\vfill
\pagebreak

\magnificat

\vspace{-1cm}

\vfill
\pagebreak

%\sideThumbs{{\scriptsize{}Fine horarum}}

\anteOrationem

\pagebreak

% Oratio. %%%
\oratioLaudes

\vspace{-1mm}
%\trOrationisI

\vfill

\rubrica{Hebdomadarius dicit iterum Dominus vobiscum. Postea cantatur a cantore:}
\vspace{2mm}

\cuminitiali{VII}{temporalia/benedicamus-tempore-paschali.gtex}

\vspace{1mm}
\fi

\end{document}

