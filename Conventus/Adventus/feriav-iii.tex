\newcommand{\lectioi}{\pars{Lectio I.} \scriptura{Is. 45, 1-7}

\noindent De libro Isaíæ prophétæ.

\noindent Hæc dicit Dóminus de uncto suo Cyro: «Apprehéndi déxteram eius, ut subíciam ante fáciem eius gentes et dorsa regum vertam et apériam coram eo iánuas; et portæ non claudéntur. Ego ante te ibo et montes humiliábo; portas ǽreas cónteram et vectes férreos confríngam. Et dabo tibi thesáuros abscónditos et divítias occúltas, ut scias quia ego Dóminus, qui vocávi te nómine tuo, Deus Israel. Propter servum meum Iacob et Israel eléctum meum, et vocávi te nómine tuo; designávi te, et non cognovísti me. Ego Dóminus, et non est ámplius: extra me non est Deus. Accínxi te, et non cognovísti me, ut sciant ab ortu solis et ab occidénte quóniam absque me nullus est. Ego Dóminus, et non est alter, formans lucem et creans ténebras, fáciens pacem et creans malum: ego Dóminus fáciens ómnia hæc.}
\newcommand{\responsoriumi}{\pars{Responsorium 1.} \scriptura{\textbf{H34}}

\vspace{-5mm}

\responsorium{II}{temporalia/resp-festina-CROCHU.gtex}{}}
\newcommand{\lectioii}{\pars{Lectio II.} \scriptura{Is. 45, 8-13}

\noindent Roráte, cæli, désuper, et nubes pluant iustítiam; aperiátur terra et gérminet salvatiónem; et iustítia oriátur simul: ego Dóminus creávi eam». Væ, qui contradícit fictóri suo, testa de vasis fictílibus terræ! Numquid dicet lutum fígulo suo: «Quid facis?» et «Opus tuum absque mánibus est?». Væ, qui dicit patri: «Quid géneras?» et mulíeri: «Quid párturis?». Hæc dicit Dóminus, Sanctus Israel, plastes eius: «Numquid ventúra interrogátis me super fílios meos et super opus mánuum meárum mandátis mihi? Ego feci terram et hóminem super eam creávi ego; manus meæ tetendérunt cælos, et omni milítiæ eórum mandávi. Ego suscitávi eum in iustítia et omnes vias eius dírigam; ipse ædificábit civitátem meam et captivitátem meam dimíttet non in prétio neque in munéribus», dicit Dóminus exercítuum.}
\newcommand{\responsoriumii}{\pars{Responsorium 2.} \scriptura{\Rbar{} Is. 45, 8 \Vbar{} Is. 16, 1; \textbf{H35}}

\vspace{-5mm}

\responsorium{II}{temporalia/resp-roratecaeli-CROCHU.gtex}{}}
\newcommand{\lectioiii}{\pars{Lectio III.} \scriptura{Ep. 31, 2-3: PL 54, 791-793}

\noindent Ex Epístolis sancti Leónis Magni papæ.

\noindent Nihil prodest Dóminum nostrum, Beátæ Maríæ Vírginis fílium, verum perfectúmque hóminem dícere, si non illíus géneris homo créditur, cuius in Evangélio prædicátur.

\noindent Dicit enim Matthǽus: Liber generatiónis Iesu Christi fílii David, fílii Abraham; et ita humánæ oríginis órdinem séquitur, ut generatiónum líneas usque ad Ioseph, cui mater Dómini erat desponsáta, dedúcat.

\noindent Lucas vero, retrórsum successiónum gradus rélegens, ad ipsum humáni géneris príncipem redit, ut Adam primum et Adam novíssimum eiúsdem osténdat esse natúræ.

\noindent Potúerat quippe omnipoténtia Fílii Dei sic ad docéndos iustificandósque hómines apparére, quómodo et patriárchis et prophétis in spécie carnis appáruit, cum aut luctámen íniit, aut sermónem conséruit, cumve offícia hospitalitátis non ábnuit, vel étiam appósitum cibum sumpsit.

\noindent Sed illæ imágines huius hóminis erant índices, cuius veritátem ex præcedéntium patrum stirpe suméndam significatiónes mýsticæ nuntiábant.

\noindent Et ídeo sacraméntum reconciliatiónis nostræ, ante témpora ætérna dispósitum, nullæ implébant figúræ, quia nondum supervénerat Spíritus Sanctus in Vírginem, nec virtus Altíssimi obumbráverat ei; ut intra intemeráta víscera, ædificánte sibi Sapiéntia domum, Verbum caro fíeret; et, forma Dei ac forma servi in unam conveniénte persónam, Creátor témporum nascerétur in témpore; et, per quem facta sunt ómnia, ipse inter ómnia gignerétur.

\noindent Nisi enim novus homo, factus in similitúdinem carnis peccáti, nostram suscíperet vetustátem, et, consubstantiális Patri, consubstantiális esse dignarétur et matri, naturámque sibi nostram solus a peccáto liber uníret, sub iugo diáboli generáliter tenerétur humána captívitas; nec uti possémus triumphántis victória, si extra nostram esset consérta natúram.

\noindent De hac autem participatióne mirábili sacraméntum nobis regeneratiónis illúxit, ut per ipsum Spíritum, per quem Christum et concéptus est et natus, étiam nos spiritáli rursus orígine nascerémur.

\noindent Propter quod ab Evangelísta de credéntibus dícitur: Qui non ex sanguínibus, neque ex voluntáte carnis, neque ex voluntáte viri, sed ex Deo nati sunt.}
\newcommand{\responsoriumiii}{\pars{Responsorium 3.} \scriptura{\Rbar{} Is. 16, 1 \Vbar{} Ps. 49, 2; \textbf{H35}}

\vspace{-5mm}

\responsorium{II}{temporalia/resp-emitteagnumdomine-CROCHU-cumdox.gtex}{}}
\newcommand{\lectiobrevis}{\pars{Lectio Brevis.} \scriptura{Is. 11, 1-3}

\noindent Egrediétur virga de stirpe Iesse, et flos de radíce eius ascéndet; et requiéscet super eum spíritus Dómini: spíritus sapiéntiæ et intelléctus, spíritus consílii et fortitúdinis, spíritus sciéntiæ et timóris Dómini; et delíciæ eius in timóre Dómini.}
\newcommand{\benedictus}{\pars{Canticum Zachariæ.} \scriptura{Gn. 49, 10}

\vspace{-4mm}

{
\grechangedim{interwordspacetext}{0.18 cm plus 0.15 cm minus 0.05 cm}{scalable}%
\antiphona{I D\textsuperscript{2}}{temporalia/ant-nonauferetursceptrum.gtex}
\grechangedim{interwordspacetext}{0.22 cm plus 0.15 cm minus 0.05 cm}{scalable}%
}

%\trAntIMagnificat

\vspace{-2mm}

\scriptura{Lc. 1, 68-79}

\vspace{-2mm}

\cantusSineNeumas
\initiumpsalmi{temporalia/benedictus-initium-i-D2-auto.gtex}

%\vspace{-1.5mm}

%\psalmusEtTranslatioT{temporalia/benedictus-XXVIII-comb.tex}{10.2cm}
\input{temporalia/benedictus-XXVIII.tex} \Abardot{}}
\newcommand{\oratio}{\pars{Oratio.}

\noindent Deus, humánæ cónditor et redémptor natúræ, qui Verbum tuum in útero perpétuæ virginitátis carnem assúmere voluísti, réspice propítius ad preces nostras, ut Unigénitus tuus, nostra humanitáte suscépta, nos divíno suo consórtio sociáre dignétur.

\noindent Per Dóminum nostrum Iesum Christum, Fílium tuum, qui tecum vivit et regnat in unitáte Spíritus Sancti, Deus, per ómnia sǽcula sæculórum.

\noindent \Rbardot{} Amen.}
\newcommand{\magnificat}{\pars{Canticum B. Mariæ V.} \scriptura{Eccli. 24, 5; Sap. 8, 1; Is. 40, 14; \textbf{H40}}

\vspace{-6mm}

{
\grechangedim{interwordspacetext}{0.18 cm plus 0.15 cm minus 0.05 cm}{scalable}%
\antiphona{II D}{temporalia/ant-osapientia.gtex}
\grechangedim{interwordspacetext}{0.22 cm plus 0.15 cm minus 0.05 cm}{scalable}%
}

%\trAntIMagnificat

\vspace{-2mm}

\scriptura{Lc. 1, 46-55}

\vspace{-2mm}

\cantusSineNeumas

\initiumpsalmi{temporalia/magnificat-initium-iisoll-D.gtex}

\vspace{-1.5mm}

%\psalmusEtTranslatioT{temporalia/magnificat-XXXII-comb.tex}{10.2cm}
\input{temporalia/magnificat-XXXII.tex} \Abardot{}

\vspace{-1cm}}
\newcommand{\hebdomada}{infra Hebdom. III Adventus.}
\newcommand{\oratioLaudes}{\cuminitiali{}{temporalia/oratio3vo.gtex}}
\newcommand{\responsoriumbreve}{\pars{Responsorium breve.} \scriptura{Is. 60, 2; \textbf{H20}}

\cuminitiali{IV}{temporalia/resp-superte.gtex}}

% LuaLaTeX

\documentclass[a4paper, twoside, 12pt]{article}
\usepackage[latin]{babel} 
%\usepackage[landscape, left=3cm, right=1.5cm, top=2cm, bottom=1cm]{geometry} % okraje stranky
%\usepackage[landscape, a4paper, mag=1166, truedimen, left=2cm, right=1.5cm, top=1.6cm, bottom=0.95cm]{geometry} % okraje stranky
\usepackage[landscape, a4paper, mag=1400, truedimen, left=0.5cm, right=0.5cm, top=0.5cm, bottom=0.5cm]{geometry} % okraje stranky

\usepackage{fontspec}
\setmainfont[FeatureFile={junicode.fea}, Ligatures={Common, TeX}, RawFeature=+fixi]{Junicode}
%\setmainfont{Junicode}

% shortcut for Junicode without ligatures (for the Czech texts)
\newfontfamily\nlfont[FeatureFile={junicode.fea}, Ligatures={Common, TeX}, RawFeature=+fixi]{Junicode}

% Hebrew font:
% http://scripts.sil.org/cms/scripts/page.php?site_id=nrsi&id=SILHebrUnic2
\newfontfamily\hebfont[Scale=1]{Ezra SIL}

\usepackage{multicol}
\usepackage{color}
\usepackage{lettrine}
\usepackage{fancyhdr}

% usual packages loading:
\usepackage{luatextra}
\usepackage{graphicx} % support the \includegraphics command and options
\usepackage{gregoriotex} % for gregorio score inclusion
\usepackage{gregoriosyms}
\usepackage{wrapfig} % figures wrapped by the text
\usepackage{parcolumns}
\usepackage[contents={},opacity=1,scale=1,color=black]{background}
\usepackage{tikzpagenodes}
\usepackage{calc}
\usepackage{longtable}
\usetikzlibrary{calc}

\setlength{\headheight}{14.5pt}

\input{conventuscommune.tex} % Often used macros
%%%% Preklady jednotlivych zpevu (nektere se opakuji, a je dobre mit je
% vsechny na jedne hromade)

% HOURS ---

\newcommand{\trAntI}{\translatioCantus{Muž boží měl kožený toulec, pečlivě
zavázaný, jenž mu visel na šíji a~často se ho dotýkal.}}

\newcommand{\trAntII}{\translatioCantus{Klíč od~něho tak dobře střežil, že
dokud žil v~těle, nikdo z~jeho žáků nezvěděl, co je uvnitř.}}

\newcommand{\trAntIII}{\translatioCantus{Ale když se odebral z~tohoto
života, schránku otevřeli a~objevili v~ní žíněné roucho a~měděný řetěz
potřísněný krví.}}

\newcommand{\trAntIV}{\translatioCantus{A když prohlédli mistrovo tělo,
nalezli jeho tělo na čtyřech místech hluboce zbrázděno ranami od řetězu.}}

\newcommand{\trAntV}{\translatioCantus{Krev vytékající z~těch ran, místy
prostoupila i~žíněným rouchem.}}

\newcommand{\trCapituli}{\translatioCantus{
Miláčkovi Boha a~lidí,
Mojžíšovi požehnané paměti,~\gredagger{}
dopřál slávu rovnou slávě svatých~\grestar{}
učinil ho mocným na postrach nepřátelům
a~jeho slovy zastavil divy.}}

\newcommand{\trLectioBrevis}{\translatioCantus{
Pamatujte na své představené,
kteří vám hlásali Boží slovo.
Uvažte, jak oni skončili život, a~napodobujte jejich víru.
Ježíš Kristus je stejný včera i~dnes i~navěky.
Nenechte se svést věelijakými cizími naukami.}}

\newcommand{\trRespLaud}{\translatioCantus{Spravedlivého vodil Hospodin~\grestar{}
po přímých stezkách. \Vbardot{} A~ukázal mu Boží království.}}

\newcommand{\trRespLaudB}{\translatioCantus{Na tvých hradbách, Jeruzaléme,
ustanovil jsem strážné;~\grestar{}
budou bdít nad mým lidem. \Vbardot{} Ani ve dne, ani v~noci nesmějí nikdy
mlčet.}}

\newcommand{\trVersus}{\translatioCantus{\Vbardot{} Ústa spravedlivého šeptají moudrost, aleluja.
\Rbardot{} A~jeho jazyk ohlašuje právo, aleluja.}}

\newcommand{\trAntBenedictus}{\translatioCantus{Když na bujné oře vložili
nosítka a~sňali jim uzdu, vydali se přímo k~cele božího muže.}}

\newcommand{\trPreces}{\translatioCantus{
\noindent S vděčností chvalme Krista, dobrého Pastýře, \gredagger{} který dal život za své ovce, \grestar{} a~pokorně ho prosme: \Rbardot{} Pane, buď pastýřem svého lidu.

\noindent Kriste, ty dáváš církvi pastýře, a~jejich službou se ujímáš svého lidu, \grestar{} dej, ať v~lásce těch, kteří nás vedou, poznáváme, jak nás miluješ. \Rbardot{} Pane, buď pastýřem svého lidu.

\noindent Ty stále konáš skrze své zástupce službu pastýře a~učitele, \grestar{} nepřestávej nás nikdy vést prostřednictvím svých služebníků. \Rbardot{} Pane, buď pastýřem svého lidu.

\noindent Ty prokazuješ svému lidu skrze jeho pastýře službu lékaře duše i~těla, \grestar{} ochraňuj náš život a~veď nás ke svatosti. \Rbardot{} Pane, buď pastýřem svého lidu.

\noindent Ty posíláš své svaté, aby slovem i~příkladem vedli tvůj lid k~tobě, \grestar{} na jejich přímluvu nás posiluj, abychom vytrvali na cestě, která vede k~věčnému životu. \Rbardot{} Pane, buď pastýřem svého lidu.}}

\newcommand{\trOrationis}{\translatioCantus{Bože, jenž nám dopřáváš radovat
se z~výroční slavnosti svatého tvého vyznavače Havla, uděl dobrotivě,
abychom když slavíme jeho narození, též se řídili podobou jeho skutků.
Skrze…}}
 % Czech translations of the proper texts

\newcommand{\annusEditionis}{2020}

\def\hebinitial#1{%
\leavevmode{\newbox\hebbox\setbox\hebbox\hbox{\hebfont{#1}\hskip 1mm}\kern -\wd\hebbox\hbox{\hebfont{#1}\hskip 1mm}}%
}

%%%% Vicekrat opakovane kousky

\newcommand{\anteOrationem}{
  \rubrica{Ante Orationem, cantatur a Superiore:}

  \pars{Supplicatio Litaniæ.}

  \cuminitiali{}{temporalia/supplicatiolitaniae.gtex}

  \pars{Oratio Dominica.}

  \cuminitiali{}{temporalia/oratiodominica.gtex}

  \rubrica{Deinde dicitur ab Hebdomadario:}

  \cuminitiali{}{temporalia/dominusvobiscum-solemnis.gtex}

  \rubrica{In choro monialium loco Dominus vobiscum dicitur:}

  \sineinitiali{temporalia/domineexaudi.gtex}
}

\setlength{\columnsep}{30pt} % prostor mezi sloupci

%%%%%%%%%%%%%%%%%%%%%%%%%%%%%%%%%%%%%%%%%%%%%%%%%%%%%%%%%%%%%%%%%%%%%%%%%%%%%%%%%%%%%%%%%%%%%%%%%%%%%%%%%%%%%
\begin{document}

% Here we set the space around the initial.
% Please report to http://home.gna.org/gregorio/gregoriotex/details for more details and options
\grechangedim{afterinitialshift}{2.2mm}{scalable}
\grechangedim{beforeinitialshift}{2.2mm}{scalable}

\grechangedim{interwordspacetext}{0.22 cm plus 0.15 cm minus 0.05 cm}{scalable}%
\grechangedim{annotationraise}{-0.2cm}{scalable}

% Here we set the initial font. Change 38 if you want a bigger initial.
% Emit the initials in red.
\grechangestyle{initial}{\color{red}\fontsize{38}{38}\selectfont}

\pagestyle{empty}

%%%% Titulni stranka
\begin{titulusOfficii}
\nomenFesti{Feria V infra Hebdom. Ultima Adventus.}
\end{titulusOfficii}

\pars{}

\scriptura{}

\pagebreak

% graphic
\renewcommand{\headrulewidth}{0pt} % no horiz. rule at the header
\fancyhf{}
\pagestyle{fancy}

\cantusSineNeumas

\hora{Ad Matutinum.}

\vspace{2mm}

\cuminitiali{}{temporalia/dominelabiamea.gtex}

\vspace{2mm}

\pars{Invitatorium.} \scriptura{Phil. 4, 4.5}

\vspace{-6mm}

\antiphona{VI}{temporalia/inv-propeestiamsimplex.gtex}

\vfill
\pagebreak

\pars{Hymnus.}

\vspace{-5mm}

\antiphona{II}{temporalia/hym-VeniRedemptor.gtex}
%{
%\vspace{-5mm}
%\setlength{\columnsep}{0pt} % prostor mezi sloupci
%\input{hym-VeniRedemptor-bohtext.tex}
%\setlength{\columnsep}{30pt} % prostor mezi sloupci
%}

\vfill
\pagebreak

\pars{Psalmus 1.} \scriptura{Ps. 61, 2; \textbf{H96}}

\vspace{-4mm}

\antiphona{III g}{temporalia/ant-nonnedeo.gtex}

%\vspace{-5mm}

\scriptura{Ps. 61}

%\vspace{-2mm}

\initiumpsalmi{temporalia/ps61-initium-iii-g-auto.gtex}

%\psalmusEtTranslatioT{temporalia/ps61-V-comb.tex}{10cm}
\input{temporalia/ps61-V.tex} \Abardot{}

\vfill
\pagebreak

\pars{Psalmus 2.} \scriptura{Ps. 65, 4; \textbf{H73}}

\vspace{-4mm}

\antiphona{E}{temporalia/ant-omnisterra.gtex}

%\vspace{-5mm}

\scriptura{Ps. 65, 1-12}

%\initiumpsalmi{temporalia/ps65i-initium-e-auto.gtex}
\initiumpsalmi{temporalia/ps65i-initium-e.gtex}

%\psalmusEtTranslatioT{temporalia/ps65i-V-comb.tex}{10cm}
\input{temporalia/ps65i-V.tex} \Abardot{}

\vfill
\pagebreak

\pars{Psalmus 3.} \scriptura{Ps. 65, 16}

\vspace{-4mm}

\antiphona{I g}{temporalia/ant-auditeomnes.gtex}

%\vspace{-5mm}

\scriptura{Ps. 65, 13-20}

\initiumpsalmi{temporalia/ps65ii-initium-i-g-auto.gtex}

%\psalmusEtTranslatioT{temporalia/ps65ii-V-comb.tex}{10cm}
\input{temporalia/ps65ii-V.tex} \Abardot{}

\vfill
\pagebreak

\pars{Psalmus 4.} \scriptura{Ps. 67, 2}

\vspace{-4mm}

\antiphona{VII a}{temporalia/ant-exsurgatdeus.gtex}

%\vspace{-5mm}

\scriptura{Ps. 67, 2-11}

\initiumpsalmi{temporalia/ps67i-initium-vii-a-auto.gtex}

%\psalmusEtTranslatioT{temporalia/ps67i-V-comb.tex}{10cm}
\input{temporalia/ps67i-V.tex} \Abardot{}

\vfill
\pagebreak

\pars{Psalmus 5.} \scriptura{Ps. 95, 2; \textbf{H98}}

\vspace{-6mm}

\antiphona{II D}{temporalia/ant-cantatedomino.gtex}

\vspace{-3mm}

\scriptura{Ps. 67, 12-24}

\vspace{-2mm}

\initiumpsalmi{temporalia/ps67ii-initium-ii-D-auto.gtex}

%\psalmusEtTranslatioT{temporalia/ps67ii-V-comb.tex}{10cm}
\input{temporalia/ps67ii-V.tex} \Abardot{}

\vfill
\pagebreak

\pars{Psalmus 6.} \scriptura{Ps. 67, 27; \textbf{H96}}

\vspace{-4mm}

\antiphona{D}{temporalia/ant-inecclesiis.gtex}

%\vspace{-5mm}

\scriptura{Ps. 67, 25-36}

\initiumpsalmi{temporalia/ps67iii-initium-d-g2-auto.gtex}

%\psalmusEtTranslatioT{temporalia/ps67iii-V-comb.tex}{10cm}
\input{temporalia/ps67iii-V.tex} \Abardot{}

\vfill
\pagebreak

\pars{Psalmus 7.} \scriptura{Ps. 68, 2}

\vspace{-4mm}

\antiphona{VIII c}{temporalia/ant-salvummefac.gtex}

%\vspace{-5mm}

\scriptura{Ps. 68, 2-14a}

\initiumpsalmi{temporalia/ps68i-initium-viii-c-auto.gtex}

%\psalmusEtTranslatioT{temporalia/ps68i-V-comb.tex}{10cm}
\input{temporalia/ps68i-V.tex}

\vfill

\antiphona{}{temporalia/ant-salvummefac.gtex}

\vfill
\pagebreak

\pars{Psalmus 8.} \scriptura{Ps. 68, 10; \textbf{H178}}

\vspace{-4mm}

\antiphona{VIII c}{temporalia/ant-zelusdomus.gtex}

%\vspace{-5mm}

\scriptura{Ps. 68, 14b-28}

\initiumpsalmi{temporalia/ps68ii-initium-viii-c-auto.gtex}

%\psalmusEtTranslatioT{temporalia/ps68ii-V-comb.tex}{10cm}
\input{temporalia/ps68ii-V.tex}

\vfill

\antiphona{}{temporalia/ant-zelusdomus.gtex}

\vfill
\pagebreak

\pars{Psalmus 9.} \scriptura{Ps. 68, 33; \textbf{H96}}

\vspace{-4mm}

\antiphona{VIII G}{temporalia/ant-quaeritedominum.gtex}

%\vspace{-5mm}

\scriptura{Ps. 68, 29-37}

\initiumpsalmi{temporalia/ps68iii-initium-viii-G-auto.gtex}

%\psalmusEtTranslatioT{temporalia/ps68iii-V-comb.tex}{10cm}
\input{temporalia/ps68iii-V.tex} \Abardot{}

\vfill
\pagebreak

\pars{Versus.} \scriptura{Mc. 1, 3; Is. 40, 3}

% Versus. %%%
\sineinitiali{temporalia/versus-voxclamantis-simplex.gtex}

\vspace{5mm}

\sineinitiali{temporalia/oratiodominica-mat.gtex}

\vspace{5mm}

\pars{Absolutio.}

\cuminitiali{}{temporalia/absolutio-exaudi.gtex}

\vfill
\pagebreak

\cuminitiali{}{temporalia/benedictio-solemn-benedictione.gtex}

\vspace{7mm}

\lectioi

\noindent \Vbardot{} Tu autem, Dómine, miserére nobis.
\noindent \Rbardot{} Deo grátias.

\vfill
\pagebreak

\responsoriumi

\vfill
\pagebreak

\cuminitiali{}{temporalia/benedictio-solemn-unigenitus.gtex}

\vspace{7mm}

\lectioii

\noindent \Vbardot{} Tu autem, Dómine, miserére nobis.
\noindent \Rbardot{} Deo grátias.

\vfill
\pagebreak

\responsoriumii

\vfill
\pagebreak

\cuminitiali{}{temporalia/benedictio-solemn-spiritus.gtex}

\vspace{7mm}

\lectioiii

\noindent \Vbardot{} Tu autem, Dómine, miserére nobis.
\noindent \Rbardot{} Deo grátias.

\vfill
\pagebreak

\responsoriumiii

\vfill
\pagebreak

\rubrica{Reliqua omittuntur, nisi Laudes separandæ sint.}

\pars{Oratio}

\noindent \Vbardot{} Dómine, exáudi oratiónem meam.

\noindent \Rbardot{} Et clamor meus ad te véniat.

\oratio

\vspace{7mm}

\pars{Conclusio}

\noindent \Vbardot{} Dómine, exáudi oratiónem meam.

\noindent \Rbardot{} Et clamor meus ad te véniat.

\noindent \Vbardot{} Benedicámus Dómino.

\noindent \Rbardot{} Deo grátias.

\noindent \Vbardot{} Fidélium ánimæ per misericórdiam Dei requiéscant in pace.

\noindent \Rbardot{} Amen.

\vfill
\pagebreak

\hora{Ad Laudes.} %%%%%%%%%%%%%%%%%%%%%%%%%%%%%%%%%%%%%%%%%%%%%%%%%%%%%
%\sideThumbs{Laudes}

\cantusSineNeumas

\vspace{0.5cm}
\grechangedim{interwordspacetext}{0.18 cm plus 0.15 cm minus 0.05 cm}{scalable}%
\cuminitiali{}{temporalia/deusinadiutorium-communis.gtex}
\grechangedim{interwordspacetext}{0.22 cm plus 0.15 cm minus 0.05 cm}{scalable}%

\vfill
%\pagebreak

\pars{Psalmus 1.} \scriptura{Is. 24, 16}

\vspace{-4mm}

\antiphona{I f}{temporalia/ant-afinibusterrae.gtex}

\scriptura{Psalmus 50.}

\initiumpsalmi{temporalia/ps50-initium-i-f-auto.gtex}

%\psalmusEtTranslatioT{temporalia/ps50-VIII-comb.tex}{10cm}
\input{temporalia/ps50-VIII.tex}

\vfill

\antiphona{}{temporalia/ant-afinibusterrae.gtex}

\vfill
\pagebreak

\pars{Psalmus 2.} \scriptura{\textbf{H38}}

\vspace{-4mm}

\antiphona{I g\textsuperscript{3}}{temporalia/ant-desionveniet.gtex}

\scriptura{Psalmus 87.}

\initiumpsalmi{temporalia/ps87-initium-i-g3-auto.gtex}

%\psalmusEtTranslatioT{temporalia/ps87-VIII-comb.tex}{10cm}
\input{temporalia/ps87-VIII.tex}

\vfill

\antiphona{}{temporalia/ant-desionveniet.gtex}

\vfill
\pagebreak

\pars{Psalmus 3.} \scriptura{Eccli. 36, 18; \textbf{H37}}

\vspace{-4mm}

\antiphona{II* b}{temporalia/ant-damercedemdomine.gtex}

\scriptura{Psalmus 89.}

\initiumpsalmi{temporalia/ps89-initium-ii_-B-auto.gtex}

%\psalmusEtTranslatioT{temporalia/ps89-VIII-comb.tex}{10cm}
\input{temporalia/ps89-VIII.tex}

\vfill

\vspace{-6mm}

\antiphona{}{temporalia/ant-damercedemdomine.gtex} % repeat the antiphon - new page

\vfill
\pagebreak

\pars{Psalmus 4.} \scriptura{Is. 12, 3; \textbf{H37}}

\vspace{-4mm}

\antiphona{V g}{temporalia/ant-haurietisaquas.gtex}

\scriptura{Canticum Moysis, Ex. 15, 1-19}

\initiumpsalmi{temporalia/moysis-initium-v-g-auto.gtex}

%\psalmusEtTranslatioT{temporalia/moysis-comb.tex}{10cm}
\input{temporalia/moysis.tex}

\antiphona{}{temporalia/ant-haurietisaquas.gtex}

\vfill
\pagebreak

\pars{Psalmus 5.} \scriptura{Cf. Ac. 2, 20-21}

\vspace{-4mm}

\antiphona{I g\textsuperscript{3}}{temporalia/ant-venietdiesdomini.gtex}

\scriptura{Psalmus 148.}

\initiumpsalmi{temporalia/ps148-initium-i-g3-auto.gtex}

%\psalmusEtTranslatioT{temporalia/ps148-II-comb.tex}{10cm}
\input{temporalia/ps148-II.tex}

\rubrica{Hic non dicitur Gloria Patri.}

\vfill
\pagebreak

%
\scriptura{Psalmus 149.}

\initiumpsalmi{temporalia/ps149-initium-i-g3-auto.gtex}

%\psalmusEtTranslatioT{temporalia/ps149-II-comb.tex}{10cm}
\input{temporalia/ps149-II.tex}

\rubrica{Hic non dicitur Gloria Patri.}

\vfill
\pagebreak

%
\scriptura{Psalmus 150.}

\initiumpsalmi{temporalia/ps150-initium-i-g3-auto.gtex}

%\psalmusEtTranslatioT{temporalia/ps150-II-comb.tex}{10cm}
\input{temporalia/ps150-II.tex}

\vfill

\vspace{-6mm}

\antiphona{}{temporalia/ant-venietdiesdomini.gtex} % repeat the antiphon - new page

\vfill
\pagebreak

\lectiobrevis

% preklad Jeruz. bible
%\trCapituliI

\vfill

\responsoriumbreve

%\trResp

\vfill
\pagebreak

\pars{Hymnus}

\cuminitiali{D}{temporalia/hym-MagnisProphetae.gtex}
\vspace{-3mm}
%\input{hym-MagnisProphetae-bohtext.tex}

\vfill
%\pagebreak

\pars{Versus.} \scriptura{Mc. 1, 3; Is. 40, 3}

% Versus. %%%
\sineinitiali{temporalia/versus-voxclamantis.gtex}

%\noindent \trVersus

\vfill
\pagebreak

\benedictus

\vfill
\pagebreak

%\sideThumbs{{\scriptsize{}Fine horarum}}

\rubrica{Ante Orationem, cantatur a Superiore:}

\pars{Supplicatio Litaniæ.}

\cuminitiali{}{temporalia/supplicatiolitaniae.gtex}

\pars{Oratio Dominica.}

\cuminitiali{}{temporalia/oratiodominica.gtex}

\vfill
\pagebreak

% Oratio. %%%
\oratio

\vspace{-1mm}
%\trOrationisI

\vfill

\rubrica{Hebdomadarius dicit Dominus vobiscum, vel, absente sacerdote vel diacono, sic concluditur:}

\vspace{2mm}

\antiphona{C}{temporalia/dominusnosbenedicat.gtex}

\rubrica{Postea cantatur a cantore:}

\vspace{2mm}

\cuminitiali{IV}{temporalia/benedicamus-dominica-advequad.gtex}

\vspace{1mm}

\vfill
\pagebreak

\hora{Ad Vesperas.} %%%%%%%%%%%%%%%%%%%%%%%%%%%%%%%%%%%%%%%%%%%%%%%%%%%%%
%\sideThumbs{Vesperæ}

\cantusSineNeumas

\vspace{0.5cm}
\grechangedim{interwordspacetext}{0.18 cm plus 0.15 cm minus 0.05 cm}{scalable}%
\cuminitiali{}{temporalia/deusinadiutorium-communis.gtex}
\grechangedim{interwordspacetext}{0.22 cm plus 0.15 cm minus 0.05 cm}{scalable}%

\vfill
%\pagebreak

\vspace{8mm}

\pars{Psalmus 1.}  \scriptura{Is. 24, 16}

\vspace{-4mm}

\antiphona{I f}{temporalia/ant-afinibusterrae.gtex}

\scriptura{Psalmus 138, 1-13}

\initiumpsalmi{temporalia/ps138i-initium-i-f-auto.gtex}

%\psalmusEtTranslatioT{temporalia/ps138i-VIII-comb.tex}{10cm}
\input{temporalia/ps138i-VIII.tex}

\vfill

\antiphona{}{temporalia/ant-afinibusterrae.gtex}

\vfill
\pagebreak

\pars{Psalmus 2.} \scriptura{\textbf{H38}}

\vspace{-4mm}

\antiphona{I g\textsuperscript{3}}{temporalia/ant-desionveniet.gtex}

\initiumpsalmi{temporalia/ps138ii-initium-i-g3-auto.gtex}

%\psalmusEtTranslatioT{temporalia/ps138ii-VIII-comb.tex}{10cm}
\input{temporalia/ps138ii-VIII.tex} \Abardot{}

\vfill
\pagebreak

\pars{Psalmus 3.} \scriptura{Eccli. 36, 18; \textbf{H37}}

\vspace{-4mm}

\antiphona{II* b}{temporalia/ant-damercedemdomine.gtex}

\vspace{-2mm}

\scriptura{Psalmus 139.}

\vspace{-2mm}

\initiumpsalmi{temporalia/ps139-initium-ii_-B-auto.gtex}

%\psalmusEtTranslatioT{temporalia/ps139-VIII-comb.tex}{10cm}
\input{temporalia/ps139-VIII.tex} \Abardot{}

\vfill
\pagebreak

\pars{Psalmus 4.} \scriptura{Cf. Ac. 2, 20-21}

\vspace{-4mm}

\antiphona{I g\textsuperscript{3}}{temporalia/ant-venietdiesdomini.gtex}

\scriptura{Psalmus 140.}

\initiumpsalmi{temporalia/ps140-initium-i-g3-auto.gtex}

%\psalmusEtTranslatioT{temporalia/ps140-VIII-comb.tex}{10cm}
\input{temporalia/ps140-VIII.tex} \Abardot{}

\vfill
\pagebreak

\pars{Capitulum.} \scriptura{Gen. 49, 10}

\grechangedim{interwordspacetext}{0.12 cm plus 0.15 cm minus 0.05 cm}{scalable}%
\cuminitiali{}{temporalia/capitulum-NosAuferetur.gtex}
\grechangedim{interwordspacetext}{0.22 cm plus 0.15 cm minus 0.05 cm}{scalable}%

% preklad Jeruz. bible
%\trCapituliI

\vfill

\pars{Responsorium breve.} \scriptura{Ps. 84, 8; \textbf{H20}}

\cuminitiali{IV}{temporalia/resp-ostendenobis.gtex}

%\trResp

\vfill
\pagebreak

\pars{Hymnus} \scriptura{Ambrosius (\olddag{} 397)}

\cuminitiali{IV}{temporalia/hym-VerbumSalutis.gtex}
\vspace{-3mm}
%\input{hym-VerbumSalutis-bohtext.tex}

\vfill
%\pagebreak

\pars{Versus.} \scriptura{Is. 45, 8}

% Versus. %%%
\sineinitiali{temporalia/versus-rorate.gtex}

%\noindent \trVersus

\vfill
\pagebreak

\magnificat

\vfill
\pagebreak

%\sideThumbs{{\scriptsize{}Fine horarum}}

\anteOrationem

\pagebreak

% Oratio. %%%
\oratioLaudes

\vspace{-1mm}
%\trOrationisI

\vfill

\rubrica{Hebdomadarius dicit iterum Dominus vobiscum, vel cantor dicit:}

\vspace{2mm}

\sineinitiali{temporalia/domineexaudi.gtex}

\rubrica{Postea cantatur a cantore:}

\vspace{2mm}

\cuminitiali{IV}{temporalia/benedicamus-dominica-advequad.gtex}

\vspace{1mm}

\end{document}

