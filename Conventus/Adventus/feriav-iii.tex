\newcommand{\oratio}{\pars{Oratio.}

\noindent Preces pópuli tui, quǽsumus, Dómine, cleménter exáudi, ut qui de Unigéniti tui in nostra carne advéntu lætántur, cum vénerit in sua maiestáte, ætérnæ vitæ prǽmium consequántur.

\pars{Pro pace in universo mundo.} \scriptura{Sir. 50, 25; 2 Esdr. 4, 20; \textbf{H416}}

\vspace{-4mm}

\antiphona{II D}{temporalia/ant-dapacemdomine.gtex}

\vfill

\noindent Deus, a quo sancta desidéria, recta consília et iusta sunt ópera: da servis tuis illam, quam mundus dare non potest, pacem; ut et corda nostra mandátis tuis dédita, et hóstium subláta formídine, témpora sint tua protectióne tranquílla.

\noindent Per Dóminum nostrum Iesum Christum, Fílium tuum, qui tecum vivit et regnat in unitáte Spíritus Sancti, Deus, per ómnia sǽcula sæculórum.

\noindent \Rbardot{} Amen.}
\newcommand{\matversus}{\noindent \Vbardot{} Audíte verbum Dómini, gentes.

\noindent \Rbardot{} Et annuntiáte illud in fínibus terræ.}
\newcommand{\lectioi}{\pars{Lectio I.} \scriptura{Is. 48, 12-21; 49, 9b-13}

\noindent De libro Isaíæ prophétæ.

\noindent Hæc dicit Dóminus: «Audi me, Iacob, et, Israel, quem ego vocávi; ego, ego primus et ego novíssimus. Manus mea fundávit terram, et déxtera mea expándit cælos; ego voco eos, et stant simul. Congregámini, omnes vos et audíte: Quis de eis annuntiávit hæc? Dóminus diléxit eum; fáciet voluntátem suam in Babylóne et bráchium suum in Chaldǽis.Ego, ego locútus sum et vocávi eum; addúxi eum, et próspera fuit via eius. Accédite ad me et audíte hoc: Non a princípio in abscóndito locútus sum; ex témpore, ántequam fíeret, ibi eram; et nunc Dóminus Deus misit me cum spíritu suo».

\noindent Hæc dicit Dóminus, redémptor tuus, Sanctus Israel: «Ego Dóminus Deus tuus docens te utília, gubérnans te in via, qua ámbulas. Utinam attendísses mandáta mea! Facta fuísset sicut flumen pax tua, et iustítia tua sicut gúrgites maris; et fuísset quasi aréna semen tuum, et stirps úteri tui ut lapílli eius; non interísset et non fuísset attrítum nomen eius a fácie mea. Egredímini de Babylóne, fúgite a Chaldǽis, in voce exsultatiónis annuntiáte; audítum fácite hoc, efférte illud usque ad extréma terræ, dícite: “Redémit Dóminus servum suum Iacob”. Non sitiérunt, cum per desértum dúceret eos; aquam de petra prodúxit eis et scidit petram, et fluxérunt aquæ.

\noindent Super vias pascéntur, et in ómnibus cóllibus decalvátis páscua eórum; non esúrient neque sítient, et non percútiet eos æstus vel sol, quia miserátor eórum reget eos et ad fontes aquárum addúcet eos. Et ponam omnes montes meos in viam, et sémitæ meæ exaltabúntur. Ecce isti de longe vénient, et ecce illi ab aquilóne et mari, et isti de terra Sinim». Laudáte, cæli, et exsúlta, terra; iubiláte, montes, laudem, quia consolátur Dóminus pópulum suum et páuperum suórum miserétur.}
\newcommand{\responsoriumi}{\pars{Responsorium 1.} \scriptura{\Rbar{} Rom. 15, 12; Cf. Ps. 71, 17 \Vbar{} Ps. 49, 2-3; \textbf{H36}}

\vspace{-5mm}

\responsorium{VIII}{temporalia/resp-radixjesse-CROCHU.gtex}{}

\rubrica{vel ad libitum:}

\vspace{3mm}

\pars{Responsorium 2.} \scriptura{\Rbardot{} Is. 49, 13; \textbf{H17}}

\vspace{-5mm}

\responsorium{II}{temporalia/resp-laetenturcaeli-CROCHU.gtex}{}}
\newcommand{\lectioii}{\pars{Lectio II.} \scriptura{Lib. 2, 19. 22-23. 26-27: CCL 14, 39-42}

\noindent Ex Expositióne sancti Ambrósii epíscopi in Lucam.

\noindent Angelus, cum abscóndita nuntiáret, ut fides astruerétur exémplo, senióris féminæ sterilísque concéptum Vírgini Maríæ nuntiávit, ut possíbile Deo omne quod ei placúerit asséreret. Ubi audívit hoc María, non quasi incrédula de oráculo nec quasi incérta de núntio nec quasi dúbitans de exémplo, sed quasi læta pro voto, religiósa pro offício, festína pro gáudio in montána perréxit. Quo enim iam Deo plena nisi ad superióra cum festinatióne conténderet? Nescit tarda molímina Sancti Spíritus grátia. Cito quoque advéntus Maríæ et præséntiæ domínicæ benefícia declarántur; \emph{simul enim ut audívit salutatiónem Maríæ Elísabeth, exsultávit infans in útero eius et repléta est Spíritu Sancto.}

\noindent Vide distinctiónem singulorúmque verbórum proprietátes. Vocem prior Elísabeth audívit, sed Ioánnes prior grátiam sensit: illa natúræ órdine audívit, iste exsultávit ratióne mystérii; illa Maríæ, iste Dómini sensit advéntum, fémina mulíeris et pignus pígnoris; istæ grátiam loquúntur, illi intus operántur pietatísque mystérium matérnis adoriúntur proféctibus duplicíque miráculo prophétant matres spíritu parvulórum.}
\newcommand{\responsoriumii}{\pars{Responsorium 2.} \scriptura{\Rbar{} Is. 16, 1 \Vbar{} Ps. 49, 2; \textbf{H35}}

\vspace{-5mm}

\responsorium{II}{temporalia/resp-emitteagnumdomine-CROCHU.gtex}{}

\rubrica{vel ad libitum:}

\vspace{3mm}

\pars{Responsorium 2.} \scriptura{\Rbardot{} Num. 24, 17 \Vbardot{} Ps. 71, 10; \textbf{H34}}

\vspace{-5mm}

\responsorium{VIII}{temporalia/resp-orieturstella-CROCHU.gtex}{}}
\newcommand{\lectioiii}{\pars{Lectio III.}

\noindent Exsultávit infans, repléta mater est. Non prius mater repléta quam fílius, sed, cum fílius esset replétus Spíritu Sancto, replévit et matrem. Exsultávit Ioánnes, \emph{exsultávit} et Maríæ \emph{spíritus.} Exsultánte Ioánne replétur Elísabeth, Maríam tamen non repléri Spíritu, sed spíritum eius exsultáre cognóvimus —incomprehensíbilis enim incomprehensibíliter operabátur in matre— et illa post concéptum replétur, ista ante concéptum. \emph{Beáta,} inquit, \emph{quæ credidísti.} Sed et vos beáti, qui audístis et credidístis; quæcúmque enim credíderit ánima, et cóncipit et génerat Dei Verbum et ópera eius agnóscit. Sit in síngulis Maríæ ánima, ut magníficet Dóminum; sit in síngulis spíritus Maríæ, ut exsúltet in Deo; si secúndum carnem una mater est Christi, secúndum fidem tamen ómnium fructus est Christus; omnis enim ánima áccipit Dei Verbum, si tamen immaculáta et immúnis a vítiis intemeráto castimóniam pudóre custódiat. Quæcúmque ígitur talis esse potúerit \emph{ánima magníficat Dóminum,} sicut ánima Maríæ magnificávit Dóminum \emph{et exsultávit spíritus eius in Deo salutári.}

\noindent Magnificátur enim Dóminus, sicut et álibi legístis: \emph{Magnificáte Dóminum mecum,} non quo Dómino áliquid humána voce possit adiúngi, sed quia magnificátur in nobis: imágo enim Dei Christus est, et ídeo, si quid iustum religiosúmque fécerit ánima, illam imáginem Dei, ad cuius est similitúdinem creáta, magníficat, et ídeo, dum magníficat eam, magnitúdinis eius quadam participatióne sublímior fit.}
\newcommand{\responsoriumiii}{\pars{Responsorium 3.} \scriptura{\Rbar{} Cf. Is. 35, 2 \Vbar{} Is. 40, 10; \textbf{H35}}

\vspace{-5mm}

\responsorium{I}{temporalia/resp-germinaveruntcampi-CROCHU-cumdox.gtex}{}

\rubrica{vel ad libitum:}

\vspace{3mm}

\pars{Responsorium 3.} \scriptura{\Rbar{} Is. 9, 6 \Vbar{} Cf. Mal. 3, 1; \textbf{H32}}

\vspace{-5mm}

\responsorium{VII}{temporalia/resp-nasceturnobisparvulus-CROCHU-cumdox.gtex}{}}
\newcommand{\laudes}{\pars{Psalmus 1.} \scriptura{Ps. 142, 8-9; \textbf{H39}}

\vspace{-4mm}

\antiphona{II* c}{temporalia/ant-adtedominelevavi.gtex}

%\vspace{-2mm}

\scriptura{Psalmus 86}

%\vspace{-2mm}

\initiumpsalmi{temporalia/ps86-initium-ii_-c-auto.gtex}

%\vspace{-1.5mm}

\input{temporalia/ps86-ii_-c.tex} \Abardot{}

\vfill
\pagebreak

\pars{Psalmus 2.} \scriptura{Eccli. 36, 18; \textbf{H37}}

\vspace{-4mm}

\antiphona{II* b}{temporalia/ant-damercedemdomine.gtex}

%\vspace{-2mm}

\scriptura{Canticum Isaiæ, Is. 40, 10-17}

%\vspace{-3mm}

\initiumpsalmi{temporalia/isaiae9-initium-ii_-B-auto.gtex}

\input{temporalia/isaiae9-ii_-B.tex} \Abardot{}

\vfill
\pagebreak

\pars{Psalmus 3.} \scriptura{\textbf{H38}}

\vspace{-4mm}

\antiphona{VIII G}{temporalia/ant-convertere.gtex}

\scriptura{Psalmus 98}

\initiumpsalmi{temporalia/ps98-initium-viii-g-auto.gtex}

\input{temporalia/ps98-viii-g.tex} \Abardot{}

\vfill
\pagebreak}
\newcommand{\lectiobrevis}{\pars{Lectio Brevis.} \scriptura{Is. 7, 14-15}

\noindent Ecce, Virgo concípiet et páriet fílium et vocábit nomen eius Emmánuel; butýrum et mel cómedet, ut ipse sciat reprobáre malum et elígere bonum.}
\newcommand{\benedictus}{\pars{Canticum Zachariæ.} \scriptura{\textbf{H37}}

\vspace{-4mm}

{
\grechangedim{interwordspacetext}{0.18 cm plus 0.15 cm minus 0.05 cm}{scalable}%
\antiphona{VIII g}{temporalia/ant-nolitetimere.gtex}
\grechangedim{interwordspacetext}{0.22 cm plus 0.15 cm minus 0.05 cm}{scalable}%
}

%\vspace{-2mm}

\scriptura{Lc. 1, 68-79}

%\vspace{-2mm}

\cantusSineNeumas
\initiumpsalmi{temporalia/benedictus-initium-viii-g-auto.gtex}

%\vspace{-1.5mm}

\input{temporalia/benedictus-viii-g.tex} \Abardot{}}
\newcommand{\preces}{\noindent Dóminum nostrum Iesum Christum, fratres caríssimi, exorémus,~\gredagger{} qui in sua misericórdia nos vísitat,~\grestar{} lætámque vocem iterémus:

\Rbardot{} Veni, Dómine Iesu.

\noindent Qui de sinu Patris egréssus,~\gredagger{} venísti ut carnis nostræ vestiméntum indúeres,~\grestar{} líbera quod períerat natúræ vitiátæ contágio.

\Rbardot{} Veni, Dómine Iesu.

\noindent Qui ventúrus, in eléctis agnoscéris gloriósus,\gredagger{} nunc véniens,~\grestar{} in peccatóribus clemens semper et pius inveniáris.

\Rbardot{} Veni, Dómine Iesu.

\noindent Gloriántes in laude tua,~\grestar{} Christe Dómine, vísita nos in salutári tuo.

\Rbardot{} Veni, Dómine Iesu.

\noindent Qui nos iam eduxísti in lucem per fidem,~\grestar{} fac nos iustítia tua pro dignis opéribus tibi plácitos.

\Rbardot{} Veni, Dómine Iesu.}
\newcommand{\vesperas}{\vspace{4mm}

\pars{Psalmus 1.}  \scriptura{Is. 24, 16}

\vspace{-4mm}

\antiphona{I f}{temporalia/ant-afinibusterrae.gtex}

\scriptura{Psalmus 138, 1-13}

\initiumpsalmi{temporalia/ps138i-initium-i-f-auto.gtex}

\input{temporalia/ps138i-i-f.tex}

\vfill

\antiphona{}{temporalia/ant-afinibusterrae.gtex}

\vfill
\pagebreak

\pars{Psalmus 2.} \scriptura{\textbf{H38}}

\vspace{-4mm}

\antiphona{I g\textsuperscript{3}}{temporalia/ant-desionveniet.gtex}

\initiumpsalmi{temporalia/ps138ii-initium-i-g3-auto.gtex}

\input{temporalia/ps138ii-i-g3.tex} \Abardot{}

\vfill
\pagebreak

\pars{Psalmus 3.} \scriptura{Eccli. 36, 18; \textbf{H37}}

\vspace{-4mm}

\antiphona{II* b}{temporalia/ant-damercedemdomine.gtex}

\vspace{-2mm}

\scriptura{Psalmus 139.}

\vspace{-2mm}

\initiumpsalmi{temporalia/ps139-initium-ii_-B-auto.gtex}

\input{temporalia/ps139-ii_-B.tex} \Abardot{}

\vfill
\pagebreak

\pars{Psalmus 4.} \scriptura{Cf. Ac. 2, 20-21}

\vspace{-4mm}

\antiphona{I g\textsuperscript{3}}{temporalia/ant-venietdiesdomini.gtex}

\scriptura{Psalmus 140.}

\initiumpsalmi{temporalia/ps140-initium-i-g3-auto.gtex}

\input{temporalia/ps140-i-g3.tex} \Abardot{}

\vfill
\pagebreak}
\newcommand{\magnificat}{\pars{Canticum B. Mariæ V.} \scriptura{Sap. 7, 26; Mal. 4, 2; Lc. 1, 78-79; \textbf{H40}}

\vspace{-5mm}

{
\grechangedim{interwordspacetext}{0.18 cm plus 0.15 cm minus 0.05 cm}{scalable}%
\antiphona{II D}{temporalia/ant-ooriens.gtex}
\grechangedim{interwordspacetext}{0.22 cm plus 0.15 cm minus 0.05 cm}{scalable}%
}

\vspace{-2mm}

\scriptura{Lc. 1, 46-55}

\vspace{-2mm}

\cantusSineNeumas

\initiumpsalmi{temporalia/magnificat-initium-iisoll-D.gtex}

\vspace{-1.5mm}

\input{temporalia/magnificat-iisoll-D.tex} \Abardot{}

\vspace{-1cm}}
\newcommand{\benedicamuslaudes}{
\cuminitiali{IV}{temporalia/benedicamus-feria-advequad.gtex}

\vfill
\pagebreak

\hora{Pro defunctæ Mariæ-Paulæ.}

\cantusSineNeumas

\pars{Responsorium.} \scriptura{Cf. Lc. 16, 22; \textbf{H389}}

\vspace{-7mm}

\responsorium{IV}{temporalia/resp-subvenite-gn.gtex}{}

\vspace{1cm}

\noindent Tibi, Dómine, commendámus ánimam fámulæ tuæ Maríæ-Paulæ, ut defúncta sǽculo, tibi vivat, et, quæ per fragilitátem humánæ conversatiónis peccáta commísit, tu vénia misericordíssimæ pietátis abstérge.

\noindent Per Christum Dóminum nostrum.

\Rbardot{} Amen.

\noindent Fidélium ánimæ per misericórdiam Dei requiéscant in pace. \Rbardot{} Amen.

\rubrica{Post Orationem dicitur (semper plurali numero):}

\noindent \Vbardot{} Réquiem ætérnam dona eis Dómine. \Rbardot{} Et lux perpétua lúceat eis.

\vspace{1cm}
\rubrica{Deinde Cantores:}
\vspace{2mm}

\sineinitiali{temporalia/requiescant.gtex}
}
\newcommand{\hebdomada}{infra Hebdom. III Adventus.}
\newcommand{\oratioLaudes}{\cuminitiali{}{temporalia/oratio3vo.gtex}}
\newcommand{\responsoriumbreve}{\pars{Responsorium breve.} \scriptura{Is. 60, 2; \textbf{H20}}

\cuminitiali{IV}{temporalia/resp-superte.gtex}}

\renewcommand{\hebdomada}{infra Hebdom. Ultima Adventus.}
\ifx\invitatorium\undefined
\newcommand{\invitatorium}{\pars{Invitatorium.} \scriptura{Phil. 4, 4.5}

\vspace{-6mm}

\antiphona{VI}{temporalia/inv-propeestiamsimplex.gtex}}
\fi
\ifx\hymnusmatutinum\undefined
\newcommand{\hymnusmatutinum}{\pars{Hymnus.}

\vspace{-5mm}

\antiphona{II}{temporalia/hym-VeniRedemptor.gtex}}
\fi
\ifx\hymnuslaudes\undefined
\newcommand{\hymnuslaudes}{\pars{Hymnus}

\cuminitiali{D}{temporalia/hym-MagnisProphetae.gtex}}
\fi
\ifx\hymnusvesperas\undefined
\newcommand{\hymnusvesperas}{\pars{Hymnus}

\cuminitiali{IV}{temporalia/hym-VerbumSalutis.gtex}}
\fi

% LuaLaTeX

\documentclass[a4paper, twoside, 12pt]{article}
\usepackage[latin]{babel}
%\usepackage[landscape, left=3cm, right=1.5cm, top=2cm, bottom=1cm]{geometry} % okraje stranky
%\usepackage[landscape, a4paper, mag=1166, truedimen, left=2cm, right=1.5cm, top=1.6cm, bottom=0.95cm]{geometry} % okraje stranky
\usepackage[landscape, a4paper, mag=1400, truedimen, left=0.5cm, right=0.5cm, top=0.5cm, bottom=0.5cm]{geometry} % okraje stranky

\usepackage{fontspec}
\setmainfont[FeatureFile={junicode.fea}, Ligatures={Common, TeX}, RawFeature=+fixi]{Junicode}
%\setmainfont{Junicode}

% shortcut for Junicode without ligatures (for the Czech texts)
\newfontfamily\nlfont[FeatureFile={junicode.fea}, Ligatures={Common, TeX}, RawFeature=+fixi]{Junicode}

\usepackage{multicol}
\usepackage{color}
\usepackage{lettrine}
\usepackage{fancyhdr}

% usual packages loading:
\usepackage{luatextra}
\usepackage{graphicx} % support the \includegraphics command and options
\usepackage{gregoriotex} % for gregorio score inclusion
\usepackage{gregoriosyms}
\usepackage{wrapfig} % figures wrapped by the text
\usepackage{parcolumns}
\usepackage[contents={},opacity=1,scale=1,color=black]{background}
\usepackage{tikzpagenodes}
\usepackage{calc}
\usepackage{longtable}
\usetikzlibrary{calc}

\setlength{\headheight}{14.5pt}

\input{conventuscommune.tex} % Often used macros

\newcommand{\annusEditionis}{2021}

%%%% Vicekrat opakovane kousky

\newcommand{\anteOrationem}{
  \rubrica{Ante Orationem, cantatur a Superiore:}

  \pars{Supplicatio Litaniæ.}

  \cuminitiali{}{temporalia/supplicatiolitaniae.gtex}

  \pars{Oratio Dominica.}

  \cuminitiali{}{temporalia/oratiodominica.gtex}

  \rubrica{Deinde dicitur ab Hebdomadario:}

  \cuminitiali{}{temporalia/dominusvobiscum-solemnis.gtex}

  \rubrica{In choro monialium loco Dominus vobiscum dicitur:}

  \sineinitiali{temporalia/domineexaudi.gtex}
}

\setlength{\columnsep}{30pt} % prostor mezi sloupci

%%%%%%%%%%%%%%%%%%%%%%%%%%%%%%%%%%%%%%%%%%%%%%%%%%%%%%%%%%%%%%%%%%%%%%%%%%%%%%%%%%%%%%%%%%%%%%%%%%%%%%%%%%%%%
\begin{document}

% Here we set the space around the initial.
% Please report to http://home.gna.org/gregorio/gregoriotex/details for more details and options
\grechangedim{afterinitialshift}{2.2mm}{scalable}
\grechangedim{beforeinitialshift}{2.2mm}{scalable}
\grechangedim{interwordspacetext}{0.22 cm plus 0.15 cm minus 0.05 cm}{scalable}%
\grechangedim{annotationraise}{-0.2cm}{scalable}

% Here we set the initial font. Change 38 if you want a bigger initial.
% Emit the initials in red.
\grechangestyle{initial}{\color{red}\fontsize{38}{38}\selectfont}

\pagestyle{empty}

%%%% Titulni stranka
\begin{titulusOfficii}
\ifx\titulus\undefined
\nomenFesti{Feria V \hebdomada{}}
\else
\titulus
\fi
\end{titulusOfficii}

\vfill

\begin{center}
%Ad usum et secundum consuetudines chori \guillemotright{}Conventus Choralis\guillemotleft.

%Editio Sancti Wolfgangi \annusEditionis
\end{center}

\scriptura{}

\pars{}

\pagebreak

\renewcommand{\headrulewidth}{0pt} % no horiz. rule at the header
\fancyhf{}
\pagestyle{fancy}

\cantusSineNeumas

\ifx\oratio\undefined
\ifx\lauda\undefined
\else
\newcommand{\oratio}{\pars{Oratio.}

\noindent Omnípotens sempitérne Deus, véspere, mane et merídie maiestátem tuam supplíciter deprecámur, ut, expúlsis de córdibus nostris peccatórum ténebris, ad veram lucem, quæ Christus est, nos fácias perveníre.

\noindent Qui tecum vivit et regnat in unitáte Spíritus Sancti, Deus, per ómnia sǽcula sæculórum.

\noindent \Rbardot{} Amen.}
\fi
\fi

\hora{Ad Matutinum.} %%%%%%%%%%%%%%%%%%%%%%%%%%%%%%%%%%%%%%%%%%%%%%%%%%%%%
%\sideThumbs{Matutinum}

\vspace{2mm}

\cuminitiali{}{temporalia/dominelabiamea.gtex}

\vfill
%\pagebreak

\vspace{2mm}

\ifx\invitatorium\undefined
\pars{Invitatorium.} \scriptura{Ps. 94, 6; Psalmus 94; \textbf{H136}}

\vspace{-6mm}

\antiphona{E}{temporalia/inv-adoremusdominum.gtex}
\else
\invitatorium
\fi

\vfill
\pagebreak

\ifx\hymnusmatutinum\undefined
\ifx\matuac\undefined
\else
\pars{Hymnus.} \scriptura{Gregorius Magnus (+604)}

{
\grechangedim{interwordspacetext}{0.10 cm plus 0.15 cm minus 0.05 cm}{scalable}%
\antiphona{IV}{temporalia/hym-NoxAtra.gtex}
\grechangedim{interwordspacetext}{0.22 cm plus 0.15 cm minus 0.05 cm}{scalable}%
}
\fi
\else
\hymnusmatutinum
\fi

\vspace{-3mm}

\vfill
\pagebreak

\ifx\matua\undefined
\else
% MAT A
\pars{Psalmus 1.} \scriptura{Ps. 17, 3; \textbf{H99}}

\vspace{-4mm}

\antiphona{VIII G}{temporalia/ant-dominusfirmamentum.gtex}

%\vspace{-2mm}

\scriptura{Ps. 17, 31-35}

%\vspace{-2mm}

\initiumpsalmi{temporalia/ps17xxxi_xxxv-initium-viii-G-auto.gtex}

\input{temporalia/ps17xxxi_xxxv-viii-G.tex} \Abardot{}

\vfill
\pagebreak

\pars{Psalmus 2.} \scriptura{Ps. 62, 9; \textbf{H393}}

\vspace{-4mm}

\antiphona{VII c trans.}{temporalia/ant-mesuscepit.gtex}

%\vspace{-2mm}

\scriptura{Ps. 17, 36-46}

%\vspace{-2mm}

\initiumpsalmi{temporalia/ps17xxxvi_xlvi-initium-vii-c-trans.gtex}

\input{temporalia/ps17xxxvi_xlvi-vii-c.tex} \Abardot{}

\vfill
\pagebreak

\pars{Psalmus 3.} \scriptura{Ps. 17, 47; \textbf{H100}}

\vspace{-4mm}

\antiphona{VII c\textsuperscript{2}}{temporalia/ant-vivitdominus.gtex}

%\vspace{-2mm}

\scriptura{Ps. 17, 47-51}

%\vspace{-2mm}

\initiumpsalmi{temporalia/ps17xlvii_li-initium-vii-c2-auto.gtex}

\input{temporalia/ps17xlvii_li-vii-c2.tex} \Abardot{}

\vfill
\pagebreak
\fi
\ifx\matuc\undefined
\else
% MAT C
\pars{Psalmus 1.} \scriptura{Lam. 1, 21; \textbf{H177}}

\vspace{-4mm}

\antiphona{VII a}{temporalia/ant-omnesinimici.gtex}

%\vspace{-2mm}

\scriptura{Ps. 88, 39-46}

%\vspace{-2mm}

\initiumpsalmi{temporalia/ps88xxxix_xlvi-initium-vii-a-auto.gtex}

\input{temporalia/ps88xxxix_xlvi-vii-a.tex} \Abardot{}

\vfill
\pagebreak

\pars{Psalmus 2.} \scriptura{Ps. 88, 53; \textbf{H98}}

\vspace{-4mm}

\antiphona{VI F}{temporalia/ant-benedictusdominusinaeternum.gtex}

%\vspace{-2mm}

\scriptura{Ps. 88, 47-53}

%\vspace{-2mm}

\initiumpsalmi{temporalia/ps88xlvii_liii-initium-vi-F-auto.gtex}

\input{temporalia/ps88xlvii_liii-vi-F.tex} \Abardot{}

\vfill
\pagebreak

\pars{Psalmus 3.} \scriptura{Ps. 89, 13}

\vspace{-4mm}

\antiphona{I g}{temporalia/ant-converteredomine.gtex}

%\vspace{-2mm}

\scriptura{Ps. 89}

%\vspace{-2mm}

\initiumpsalmi{temporalia/ps89-initium-i-g-auto.gtex}

\input{temporalia/ps89-i-g.tex}

\vfill

\antiphona{}{temporalia/ant-converteredomine.gtex}

\vfill
\pagebreak
\fi

\pars{Versus.}

\ifx\matversus\undefined
\ifx\matua\undefined
\else
\noindent \Vbardot{} Révela, Dómine, óculos meos.

\noindent \Rbardot{} Et considerábo mirabília de lege tua.
\fi
\ifx\matuc\undefined
\else
\noindent \Vbardot{} Audies de ore meo verbum.

\noindent \Rbardot{} Et annuntiábis eis ex me.
\fi
\else
\matversus
\fi

\vspace{5mm}

\sineinitiali{temporalia/oratiodominica-mat.gtex}

\vspace{5mm}

\pars{Absolutio.}

\cuminitiali{}{temporalia/absolutio-exaudi.gtex}

\vfill
\pagebreak

\cuminitiali{}{temporalia/benedictio-solemn-benedictione.gtex}

\vspace{7mm}

\lectioi

\noindent \Vbardot{} Tu autem, Dómine, miserére nobis.
\noindent \Rbardot{} Deo grátias.

\vfill
\pagebreak

\responsoriumi

\vfill
\pagebreak

\cuminitiali{}{temporalia/benedictio-solemn-unigenitus.gtex}

\vspace{7mm}

\lectioii

\noindent \Vbardot{} Tu autem, Dómine, miserére nobis.
\noindent \Rbardot{} Deo grátias.

\vfill
\pagebreak

\responsoriumii

\vfill
\pagebreak

\cuminitiali{}{temporalia/benedictio-solemn-spiritus.gtex}

\vspace{7mm}

\lectioiii

\noindent \Vbardot{} Tu autem, Dómine, miserére nobis.
\noindent \Rbardot{} Deo grátias.

\vfill
\pagebreak

\responsoriumiii

\vfill
\pagebreak

\rubrica{Reliqua omittuntur, nisi Laudes separandæ sint.}

\sineinitiali{temporalia/domineexaudi.gtex}

\vfill

\oratio

\vfill

\noindent \Vbardot{} Dómine, exáudi oratiónem meam.
\Rbardot{} Et clamor meus ad te véniat.

\vfill

\noindent \Vbardot{} Benedicámus Dómino.
\noindent \Rbardot{} Deo grátias.

\vfill

\noindent \Vbardot{} Fidélium ánimæ per misericórdiam Dei requiéscant in pace.
\Rbardot{} Amen.

\vfill
\pagebreak

\hora{Ad Laudes.} %%%%%%%%%%%%%%%%%%%%%%%%%%%%%%%%%%%%%%%%%%%%%%%%%%%%%
%\sideThumbs{Laudes}

\cantusSineNeumas

\vspace{0.5cm}
\grechangedim{interwordspacetext}{0.18 cm plus 0.15 cm minus 0.05 cm}{scalable}%
\cuminitiali{}{temporalia/deusinadiutorium-communis.gtex}
\grechangedim{interwordspacetext}{0.22 cm plus 0.15 cm minus 0.05 cm}{scalable}%

\vfill
\pagebreak

\ifx\hymnuslaudes\undefined
\ifx\laudac\undefined
\else
\pars{Hymnus}

\grechangedim{interwordspacetext}{0.16 cm plus 0.15 cm minus 0.05 cm}{scalable}%
\cuminitiali{I}{temporalia/hym-SolEcce.gtex}
\grechangedim{interwordspacetext}{0.22 cm plus 0.15 cm minus 0.05 cm}{scalable}%
\vspace{-3mm}
\fi
\else
\hymnuslaudes
\fi

\vfill
\pagebreak

\ifx\lauda\undefined
\else
\pars{Psalmus 1.}

\vspace{-4mm}

\antiphona{VIII G}{temporalia/ant-exsurgamdiluculo.gtex}

%\vspace{-2mm}

\scriptura{Psalmus 56}

%\vspace{-2mm}

\initiumpsalmi{temporalia/ps56-initium-viii-g-auto.gtex}

%\vspace{-1.5mm}

\input{temporalia/ps56-viii-g.tex} \Abardot{}

\vfill
\pagebreak

\pars{Psalmus 2.} \scriptura{Ier. 31, 14}

\vspace{-4mm}

\antiphona{IV* e}{temporalia/ant-populusmeusait.gtex}

%\vspace{-2mm}

\scriptura{Canticum Ieremiæ, 1 Ier. 31, 10-14}

%\vspace{-3mm}

\initiumpsalmi{temporalia/jeremiae3-initium-iv_-e-auto.gtex}

\input{temporalia/jeremiae3-iv_-e.tex} \Abardot{}

\vfill
\pagebreak

\pars{Psalmus 3.} \scriptura{Ps. 95, 4; \textbf{H94}}

\vspace{-4mm}

\antiphona{IV a}{temporalia/ant-magnusdominus.gtex}

\scriptura{Psalmus 47}

\initiumpsalmi{temporalia/ps47-initium-iv-a-auto.gtex}

\input{temporalia/ps47-iv-a.tex} \Abardot{}

\vfill
\pagebreak
\fi
\ifx\laudc\undefined
\else
\pars{Psalmus 1.} \scriptura{Ps. 86, 1; \textbf{H98}}

\vspace{-4mm}

\antiphona{I g}{temporalia/ant-fundamentaeius.gtex}

%\vspace{-2mm}

\scriptura{Psalmus 86}

%\vspace{-2mm}

\initiumpsalmi{temporalia/ps86-initium-i-g-auto.gtex}

%\vspace{-1.5mm}

\input{temporalia/ps86-i-g.tex} \Abardot{}

\vfill
\pagebreak

\pars{Psalmus 2.}

\vspace{-4mm}

\antiphona{II D}{temporalia/ant-eccedominusnosterbrachio.gtex}

%\vspace{-2mm}

\scriptura{Canticum Isaiæ, Is. 40, 10-17}

%\vspace{-3mm}

\initiumpsalmi{temporalia/isaiae9-initium-ii-D-auto.gtex}

\input{temporalia/isaiae9-ii-D.tex} \Abardot{}

\vfill
\pagebreak

\pars{Psalmus 3.} \scriptura{Ps. 144, 17}

\vspace{-4mm}

\antiphona{E}{temporalia/ant-iustusetsanctus.gtex}

\scriptura{Psalmus 98}

\initiumpsalmi{temporalia/ps98-initium-e.gtex}

\input{temporalia/ps98-e.tex} \Abardot{}

\vfill
\pagebreak
\fi

\ifx\lectiobrevis\undefined
\ifx\lauda\undefined
\else
\pars{Lectio Brevis.} \scriptura{Is. 66, 1-2}

\noindent Hæc dicit Dóminus: Cælum thronus meus, terra autem scabéllum pedum meórum. Quæ ista domus, quam ædificábitis mihi, et quis iste locus quiétis meæ? Omnia hæc manus mea fecit et mea sunt univérsa ista, dicit Dóminus. Ad hunc autem respíciam, ad paupérculum et contrítum spíritu et treméntem sermónes meos.
\fi
\else
\lectiobrevis
\fi

\vfill

\ifx\responsoriumbreve\undefined
\ifx\laudac\undefined
\else
\pars{Responsorium breve.} \scriptura{Ps. 118, 145}

\cuminitiali{VI}{temporalia/resp-clamaviintotocorde.gtex}
\fi
\else
\responsoriumbreve
\fi

\vfill
\pagebreak

\ifx\benedictus\undefined
\ifx\laudac\undefined
\else
\pars{Canticum Zachariæ.} \scriptura{Lc. 1, 74.75; \textbf{H423}}

%\vspace{-4mm}

{
\grechangedim{interwordspacetext}{0.18 cm plus 0.15 cm minus 0.05 cm}{scalable}%
\antiphona{VII a}{temporalia/ant-insanctitate.gtex}
\grechangedim{interwordspacetext}{0.22 cm plus 0.15 cm minus 0.05 cm}{scalable}%
}

%\vspace{-3mm}

\scriptura{Lc. 1, 68-79}

%\vspace{-2mm}

\cantusSineNeumas
\initiumpsalmi{temporalia/benedictus-initium-vii-a-auto.gtex}

%\vspace{-1.5mm}

\input{temporalia/benedictus-vii-a.tex} \Abardot{}
\fi
\else
\benedictus
\fi

\vspace{-1cm}

\vfill
\pagebreak

%\sideThumbs{{\scriptsize{}Fine horarum}}

\pars{Preces.}

\sineinitiali{}{temporalia/tonusprecum.gtex}

\ifx\preces\undefined
\ifx\lauda\undefined
\else
\noindent Grátias agámus Christo, qui lumen huius diéi nobis concédit, \gredagger{} et ad eum clamémus:

\Rbardot{} Bénedic et sanctífica nos, Dómine.

\noindent Qui te pro peccátis nostris hóstiam obtulísti, \gredagger{} incépta et propósita suscípias hodiérna.

\Rbardot{} Bénedic et sanctífica nos, Dómine.

\noindent Qui óculos nostros lucis dono lætíficas novæ, \gredagger{} lúcifer oriáris in córdibus nostris.

\Rbardot{} Bénedic et sanctífica nos, Dómine.

\noindent Tríbue hódie nos esse ómnibus longánimes, \gredagger{} ut imitatóres tui fíeri possímus.

\Rbardot{} Bénedic et sanctífica nos, Dómine.

\noindent Audítam, Dómine, fac nobis mane misericórdiam tuam. \gredagger{} Sit hódie gáudium tuum fortitúdo nostra.

\Rbardot{} Bénedic et sanctífica nos, Dómine.
\fi
\ifx\laudc\undefined
\else
\noindent Christo, bono pastóri, qui pro suis óvibus ánimam pósuit, \gredagger{} laudes grati exsolvámus et supplicémus, dicéntes:

\Rbardot{} Pasce pópulum tuum, Dómine.

\noindent Christe, qui in sanctis pastóribus misericórdiam et dilectiónem tuam dignátus es osténdere, \gredagger{} numquam désinas per eos nobíscum misericórditer ágere.

\Rbardot{} Pasce pópulum tuum, Dómine.

\noindent Qui múnere pastóris animárum fungi per tuos vicários pergis, \gredagger{} ne destíteris nos ipse per rectóres nostros dirígere.

\Rbardot{} Pasce pópulum tuum, Dómine.

\noindent Qui in sanctis tuis, populórum dúcibus, córporum animarúmque médicus exstitísti, \gredagger{} numquam cesses ministérium in nos vitæ et sanctitátis perágere.

\Rbardot{} Pasce pópulum tuum, Dómine.

\noindent Qui, prudéntia et caritáte sanctórum, tuum gregem erudísti, \gredagger{} nos in sanctitáte iúgiter per pastóres nostros ædífica.

\Rbardot{} Pasce pópulum tuum, Dómine.
\fi
\else
\preces
\fi

\vfill

\pars{Oratio Dominica.}

\cuminitiali{}{temporalia/oratiodominicaalt.gtex}

\vfill
\pagebreak

\rubrica{vel:}

\pars{Supplicatio Litaniæ.}

\cuminitiali{}{temporalia/supplicatiolitaniae.gtex}

\vfill

\pars{Oratio Dominica.}

\cuminitiali{}{temporalia/oratiodominica.gtex}

\vfill
\pagebreak

% Oratio. %%%
\oratio

\vspace{-1mm}

\vfill

\rubrica{Hebdomadarius dicit Dominus vobiscum, vel, absente sacerdote vel diacono, sic concluditur:}

\vspace{2mm}

\antiphona{C}{temporalia/dominusnosbenedicat.gtex}

\rubrica{Postea cantatur a cantore:}

\vspace{2mm}

\cuminitiali{IV}{temporalia/benedicamus-feria-laudes.gtex}

\vspace{1mm}

\vfill
\pagebreak

\end{document}

