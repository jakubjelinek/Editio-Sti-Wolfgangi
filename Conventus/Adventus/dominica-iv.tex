\newcommand{\titulus}{\nomenFesti{Dominica IV Adventus.}}
\newcommand{\oratio}{\pars{Oratio.}

\noindent Festína, quǽsumus, ne tardáveris, Dómine Iesu,~\gredagger{} ut advéntus tui consolatiónibus sublevéntur,~\grestar{} qui in tua pietáte confídunt.

\pars{Pro pace in universo mundo.} \scriptura{Sir. 50, 25; 2 Esdr. 4, 20; \textbf{H416}}

\vspace{-4mm}

\antiphona{II D}{temporalia/ant-dapacemdomine.gtex}

\vfill

\noindent Deus, a quo sancta desidéria, recta consília et iusta sunt ópera: da servis tuis illam, quam mundus dare non potest, pacem; ut et corda nostra mandátis tuis dédita, et hóstium subláta formídine, témpora sint tua protectióne tranquílla.

\noindent Per Dóminum nostrum Iesum Christum, Fílium tuum, qui tecum vivit et regnat in unitáte Spíritus Sancti, Deus, per ómnia sǽcula sæculórum.

\noindent \Rbardot{} Amen.}
\newcommand{\vesperasi}{\cantusSineNeumas

\vspace{0.5cm}
\grechangedim{interwordspacetext}{0.18 cm plus 0.15 cm minus 0.05 cm}{scalable}%
\cuminitiali{}{temporalia/deusinadiutorium-communis.gtex}
\grechangedim{interwordspacetext}{0.22 cm plus 0.15 cm minus 0.05 cm}{scalable}%

\vfill
\pagebreak

\pars{Psalmus 1.} \scriptura{Ioel 2, 1; \textbf{H33}}

\vspace{-4mm}

\antiphona{I g}{temporalia/ant-canitetuba.gtex}

\scriptura{Psalmus 144, 10-21.}

\initiumpsalmi{temporalia/ps144ii-initium-i-g-auto.gtex}

\input{temporalia/ps144ii-i-g.tex}

\vfill

\antiphona{}{temporalia/ant-canitetuba.gtex}

\vspace{-1cm}

\vfill
\pagebreak

\pars{Psalmus 2.} \scriptura{Ag. 2, 7; \textbf{H33}}

\vspace{-4mm}

\antiphona{I f}{temporalia/ant-eccevenietdesideratus.gtex}

\scriptura{Psalmus 145.}

\initiumpsalmi{temporalia/ps145-initium-i-f-auto.gtex}

\input{temporalia/ps145-i-f.tex} \Abardot{}

\vfill
\pagebreak

\pars{Psalmus 3.} \scriptura{Is. 40, 4; \textbf{H33}}

\vspace{-4mm}

\antiphona{I g}{temporalia/ant-eruntprava.gtex}

%\vspace{-2mm}

\scriptura{Psalmus 146.}

%\vspace{-2mm}

\initiumpsalmi{temporalia/ps146-initium-i-g-auto.gtex}

%\vspace{-1.5mm}

\input{temporalia/ps146-i-g.tex} \Abardot{}

\vfill
\pagebreak

\pars{Psalmus 4.} \scriptura{Sap. 18, 15; \textbf{H33}}

\vspace{-4mm}

\antiphona{II D}{temporalia/ant-omnipotenssermotuus.gtex}

\scriptura{Psalmus 147.}

\initiumpsalmi{temporalia/ps147-initium-ii-D-auto.gtex}

\input{temporalia/ps147-ii-D.tex} \Abardot{}

\vfill
\pagebreak

\pars{Capitulum.} \scriptura{1 Cor. 4, 1-2}

\cuminitiali{}{temporalia/capitulum-FratresSic.gtex}}
\newcommand{\hymnusvesperas}{\pars{Hymnus.} \scriptura{Ambrosius (\olddag{} 397)}

\vspace{-5mm}

\antiphona{IV}{temporalia/hym-VerbumSalutis.gtex}}
\newcommand{\magnificati}{\pars{Canticum B. Mariæ V.} \scriptura{Is. 7, 14; 33, 22; Gn. 49, 10; \textbf{H41}}

\vspace{-6.5mm}

{
\grechangedim{interwordspacetext}{0.18 cm plus 0.15 cm minus 0.05 cm}{scalable}%
\antiphona{II D}{temporalia/ant-oemmanuel.gtex}
\grechangedim{interwordspacetext}{0.22 cm plus 0.15 cm minus 0.05 cm}{scalable}%
}

\vspace{-3mm}

\scriptura{Lc. 1, 46-55}

\vspace{-2mm}

\cantusSineNeumas

\initiumpsalmi{temporalia/magnificat-initium-iisoll-D.gtex}

\vspace{-1.5mm}

\input{temporalia/magnificat-iisoll-D.tex} \Abardot{}

\vspace{-1cm}}
\newcommand{\invitatorium}{\pars{Invitatorium.} \scriptura{\textbf{H41}}

\vspace{-6mm}

\antiphona{E}{temporalia/inv-hodiescietis.gtex}}
\newcommand{\hymnusmatutinum}{\pars{Hymnus.}

\vspace{-5mm}

\antiphona{II}{temporalia/hym-VeniRedemptor.gtex}}
\newcommand{\nocturnoii}{\vspace{-4mm}

\pars{Psalmus 4.} \scriptura{2. Par. 20, 17}

\vspace{-4mm}

\antiphona{VIII G}{temporalia/ant-iudaeaetierusalem.gtex}

\vspace{-2mm}

\scriptura{Ps. 23}

%\vspace{-2mm}

\initiumpsalmi{temporalia/ps23-initium-viii-G-auto.gtex}

\input{temporalia/ps23-viii-G.tex} \Abardot{}

\vfill
\pagebreak

\pars{Psalmus 2.} \scriptura{Cf. Ex. 16, 6-7; \textbf{H42}}

\vspace{-4mm}

\antiphona{VIII a}{temporalia/ant-hodiescietis.gtex}

\vspace{-3mm}

\scriptura{Ps. 65, 1-12}

\vspace{-1mm}

\initiumpsalmi{temporalia/ps65i-initium-viii-a-auto.gtex}

\vspace{-1.5mm}

\input{temporalia/ps65i-viii-a.tex} \Abardot{}

\vspace{-5mm}

\vfill
\pagebreak

\pars{Psalmus 3.} \scriptura{Cf. Dt. 32, 2; \textbf{H43}}

\vspace{-4mm}

\antiphona{II* b}{temporalia/ant-exspectetur.gtex}

%\vspace{-5mm}

\scriptura{Ps. 65, 13-20}

%\vspace{-2mm}

\initiumpsalmi{temporalia/ps65ii-initium-ii_-B-auto.gtex}

\input{temporalia/ps65ii-ii_-B.tex} \Abardot{}

\vfill
\pagebreak}
\newcommand{\lectioi}{\pars{Lectio I.} \scriptura{Is. 51, 17-20}

\noindent De libro Isaíæ prophétæ.

\noindent Eleváre, eleváre, consúrge, Ierúsalem, quæ bibísti de manu Dómini cálicem iræ eius; póculum sopóris bibísti, epotásti.

\noindent Non est qui susténtet eam ex ómnibus fíliis, quos génuit;et non est qui apprehéndat manum eius ex ómnibus fíliis, quos enutrívit.

\noindent Duo sunt quæ occurrérunt tibi; quis contristábitur super te? Vástitas et contrítio et fames et gládius; quis consolábitur te?

\noindent Fílii tui defecérunt, iácent in cápite ómnium viárum sicut oryx illaqueátus, pleni indignatióne Dómini, increpatióne Dei tui.}
\newcommand{\responsoriumi}{\pars{Responsorium 1.} \scriptura{\Rbardot{} Cf. Ioel 2, 1 \Vbardot{} Ps. 106, 3; \textbf{H31}}

\vspace{-5mm}

\responsorium{I}{temporalia/resp-canitetuba-CROCHU.gtex}{}}
\newcommand{\lectioii}{\pars{Lectio II.} \scriptura{Is. 51, 21-23; 52, 1-2.7-10}

\noindent Idcírco audi hoc, paupércula et ébria, sed non a vino.

\noindent Hæc dicit dominátor tuus, Dóminus et Deus tuus, qui conténdit pro pópulo suo: «Ecce tuli de manu tua cálicem sopóris, póculum indignatiónis meæ; non adícies, ut bibas illum ultra. Et ponam illum in manu eórum, qui te humiliavérunt et dixérunt tibi: “Incurváre, ut transeámus”; et ponébas ut terram dorsum tuum et quasi viam transeúntibus».

\noindent Consúrge, consúrge, indúere fortitúdine tua, Sion; indúere vestiméntis glóriæ tuæ, Ierúsalem, cívitas sanctitátis, quia non adíciet ultra ut pertránseat per te incircumcísus et immúndus. Excútere de púlvere, consúrge, captíva Ierúsalem; solve víncula colli tui, captíva fília Sion.

\noindent Quam pulchri super montes pedes annuntiántis, prædicántis pacem, annuntiántis bonum, prædicántis salútem, dicéntis Sion: «Regnávit Deus tuus!».}
\newcommand{\responsoriumii}{\pars{Responsorium 2.} \scriptura{\Rbar{} Cf. E. 33, 1 \Vbar{} Lev. 11, 44; \textbf{H31}}

\vspace{-5mm}

\responsorium{VII}{temporalia/resp-vigesimaquarta-CROCHU.gtex}{}}
\newcommand{\lectioiii}{\pars{Lectio III.} \scriptura{Is. 52, 8-10}

\noindent Vox speculatórum tuórum: levavérunt vocem, simul exsultábunt, quia óculo ad óculum vidébunt, cum redíerit Dóminus ad Sion.

\noindent Gaudéte et exsultáte simul, desérta Ierúsalem, quia consolátus est Dóminus pópulum suum, redémit Ierúsalem.

\noindent Nudávit Dóminus bráchium sanctum suum in óculis ómnium géntium; et vidébunt omnes fines terræ salutáre Dei nostri.}
\newcommand{\responsoriumiii}{\pars{Responsorium 3.} \scriptura{\Rbar{} Ier. 31, 21 \Vbar{} Ps. 106, 3; \textbf{H32}}

\vspace{-5mm}

\responsorium{VIII}{temporalia/resp-virgoisrael-CROCHU-cumdox.gtex}{}

\rubrica{vel ad libitum:}

\vspace{3mm}

\pars{Responsorium 3.} \scriptura{\textbf{H42}}

\vspace{-5mm}

\responsorium{VIII}{temporalia/resp-sanctificaminifilii-CROCHU-cumdox.gtex}{}}
\newcommand{\lectioiv}{\pars{Lectio IV.} \scriptura{Sermo 185: PL 38, 997-999}

\noindent Ex Sermónibus sancti Augustíni epíscopi.

\noindent Expergíscere, homo: pro te Deus factus est homo. \textit{Surge, qui dormis, et exsúrge a mórtuis, et illuminábit te Christus.} Pro te, inquam, Deus factus est homo.

\noindent In ætérnum mórtuus esses, nisi in témpore natus esset. Numquam liberaréris a carne peccáti, nisi suscepísset similitúdinem carnis peccáti. Perpétua te possidéret miséria, nisi fíeret hæc misericórdia. Non revixísses, nisi tuæ morti convenísset. Defecísses, nisi subvenísset. Perísses, nisi venísset.}
\newcommand{\responsoriumiv}{\pars{Responsorium 4.} \scriptura{\Rbar{} Gal. 4, 4; Rom. 8, 3 \Vbar{} Is. 14, 1; \textbf{H32}}

\vspace{-5mm}

\responsorium{V}{temporalia/resp-ecceiamveniet-CROCHU.gtex}{}}
\newcommand{\lectiov}{\pars{Lectio V.}

\noindent Celebrémus læti nostræ salútis et redemptiónis advéntum. Celebrémus festum diem, quo magnus et ætérnus dies ex magno et ætérno die venit in hunc nostrum tam brevem temporálem diem.

\noindent Hic \textit{est nobis factus iustítia et sanctificátio et redémptio: ut, quemádmodum scriptum est: Qui gloriátur, in Dómino gloriétur.}

\noindent \textit{Véritas ergo de terra orta est:} Christus qui dixit: \textit{Ego sum véritas}, de Vírgine natus est. \textit{Et iustítia de cælo prospéxit:} quóniam credens in eum qui natus est, non homo a se ipso, sed a Deo iustificátus est.

\noindent \textit{Véritas de terra orta est:} quia \textit{Verbum caro factum est.} Et iustítia de cælo prospéxit: quia \textit{omne datum óptimum et omne donum perféctum desúrsum est.}

\noindent \textit{Véritas de terra orta est,} caro de María. \textit{Et iustítia de cælo prospéxit:} quia \textit{non potest homo accípere quidquam, nisi fúerit ei datum de cælo.}

\noindent \textit{Iustificáti ex fide, pacem habeámus ad Deum:} quia \textit{iustítia et pax osculátæ sunt ínvicem. Per Dóminum nostrum Iesum Christum:} quia \textit{Véritas de terra orta est. Per quem et accéssum habémus in grátiam istam, in qua stamus, et gloriámur in spe glóriæ Dei.} Non ait: «glóriæ nostræ»; sed, \textit{glóriæ Dei:} quia \textit{iustítia} non de nobis procéssit, sed \textit{de cælo prospéxit.} Ergo \textit{qui gloriátur,} non in se, sed \textit{in Dómino gloriétur.}}
\newcommand{\responsoriumv}{\pars{Responsorium 5.} \scriptura{\Rbar{} Is. 9, 6 \Vbar{} Cf. Mal. 3, 1; \textbf{H32}}

\vspace{-5mm}

\responsorium{VII}{temporalia/resp-nasceturnobisparvulus-CROCHU-cumdox.gtex}{}}
\newcommand{\lectiovi}{\pars{Lectio VI.}

\noindent Hinc enim et nato ex Vírgine Dómino præcónium vocis angélicæ factum est: \textit{Glória in excélsis Deo et in terra pax homínibus bonæ voluntátis.}

\noindent In terra enim pax unde, nisi quia \textit{Véritas de terra orta est,} id est, Christus de carne natus est? Et \textit{ipse est pax nostra qui fecit útraque unum:} ut essémus hómines bonæ voluntátis, suáviter conéxi vínculo unitátis.

\noindent In hac ígitur grátia gaudeámus, ut sit glória nostra testimónium consciéntiæ nostræ: ubi non in nobis, sed in Dómino gloriémur. Hinc enim dictum est, \textit{Glória mea et exáltans caput meum.} Nam quæ maior grátia Dei nobis pótuit illucéscere, quam ut, habens unigénitum Fílium, fáceret eum hóminis Fílium, atque ita vicíssim hóminis fílium fáceret Dei fílium?

\noindent Quære méritum, quære causam, quære iustítiam; et vide utrum invénias nisi grátiam.}
\newcommand{\responsoriumvi}{\pars{Responsorium 6.} \scriptura{\Rbar{} Is. 1, 11 \Vbar{} Is. 45, 8; \textbf{H36}}

\vspace{-5mm}

\responsorium{I}{temporalia/resp-egredieturvirga-CROCHU-cumdox.gtex}{}

\rubrica{vel ad libitum:}

\vspace{3mm}

\pars{Responsorium 6.} \scriptura{\Rbardot{} 2 Chr. 20, 17 \Vbardot{} Is. 26, 19; \textbf{H42}}

\vspace{-5mm}

\responsorium{VIII}{temporalia/resp-constantesestote-CROCHU-cumdox.gtex}{}}
\newcommand{\evangelium}{
\pars{Versus.} \scriptura{Cf. Mch. 1, 3}

% Versus. %%%
\sineinitiali{temporalia/versus-egredietur.gtex}

\vspace{5mm}

\sineinitiali{temporalia/oratiodominica-mat.gtex}

\vspace{5mm}

\pars{Absolutio.}

\cuminitiali{}{temporalia/absolutio-avinculis.gtex}

\vfill
\pagebreak

\cuminitiali{}{temporalia/benedictio-solemn-evangelica.gtex}

\vspace{7mm}

\pars{Evangelium} \scriptura{Lc. 1, 26-38}

\noindent Léctio sancti Evangélii secúndum Lucam.

\noindent In illo témpore:

\noindent Missus est ángelus Gábriel a Deo in civitátem Galilǽæ, cui nomen Názareth, ad vírginem desponsátam viro, cui nomen erat Ioseph de domo David, et nomen vírginis María.

\noindent Et ingréssus ad eam dixit: «Ave, grátia plena, Dóminus tecum». Ipsa autem turbáta est in sermóne eius et cogitábat qualis esset ista salutátio.

\noindent Et ait ángelus ei: «Ne tímeas, María; invenísti enim grátiam apud Deum. Et ecce concípies in útero et páries fílium et vocábis nomen eius Iesum. Hic erit magnus et Fílius Altíssimi vocábitur, et dabit illi Dóminus Deus sedem David patris eius, et regnábit super domum Iacob in ætérnum, et regni eius non erit finis».

\noindent Dixit autem María ad ángelum: «Quómodo fiet istud, quóniam virum non cognósco?».

\noindent Et respóndens ángelus dixit ei: «Spíritus Sanctus supervéniet in te, et virtus Altíssimi obumbrábit tibi: ideóque et quod nascétur sanctum, vocábitur Fílius Dei. Et ecce Elísabeth cognáta tua et ipsa concépit fílium in senécta sua, et hic mensis est sextus illi, quæ vocátur stérilis, quia non erit impossíbile apud Deum omne verbum».

\noindent Dixit autem María: «Ecce ancílla Dómini; fiat mihi secúndum verbum tuum». Et discéssit ab illa ángelus.

\scriptura{Sermo 2 de Advéntus, 1. 2-3 : SC 166,104. 108}

\noindent Ex Sermónibus beáti Guerríci abbátis.

\noindent \emph{Ecce venit Rex: occurrámus óbviam Salvatóri nostro.} Pulchre Sálomon ait: \emph{Aqua frígida ánimæ sitiénti núntius bonus de terra longínqua.} Bonus útique núntius, qui advéntum Salvatóris núntiat, reconciliatiónem mundi, bona superventúri sǽculi. \emph{Quam pulchri pedes annuntiántium pacem, annuntiántium bona.} Multi síquidem, non unus, multi, inquam, sed in uno spíritu longa série ab inítio sǽculi nobis supervenére núntii; et ómnium vox símilis, una senténtia: \emph{Venit, ecce venit.}

\noindent Veni ergo, Dómine, \emph{salvum me fac et salvus ero;} veni et \emph{osténde fáciem tuam et salvi érimus. Te enim exspectávimus; esto salus nostra in témpore tribulatiónis.} Sic prophétæ et iusti desidério et afféctu tanto ante Christo ventúro occurrébant, desiderántes si fíeri posset óculis vidére quod spíritu prævidébant.

\noindent Unde Dóminus discípulis dicébat: \emph{Beáti óculi, qui vident quæ vos vidétis. Dico enim vobis, quod multi prophétæ et iusti voluérunt vidére quæ vos vidétis et non vidérunt.} Abraham quoque pater noster exultávit ut vidéret diem Christi. \emph{Vidit,} sed apud ínferos, \emph{et gavísus est.} In quo útique tepor et durítia cordis nostri sugillátur: si non vidélicet cum gáudio spiritáli Christi nascéntis diem anniversárium exspectámus, qui nobis in próximo vidéndus, Dómino annuénte, promíttitur.

\noindent Hoc sane gáudium nostrum tale vidétur exígere Scriptúra, ut spíritus noster levans se super se Christo veniénti quodámmodo occúrrere géstiat, desidério se exténdens in anterióra, impatiénsque morárum iam vidére conténdat futúra. Ego namque non solum ad secúndum advéntum sed étiam ad primum árbitror pertinére, quod tot locis Scripturárum ei monémur occúrrere. Quómodo? inquis. Quia vidélicet, sicut secúndo advéntui occurrémus motu et exsultatióne córporis, sic et primo occurréndum est afféctu et exsultatióne cordis.

\vfill
\pagebreak

\pars{Responsorium 7.} \scriptura{\Vbar{} Ps. 49, 2; \textbf{H42}}

\vspace{-5mm}

\responsorium{VIII}{temporalia/resp-deillaocculta-CROCHU-cumdox.gtex}{}

\vfill
\pagebreak

{
\pars{Hymnus Ambrosianus} \scriptura{Tonus Cisterciensis}

\vspace{-2mm}

\grechangedim{interwordspacetext}{0.27 cm plus 0.15 cm minus 0.05 cm}{scalable}%
\cuminitiali{IV}{temporalia/tedeum-cisterciense.gtex}
\grechangedim{interwordspacetext}{0.22 cm plus 0.15 cm minus 0.05 cm}{scalable}%
}

\vfill
\pagebreak}
\newcommand{\hymnuslaudes}{\pars{Hymnus}

\cuminitiali{D}{temporalia/hym-MagnisProphetae.gtex}}
\newcommand{\laudes}{\pars{Psalmus 1.} \scriptura{Mt. 2, 6; \textbf{H37}}

\vspace{-4.5mm}

\antiphona{III a}{temporalia/ant-tubethlehem.gtex}

\scriptura{Psalmus 117.}

\initiumpsalmi{temporalia/ps117-initium-iii-a-auto.gtex}

\input{temporalia/ps117-iii-a.tex}

\vfill

\antiphona{}{temporalia/ant-tubethlehem.gtex}

\vfill
\pagebreak

\pars{Psalmus 2.} \scriptura{Lc. 21, 28; \textbf{H43}}

\vspace{-4mm}

\antiphona{I g}{temporalia/ant-levatecapita.gtex}

\scriptura{Canticum Danielis, Dan. 3, 52-57}

%\vspace{-2mm}

\initiumpsalmi{temporalia/dan33-initium-i-g-auto.gtex}

\input{temporalia/dan33-i-g.tex} \Abardot{}

\vfill
\pagebreak

\pars{Psalmus 3.} \scriptura{1 Sam. 11, 9; \textbf{H43}}

\vspace{-4mm}

\antiphona{VIII c}{temporalia/ant-crastinaerit.gtex}

\scriptura{Psalmus 150.}

%\vspace{-2mm}

\initiumpsalmi{temporalia/ps150-initium-viii-c-auto.gtex}

\input{temporalia/ps150-viii-c.tex} \Abardot{}

\vfill
\pagebreak}
\newcommand{\lectiobrevis}{\pars{Lectio Brevis.} \scriptura{Is. 11, 1-3}

\noindent Egrediétur virga de stirpe Iesse, et flos de radíce eius ascéndet; et requiéscet super eum spíritus Dómini: spíritus sapiéntiæ et intelléctus, spíritus consílii et fortitúdinis, spíritus sciéntiæ et timóris Dómini; et delíciæ eius in timóre Dómini.}
\newcommand{\responsoriumbreve}{\pars{Responsorium breve.}

\cuminitiali{VI}{temporalia/resp-crastinadie.gtex}}
\newcommand{\benedictus}{\pars{Canticum Zachariæ.} \scriptura{Lc. 2, 6-7; \textbf{H71}}

\vspace{-4mm}

{
\grechangedim{interwordspacetext}{0.18 cm plus 0.15 cm minus 0.05 cm}{scalable}%
\antiphona{VIII G}{temporalia/ant-completisunt.gtex}
\grechangedim{interwordspacetext}{0.22 cm plus 0.15 cm minus 0.05 cm}{scalable}%
}

%\vspace{-2mm}

\scriptura{Lc. 1, 68-79}

%\vspace{-2mm}

\cantusSineNeumas
\initiumpsalmi{temporalia/benedictus-initium-viii-G-auto.gtex}

%\vspace{-1.5mm}

\input{temporalia/benedictus-viii-G.tex} \Abardot{}}
\newcommand{\preces}{\noindent Christum redemptórem, fratres caríssimi, devotióne mentis orémus,~\gredagger{} qui véniet cum glória et potestáte magna,~\grestar{} atque súpplices invocémus:

\Rbardot{} Veni, Dómine Iesu.

\noindent Christe Dómine, qui excélsus in fortitúdine vénies,~\gredagger{} réspice humília nostra,~\grestar{} ut dignos nos fácias munéribus tuis.

\Rbardot{} Veni, Dómine Iesu.

\noindent Qui Evangélium homínibus manifestáre venísti,~\grestar{} da tuam nos semper prædicáre salútem.

\Rbardot{} Veni, Dómine Iesu.

\noindent Tu, qui es benedíctus et vivis et ómnia regis,~\grestar{} da nos gaudéntes exspectáre beátam spem et advéntum magnificéntiæ tuæ.

\Rbardot{} Veni, Dómine Iesu.

\noindent Et nos, qui advéntus tui grátiæ anhelámus,~\grestar{} divinitátis tuæ múnere consoláre.

\Rbardot{} Veni, Dómine Iesu.}
\newcommand{\sinevesperas}{Sine vesperas.}
\newcommand{\hebdomada}{infra Hebdom. IV post Pentecosten.}
\newcommand{\oratioLaudes}{\cuminitiali{}{temporalia/oratio4.gtex}}

% LuaLaTeX

\documentclass[a4paper, twoside, 12pt]{article}
\usepackage[latin]{babel}
%\usepackage[landscape, left=3cm, right=1.5cm, top=2cm, bottom=1cm]{geometry} % okraje stranky
%\usepackage[landscape, a4paper, mag=1166, truedimen, left=2cm, right=1.5cm, top=1.6cm, bottom=0.95cm]{geometry} % okraje stranky
\usepackage[landscape, a4paper, mag=1400, truedimen, left=0.5cm, right=0.5cm, top=0.5cm, bottom=0.5cm]{geometry} % okraje stranky

\usepackage{fontspec}
\setmainfont[FeatureFile={junicode.fea}, Ligatures={Common, TeX}, RawFeature=+fixi]{Junicode}
%\setmainfont{Junicode}

% shortcut for Junicode without ligatures (for the Czech texts)
\newfontfamily\nlfont[FeatureFile={junicode.fea}, Ligatures={Common, TeX}, RawFeature=+fixi]{Junicode}

\usepackage{multicol}
\usepackage{color}
\usepackage{lettrine}
\usepackage{fancyhdr}

% usual packages loading:
\usepackage{luatextra}
\usepackage{graphicx} % support the \includegraphics command and options
\usepackage{gregoriotex} % for gregorio score inclusion
\usepackage{gregoriosyms}
\usepackage{wrapfig} % figures wrapped by the text
\usepackage{parcolumns}
\usepackage[contents={},opacity=1,scale=1,color=black]{background}
\usepackage{tikzpagenodes}
\usepackage{calc}
\usepackage{longtable}
\usetikzlibrary{calc}

\setlength{\headheight}{14.5pt}

\input{conventuscommune.tex} % Often used macros
%%%% Preklady jednotlivych zpevu (nektere se opakuji, a je dobre mit je
% vsechny na jedne hromade)

% HOURS ---

\newcommand{\trAntI}{\translatioCantus{Muž boží měl kožený toulec, pečlivě
zavázaný, jenž mu visel na šíji a~často se ho dotýkal.}}

\newcommand{\trAntII}{\translatioCantus{Klíč od~něho tak dobře střežil, že
dokud žil v~těle, nikdo z~jeho žáků nezvěděl, co je uvnitř.}}

\newcommand{\trAntIII}{\translatioCantus{Ale když se odebral z~tohoto
života, schránku otevřeli a~objevili v~ní žíněné roucho a~měděný řetěz
potřísněný krví.}}

\newcommand{\trAntIV}{\translatioCantus{A když prohlédli mistrovo tělo,
nalezli jeho tělo na čtyřech místech hluboce zbrázděno ranami od řetězu.}}

\newcommand{\trAntV}{\translatioCantus{Krev vytékající z~těch ran, místy
prostoupila i~žíněným rouchem.}}

\newcommand{\trCapituli}{\translatioCantus{
Miláčkovi Boha a~lidí,
Mojžíšovi požehnané paměti,~\gredagger{}
dopřál slávu rovnou slávě svatých~\grestar{}
učinil ho mocným na postrach nepřátelům
a~jeho slovy zastavil divy.}}

\newcommand{\trLectioBrevis}{\translatioCantus{
Pamatujte na své představené,
kteří vám hlásali Boží slovo.
Uvažte, jak oni skončili život, a~napodobujte jejich víru.
Ježíš Kristus je stejný včera i~dnes i~navěky.
Nenechte se svést věelijakými cizími naukami.}}

\newcommand{\trRespLaud}{\translatioCantus{Spravedlivého vodil Hospodin~\grestar{}
po přímých stezkách. \Vbardot{} A~ukázal mu Boží království.}}

\newcommand{\trRespLaudB}{\translatioCantus{Na tvých hradbách, Jeruzaléme,
ustanovil jsem strážné;~\grestar{}
budou bdít nad mým lidem. \Vbardot{} Ani ve dne, ani v~noci nesmějí nikdy
mlčet.}}

\newcommand{\trVersus}{\translatioCantus{\Vbardot{} Ústa spravedlivého šeptají moudrost, aleluja.
\Rbardot{} A~jeho jazyk ohlašuje právo, aleluja.}}

\newcommand{\trAntBenedictus}{\translatioCantus{Když na bujné oře vložili
nosítka a~sňali jim uzdu, vydali se přímo k~cele božího muže.}}

\newcommand{\trPreces}{\translatioCantus{
\noindent S vděčností chvalme Krista, dobrého Pastýře, \gredagger{} který dal život za své ovce, \grestar{} a~pokorně ho prosme: \Rbardot{} Pane, buď pastýřem svého lidu.

\noindent Kriste, ty dáváš církvi pastýře, a~jejich službou se ujímáš svého lidu, \grestar{} dej, ať v~lásce těch, kteří nás vedou, poznáváme, jak nás miluješ. \Rbardot{} Pane, buď pastýřem svého lidu.

\noindent Ty stále konáš skrze své zástupce službu pastýře a~učitele, \grestar{} nepřestávej nás nikdy vést prostřednictvím svých služebníků. \Rbardot{} Pane, buď pastýřem svého lidu.

\noindent Ty prokazuješ svému lidu skrze jeho pastýře službu lékaře duše i~těla, \grestar{} ochraňuj náš život a~veď nás ke svatosti. \Rbardot{} Pane, buď pastýřem svého lidu.

\noindent Ty posíláš své svaté, aby slovem i~příkladem vedli tvůj lid k~tobě, \grestar{} na jejich přímluvu nás posiluj, abychom vytrvali na cestě, která vede k~věčnému životu. \Rbardot{} Pane, buď pastýřem svého lidu.}}

\newcommand{\trOrationis}{\translatioCantus{Bože, jenž nám dopřáváš radovat
se z~výroční slavnosti svatého tvého vyznavače Havla, uděl dobrotivě,
abychom když slavíme jeho narození, též se řídili podobou jeho skutků.
Skrze…}}
 % Czech translations of the proper texts

\newcommand{\annusEditionis}{2020}

%%%% Vicekrat opakovane kousky

\newcommand{\anteOrationem}{
  \rubrica{Ante Orationem, cantatur a Superiore:}

  \pars{Supplicatio Litaniæ.}

  \cuminitiali{}{temporalia/supplicatiolitaniae.gtex}

  \pars{Oratio Dominica.}

  \cuminitiali{}{temporalia/oratiodominica.gtex}

  \rubrica{Deinde dicitur ab Hebdomadario:}

  \cuminitiali{}{temporalia/dominusvobiscum-solemnis.gtex}

  \rubrica{In choro monialium loco Dominus vobiscum dicitur:}

  \sineinitiali{temporalia/domineexaudi.gtex}
}

\setlength{\columnsep}{30pt} % prostor mezi sloupci

%%%%%%%%%%%%%%%%%%%%%%%%%%%%%%%%%%%%%%%%%%%%%%%%%%%%%%%%%%%%%%%%%%%%%%%%%%%%%%%%%%%%%%%%%%%%%%%%%%%%%%%%%%%%%
\begin{document}

% Here we set the space around the initial.
% Please report to http://home.gna.org/gregorio/gregoriotex/details for more details and options
\grechangedim{afterinitialshift}{2.2mm}{scalable}
\grechangedim{beforeinitialshift}{2.2mm}{scalable}
\grechangedim{interwordspacetext}{0.22 cm plus 0.15 cm minus 0.05 cm}{scalable}%
\grechangedim{annotationraise}{-0.2cm}{scalable}

% Here we set the initial font. Change 38 if you want a bigger initial.
% Emit the initials in red.
\grechangestyle{initial}{\color{red}\fontsize{38}{38}\selectfont}

\pagestyle{empty}

%%%% Titulni stranka
\begin{titulusOfficii}
\titulus{}
\end{titulusOfficii}

% graphic
%\vspace{1.5cm}
%\begin{center}
%\includegraphics[width=8cm]{emmaus.jpg}
%\end{center}

\vfill

\begin{center}
%Ad usum et secundum consuetudines chori \guillemotright{}Conventus Choralis\guillemotleft.

%Editio Sancti Wolfgangi \annusEditionis
\end{center}

\pagebreak

\renewcommand{\headrulewidth}{0pt} % no horiz. rule at the header
\fancyhf{}
\pagestyle{fancy}

\pars{Oratio ante divinum Officium.}

\lettrine{{\color{red}A}}{peri,} Dómine, os meum ad benedicéndum nomen sanctum tuum:
munda quoque cor meum ab ómnibus vanis, pervérsis, et aliénis
cogitatiónibus:
intelléctum illúmina, afféctum inflámma,
ut digne, atténte ac devóte hoc Offícium recitáre váleam,
et exaudíri mérear ante conspéctum Divínæ Maiestátis tuæ.
Per Christum, Dóminum nostrum.
\Rbardot{} Amen.

Dómine, in unióne illíus divínæ intentiónis,
qua ipse in terris laudes Deo persolvísti,
has tibi Horas \rubricatum{(vel \textnormal{hanc tibi Horam})} persólvo.

%\trOratioAnteOfficium

\vfill

\pars{Oratio post divinum Officium.}

\rubrica{
  Orationem sequentem devote post Officium recitantibus
  Leo Papa X. defectus, et culpas in eo persolvendo ex humana
  fragilitate contractas, indulsit, et dicitur flexis genibus.
}

\lettrine{{\color{red}S}}{acrosánctæ} et indivíduæ Trinitáti,
crucifíxi Dómini nostri Iesu Christi humanitáti,
beatíssimæ et gloriosíssimæ sempérque Vírginis Maríæ
fecúndæ integritáti, 
et ómnium Sanctórum universitáti
sit sempitérna laus, honor, virtus et glória
ab omni creatúra,
nobísque remíssio ómnium peccatórum,
per infiníta sǽcula sæculórum.
\Rbardot{} Amen.

\noindent \Vbardot{} Beáta víscera Maríæ Virginis, quæ portavérunt
ætérni Patris Fílium.\\
\Rbardot{} Et beáta úbera, quæ lactavérunt Christum Dominum.

\rubrica{Et dicitur secreto \textnormal{Pater noster.} et \textnormal{Ave María.}}

%\trOratioPostOfficium

\vfill

\hora{Ad I. Vesperas.} %%%%%%%%%%%%%%%%%%%%%%%%%%%%%%%%%%%%%%%%%%%%%%%%%%%%%
%\sideThumbs{I. Vesperæ}

\cantusSineNeumas

\vspace{0.5cm}
\grechangedim{interwordspacetext}{0.18 cm plus 0.15 cm minus 0.05 cm}{scalable}%
\cuminitiali{}{temporalia/deusinadiutorium-solemnis.gtex}
\grechangedim{interwordspacetext}{0.22 cm plus 0.15 cm minus 0.05 cm}{scalable}%

\vfill
\pagebreak

\pars{Psalmus 1.} \scriptura{Ps. 144, 13; \textbf{H100}}

\vspace{-4mm}

\antiphona{VII c\textsuperscript{2}}{temporalia/ant-regnumtuum.gtex}

\scriptura{Psalmus 144, 10-21.}

\initiumpsalmi{temporalia/ps144ii-initium-vii-c2-auto.gtex}

%\psalmusEtTranslatioT{temporalia/ps144ii-VII-comb.tex}{10cm}
\input{temporalia/ps144ii-VII.tex} \Abardot{}

\vspace{-1cm}

\vfill
\pagebreak

\pars{Psalmus 2.} \scriptura{Ps. 145, 2; \textbf{H100}}

\vspace{-4mm}

\antiphona{IV E}{temporalia/ant-laudabodeum.gtex}

\scriptura{Psalmus 145.}

\initiumpsalmi{temporalia/ps145-initium-iv-E-auto.gtex}

%\psalmusEtTranslatioT{temporalia/ps145-VII-comb.tex}{10cm}
\input{temporalia/ps145-VII.tex} \Abardot{}

\vfill
\pagebreak

\pars{Psalmus 3.} \scriptura{Ps. 146, 1; \textbf{H101}}

\vspace{-4mm}

\antiphona{VIII a}{temporalia/ant-deonostro.gtex}

\scriptura{Psalmus 146.}

\initiumpsalmi{temporalia/ps146-initium-viii-A-auto.gtex}

%\psalmusEtTranslatioT{temporalia/ps146-VII-comb.tex}{10cm}
\input{temporalia/ps146-VII.tex} \Abardot{}

\vfill
\pagebreak

\pars{Psalmus 4.} \scriptura{Ps. 147, 1}

\vspace{-4mm}

\antiphona{E}{temporalia/ant-laudajerusalem.gtex}

\scriptura{Psalmus 147.}

\initiumpsalmi{temporalia/ps147-initium-e-auto.gtex}

%\psalmusEtTranslatioT{temporalia/ps147-VII-comb.tex}{10cm}
\input{temporalia/ps147-VII.tex} \Abardot{}

\vfill
\pagebreak

\pars{Capitulum.} \scriptura{Rom. 11, 33}

\grechangedim{interwordspacetext}{0.12 cm plus 0.15 cm minus 0.05 cm}{scalable}%
\cuminitiali{}{temporalia/capitulum-OAltitudo.gtex}
\grechangedim{interwordspacetext}{0.22 cm plus 0.15 cm minus 0.05 cm}{scalable}

% preklad Jeruz. bible
%\trCapituliI

\vfill

\pars{Responsorium breve.} \scriptura{Ps. 146, 5}

\cuminitiali{VI}{temporalia/resp-magnusdominusnoster.gtex}

%\trResp

\vfill
\pagebreak

\pars{Hymnus} \scriptura{Ambrosius (\olddag{} 397)}

\cuminitiali{I}{temporalia/hym-OLuxBeata-aestivalis.gtex}
\vspace{-3mm}
%\input{hym-OLuxBeata-bohtext.tex}

\vfill
%\pagebreak

\pars{Versus.}

% Versus. %%%
\sineinitiali{temporalia/versus-vespertina.gtex}

%\noindent \trVersus

\vfill
\pagebreak

\magnificati

\vfill
\pagebreak

%\sideThumbs{{\scriptsize{}Fine horarum}}

\anteOrationem

\pagebreak

% Oratio. %%%
\oratioLaudes

\vspace{-1mm}
%\trOrationisI

\vfill

\rubrica{Hebdomadarius dicit iterum Dominus vobiscum, vel cantor dicit:}

\vspace{2mm}

\sineinitiali{temporalia/domineexaudi.gtex}

\rubrica{Postea cantatur a cantore:}

\vspace{2mm}

\cuminitiali{I}{temporalia/benedicamus-dominica-perannum.gtex}

\vspace{1mm}

\vfill
\pagebreak

\hora{Ad Matutinum.} %%%%%%%%%%%%%%%%%%%%%%%%%%%%%%%%%%%%%%%%%%%%%%%%%%%%%
%\sideThumbs{Matutinum}

\vspace{2mm}

\cuminitiali{}{temporalia/dominelabiamea.gtex}

\vspace{2mm}

\pars{Invitatorium.} \scriptura{Ps. 94, 1; Psalmus 94}

\vspace{-6mm}

\antiphona{E}{temporalia/inv-veniteexsultemus.gtex}

\vfill
\pagebreak

\pars{Hymnus.} \scriptura{Adamus Sancti Victoris (\olddag 1146)}

\vspace{-5mm}

\antiphona{VII}{temporalia/hym-SalveDies.gtex}

\scriptura{Non dicitur \textnormal{Amen} in fine.}
%{
%\vspace{-5mm}
%\setlength{\columnsep}{0pt} % prostor mezi sloupci
%\input{hym-SalveDies-bohtext.tex}
%\setlength{\columnsep}{30pt} % prostor mezi sloupci
%}

\vfill
\pagebreak

\subhora{In I. Nocturno}

\pars{Psalmus 1.} \scriptura{Ps. 1, 1}

\vspace{-4mm}

\antiphona{VIII G}{temporalia/ant-beatusvir.gtex}

%\vspace{-5mm}

\scriptura{Ps. 1}

%\vspace{-2mm}

\initiumpsalmi{temporalia/ps1-initium-viii-G-auto.gtex}

%\psalmusEtTranslatioT{temporalia/ps1-I-comb.tex}{10cm}
\input{temporalia/ps1-I.tex} \Abardot{}

\vfill
\pagebreak

\pars{Psalmus 2.} \scriptura{Ps. 2, 11; \textbf{H93}}

\vspace{-4mm}

\antiphona{VII a}{temporalia/ant-servitedomino.gtex}

\vspace{-3mm}

\scriptura{Ps. 2}

\vspace{-2mm}

\initiumpsalmi{temporalia/ps2-initium-vii-a-auto.gtex}

%\psalmusEtTranslatioT{temporalia/ps2-I-comb.tex}{10cm}
\input{temporalia/ps2-I.tex} \Abardot{}

\vfill
\pagebreak

\pars{Psalmus 3.} \scriptura{Ps. 3, 7}

\vspace{-4mm}

\antiphona{VI F}{temporalia/ant-exsurgedominesalvum.gtex}

%\vspace{-5mm}

\scriptura{Ps. 3}

\initiumpsalmi{temporalia/ps3-initium-vi-F-auto.gtex}

%\psalmusEtTranslatioT{temporalia/ps3-I-comb.tex}{10cm}
\input{temporalia/ps3-I.tex} \Abardot{}

\vfill
\pagebreak

\pars{Versus.} \scriptura{Ps. 118, 55}

% Versus. %%%
\sineinitiali{temporalia/versus-memorfui.gtex}

\vspace{5mm}

\sineinitiali{temporalia/oratiodominica-mat.gtex}

\vspace{5mm}

\pars{Absolutio.}

\cuminitiali{}{temporalia/absolutio-exaudi.gtex}

\vfill
\pagebreak

\cuminitiali{}{temporalia/benedictio-solemn-benedictione.gtex}

\vspace{7mm}

\lectioi

\noindent \Vbardot{} Tu autem, Dómine, miserére nobis.
\noindent \Rbardot{} Deo grátias.

\vfill
\pagebreak

\responsoriumi

\vfill
\pagebreak

\cuminitiali{}{temporalia/benedictio-solemn-unigenitus.gtex}

\vspace{7mm}

\lectioii

\noindent \Vbardot{} Tu autem, Dómine, miserére nobis.
\noindent \Rbardot{} Deo grátias.

\vfill
\pagebreak

\responsoriumii

\vfill
\pagebreak

\cuminitiali{}{temporalia/benedictio-solemn-spiritus.gtex}

\vspace{7mm}

\lectioiii

\noindent \Vbardot{} Tu autem, Dómine, miserére nobis.
\noindent \Rbardot{} Deo grátias.

\vfill
\pagebreak

\responsoriumiii

\vfill
\pagebreak

\subhora{In II. Nocturno}

\pars{Psalmus 4.} \scriptura{Ps. 8, 2}

\vspace{-4mm}

\antiphona{I g}{temporalia/ant-quamadmirabileest.gtex}

%\vspace{-5mm}

\scriptura{Ps. 8}

%A\vspace{-2mm}

\initiumpsalmi{temporalia/ps8-initium-i-g-auto.gtex}

%\psalmusEtTranslatioT{temporalia/ps8-I-comb.tex}{10cm}
\input{temporalia/ps8-I.tex} \Abardot{}

\vfill
\pagebreak

\pars{Psalmus 5.} \scriptura{Ps. 9, 5}

\vspace{-4mm}

\antiphona{VIII G}{temporalia/ant-sedistisuperthronum.gtex}

%\vspace{-5mm}

\scriptura{Ps. 9, 2-11}

\initiumpsalmi{temporalia/ps9ii_xi-initium-viii-G-auto.gtex}

%\psalmusEtTranslatioT{temporalia/ps9ii_xi-I-comb.tex}{10cm}
\input{temporalia/ps9ii_xi-I.tex} \Abardot{}

\vfill
\pagebreak

\pars{Psalmus 6.} \scriptura{Ps. 9, 20}

\vspace{-4mm}

\antiphona{I g\textsuperscript{3}}{temporalia/ant-exsurgedominenon.gtex}

%\vspace{-5mm}

\scriptura{Ps. 9, 12-21}

\initiumpsalmi{temporalia/ps9xii_xxi-initium-i-g3-auto.gtex}

%\psalmusEtTranslatioT{temporalia/ps9xii_xxi-I-comb.tex}{10cm}
\input{temporalia/ps9xii_xxi-I.tex} \Abardot{}

\vfill
\pagebreak

\pars{Versus.} \scriptura{Ps. 118, 62}

% Versus. %%%
\sineinitiali{temporalia/versus-medianocte.gtex}

\vspace{5mm}

\sineinitiali{temporalia/oratiodominica-mat.gtex}

\vspace{5mm}

\pars{Absolutio.}

\cuminitiali{}{temporalia/absolutio-ipsius.gtex}

\vfill
\pagebreak

\cuminitiali{}{temporalia/benedictio-solemn-deus.gtex}

\vspace{7mm}

\lectioiv

\noindent \Vbardot{} Tu autem, Dómine, miserére nobis.
\noindent \Rbardot{} Deo grátias.

\vfill
\pagebreak

\responsoriumiv

\vfill
\pagebreak

\cuminitiali{}{temporalia/benedictio-solemn-christus.gtex}

\vspace{7mm}

\lectiov

\noindent \Vbardot{} Tu autem, Dómine, miserére nobis.
\noindent \Rbardot{} Deo grátias.

\vfill
\pagebreak

\responsoriumv

\vfill
\pagebreak

\cuminitiali{}{temporalia/benedictio-solemn-ignem.gtex}

\vspace{7mm}

\lectiovi

\noindent \Vbardot{} Tu autem, Dómine, miserére nobis.
\noindent \Rbardot{} Deo grátias.

\vfill
\pagebreak

\responsoriumvi

\vfill
\pagebreak

\subhora{In III. Nocturno}

\pars{Psalmus 7.} \scriptura{Ps. 9, 22}

\vspace{-4mm}

\antiphona{II D}{temporalia/ant-utquiddomine.gtex}

\vspace{-4mm}

\scriptura{Ps. 9, 22-32}

%\vspace{-2mm}

\initiumpsalmi{temporalia/ps9xxii_xxxii-initium-ii-D-auto.gtex}

%\psalmusEtTranslatioT{temporalia/ps9xxii_xxxii-I-comb.tex}{10cm}
\input{temporalia/ps9xxii_xxxii-I.tex} \Abardot{}

\vfill
\pagebreak

\pars{Psalmus 8.}\scriptura{Ex. 15, 18}

\vspace{-4mm}

\antiphona{IV* e}{temporalia/ant-inaeternum.gtex}

%\vspace{-4mm}

\scriptura{Ps. 9, 33-39}

\initiumpsalmi{temporalia/ps9xxxiii_xxxix-initium-iv_-e-auto.gtex}

%\psalmusEtTranslatioT{temporalia/ps9xxxiii_xxxix-I-comb.tex}{10cm}
\input{temporalia/ps9xxxiii_xxxix-I.tex} \Abardot{}

\vfill
\pagebreak

\pars{Psalmus 9.} \scriptura{Ps. 10, 8}

\vspace{-4mm}

\antiphona{II* f}{temporalia/ant-justusdominus.gtex}

%\vspace{-4mm}

\scriptura{Ps. 10}

%\initiumpsalmi{temporalia/ps10-initium-iv-c-auto.gtex}
\initiumpsalmi{temporalia/ps10-initium-ii_-f.gtex}

%\psalmusEtTranslatioT{temporalia/ps10-I-comb.tex}{10cm}
\input{temporalia/ps10-I.tex} \Abardot{}

\vfill
\pagebreak

\pars{Versus.} \scriptura{Ps. 118, 148}

% Versus. %%%
\sineinitiali{temporalia/versus-praevenerunt.gtex}

\vspace{5mm}

\sineinitiali{temporalia/oratiodominica-mat.gtex}

\vspace{5mm}

\pars{Absolutio.}

\cuminitiali{}{temporalia/absolutio-avinculis.gtex}

\vfill
\pagebreak

\cuminitiali{}{temporalia/benedictio-solemn-evangelica.gtex}

\vspace{7mm}

\lectiovii

\noindent \Vbardot{} Tu autem, Dómine, miserére nobis.
\noindent \Rbardot{} Deo grátias.

\vfill
\pagebreak

\responsoriumvii

\vfill
\pagebreak

\cuminitiali{}{temporalia/benedictio-solemn-divinum.gtex}

\vspace{7mm}

\lectioviii

\noindent \Vbardot{} Tu autem, Dómine, miserére nobis.
\noindent \Rbardot{} Deo grátias.

\vfill
\pagebreak

\responsoriumviii

\vfill
\pagebreak

\cuminitiali{}{temporalia/benedictio-solemn-adsocietatem.gtex}

\vspace{7mm}

\lectioix

\noindent \Vbardot{} Tu autem, Dómine, miserére nobis.
\noindent \Rbardot{} Deo grátias.

\vfill
\pagebreak

% Te Deum

{
\pars{Hymnus Ambrosianus} \scriptura{Tonus Solemnis}

\vspace{-2mm}

\grechangedim{interwordspacetext}{0.26 cm plus 0.15 cm minus 0.05 cm}{scalable}%
\cuminitiali{III}{temporalia/tedeum-solemnis-gn.gtex}
\grechangedim{interwordspacetext}{0.22 cm plus 0.15 cm minus 0.05 cm}{scalable}%
}

\vfill
\pagebreak

\rubrica{Reliqua omittuntur, nisi Laudes separandæ sint.}

\pars{Oratio}

\noindent \Vbardot{} Dómine, exáudi oratiónem meam.

\noindent \Rbardot{} Et clamor meus ad te véniat.

Orémus:

\oratioLaudes

\vspace{7mm}

\pars{Conclusio}

\noindent \Vbardot{} Dómine, exáudi oratiónem meam.

\noindent \Rbardot{} Et clamor meus ad te véniat.

\noindent \Vbardot{} Benedicámus Dómino, allelúia, allelúia.

\noindent \Rbardot{} Deo grátias, allelúia, allelúia.

\noindent \Vbardot{} Fidélium ánimæ per misericórdiam Dei requiéscant in pace.

\noindent \Rbardot{} Amen.

\vfill
\pagebreak

\hora{Ad Laudes.} %%%%%%%%%%%%%%%%%%%%%%%%%%%%%%%%%%%%%%%%%%%%%%%%%%%%%
%\sideThumbs{Laudes}

\cantusSineNeumas

\vspace{0.5cm}
\grechangedim{interwordspacetext}{0.18 cm plus 0.15 cm minus 0.05 cm}{scalable}%
\cuminitiali{}{temporalia/deusinadiutorium-alter.gtex}
\grechangedim{interwordspacetext}{0.22 cm plus 0.15 cm minus 0.05 cm}{scalable}%

\vfill
%\pagebreak

\pars{Psalmus 1.}

\vspace{-4mm}

\antiphona{VI F}{temporalia/ant-alleluia1.gtex}

\scriptura{Psalmus 50.}

\initiumpsalmi{temporalia/ps50-initium-vi-F-auto.gtex}

%\psalmusEtTranslatioT{temporalia/ps50-I-comb.tex}{10cm}
\input{temporalia/ps50-I.tex}

\vfill
\pagebreak

\pars{Psalmus 2.}

\scriptura{Psalmus 117.}

\initiumpsalmi{temporalia/ps117-initium-vi-F-auto.gtex}

%\psalmusEtTranslatioT{temporalia/ps117-I-comb.tex}{10cm}
\input{temporalia/ps117-I.tex}

\vfill
\pagebreak

\pars{Psalmus 3.}

\scriptura{Psalmus 62.}

\initiumpsalmi{temporalia/ps62-initium-vi-F-auto.gtex}

%\psalmusEtTranslatioT{temporalia/ps62-I-comb.tex}{10cm}
\input{temporalia/ps62-I.tex}

\vfill

\vspace{-6mm}

\antiphona{}{temporalia/ant-alleluia1.gtex} % repeat the antiphon - new page

\vfill
\pagebreak

\pars{Psalmus 4.} \scriptura{Dan. 3, 22-26; \textbf{H422}}

\vspace{-4mm}

\antiphona{VIII G}{temporalia/ant-trespueri.gtex}

\scriptura{Canticum trium puerorum, Dan. 3, 57-88 et 56}

\initiumpsalmi{temporalia/dan3-initium-viii-G-auto.gtex}

%\psalmusEtTranslatioT{temporalia/dan3-comb.tex}{10cm}
\input{temporalia/dan3.tex}

\rubrica{Hic non dicitur Gloria Patri, neque Amen.}

\vfill

\vspace{-6mm}

\antiphona{}{temporalia/ant-trespueri.gtex} % repeat the antiphon - new page

\vfill
\pagebreak

\pars{Psalmus 5.}

\vspace{-4mm}

\antiphona{VIII G}{temporalia/ant-alleluia2.gtex}

\scriptura{Psalmus 148.}

\initiumpsalmi{temporalia/ps148-initium-viii-G-auto.gtex}

%\psalmusEtTranslatioT{temporalia/ps148-I-comb.tex}{10cm}
\input{temporalia/ps148-I.tex}

\rubrica{Hic non dicitur Gloria Patri.}

\vfill
\pagebreak

%
\scriptura{Psalmus 149.}

\initiumpsalmi{temporalia/ps149-initium-viii-G-auto.gtex}

%\psalmusEtTranslatioT{temporalia/ps149-I-comb.tex}{10cm}
\input{temporalia/ps149-I.tex}

\rubrica{Hic non dicitur Gloria Patri.}

\vfill
\pagebreak

%
\scriptura{Psalmus 150.}

\initiumpsalmi{temporalia/ps150-initium-viii-G-auto.gtex}

%\psalmusEtTranslatioT{temporalia/ps150-I-comb.tex}{10cm}
\input{temporalia/ps150-I.tex}

\vfill

\vspace{-6mm}

\antiphona{}{temporalia/ant-alleluia2.gtex} % repeat the antiphon - new page

\vfill
\pagebreak

\pars{Capitulum.} \scriptura{Ac. 7, 12}

\grechangedim{interwordspacetext}{0.12 cm plus 0.15 cm minus 0.05 cm}{scalable}%
\cuminitiali{}{temporalia/capitulum-Benedictio.gtex}
\grechangedim{interwordspacetext}{0.22 cm plus 0.15 cm minus 0.05 cm}{scalable}

% preklad Jeruz. bible
%\trCapituliI

\vfill

\pars{Responsorium breve.} \scriptura{Ps. 118, 36-37}

\cuminitiali{IV}{temporalia/resp-inclinacormeum.gtex}

%\trResp

\vfill
\pagebreak

\pars{Hymnus} \scriptura{Gregorius Magnus (\olddag{} 604)}

\cuminitiali{IV}{temporalia/hym-EcceJamNoctis.gtex}
\vspace{-3mm}
%\input{hym-EcceJamNocis-bohtext.tex}

\vfill
%\pagebreak

\pars{Versus.} \scriptura{Ps. 92, 1}

% Versus. %%%
\sineinitiali{temporalia/versus-dominusregnavit.gtex}

%\noindent \trVersus

\vfill
\pagebreak

\benedictus

\vspace{-1cm}

\vfill
\pagebreak

%\sideThumbs{{\scriptsize{}Fine horarum}}

\anteOrationem

\pagebreak

% Oratio. %%%
\oratioLaudes

\vspace{-1mm}
%\trOrationisI

\vfill

\rubrica{Hebdomadarius dicit iterum Dominus vobiscum, vel cantor dicit:}

\vspace{2mm}

\sineinitiali{temporalia/domineexaudi.gtex}

\rubrica{Postea cantatur a cantore:}

\vspace{2mm}

\cuminitiali{I}{temporalia/benedicamus-dominica-perannum.gtex}

\vspace{1mm}

\vfill
\pagebreak

\hora{Ad II. Vesperas.} %%%%%%%%%%%%%%%%%%%%%%%%%%%%%%%%%%%%%%%%%%%%%%%%%%%%%
%\sideThumbs{II. Vesperæ}

\cantusSineNeumas

%\vspace{0.5cm}
\grechangedim{interwordspacetext}{0.18 cm plus 0.15 cm minus 0.05 cm}{scalable}%
\cuminitiali{}{temporalia/deusinadiutorium-solemnis.gtex}
\grechangedim{interwordspacetext}{0.22 cm plus 0.15 cm minus 0.05 cm}{scalable}%

\vfill
%\pagebreak

\vspace{-2mm}

\pars{Psalmus 1.} \scriptura{Ps. 109, 1; \textbf{H91}}

\vspace{-4mm}

\antiphona{VII c\textsuperscript{2}}{temporalia/ant-dixitdominus.gtex}

\vspace{-4mm}

\scriptura{Psalmus 109.}

\initiumpsalmi{temporalia/ps109-initium-vii-c2-auto.gtex}

%\psalmusEtTranslatioT{temporalia/ps109-I-comb.tex}{10cm}
\input{temporalia/ps109-I.tex} \Abardot{}

\vspace{-1cm}

\vfill
\pagebreak

\pars{Psalmus 2.} \scriptura{Ps. 110, 8; \textbf{H91}}

\vspace{-4mm}

\antiphona{IV g}{temporalia/ant-fideliaomnia.gtex}

\scriptura{Psalmus 110.}

\initiumpsalmi{temporalia/ps110-initium-iv-g-auto.gtex}

%\psalmusEtTranslatioT{temporalia/ps110-I-comb.tex}{10cm}
\input{temporalia/ps110-I.tex} \Abardot{}

\vfill
\pagebreak

\pars{Psalmus 3.} \scriptura{Ps. 111, 1; \textbf{H92}}

\vspace{-4mm}

\antiphona{IV a}{temporalia/ant-inmandatis.gtex}

\scriptura{Psalmus 111.}

\initiumpsalmi{temporalia/ps111-initium-iv-a-auto.gtex}

%\psalmusEtTranslatioT{temporalia/ps111-I-comb.tex}{10cm}
\input{temporalia/ps111-I.tex} \Abardot{}

\vfill
\pagebreak

\pars{Psalmus 4.} \scriptura{Ps. 112, 2; \textbf{H92}}

\vspace{-4mm}

\antiphona{VII c}{temporalia/ant-sitnomendomini.gtex}

\scriptura{Psalmus 112.}

\initiumpsalmi{temporalia/ps112-initium-vii-c-auto.gtex}

%\psalmusEtTranslatioT{temporalia/ps112-I-comb.tex}{10cm}
\input{temporalia/ps112-I.tex} \Abardot{}

\vfill
\pagebreak

\pars{Capitulum.} \scriptura{2 Cor. 1, 3-4}

\grechangedim{interwordspacetext}{0.12 cm plus 0.15 cm minus 0.05 cm}{scalable}%
\cuminitiali{}{temporalia/capitulum-BenedictusDeus.gtex}
\grechangedim{interwordspacetext}{0.22 cm plus 0.15 cm minus 0.05 cm}{scalable}

% preklad Jeruz. bible
%\trCapituliI

\vfill

\pars{Responsorium breve.} \scriptura{Ps. 103, 24}

\cuminitiali{VI}{temporalia/resp-quammagnificata.gtex}

%\trResp

\vfill
\pagebreak

\pars{Hymnus} \scriptura{Gregorius Magnus (\olddag{} 604)}

\cuminitiali{I}{temporalia/hym-LucisCreator-aestivalis.gtex}
\vspace{-3mm}
%\begin{translatioMulticol}{3}
Tvůrce světa předobrý,\\
tys ustanovil denní řád\\
a proudy světla rozhodil,\\
když světu základy jsi klad.\\
\\
A spojils ráno s večerem\\
a dnem tu dobu nazýváš;\\
hle padá temné noci stín -\\
slyš prosbu, vyslyš nářek náš.\columnbreak

Ach, nedej, by nás stihla smrt,\\
když svědomí nám tíží hřích,\\
když nemyslíme na věčnost\\
v té síti hříchů šalebných.\\
\\
Vzbuď naši touhu po nebi,\\
kde věčný život čeká nás,\\
a pomoz odložit vše zlé\\
a smýti z duše každý kaz.\columnbreak

To splň nám, dobrý Otče náš,\\
i ty, jenž rovné božství máš,\\
i Duchu, který těšíš nás\\
a vládneš, Bože, v každý čas.\\
Amen. 
\end{translatioMulticol}


\vfill
%\pagebreak

\pars{Versus.} \scriptura{Ps. 140, 2}

% Versus. %%%
\sineinitiali{temporalia/versus-dirigatur.gtex}

%\noindent \trVersus

\vfill
\pagebreak

\magnificatii

\vfill
\pagebreak

%\sideThumbs{{\scriptsize{}Fine horarum}}

\anteOrationem

\pagebreak

% Oratio. %%%
\oratioLaudes

\vspace{-1mm}
%\trOrationisI

\vfill

\rubrica{Hebdomadarius dicit iterum Dominus vobiscum, vel cantor dicit:}

\vspace{2mm}

\sineinitiali{temporalia/domineexaudi.gtex}

\rubrica{Postea cantatur a cantore:}

\vspace{2mm}

\cuminitiali{I}{temporalia/benedicamus-dominica-perannum.gtex}

\vspace{1mm}

\end{document}

