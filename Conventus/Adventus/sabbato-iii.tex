\newcommand{\lectioi}{\pars{Lectio I.} \scriptura{Is. 47, 1.3b-5.11.13-15}

\noindent De libro Isaíæ prophétæ.

\noindent Descénde, sede in púlvere, virgo fília Bábylon; sede in terra sine sólio, fília Chaldæórum, quia ultra non vocáberis mollis et ténera. «Ultiónem cápiam, némini parcam», dicit Redémptor noster, Dóminus exercítuum nomen illíus, Sanctus Israel. Sede tacens et intra in ténebras, fília Chaldæórum, quia non vocáberis ultra dómina regnórum. Véniet super te malum, et néscies avértere; et írruet super te calámitas, quam non póteris expiáre; véniet super te repénte miséria, quam néscies. Defecísti in multitúdine consiliórum tuórum; stent et salvent te, qui metiúntur cælum, qui contemplántur sídera et annúntiant síngulis novilúniis ventúra tibi. Ecce facti sunt quasi stípula, ignis combússit eos. Non liberábunt seípsos de manu flammæ; non sunt prunæ, quibus calefíant, nec focus, ut sédeant ad eum. Sic fiunt tibi incantatóres tui, in quibuscúmque laborásti ab adulescéntia tua; unusquísque in via sua errat, non est qui salvet te.}
\newcommand{\responsoriumi}{\pars{Responsorium 1.} \scriptura{\Rbar{} Rom. 15, 12; Cf. Ps. 71, 17 \Vbar{} Ps. 49, 2-3; \textbf{H36}}

\vspace{-5mm}

\responsorium{VIII}{temporalia/resp-radixjesse-CROCHU.gtex}{}}
\newcommand{\lectioii}{\pars{Lectio II.} \scriptura{Lib. 3, 20, 2-3: SCh 34, 342-344}

\noindent Ex Tractátu sancti Irenǽi epíscopi Advérsus hǽreses.

\noindent Glória hóminis Deus; operatiónis vero Dei et omnis sapiéntiæ eius et virtútis receptáculum homo.

\noindent Quemádmodum médicus in iis qui ægrótant probátur, sic et Deus in homínibus manifestátur. Quaprópter et Paulus ait: Conclúsit autem Deus ómnia in incredulitáte, ut ómnium misereátur; dicens hoc de hómine, qui fuit inobáudiens Deo et proiéctus de immortalitáte, dehinc misericórdiam consecútus est, per Fílium Dei eam quæ est per ipsum percípiens adoptiónem.

\noindent Hic enim tenens sine inflatióne et iactántia veram glóriam de iis quæ facta sunt et de eo qui fecit, qui est potentíssimus ómnium Deus quique ómnibus ut sint prǽstitit, et manens in dilectióne eius et subiectióne et gratiárum actióne, maiórem ab eo glóriam percípiet, provéctus accípiens dum consímilis fiat eius qui pro eo mórtuus est.}
\newcommand{\responsoriumii}{\pars{Responsorium 2.} \scriptura{\Rbar{} Is. 1, 11 \Vbar{} Is. 45, 8; \textbf{H36}}

\vspace{-5mm}

\responsorium{I}{temporalia/resp-egredieturvirga-CROCHU.gtex}{}}
\newcommand{\lectioiii}{\pars{Lectio III.}

\noindent Quóniam et ipse in similitúdinem carnis peccáti factus est, uti condemnáret peccátum et iam quasi condemnátum proíceret illud extra carnem, provocáret autem in similitúdinem suam hóminem, imitatórem eum assígnans Deo et in patérnam impónens régulam ad vidéndum Deum et cápere Patrem donans: Verbum Dei quod habitávit in hómine et Fílius hóminis factus est, ut assuésceret hóminem percípere Deum et assuésceret Deum habitáre in hómine, secúndum plácitum Patris.

\noindent Propter hoc ergo signum salútis nostræ eum, qui ex Vírgine Emmánuel est, ipse dedit Dóminus, quóniam ipse Dóminus erat qui salvábat eos, quia per semetípsos non habébant salvári; et propter hoc Paulus infirmitátem hóminis annúntians ait: Scio enim quóniam non inhábitat in carne mea bonum, signíficans quóniam non a nobis sed a Deo est bonum salútis nostræ; et íterum: Miser ego homo, quis me liberábit de córpore mortis huius? deínde infert liberatórem, grátia Iesu Christi Dómini nostri.

\noindent Hoc autem et Isaías: Confortámini, inquit, manus resolútæ et génua debília, adhortámini, pusillánimes sensu, confortámini, ne timeátis. Ecce Deus noster iudícium et retributúrus est, ipse véniet et salvábit nos; hoc quóniam non a nobis, sed a Dei adiuménto habúimus salvári.}
\newcommand{\responsoriumiii}{\pars{Responsorium 3.} \scriptura{\Rbar{} Cf. Is. 35, 2 \Vbar{} Is. 40, 10; \textbf{H35}}

\vspace{-5mm}

\responsorium{I}{temporalia/resp-germinaveruntcampi-CROCHU-cumdox.gtex}{}}
\newcommand{\lectiobrevis}{\pars{Lectio Brevis.} \scriptura{Is. 2, 3}

\noindent Veníte et ascendámus ad montem Dómini, ad domum Dei Iacob, ut dóceat nos vias suas, et ambulémus in sémitis eius; quia de Sion exíbit lex et verbum Dómini de Ierúsalem.}
\newcommand{\benedictus}{\pars{Canticum Zachariæ.} \scriptura{Mal. 4, 2; \textbf{H43}}

\vspace{-4mm}

{
\grechangedim{interwordspacetext}{0.18 cm plus 0.15 cm minus 0.05 cm}{scalable}%
\antiphona{VIII G}{temporalia/ant-orietursicutsol.gtex}
\grechangedim{interwordspacetext}{0.22 cm plus 0.15 cm minus 0.05 cm}{scalable}%
}

%\trAntIMagnificat

%\vspace{-2mm}

\scriptura{Lc. 1, 68-79}

%\vspace{-2mm}

\cantusSineNeumas
\initiumpsalmi{temporalia/benedictus-initium-viii-G-auto.gtex}

%\vspace{-1.5mm}

\input{temporalia/benedictus-viii-G.tex} \Abardot{}}
\newcommand{\preces}{\noindent Christum redemptórem, fratres caríssimi, deprecémur, qui véniet, ut redeúntes ad se a mortis potestáte líberet, \gredagger{} et súpplices implorémus:

\Rbardot{} Veni, Dómine Iesu.

\noindent Tuum, Dómine, dum annuntiámus advéntum, \gredagger{} munda cor nostrum ab omni spíritu vanitátis.

\Rbardot{} Veni, Dómine Iesu.

\noindent Ecclésia tua, Dómine, quam fundásti, \gredagger{} te per omnes gentes magníficet.

\Rbardot{} Veni, Dómine Iesu.

\noindent Lex tua, Dómine, illúminans óculos, \gredagger{} prótegat pópulos tíbimet confiténtes.

\Rbardot{} Veni, Dómine Iesu.

\noindent Qui gáudia advéntus tui nobis ab Ecclésia prænuntiári concédis, \gredagger{} fac ut promptíssima devotióne te excipiámus.

\Rbardot{} Veni, Dómine Iesu.}
\newcommand{\oratio}{\pars{Oratio.}

\noindent Deus, qui splendórem glóriæ tuæ per sacræ Vírginis partum mundo dignátus es reveláre, tríbue, quǽsumus, ut tantæ incarnatiónis mystérium et fídei integritáte colámus et devóto semper obséquio frequentémus.

\noindent Per Dóminum nostrum Iesum Christum, Fílium tuum, qui tecum vivit et regnat in unitáte Spíritus Sancti, Deus, per ómnia sǽcula sæculórum.

\noindent \Rbardot{} Amen.}
\newcommand{\hebdomada}{infra Hebdom. III Adventus.}
\newcommand{\oratioLaudes}{\cuminitiali{}{temporalia/oratio3vo.gtex}}
\newcommand{\responsoriumbreve}{\pars{Responsorium breve.} \scriptura{Is. 60, 2; \textbf{H20}}

\cuminitiali{IV}{temporalia/resp-superte.gtex}}

% LuaLaTeX

\documentclass[a4paper, twoside, 12pt]{article}
\usepackage[latin]{babel} 
%\usepackage[landscape, left=3cm, right=1.5cm, top=2cm, bottom=1cm]{geometry} % okraje stranky
%\usepackage[landscape, a4paper, mag=1166, truedimen, left=2cm, right=1.5cm, top=1.6cm, bottom=0.95cm]{geometry} % okraje stranky
\usepackage[landscape, a4paper, mag=1400, truedimen, left=0.5cm, right=0.5cm, top=0.5cm, bottom=0.5cm]{geometry} % okraje stranky

\usepackage{fontspec}
\setmainfont[FeatureFile={junicode.fea}, Ligatures={Common, TeX}, RawFeature=+fixi]{Junicode}
%\setmainfont{Junicode}

% shortcut for Junicode without ligatures (for the Czech texts)
\newfontfamily\nlfont[FeatureFile={junicode.fea}, Ligatures={Common, TeX}, RawFeature=+fixi]{Junicode}

% Hebrew font:
% http://scripts.sil.org/cms/scripts/page.php?site_id=nrsi&id=SILHebrUnic2
\newfontfamily\hebfont[Scale=1]{Ezra SIL}

\usepackage{multicol}
\usepackage{color}
\usepackage{lettrine}
\usepackage{fancyhdr}

% usual packages loading:
\usepackage{luatextra}
\usepackage{graphicx} % support the \includegraphics command and options
\usepackage{gregoriotex} % for gregorio score inclusion
\usepackage{gregoriosyms}
\usepackage{wrapfig} % figures wrapped by the text
\usepackage{parcolumns}
\usepackage[contents={},opacity=1,scale=1,color=black]{background}
\usepackage{tikzpagenodes}
\usepackage{calc}
\usepackage{longtable}
\usetikzlibrary{calc}

\setlength{\headheight}{14.5pt}

\input{conventuscommune.tex} % Often used macros
%%%% Preklady jednotlivych zpevu (nektere se opakuji, a je dobre mit je
% vsechny na jedne hromade)

% HOURS ---

\newcommand{\trAntI}{\translatioCantus{Muž boží měl kožený toulec, pečlivě
zavázaný, jenž mu visel na šíji a~často se ho dotýkal.}}

\newcommand{\trAntII}{\translatioCantus{Klíč od~něho tak dobře střežil, že
dokud žil v~těle, nikdo z~jeho žáků nezvěděl, co je uvnitř.}}

\newcommand{\trAntIII}{\translatioCantus{Ale když se odebral z~tohoto
života, schránku otevřeli a~objevili v~ní žíněné roucho a~měděný řetěz
potřísněný krví.}}

\newcommand{\trAntIV}{\translatioCantus{A když prohlédli mistrovo tělo,
nalezli jeho tělo na čtyřech místech hluboce zbrázděno ranami od řetězu.}}

\newcommand{\trAntV}{\translatioCantus{Krev vytékající z~těch ran, místy
prostoupila i~žíněným rouchem.}}

\newcommand{\trCapituli}{\translatioCantus{
Miláčkovi Boha a~lidí,
Mojžíšovi požehnané paměti,~\gredagger{}
dopřál slávu rovnou slávě svatých~\grestar{}
učinil ho mocným na postrach nepřátelům
a~jeho slovy zastavil divy.}}

\newcommand{\trLectioBrevis}{\translatioCantus{
Pamatujte na své představené,
kteří vám hlásali Boží slovo.
Uvažte, jak oni skončili život, a~napodobujte jejich víru.
Ježíš Kristus je stejný včera i~dnes i~navěky.
Nenechte se svést věelijakými cizími naukami.}}

\newcommand{\trRespLaud}{\translatioCantus{Spravedlivého vodil Hospodin~\grestar{}
po přímých stezkách. \Vbardot{} A~ukázal mu Boží království.}}

\newcommand{\trRespLaudB}{\translatioCantus{Na tvých hradbách, Jeruzaléme,
ustanovil jsem strážné;~\grestar{}
budou bdít nad mým lidem. \Vbardot{} Ani ve dne, ani v~noci nesmějí nikdy
mlčet.}}

\newcommand{\trVersus}{\translatioCantus{\Vbardot{} Ústa spravedlivého šeptají moudrost, aleluja.
\Rbardot{} A~jeho jazyk ohlašuje právo, aleluja.}}

\newcommand{\trAntBenedictus}{\translatioCantus{Když na bujné oře vložili
nosítka a~sňali jim uzdu, vydali se přímo k~cele božího muže.}}

\newcommand{\trPreces}{\translatioCantus{
\noindent S vděčností chvalme Krista, dobrého Pastýře, \gredagger{} který dal život za své ovce, \grestar{} a~pokorně ho prosme: \Rbardot{} Pane, buď pastýřem svého lidu.

\noindent Kriste, ty dáváš církvi pastýře, a~jejich službou se ujímáš svého lidu, \grestar{} dej, ať v~lásce těch, kteří nás vedou, poznáváme, jak nás miluješ. \Rbardot{} Pane, buď pastýřem svého lidu.

\noindent Ty stále konáš skrze své zástupce službu pastýře a~učitele, \grestar{} nepřestávej nás nikdy vést prostřednictvím svých služebníků. \Rbardot{} Pane, buď pastýřem svého lidu.

\noindent Ty prokazuješ svému lidu skrze jeho pastýře službu lékaře duše i~těla, \grestar{} ochraňuj náš život a~veď nás ke svatosti. \Rbardot{} Pane, buď pastýřem svého lidu.

\noindent Ty posíláš své svaté, aby slovem i~příkladem vedli tvůj lid k~tobě, \grestar{} na jejich přímluvu nás posiluj, abychom vytrvali na cestě, která vede k~věčnému životu. \Rbardot{} Pane, buď pastýřem svého lidu.}}

\newcommand{\trOrationis}{\translatioCantus{Bože, jenž nám dopřáváš radovat
se z~výroční slavnosti svatého tvého vyznavače Havla, uděl dobrotivě,
abychom když slavíme jeho narození, též se řídili podobou jeho skutků.
Skrze…}}
 % Czech translations of the proper texts

\newcommand{\annusEditionis}{2020}

\def\hebinitial#1{%
\leavevmode{\newbox\hebbox\setbox\hebbox\hbox{\hebfont{#1}\hskip 1mm}\kern -\wd\hebbox\hbox{\hebfont{#1}\hskip 1mm}}%
}

%%%% Vicekrat opakovane kousky

\newcommand{\anteOrationem}{
  \rubrica{Ante Orationem, cantatur a Superiore:}

  \pars{Supplicatio Litaniæ.}

  \cuminitiali{}{temporalia/supplicatiolitaniae.gtex}

  \pars{Oratio Dominica.}

  \cuminitiali{}{temporalia/oratiodominica.gtex}
}

\setlength{\columnsep}{30pt} % prostor mezi sloupci

%%%%%%%%%%%%%%%%%%%%%%%%%%%%%%%%%%%%%%%%%%%%%%%%%%%%%%%%%%%%%%%%%%%%%%%%%%%%%%%%%%%%%%%%%%%%%%%%%%%%%%%%%%%%%
\begin{document}

% Here we set the space around the initial.
% Please report to http://home.gna.org/gregorio/gregoriotex/details for more details and options
\grechangedim{afterinitialshift}{2.2mm}{scalable}
\grechangedim{beforeinitialshift}{2.2mm}{scalable}

\grechangedim{interwordspacetext}{0.22 cm plus 0.15 cm minus 0.05 cm}{scalable}%
\grechangedim{annotationraise}{-0.2cm}{scalable}

% Here we set the initial font. Change 38 if you want a bigger initial.
% Emit the initials in red.
\grechangestyle{initial}{\color{red}\fontsize{38}{38}\selectfont}

\pagestyle{empty}

%%%% Titulni stranka
\begin{titulusOfficii}
\nomenFesti{Sabbato infra Hebdom. Ultima Adventus.}
\end{titulusOfficii}

\pars{}

\scriptura{}

\pagebreak

% graphic
\renewcommand{\headrulewidth}{0pt} % no horiz. rule at the header
\fancyhf{}
\pagestyle{fancy}

\cantusSineNeumas

\hora{Ad Matutinum.}

\vspace{2mm}

\cuminitiali{}{temporalia/dominelabiamea.gtex}

\vspace{2mm}

\pars{Invitatorium.} \scriptura{Phil. 4, 4.5}

\vspace{-6mm}

\antiphona{VI}{temporalia/inv-propeestiamsimplex.gtex}

\vfill
\pagebreak

\pars{Hymnus.}

\vspace{-5mm}

{
\grechangedim{interwordspacetext}{0.30 cm plus 0.15 cm minus 0.05 cm}{scalable}%
\antiphona{II}{temporalia/hym-VeniRedemptor.gtex}
\grechangedim{interwordspacetext}{0.22 cm plus 0.15 cm minus 0.05 cm}{scalable}%
}
%{
%\vspace{-5mm}
%\setlength{\columnsep}{0pt} % prostor mezi sloupci
%\input{hym-VeniRedemptor-bohtext.tex}
%\setlength{\columnsep}{30pt} % prostor mezi sloupci
%}

\vfill
\pagebreak

\pars{Psalmus 1.} \scriptura{Ps. 104, 3; \textbf{H99}}

\vspace{-6mm}

\antiphona{D}{temporalia/ant-laeteturcor.gtex}

\vspace{-4mm}

\scriptura{Ps. 104, 1-15}

\vspace{-2mm}

\initiumpsalmi{temporalia/ps104i-initium-d-g-auto.gtex}

\vspace{-1.5mm}

\input{temporalia/ps104i-d-g.tex} \Abardot{}

\vfill
\pagebreak

\pars{Psalmus 2.} \scriptura{Ps. 113, 1; \textbf{H94}}

\vspace{-4mm}

\antiphona{VIII a}{temporalia/ant-domusiacob.gtex}

%\vspace{-5mm}

\scriptura{Ps. 104, 16-27}

%\vspace{-2mm}

\initiumpsalmi{temporalia/ps104ii-initium-viii-a-auto.gtex}

\input{temporalia/ps104ii-viii-a.tex} \Abardot{}

\vfill
\pagebreak

\pars{Psalmus 3.} \scriptura{Ps. 104, 43}

\vspace{-4mm}

\antiphona{IV E}{temporalia/ant-eduxitdeus.gtex}

%\vspace{-5mm}

\scriptura{Ps. 104, 28-45}

%\vspace{-2mm}

\initiumpsalmi{temporalia/ps104iii-initium-iv-E-auto.gtex}

\input{temporalia/ps104iii-iv-E.tex}

\vfill

\antiphona{}{temporalia/ant-eduxitdeus.gtex}

\vfill
\pagebreak

\pars{Psalmus 4.} \scriptura{Ps. 105, 4; \textbf{H100}}

\vspace{-4mm}

\antiphona{E}{temporalia/ant-visitanos.gtex}

%\vspace{-5mm}

\scriptura{Ps. 105, 1-15}

%\vspace{-2mm}

\initiumpsalmi{temporalia/ps105i-initium-e.gtex}

\input{temporalia/ps105i-e.tex}

\vfill

\antiphona{}{temporalia/ant-visitanos.gtex}

\vfill
\pagebreak

\pars{Psalmus 5.} \scriptura{Ps. 117, 6; \textbf{H156}}

\vspace{-8mm}

\antiphona{VIII G}{temporalia/ant-dominusmihi.gtex}

\vspace{-3mm}

\scriptura{Ps. 105, 16-31}

\vspace{-2.5mm}

\initiumpsalmi{temporalia/ps105ii-initium-viii-G-auto.gtex}

\vspace{-1.5mm}

\input{temporalia/ps105ii-viii-G.tex} \Abardot{}

\vspace{-5mm}

\vfill
\pagebreak

\pars{Psalmus 6.} \scriptura{Ps. 105, 44}

\vspace{-4mm}

\antiphona{VII a}{temporalia/ant-cumtribularentur.gtex}

%\vspace{-5mm}

\scriptura{Ps. 105, 32-48}

%\vspace{-2mm}

\initiumpsalmi{temporalia/ps105iii-initium-vii-a-auto.gtex}

\input{temporalia/ps105iii-vii-a.tex}

\vfill

\antiphona{}{temporalia/ant-cumtribularentur.gtex}

\vfill
\pagebreak

\pars{Psalmus 7.} \scriptura{Ps. 106, 8}

\vspace{-4mm}

\antiphona{IV* e}{temporalia/ant-confiteanturdomino.gtex}

%\vspace{-5mm}

\scriptura{Ps. 106, 1-14}

%\vspace{-2mm}

\initiumpsalmi{temporalia/ps106i-initium-iv_-e-auto.gtex}

\input{temporalia/ps106i-iv_-e.tex} \Abardot{}

\vfill
\pagebreak

\pars{Psalmus 8.} \scriptura{Ps. 24, 17; \textbf{H100}}

\vspace{-4mm}

\antiphona{C}{temporalia/ant-denecessitatibus.gtex}

%\vspace{-5mm}

\scriptura{Ps. 106, 15-30}

%\vspace{-2mm}

\initiumpsalmi{temporalia/ps106ii-initium-c-c2-auto.gtex}

\input{temporalia/ps106ii-c-c2.tex}

\vfill

\antiphona{}{temporalia/ant-denecessitatibus.gtex}

\vfill
\pagebreak

\pars{Psalmus 9.} \scriptura{Ps. 106, 24}

\vspace{-4mm}

\antiphona{III a\textsuperscript{2}}{temporalia/ant-ipsividerunt.gtex}

%\vspace{-5mm}

\scriptura{Ps. 106, 31-43}

%\vspace{-2mm}

\initiumpsalmi{temporalia/ps106iii-initium-iii-a2-auto.gtex}

\input{temporalia/ps106iii-iii-a2.tex} \Abardot{}

\vfill
\pagebreak

\pars{Versus} \scriptura{Mc. 1, 3; Is. 40, 3}

% Versus. %%%
\sineinitiali{temporalia/versus-voxclamantis-simplex.gtex}

\vspace{5mm}

\sineinitiali{temporalia/oratiodominica-mat.gtex}

\vspace{5mm}

\pars{Absolutio.}

\cuminitiali{}{temporalia/absolutio-avinculis.gtex}

\vfill
\pagebreak

\vspace{7mm}

\cuminitiali{}{temporalia/benedictio-solemn-ille.gtex}

\lectioi

\noindent \Vbardot{} Tu autem, Dómine, miserére nobis.
\noindent \Rbardot{} Deo grátias.

\vfill
\pagebreak

\responsoriumi

\vfill
\pagebreak

\cuminitiali{}{temporalia/benedictio-solemn-divinum.gtex}

\vspace{7mm}

\lectioii

\noindent \Vbardot{} Tu autem, Dómine, miserére nobis.
\noindent \Rbardot{} Deo grátias.

\vfill
\pagebreak

\responsoriumii

\vfill
\pagebreak

\cuminitiali{}{temporalia/benedictio-solemn-adsocietatem.gtex}

\vspace{7mm}

\lectioiii

\noindent \Vbardot{} Tu autem, Dómine, miserére nobis.
\noindent \Rbardot{} Deo grátias.

\vfill
\pagebreak

\responsoriumiii

\vfill
\pagebreak

\rubrica{Reliqua omittuntur, nisi Laudes separandæ sint.}

\pars{Oratio}

\noindent \Vbardot{} Dómine, exáudi oratiónem meam.

\noindent \Rbardot{} Et clamor meus ad te véniat.

\oratio

\vspace{7mm}

\pars{Conclusio}

\noindent \Vbardot{} Dómine, exáudi oratiónem meam.

\noindent \Rbardot{} Et clamor meus ad te véniat.

\noindent \Vbardot{} Benedicámus Dómino, allelúia, allelúia.

\noindent \Rbardot{} Deo grátias, allelúia, allelúia.

\noindent \Vbardot{} Fidélium ánimæ per misericórdiam Dei requiéscant in pace.

\noindent \Rbardot{} Amen.

\vfill
\pagebreak

\hora{Ad Laudes.} %%%%%%%%%%%%%%%%%%%%%%%%%%%%%%%%%%%%%%%%%%%%%%%%%%%%%
%\sideThumbs{Laudes}

\cantusSineNeumas

\vspace{0.5cm}
\grechangedim{interwordspacetext}{0.18 cm plus 0.15 cm minus 0.05 cm}{scalable}%
\cuminitiali{}{temporalia/deusinadiutorium-communis.gtex}
\grechangedim{interwordspacetext}{0.22 cm plus 0.15 cm minus 0.05 cm}{scalable}%

\vfill

\pars{Psalmus 1.} \scriptura{Is. 45, 8; \textbf{H37}}

\vspace{-4.5mm}

\antiphona{II* a}{temporalia/ant-roratecaeli.gtex}

\scriptura{Psalmus 50.}

\initiumpsalmi{temporalia/ps50-initium-ii_-a-auto.gtex}

\input{temporalia/ps50-ii_-a.tex}

\vfill
\antiphona{}{temporalia/ant-roratecaeli.gtex}

\vfill
\pagebreak

\pars{Psalmus 2.} \scriptura{Ps. 142, 8-9; \textbf{H39}}

\vspace{-4mm}

\antiphona{II* c}{temporalia/ant-adtedominelevavi.gtex}

\scriptura{Psalmus 142.}

\initiumpsalmi{temporalia/ps142-initium-ii_-c-auto.gtex}

\input{temporalia/ps142-ii_-c.tex}

\vfill

\antiphona{}{temporalia/ant-adtedominelevavi.gtex}

\vfill
\pagebreak

\pars{Psalmus 3.} \scriptura{Ex. 15, 2; \textbf{H39}}

\vspace{-4mm}

\antiphona{VIII G}{temporalia/ant-eccedeusmeus.gtex}

\scriptura{Canticum Moysi, Dt. 32, 1-32}

\initiumpsalmi{temporalia/moysis2i-initium-viii-G-auto.gtex}

\input{temporalia/moysis2i-viii-G.tex}

\vfill

\antiphona{}{temporalia/ant-eccedeusmeus.gtex}

\vfill
\pagebreak

\pars{Psalmus 4.} \scriptura{Cf. Dt. 32, 2; \textbf{H43}}

\vspace{-4mm}

\antiphona{II* b}{temporalia/ant-exspectetur.gtex}

\scriptura{Canticum Moysis, Dt. 32, 33-65}

\initiumpsalmi{temporalia/moysis2ii-initium-ii_-B-auto.gtex}

\input{temporalia/moysis2ii-ii_-B.tex}

\vfill

\antiphona{}{temporalia/ant-exspectetur.gtex}

\vfill
\pagebreak

\pars{Psalmus 5.}\scriptura{Cf. Am. 4, 12; \textbf{H30}}

\vspace{-4mm}

\antiphona{II D}{temporalia/ant-paratusesto.gtex}

\scriptura{Psalmus 148.}

\initiumpsalmi{temporalia/ps148-initium-ii-D-auto.gtex}

\input{temporalia/ps148-ii-D-sinedox.tex}

\rubrica{Hic non dicitur Gloria Patri.}

\vfill
\pagebreak

%
\scriptura{Psalmus 149.}

\initiumpsalmi{temporalia/ps149-initium-ii-D-auto.gtex}

\input{temporalia/ps149-ii-D-sinedox.tex}

\rubrica{Hic non dicitur Gloria Patri.}

\vfill
\pagebreak

%
\scriptura{Psalmus 150.}

\initiumpsalmi{temporalia/ps150-initium-ii-D-auto.gtex}

\input{temporalia/ps150-ii-D.tex}

\vfill

\vspace{-6mm}

\antiphona{}{temporalia/ant-paratusesto.gtex} % repeat the antiphon - new page

\vfill
\pagebreak

\lectiobrevis

% preklad Jeruz. bible
%\trCapituliI

\vfill

\responsoriumbreve

%\trResp

\vfill
\pagebreak

\pars{Hymnus}

\cuminitiali{I}{temporalia/hym-MagnisProphetae.gtex}
\vspace{-3mm}
%\input{hym-MagnisProphetae-bohtext.tex}

\vfill
%\pagebreak

\pars{Versus.} \scriptura{Mc. 1, 3; Is. 40, 3}

% Versus. %%%
\sineinitiali{temporalia/versus-voxclamantis.gtex}

%\noindent \trVersus

\vfill
\pagebreak

\benedictus

\vfill
\pagebreak

%\sideThumbs{{\scriptsize{}Fine horarum}}

\ifx\preces\undefined
\rubrica{Ante Orationem, cantatur a Superiore:}

\pars{Supplicatio Litaniæ.}

\cuminitiali{}{temporalia/supplicatiolitaniae.gtex}

\pars{Oratio Dominica.}

\cuminitiali{}{temporalia/oratiodominica.gtex}
\else
\pars{Preces.}

\sineinitiali{}{temporalia/tonusprecum.gtex}

\preces

\vfill

\pars{Oratio Dominica.}

\cuminitiali{}{temporalia/oratiodominicaalt.gtex}

\vfill
\pagebreak

\rubrica{vel:}

\pars{Supplicatio Litaniæ.}

\cuminitiali{}{temporalia/supplicatiolitaniae.gtex}

\vfill

\pars{Oratio Dominica.}

\cuminitiali{}{temporalia/oratiodominica.gtex}
\fi

\vfill
\pagebreak

% Oratio. %%%
\oratio

\vspace{-1mm}
%\trOrationisI

\vfill

\rubrica{Hebdomadarius dicit Dominus vobiscum, vel, absente sacerdote vel diacono, sic concluditur:}

\vspace{2mm}

\antiphona{C}{temporalia/dominusnosbenedicat.gtex}

\rubrica{Postea cantatur a cantore:}

\vspace{2mm}

\cuminitiali{IV}{temporalia/benedicamus-dominica-advequad.gtex}

\vfill

\vspace{1mm}

\end{document}

