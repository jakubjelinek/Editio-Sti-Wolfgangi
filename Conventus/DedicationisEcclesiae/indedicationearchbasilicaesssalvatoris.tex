\newcommand{\titulus}{\dies{Die 9. Novembris.}
\nomenFesti{In Dedicatione Archbasilicæ Ss. Salvatoris.}
\celebratio{Duplex 2. classis.}}
\newcommand{\lectioi}{\pars{Lectio I.} \scriptura{Ap. 21, 9-11}

\noindent De libro Apocalýpsis beáti Ioánnis Apóstoli.

\noindent Et venit unus de septem ángelis habéntibus phíalas plenas septem plagis novíssimis, et locútus est mecum, dicens: Veni, et osténdam tibi sponsam, uxórem Agni. Et sústulit me in spíritu in montem magnum et altum, et osténdit mihi civitátem sanctam Ierúsalem descendéntem de cælo a Deo, habéntem claritátem Dei: et lumen eius símile lápidi pretióso tamquam lápidi iáspidis, sicut crystállum.}
\newcommand{\lectioii}{\pars{Lectio II.} \scriptura{Ap. 21, 12-15}

\noindent Et habébat murum magnum, et altum, habéntem portas duódecim: et in portis ángelos duódecim, et nómina inscrípta, quæ sunt nómina duódecim tríbuum filiórum Israël: ab oriénte portæ tres, et ab aquilóne portæ tres, et ab austro portæ tres, et ab occásu portæ tres. Et murus civitátis habens fundaménta duódecim, et in ipsis duódecim nómina duódecim apostolórum Agni. Et qui loquebátur mecum, habébat mensúram arundíneam áuream, ut metirétur civitátem, et portas eius, et murum.}
\newcommand{\lectioiii}{\pars{Lectio III.} \scriptura{Ap. 21, 16-18}

\noindent Et cívitas in quadro pósita est, et longitúdo eius tanta est quanta et latitúdo: et mensus est civitátem de arúndine áurea per stádia duódecim míllia: et longitúdo, et altitúdo, et latitúdo eius æquália sunt. Et mensus est murum eius centum quadragínta quátuor cubitórum, mensúra hóminis, quæ est ángeli. Et erat structúra muri eius ex lápide iáspide: ipsa vero cívitas aurum mundum símile vitro mundo.}
% LuaLaTeX

\documentclass[a4paper, twoside, 12pt]{article}
\usepackage[latin]{babel}
%\usepackage[landscape, left=3cm, right=1.5cm, top=2cm, bottom=1cm]{geometry} % okraje stranky
%\usepackage[landscape, a4paper, mag=1166, truedimen, left=2cm, right=1.5cm, top=1.6cm, bottom=0.95cm]{geometry} % okraje stranky
\usepackage[landscape, a4paper, mag=1400, truedimen, left=0.5cm, right=0.5cm, top=0.5cm, bottom=0.5cm]{geometry} % okraje stranky

\usepackage{fontspec}
\setmainfont[FeatureFile={junicode.fea}, Ligatures={Common, TeX}, RawFeature=+fixi]{Junicode}
%\setmainfont{Junicode}

% shortcut for Junicode without ligatures (for the Czech texts)
\newfontfamily\nlfont[FeatureFile={junicode.fea}, Ligatures={Common, TeX}, RawFeature=+fixi]{Junicode}

\usepackage{multicol}
\usepackage{color}
\usepackage{lettrine}
\usepackage{fancyhdr}

% usual packages loading:
\usepackage{luatextra}
\usepackage{graphicx} % support the \includegraphics command and options
\usepackage{gregoriotex} % for gregorio score inclusion
\usepackage{gregoriosyms}
\usepackage{wrapfig} % figures wrapped by the text
\usepackage{parcolumns}
\usepackage[contents={},opacity=1,scale=1,color=black]{background}
\usepackage{tikzpagenodes}
\usepackage{calc}
\usepackage{longtable}
\usetikzlibrary{calc}

\setlength{\headheight}{14.5pt}

\input{conventuscommune.tex} % Often used macros
%%%% Preklady jednotlivych zpevu (nektere se opakuji, a je dobre mit je
% vsechny na jedne hromade)

% HOURS ---

\newcommand{\trAntI}{\translatioCantus{Muž boží měl kožený toulec, pečlivě
zavázaný, jenž mu visel na šíji a~často se ho dotýkal.}}

\newcommand{\trAntII}{\translatioCantus{Klíč od~něho tak dobře střežil, že
dokud žil v~těle, nikdo z~jeho žáků nezvěděl, co je uvnitř.}}

\newcommand{\trAntIII}{\translatioCantus{Ale když se odebral z~tohoto
života, schránku otevřeli a~objevili v~ní žíněné roucho a~měděný řetěz
potřísněný krví.}}

\newcommand{\trAntIV}{\translatioCantus{A když prohlédli mistrovo tělo,
nalezli jeho tělo na čtyřech místech hluboce zbrázděno ranami od řetězu.}}

\newcommand{\trAntV}{\translatioCantus{Krev vytékající z~těch ran, místy
prostoupila i~žíněným rouchem.}}

\newcommand{\trCapituli}{\translatioCantus{
Miláčkovi Boha a~lidí,
Mojžíšovi požehnané paměti,~\gredagger{}
dopřál slávu rovnou slávě svatých~\grestar{}
učinil ho mocným na postrach nepřátelům
a~jeho slovy zastavil divy.}}

\newcommand{\trLectioBrevis}{\translatioCantus{
Pamatujte na své představené,
kteří vám hlásali Boží slovo.
Uvažte, jak oni skončili život, a~napodobujte jejich víru.
Ježíš Kristus je stejný včera i~dnes i~navěky.
Nenechte se svést věelijakými cizími naukami.}}

\newcommand{\trRespLaud}{\translatioCantus{Spravedlivého vodil Hospodin~\grestar{}
po přímých stezkách. \Vbardot{} A~ukázal mu Boží království.}}

\newcommand{\trRespLaudB}{\translatioCantus{Na tvých hradbách, Jeruzaléme,
ustanovil jsem strážné;~\grestar{}
budou bdít nad mým lidem. \Vbardot{} Ani ve dne, ani v~noci nesmějí nikdy
mlčet.}}

\newcommand{\trVersus}{\translatioCantus{\Vbardot{} Ústa spravedlivého šeptají moudrost, aleluja.
\Rbardot{} A~jeho jazyk ohlašuje právo, aleluja.}}

\newcommand{\trAntBenedictus}{\translatioCantus{Když na bujné oře vložili
nosítka a~sňali jim uzdu, vydali se přímo k~cele božího muže.}}

\newcommand{\trPreces}{\translatioCantus{
\noindent S vděčností chvalme Krista, dobrého Pastýře, \gredagger{} který dal život za své ovce, \grestar{} a~pokorně ho prosme: \Rbardot{} Pane, buď pastýřem svého lidu.

\noindent Kriste, ty dáváš církvi pastýře, a~jejich službou se ujímáš svého lidu, \grestar{} dej, ať v~lásce těch, kteří nás vedou, poznáváme, jak nás miluješ. \Rbardot{} Pane, buď pastýřem svého lidu.

\noindent Ty stále konáš skrze své zástupce službu pastýře a~učitele, \grestar{} nepřestávej nás nikdy vést prostřednictvím svých služebníků. \Rbardot{} Pane, buď pastýřem svého lidu.

\noindent Ty prokazuješ svému lidu skrze jeho pastýře službu lékaře duše i~těla, \grestar{} ochraňuj náš život a~veď nás ke svatosti. \Rbardot{} Pane, buď pastýřem svého lidu.

\noindent Ty posíláš své svaté, aby slovem i~příkladem vedli tvůj lid k~tobě, \grestar{} na jejich přímluvu nás posiluj, abychom vytrvali na cestě, která vede k~věčnému životu. \Rbardot{} Pane, buď pastýřem svého lidu.}}

\newcommand{\trOrationis}{\translatioCantus{Bože, jenž nám dopřáváš radovat
se z~výroční slavnosti svatého tvého vyznavače Havla, uděl dobrotivě,
abychom když slavíme jeho narození, též se řídili podobou jeho skutků.
Skrze…}}
 % Czech translations of the proper texts

\newcommand{\annusEditionis}{2020}

%%%% Vicekrat opakovane kousky

\newcommand{\anteOrationem}{
  \rubrica{Ante Orationem, cantatur a Superiore:}

  \pars{Supplicatio Litaniæ.}

  \cuminitiali{}{temporalia/supplicatiolitaniae.gtex}

  \pars{Oratio Dominica.}

  \cuminitiali{}{temporalia/oratiodominica.gtex}

  \rubrica{Deinde dicitur ab Hebdomadario:}

  \cuminitiali{}{temporalia/dominusvobiscum-solemnis.gtex}

  \rubrica{In choro monialium loco Dominus vobiscum dicitur:}

  \sineinitiali{temporalia/domineexaudi.gtex}
}

\setlength{\columnsep}{30pt} % prostor mezi sloupci

%%%%%%%%%%%%%%%%%%%%%%%%%%%%%%%%%%%%%%%%%%%%%%%%%%%%%%%%%%%%%%%%%%%%%%%%%%%%%%%%%%%%%%%%%%%%%%%%%%%%%%%%%%%%%
\begin{document}

% Here we set the space around the initial.
% Please report to http://home.gna.org/gregorio/gregoriotex/details for more details and options
\grechangedim{afterinitialshift}{2.2mm}{scalable}
\grechangedim{beforeinitialshift}{2.2mm}{scalable}
\grechangedim{interwordspacetext}{0.22 cm plus 0.15 cm minus 0.05 cm}{scalable}%
\grechangedim{annotationraise}{-0.2cm}{scalable}

% Here we set the initial font. Change 38 if you want a bigger initial.
% Emit the initials in red.
\grechangestyle{initial}{\color{red}\fontsize{38}{38}\selectfont}

\pagestyle{empty}

%%%% Titulni stranka
\begin{titulusOfficii}
\titulus
\end{titulusOfficii}

\vfill

\begin{center}
%Ad usum et secundum consuetudines chori \guillemotright{}Conventus Choralis\guillemotleft.

%Editio Sancti Wolfgangi \annusEditionis
\end{center}

\scriptura{}
\pars{}

\pagebreak

\renewcommand{\headrulewidth}{0pt} % no horiz. rule at the header
\fancyhf{}
\pagestyle{fancy}

\cantusSineNeumas

\pars{Oratio ante divinum Officium.}

\lettrine{{\color{red}A}}{peri,} Dómine, os meum ad benedicéndum nomen sanctum tuum:
munda quoque cor meum ab ómnibus vanis, pervérsis, et aliénis
cogitatiónibus:
intelléctum illúmina, afféctum inflámma,
ut digne, atténte ac devóte hoc Offícium recitáre váleam,
et exaudíri mérear ante conspéctum Divínæ Maiestátis tuæ.
Per Christum, Dóminum nostrum.
\Rbardot{} Amen.

Dómine, in unióne illíus divínæ intentiónis,
qua ipse in terris laudes Deo persolvísti,
has tibi Horas \rubricatum{(vel \textnormal{hanc tibi Horam})} persólvo.

%\trOratioAnteOfficium

\vfill

\pars{Oratio post divinum Officium.}

\rubrica{
  Orationem sequentem devote post Officium recitantibus
  Leo Papa X. defectus, et culpas in eo persolvendo ex humana
  fragilitate contractas, indulsit, et dicitur flexis genibus.
}

\lettrine{{\color{red}S}}{acrosánctæ} et indivíduæ Trinitáti,
crucifíxi Dómini nostri Iesu Christi humanitáti,
beatíssimæ et gloriosíssimæ sempérque Vírginis Maríæ
fecúndæ integritáti, 
et ómnium Sanctórum universitáti
sit sempitérna laus, honor, virtus et glória
ab omni creatúra,
nobísque remíssio ómnium peccatórum,
per infiníta sǽcula sæculórum.
\Rbardot{} Amen.

\noindent \Vbardot{} Beáta víscera Maríæ Virginis, quæ portavérunt
ætérni Patris Fílium.\\
\Rbardot{} Et beáta úbera, quæ lactavérunt Christum Dominum.

\rubrica{Et dicitur secreto \textnormal{Pater noster.} et \textnormal{Ave María.}}

%\trOratioPostOfficium

\vfill

\pars{} \scriptura{}

\hora{Ad Matutinum.} %%%%%%%%%%%%%%%%%%%%%%%%%%%%%%%%%%%%%%%%%%%%%%%%%%%%%%%%%%
%\sideThumbs{Matutinum}

\vspace{2mm}

\cantusSineNeumas

\cuminitiali{}{temporalia/dominelabiamea.gtex}

%\vspace{5mm}

\vfill
\pagebreak

\ifx\invitatorium\undefined
\pars{Invitatorium.} \scriptura{\textbf{Aug. LX 182r}}

\vspace{-2mm}

\antiphona{II}{temporalia/inv-exsultemusdomino.gtex}
\else
\invitatorium
\fi

\vfill
\pagebreak

\pars{Hymnus.}

\vspace{-5mm}

\antiphona{IV}{temporalia/hym-UrbsJerusalem.gtex}

\vfill
\pagebreak

\ifx\matutinum\undefined
\pars{Psalmus 1.} \scriptura{Ps. 23, 7; \textbf{H327}}

\vspace{-4mm}

\antiphona{III a}{temporalia/ant-tolliteportas.gtex}

\scriptura{Psalmus 23.}

\initiumpsalmi{temporalia/ps23-initium-iii-a-auto.gtex}

\input{temporalia/ps23-iii-a.tex} \Abardot{}

\vfill
\pagebreak

\pars{Psalmus 2.} \scriptura{Gn. 28, 21-22; \textbf{H327}}

\vspace{-4mm}

\antiphona{II* b}{temporalia/ant-eritmihi.gtex}

\scriptura{Psalmus 45.}

\initiumpsalmi{temporalia/ps45-initium-ii_-B-auto.gtex}

\input{temporalia/ps45-ii_-B.tex} \Abardot{}

\vfill
\pagebreak

\pars{Psalmus 3.} \scriptura{Ex. 24, 4; \textbf{H327}}

\vspace{-4mm}

\antiphona{VI F}{temporalia/ant-aedificavitmoyses.gtex}

\scriptura{Psalmus 47.}

\initiumpsalmi{temporalia/ps47-initium-vi-F-auto.gtex}

\input{temporalia/ps47-vi-F.tex} \Abardot{}

\vfill
\pagebreak

\pars{Psalmus 4.} \scriptura{Gn. 28, 12; \textbf{H328}}

\vspace{-4mm}

\antiphona{VII c}{temporalia/ant-nonesthic.gtex}

\scriptura{Psalmus 83.}

\initiumpsalmi{temporalia/ps83-initium-vii-c-auto.gtex}

\input{temporalia/ps83-vii-c.tex} \Abardot{}

\vfill
\pagebreak

\pars{Psalmus 5.} \scriptura{Gn. 28, 12; \textbf{H328}}

\vspace{-4mm}

\antiphona{VII a}{temporalia/ant-viditjacobscalam.gtex}

\scriptura{Psalmus 86.}

\initiumpsalmi{temporalia/ps86-initium-vii-a-auto.gtex}

\input{temporalia/ps86-vii-a.tex} \Abardot{}

\vfill
\pagebreak

\pars{Psalmus 6.} \scriptura{Bar. 2, 16}

\vspace{-4mm}

\antiphona{VI F}{temporalia/ant-respicedomine.gtex}

%\vspace{-2mm}

\scriptura{Psalmus 87.}

%\vspace{-2mm}

\initiumpsalmi{temporalia/ps87-initium-vi-F-auto.gtex}

%\vspace{-1mm}

\input{temporalia/ps87-vi-F.tex}

\vfill

\antiphona{}{temporalia/ant-respicedomine.gtex}

\vfill
\pagebreak
\else
\matutinum
\fi

\ifx\matversus\undefined
\pars{Versus.} \scriptura{Ps. 92, 5}

\sineinitiali{temporalia/versus-domumtuam-communis.gtex}
\else
\matversus
\fi

\vspace{5mm}

\sineinitiali{temporalia/oratiodominica-mat.gtex}

\vspace{5mm}

\pars{Absolutio.}

\cuminitiali{}{temporalia/absolutio-exaudi.gtex}

%\trMatAbsolutioI

\vfill
\pagebreak

\cuminitiali{}{temporalia/benedictio-solemn-benedictione.gtex}

%\trMatBenedictioI

\vspace{7mm}

\lectioi

\noindent \Vbardot{} Tu autem, Dómine, miserére nobis.
\noindent \Rbardot{} Deo grátias.

\vfill
\pagebreak

\ifx\responsoriumi\undefined
\pars{Responsorium 1.} \scriptura{\Rbardot{} Cf. 2 Chr. 7, 6 \Vbardot{} Is. 2, 2; \textbf{H327}}

\vspace{-5mm}

\responsorium{I}{temporalia/resp-indedicationetempli-CROCHU.gtex}{}
\else
\responsoriumi
\fi

\vfill
\pagebreak

\cuminitiali{}{temporalia/benedictio-solemn-unigenitus.gtex}

\vspace{7mm}

\lectioii

\noindent \Vbardot{} Tu autem, Dómine, miserére nobis.
\noindent \Rbardot{} Deo grátias.

\vfill
\pagebreak

\ifx\responsoriumii\undefined
\pars{Responsorium 2.} \scriptura{\Rbardot{} Is. 2, 2 \Vbardot{} Ps. 125, 6; \textbf{H328}}

\vspace{-5mm}

\responsorium{II}{temporalia/resp-fundataest-CROCHU.gtex}{}
\else
\responsoriumii
\fi

\vfill
\pagebreak

\cuminitiali{}{temporalia/benedictio-solemn-spiritus.gtex}

\vspace{7mm}

\lectioiii

\noindent \Vbardot{} Tu autem, Dómine, miserére nobis.
\noindent \Rbardot{} Deo grátias.

\vfill
\pagebreak

\ifx\responsoriumiii\undefined
\pars{Responsorium 3.} \scriptura{\Rbardot{} 1 Sam. 7, 29 \Vbardot{} Ps. 92, 5; \textbf{H328}}

\vspace{-5mm}

\responsorium{VIII}{temporalia/resp-benedicdominedomum-CROCHU.gtex}{}
\else
\responsoriumiii
\fi

\vfill
\pagebreak

\ifx\duplexmaius\undefined
% Te Deum

\ifx\tedeumromanum\undefined
\ifx\tedeumsimplex\undefined
\ifx\tedeummonasticum\undefined
{
\pars{Hymnus Ambrosianus} \scriptura{Tonus Solemnis}

\vspace{-2mm}

\grechangedim{interwordspacetext}{0.26 cm plus 0.15 cm minus 0.05 cm}{scalable}%
\cuminitiali{III}{temporalia/tedeum-solemnis-gn.gtex}
\grechangedim{interwordspacetext}{0.22 cm plus 0.15 cm minus 0.05 cm}{scalable}%
}
\else
{
\pars{Hymnus Ambrosianus} \scriptura{Tonus Monasticus}

\vspace{-2mm}

\grechangedim{interwordspacetext}{0.26 cm plus 0.15 cm minus 0.05 cm}{scalable}%
\cuminitiali{III}{temporalia/tedeum-monasticum-am34.gtex}
\grechangedim{interwordspacetext}{0.22 cm plus 0.15 cm minus 0.05 cm}{scalable}%
}
\fi
\else
{
\pars{Hymnus Ambrosianus} \scriptura{Tonus Simplex}

\vspace{-2mm}

\grechangedim{interwordspacetext}{0.30 cm plus 0.15 cm minus 0.05 cm}{scalable}%
\cuminitiali{III}{temporalia/tedeum-simplex-gn.gtex}
\grechangedim{interwordspacetext}{0.22 cm plus 0.15 cm minus 0.05 cm}{scalable}%
}
\fi
\else
{
\pars{Hymnus Ambrosianus} \scriptura{Alio modo, iuxta morem Romanum}

\vspace{-2mm}

\grechangedim{interwordspacetext}{0.26 cm plus 0.15 cm minus 0.05 cm}{scalable}%
\cuminitiali{III}{temporalia/tedeum-romanum-gn.gtex}
\grechangedim{interwordspacetext}{0.22 cm plus 0.15 cm minus 0.05 cm}{scalable}%
}
\fi

\vfill
\pagebreak
\fi

\sineinitiali{temporalia/domineexaudi.gtex}

\vfill

\pars{Oratio.}

\ifx\oratioLaudes\undefined
\ifx\duplexmaius\undefined
\cuminitiali{}{temporalia/oratio.gtex}
\else
\cuminitiali{}{temporalia/oratio2.gtex}
\fi
\else
\oratioLaudes
\fi
%\trOrationis

\vfill

\noindent \Vbardot{} Dómine, exáudi oratiónem meam.
\Rbardot{} Et clamor meus ad te véniat.

\vfill

% Nocturnale Romanum 2002, p. LXXVI Benedicamus Domino seems to match
% the one from Solemn Laudes.
\cuminitiali{V}{temporalia/benedicamus-solemnis-laud.gtex}

\vfill

\noindent \Vbardot{} Fidélium ánimæ per misericórdiam Dei requiéscant in pace.
\Rbardot{} Amen.

%\trFideliumAnimae

\vfill
\pagebreak

\hora{Ad Laudes.} %%%%%%%%%%%%%%%%%%%%%%%%%%%%%%%%%%%%%%%%%%%%%%%%%%%%%%%%%%
%\sideThumbs{Laudes}

% Psalmi festivi (AM33, pg. 721):
% 66 // 92, 99, 62, Dan3, 148+149+150

\vspace{4mm}

\cuminitiali{}{temporalia/deusinadiutorium-alter.gtex}
%\vspace{1cm}

\cantusSineNeumas

\ifx\laudes\undefined

\vspace{4mm}

\pars{Psalmus 1.} \scriptura{Ps. 92, 5; \textbf{H329}}

\vspace{-4mm}

\antiphona{VII a}{temporalia/ant-domumtuam.gtex}

%\trAntI

%\vspace{-2mm}

\scriptura{Ps. 92}

%\vspace{-2mm}

\initiumpsalmi{temporalia/ps92-initium-vii-a-auto.gtex}

%\vspace{-1.5mm}

\input{temporalia/ps92-vii-a.tex} \Abardot{}

\vfill
\pagebreak

\pars{Psalmus 2.} \scriptura{Mt. 7, 24.25; \textbf{H329}}

\antiphona{I f}{temporalia/ant-haecestdomus.gtex}

%\trAntII

\scriptura{Ps. 99}

\initiumpsalmi{temporalia/ps99-initium-i-f-auto.gtex}

\input{temporalia/ps99-i-f.tex} \Abardot{}

\vfill
\pagebreak

\pars{Psalmus 3.} \scriptura{Is. 56, 8; Mt. 21, 13; Mc. 11, 17; \textbf{H329}}

\vspace{-4mm}

\antiphona{I d}{temporalia/ant-domusmea.gtex}

%\vspace{-2mm}

%\trAntIII

\scriptura{Ps. 62.}

\initiumpsalmi{temporalia/ps62-initium-i-d-auto.gtex}

%\vspace{-6mm}

\input{temporalia/ps62-i-d.tex} \Abardot{}

\vfill
\pagebreak

\pars{Psalmus 4.} \scriptura{Mt. 7, 24.25; \textbf{H329}}

\vspace{-4mm}

\antiphona{VIII c}{temporalia/ant-benefundataest.gtex}

\vspace{-2mm}

%\trAntIV

\scriptura{Canticum trium puerorum, Dan. 3, 57-88 et 56}

\vspace{-2mm}

\initiumpsalmi{temporalia/dan3-initium-viii-c-auto.gtex}

\input{temporalia/dan3-viii-c-sinedox.tex}

\rubrica{Hic non dicitur Gloria Patri, neque Amen.}
\vspace{1cm}

\antiphona{}{temporalia/ant-benefundataest.gtex} % repeat the antiphon - new page

\vfill
\pagebreak

\pars{Psalmus 5.} \scriptura{Ap. 21, 18.19; \textbf{H329}}

\vspace{-4mm}

\antiphona{I g}{temporalia/ant-lapidespretiosi.gtex}

\vspace{-2mm}

%\trAntV

\scriptura{Ps. 148}

\vspace{-2mm}

\initiumpsalmi{temporalia/ps148-initium-i-g-auto.gtex}

%\vspace{-1.5mm}

\input{temporalia/ps148-i-g-sinedox.tex}

\rubrica{Hic non dicitur Gloria Patri.}

\vspace{-5mm}

\vfill
\pagebreak

%
\scriptura{Ps. 149}

\initiumpsalmi{temporalia/ps149-initium-i-g-auto.gtex}

\input{temporalia/ps149-i-g-sinedox.tex}

\rubrica{Hic non dicitur Gloria Patri.}

\vfill
\pagebreak

%
\scriptura{Ps. 150}

\initiumpsalmi{temporalia/ps150-initium-i-g-auto.gtex}

\input{temporalia/ps150-i-g.tex}

\antiphona{}{temporalia/ant-lapidespretiosi.gtex} % repeat the antiphon - new page

\vfill
\pagebreak

\cantusSineNeumas

\pars{Capitulum.} \scriptura{Ap. 21, 2}

\cuminitiali{}{temporalia/capitulum-VidiCivitatem.gtex}

% preklad Jeruz. bible
%\trCapituli

\vfill
\pars{Responsorium breve.} \scriptura{Ps. 92, 5}

\antiphona{VI}{temporalia/resp-domumtuam.gtex}

%\trRespVesp

\vfill
\pagebreak

% Hymnus. %%%
\pars{Hymnus.}

{
\grechangedim{interwordspacetext}{0.20 cm plus 0.15 cm minus 0.05 cm}{scalable}%
\ifx\duplexmaius\undefined
\cuminitiali{IV}{temporalia/hym-AngularisFundamentum.gtex}
\else
\cuminitiali{IV}{temporalia/hym-IamBone.gtex}
\fi
\grechangedim{interwordspacetext}{0.22 cm plus 0.15 cm minus 0.05 cm}{scalable}%
}

\vfill

\pars{Versus.} \scriptura{Mt. 7, 24.25}

% Versus. %%%
\sineinitiali{temporalia/versus-haecest.gtex}
    
\noindent %\trVersus

\vfill
\pagebreak

\pars{Canticum Zachariæ.} \scriptura{Gn. 28, 18.17; \textbf{H329}}

\vspace{-4mm}

\antiphona{IV E}{temporalia/ant-manesurgensjacob.gtex}

%\trAntBenedictus

%\vspace{-3mm}

\scriptura{Lc. 1, 68-79}

%\vspace{-2mm}

\initiumpsalmi{temporalia/benedictus-initium-ivsoll-E.gtex}

%\vspace{-1.5mm}

\input{temporalia/benedictus-ivsoll-E.tex}

\vfill

\antiphona{}{temporalia/ant-manesurgensjacob.gtex}

\vfill
\pagebreak

\cantusSineNeumas

\anteOrationem

\pagebreak

% Oratio. %%%
\pars{Oratio.}

\ifx\duplexmaius\undefined
\cuminitiali{}{temporalia/oratio.gtex}
\else
\cuminitiali{}{temporalia/oratio2.gtex}
\fi
%\trOrationis

\vfill

\rubrica{Hebdomadarius dicit iterum Dominus vobiscum, vel cantor dicit:}

\vspace{2mm}

\sineinitiali{temporalia/domineexaudi.gtex}

\rubrica{Postea cantatur a cantore:}

\vspace{2mm}

\ifx\duplexmaius\undefined
\cuminitiali{II}{temporalia/benedicamus-solemnism-laud.gtex}
\else
\cuminitiali{VIII}{temporalia/benedicamus-duplexmajus-laudes.gtex}
\fi

\vspace{1mm}

\vfill
\pagebreak
\else
\laudes
\fi

\hora{Ad Vesperas.} %%%%%%%%%%%%%%%%%%%%%%%%%%%%%%%%%%%%%%%%%%%%%%%%%%%%%
%\sideThumbs{II. Vesperæ}

%\vspace{5mm}
\grechangedim{interwordspacetext}{0.18 cm plus 0.15 cm minus 0.05 cm}{scalable}%
\cuminitiali{}{temporalia/deusinadiutorium-solemnis.gtex}
\grechangedim{interwordspacetext}{0.22 cm plus 0.15 cm minus 0.05 cm}{scalable}%

\vspace{3mm}

%\vfill
%\pagebreak

\pars{Psalmus 1.} \scriptura{Ps. 92, 5; \textbf{H329}}

\vspace{-5mm}

\ifx\temporepaschalis\undefined
\antiphona{VII a}{temporalia/ant-domumtuam.gtex}
\else
\antiphona{VII a}{temporalia/ant-domumtuam-tp.gtex}
\fi

\vspace{-2mm}

%\trAntI

\scriptura{Ps. 109}

\vspace{-2mm}

\initiumpsalmi{temporalia/ps109-initium-vii-a-auto.gtex}

\vspace{-1mm}

\input{temporalia/ps109-vii-a.tex} \Abardot{}

\vfill
\pagebreak

\pars{Psalmus 2.} \scriptura{Mt. 7, 24.25; \textbf{H329}}

\ifx\temporepaschalis\undefined
\antiphona{I f}{temporalia/ant-haecestdomus.gtex}
\else
\antiphona{I f}{temporalia/ant-haecestdomus-tp.gtex}
\fi

%\trAntII

\scriptura{Ps. 121}

\initiumpsalmi{temporalia/ps121-initium-i-f-auto.gtex}

\input{temporalia/ps121-i-f.tex} \Abardot{}

\vfill
\pagebreak

\pars{Psalmus 3.} \scriptura{Is. 56, 7; Mt. 21, 13; Mc. 11, 17; \textbf{H329}}

\ifx\temporepaschalis\undefined
\antiphona{I d}{temporalia/ant-domusmea.gtex}
\else
\antiphona{I d}{temporalia/ant-domusmea-tp.gtex}
\fi

%\trAntIII

\scriptura{Ps. 126}

\initiumpsalmi{temporalia/ps126-initium-i-d-auto.gtex}

\input{temporalia/ps126-i-d.tex} \Abardot{}

\vfill
\pagebreak

\pars{Psalmus 4.} \scriptura{Ap. 21, 18.19; \textbf{H329}}

\ifx\temporepaschalis\undefined
\antiphona{I g}{temporalia/ant-lapidespretiosi.gtex}
\else
\antiphona{I g}{temporalia/ant-lapidespretiosi-tp.gtex}
\fi

%\trAntIV

\scriptura{Ps. 147}

\initiumpsalmi{temporalia/ps147-initium-i-g-auto.gtex}

\input{temporalia/ps147-i-g.tex} \Abardot{}

\vfill
\pagebreak

% Capitulum. %%%
\pars{Capitulum.} \scriptura{Ap. 21, 2}

\cuminitiali{}{temporalia/capitulum-VidiCivitatem.gtex}

% preklad Jeruz. bible
%\trCapituli

\vfill
\pars{Responsorium breve.} \scriptura{Ps. 92, 5}

\vspace{-5mm}

\ifx\temporepaschalis\undefined
\responsorium{VI}{temporalia/resp-domumtuam.gtex}{}
\else
\responsorium{VI}{temporalia/resp-domumtuam-tp.gtex}{}
\fi

%\trRespVesp

\vfill
\pagebreak

% Hymnus. %%%
\pars{Hymnus.}

{
\grechangedim{interwordspacetext}{0.20 cm plus 0.15 cm minus 0.05 cm}{scalable}%
\cuminitiali{IV}{temporalia/hym-UrbsBeata.gtex}
\grechangedim{interwordspacetext}{0.22 cm plus 0.15 cm minus 0.05 cm}{scalable}%
}

\vfill
%\pagebreak

\pars{Versus.} \scriptura{Ps. 92, 5}

% Versus. %%%
\sineinitiali{temporalia/versus-domumtuam.gtex}
    
\noindent %\trVersus

\vfill
\pagebreak

\ifx\magnificat\undefined
\pars{Canticum B. Mariæ V.} \scriptura{Lc. 19, 5.6.9; \textbf{H330}}

\vspace{-6mm}

\antiphona{VIII G\textsuperscript{2}}{temporalia/ant-zachaeefestinans.gtex}

%\trAntMagnificatII

\vspace{-3mm}

\scriptura{Lc. 1, 46-55}

\vspace{-2mm}

\initiumpsalmi{temporalia/magnificat-initium-viiisoll-G2.gtex}

\vspace{-1.5mm}

\input{temporalia/magnificat-viiisoll-G2.tex} \Abardot{}
\else
\magnificat
\fi

\vfill
\pagebreak

\anteOrationem

\pagebreak

%% Oratio. %%%
\pars{Oratio.}

\ifx\oratioLaudes\undefined
\ifx\duplexmaius\undefined
\cuminitiali{}{temporalia/oratio.gtex}
\else
\cuminitiali{}{temporalia/oratio2.gtex}
\fi
\else
\oratioLaudes
\fi
%\trOrationis

\vfill

\rubrica{Hebdomadarius dicit iterum Dominus vobiscum, vel cantor dicit:}

\vspace{2mm}

\sineinitiali{temporalia/domineexaudi.gtex}

\rubrica{Postea cantatur a cantore:}

\vspace{2mm}

\ifx\temporepaschalis\undefined
\ifx\duplexmaius\undefined
\cuminitiali{II}{temporalia/benedicamus-solemnism-2vesp.gtex}
\else
\cuminitiali{II}{temporalia/benedicamus-duplexmajus-vesperae.gtex}
\fi
\else
\cuminitiali{VII}{temporalia/benedicamus-tempore-paschali.gtex}
\fi

\vspace{1mm}

\vfill
\pagebreak

\end{document}

