\documentclass[options]{article}
\begin{document}
	Ex Sermónibus Ioánnis Medíocris Neapolitáni epíscopi
	\begin{flushright}
			(Sermo 7: PLS 4, 785-786)
	\end{flushright}	
	\emph{Dóminus illuminátio mea et salus mea: quem timébo?} Magnum servum, qui sciébat quómodo illuminabátur, unde illuminabátur, qualis erat qui illuminabátur. Vidébat lucem, non istam, quæ ad vésperum tendit; sed illam lucem, \emph{quam óculus non videt.} Mentes ex hac luce illuminátæ non ruunt in peccáta, non offéndunt in vítiis.\\
	Dicébat enim Dóminus: \emph{Ambuláte dum habétis lucem in vobis.} De qua enim luce dicébat, nisi de ipso? qui dixit: \emph{Ego lux veni in mundum}, ut qui vident, non vídeant, et cæci lumen recípiant. Iste ergo Dóminus illuminátio nostra, sol iustítiæ, qui radiávit Ecclésiam suam cathólicam ubíque diffúsam, et clamábat Prophéta in figúra eius: \emph{Dóminus illuminátio mea et salus mea: quem timébo?}\\
	Illuminátus homo intérior non cláudicat, a via non recédit, totum tólerat. Qui de longínquo videt pátriam, adversitátes sústinet, in temporálibus non contristátur, sed in Deo confirmátur; déprimit cor et sústinet, et humilitáte sua patiéntiam habet. Lumen istud verum, \emph{quod illúminat omnem hóminem veniéntem in hunc mundum,} dat se metuéntibus, infúndit cui vult, ubi vult, revélat se cui Fílius vult.\\
	Qui sedébat in ténebris et umbra mortis, in ténebris malórum et umbra peccatórum, orta luce horret sibi et ádicit se, p\'{æ}nitet, erubéscit et dicit: \emph{Dóminus illuminátio mea et salus mea: quem timébo?} Magna salus, fratres mei. Ista salus infirmitátem non timet, lassitúdinem non formídat, dolórem non videt. Debémus ergo plene atque perfécte non tantum lingua, sed étiam mente clamáre et dícere: \emph{Dóminus illuminátio mea et salus mea: quem timébo?} Si ipse illúminat, ipse salvat, quem timébo? Véniant calígines suggestiónum, Dóminus illuminátio mea. Veníre possunt, profícere non possunt, cor nostrum impugnándo, non tamen vincéndo. Véniat c\'{æ}citas cupiditátum, \emph{Dóminus illuminátio mea.} Fortitúdo ergo nostra ipse est, qui se dat nobis et nos ipsos damus illi. Cúrrite ad médicum cum potéstis, ne cum velítis non possítis.\\
	\\
	resp-emittedominesapientiam-CROCHU.gabc
\end{document}