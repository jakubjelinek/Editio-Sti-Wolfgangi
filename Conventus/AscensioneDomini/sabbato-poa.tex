\newcommand{\titulus}{\nomenFesti{Sabbato in Vigilia Pentecostes.}
\celebratio{1 Classis.  Semiduplex.}}
\newcommand{\dominicavelpostoctavam}{Sabbato}
\newcommand{\postoctavam}{Sabbato}
\newcommand{\sabbato}{Sabbato}
\newcommand{\lectioi}{\pars{Lectio I.} \scriptura{Iu. 1, 1-4}

\noindent Incipit Epístola cathólica beáti Iudæ Apóstoli.

\noindent Iudas, Iesu Christi servus, frater autem Iacóbi, his qui sunt in Deo Patre diléctis, et Christo Iesu conservátis, et vocátis. Misericórdia vobis, et pax, et cáritas adimpleátur. Caríssimi, omnem sollicitúdinem fáciens scribéndi vobis de commúni vestra salúte, necésse hábui scríbere vobis: déprecans supercertári semel tráditæ sanctis fídei. Subintroiérunt enim quidam hómines (qui olim præscrípti sunt in hoc iudícium) ímpii, Dei nostri grátiam transferéntes in luxúriam, et solum Dominatórem, et Dóminum nostrum Iesum Christum negántes.}
\newcommand{\lectioii}{\pars{Lectio II.} \scriptura{Iu. 1, 5-8}

\noindent Commonére autem vos volo, sciéntes semel ómnia, quóniam Iesus pópulum de terra Ægýpti salvans, secúndo eos, qui non credidérunt, pérdidit: ángelos vero, qui non servavérunt suum principátum, sed dereliquérunt suum domicílium, in iudícium magni diéi, vínculis ætérnis sub calígine reservávit. Sicut Sódoma, et Gomórrha, et finítimæ civitátes símili modo exfornicátæ, et abeúntes post carnem álteram, factæ sunt exémplum, ignis ætérni pœnam sustinéntes. Simíliter et hi carnem quidem máculant, dominatiónem autem spernunt, maiestátem autem blasphémant.}
\newcommand{\lectioiii}{\pars{Lectio III.} \scriptura{Iu. 1, 9-13}

\noindent Cum Míchael Archángelus cum diábolo dísputans altercarétur de Móysi córpore, non est ausus iudícium inférre blasphémiæ: sed dixit: Imperet tibi Dóminus. Hi autem quæcúmque quidem ignórant, blasphémant: quæcúmque autem naturáliter, tamquam muta animália, norunt, in his corrumpúntur. Væ illis, quia in via Cain abiérunt, et erróre Bálaam mercéde effúsi sunt, et in contradictióne Core periérunt! Hi sunt in épulis suis máculæ, convivántes sine timóre, semetípsos pascéntes, nubes sine aqua, quæ a ventis circumferéntur, árbores autumnáles, infructuósæ, bis mórtuæ, eradicátæ, fluctus feri maris, despumántes suas confusiónes, sídera errántia: quibus procélla tenebrárum serváta est in ætérnum.}
\newcommand{\lectioiv}{\pars{Lectio IV.} \scriptura{Lib. 4. cap. 1. tom. 9}

\noindent Ex Tractátu sancti Augustíni Epíscopi de Symbolo ad Catechúmenos.

\noindent Dum per sacratíssimum crucis signum vos suscépit in útero sancta mater Ecclésia, quæ sicut et fratres vestros cum summa lætítia spiritáliter páriet, nova proles futúra tantæ matris, quoúsque per lavácrum sanctum regenerátos veræ luci restítuat, cóngruis aliméntis eos, quos portat, pascat in útero, et ad diem partus sui lætos læta perdúcat: quóniam non tenétur hæc senténtia Hevæ, quæ in tristítia et gémitu parit fílios; nec ipsos gaudéntes, sed pótius flentes. Hæc enim solvit, quod illa ligáverat: ut prolem, quam per inobediéntiam sui, morti donávit, hæc per obediéntiam restítuat vitæ. Omnia sacraménta, quæ acta sunt et agúntur in vobis per ministérium servórum Dei, exorcísmis, oratiónibus, cánticis spirituálibus, insufflatiónibus, cilício, inclinatióne cervícum, humilitáte pedum, pavor ipse omni securitáte appeténdus: hæc ómnia, ut dixi, escæ sunt, quæ vos refíciunt in útero, ut renátos ex baptísmo hílares vos mater exhíbeat Christo.}
\newcommand{\lectiov}{\pars{Lectio V.}

\noindent Accepístis et sýmbolum, protectiónem parturiéntis contra venéna serpéntis. In Apocalýpsi Ioánnis Apóstoli scriptum est hoc, quod staret draco in conspéctu mulíeris, quæ paritúra erat, ut cum peperísset, natum eius coméderet. Dracónem diábolum esse, nullus vestrum ignórat: mulíerem illam Vírginem Maríam significásse, quæ caput nostrum íntegra íntegrum péperit; quæ étiam ipsa figúram in se sanctæ Ecclésiæ demonstrávit: ut quómodo Fílium páriens, Virgo permánsit, ita et hæc omni témpore membra eius páriat, et virginitátem non amíttat. Ipsas senténtias sacratíssimi sýmboli adiuvánte Dómino exponéndas suscépimus, ut, quid síngulæ contíneant, vestris sénsibus intimémus. Paráta sunt corda vestra, quia exclúsus est inimícus de córdibus vestris.}
\newcommand{\lectiovi}{\pars{Lectio VI.}

\noindent Huic vos renuntiáre proféssi estis: in qua professióne, non homínibus, sed Deo et Angelis eius conscribéntibus dixístis, Renúntio. Renuntiáte non solum vócibus, sed étiam móribus: non tantum sono linguæ, sed et actu vitæ: nec tantum lábiis sonántibus, sed opéribus pronuntiántibus. Scitóte vos cum cállido, antíquo, et veternóso inimíco suscepísse certámen: non in vobis post renuntiatiónem invéniat ópera sua, non iure vos áttrahat in servitútem suam. Deprehénderis enim, et detégeris Christiáne, quando áliud agis, et áliud profitéris: fidélis in nómine, áliud demónstrans in ópere, non tenens promissiónis tuæ fidem: modo ingrédiens ecclésiam oratiónes fúndere, post módicum in spectáculis cum histriónibus impudíce clamáre. Quid tibi cum pompis diáboli, quibus renuntiásti?}
\newcommand{\lectiovii}{\pars{Lectio VII.} \scriptura{Io. 14, 15-21}

\noindent Léctio sancti Evangélii secúndum Ioánnem.

\noindent In illo témpore: Dixit Iesus discípulis suis: Si dilígitis me, mandáta mea serváte. Et ego rogábo Patrem, et álium Paráclitum dabit vobis. Et réliqua.

\scriptura{Tractatus 74 in Ioannem, sub finem, et 75}

\noindent Homilía sancti Augustíni Epíscopi.

\noindent Quod ait, Rogábo Patrem, et álium Paráclitum dabit vobis: osténdit et seípsum esse Paráclitum. Paráclitus enim Latíne dícitur advocátus: et dictum est de Christo: Advocátum habémus ad Patrem, Iesum Christum iustum. Sic autem mundum dixit non posse accípere Spíritum Sanctum, sicut étiam dictum est: Prudéntia carnis inimíca est Deo: legi enim Dei non est subiécta, nec enim potest: velut si dicámus: Iniustítia iustítia esse non potest. Mundum quippe ait hoc loco, mundi signíficans dilectóres: quæ diléctio non est a Patre. Et ídeo dilectióni huius mundi, de qua satágimus ut minuátur et consumátur in nobis, contrária est diléctio Dei, quæ diffúnditur in córdibus nostris per Spíritum Sanctum, qui datus est nobis.}
\newcommand{\lectioviii}{\pars{Lectio VIII.}

\noindent Mundus ergo eum accípere non potest, quia non videt eum, neque scit eum. Non enim habet invisíbiles óculos mundána diléctio, per quos vidéri Spíritus Sanctus potest, qui vidéri nisi invisibíliter non potest. Vos autem, inquit, cognoscétis eum: quia apud vos manébit, et in vobis erit. Erit in eis, ut máneat; non manébit, ut sit: prius est enim esse alícubi, quam manére. Sed ne putárent quod dictum est, Apud vos manébit; ita dictum, quemádmodum apud hóminem hospes visibíliter manére consuévit, expósuit quid díxerit: Apud vos manébit, cum adiúnxit et dixit, In vobis erit.}
\newcommand{\lectioix}{\pars{Lectio IX.}

\noindent Ergo invisibíliter vidétur. Nec, si non sit in nobis, potest esse in nobis eius sciéntia: sic enim a nobis vidétur in nobis et nostra consciéntia. Nam fáciem vidémus altérius, nostram vidére non póssumus: consciéntiam vero nostram vidémus, altérius non vidémus. Sed consciéntia nunquam est nisi in nobis: Spíritus autem Sanctus potest esse étiam sine nobis. Datur quippe ut sit et in nobis: sed vidéri et sciri, quemádmodum vidéndus et sciéndus est, non potest a nobis, si non sit in nobis. Post promissiónem Spíritus Sancti, ne quisquam putáret, quod ita eum Dóminus datúrus fúerit velut pro seípso, ut non et ipse cum eis esset futúrus, adiécit atque ait: Non relínquam vos órphanos, véniam ad vos. Quamvis ergo nos Fílius Dei suo Patri adoptáverit fílios, et eúndem Patrem nos volúerit habére per grátiam, qui eius Pater est per natúram: tamen étiam ipse circa nos patérnum afféctum quodámmodo demónstrat, cum dicit: Non relínquam vos órphanos.}
% LuaLaTeX

\documentclass[a4paper, twoside, 12pt]{article}
\usepackage[latin]{babel}
%\usepackage[landscape, left=3cm, right=1.5cm, top=2cm, bottom=1cm]{geometry} % okraje stranky
%\usepackage[landscape, a4paper, mag=1166, truedimen, left=2cm, right=1.5cm, top=1.6cm, bottom=0.95cm]{geometry} % okraje stranky
\usepackage[landscape, a4paper, mag=1400, truedimen, left=0.5cm, right=0.5cm, top=0.5cm, bottom=0.5cm]{geometry} % okraje stranky

\usepackage{fontspec}
\setmainfont[FeatureFile={junicode.fea}, Ligatures={Common, TeX}, RawFeature=+fixi]{Junicode}
%\setmainfont{Junicode}

% shortcut for Junicode without ligatures (for the Czech texts)
\newfontfamily\nlfont[FeatureFile={junicode.fea}, Ligatures={Common, TeX}, RawFeature=+fixi]{Junicode}

\usepackage{multicol}
\usepackage{color}
\usepackage{lettrine}
\usepackage{fancyhdr}

% usual packages loading:
\usepackage{luatextra}
\usepackage{graphicx} % support the \includegraphics command and options
\usepackage{gregoriotex} % for gregorio score inclusion
\usepackage{gregoriosyms}
\usepackage{wrapfig} % figures wrapped by the text
\usepackage{parcolumns}
\usepackage[contents={},opacity=1,scale=1,color=black]{background}
\usepackage{tikzpagenodes}
\usepackage{calc}
\usepackage{longtable}
\usetikzlibrary{calc}

\setlength{\headheight}{14.5pt}

\input{conventuscommune.tex} % Often used macros
%%%% Preklady jednotlivych zpevu (nektere se opakuji, a je dobre mit je
% vsechny na jedne hromade)

% HOURS ---

\newcommand{\trAntI}{\translatioCantus{Muž boží měl kožený toulec, pečlivě
zavázaný, jenž mu visel na šíji a~často se ho dotýkal.}}

\newcommand{\trAntII}{\translatioCantus{Klíč od~něho tak dobře střežil, že
dokud žil v~těle, nikdo z~jeho žáků nezvěděl, co je uvnitř.}}

\newcommand{\trAntIII}{\translatioCantus{Ale když se odebral z~tohoto
života, schránku otevřeli a~objevili v~ní žíněné roucho a~měděný řetěz
potřísněný krví.}}

\newcommand{\trAntIV}{\translatioCantus{A když prohlédli mistrovo tělo,
nalezli jeho tělo na čtyřech místech hluboce zbrázděno ranami od řetězu.}}

\newcommand{\trAntV}{\translatioCantus{Krev vytékající z~těch ran, místy
prostoupila i~žíněným rouchem.}}

\newcommand{\trCapituli}{\translatioCantus{
Miláčkovi Boha a~lidí,
Mojžíšovi požehnané paměti,~\gredagger{}
dopřál slávu rovnou slávě svatých~\grestar{}
učinil ho mocným na postrach nepřátelům
a~jeho slovy zastavil divy.}}

\newcommand{\trLectioBrevis}{\translatioCantus{
Pamatujte na své představené,
kteří vám hlásali Boží slovo.
Uvažte, jak oni skončili život, a~napodobujte jejich víru.
Ježíš Kristus je stejný včera i~dnes i~navěky.
Nenechte se svést věelijakými cizími naukami.}}

\newcommand{\trRespLaud}{\translatioCantus{Spravedlivého vodil Hospodin~\grestar{}
po přímých stezkách. \Vbardot{} A~ukázal mu Boží království.}}

\newcommand{\trRespLaudB}{\translatioCantus{Na tvých hradbách, Jeruzaléme,
ustanovil jsem strážné;~\grestar{}
budou bdít nad mým lidem. \Vbardot{} Ani ve dne, ani v~noci nesmějí nikdy
mlčet.}}

\newcommand{\trVersus}{\translatioCantus{\Vbardot{} Ústa spravedlivého šeptají moudrost, aleluja.
\Rbardot{} A~jeho jazyk ohlašuje právo, aleluja.}}

\newcommand{\trAntBenedictus}{\translatioCantus{Když na bujné oře vložili
nosítka a~sňali jim uzdu, vydali se přímo k~cele božího muže.}}

\newcommand{\trPreces}{\translatioCantus{
\noindent S vděčností chvalme Krista, dobrého Pastýře, \gredagger{} který dal život za své ovce, \grestar{} a~pokorně ho prosme: \Rbardot{} Pane, buď pastýřem svého lidu.

\noindent Kriste, ty dáváš církvi pastýře, a~jejich službou se ujímáš svého lidu, \grestar{} dej, ať v~lásce těch, kteří nás vedou, poznáváme, jak nás miluješ. \Rbardot{} Pane, buď pastýřem svého lidu.

\noindent Ty stále konáš skrze své zástupce službu pastýře a~učitele, \grestar{} nepřestávej nás nikdy vést prostřednictvím svých služebníků. \Rbardot{} Pane, buď pastýřem svého lidu.

\noindent Ty prokazuješ svému lidu skrze jeho pastýře službu lékaře duše i~těla, \grestar{} ochraňuj náš život a~veď nás ke svatosti. \Rbardot{} Pane, buď pastýřem svého lidu.

\noindent Ty posíláš své svaté, aby slovem i~příkladem vedli tvůj lid k~tobě, \grestar{} na jejich přímluvu nás posiluj, abychom vytrvali na cestě, která vede k~věčnému životu. \Rbardot{} Pane, buď pastýřem svého lidu.}}

\newcommand{\trOrationis}{\translatioCantus{Bože, jenž nám dopřáváš radovat
se z~výroční slavnosti svatého tvého vyznavače Havla, uděl dobrotivě,
abychom když slavíme jeho narození, též se řídili podobou jeho skutků.
Skrze…}}
 % Czech translations of the proper texts

\newcommand{\annusEditionis}{2020}

%%%% Vicekrat opakovane kousky

\newcommand{\anteOrationem}{
  \rubrica{Ante Orationem, cantatur a Superiore:}

  \pars{Supplicatio Litaniæ.}

  \cuminitiali{}{temporalia/supplicatiolitaniae.gtex}

  \pars{Oratio Dominica.}

  \cuminitiali{}{temporalia/oratiodominica.gtex}

  \rubrica{Deinde dicitur ab Hebdomadario:}

  \cuminitiali{}{temporalia/dominusvobiscum-solemnis.gtex}

  \rubrica{In choro monialium loco Dominus vobiscum dicitur:}

  \sineinitiali{temporalia/domineexaudi.gtex}
}

\ifx\dominicavelpostoctavam\undefined
\newcommand{\capitulumLaudes}{\pars{Capitulum.} \scriptura{Ac. 1, 1-2}

\grechangedim{interwordspacetext}{0.12 cm plus 0.15 cm minus 0.05 cm}{scalable}%
\cuminitiali{}{temporalia/capitulum-PrimumQuidem.gtex}
\grechangedim{interwordspacetext}{0.22 cm plus 0.15 cm minus 0.05 cm}{scalable}}
\else
\newcommand{\capitulumLaudes}{\pars{Capitulum.} \scriptura{1 Ptr. 4, 7-8}

\grechangedim{interwordspacetext}{0.12 cm plus 0.15 cm minus 0.05 cm}{scalable}%
\cuminitiali{}{temporalia/capitulum-CarissimiEstote.gtex}
\grechangedim{interwordspacetext}{0.22 cm plus 0.15 cm minus 0.05 cm}{scalable}}
\fi

\setlength{\columnsep}{30pt} % prostor mezi sloupci

%%%%%%%%%%%%%%%%%%%%%%%%%%%%%%%%%%%%%%%%%%%%%%%%%%%%%%%%%%%%%%%%%%%%%%%%%%%%%%%%%%%%%%%%%%%%%%%%%%%%%%%%%%%%%
\begin{document}

% Here we set the space around the initial.
% Please report to http://home.gna.org/gregorio/gregoriotex/details for more details and options
\grechangedim{afterinitialshift}{2.2mm}{scalable}
\grechangedim{beforeinitialshift}{2.2mm}{scalable}
\grechangedim{interwordspacetext}{0.22 cm plus 0.15 cm minus 0.05 cm}{scalable}%
\grechangedim{annotationraise}{-0.2cm}{scalable}

% Here we set the initial font. Change 38 if you want a bigger initial.
% Emit the initials in red.
\grechangestyle{initial}{\color{red}\fontsize{38}{38}\selectfont}

\pagestyle{empty}

\newcommand{\vesperas}{
\pars{Psalmus 2.} \scriptura{Ac. 1, 10; \textbf{H265}}

\vspace{-0.4cm}

\antiphona{VIII G\textsuperscript{2}}{temporalia/ant-cumqueintuerentur.gtex}

\scriptura{Psalmus 110.}

\initiumpsalmi{temporalia/ps110-initium-viii-G2-auto.gtex}

\input{temporalia/ps110-viii-G2.tex} \Abardot{}

\vfill
\pagebreak

\pars{Psalmus 3.} \scriptura{Lc. 24, 50.51; \textbf{H265}}

\vspace{-0.4cm}

\antiphona{IV A*}{temporalia/ant-elevatismanibus.gtex}

\scriptura{Psalmus 111.}

\initiumpsalmi{temporalia/ps111-initium-iv-A_-auto.gtex}

\input{temporalia/ps111-iv-A_.tex} \Abardot{}

\vfill
\pagebreak

\pars{Psalmus 4.} \scriptura{Ac. 1, 9; \textbf{H265}}

\vspace{-0.4cm}

\antiphona{VIII G}{temporalia/ant-videntibusillis.gtex}

\scriptura{Psalmus 112.}

\initiumpsalmi{temporalia/ps112-initium-viii-G-auto.gtex}

\input{temporalia/ps112-viii-G.tex} \Abardot{}

%\vfill

%\vspace{-6mm}

%\antiphona{}{temporalia/ant-videntibusillis.gtex} % repeat the antiphon - new page

\vfill
\pagebreak

\capitulumLaudes

\vfill

\pars{Responsorium breve.} \scriptura{Cf. Ps. 67, 19}

\ifx\festum\undefined
\cuminitiali{VI}{temporalia/resp-ascendenschristusinaltum-simplex.gtex}
\else
\cuminitiali{VI}{temporalia/resp-ascendenschristusinaltum.gtex}
\fi

\vfill
\pagebreak

\pars{Hymnus}

\cuminitiali{IV}{temporalia/hym-JesuNostraRedemptio.gtex}
\vspace{-3mm}
%\begin{translatioMulticol}{3}
Výkupné naše, Ježíši,\\
lásko a tužbo nejčistší,\\
tys Tvůrce věcí stvořených\\
a člověk věků posledních.\\
\\
Jaký tě musil soucit vést,\\
žes naše hříchy za své vzal,\\
že chtěl jsi muky smrti nést,\\
bys kletbu smrti z lidí sňal.\columnbreak

Pronikáš v žalář pekelný,\\
propouštíš z něho zajatce.\\
Vítězi, slávou oděný,\\
po boku trůníš u Otce.\\
\\
Kéž donutí té soucit týž,\\
že rány vin v nás zacelíš,\\
nás podle slibu ušetříš\\
a vlídnou tváří potěšíš.\columnbreak

Ty budiž naší radostí,\\
odměnou ve tvé věčnosti,\\
kéž naše sláva veškerá\\
jen z tebe věčně vyvěrá.\\
Amen.
\end{translatioMulticol}


\vfill
%\pagebreak

\pars{Versus.} \scriptura{Ps. 46, 6}

% Versus. %%%
\ifx\festum\undefined
\sineinitiali{temporalia/versus-ascenditdeus-communis.gtex}
\else
\sineinitiali{temporalia/versus-ascenditdeus.gtex}
\fi

\vfill
\pagebreak
}

%%%% Titulni stranka
\begin{titulusOfficii}
\titulus
\end{titulusOfficii}

% graphic
%\vspace{1.5cm}
%\begin{center}
%\includegraphics[width=8cm]{emmaus.jpg}
%\end{center}

\vfill

\begin{center}
%Ad usum et secundum consuetudines chori \guillemotright{}Conventus Choralis\guillemotleft.

%Editio Sancti Wolfgangi \annusEditionis
\end{center}

\pagebreak

\renewcommand{\headrulewidth}{0pt} % no horiz. rule at the header
\fancyhf{}
\pagestyle{fancy}

\cantusSineNeumas

\ifx\festumveldominica\undefined
\else
\pars{Oratio ante divinum Officium.}

\lettrine{{\color{red}A}}{peri,} Dómine, os meum ad benedicéndum nomen sanctum tuum:
munda quoque cor meum ab ómnibus vanis, pervérsis, et aliénis
cogitatiónibus:
intelléctum illúmina, afféctum inflámma,
ut digne, atténte ac devóte hoc Offícium recitáre váleam,
et exaudíri mérear ante conspéctum Divínæ Maiestátis tuæ.
Per Christum, Dóminum nostrum.
\Rbardot{} Amen.

Dómine, in unióne illíus divínæ intentiónis,
qua ipse in terris laudes Deo persolvísti,
has tibi Horas \rubricatum{(vel \textnormal{hanc tibi Horam})} persólvo.

\vfill

\pars{Oratio post divinum Officium.}

\rubrica{
  Orationem sequentem devote post Officium recitantibus
  Leo Papa X. defectus, et culpas in eo persolvendo ex humana
  fragilitate contractas, indulsit, et dicitur flexis genibus.
}

\lettrine{{\color{red}S}}{acrosánctæ} et indivíduæ Trinitáti,
crucifíxi Dómini nostri Iesu Christi humanitáti,
beatíssimæ et gloriosíssimæ sempérque Vírginis Maríæ
fecúndæ integritáti, 
et ómnium Sanctórum universitáti
sit sempitérna laus, honor, virtus et glória
ab omni creatúra,
nobísque remíssio ómnium peccatórum,
per infiníta sǽcula sæculórum.
\Rbardot{} Amen.

\noindent \Vbardot{} Beáta víscera Maríæ Virginis, quæ portavérunt
ætérni Patris Fílium.\\
\Rbardot{} Et beáta úbera, quæ lactavérunt Christum Dominum.

\rubrica{Et dicitur secreto \textnormal{Pater noster.} et \textnormal{Ave María.}}

\vfill

\hora{Ad I. Vesperas.} %%%%%%%%%%%%%%%%%%%%%%%%%%%%%%%%%%%%%%%%%%%%%%%%%%%%%
%\sideThumbs{I. Vesperæ}

\vspace{0.5cm}
\grechangedim{interwordspacetext}{0.18 cm plus 0.15 cm minus 0.05 cm}{scalable}%
\cuminitiali{}{temporalia/deusinadiutorium-solemnis.gtex}
\grechangedim{interwordspacetext}{0.22 cm plus 0.15 cm minus 0.05 cm}{scalable}%

\vfill
\pagebreak

\pars{Psalmus 1.} \scriptura{Ac. 1, 11; \textbf{H265}}

\vspace{-0.4cm}

\antiphona{VII a}{temporalia/ant-virigalilaeiquidaspicitis.gtex}

\scriptura{Psalmus 109.}

\initiumpsalmi{temporalia/ps109-initium-vii-a-auto.gtex}

\input{temporalia/ps109-vii-a.tex} \Abardot{}

\vspace{-1cm}

\vfill
\pagebreak

\vesperas
\ifx\dominicavelpostoctavam\undefined
\pars{Canticum B. Mariæ V.} \scriptura{Io. 17, 6.9}

\vspace{-3mm}

{
\grechangedim{interwordspacetext}{0.18 cm plus 0.15 cm minus 0.05 cm}{scalable}%
\antiphona{VI F}{temporalia/ant-patermanifestavi.gtex}
\grechangedim{interwordspacetext}{0.22 cm plus 0.15 cm minus 0.05 cm}{scalable}%
}

\vspace{-2mm}

\scriptura{Lc. 1, 46-55}

\vspace{-2mm}

\cantusSineNeumas
\initiumpsalmi{temporalia/magnificat-initium-visoll-F.gtex}

\input{temporalia/magnificat-visoll-F.tex} \Abardot{}
\else
\pars{Canticum B. Mariæ V.} \scriptura{Io. 15, 26; \textbf{H267}}

\vspace{-6mm}

{
\grechangedim{interwordspacetext}{0.18 cm plus 0.15 cm minus 0.05 cm}{scalable}%
\antiphona{VIII G}{temporalia/ant-cumveneritparaclitus.gtex}
\grechangedim{interwordspacetext}{0.22 cm plus 0.15 cm minus 0.05 cm}{scalable}%
}

\vspace{-3mm}

\scriptura{Lc. 1, 46-55}

\vspace{-2mm}

\cantusSineNeumas
\initiumpsalmi{temporalia/magnificat-initium-viiisoll-G.gtex}

\vspace{-1.5mm}

\input{temporalia/magnificat-viiisoll-G.tex} \Abardot{}
\fi

\vspace{-1cm}

\vfill
\pagebreak

%\sideThumbs{{\scriptsize{}Fine horarum}}

\anteOrationem

\pagebreak

% Oratio. %%%
\cuminitiali{}{temporalia/oratio.gtex}

\vspace{-1mm}

\vfill

\rubrica{Hebdomadarius dicit iterum Dominus vobiscum, vel cantor dicit:}

\vspace{2mm}

\sineinitiali{temporalia/domineexaudi.gtex}

\rubrica{Postea cantatur a cantore:}

\vspace{2mm}

\ifx\festum\undefined
\cuminitiali{VII}{temporalia/benedicamus-tempore-paschali.gtex}
\else
\cuminitiali{II}{temporalia/benedicamus-solemnism-1vesp.gtex}
\fi

\vspace{1mm}

\vfill
\pagebreak
\fi

\ifx\festum\undefined
\else
\hora{Ad Completorium.} %%%%%%%%%%%%%%%%%%%%%%%%%%%%%%%%%%%%%%%%%%%%%%%%%%%%%%%%%%
%\sideThumbs{{\scriptsize{}Completorium}}

\rubrica{Lector petit benedictionem, dicens:}

\cuminitiali{}{temporalia/jubedomnebenedicere.gtex}

\vfill

\pars{Benedictio.}

\cuminitiali{}{temporalia/benedictio-noctemquietam.gtex}

\vfill

\pars{Lectio brevis.} \scriptura{1Ptr. 5, 8-9}

\cuminitiali{}{temporalia/lectiobrevis-fratressobrii.gtex}

\vfill

\noindent \Vbardot{} Adiutórium nostrum in nómine Dómini.

\noindent \Rbardot{} Qui fecit cælum, et terram.

\vfill
\pagebreak

\pars{Confessio.}

\noindent Confíteor Deo omnipoténti, beátæ Maríæ semper Vírgini, beáto
Michaéli Archángelo, beáto Ioánni Baptístæ, sanctis Apóstolis Petro
et Paulo, ómnibus Sanctis, et vobis fratres: quia peccávi nimis cogitatióne,
verbo et ópere: mea culpa, mea culpa, mea máxima culpa.
Ideo precor beátam Maríam semper Vírginem, beátum Michaélem
Archángelum, beátum Ioánnem Baptístam, sanctos Apóstolos Petrum
et Paulum, omnes Sanctos, et vos fratres, oráre pro me ad Dóminum
Deum nostrum.

\vfill

\noindent \Vbardot{} Misereátur nostri omnípotens Deus, et, dimíssis peccátis nostris, perdúcat
nos ad vitam ætérnam. \Rbardot{} Amen.

\vfill

\noindent \Vbardot{} Indulgéntiam, absolutiónem et remissiónem peccatórum nostrórum tríbuat nobis
omnípotens et miséricors Dóminus. \Rbardot{} Amen.

\vfill

\rubrica{Et facta absolutione dicitur:}

\sineinitiali{temporalia/convertenosdeus.gtex}

\vfill

\cuminitiali{}{temporalia/deusinadiutorium-communis.gtex}

\vfill
\pagebreak

\pars{Psalmus 1.}

\antiphona{VIII G}{temporalia/ant-alleluia-compl.gtex}

\scriptura{Ps. 4}

\initiumpsalmi{temporalia/ps4-initium-viii-G-auto.gtex}

\input{temporalia/ps4-viii-G.tex}

\vfill
\pagebreak

\pars{Psalmus 2.} \scriptura{Ps. 90}

\initiumpsalmi{temporalia/ps90-initium-viii-G-auto.gtex}

\input{temporalia/ps90-viii-G.tex}

\pagebreak

\pars{Psalmus 3.} \scriptura{Ps. 133}

\initiumpsalmi{temporalia/ps133-initium-viii-G-auto.gtex}

\input{temporalia/ps133-viii-G.tex}

\vfill

\antiphona{VIII G}{temporalia/ant-alleluia-compl.gtex}

\vfill

\pars{Hymnus.}

\antiphona{I}{temporalia/hym-TeLucis.gtex}
%\input{hym-TeLucis-bohtext.tex}

\pagebreak

\pars{Capitulum.} \scriptura{Ier. 14, 9}

\cuminitiali{}{temporalia/capitulum-tuautem.gtex}

\vfill

\pars{Responsorium breve.} \scriptura{Ps. 30, 6}

\cuminitiali{VI}{temporalia/resp-inmanus-tp.gtex}

\vfill

\pars{Versus.} \scriptura{Ps. 16, 8}

\sineinitiali{temporalia/versus-custodi.gtex}

\vfill
\pagebreak

\cantusCumNeumis

\pars{Canticum Simeonis.}

\vspace{-3mm}

\antiphona{III a}{temporalia/ant-salvanos-antiquo-tp.gtex}

\scriptura{Lc. 2, 29-32}

\vspace{-2mm}

\initiumpsalmi{temporalia/nuncdimittis-initium-iii-a-auto.gtex}

\input{temporalia/nuncdimittis-iii-a.tex} \Abardot{}

\vfill

\rubrica{Ante Orationem, cantatur a Superiore:}

\pars{Supplicatio Litaniæ.}

\cuminitiali{}{temporalia/supplicatiolitaniae.gtex}

\vspace{7mm}

\pars{Oratio Dominica.}

\noindent Pater noster.

\vfill
\pagebreak

\sineinitiali{temporalia/domineexaudi-simplex.gtex}

\vspace{7mm}

\pars{Oratio.}

\cantusSineNeumas

\cuminitiali{}{temporalia/oratio-visita.gtex}

\vfill

%\sineinitiali{temporalia/domineexaudi-communis.gtex}

\noindent \Vbardot{} Dómine, exáudi oratiónem meam. \Rbardot{} Et clamor meus ad te véniat.

\vfill

%\vfill

\sineinitiali{temporalia/benedicamus-minor.gtex}

\vfill

\pars{Benedictio.}

\noindent Benedícat et custódiat nos omnípotens et miséricors Dóminus,~\gredagger{}
Pater, et Fílius, et Spíritus Sanctus. \Rbardot{} Amen.

\vfill
\pagebreak

\pars{Antiphona finalis B. M. V.}

\antiphona{V}{temporalia/an_regina_caeli_simplex.gtex}

\vspace{7mm}

\sineinitiali{temporalia/versus-gaude.gtex}

\vfill
\pagebreak
\fi

\hora{Ad Matutinum.} %%%%%%%%%%%%%%%%%%%%%%%%%%%%%%%%%%%%%%%%%%%%%%%%%%%%%
%\sideThumbs{Matutinum}

\vspace{2mm}

\cuminitiali{}{temporalia/dominelabiamea.gtex}

\vspace{2mm}

\pars{Invitatorium.} \scriptura{Cantor; Psalmus 94; \textbf{H261}}

\vspace{-6mm}

\antiphona{V}{temporalia/inv-alleluiachristumdominum.gtex}

\vfill
\pagebreak

\pars{Hymnus.}

\vspace{-5mm}

\scriptura{Anonymus X. sæculi; \textbf{AR488}}

\antiphona{IV}{temporalia/hym-AEterneRexAltissime.gtex}
%{
%\vspace{-5mm}
%\setlength{\columnsep}{0pt} % prostor mezi sloupci
%\input{hym-RexSempiterne-bohtext.tex}
%\setlength{\columnsep}{30pt} % prostor mezi sloupci
%}

\vfill
\pagebreak

\subhora{In I. Nocturno}

\pars{Psalmus 1.} \scriptura{Ps. 8, 2; \textbf{H262}}

%\vspace{-5mm}

\antiphona{IV A*}{temporalia/ant-elevataestmagnificentiatua.gtex}

%\vspace{-5mm}

\scriptura{Ps. 8}

%\vspace{-2mm}

\initiumpsalmi{temporalia/ps8-initium-iv-A_-auto.gtex}

\input{temporalia/ps8-iv-A_.tex} \Abardot{}

\vfill
\pagebreak

\pars{Psalmus 2.} \scriptura{Ps. 10, 5; \textbf{H262}}

%\vspace{-5mm}

\antiphona{VIII c}{temporalia/ant-dominusintemplosanctosuo.gtex}

%\vspace{-5mm}

\scriptura{Ps. 10}

\initiumpsalmi{temporalia/ps10-initium-viii-C-auto.gtex}

\input{temporalia/ps10-viii-C.tex} \Abardot{}

\vfill
\pagebreak

\pars{Psalmus 3.} \scriptura{Ps. 18, 7; \textbf{H262}}

%\vspace{-5mm}

\antiphona{IV A*}{temporalia/ant-asummocoeloegressioejus.gtex}

%\vspace{-5mm}

\scriptura{Ps. 18}

\initiumpsalmi{temporalia/ps18-initium-iv-A_-auto.gtex}

\input{temporalia/ps18-iv-A_.tex}

\vfill

\antiphona{}{temporalia/ant-asummocoeloegressioejus.gtex}

\vfill
\pagebreak

\noindent \Vbardot{} Ascéndit Deus in iubilatióne, allelúia.
\noindent \Rbardot{} Et Dóminus in voce tubæ, allelúia.

\vspace{5mm}

\sineinitiali{temporalia/oratiodominica-mat.gtex}

\vspace{5mm}

\pars{Absolutio.}

\cuminitiali{}{temporalia/absolutio-exaudi.gtex}

\vfill
\pagebreak

\cuminitiali{}{temporalia/benedictio-solemn-benedictione.gtex}

\vspace{7mm}

\lectioi

\noindent \Vbardot{} Tu autem, Dómine, miserére nobis.
\noindent \Rbardot{} Deo grátias.

\vfill
\pagebreak

\pars{Responsorium 1.} \scriptura{\Rbardot{} Ac. 1, 3.9; \Vbardot{} ibid. 1, 4; \textbf{H262}}

\vspace{-5mm}

\responsorium{III transp.}{temporalia/resp-postpassionemsuam-sinedox.gtex}{}

\vfill
\pagebreak

\cuminitiali{}{temporalia/benedictio-solemn-unigenitus.gtex}

\vspace{7mm}

\lectioii

\noindent \Vbardot{} Tu autem, Dómine, miserére nobis.
\noindent \Rbardot{} Deo grátias.

\vfill
\pagebreak

\pars{Responsorium 2.} \scriptura{\Rbardot{} Cantor; \Vbardot{} Ps. 18, 7; \textbf{H262}}

\vspace{-5mm}

\responsorium{II}{temporalia/resp-omnispulchritudodomini-sinedox.gtex}{}

\vfill
\pagebreak

\cuminitiali{}{temporalia/benedictio-solemn-spiritus.gtex}

\vspace{7mm}

\lectioiii

\noindent \Vbardot{} Tu autem, Dómine, miserére nobis.
\noindent \Rbardot{} Deo grátias.

\vfill
\pagebreak

\pars{Responsorium 3.} \scriptura{\Rbardot{} Ps. 20, 14; \Vbardot{} Ps. 8, 2; \textbf{H262}}

\vspace{-5mm}

\responsorium{VII}{temporalia/resp-exaltaredomine-cumdox.gtex}{}

\vfill
\pagebreak

\subhora{In II. Nocturno}

\pars{Psalmus 4.} \scriptura{Ps. 20, 14; \textbf{H262}}

\vspace{-5mm}

\antiphona{IV A*}{temporalia/ant-exaltaredomine.gtex}

\vspace{-2mm}

\scriptura{Ps. 20}

\vspace{-1mm}

\initiumpsalmi{temporalia/ps20-initium-iv-A_-auto.gtex}

\input{temporalia/ps20-iv-A_.tex} \Abardot{}

\vfill
\pagebreak

\pars{Psalmus 5.} \scriptura{Ps. 29, 2; \textbf{H262}}

\vspace{-5.5mm}

\antiphona{VIII G}{temporalia/ant-exaltabotedomine.gtex}

\vspace{-3mm}

\scriptura{Ps. 29}

\vspace{-2mm}

\initiumpsalmi{temporalia/ps29-initium-viii-G-auto.gtex}

\vspace{-1.5mm}

\input{temporalia/ps29-viii-G.tex} \Abardot{}

\vspace{-1cm}

\vfill
\pagebreak

\pars{Psalmus 6.} \scriptura{Ps. 46, 6; \textbf{H262}}

%\vspace{-5mm}

\antiphona{IV A*}{temporalia/ant-ascenditdeus.gtex}

%\vspace{-5mm}

\scriptura{Ps. 46}

\initiumpsalmi{temporalia/ps46-initium-iv-A_-auto.gtex}

\input{temporalia/ps46-iv-A_.tex} \Abardot{}

\vfill
\pagebreak

\noindent \Vbardot{} Ascéndens Christus in altum, allelúia.
\noindent \Rbardot{} Captívam duxit captivitátem, allelúia.

\vspace{5mm}

\sineinitiali{temporalia/oratiodominica-mat.gtex}

\vspace{5mm}

\pars{Absolutio.}

\cuminitiali{}{temporalia/absolutio-ipsius.gtex}

\vfill
\pagebreak

\cuminitiali{}{temporalia/benedictio-solemn-deus.gtex}

\vspace{7mm}

\lectioiv

\noindent \Vbardot{} Tu autem, Dómine, miserére nobis.
\noindent \Rbardot{} Deo grátias.

\vfill
\pagebreak

\pars{Responsorium 4.} \scriptura{\Rbardot{} Tob. 12, 20 \& Io. 14, 27; \Vbardot{} Io. 16, 7; \textbf{H263}}

\vspace{-5mm}

\responsorium{IV}{temporalia/resp-tempusest-sinedox.gtex}{}

\vfill
\pagebreak

\cuminitiali{}{temporalia/benedictio-solemn-christus.gtex}

\vspace{7mm}

\lectiov

\noindent \Vbardot{} Tu autem, Dómine, miserére nobis.
\noindent \Rbardot{} Deo grátias.

\vfill
\pagebreak

\pars{Responsorium 5.} \scriptura{\Rbardot{} Cantor super Ioannem; \Vbardot{} Io. 14, 16; \textbf{H263}}

\vspace{-5mm}

\responsorium{III}{temporalia/resp-nonconturbetur-sinedox.gtex}{}

\vfill
\pagebreak

\cuminitiali{}{temporalia/benedictio-solemn-ignem.gtex}

\vspace{7mm}

\lectiovi

\noindent \Vbardot{} Tu autem, Dómine, miserére nobis.
\noindent \Rbardot{} Deo grátias.

\vfill
\pagebreak

\pars{Responsorium 6.} \scriptura{\Rbardot{} Eph. 4, 8; \Vbardot{} Ps. 46, 6; \textbf{H263}}

\vspace{-5mm}

\responsorium{IV}{temporalia/resp-ascendensinaltum-cumdox.gtex}{}

\vfill
\pagebreak

\subhora{In III. Nocturno}

\pars{Psalmus 7.} \scriptura{Ps. 96, 9; \textbf{H263}}

\vspace{-5mm}

\antiphona{VI F}{temporalia/ant-nimisexaltatusest.gtex}

\vspace{-4mm}

\scriptura{Ps. 96}

%\vspace{-2mm}

\initiumpsalmi{temporalia/ps96-initium-vi-F-auto.gtex}

\input{temporalia/ps96-vi-F.tex} \Abardot{}

\vfill
\pagebreak

\pars{Psalmus 8.} \scriptura{Ps. 98, 2; \textbf{H263}}

\vspace{-5mm}

\antiphona{VI F}{temporalia/ant-dominusinsion.gtex}

\vspace{-4mm}

\scriptura{Ps. 98}

\initiumpsalmi{temporalia/ps98-initium-vi-F-auto.gtex}

\input{temporalia/ps98-vi-F.tex} \Abardot{}

\vfill
\pagebreak

\pars{Psalmus 9.} \scriptura{Ps. 102, 19; \textbf{H263}}

\vspace{-5mm}

\antiphona{VI F}{temporalia/ant-dominusincoelo.gtex}

\vspace{-4mm}

\scriptura{Ps. 102}

\initiumpsalmi{temporalia/ps102-initium-vi-F-auto.gtex}

\input{temporalia/ps102-vi-F.tex}

\vfill

\antiphona{}{temporalia/ant-dominusincoelo.gtex}

\vfill
\pagebreak

\noindent \Vbardot{} Ascéndo ad Patrem meum, et Patrem vestrum, allelúia.
\noindent \Rbardot{} Deum meum, et Deum vestrum, allelúia.

\vspace{5mm}

\sineinitiali{temporalia/oratiodominica-mat.gtex}

\vspace{5mm}

\pars{Absolutio.}

\cuminitiali{}{temporalia/absolutio-avinculis.gtex}

\vfill
\pagebreak

\cuminitiali{}{temporalia/benedictio-solemn-evangelica.gtex}

\vspace{7mm}

\lectiovii

\noindent \Vbardot{} Tu autem, Dómine, miserére nobis.
\noindent \Rbardot{} Deo grátias.

\vfill
\pagebreak

\pars{Responsorium 7.} \scriptura{\Rbardot{} Io. 14, 16.17; \Vbardot{} ibid. 16, 7; \textbf{Sar.275}}

\vspace{-5mm}

\responsorium{III}{temporalia/resp-egorogabopatrem-sinedox.gtex}{}

\vfill
\pagebreak

\cuminitiali{}{temporalia/benedictio-solemn-divinum.gtex}

\vspace{7mm}

\lectioviii

\noindent \Vbardot{} Tu autem, Dómine, miserére nobis.
\noindent \Rbardot{} Deo grátias.

\vfill
\pagebreak

\ifx\dominicavelpostoctavam\undefined
\pars{Responsorium 8.} \scriptura{\Rbardot{} Ps. 103, 3; \Vbardot{} Ps. 103, 1.2; \textbf{H264}}

\vspace{-5mm}

\responsorium{II}{temporalia/resp-ponitnubem-cumdox.gtex}{}
\else
\pars{Responsorium 8.} \scriptura{\Rbardot{} Io. 16, 7; \Vbardot{} ibid. 16, 13; \textbf{Sar.272}}

\vspace{-5mm}

\responsorium{III}{temporalia/resp-sienimnonabiero-cumdox.gtex}{}
\fi

\vfill
\pagebreak

\cuminitiali{}{temporalia/benedictio-solemn-adsocietatem.gtex}

\vspace{7mm}

\lectioix

\noindent \Vbardot{} Tu autem, Dómine, miserére nobis.
\noindent \Rbardot{} Deo grátias.

\vfill
\pagebreak

% Te Deum

%\pars{Hymnus Ambrosianus}

\vspace{-5mm}

\cuminitiali{III}{temporalia/tedeum-solemnis.gtex}

\vfill
\pagebreak

\rubrica{Reliqua omittuntur, nisi Laudes separandæ sint.}

\pars{Oratio}

\noindent \Vbardot{} Dómine, exáudi oratiónem meam.

\noindent \Rbardot{} Et clamor meus ad te véniat.

Orémus:

\ifx\dominicavelpostoctavam\undefined
\noindent Concéde, quǽsumus, omnípotens Deus:~\gredagger{} ut, qui hodiérna die Unigénitum tuum, Redemptórem nostrum, ad cælos ascendísse crédimus;~\grestar{} ipsi quoque mente in cæléstibus habitémus. Per eúmdem Dóminum.
\else
\noindent Omnípotens sempitérne Deus:~\gredagger{} fac nos tibi semper et devótam gérere voluntátem;~\grestar{} et majestáti tuæ sincéro corde servíre. Per Dóminum.
\fi

\noindent \Rbardot{} Amen.

\vspace{7mm}

\pars{Conclusio}

\noindent \Vbardot{} Dómine, exáudi oratiónem meam.

\noindent \Rbardot{} Et clamor meus ad te véniat.

\noindent \Vbardot{} Benedicámus Dómino, allelúia, allelúia.

\noindent \Rbardot{} Deo grátias, allelúia, allelúia.

\noindent \Vbardot{} Fidélium ánimæ per misericórdiam Dei requiéscant in pace.

\noindent \Rbardot{} Amen.

\vfill
\pagebreak

\hora{Ad Laudes.} %%%%%%%%%%%%%%%%%%%%%%%%%%%%%%%%%%%%%%%%%%%%%%%%%%%%%
%\sideThumbs{Laudes}

\cantusSineNeumas

\ifx\postoctavam\undefined
\vspace{0.5cm}
\grechangedim{interwordspacetext}{0.18 cm plus 0.15 cm minus 0.05 cm}{scalable}%
\ifx\festumveldominica\undefined
\cuminitiali{}{temporalia/deusinadiutorium-communis.gtex}
\else
\cuminitiali{}{temporalia/deusinadiutorium-alter.gtex}
\fi
\grechangedim{interwordspacetext}{0.22 cm plus 0.15 cm minus 0.05 cm}{scalable}%

\vfill
%\pagebreak
\else
\rubrica{Absolute incipitur Officium ab Antiphona primi Psalmi.}

\vspace{7mm}
\fi

\pars{Psalmus 1.} \scriptura{Ac. 1, 11; \textbf{H265}}

\vspace{-0.4cm}

\antiphona{VII a}{temporalia/ant-virigalilaeiquidaspicitis.gtex}

\scriptura{Psalmus 92.}

\initiumpsalmi{temporalia/ps92-initium-vii-a-auto.gtex}

\ifx\postoctavam\undefined
\input{temporalia/ps92-vii-a.tex}

\vfill

\vspace{-1cm}

\antiphona{}{temporalia/ant-virigalilaeiquidaspicitis.gtex}
\else
\input{temporalia/ps92-vii-a.tex} \Abardot{}
\fi

\vfill
\pagebreak

\pars{Psalmus 2.} \scriptura{Ac. 1, 10; \textbf{H265}}

\vspace{-0.4cm}

\antiphona{VIII G\textsuperscript{2}}{temporalia/ant-cumqueintuerentur.gtex}

\scriptura{Psalmus 99.}

\initiumpsalmi{temporalia/ps99-initium-viii-G2-auto.gtex}

\input{temporalia/ps99-viii-G2.tex} \Abardot{}

\vfill
\pagebreak

\pars{Psalmus 3.} \scriptura{Lc. 24, 50.51; \textbf{H265}}

\vspace{-0.4cm}

\antiphona{IV A*}{temporalia/ant-elevatismanibus.gtex}

\scriptura{Psalmus 62.}

\initiumpsalmi{temporalia/ps62-initium-iv-A_-auto.gtex}

\input{temporalia/ps62-iv-A_.tex} \Abardot{}

%\vfill

%\vspace{-6mm}

%\antiphona{}{temporalia/ant-elevatismanibus.gtex} % repeat the antiphon - new page

\vfill
\pagebreak

\pars{Psalmus 4.} \scriptura{\textbf{H265}}

\vspace{-0.4cm}

\antiphona{VIII G\textsuperscript{2}}{temporalia/ant-exaltateregemregum.gtex}

\scriptura{Canticum trium puerorum, Dan. 3, 57-88 et 56}

\initiumpsalmi{temporalia/dan3-initium-viii-G2-auto.gtex}

\input{temporalia/dan3-viii-G2-sinedox.tex}

\rubrica{Hic non dicitur Gloria Patri, neque Amen.}

\vfill

\vspace{-6mm}

\antiphona{}{temporalia/ant-exaltateregemregum.gtex} % repeat the antiphon - new page

\vfill
\pagebreak

\pars{Psalmus 5.} \scriptura{Ac. 1, 9; \textbf{H265}}

\vspace{-0.4cm}

\antiphona{VIII G}{temporalia/ant-videntibusillis.gtex}

\scriptura{Psalmus 148.}

\initiumpsalmi{temporalia/ps148-initium-viii-G-auto.gtex}

\input{temporalia/ps148-viii-G-sinedox.tex}

\rubrica{Hic non dicitur Gloria Patri.}

\vfill
\pagebreak

%
\scriptura{Psalmus 149.}

\initiumpsalmi{temporalia/ps149-initium-viii-G-auto.gtex}

\input{temporalia/ps149-viii-G-sinedox.tex}

\rubrica{Hic non dicitur Gloria Patri.}

\vfill
\pagebreak

%
\scriptura{Psalmus 150.}

\initiumpsalmi{temporalia/ps150-initium-viii-G-auto.gtex}

\input{temporalia/ps150-viii-G.tex}

\vfill

\vspace{-6mm}

\antiphona{}{temporalia/ant-videntibusillis.gtex} % repeat the antiphon - new page

\vfill
\pagebreak

\capitulumLaudes

\vfill

\pars{Responsorium breve.} \scriptura{Ps. 46, 6}

\cuminitiali{VI}{temporalia/resp-ascenditdeus.gtex}

\vfill
\pagebreak

\pars{Hymnus}

\cuminitiali{VIII}{temporalia/hym-JesuNostraRedemptio.gtex}
\vspace{-3mm}
%\begin{translatioMulticol}{3}
Výkupné naše, Ježíši,\\
lásko a tužbo nejčistší,\\
tys Tvůrce věcí stvořených\\
a člověk věků posledních.\\
\\
Jaký tě musil soucit vést,\\
žes naše hříchy za své vzal,\\
že chtěl jsi muky smrti nést,\\
bys kletbu smrti z lidí sňal.\columnbreak

Pronikáš v žalář pekelný,\\
propouštíš z něho zajatce.\\
Vítězi, slávou oděný,\\
po boku trůníš u Otce.\\
\\
Kéž donutí té soucit týž,\\
že rány vin v nás zacelíš,\\
nás podle slibu ušetříš\\
a vlídnou tváří potěšíš.\columnbreak

Ty budiž naší radostí,\\
odměnou ve tvé věčnosti,\\
kéž naše sláva veškerá\\
jen z tebe věčně vyvěrá.\\
Amen.
\end{translatioMulticol}


\vfill
%\pagebreak

\pars{Versus.}

% Versus. %%%
\ifx\festum\undefined
\sineinitiali{temporalia/versus-dominusincaelo-communis.gtex}
\else
\sineinitiali{temporalia/versus-dominusincaelo.gtex}
\fi

\vfill
\pagebreak

\ifx\dominicavelpostoctavam\undefined
\pars{Canticum Zachariæ.} \scriptura{Io. 20, 17; \textbf{H265}}

%\vspace{-6mm}

{
\grechangedim{interwordspacetext}{0.18 cm plus 0.15 cm minus 0.05 cm}{scalable}%
\antiphona{VII a}{temporalia/ant-ascendoadpatrem.gtex}
\grechangedim{interwordspacetext}{0.22 cm plus 0.15 cm minus 0.05 cm}{scalable}%
}

%\vspace{-3mm}

\scriptura{Lc. 1, 68-79}

%\vspace{-2.5mm}

\cantusSineNeumas
\initiumpsalmi{temporalia/benedictus-initium-viisoll-a-auto.gtex}

%\vspace{-1.5mm}

\input{temporalia/benedictus-viisoll-a.tex} \Abardot{}
\else
\pars{Canticum Zachariæ.} \scriptura{Io. 15, 26; \textbf{H267}}

\vspace{-3mm}

{
\grechangedim{interwordspacetext}{0.18 cm plus 0.15 cm minus 0.05 cm}{scalable}%
\antiphona{VIII G}{temporalia/ant-cumveneritparaclitus.gtex}
\grechangedim{interwordspacetext}{0.22 cm plus 0.15 cm minus 0.05 cm}{scalable}%
}

%\vspace{-3mm}

\scriptura{Lc. 1, 68-79}

%\vspace{-2.5mm}

\cantusSineNeumas
\initiumpsalmi{temporalia/benedictus-initium-viiisoll-G-auto.gtex}

%\vspace{-1.5mm}

\input{temporalia/benedictus-viiisoll-G.tex}

\vfill

\antiphona{}{temporalia/ant-cumveneritparaclitus.gtex}
\fi

\vspace{-1cm}

\vfill
\pagebreak

%\sideThumbs{{\scriptsize{}Fine horarum}}

\anteOrationem

\pagebreak

% Oratio. %%%
\ifx\dominicavelpostoctavam\undefined
\cuminitiali{}{temporalia/oratio.gtex}
\else
\cuminitiali{}{temporalia/oratio2.gtex}
\fi

\vspace{-1mm}

\vfill

\rubrica{Hebdomadarius dicit iterum Dominus vobiscum, vel cantor dicit:}

\vspace{2mm}

\sineinitiali{temporalia/domineexaudi.gtex}

\rubrica{Postea cantatur a cantore:}

\vspace{2mm}

\ifx\festum\undefined
\cuminitiali{VII}{temporalia/benedicamus-tempore-paschali.gtex}
\else
\cuminitiali{II}{temporalia/benedicamus-solemnism-laud.gtex}
\fi

\vspace{1mm}

\vfill
\pagebreak

\ifx\sabbato\undefined
\ifx\festumveldominica\undefined
\hora{Ad Vesperas.} %%%%%%%%%%%%%%%%%%%%%%%%%%%%%%%%%%%%%%%%%%%%%%%%%%%%%
%\sideThumbs{Vesperæ}
\else
\hora{Ad II. Vesperas.} %%%%%%%%%%%%%%%%%%%%%%%%%%%%%%%%%%%%%%%%%%%%%%%%%%%%%
%\sideThumbs{II. Vesperæ}
\fi

\cantusSineNeumas

\vspace{0.5cm}
\grechangedim{interwordspacetext}{0.18 cm plus 0.15 cm minus 0.05 cm}{scalable}%
\cuminitiali{}{temporalia/deusinadiutorium-solemnis.gtex}
\grechangedim{interwordspacetext}{0.22 cm plus 0.15 cm minus 0.05 cm}{scalable}%

\vfill
%\pagebreak

\vspace{-2mm}

\pars{Psalmus 1.} \scriptura{Ac. 1, 11; \textbf{H265}}

\vspace{-0.4cm}

\antiphona{VII a}{temporalia/ant-virigalilaeiquidaspicitis.gtex}

\scriptura{Psalmus 109.}

\initiumpsalmi{temporalia/ps109-initium-vii-a-auto.gtex}

\input{temporalia/ps109-vii-a.tex}

\vfill

\antiphona{}{temporalia/ant-virigalilaeiquidaspicitis.gtex}

\vspace{-1cm}

\vfill
\pagebreak

\vesperas

\ifx\dominicavelpostoctavam\undefined
\pars{Canticum B. Mariæ V.} \scriptura{Cf. Ps. 23, 7.10; Io. 14, 17.18; Lc. 24, 49; Eph. 4, 10; \textbf{H266}}

\vspace{-5.5mm}

{
\grechangedim{interwordspacetext}{0.18 cm plus 0.15 cm minus 0.05 cm}{scalable}%
\antiphona{II D}{temporalia/ant-orexgloriae.gtex}
\grechangedim{interwordspacetext}{0.22 cm plus 0.15 cm minus 0.05 cm}{scalable}%
}

\vspace{-3mm}

\scriptura{Lc. 1, 46-55}

\vspace{-2.5mm}

\cantusSineNeumas
\initiumpsalmi{temporalia/magnificat-initium-iisoll-D.gtex}

\vspace{-1.5mm}

\input{temporalia/magnificat-iisoll-D.tex} \Abardot{}
\else
\pars{Canticum B. Mariæ V.} \scriptura{Io. 16, 4}

{
\grechangedim{interwordspacetext}{0.18 cm plus 0.15 cm minus 0.05 cm}{scalable}%
\antiphona{VIII G}{temporalia/ant-haeclocutussumvobis.gtex}
\grechangedim{interwordspacetext}{0.22 cm plus 0.15 cm minus 0.05 cm}{scalable}%
}

\scriptura{Lc. 1, 46-55}

\cantusSineNeumas
\initiumpsalmi{temporalia/magnificat-initium-viiisoll-G.gtex}

\input{temporalia/magnificat-viiisoll-G.tex} \Abardot{}
\fi

\vspace{-1cm}

\vfill
\pagebreak

%\sideThumbs{{\scriptsize{}Fine horarum}}

\anteOrationem

\pagebreak

% Oratio. %%%
\ifx\dominicavelpostoctavam\undefined
\cuminitiali{}{temporalia/oratio.gtex}
\else
\cuminitiali{}{temporalia/oratio2.gtex}
\fi

\vspace{-1mm}

\vfill

\rubrica{Hebdomadarius dicit iterum Dominus vobiscum, vel cantor dicit:}

\vspace{2mm}

\sineinitiali{temporalia/domineexaudi.gtex}

\rubrica{Postea cantatur a cantore:}

\vspace{2mm}

\ifx\festum\undefined
\cuminitiali{VII}{temporalia/benedicamus-tempore-paschali.gtex}
\else
\cuminitiali{II}{temporalia/benedicamus-solemnism-2vesp.gtex}
\fi

\vspace{1mm}
\fi

\end{document}

