\newcommand{\titulus}{\nomenFesti{Dominica infra Octavam Ascensionis.}
\celebratio{Semiduplex.}}
\newcommand{\festumveldominica}{Dominica}
\newcommand{\dominicavelpostoctavam}{Dominica}
\newcommand{\dominica}{Dominica}
\newcommand{\lectioi}{\pars{Lectio I.} \scriptura{1 Io. 1, 1-5}

\noindent Incipit Epístola prima beáti Ioánnis Apóstoli.

\noindent Quod fuit ab inítio, quod audívimus, quod vídimus óculis nostris, quod perspéximus, et manus nostræ contrectavérunt de verbo vitæ: et vita manifestáta est, et vídimus, et testámur, et annuntiámus vobis vitam ætérnam, quæ erat apud Patrem, et appáruit nobis: quod vídimus et audívimus, annuntiámus vobis, ut et vos societátem habeátis nobíscum, et socíetas nostra sit cum Patre, et cum Fílio eius Iesu Christo. Et hæc scríbimus vobis ut gaudeátis, et gáudium vestrum sit plenum. Et hæc est annuntiátio, quam audívimus ab eo, et annuntiámus vobis: quóniam Deus lux est, et ténebræ in eo non sunt ullæ.}
\newcommand{\lectioii}{\pars{Lectio II.} \scriptura{1 Io. 1, 6-10}

\noindent Si dixérimus quóniam societátem habémus cum eo, et in ténebris ambulámus, mentímur, et veritátem non fácimus. Si autem in luce ambulámus sicut et ipse est in luce, societátem habémus ad ínvicem, et sanguis Iesu Christi, Fílii eius, emúndat nos ab omni peccáto. Si dixérimus quóniam peccátum non habémus, ipsi nos sedúcimus, et véritas in nobis non est. Si confiteámur peccáta nostra: fidélis est, et iustus, ut remíttat nobis peccáta nostra, et emúndet nos ab omni iniquitáte. Si dixérimus quóniam non peccávimus, mendácem fácimus eum, et verbum eius non est in nobis.}
\newcommand{\lectioiii}{\pars{Lectio III.} \scriptura{1 Io. 2, 1-6}

\noindent Filíoli mei, hæc scribo vobis, ut non peccétis. Sed et si quis peccáverit, advocátum habémus apud Patrem, Iesum Christum iustum: et ipse est propitiátio pro peccátis nostris: non pro nostris autem tantum, sed étiam pro totíus mundi. Et in hoc scimus quóniam cognóvimus eum, si mandáta eius observémus. Qui dicit se nosse eum, et mandáta eius non custódit, mendax est, et in hoc véritas non est. Qui autem servat verbum eius, vere in hoc cáritas Dei perfécta est: et in hoc scimus quóniam in ipso sumus. Qui dicit se in ipso manére, debet, sicut ille ambulávit, et ipse ambuláre.}
\newcommand{\lectioiv}{\pars{Lectio IV.} \scriptura{Sermo 2 de Ascensióne Dómini, qui est 175 de Tempore}

\noindent Sermo sancti Augustíni Epíscopi.

\noindent Salvátor noster, dilectíssimi fratres, ascéndit in cælum: non ergo turbémur in terra. Ibi sit mens, et hic erit réquies. Ascendámus cum Christo ínterim corde: cum dies eius promíssus advénerit, sequémur et córpore. Scire tamen debémus, fratres, quia cum Christo non ascéndit supérbia, non avarítia, non luxúria: nullum vítium nostrum ascéndit cum médico nostro. Et ídeo si post médicum desiderámus ascéndere, debémus vítia et peccáta depónere. Omnes enim quasi quibúsdam compédibus nos premunt, et peccatórum nos rétibus ligáre conténdunt: et ídeo cum Dei adiutório, secúndum quod ait Psalmísta: Dirumpámus víncula eórum: ut secúri possímus dícere Dómino: Dirupísti víncula mea, tibi sacrificábo hóstiam laudis.}
\newcommand{\lectiov}{\pars{Lectio V.}

\noindent Resurréctio Dómini spes nostra est: ascénsio Dómini glorificátio nostra est. Ascensiónis hódie solémnia celebrámus. Si ergo recte, si fidéliter, si devóte, si sancte, si pie ascensiónem Dómini celebrámus, ascendámus cum illo, et sursum corda habeámus. Ascendéntes autem non extollámur, nec de nostris, quasi de própriis méritis præsumámus. Sursum autem corda habére debémus ad Dóminum. Sursum enim cor non ad Dóminum, supérbia vocátur: sursum autem cor ad Dóminum, refúgium vocátur. Vidéte, fratres, magnum miráculum. Altus est Deus: érigis te, et fugit a te: humílias te, et descéndit ad te. Quare hoc? Quia excélsus est, et humília réspicit, et alta de longe cognóscit. Humília de próximo réspicit, ut attóllat: alta, id est, supérba, de longe cognóscit, ut déprimat.}
\newcommand{\lectiovi}{\pars{Lectio VI.}

\noindent Resurréxit enim Christus, ut spem nobis daret, quia surgit homo, qui móritur: ne moriéndo desperarémus, et vitam nostram in morte finítam putarémus, secúros nos fecit. Sollíciti enim erámus de ipsa ánima: et ille nobis resurgéndo, de carnis resurrectióne fidúciam dedit. Crede ergo, ut mundéris. Prius te opórtet crédere, ut póstea per fidem Deum mereáris aspícere. Deum enim vidére vis? Audi ipsum: Beáti mundo corde: quóniam ipsi Deum vidébunt. Prius ergo cógita de corde mundándo: quidquid ibi vides, quod dísplicet Deo, tolle.}
\newcommand{\lectiovii}{\pars{Lectio VII.} \scriptura{Io. 15, 26-27; 16, 1-4}

\noindent Léctio sancti Evangélii secúndum Ioánnem.

\noindent In illo témpore: Dixit Iesus discípulis suis: Cum vénerit Paráclitus, quem ego mittam vobis a Patre, Spíritum veritátis, qui a Patre procédit, ille testimónium perhibébit de me. Et réliqua.

\scriptura{Tractatus 92 in Ioannem}

\noindent Homilía sancti Augustíni Epíscopi.

\noindent Dóminus Iesus in sermóne, quem locútus est discípulis suis post cenam, próximus passióni, tamquam itúrus, et relictúrus eos præséntia corporáli, cum ómnibus autem suis usque in consummatiónem sǽculi futúrus præséntia spiritáli, exhortátus est eos ad perferéndas persecutiónes impiórum, quos mundi nómine nuncupávit. Ex quo tamen mundo étiam ipsos discípulos se elegísse dixit: ut scirent se Dei grátia esse, quod sunt; suis autem vítiis fuísse, quod fuérunt.}
\newcommand{\lectioviii}{\pars{Lectio VIII.}

\noindent Deínde persecutóres et suos et ipsórum, Iudǽos evidénter expréssit: ut omníno apparéret étiam ipsos mundi damnábilis appellatióne conclúsos, qui persequúntur sanctos. Cumque de illis díceret, quod ignorárent eum a quo missus est; et tamen odíssent et Fílium, et Patrem, hoc est, et eum qui missus est, et eum a quo missus est: (de quibus ómnibus in áliis semónibus iam disserúimus) ad hoc pervénit, ubi ait: Ut adimpleátur sermo, qui in lege eórum scriptus est: Quia ódio habuérunt me gratis.}
\newcommand{\lectioix}{\pars{Lectio IX.}

\noindent Deínde tamquam consequénter adiúnxit, unde modo disputáre suscépimus: Cum autem vénerit Paráclitus, quem ego mittam vobis a Patre, Spíritum veritátis, qui a Patre procédit, ille testimónium perhibébit de me: et vos testimónium perhibébitis, quia ab inítio mecum estis. Quid hoc pértinet ad illud quod díxerat: Nunc autem et vidérunt, et odérunt et me, et Patrem meum: sed ut impleátur sermo, qui in lege eórum scriptus est: Quia ódio habuérunt me gratis? An quia Paráclitus quando venit Spíritus veritátis, eos, qui vidérunt, et odérunt, testimónio manifestióre convícit? Immo vero étiam áliquos ex illis qui vidérunt, et adhuc óderant, ad fidem, quæ per dilectiónem operátur, sui manifestatióne convértit.}
% LuaLaTeX

\documentclass[a4paper, twoside, 12pt]{article}
\usepackage[latin]{babel}
%\usepackage[landscape, left=3cm, right=1.5cm, top=2cm, bottom=1cm]{geometry} % okraje stranky
%\usepackage[landscape, a4paper, mag=1166, truedimen, left=2cm, right=1.5cm, top=1.6cm, bottom=0.95cm]{geometry} % okraje stranky
\usepackage[landscape, a4paper, mag=1400, truedimen, left=0.5cm, right=0.5cm, top=0.5cm, bottom=0.5cm]{geometry} % okraje stranky

\usepackage{fontspec}
\setmainfont[FeatureFile={junicode.fea}, Ligatures={Common, TeX}, RawFeature=+fixi]{Junicode}
%\setmainfont{Junicode}

% shortcut for Junicode without ligatures (for the Czech texts)
\newfontfamily\nlfont[FeatureFile={junicode.fea}, Ligatures={Common, TeX}, RawFeature=+fixi]{Junicode}

\usepackage{multicol}
\usepackage{color}
\usepackage{lettrine}
\usepackage{fancyhdr}

% usual packages loading:
\usepackage{luatextra}
\usepackage{graphicx} % support the \includegraphics command and options
\usepackage{gregoriotex} % for gregorio score inclusion
\usepackage{gregoriosyms}
\usepackage{wrapfig} % figures wrapped by the text
\usepackage{parcolumns}
\usepackage[contents={},opacity=1,scale=1,color=black]{background}
\usepackage{tikzpagenodes}
\usepackage{calc}
\usepackage{longtable}
\usetikzlibrary{calc}

\setlength{\headheight}{14.5pt}

\input{conventuscommune.tex} % Often used macros
%%%% Preklady jednotlivych zpevu (nektere se opakuji, a je dobre mit je
% vsechny na jedne hromade)

% HOURS ---

\newcommand{\trAntI}{\translatioCantus{Muž boží měl kožený toulec, pečlivě
zavázaný, jenž mu visel na šíji a~často se ho dotýkal.}}

\newcommand{\trAntII}{\translatioCantus{Klíč od~něho tak dobře střežil, že
dokud žil v~těle, nikdo z~jeho žáků nezvěděl, co je uvnitř.}}

\newcommand{\trAntIII}{\translatioCantus{Ale když se odebral z~tohoto
života, schránku otevřeli a~objevili v~ní žíněné roucho a~měděný řetěz
potřísněný krví.}}

\newcommand{\trAntIV}{\translatioCantus{A když prohlédli mistrovo tělo,
nalezli jeho tělo na čtyřech místech hluboce zbrázděno ranami od řetězu.}}

\newcommand{\trAntV}{\translatioCantus{Krev vytékající z~těch ran, místy
prostoupila i~žíněným rouchem.}}

\newcommand{\trCapituli}{\translatioCantus{
Miláčkovi Boha a~lidí,
Mojžíšovi požehnané paměti,~\gredagger{}
dopřál slávu rovnou slávě svatých~\grestar{}
učinil ho mocným na postrach nepřátelům
a~jeho slovy zastavil divy.}}

\newcommand{\trLectioBrevis}{\translatioCantus{
Pamatujte na své představené,
kteří vám hlásali Boží slovo.
Uvažte, jak oni skončili život, a~napodobujte jejich víru.
Ježíš Kristus je stejný včera i~dnes i~navěky.
Nenechte se svést věelijakými cizími naukami.}}

\newcommand{\trRespLaud}{\translatioCantus{Spravedlivého vodil Hospodin~\grestar{}
po přímých stezkách. \Vbardot{} A~ukázal mu Boží království.}}

\newcommand{\trRespLaudB}{\translatioCantus{Na tvých hradbách, Jeruzaléme,
ustanovil jsem strážné;~\grestar{}
budou bdít nad mým lidem. \Vbardot{} Ani ve dne, ani v~noci nesmějí nikdy
mlčet.}}

\newcommand{\trVersus}{\translatioCantus{\Vbardot{} Ústa spravedlivého šeptají moudrost, aleluja.
\Rbardot{} A~jeho jazyk ohlašuje právo, aleluja.}}

\newcommand{\trAntBenedictus}{\translatioCantus{Když na bujné oře vložili
nosítka a~sňali jim uzdu, vydali se přímo k~cele božího muže.}}

\newcommand{\trPreces}{\translatioCantus{
\noindent S vděčností chvalme Krista, dobrého Pastýře, \gredagger{} který dal život za své ovce, \grestar{} a~pokorně ho prosme: \Rbardot{} Pane, buď pastýřem svého lidu.

\noindent Kriste, ty dáváš církvi pastýře, a~jejich službou se ujímáš svého lidu, \grestar{} dej, ať v~lásce těch, kteří nás vedou, poznáváme, jak nás miluješ. \Rbardot{} Pane, buď pastýřem svého lidu.

\noindent Ty stále konáš skrze své zástupce službu pastýře a~učitele, \grestar{} nepřestávej nás nikdy vést prostřednictvím svých služebníků. \Rbardot{} Pane, buď pastýřem svého lidu.

\noindent Ty prokazuješ svému lidu skrze jeho pastýře službu lékaře duše i~těla, \grestar{} ochraňuj náš život a~veď nás ke svatosti. \Rbardot{} Pane, buď pastýřem svého lidu.

\noindent Ty posíláš své svaté, aby slovem i~příkladem vedli tvůj lid k~tobě, \grestar{} na jejich přímluvu nás posiluj, abychom vytrvali na cestě, která vede k~věčnému životu. \Rbardot{} Pane, buď pastýřem svého lidu.}}

\newcommand{\trOrationis}{\translatioCantus{Bože, jenž nám dopřáváš radovat
se z~výroční slavnosti svatého tvého vyznavače Havla, uděl dobrotivě,
abychom když slavíme jeho narození, též se řídili podobou jeho skutků.
Skrze…}}
 % Czech translations of the proper texts

\newcommand{\annusEditionis}{2020}

%%%% Vicekrat opakovane kousky

\newcommand{\anteOrationem}{
  \rubrica{Ante Orationem, cantatur a Superiore:}

  \pars{Supplicatio Litaniæ.}

  \cuminitiali{}{temporalia/supplicatiolitaniae.gtex}

  \pars{Oratio Dominica.}

  \cuminitiali{}{temporalia/oratiodominica.gtex}

  \rubrica{Deinde dicitur ab Hebdomadario:}

  \cuminitiali{}{temporalia/dominusvobiscum-solemnis.gtex}

  \rubrica{In choro monialium loco Dominus vobiscum dicitur:}

  \sineinitiali{temporalia/domineexaudi.gtex}
}

\ifx\dominicavelpostoctavam\undefined
\newcommand{\capitulumLaudes}{\pars{Capitulum.} \scriptura{Ac. 1, 1-2}

\grechangedim{interwordspacetext}{0.12 cm plus 0.15 cm minus 0.05 cm}{scalable}%
\cuminitiali{}{temporalia/capitulum-PrimumQuidem.gtex}
\grechangedim{interwordspacetext}{0.22 cm plus 0.15 cm minus 0.05 cm}{scalable}}
\else
\newcommand{\capitulumLaudes}{\pars{Capitulum.} \scriptura{1 Ptr. 4, 7-8}

\grechangedim{interwordspacetext}{0.12 cm plus 0.15 cm minus 0.05 cm}{scalable}%
\cuminitiali{}{temporalia/capitulum-CarissimiEstote.gtex}
\grechangedim{interwordspacetext}{0.22 cm plus 0.15 cm minus 0.05 cm}{scalable}}
\fi

\setlength{\columnsep}{30pt} % prostor mezi sloupci

%%%%%%%%%%%%%%%%%%%%%%%%%%%%%%%%%%%%%%%%%%%%%%%%%%%%%%%%%%%%%%%%%%%%%%%%%%%%%%%%%%%%%%%%%%%%%%%%%%%%%%%%%%%%%
\begin{document}

% Here we set the space around the initial.
% Please report to http://home.gna.org/gregorio/gregoriotex/details for more details and options
\grechangedim{afterinitialshift}{2.2mm}{scalable}
\grechangedim{beforeinitialshift}{2.2mm}{scalable}
\grechangedim{interwordspacetext}{0.22 cm plus 0.15 cm minus 0.05 cm}{scalable}%
\grechangedim{annotationraise}{-0.2cm}{scalable}

% Here we set the initial font. Change 38 if you want a bigger initial.
% Emit the initials in red.
\grechangestyle{initial}{\color{red}\fontsize{38}{38}\selectfont}

\pagestyle{empty}

\newcommand{\vesperas}{
\pars{Psalmus 2.} \scriptura{Ac. 1, 10; \textbf{H265}}

\vspace{-0.4cm}

\antiphona{VIII G\textsuperscript{2}}{temporalia/ant-cumqueintuerentur.gtex}

\scriptura{Psalmus 110.}

\initiumpsalmi{temporalia/ps110-initium-viii-G2-auto.gtex}

\input{temporalia/ps110-viii-G2.tex} \Abardot{}

\vfill
\pagebreak

\pars{Psalmus 3.} \scriptura{Lc. 24, 50.51; \textbf{H265}}

\vspace{-0.4cm}

\antiphona{IV A*}{temporalia/ant-elevatismanibus.gtex}

\scriptura{Psalmus 111.}

\initiumpsalmi{temporalia/ps111-initium-iv-A_-auto.gtex}

\input{temporalia/ps111-iv-A_.tex} \Abardot{}

\vfill
\pagebreak

\pars{Psalmus 4.} \scriptura{Ac. 1, 9; \textbf{H265}}

\vspace{-0.4cm}

\antiphona{VIII G}{temporalia/ant-videntibusillis.gtex}

\scriptura{Psalmus 112.}

\initiumpsalmi{temporalia/ps112-initium-viii-G-auto.gtex}

\input{temporalia/ps112-viii-G.tex} \Abardot{}

%\vfill

%\vspace{-6mm}

%\antiphona{}{temporalia/ant-videntibusillis.gtex} % repeat the antiphon - new page

\vfill
\pagebreak

\capitulumLaudes

\vfill

\pars{Responsorium breve.} \scriptura{Cf. Ps. 67, 19}

\ifx\festum\undefined
\cuminitiali{VI}{temporalia/resp-ascendenschristusinaltum-simplex.gtex}
\else
\cuminitiali{VI}{temporalia/resp-ascendenschristusinaltum.gtex}
\fi

\vfill
\pagebreak

\pars{Hymnus}

\cuminitiali{IV}{temporalia/hym-JesuNostraRedemptio.gtex}
\vspace{-3mm}
%\begin{translatioMulticol}{3}
Výkupné naše, Ježíši,\\
lásko a tužbo nejčistší,\\
tys Tvůrce věcí stvořených\\
a člověk věků posledních.\\
\\
Jaký tě musil soucit vést,\\
žes naše hříchy za své vzal,\\
že chtěl jsi muky smrti nést,\\
bys kletbu smrti z lidí sňal.\columnbreak

Pronikáš v žalář pekelný,\\
propouštíš z něho zajatce.\\
Vítězi, slávou oděný,\\
po boku trůníš u Otce.\\
\\
Kéž donutí té soucit týž,\\
že rány vin v nás zacelíš,\\
nás podle slibu ušetříš\\
a vlídnou tváří potěšíš.\columnbreak

Ty budiž naší radostí,\\
odměnou ve tvé věčnosti,\\
kéž naše sláva veškerá\\
jen z tebe věčně vyvěrá.\\
Amen.
\end{translatioMulticol}


\vfill
%\pagebreak

\pars{Versus.} \scriptura{Ps. 46, 6}

% Versus. %%%
\ifx\festum\undefined
\sineinitiali{temporalia/versus-ascenditdeus-communis.gtex}
\else
\sineinitiali{temporalia/versus-ascenditdeus.gtex}
\fi

\vfill
\pagebreak
}

%%%% Titulni stranka
\begin{titulusOfficii}
\titulus
\end{titulusOfficii}

% graphic
%\vspace{1.5cm}
%\begin{center}
%\includegraphics[width=8cm]{emmaus.jpg}
%\end{center}

\vfill

\begin{center}
%Ad usum et secundum consuetudines chori \guillemotright{}Conventus Choralis\guillemotleft.

%Editio Sancti Wolfgangi \annusEditionis
\end{center}

\pagebreak

\renewcommand{\headrulewidth}{0pt} % no horiz. rule at the header
\fancyhf{}
\pagestyle{fancy}

\cantusSineNeumas

\ifx\festumveldominica\undefined
\else
\pars{Oratio ante divinum Officium.}

\lettrine{{\color{red}A}}{peri,} Dómine, os meum ad benedicéndum nomen sanctum tuum:
munda quoque cor meum ab ómnibus vanis, pervérsis, et aliénis
cogitatiónibus:
intelléctum illúmina, afféctum inflámma,
ut digne, atténte ac devóte hoc Offícium recitáre váleam,
et exaudíri mérear ante conspéctum Divínæ Maiestátis tuæ.
Per Christum, Dóminum nostrum.
\Rbardot{} Amen.

Dómine, in unióne illíus divínæ intentiónis,
qua ipse in terris laudes Deo persolvísti,
has tibi Horas \rubricatum{(vel \textnormal{hanc tibi Horam})} persólvo.

\vfill

\pars{Oratio post divinum Officium.}

\rubrica{
  Orationem sequentem devote post Officium recitantibus
  Leo Papa X. defectus, et culpas in eo persolvendo ex humana
  fragilitate contractas, indulsit, et dicitur flexis genibus.
}

\lettrine{{\color{red}S}}{acrosánctæ} et indivíduæ Trinitáti,
crucifíxi Dómini nostri Iesu Christi humanitáti,
beatíssimæ et gloriosíssimæ sempérque Vírginis Maríæ
fecúndæ integritáti, 
et ómnium Sanctórum universitáti
sit sempitérna laus, honor, virtus et glória
ab omni creatúra,
nobísque remíssio ómnium peccatórum,
per infiníta sǽcula sæculórum.
\Rbardot{} Amen.

\noindent \Vbardot{} Beáta víscera Maríæ Virginis, quæ portavérunt
ætérni Patris Fílium.\\
\Rbardot{} Et beáta úbera, quæ lactavérunt Christum Dominum.

\rubrica{Et dicitur secreto \textnormal{Pater noster.} et \textnormal{Ave María.}}

\vfill

\hora{Ad I. Vesperas.} %%%%%%%%%%%%%%%%%%%%%%%%%%%%%%%%%%%%%%%%%%%%%%%%%%%%%
%\sideThumbs{I. Vesperæ}

\vspace{0.5cm}
\grechangedim{interwordspacetext}{0.18 cm plus 0.15 cm minus 0.05 cm}{scalable}%
\cuminitiali{}{temporalia/deusinadiutorium-solemnis.gtex}
\grechangedim{interwordspacetext}{0.22 cm plus 0.15 cm minus 0.05 cm}{scalable}%

\vfill
\pagebreak

\pars{Psalmus 1.} \scriptura{Ac. 1, 11; \textbf{H265}}

\vspace{-0.4cm}

\antiphona{VII a}{temporalia/ant-virigalilaeiquidaspicitis.gtex}

\scriptura{Psalmus 109.}

\initiumpsalmi{temporalia/ps109-initium-vii-a-auto.gtex}

\input{temporalia/ps109-vii-a.tex} \Abardot{}

\vspace{-1cm}

\vfill
\pagebreak

\vesperas
\ifx\dominicavelpostoctavam\undefined
\pars{Canticum B. Mariæ V.} \scriptura{Io. 17, 6.9}

\vspace{-3mm}

{
\grechangedim{interwordspacetext}{0.18 cm plus 0.15 cm minus 0.05 cm}{scalable}%
\antiphona{VI F}{temporalia/ant-patermanifestavi.gtex}
\grechangedim{interwordspacetext}{0.22 cm plus 0.15 cm minus 0.05 cm}{scalable}%
}

\vspace{-2mm}

\scriptura{Lc. 1, 46-55}

\vspace{-2mm}

\cantusSineNeumas
\initiumpsalmi{temporalia/magnificat-initium-visoll-F.gtex}

\input{temporalia/magnificat-visoll-F.tex} \Abardot{}
\else
\pars{Canticum B. Mariæ V.} \scriptura{Io. 15, 26; \textbf{H267}}

\vspace{-6mm}

{
\grechangedim{interwordspacetext}{0.18 cm plus 0.15 cm minus 0.05 cm}{scalable}%
\antiphona{VIII G}{temporalia/ant-cumveneritparaclitus.gtex}
\grechangedim{interwordspacetext}{0.22 cm plus 0.15 cm minus 0.05 cm}{scalable}%
}

\vspace{-3mm}

\scriptura{Lc. 1, 46-55}

\vspace{-2mm}

\cantusSineNeumas
\initiumpsalmi{temporalia/magnificat-initium-viiisoll-G.gtex}

\vspace{-1.5mm}

\input{temporalia/magnificat-viiisoll-G.tex} \Abardot{}
\fi

\vspace{-1cm}

\vfill
\pagebreak

%\sideThumbs{{\scriptsize{}Fine horarum}}

\anteOrationem

\pagebreak

% Oratio. %%%
\cuminitiali{}{temporalia/oratio.gtex}

\vspace{-1mm}

\vfill

\rubrica{Hebdomadarius dicit iterum Dominus vobiscum, vel cantor dicit:}

\vspace{2mm}

\sineinitiali{temporalia/domineexaudi.gtex}

\rubrica{Postea cantatur a cantore:}

\vspace{2mm}

\ifx\festum\undefined
\cuminitiali{VII}{temporalia/benedicamus-tempore-paschali.gtex}
\else
\cuminitiali{II}{temporalia/benedicamus-solemnism-1vesp.gtex}
\fi

\vspace{1mm}

\vfill
\pagebreak
\fi

\ifx\festum\undefined
\else
\hora{Ad Completorium.} %%%%%%%%%%%%%%%%%%%%%%%%%%%%%%%%%%%%%%%%%%%%%%%%%%%%%%%%%%
%\sideThumbs{{\scriptsize{}Completorium}}

\rubrica{Lector petit benedictionem, dicens:}

\cuminitiali{}{temporalia/jubedomnebenedicere.gtex}

\vfill

\pars{Benedictio.}

\cuminitiali{}{temporalia/benedictio-noctemquietam.gtex}

\vfill

\pars{Lectio brevis.} \scriptura{1Ptr. 5, 8-9}

\cuminitiali{}{temporalia/lectiobrevis-fratressobrii.gtex}

\vfill

\noindent \Vbardot{} Adiutórium nostrum in nómine Dómini.

\noindent \Rbardot{} Qui fecit cælum, et terram.

\vfill
\pagebreak

\pars{Confessio.}

\noindent Confíteor Deo omnipoténti, beátæ Maríæ semper Vírgini, beáto
Michaéli Archángelo, beáto Ioánni Baptístæ, sanctis Apóstolis Petro
et Paulo, ómnibus Sanctis, et vobis fratres: quia peccávi nimis cogitatióne,
verbo et ópere: mea culpa, mea culpa, mea máxima culpa.
Ideo precor beátam Maríam semper Vírginem, beátum Michaélem
Archángelum, beátum Ioánnem Baptístam, sanctos Apóstolos Petrum
et Paulum, omnes Sanctos, et vos fratres, oráre pro me ad Dóminum
Deum nostrum.

\vfill

\noindent \Vbardot{} Misereátur nostri omnípotens Deus, et, dimíssis peccátis nostris, perdúcat
nos ad vitam ætérnam. \Rbardot{} Amen.

\vfill

\noindent \Vbardot{} Indulgéntiam, absolutiónem et remissiónem peccatórum nostrórum tríbuat nobis
omnípotens et miséricors Dóminus. \Rbardot{} Amen.

\vfill

\rubrica{Et facta absolutione dicitur:}

\sineinitiali{temporalia/convertenosdeus.gtex}

\vfill

\cuminitiali{}{temporalia/deusinadiutorium-communis.gtex}

\vfill
\pagebreak

\pars{Psalmus 1.}

\antiphona{VIII G}{temporalia/ant-alleluia-compl.gtex}

\scriptura{Ps. 4}

\initiumpsalmi{temporalia/ps4-initium-viii-G-auto.gtex}

\input{temporalia/ps4-viii-G.tex}

\vfill
\pagebreak

\pars{Psalmus 2.} \scriptura{Ps. 90}

\initiumpsalmi{temporalia/ps90-initium-viii-G-auto.gtex}

\input{temporalia/ps90-viii-G.tex}

\pagebreak

\pars{Psalmus 3.} \scriptura{Ps. 133}

\initiumpsalmi{temporalia/ps133-initium-viii-G-auto.gtex}

\input{temporalia/ps133-viii-G.tex}

\vfill

\antiphona{VIII G}{temporalia/ant-alleluia-compl.gtex}

\vfill

\pars{Hymnus.}

\antiphona{I}{temporalia/hym-TeLucis.gtex}
%\input{hym-TeLucis-bohtext.tex}

\pagebreak

\pars{Capitulum.} \scriptura{Ier. 14, 9}

\cuminitiali{}{temporalia/capitulum-tuautem.gtex}

\vfill

\pars{Responsorium breve.} \scriptura{Ps. 30, 6}

\cuminitiali{VI}{temporalia/resp-inmanus-tp.gtex}

\vfill

\pars{Versus.} \scriptura{Ps. 16, 8}

\sineinitiali{temporalia/versus-custodi.gtex}

\vfill
\pagebreak

\cantusCumNeumis

\pars{Canticum Simeonis.}

\vspace{-3mm}

\antiphona{III a}{temporalia/ant-salvanos-antiquo-tp.gtex}

\scriptura{Lc. 2, 29-32}

\vspace{-2mm}

\initiumpsalmi{temporalia/nuncdimittis-initium-iii-a-auto.gtex}

\input{temporalia/nuncdimittis-iii-a.tex} \Abardot{}

\vfill

\rubrica{Ante Orationem, cantatur a Superiore:}

\pars{Supplicatio Litaniæ.}

\cuminitiali{}{temporalia/supplicatiolitaniae.gtex}

\vspace{7mm}

\pars{Oratio Dominica.}

\noindent Pater noster.

\vfill
\pagebreak

\sineinitiali{temporalia/domineexaudi-simplex.gtex}

\vspace{7mm}

\pars{Oratio.}

\cantusSineNeumas

\cuminitiali{}{temporalia/oratio-visita.gtex}

\vfill

%\sineinitiali{temporalia/domineexaudi-communis.gtex}

\noindent \Vbardot{} Dómine, exáudi oratiónem meam. \Rbardot{} Et clamor meus ad te véniat.

\vfill

%\vfill

\sineinitiali{temporalia/benedicamus-minor.gtex}

\vfill

\pars{Benedictio.}

\noindent Benedícat et custódiat nos omnípotens et miséricors Dóminus,~\gredagger{}
Pater, et Fílius, et Spíritus Sanctus. \Rbardot{} Amen.

\vfill
\pagebreak

\pars{Antiphona finalis B. M. V.}

\antiphona{V}{temporalia/an_regina_caeli_simplex.gtex}

\vspace{7mm}

\sineinitiali{temporalia/versus-gaude.gtex}

\vfill
\pagebreak
\fi

\hora{Ad Matutinum.} %%%%%%%%%%%%%%%%%%%%%%%%%%%%%%%%%%%%%%%%%%%%%%%%%%%%%
%\sideThumbs{Matutinum}

\vspace{2mm}

\cuminitiali{}{temporalia/dominelabiamea.gtex}

\vspace{2mm}

\pars{Invitatorium.} \scriptura{Cantor; Psalmus 94; \textbf{H261}}

\vspace{-6mm}

\antiphona{V}{temporalia/inv-alleluiachristumdominum.gtex}

\vfill
\pagebreak

\pars{Hymnus.}

\vspace{-5mm}

\scriptura{Anonymus X. sæculi; \textbf{AR488}}

\antiphona{IV}{temporalia/hym-AEterneRexAltissime.gtex}
%{
%\vspace{-5mm}
%\setlength{\columnsep}{0pt} % prostor mezi sloupci
%\input{hym-RexSempiterne-bohtext.tex}
%\setlength{\columnsep}{30pt} % prostor mezi sloupci
%}

\vfill
\pagebreak

\subhora{In I. Nocturno}

\pars{Psalmus 1.} \scriptura{Ps. 8, 2; \textbf{H262}}

%\vspace{-5mm}

\antiphona{IV A*}{temporalia/ant-elevataestmagnificentiatua.gtex}

%\vspace{-5mm}

\scriptura{Ps. 8}

%\vspace{-2mm}

\initiumpsalmi{temporalia/ps8-initium-iv-A_-auto.gtex}

\input{temporalia/ps8-iv-A_.tex} \Abardot{}

\vfill
\pagebreak

\pars{Psalmus 2.} \scriptura{Ps. 10, 5; \textbf{H262}}

%\vspace{-5mm}

\antiphona{VIII c}{temporalia/ant-dominusintemplosanctosuo.gtex}

%\vspace{-5mm}

\scriptura{Ps. 10}

\initiumpsalmi{temporalia/ps10-initium-viii-C-auto.gtex}

\input{temporalia/ps10-viii-C.tex} \Abardot{}

\vfill
\pagebreak

\pars{Psalmus 3.} \scriptura{Ps. 18, 7; \textbf{H262}}

%\vspace{-5mm}

\antiphona{IV A*}{temporalia/ant-asummocoeloegressioejus.gtex}

%\vspace{-5mm}

\scriptura{Ps. 18}

\initiumpsalmi{temporalia/ps18-initium-iv-A_-auto.gtex}

\input{temporalia/ps18-iv-A_.tex}

\vfill

\antiphona{}{temporalia/ant-asummocoeloegressioejus.gtex}

\vfill
\pagebreak

\noindent \Vbardot{} Ascéndit Deus in iubilatióne, allelúia.
\noindent \Rbardot{} Et Dóminus in voce tubæ, allelúia.

\vspace{5mm}

\sineinitiali{temporalia/oratiodominica-mat.gtex}

\vspace{5mm}

\pars{Absolutio.}

\cuminitiali{}{temporalia/absolutio-exaudi.gtex}

\vfill
\pagebreak

\cuminitiali{}{temporalia/benedictio-solemn-benedictione.gtex}

\vspace{7mm}

\lectioi

\noindent \Vbardot{} Tu autem, Dómine, miserére nobis.
\noindent \Rbardot{} Deo grátias.

\vfill
\pagebreak

\pars{Responsorium 1.} \scriptura{\Rbardot{} Ac. 1, 3.9; \Vbardot{} ibid. 1, 4; \textbf{H262}}

\vspace{-5mm}

\responsorium{III transp.}{temporalia/resp-postpassionemsuam-sinedox.gtex}{}

\vfill
\pagebreak

\cuminitiali{}{temporalia/benedictio-solemn-unigenitus.gtex}

\vspace{7mm}

\lectioii

\noindent \Vbardot{} Tu autem, Dómine, miserére nobis.
\noindent \Rbardot{} Deo grátias.

\vfill
\pagebreak

\pars{Responsorium 2.} \scriptura{\Rbardot{} Cantor; \Vbardot{} Ps. 18, 7; \textbf{H262}}

\vspace{-5mm}

\responsorium{II}{temporalia/resp-omnispulchritudodomini-sinedox.gtex}{}

\vfill
\pagebreak

\cuminitiali{}{temporalia/benedictio-solemn-spiritus.gtex}

\vspace{7mm}

\lectioiii

\noindent \Vbardot{} Tu autem, Dómine, miserére nobis.
\noindent \Rbardot{} Deo grátias.

\vfill
\pagebreak

\pars{Responsorium 3.} \scriptura{\Rbardot{} Ps. 20, 14; \Vbardot{} Ps. 8, 2; \textbf{H262}}

\vspace{-5mm}

\responsorium{VII}{temporalia/resp-exaltaredomine-cumdox.gtex}{}

\vfill
\pagebreak

\subhora{In II. Nocturno}

\pars{Psalmus 4.} \scriptura{Ps. 20, 14; \textbf{H262}}

\vspace{-5mm}

\antiphona{IV A*}{temporalia/ant-exaltaredomine.gtex}

\vspace{-2mm}

\scriptura{Ps. 20}

\vspace{-1mm}

\initiumpsalmi{temporalia/ps20-initium-iv-A_-auto.gtex}

\input{temporalia/ps20-iv-A_.tex} \Abardot{}

\vfill
\pagebreak

\pars{Psalmus 5.} \scriptura{Ps. 29, 2; \textbf{H262}}

\vspace{-5.5mm}

\antiphona{VIII G}{temporalia/ant-exaltabotedomine.gtex}

\vspace{-3mm}

\scriptura{Ps. 29}

\vspace{-2mm}

\initiumpsalmi{temporalia/ps29-initium-viii-G-auto.gtex}

\vspace{-1.5mm}

\input{temporalia/ps29-viii-G.tex} \Abardot{}

\vspace{-1cm}

\vfill
\pagebreak

\pars{Psalmus 6.} \scriptura{Ps. 46, 6; \textbf{H262}}

%\vspace{-5mm}

\antiphona{IV A*}{temporalia/ant-ascenditdeus.gtex}

%\vspace{-5mm}

\scriptura{Ps. 46}

\initiumpsalmi{temporalia/ps46-initium-iv-A_-auto.gtex}

\input{temporalia/ps46-iv-A_.tex} \Abardot{}

\vfill
\pagebreak

\noindent \Vbardot{} Ascéndens Christus in altum, allelúia.
\noindent \Rbardot{} Captívam duxit captivitátem, allelúia.

\vspace{5mm}

\sineinitiali{temporalia/oratiodominica-mat.gtex}

\vspace{5mm}

\pars{Absolutio.}

\cuminitiali{}{temporalia/absolutio-ipsius.gtex}

\vfill
\pagebreak

\cuminitiali{}{temporalia/benedictio-solemn-deus.gtex}

\vspace{7mm}

\lectioiv

\noindent \Vbardot{} Tu autem, Dómine, miserére nobis.
\noindent \Rbardot{} Deo grátias.

\vfill
\pagebreak

\pars{Responsorium 4.} \scriptura{\Rbardot{} Tob. 12, 20 \& Io. 14, 27; \Vbardot{} Io. 16, 7; \textbf{H263}}

\vspace{-5mm}

\responsorium{IV}{temporalia/resp-tempusest-sinedox.gtex}{}

\vfill
\pagebreak

\cuminitiali{}{temporalia/benedictio-solemn-christus.gtex}

\vspace{7mm}

\lectiov

\noindent \Vbardot{} Tu autem, Dómine, miserére nobis.
\noindent \Rbardot{} Deo grátias.

\vfill
\pagebreak

\pars{Responsorium 5.} \scriptura{\Rbardot{} Cantor super Ioannem; \Vbardot{} Io. 14, 16; \textbf{H263}}

\vspace{-5mm}

\responsorium{III}{temporalia/resp-nonconturbetur-sinedox.gtex}{}

\vfill
\pagebreak

\cuminitiali{}{temporalia/benedictio-solemn-ignem.gtex}

\vspace{7mm}

\lectiovi

\noindent \Vbardot{} Tu autem, Dómine, miserére nobis.
\noindent \Rbardot{} Deo grátias.

\vfill
\pagebreak

\pars{Responsorium 6.} \scriptura{\Rbardot{} Eph. 4, 8; \Vbardot{} Ps. 46, 6; \textbf{H263}}

\vspace{-5mm}

\responsorium{IV}{temporalia/resp-ascendensinaltum-cumdox.gtex}{}

\vfill
\pagebreak

\subhora{In III. Nocturno}

\pars{Psalmus 7.} \scriptura{Ps. 96, 9; \textbf{H263}}

\vspace{-5mm}

\antiphona{VI F}{temporalia/ant-nimisexaltatusest.gtex}

\vspace{-4mm}

\scriptura{Ps. 96}

%\vspace{-2mm}

\initiumpsalmi{temporalia/ps96-initium-vi-F-auto.gtex}

\input{temporalia/ps96-vi-F.tex} \Abardot{}

\vfill
\pagebreak

\pars{Psalmus 8.} \scriptura{Ps. 98, 2; \textbf{H263}}

\vspace{-5mm}

\antiphona{VI F}{temporalia/ant-dominusinsion.gtex}

\vspace{-4mm}

\scriptura{Ps. 98}

\initiumpsalmi{temporalia/ps98-initium-vi-F-auto.gtex}

\input{temporalia/ps98-vi-F.tex} \Abardot{}

\vfill
\pagebreak

\pars{Psalmus 9.} \scriptura{Ps. 102, 19; \textbf{H263}}

\vspace{-5mm}

\antiphona{VI F}{temporalia/ant-dominusincoelo.gtex}

\vspace{-4mm}

\scriptura{Ps. 102}

\initiumpsalmi{temporalia/ps102-initium-vi-F-auto.gtex}

\input{temporalia/ps102-vi-F.tex}

\vfill

\antiphona{}{temporalia/ant-dominusincoelo.gtex}

\vfill
\pagebreak

\noindent \Vbardot{} Ascéndo ad Patrem meum, et Patrem vestrum, allelúia.
\noindent \Rbardot{} Deum meum, et Deum vestrum, allelúia.

\vspace{5mm}

\sineinitiali{temporalia/oratiodominica-mat.gtex}

\vspace{5mm}

\pars{Absolutio.}

\cuminitiali{}{temporalia/absolutio-avinculis.gtex}

\vfill
\pagebreak

\cuminitiali{}{temporalia/benedictio-solemn-evangelica.gtex}

\vspace{7mm}

\lectiovii

\noindent \Vbardot{} Tu autem, Dómine, miserére nobis.
\noindent \Rbardot{} Deo grátias.

\vfill
\pagebreak

\pars{Responsorium 7.} \scriptura{\Rbardot{} Io. 14, 16.17; \Vbardot{} ibid. 16, 7; \textbf{Sar.275}}

\vspace{-5mm}

\responsorium{III}{temporalia/resp-egorogabopatrem-sinedox.gtex}{}

\vfill
\pagebreak

\cuminitiali{}{temporalia/benedictio-solemn-divinum.gtex}

\vspace{7mm}

\lectioviii

\noindent \Vbardot{} Tu autem, Dómine, miserére nobis.
\noindent \Rbardot{} Deo grátias.

\vfill
\pagebreak

\ifx\dominicavelpostoctavam\undefined
\pars{Responsorium 8.} \scriptura{\Rbardot{} Ps. 103, 3; \Vbardot{} Ps. 103, 1.2; \textbf{H264}}

\vspace{-5mm}

\responsorium{II}{temporalia/resp-ponitnubem-cumdox.gtex}{}
\else
\pars{Responsorium 8.} \scriptura{\Rbardot{} Io. 16, 7; \Vbardot{} ibid. 16, 13; \textbf{Sar.272}}

\vspace{-5mm}

\responsorium{III}{temporalia/resp-sienimnonabiero-cumdox.gtex}{}
\fi

\vfill
\pagebreak

\cuminitiali{}{temporalia/benedictio-solemn-adsocietatem.gtex}

\vspace{7mm}

\lectioix

\noindent \Vbardot{} Tu autem, Dómine, miserére nobis.
\noindent \Rbardot{} Deo grátias.

\vfill
\pagebreak

% Te Deum

%\pars{Hymnus Ambrosianus}

\vspace{-5mm}

\cuminitiali{III}{temporalia/tedeum-solemnis.gtex}

\vfill
\pagebreak

\rubrica{Reliqua omittuntur, nisi Laudes separandæ sint.}

\pars{Oratio}

\noindent \Vbardot{} Dómine, exáudi oratiónem meam.

\noindent \Rbardot{} Et clamor meus ad te véniat.

Orémus:

\ifx\dominicavelpostoctavam\undefined
\noindent Concéde, quǽsumus, omnípotens Deus:~\gredagger{} ut, qui hodiérna die Unigénitum tuum, Redemptórem nostrum, ad cælos ascendísse crédimus;~\grestar{} ipsi quoque mente in cæléstibus habitémus. Per eúmdem Dóminum.
\else
\noindent Omnípotens sempitérne Deus:~\gredagger{} fac nos tibi semper et devótam gérere voluntátem;~\grestar{} et majestáti tuæ sincéro corde servíre. Per Dóminum.
\fi

\noindent \Rbardot{} Amen.

\vspace{7mm}

\pars{Conclusio}

\noindent \Vbardot{} Dómine, exáudi oratiónem meam.

\noindent \Rbardot{} Et clamor meus ad te véniat.

\noindent \Vbardot{} Benedicámus Dómino, allelúia, allelúia.

\noindent \Rbardot{} Deo grátias, allelúia, allelúia.

\noindent \Vbardot{} Fidélium ánimæ per misericórdiam Dei requiéscant in pace.

\noindent \Rbardot{} Amen.

\vfill
\pagebreak

\hora{Ad Laudes.} %%%%%%%%%%%%%%%%%%%%%%%%%%%%%%%%%%%%%%%%%%%%%%%%%%%%%
%\sideThumbs{Laudes}

\cantusSineNeumas

\ifx\postoctavam\undefined
\vspace{0.5cm}
\grechangedim{interwordspacetext}{0.18 cm plus 0.15 cm minus 0.05 cm}{scalable}%
\ifx\festumveldominica\undefined
\cuminitiali{}{temporalia/deusinadiutorium-communis.gtex}
\else
\cuminitiali{}{temporalia/deusinadiutorium-alter.gtex}
\fi
\grechangedim{interwordspacetext}{0.22 cm plus 0.15 cm minus 0.05 cm}{scalable}%

\vfill
%\pagebreak
\else
\rubrica{Absolute incipitur Officium ab Antiphona primi Psalmi.}

\vspace{7mm}
\fi

\pars{Psalmus 1.} \scriptura{Ac. 1, 11; \textbf{H265}}

\vspace{-0.4cm}

\antiphona{VII a}{temporalia/ant-virigalilaeiquidaspicitis.gtex}

\scriptura{Psalmus 92.}

\initiumpsalmi{temporalia/ps92-initium-vii-a-auto.gtex}

\ifx\postoctavam\undefined
\input{temporalia/ps92-vii-a.tex}

\vfill

\vspace{-1cm}

\antiphona{}{temporalia/ant-virigalilaeiquidaspicitis.gtex}
\else
\input{temporalia/ps92-vii-a.tex} \Abardot{}
\fi

\vfill
\pagebreak

\pars{Psalmus 2.} \scriptura{Ac. 1, 10; \textbf{H265}}

\vspace{-0.4cm}

\antiphona{VIII G\textsuperscript{2}}{temporalia/ant-cumqueintuerentur.gtex}

\scriptura{Psalmus 99.}

\initiumpsalmi{temporalia/ps99-initium-viii-G2-auto.gtex}

\input{temporalia/ps99-viii-G2.tex} \Abardot{}

\vfill
\pagebreak

\pars{Psalmus 3.} \scriptura{Lc. 24, 50.51; \textbf{H265}}

\vspace{-0.4cm}

\antiphona{IV A*}{temporalia/ant-elevatismanibus.gtex}

\scriptura{Psalmus 62.}

\initiumpsalmi{temporalia/ps62-initium-iv-A_-auto.gtex}

\input{temporalia/ps62-iv-A_.tex} \Abardot{}

%\vfill

%\vspace{-6mm}

%\antiphona{}{temporalia/ant-elevatismanibus.gtex} % repeat the antiphon - new page

\vfill
\pagebreak

\pars{Psalmus 4.} \scriptura{\textbf{H265}}

\vspace{-0.4cm}

\antiphona{VIII G\textsuperscript{2}}{temporalia/ant-exaltateregemregum.gtex}

\scriptura{Canticum trium puerorum, Dan. 3, 57-88 et 56}

\initiumpsalmi{temporalia/dan3-initium-viii-G2-auto.gtex}

\input{temporalia/dan3-viii-G2-sinedox.tex}

\rubrica{Hic non dicitur Gloria Patri, neque Amen.}

\vfill

\vspace{-6mm}

\antiphona{}{temporalia/ant-exaltateregemregum.gtex} % repeat the antiphon - new page

\vfill
\pagebreak

\pars{Psalmus 5.} \scriptura{Ac. 1, 9; \textbf{H265}}

\vspace{-0.4cm}

\antiphona{VIII G}{temporalia/ant-videntibusillis.gtex}

\scriptura{Psalmus 148.}

\initiumpsalmi{temporalia/ps148-initium-viii-G-auto.gtex}

\input{temporalia/ps148-viii-G-sinedox.tex}

\rubrica{Hic non dicitur Gloria Patri.}

\vfill
\pagebreak

%
\scriptura{Psalmus 149.}

\initiumpsalmi{temporalia/ps149-initium-viii-G-auto.gtex}

\input{temporalia/ps149-viii-G-sinedox.tex}

\rubrica{Hic non dicitur Gloria Patri.}

\vfill
\pagebreak

%
\scriptura{Psalmus 150.}

\initiumpsalmi{temporalia/ps150-initium-viii-G-auto.gtex}

\input{temporalia/ps150-viii-G.tex}

\vfill

\vspace{-6mm}

\antiphona{}{temporalia/ant-videntibusillis.gtex} % repeat the antiphon - new page

\vfill
\pagebreak

\capitulumLaudes

\vfill

\pars{Responsorium breve.} \scriptura{Ps. 46, 6}

\cuminitiali{VI}{temporalia/resp-ascenditdeus.gtex}

\vfill
\pagebreak

\pars{Hymnus}

\cuminitiali{VIII}{temporalia/hym-JesuNostraRedemptio.gtex}
\vspace{-3mm}
%\begin{translatioMulticol}{3}
Výkupné naše, Ježíši,\\
lásko a tužbo nejčistší,\\
tys Tvůrce věcí stvořených\\
a člověk věků posledních.\\
\\
Jaký tě musil soucit vést,\\
žes naše hříchy za své vzal,\\
že chtěl jsi muky smrti nést,\\
bys kletbu smrti z lidí sňal.\columnbreak

Pronikáš v žalář pekelný,\\
propouštíš z něho zajatce.\\
Vítězi, slávou oděný,\\
po boku trůníš u Otce.\\
\\
Kéž donutí té soucit týž,\\
že rány vin v nás zacelíš,\\
nás podle slibu ušetříš\\
a vlídnou tváří potěšíš.\columnbreak

Ty budiž naší radostí,\\
odměnou ve tvé věčnosti,\\
kéž naše sláva veškerá\\
jen z tebe věčně vyvěrá.\\
Amen.
\end{translatioMulticol}


\vfill
%\pagebreak

\pars{Versus.}

% Versus. %%%
\ifx\festum\undefined
\sineinitiali{temporalia/versus-dominusincaelo-communis.gtex}
\else
\sineinitiali{temporalia/versus-dominusincaelo.gtex}
\fi

\vfill
\pagebreak

\ifx\dominicavelpostoctavam\undefined
\pars{Canticum Zachariæ.} \scriptura{Io. 20, 17; \textbf{H265}}

%\vspace{-6mm}

{
\grechangedim{interwordspacetext}{0.18 cm plus 0.15 cm minus 0.05 cm}{scalable}%
\antiphona{VII a}{temporalia/ant-ascendoadpatrem.gtex}
\grechangedim{interwordspacetext}{0.22 cm plus 0.15 cm minus 0.05 cm}{scalable}%
}

%\vspace{-3mm}

\scriptura{Lc. 1, 68-79}

%\vspace{-2.5mm}

\cantusSineNeumas
\initiumpsalmi{temporalia/benedictus-initium-viisoll-a-auto.gtex}

%\vspace{-1.5mm}

\input{temporalia/benedictus-viisoll-a.tex} \Abardot{}
\else
\pars{Canticum Zachariæ.} \scriptura{Io. 15, 26; \textbf{H267}}

\vspace{-3mm}

{
\grechangedim{interwordspacetext}{0.18 cm plus 0.15 cm minus 0.05 cm}{scalable}%
\antiphona{VIII G}{temporalia/ant-cumveneritparaclitus.gtex}
\grechangedim{interwordspacetext}{0.22 cm plus 0.15 cm minus 0.05 cm}{scalable}%
}

%\vspace{-3mm}

\scriptura{Lc. 1, 68-79}

%\vspace{-2.5mm}

\cantusSineNeumas
\initiumpsalmi{temporalia/benedictus-initium-viiisoll-G-auto.gtex}

%\vspace{-1.5mm}

\input{temporalia/benedictus-viiisoll-G.tex}

\vfill

\antiphona{}{temporalia/ant-cumveneritparaclitus.gtex}
\fi

\vspace{-1cm}

\vfill
\pagebreak

%\sideThumbs{{\scriptsize{}Fine horarum}}

\anteOrationem

\pagebreak

% Oratio. %%%
\ifx\dominicavelpostoctavam\undefined
\cuminitiali{}{temporalia/oratio.gtex}
\else
\cuminitiali{}{temporalia/oratio2.gtex}
\fi

\vspace{-1mm}

\vfill

\rubrica{Hebdomadarius dicit iterum Dominus vobiscum, vel cantor dicit:}

\vspace{2mm}

\sineinitiali{temporalia/domineexaudi.gtex}

\rubrica{Postea cantatur a cantore:}

\vspace{2mm}

\ifx\festum\undefined
\cuminitiali{VII}{temporalia/benedicamus-tempore-paschali.gtex}
\else
\cuminitiali{II}{temporalia/benedicamus-solemnism-laud.gtex}
\fi

\vspace{1mm}

\vfill
\pagebreak

\ifx\sabbato\undefined
\ifx\festumveldominica\undefined
\hora{Ad Vesperas.} %%%%%%%%%%%%%%%%%%%%%%%%%%%%%%%%%%%%%%%%%%%%%%%%%%%%%
%\sideThumbs{Vesperæ}
\else
\hora{Ad II. Vesperas.} %%%%%%%%%%%%%%%%%%%%%%%%%%%%%%%%%%%%%%%%%%%%%%%%%%%%%
%\sideThumbs{II. Vesperæ}
\fi

\cantusSineNeumas

\vspace{0.5cm}
\grechangedim{interwordspacetext}{0.18 cm plus 0.15 cm minus 0.05 cm}{scalable}%
\cuminitiali{}{temporalia/deusinadiutorium-solemnis.gtex}
\grechangedim{interwordspacetext}{0.22 cm plus 0.15 cm minus 0.05 cm}{scalable}%

\vfill
%\pagebreak

\vspace{-2mm}

\pars{Psalmus 1.} \scriptura{Ac. 1, 11; \textbf{H265}}

\vspace{-0.4cm}

\antiphona{VII a}{temporalia/ant-virigalilaeiquidaspicitis.gtex}

\scriptura{Psalmus 109.}

\initiumpsalmi{temporalia/ps109-initium-vii-a-auto.gtex}

\input{temporalia/ps109-vii-a.tex}

\vfill

\antiphona{}{temporalia/ant-virigalilaeiquidaspicitis.gtex}

\vspace{-1cm}

\vfill
\pagebreak

\vesperas

\ifx\dominicavelpostoctavam\undefined
\pars{Canticum B. Mariæ V.} \scriptura{Cf. Ps. 23, 7.10; Io. 14, 17.18; Lc. 24, 49; Eph. 4, 10; \textbf{H266}}

\vspace{-5.5mm}

{
\grechangedim{interwordspacetext}{0.18 cm plus 0.15 cm minus 0.05 cm}{scalable}%
\antiphona{II D}{temporalia/ant-orexgloriae.gtex}
\grechangedim{interwordspacetext}{0.22 cm plus 0.15 cm minus 0.05 cm}{scalable}%
}

\vspace{-3mm}

\scriptura{Lc. 1, 46-55}

\vspace{-2.5mm}

\cantusSineNeumas
\initiumpsalmi{temporalia/magnificat-initium-iisoll-D.gtex}

\vspace{-1.5mm}

\input{temporalia/magnificat-iisoll-D.tex} \Abardot{}
\else
\pars{Canticum B. Mariæ V.} \scriptura{Io. 16, 4}

{
\grechangedim{interwordspacetext}{0.18 cm plus 0.15 cm minus 0.05 cm}{scalable}%
\antiphona{VIII G}{temporalia/ant-haeclocutussumvobis.gtex}
\grechangedim{interwordspacetext}{0.22 cm plus 0.15 cm minus 0.05 cm}{scalable}%
}

\scriptura{Lc. 1, 46-55}

\cantusSineNeumas
\initiumpsalmi{temporalia/magnificat-initium-viiisoll-G.gtex}

\input{temporalia/magnificat-viiisoll-G.tex} \Abardot{}
\fi

\vspace{-1cm}

\vfill
\pagebreak

%\sideThumbs{{\scriptsize{}Fine horarum}}

\anteOrationem

\pagebreak

% Oratio. %%%
\ifx\dominicavelpostoctavam\undefined
\cuminitiali{}{temporalia/oratio.gtex}
\else
\cuminitiali{}{temporalia/oratio2.gtex}
\fi

\vspace{-1mm}

\vfill

\rubrica{Hebdomadarius dicit iterum Dominus vobiscum, vel cantor dicit:}

\vspace{2mm}

\sineinitiali{temporalia/domineexaudi.gtex}

\rubrica{Postea cantatur a cantore:}

\vspace{2mm}

\ifx\festum\undefined
\cuminitiali{VII}{temporalia/benedicamus-tempore-paschali.gtex}
\else
\cuminitiali{II}{temporalia/benedicamus-solemnism-2vesp.gtex}
\fi

\vspace{1mm}
\fi

\end{document}

