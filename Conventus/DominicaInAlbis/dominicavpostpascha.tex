\newcommand{\titulus}{\nomenFesti{Dominica V post Pascha.}
\celebratio{Semiduplex.}}
\newcommand{\capitulumLaudes}{\pars{Capitulum.} \scriptura{Iac. 1, 22-24}

\grechangedim{interwordspacetext}{0.12 cm plus 0.15 cm minus 0.05 cm}{scalable}%
\cuminitiali{}{temporalia/capitulum-CarissimiEstote.gtex}
\grechangedim{interwordspacetext}{0.22 cm plus 0.15 cm minus 0.05 cm}{scalable}}
\newcommand{\magnificati}{\pars{Canticum B. Mariæ V.} \scriptura{Io. 16, 24; \textbf{H244}}

%\vspace{-6mm}

{
\grechangedim{interwordspacetext}{0.18 cm plus 0.15 cm minus 0.05 cm}{scalable}%
\antiphona{II D}{temporalia/ant-usquemodo.gtex}
\grechangedim{interwordspacetext}{0.22 cm plus 0.15 cm minus 0.05 cm}{scalable}%
}

%\trAntIMagnificat

%\vspace{-4mm}

\scriptura{Lc. 1, 46-55}

\cantusSineNeumas
\initiumpsalmi{temporalia/magnificat-initium-iisoll-D.gtex}

%\vspace{-7mm}

%\psalmusEtTranslatioT{temporalia/magnificat-IX-comb.tex}{10.2cm}
\input{temporalia/magnificat-IX.tex} \Abardot{}}
\newcommand{\lectioi}{\pars{Lectio I.} \scriptura{1 Pet. 1, 1-5}

\noindent Incipit Epístola prima beáti Petri Apóstoli.

\noindent Petrus Apóstolus Iesu Christi, eléctis ádvenis dispersiónis Ponti, Galátiæ, Cappadóciæ, Asiæ, et Bithýniæ, secúndum præsciéntiam Dei Patris, in sanctificatiónem Spíritus, in obediéntiam, et aspersiónem sánguinis Iesu Christi. Grátia vobis, et pax multiplicétur. Benedíctus Deus et Pater Dómini nostri Iesu Christi, qui secúndum misericórdiam suam magnam regenerávit nos in spem vivam, per resurrectiónem Iesu Christi ex mórtuis, in hæreditátem incorruptíbilem, et incontaminátam, et immarcescíbilem, conservátam in cælis in vobis, qui in virtúte Dei custodímini per fidem in salútem, parátam revelári in témpore novíssimo.}
\newcommand{\responsoriumi}{\pars{Responsorium 1.} \scriptura{\Rbardot{} Ps. 136, 5.6; \Vbardot{} Ps. 136, 1; \textbf{H250}}

\vspace{-5mm}

\responsorium{VIII}{temporalia/resp-sioblitusfuerotui-sinedox.gtex}{}}
\newcommand{\lectioii}{\pars{Lectio II.} \scriptura{1 Pet. 1, 6-12}

\noindent In quo exsultábitis, módicum nunc si opórtet contristári in váriis tentatiónibus: ut probátio vestræ fídei multo pretiósior auro (quod per ignem probátur) inveniátur in laudem, et glóriam, et honórem in revelatióne Iesu Christi: quem cum non vidéritis, dilígitis: in quem nunc quoque non vidéntes créditis: credéntes autem exsultábitis lætítia inenarrábili, et glorificáta: reportántes finem fídei vestræ, salútem animárum. De qua salúte exquisiérunt, atque scrutáti sunt prophétæ, qui de futúra in vobis grátia prophetavérunt: scrutántes in quod vel quale tempus significáret in eis Spíritus Christi: prænúntians eas quæ in Christo sunt passiónes, et posterióres glórias: quibus revelátum est quia non sibimetípsis, vobis autem ministrábant ea quæ nunc nuntiáta sunt vobis per eos qui evangelizavérunt vobis, Spíritu Sancto misso de cælo, in quem desíderant ángeli prospícere.}
\newcommand{\responsoriumii}{\pars{Responsorium 2.} \scriptura{\Rbardot{} Ps. 76, 17.18; \Vbardot{} Ps. 76, 19; \textbf{H250}}

\vspace{-5mm}

\responsorium{II}{temporalia/resp-videruntteaquaedeus-sinedox.gtex}{}}
\newcommand{\lectioiii}{\pars{Lectio III.} \scriptura{1 Pet. 1, 13-21}

\noindent Propter quod succíncti lumbos mentis vestræ, sóbrii, perfécte speráte in eam, quæ offértur vobis, grátiam, in revelatiónem Iesu Christi: quasi fílii obediéntiæ, non configuráti prióribus ignorántiæ vestræ desidériis: sed secúndum eum qui vocávit vos, Sanctum: et ipsi in omni conversatióne sancti sitis: quóniam scriptum est: Sancti éritis, quóniam ego sanctus sum. Et si patrem invocátis eum, qui sine acceptióne personárum iúdicat secúndum uniuscuiúsque opus, in timóre incolátus vestri témpore conversámini. Sciéntes quod non corruptibílibus, auro vel argénto, redémpti estis de vana vestra conversatióne patérnæ traditiónis: sed pretióso sánguine quasi agni immaculáti Christi, et incontamináti: præcógniti quidem ante mundi constitutiónem, manifestáti autem novíssimis tempóribus propter vos, qui per ipsum fidéles estis in Deo, qui suscitávit eum a mórtuis, et dedit ei glóriam, ut fides vestra et spes esset in Deo.}
\newcommand{\responsoriumiii}{\pars{Responsorium 3.} \scriptura{\Rbardot{} Ps. 21, 23; \Vbardot{} Ps. 65, 10; \textbf{H250}}

\vspace{-5mm}

\responsorium{II}{temporalia/resp-narrabonomentuum-cumdox.gtex}{}}
\newcommand{\lectioiv}{\pars{Lectio IV.} \scriptura{Post medium}

\noindent Ex libro sancti Ambrósii Epíscopi de fide resurrectiónis.

\noindent Quóniam Dei mori non póterat Sapiéntia, resúrgere autem non póterat quod mórtuum non erat; assúmitur caro, quæ mori posset: ut dum móritur quod solet, quod mórtuum fúerat, hoc resúrgeret. Neque enim póterat esse, nisi per hóminem, resurréctio: quóniam sicut per hóminem mors, ita et per hóminem resurréctio mortuórum. Ergo resurréxit homo, quóniam homo mórtuus est: resuscitátus homo, sed resúscitans Deus. Tunc secúndum carnem homo, nunc per ómnia Deus. Nunc enim secúndum carnem iam nóvimus Christum, sed carnis grátiam tenémus, ut ipsum primítias quiescéntium, ipsum primogénitum ex mórtuis novérimus.}
\newcommand{\responsoriumiv}{\pars{Responsorium 4.} \scriptura{\Rbardot{} Ps. 67, 27; \Vbardot{} Ps. 65, 2; \textbf{H251}}

\vspace{-5mm}

\responsorium{IV}{temporalia/resp-inecclesiisbenedicitedeo-sinedox.gtex}{}}
\newcommand{\lectiov}{\pars{Lectio V.}

\noindent Primítiæ útique eiúsdem sunt géneris atque natúræ, cuius et réliqui fructus: quorum pro lætióre provéntu primitíva Deo múnera deferúntur; sacrum munus pro ómnibus, et quasi reparátæ quædam liba natúræ. Primítiæ ergo quiescéntium Christus. Sed utrum suórum quiescéntium, qui quasi mortis exsórtes, dulci quodam sopóre tenéntur, an ómnium mortuórum? Sed sicut in Adam omnes moriúntur, ita et in Christo omnes vivificabúntur. Itaque sicut primítiæ mortis in Adam, ita étiam primítiæ resurrectiónis in Christo omnes resúrgent. Sed nemo despéret, neque iustus dóleat commúne consórtium resurgéndi, cum præcípuum fructum virtútis exspéctet. Omnes quidem resúrgent, sed unusquísque, ut ait Apóstolus, in suo órdine. Commúnis est divínæ fructus cleméntiæ, sed distínctus ordo meritórum.}
\newcommand{\responsoriumv}{\pars{Responsorium 5.} \scriptura{\Rbardot{} Ps. 118, 10; \Vbardot{} Ps. 118, 12; \textbf{H250}}

\vspace{-5mm}

\responsorium{III}{temporalia/resp-intotocordemeo-sinedox.gtex}{}}
\newcommand{\lectiovi}{\pars{Lectio VI.}

\noindent Advértimus, quam grave sit sacrilégium, resurrectiónem non crédere. Si enim non resurgémus, ergo Christus gratis mórtuus est, ergo Christus non resurréxit. Si enim nobis non resurréxit, útique non resurréxit, qui sibi cur resúrgeret, non habébat. Resurréxit in eo mundus, resurréxit in eo cælum, resurréxit in eo terra. Erit enim cælum novum, et terra nova. Sibi autem non erat necessária resurréctio, quem mortis víncula non tenébant. Nam etsi secúndum hóminem mórtuus, in ipsis tamen erat liber inférnis. Vis scire quam liber? Factus sum sicut homo sine adiutório, inter mórtuos liber. Et bene liber, qui se póterat suscitáre, iuxta quod scriptum est: Sólvite hoc templum, et in tríduo resuscitábo illud. Et bene liber, qui álios descénderat redemptúrus.}
\newcommand{\responsoriumvi}{\pars{Responsorium 6.} \scriptura{\Rbardot{} Ps. 136, 3.4; \Vbardot{} Ps. 136, 3; \textbf{H251}}

\vspace{-5mm}

\responsorium{VIII}{temporalia/resp-hymnumcantatenobis-cumdox.gtex}{}}
\newcommand{\lectiovii}{\pars{Lectio VII.} \scriptura{Io. 16, 23-30}

\noindent Léctio sancti Evangélii secúndum Ioánnem.

\noindent In illo témpore: Dixit Iesus discípulis suis: Amen, amen dico vobis: si quid petiéritis Patrem in nómine meo, dabit vobis. Et réliqua.

\scriptura{Tractatus 102 in Ioannem}

\noindent Homilía sancti Augustíni Epíscopi.

\noindent Dómini verba nunc ista tractánda sunt: Amen, amen, dico vobis: Si quid petiéritis Patrem in nómine meo, dabit vobis. Iam dictum est in superióribus huius Domínici sermónis pártibus, propter eos, qui nonnúlla petunt a Patre in Christi nómine, nec accípiunt: non peti in nómine Salvatóris, quidquid pétitur contra ratiónem salútis. Non enim sonum litterárum ac syllabárum, sed quod sonus ipse signíficat, et quod eo sono recte ac veráciter intellégitur, hoc accipiéndus est dícere, cum dicit: In nómine meo.}
\newcommand{\responsoriumvii}{\pars{Responsorium 7.} \scriptura{\Rbardot{} Ps. 32, 3.2; \Vbardot{} Ps. 117, 28; \textbf{Sar.263}}

\vspace{-5mm}

\responsorium{VIII}{temporalia/resp-deuscanticumnovum-sinedox.gtex}{}}
\newcommand{\lectioviii}{\pars{Lectio VIII.}

\noindent Unde qui hoc sentit de Christo, quod non est de único Dei Fílio sentiéndum, non petit in eius nómine, etiámsi non táceat lítteris ac sýllabis Christum: quóniam in eius nómine petit, quem cógitat cum petit. Qui vero quod est de illo sentiéndum sentit, ipse in eius nómine petit: et áccipit quod petit, si non contra suam salútem sempitérnam petit. Accipit autem quando debet accípere. Quædam enim non negántur: sed ut cóngruo dentur témpore differúntur. Ita sane intelligéndum est quod ait: Dabit vobis: ut ea benefícia significáta sciántur his verbis, quæ ad eos, qui petunt, próprie pértinent. Exaudiúntur quippe omnes Sancti pro seípsis, non autem pro ómnibus exaudiúntur vel amícis, vel inimícis suis, vel quibúslibet áliis: quia non utcúmque dictum est, Dabit; sed, Dabit vobis.}
\newcommand{\responsoriumviii}{\pars{Responsorium 8.} \scriptura{\Rbardot{} Ps. 91, 2; \Vbardot{} Ps. 91, 4; \textbf{H252}}

\vspace{-5mm}

\responsorium{VII}{temporalia/resp-bonumestconfiteri-cumdox.gtex}{}}
\newcommand{\lectioix}{\pars{Lectio IX.}

\noindent Usque modo, inquit, non petístis quidquam in nómine meo. Pétite, et accipiétis, ut gáudium vestrum sit plenum. Hoc quod dicit, gáudium plenum, profécto non carnále, sed spiritále gáudium est: et quando tantum erit, ut áliquid ei iam non sit addéndum, proculdúbio tunc erit plenum. Quidquid ergo pétitur, quod pertíneat ad hoc gáudium consequéndum, hoc est in nómine Christi peténdum, si divínam intellígimus grátiam, si vere beátam póscimus vitam. Quidquid autem áliud pétitur, nihil pétitur: non quia nulla omníno res est, sed quia in tantæ rei comparatióne quidquid áliud concupíscitur, nihil est.}
\newcommand{\benedictus}{\pars{Canticum Zachariæ.} \scriptura{Io. 16, 24; \textbf{H244}}

%\vspace{-6mm}

{
\grechangedim{interwordspacetext}{0.18 cm plus 0.15 cm minus 0.05 cm}{scalable}%
\antiphona{II D}{temporalia/ant-usquemodo.gtex}
\grechangedim{interwordspacetext}{0.22 cm plus 0.15 cm minus 0.05 cm}{scalable}%
}

%\trAntIMagnificat

%\vspace{-3mm}

\scriptura{Lc. 1, 68-79}

%\vspace{-2.5mm}

\cantusSineNeumas
\initiumpsalmi{temporalia/benedictus-initium-iisoll-D-auto.gtex}

%\vspace{-1.5mm}

%\psalmusEtTranslatioT{temporalia/benedictus-V-comb.tex}{10.2cm}
\input{temporalia/benedictus-V.tex} \Abardot{}}
\newcommand{\magnificatii}{\pars{Canticum B. Mariæ V.} \scriptura{Io. 16, 27.29; \textbf{H244}}

\vspace{-5mm}

{
\grechangedim{interwordspacetext}{0.18 cm plus 0.15 cm minus 0.05 cm}{scalable}%
\antiphona{VIII G\textsuperscript{2}}{temporalia/ant-petiteetaccipietisutgaudium.gtex}
\grechangedim{interwordspacetext}{0.22 cm plus 0.15 cm minus 0.05 cm}{scalable}%
}

%\trAntIMagnificat

\vspace{-3mm}

\scriptura{Lc. 1, 46-55}

\vspace{-2mm}

\cantusSineNeumas
\initiumpsalmi{temporalia/magnificat-initium-viiisoll-G2.gtex}

%\vspace{-1mm}

%\psalmusEtTranslatioT{temporalia/magnificat-III-comb.tex}{10.2cm}
\input{temporalia/magnificat-III.tex} \Abardot{}}
\newcommand{\oratioMatutinum}{\noindent Deus, a quo bona cuncta procédunt, largíre supplícibus tuis:~\gredagger{} ut cogitémus, te inspiránte, quæ recta sunt;~\grestar{} et, te gubernánte, éadem faciámus. Per Dóminum.}
\newcommand{\oratioLaudes}{\cuminitiali{}{temporalia/oratio5.gtex}}
\include{dominicatp}
