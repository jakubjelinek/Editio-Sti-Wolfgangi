\newcommand{\titulus}{\nomenFesti{Dominica III post Pascha.}
\celebratio{Semiduplex.}}
\newcommand{\capitulumLaudes}{\pars{Capitulum.} \scriptura{1 Petr. 2, 11}

\grechangedim{interwordspacetext}{0.12 cm plus 0.15 cm minus 0.05 cm}{scalable}%
\cuminitiali{}{temporalia/capitulum-CarissimiObsecro.gtex}
\grechangedim{interwordspacetext}{0.22 cm plus 0.15 cm minus 0.05 cm}{scalable}}
\newcommand{\magnificati}{\pars{Canticum B. Mariæ V.} \scriptura{Io. 16, 17; \textbf{H243}}

\vspace{-6mm}

{
\grechangedim{interwordspacetext}{0.18 cm plus 0.15 cm minus 0.05 cm}{scalable}%
\antiphona{VI F}{temporalia/ant-modicumetnon.gtex}
\grechangedim{interwordspacetext}{0.22 cm plus 0.15 cm minus 0.05 cm}{scalable}%
}

%\trAntIMagnificat

\vspace{-4mm}

\scriptura{Lc. 1, 46-55}

\cantusSineNeumas
\initiumpsalmi{temporalia/magnificat-initium-visoll-F.gtex}

%\vspace{-7mm}

%\psalmusEtTranslatioT{temporalia/magnificat-V-comb.tex}{10.2cm}
\input{temporalia/magnificat-V.tex} \Abardot{}}
\newcommand{\lectioi}{\pars{Lectio I.} \scriptura{Ap. 1, 1-6}

\noindent Incipit liber Apocalýpsis beáti Ioánnis Apóstoli.

\noindent Apocalýpsis Iesu Christi, quam dedit illi Deus palam fácere servis suis, quæ opórtet fíeri cito: et significávit, mittens per ángelum suum servo suo Ioánni, qui testimónium perhíbuit verbo Dei, et testimónium Iesu Christi, quæcúmque vidit. Beátus qui legit, et audit verba prophetíæ huius, et servat ea, quæ in ea scripta sunt: tempus enim prope est. Ioánnes septem ecclésiis, quæ sunt in Asia. Grátia vobis, et pax ab eo, qui est, et qui erat, et qui ventúrus est: et a septem spirítibus qui in conspéctu throni eius sunt: et a Iesu Christo, qui est testis fidélis, primogénitus mortuórum, et princeps regum terræ, qui diléxit nos, et lavit nos a peccátis nostris in sánguine suo, et fecit nos regnum, et sacerdótes Deo et Patri suo: ipsi glória et impérium in sǽcula sæculórum. Amen.}
\newcommand{\responsoriumi}{\pars{Responsorium 1.} \scriptura{\Rbardot{} Ap. 5, 9; \Vbardot{} ibid. 5, 10; \textbf{H247}}

\vspace{-5mm}

\responsorium{VII}{temporalia/resp-dignusesdomine-sinedox.gtex}{}}
\newcommand{\lectioii}{\pars{Lectio II.} \scriptura{Ap. 1, 7-11}

\noindent Ecce venit cum núbibus, et vidébit eum omnis óculus, et qui eum pupugérunt. Et plangent se super eum omnes tribus terræ. Etiam: amen. Ego sum alpha et ómega, princípium et finis, dicit Dóminus Deus: qui est, et qui erat, et qui ventúrus est, omnípotens. Ego Ioánnes frater vester, et párticeps in tribulatióne, et regno, et patiéntia in Christo Iesu: fui in ínsula, quæ appellátur Patmos, propter verbum Dei, et testimónium Iesu: fui in spíritu in domínica die, et audívi post me vocem magnam tamquam tubæ, dicéntis: Quod vides, scribe in libro: et mitte septem ecclésiis, quæ sunt in Asia, Epheso, et Smyrnæ, et Pérgamo, et Thyatíræ, et Sardis, et Philadelphíæ, et Laodicíæ.}
\newcommand{\responsoriumii}{\pars{Responsorium 2.} \scriptura{\Rbardot{} Eccli. 24, 23.26; \Vbardot{} ibid. 24, 25; \textbf{H247}}

\vspace{-5mm}

\responsorium{III}{temporalia/resp-egosicutvitis-sinedox.gtex}{}}
\newcommand{\lectioiii}{\pars{Lectio III.} \scriptura{Ap. 1, 12-19}

\noindent Et convérsus sum ut vidérem vocem, quæ loquebátur mecum: et convérsus vidi septem candelábra áurea: et in médio septem candelabrórum aureórum, símilem Fílio hóminis vestítum podére, et præcínctum ad mamíllas zona áurea: caput autem eius, et capílli erant cándidi tamquam lana alba, et tamquam nix, et óculi eius tamquam flamma ignis: et pedes eius símiles aurichálco, sicut in camíno ardénti, et vox illíus tamquam vox aquárum multárum: et habébat in déxtera sua stellas septem: et de ore eius gládius utráque parte acútus exíbat: et fácies eius sicut sol lucet in virtúte sua. Et cum vidíssem eum, cécidi ad pedes eius tamquam mórtuus. Et pósuit déxteram suam super me, dicens: Noli timére: ego sum primus, et novíssimus, et vivus, et fui mórtuus, et ecce sum vivens in sǽcula sæculórum: et hábeo claves mortis, et inférni. Scribe ergo quæ vidísti, et quæ sunt, et quæ opórtet fíeri post hæc.}
\newcommand{\responsoriumiii}{\pars{Responsorium 3.} \scriptura{\Rbardot{} Ap. 14, 2 \& Ap. 11, 15 \& Fulgentius Ruspensis (pseudo): Liber de Trinitate, cap. 6; \Vbardot{} Ap. 19, 5; \textbf{H247}}

\vspace{-5mm}

\responsorium{VII}{temporalia/resp-audivivocemincaelotamquam-cumdox.gtex}{}}
\newcommand{\lectioiv}{\pars{Lectio IV.} \scriptura{Sermo 147 de Tempore}

\noindent Sermo sancti Augustíni Epíscopi.

\noindent Diébus his sanctis resurrectióni Dómini dedicátis, quantum donánte ipso póssumus, de carnis resurrectióne tractémus. Hæc enim est fides nostra: hoc donum in Dómini nostri Iesu Christi nobis carne promíssum est, et in ipso præcéssit exémplum. Vóluit enim nobis, quod promísit in fine, non solum prænuntiáre, sed étiam demonstráre. Illi quidem qui tunc fuérunt, cum illum vidérent, et cum expavéscerent, et spíritum se vidére créderent, soliditátem córporis tenuérunt. Locútus est enim non solum verbis ad aures eórum, sed étiam spécie ad óculos eórum: parúmque erat se præbére cernéndum, nisi étiam offérret pertractándum atque palpándum.}
\newcommand{\responsoriumiv}{\pars{Responsorium 4.} \scriptura{\Rbardot{} Ap. 21, 9.2; \Vbardot{} ibid. 21, 10; \textbf{H247}}

\vspace{-5mm}

\responsorium{II}{temporalia/resp-locutusestadmeunus-sinedox.gtex}{}}
\newcommand{\lectiov}{\pars{Lectio V.}

\noindent Ait enim: Quid turbáti estis, et cogitatiónes ascéndunt in cor vestrum? Putavérunt enim se spíritum vidére. Quid turbáti estis, inquit, et cogitatiónes ascéndunt in cor vestrum? Vidéte manus meas, et pedes meos: palpáte, et vidéte: quia spíritus ossa et carnem non habet, sicut me vidétis habére. Contra istam evidéntiam disputábant hómines. Quid enim áliud fácerent hómines, qui ea, quæ sunt hóminum, sápiunt, quam sic disputáre de Deo contra Deum? Ille enim Deus est, isti hómines sunt. Sed Deus novit cogitatiónes hóminum, quóniam vanæ sunt.}
\newcommand{\responsoriumv}{\pars{Responsorium 5.} \scriptura{\Rbardot{} Ap. 14, 7; \Vbardot{} ibid. 14, 6.7; \textbf{H248}}

\vspace{-5mm}

\responsorium{I}{temporalia/resp-audivivocemincaeloangelorum-sinedox.gtex}{}}
\newcommand{\lectiovi}{\pars{Lectio VI.}

\noindent In hómine carnáli tota régula intelligéndi est consuetúdo cernéndi. Quod solent vidére, credunt: quod non solent, non credunt. Præter consuetúdinem facit Deus mirácula, quia Deus est. Maióra quidem mirácula sunt, tot quotídie hómines nasci, qui non erant, quam paucos resurrexísse, qui erant: et tamen ista mirácula non consideratióne comprehénsa sunt, sed assiduitáte viluérunt. Resurréxit Christus: absolúta est res. Corpus erat, caro erat: pepéndit in cruce, emísit ánimam, pósita est caro in sepúlcro. Exhíbuit illam vivam, qui vivébat in illa. Quare mirámur? quare non crédimus? Deus est, qui fecit.}
\newcommand{\responsoriumvi}{\pars{Responsorium 6.} \scriptura{\Rbardot{} Cantor; \Vbardot{} ibidem; \textbf{H249}}

\vspace{-5mm}

\responsorium{III}{temporalia/resp-veniensalibano-cumdox.gtex}{}}
\newcommand{\lectiovii}{\pars{Lectio VII.} \scriptura{Io. 16, 16-22}

\noindent Léctio sancti Evangélii secúndum Ioánnem.

\noindent In illo témpore: Dixit Iesus discípulis suis: Módicum, et iam non vidébitis me; et íterum módicum, et vidébitis me: quia vado ad Patrem. Et réliqua.

\scriptura{Tr. 101 in Ioann., sub fine}

\noindent Homilía sancti Augustíni Epíscopi.

\noindent Módicum est hoc totum spátium, quo præsens pérvolat sǽculum. Unde dicit idem ipse Evangelísta in Epístola sua: Novíssima hora est. Ideo namque áddidit: Quia vado ad Patrem: quod ad priórem senténtiam referéndum est, ubi ait: Módicum et iam non vidébitis me: non ad posteriórem, ubi ait: Et íterum módicum, et vidébitis me. Eúndo quippe ad Patrem, factúrus erat ut eum non vidérent. Ac per hoc non ídeo dictum est, quia fúerat moritúrus, et donec resúrgeret, ab eórum aspéctibus recessúrus: sed quod esset itúrus ad Patrem, quod fecit posteáquam resurréxit, et cum eis per quadragínta dies conversátus, ascéndit in cælum.}
\newcommand{\responsoriumvii}{\pars{Responsorium 7.} \scriptura{\Rbardot{} 1 Paralip. 15, 28.27; \Vbardot{} ibid. 15, 14.28; \textbf{H248}}

\vspace{-5mm}

\responsorium{III}{temporalia/resp-decantabatpopulus-sinedox.gtex}{}}
\newcommand{\lectioviii}{\pars{Lectio VIII.}

\noindent Illis ergo ait: Módicum, et iam non vidébitis me; qui eum corporáliter tunc vidébant: quia itúrus erat ad Patrem, et eum deínceps mortálem visúri non erant, qualem, cum ista loquerétur, vidébant. Quod vero áddidit: Et íterum módicum, et vidébitis me: univérsæ promísit Ecclésiæ, sicut univérsæ promísit: Ecce ego vobíscum sum usque ad consummatiónem sǽculi. Non tardat Dóminus promíssum. Módicum et vidébimus eum: ubi iam nihil rogémus, nihil interrogémus, quia nihil desiderándum remanébit, nihil quæréndum latébit.}
\newcommand{\responsoriumviii}{\pars{Responsorium 8.} \scriptura{\Rbardot{} Io. 16, 20; \Vbardot{} Io. 16, 20; \textbf{Sar.pl.I.}}

\vspace{-5mm}

\responsorium{VII}{temporalia/resp-tristitiavestra-N450-cumdox.gtex}{}}
\newcommand{\lectioix}{\pars{Lectio IX.}

\noindent Hoc módicum longum nobis vidétur, quóniam adhuc ágitur; cum finítum fúerit, tunc sentiémus quam módicum fúerit. Non ergo sit gáudium nostrum quale habet mundus, de quo dictum est: Mundus autem gaudébit. Nec tamen in huius desidérii parturitióne sine gáudio tristes simus: sed, sicut ait Apóstolus: Spe gaudéntes: In tribulatióne patiéntes: quia et ipsa partúriens, cui comparáti sumus, plus gaudet de mox futúra prole, quam tristis est de præsénti dolóre. Sed huius sermónis iste sit finis: habent enim quæstiónem molestíssimam, quæ sequúntur: nec brevitáte coarctánda sunt, ut possint commódius, si Dóminus volúerit, explicári.}
\newcommand{\benedictus}{\pars{Canticum Zachariæ.} \scriptura{Io. 16, 17; \textbf{H243}}

\vspace{-6mm}

{
\grechangedim{interwordspacetext}{0.18 cm plus 0.15 cm minus 0.05 cm}{scalable}%
\antiphona{VI F}{temporalia/ant-modicumetnon.gtex}
\grechangedim{interwordspacetext}{0.22 cm plus 0.15 cm minus 0.05 cm}{scalable}%
}

%\trAntIMagnificat

\vspace{-3mm}

\scriptura{Lc. 1, 68-79}

\vspace{-2.5mm}

\cantusSineNeumas
\initiumpsalmi{temporalia/benedictus-initium-visoll-F-auto.gtex}

\vspace{-1.5mm}

%\psalmusEtTranslatioT{temporalia/benedictus-III-comb.tex}{10.2cm}
\input{temporalia/benedictus-III.tex} \Abardot{}}
\newcommand{\magnificatii}{\pars{Canticum B. Mariæ V.} \scriptura{Io. 16, 20; \textbf{H243}}

\vspace{-6mm}

{
\grechangedim{interwordspacetext}{0.18 cm plus 0.15 cm minus 0.05 cm}{scalable}%
\antiphona{VIII G}{temporalia/ant-amenamendico.gtex}
\grechangedim{interwordspacetext}{0.22 cm plus 0.15 cm minus 0.05 cm}{scalable}%
}

%\trAntIMagnificat

\vspace{-3mm}

\scriptura{Lc. 1, 46-55}

\vspace{-2mm}

\cantusSineNeumas
\initiumpsalmi{temporalia/magnificat-initium-viiisoll-G.gtex}

%\vspace{-7mm}

%\psalmusEtTranslatioT{temporalia/magnificat-VI-comb.tex}{10.2cm}
\input{temporalia/magnificat-VI.tex} \Abardot{}}
\newcommand{\oratioMatutinum}{\noindent Deus, qui errántibus, ut in viam possint redíre justítiæ, veritátis tuæ lumen osténdis:~\gredagger{} da cunctis qui christiána professióne censéntur, et illa respúere quæ huic inimíca sunt nómini;~\grestar{} et ea quæ sunt apta sectári. Per Dóminum.}
\newcommand{\oratioLaudes}{\cuminitiali{}{temporalia/oratio3.gtex}}
\include{dominicatp}
