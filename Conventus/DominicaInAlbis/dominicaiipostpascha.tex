\newcommand{\titulus}{\nomenFesti{Dominica II post Pascha.}
\celebratio{Semiduplex.}}
\newcommand{\capitulumLaudes}{\pars{Capitulum.} \scriptura{1 Petr. 2, 21-22}

\grechangedim{interwordspacetext}{0.12 cm plus 0.15 cm minus 0.05 cm}{scalable}%
\cuminitiali{}{temporalia/capitulum-CarissimiChristusPassus.gtex}
\grechangedim{interwordspacetext}{0.22 cm plus 0.15 cm minus 0.05 cm}{scalable}}
\newcommand{\magnificati}{\pars{Canticum B. Mariæ V.} \scriptura{Cf. Io. 10, 2.14; Io. 14, 6; \textbf{H242}}

\vspace{-6mm}

{
\grechangedim{interwordspacetext}{0.18 cm plus 0.15 cm minus 0.05 cm}{scalable}%
\antiphona{VIII G\textsuperscript{2}}{temporalia/ant-egosumpastorovium.gtex}
\grechangedim{interwordspacetext}{0.22 cm plus 0.15 cm minus 0.05 cm}{scalable}%
}

%\trAntIMagnificat

\vspace{-4mm}

\scriptura{Lc. 1, 46-55}

\cantusSineNeumas
\initiumpsalmi{temporalia/magnificat-initium-viiisoll-G2.gtex}

%\vspace{-7mm}

%\psalmusEtTranslatioT{temporalia/magnificat-III-comb.tex}{10.2cm}
\input{temporalia/magnificat-III.tex} \Abardot{}}
\newcommand{\lectioi}{\pars{Lectio I.} \scriptura{Ac. 13, 13-20}

\noindent De Actibus Apostolórum.

\noindent Cum a Papho navigássent Paulus et qui cum eo erant, venérunt Pergen Pamphýliæ. Ioánnes autem discédens ab eis, revérsus est Ierosólymam. Illi vero pertranseúntes Pergen, venérunt Antiochíam Pisídiæ: et ingréssi synagógam die sabbatórum, sedérunt. Post lectiónem autem legis et prophetárum, misérunt príncipes synagógæ ad eos, dicéntes: Viri fratres, si quis est in vobis sermo exhortatiónis ad plebem, dícite. Surgens autem Paulus, et manu siléntium indícens, ait: Viri Israëlítæ, et qui timétis Deum, audíte: Deus plebis Israël elégit patres nostros, et plebem exaltávit cum essent íncolæ in terra Ægýpti, et in brácchio excélso edúxit eos ex ea, et per quadragínta annórum tempus mores eórum sustínuit in desérto. Et déstruens gentes septem in terra Chánaan, sorte distríbuit eis terram eórum, quasi post quadringéntos et quinquagínta annos: et post hæc dedit iúdices, usque ad Sámuel prophétam.}
\newcommand{\responsoriumi}{\pars{Responsorium 1.} \scriptura{\Rbardot{} Mt. 28, 2; \Vbardot{} Bed. Ven. «In Marc. 4.16»; \textbf{H228}}

\vspace{-5mm}

\responsorium{III}{temporalia/resp-angelusdominidescendit.gtex}{}}
\newcommand{\lectioii}{\pars{Lectio II.} \scriptura{Ac. 13, 21-25}

\noindent Et exínde postulavérunt regem: et dedit illis Deus Saul fílium Cis, virum de tribu Béniamin, annis quadragínta: et amóto illo, suscitávit illis David regem: cui testimónium pérhibens, dixit: Invéni David fílium Iesse, virum secúndum cor meum, qui fáciet omnes voluntátes meas. Huius Deus ex sémine secúndum promissiónem edúxit Israël salvatórem Iesum, prædicánte Ioánne ante fáciem advéntus eius baptísmum pœniténtiæ omni pópulo Israël. Cum impléret autem Ioánnes cursum suum, dicébat: Quem me arbitrámini esse, non sum ego: sed ecce venit post me, cuius non sum dignus calceaménta pedum sólvere.}
\newcommand{\responsoriumii}{\pars{Responsorium 2.} \scriptura{\Rbardot{} Mt. 28, 5.6; \Vbardot{} Mc. 16, 6; \textbf{H229}}

\vspace{-5mm}

\responsorium{V}{temporalia/resp-angelusdominilocutus.gtex}{}}
\newcommand{\lectioiii}{\pars{Lectio III.} \scriptura{Ac. 13, 26-33}

\noindent Viri fratres, fílii géneris Abraham, et qui in vobis timent Deum, vobis verbum salútis huius missum est. Qui enim habitábant Ierúsalem, et príncipes eius hunc ignorántes, et voces prophetárum quæ per omne sábbatum legúntur, iudicántes implevérunt, et nullam causam mortis inveniéntes in eo, petiérunt a Piláto ut interfícerent eum. Cumque consummássent ómnia quæ de eo scripta erant, deponéntes eum de ligno, posuérunt eum in monuménto. Deus vero suscitávit eum a mórtuis tértia die: qui visus est per dies multos his qui simul ascénderant cum eo de Galilǽa in Ierúsalem: qui usque nunc sunt testes eius ad plebem. Et nos vobis annuntiámus eam, quæ ad patres nostros repromíssio facta est: quóniam hanc Deus adimplévit fíliis nostris resúscitans Iesum, sicut et in psalmo secúndo scriptum est: Fílius meus es tu, ego hódie génui te.}
\newcommand{\responsoriumiii}{\pars{Responsorium 3.} \scriptura{\Rbardot{} Mc. 16, 1; \Vbardot{} ibid. 16, 2; \textbf{H229}}

\vspace{-5mm}

\responsorium{IV}{temporalia/resp-dumtransisset.gtex}{}}
\newcommand{\lectioiv}{\pars{Lectio IV.} \scriptura{Sermo 1 de Ascensione Domini, post initium}

\noindent Sermo sancti Leónis Papæ.

\noindent Hi dies, dilectíssimi, qui inter resurrectiónem Dómini ascensionémque fluxérunt, non otióso transiére decúrsu, sed magna in eis confirmáta sacraménta, magna sunt reveláta mystéria. In iis metus diræ mortis aufértur, et non solum ánimæ, sed étiam carnis immortálitas declarátur. In iis per insufflatiónem Dómini infúnditur Apóstolis ómnibus Spíritus Sanctus: et beáto Apóstolo Petro supra céteros, post regni claves, ovílis Domínici cura mandátur.}
\newcommand{\responsoriumiv}{\pars{Responsorium 4.} \scriptura{\Rbardot{} Mt. 28, 1 \& Cantor; \Vbardot{} ibidem; \textbf{H232}}

\vspace{-5mm}

\responsorium{VIII}{temporalia/resp-mariamagdalene.gtex}{}}
\newcommand{\lectiov}{\pars{Lectio V.}

\noindent In iis diébus, duóbus discípulis tértius in via Dóminus comes iúngitur, et ad omnem nostræ ambiguitátis calíginem detergéndam, pavéntium ac trepidántium tárditas increpátur. Flammam fídei illumináta corda concípiunt: et quæ erant tépida, reseránte Scriptúras Dómino, efficiúntur ardéntia. In fractióne quoque panis, convescéntium aperiúntur obtútus: multo felícius eórum óculis patefáctis, quibus natúræ suæ manifestáta est glorificátio, quam illórum géneris nostri príncipum, quibus prævaricatiónis suæ est ingésta confúsio.}
\newcommand{\responsoriumv}{\pars{Responsorium 5.} \scriptura{\Rbardot{} Io. 10, 11; \Vbardot{} I Cor. 5, 7; \textbf{H237}}

\vspace{-5mm}

\responsorium{I}{temporalia/resp-surrexitpastor.gtex}{}}
\newcommand{\lectiovi}{\pars{Lectio VI.}

\noindent Inter hæc autem, aliáque mirácula, cum discípuli trépidis cogitatiónibus æstuárent, et apparuísset in médio eórum Dóminus, dixissétque, Pax vobis: ne hoc remanéret in eórum opiniónibus, quod volvebátur in córdibus (putábant enim se spíritum vidére, non carnem) redárguit cogitatiónes a veritáte discórdes: íngerit dubitántium óculis manéntia in mánibus suis et pédibus crucis signa; et ut diligéntius pertractétur, invítat. Quia ad sanánda infidélium cordium vúlnera, clavórum et lánceæ erant serváta vestígia: ut non dúbia fide, sed constantíssima sciéntia tenerétur, eam natúram in Dei Patris consessúram throno, quæ iacúerat in sepúlcro.}
\newcommand{\responsoriumvi}{\pars{Responsorium 6.} \scriptura{\Rbardot{} Ac. 4, 33; \Vbardot{} ibid. 4, 31; \textbf{H234}}

\vspace{-5mm}

\responsorium{III}{temporalia/resp-virtutemagna.gtex}{}}
\newcommand{\lectiovii}{\pars{Lectio VII.} \scriptura{Io. 10, 11-16}

\noindent Léctio sancti Evangélii secúndum Ioánnem.

\noindent In illo témpore: Dixit Iesus pharisǽis: Ego sum pastor bonus. Bonus pastor ánimam suam dat pro óvibus suis. Et réliqua.

\scriptura{Homilia 14 in Evangelia}

\noindent Homilía sancti Gregórii Papæ.

\noindent Audístis, fratres caríssimi, ex lectióne evangélica eruditiónem vestram: audístis et perículum nostrum. Ecce enim is, qui non ex accidénti dono, sed essentiáliter bonus est, dicit: Ego sum pastor bonus. Atque eiúsdem bonitátis formam, quam nos imitémur, adiúngit, dicens: Bonus pastor ánimam suam ponit pro óvibus suis. Fecit quod mónuit: osténdit quod iussit. Bonus pastor pro óvibus suis ánimam suam pósuit, ut in sacraménto nostro corpus suum et sánguinem vérteret, et oves quas redémerat, carnis suæ aliménto satiáret.}
\newcommand{\responsoriumvii}{\pars{Responsorium 7.} \scriptura{\Rbardot{} Cant. 4, 11; \Vbardot{} Parab. 14, 33; \textbf{H236}}

\vspace{-5mm}

\responsorium{VII}{temporalia/resp-deoreprudentis.gtex}{}}
\newcommand{\lectioviii}{\pars{Lectio VIII.}

\noindent Osténsa nobis est de contémptu mortis via, quam sequámur: appósita est forma, cui imprimámur. Primum nobis est, exterióra nostra misericórditer óvibus eius impéndere: postrémum vero, si necésse sit, étiam mortem nostram pro eísdem óvibus ministráre. A primo autem hoc mínimo pervenítur ad postrémum maius. Sed cum incomparabíliter longe sit mélior ánima, qua vívimus, quam terréna substántia, quam extérius possidémus: qui non dat pro óvibus substántiam suam, quando pro his datúrus est ánimam suam?}
\newcommand{\responsoriumviii}{\pars{Responsorium 8.} \scriptura{\Rbardot{} Io. 20, 19-20; \Vbardot{} ibid. 20, 19; \textbf{H232}}

\vspace{-5mm}

\responsorium{VII}{temporalia/resp-surgensjesus.gtex}{}}
\newcommand{\lectioix}{\pars{Lectio IX.}

\noindent Et sunt nonnúlli, qui dum plus terrénam substántiam quam oves díligunt, mérito nomen pastóris perdunt: de quibus prótinus súbditur: Mercenárius autem, et qui non est pastor, cuius non sunt oves própriæ, videt lupum veniéntem, et dimíttet oves, et fugit. Non pastor, sed mercenárius vocátur, qui non pro amóre íntimo oves Domínicas, sed ad temporáles mercédes pascit. Mercenárius quippe est, qui locum quidem pastóris tenet, sed lucra animárum non quærit: terrénis cómmodis ínhiat, honóre prælatiónis gaudet, temporálibus lucris páscitur, impénsa sibi ab homínibus reveréntia lætátur.}
\newcommand{\benedictus}{\pars{Canticum Zachariæ.} \scriptura{Cf. Io. 10, 2.14; Io. 14, 6; \textbf{H242}}

\vspace{-7mm}

{
\grechangedim{interwordspacetext}{0.18 cm plus 0.15 cm minus 0.05 cm}{scalable}%
\antiphona{I D}{temporalia/ant-egosumpastorovium.gtex}
\grechangedim{interwordspacetext}{0.22 cm plus 0.15 cm minus 0.05 cm}{scalable}%
}

%\trAntIMagnificat

\vspace{-3mm}

\scriptura{Lc. 1, 68-79}

\vspace{-3mm}

\cantusSineNeumas
\initiumpsalmi{temporalia/benedictus-initium-viiisoll-G2-auto.gtex}

%\psalmusEtTranslatioT{temporalia/benedictus-II-comb.tex}{10.2cm}
\input{temporalia/benedictus-II.tex} \Abardot{}}
\newcommand{\magnificatii}{\pars{Canticum B. Mariæ V.} \scriptura{Cf. Io. 10, 14}

\vspace{-4mm}

{
\grechangedim{interwordspacetext}{0.18 cm plus 0.15 cm minus 0.05 cm}{scalable}%
\antiphona{III a}{temporalia/ant-egosumpastorbonus.gtex}
\grechangedim{interwordspacetext}{0.22 cm plus 0.15 cm minus 0.05 cm}{scalable}%
}

%\trAntIMagnificat

\vspace{-3mm}

\scriptura{Lc. 1, 46-55}

\cantusSineNeumas
\initiumpsalmi{temporalia/magnificat-initium-iiisoll-a.gtex}

%\vspace{-7mm}

%\psalmusEtTranslatioT{temporalia/magnificat-IV-comb.tex}{10.2cm}
\input{temporalia/magnificat-IV.tex} \Abardot{}}
\newcommand{\oratioMatutinum}{\noindent Deus, qui in Fílii tui humilitáte iacéntem mundum erexísti:~\gredagger{} fidélibus tuis perpétuam concéde lætítiam;~\grestar{} ut quos perpétuæ mortis eripuísti cásibus, gáudiis fácias pérfrui sempitérnis. Per eúmdem Dóminum.}
\newcommand{\oratioLaudes}{\cuminitiali{}{temporalia/oratio2.gtex}}
\include{dominicatp}
