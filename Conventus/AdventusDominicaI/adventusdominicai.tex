% LuaLaTeX

\documentclass[a4paper, twoside, 12pt]{article}
\usepackage[latin]{babel} 
%\usepackage[landscape, left=3cm, right=1.5cm, top=2cm, bottom=1cm]{geometry} % okraje stranky
\usepackage[landscape, a4paper, mag=1166, truedimen, left=2cm, right=1.5cm, top=1.6cm, bottom=0.95cm]{geometry} % okraje stranky

\usepackage{fontspec}
\setmainfont[FeatureFile={junicode.fea}, Ligatures={Common, TeX}, RawFeature=+fixi]{Junicode}
%\setmainfont{Junicode}

% shortcut for Junicode without ligatures (for the Czech texts)
\newfontfamily\nlfont[FeatureFile={junicode.fea}, Ligatures={Common, TeX}, RawFeature=+fixi]{Junicode}

% Hebrew font:
% http://scripts.sil.org/cms/scripts/page.php?site_id=nrsi&id=SILHebrUnic2
\newfontfamily\hebfont[Scale=1]{Ezra SIL}

\usepackage{multicol}
\usepackage{color}
\usepackage{lettrine}
\usepackage{fancyhdr}

% usual packages loading:
\usepackage{luatextra}
\usepackage{graphicx} % support the \includegraphics command and options
\usepackage{gregoriotex} % for gregorio score inclusion
\usepackage{gregoriosyms}
\usepackage{wrapfig} % figures wrapped by the text
\usepackage{parcolumns}
\usepackage[contents={},opacity=1,scale=1,color=black]{background}
\usepackage{tikzpagenodes}
\usepackage{calc}
\usepackage{longtable}
\usetikzlibrary{calc}

\setlength{\headheight}{14.5pt}

\input{conventuscommune.tex} % Often used macros
%%%% Preklady jednotlivych zpevu (nektere se opakuji, a je dobre mit je
% vsechny na jedne hromade)

% HOURS ---

\newcommand{\trAntI}{\translatioCantus{Muž boží měl kožený toulec, pečlivě
zavázaný, jenž mu visel na šíji a~často se ho dotýkal.}}

\newcommand{\trAntII}{\translatioCantus{Klíč od~něho tak dobře střežil, že
dokud žil v~těle, nikdo z~jeho žáků nezvěděl, co je uvnitř.}}

\newcommand{\trAntIII}{\translatioCantus{Ale když se odebral z~tohoto
života, schránku otevřeli a~objevili v~ní žíněné roucho a~měděný řetěz
potřísněný krví.}}

\newcommand{\trAntIV}{\translatioCantus{A když prohlédli mistrovo tělo,
nalezli jeho tělo na čtyřech místech hluboce zbrázděno ranami od řetězu.}}

\newcommand{\trAntV}{\translatioCantus{Krev vytékající z~těch ran, místy
prostoupila i~žíněným rouchem.}}

\newcommand{\trCapituli}{\translatioCantus{
Miláčkovi Boha a~lidí,
Mojžíšovi požehnané paměti,~\gredagger{}
dopřál slávu rovnou slávě svatých~\grestar{}
učinil ho mocným na postrach nepřátelům
a~jeho slovy zastavil divy.}}

\newcommand{\trLectioBrevis}{\translatioCantus{
Pamatujte na své představené,
kteří vám hlásali Boží slovo.
Uvažte, jak oni skončili život, a~napodobujte jejich víru.
Ježíš Kristus je stejný včera i~dnes i~navěky.
Nenechte se svést věelijakými cizími naukami.}}

\newcommand{\trRespLaud}{\translatioCantus{Spravedlivého vodil Hospodin~\grestar{}
po přímých stezkách. \Vbardot{} A~ukázal mu Boží království.}}

\newcommand{\trRespLaudB}{\translatioCantus{Na tvých hradbách, Jeruzaléme,
ustanovil jsem strážné;~\grestar{}
budou bdít nad mým lidem. \Vbardot{} Ani ve dne, ani v~noci nesmějí nikdy
mlčet.}}

\newcommand{\trVersus}{\translatioCantus{\Vbardot{} Ústa spravedlivého šeptají moudrost, aleluja.
\Rbardot{} A~jeho jazyk ohlašuje právo, aleluja.}}

\newcommand{\trAntBenedictus}{\translatioCantus{Když na bujné oře vložili
nosítka a~sňali jim uzdu, vydali se přímo k~cele božího muže.}}

\newcommand{\trPreces}{\translatioCantus{
\noindent S vděčností chvalme Krista, dobrého Pastýře, \gredagger{} který dal život za své ovce, \grestar{} a~pokorně ho prosme: \Rbardot{} Pane, buď pastýřem svého lidu.

\noindent Kriste, ty dáváš církvi pastýře, a~jejich službou se ujímáš svého lidu, \grestar{} dej, ať v~lásce těch, kteří nás vedou, poznáváme, jak nás miluješ. \Rbardot{} Pane, buď pastýřem svého lidu.

\noindent Ty stále konáš skrze své zástupce službu pastýře a~učitele, \grestar{} nepřestávej nás nikdy vést prostřednictvím svých služebníků. \Rbardot{} Pane, buď pastýřem svého lidu.

\noindent Ty prokazuješ svému lidu skrze jeho pastýře službu lékaře duše i~těla, \grestar{} ochraňuj náš život a~veď nás ke svatosti. \Rbardot{} Pane, buď pastýřem svého lidu.

\noindent Ty posíláš své svaté, aby slovem i~příkladem vedli tvůj lid k~tobě, \grestar{} na jejich přímluvu nás posiluj, abychom vytrvali na cestě, která vede k~věčnému životu. \Rbardot{} Pane, buď pastýřem svého lidu.}}

\newcommand{\trOrationis}{\translatioCantus{Bože, jenž nám dopřáváš radovat
se z~výroční slavnosti svatého tvého vyznavače Havla, uděl dobrotivě,
abychom když slavíme jeho narození, též se řídili podobou jeho skutků.
Skrze…}}
 % Czech translations of the proper texts

\newcommand{\annusEditionis}{2019}

\def\hebinitial#1{%
\leavevmode{\newbox\hebbox\setbox\hebbox\hbox{\hebfont{#1}\hskip 1mm}\kern -\wd\hebbox\hbox{\hebfont{#1}\hskip 1mm}}%
}

%%%% Vicekrat opakovane kousky

\newcommand{\anteOrationem}{
  \rubrica{Ante Orationem, cantatur a Superiore:}

  \pars{Supplicatio Litaniæ.}

  \cuminitiali{}{temporalia/supplicatiolitaniae.gtex}

  \pars{Oratio Dominica.}

  \cuminitiali{}{temporalia/oratiodominica.gtex}

  \rubrica{Deinde dicitur ab Hebdomadario:}

  \cuminitiali{}{temporalia/dominusvobiscum-solemnis.gtex}

  \rubrica{In choro monialium loco Dominus vobiscum dicitur:}

  \sineinitiali{temporalia/domineexaudi.gtex}
}

\newcommand{\tuAutem}{
  \vfill

  \sineinitiali{temporalia/tuautem.gtex}
}

\setlength{\columnsep}{30pt} % prostor mezi sloupci

%%%%%%%%%%%%%%%%%%%%%%%%%%%%%%%%%%%%%%%%%%%%%%%%%%%%%%%%%%%%%%%%%%%%%%%%%%%%%%%%%%%%%%%%%%%%%%%%%%%%%%%%%%%%%
\begin{document}

% Here we set the space around the initial.
% Please report to http://home.gna.org/gregorio/gregoriotex/details for more details and options
\grechangedim{afterinitialshift}{2.2mm}{scalable}
\grechangedim{beforeinitialshift}{2.2mm}{scalable}

\grechangedim{interwordspacetext}{0.32 cm plus 0.15 cm minus 0.05 cm}{scalable}%
\grechangedim{annotationraise}{-0.2cm}{scalable}

% Here we set the initial font. Change 38 if you want a bigger initial.
% Emit the initials in red.
\grechangestyle{initial}{\color{red}\fontsize{38}{38}\selectfont}

\pagestyle{empty}

%%%% Titulni stranka
\begin{titulusOfficii}
\nomenFesti{Dominica I Adventus.}
\celebratio{II Classis. Semiduplex.}
\end{titulusOfficii}

% graphic
\vfill
\begin{center}
\includegraphics[width=15cm]{../AdventusDominicaII/imagines/imago_Sion.jpg}
\end{center}

\vfill

\begin{center}
Ad usum et secundum consuetudines chori \guillemotright Conventus Choralis\guillemotleft.

Editio Sancti Wolfgangi \annusEditionis
\end{center}

\pagebreak

\renewcommand{\headrulewidth}{0pt} % no horiz. rule at the header
\fancyhf{}
\pagestyle{fancy}

\pars{Léctio sancti Evangélii secúndum Lucam.} \scriptura{Lc. 21, 25-33}

\textusEtTranslatio{
  In illo témpore:
  Dixit Iesus discípulis suis:
  Erunt signa in sole, et luna, et stellis,
  et in terris pressúra géntium præ confusióne sónitus maris, et flúctuum:
  arescéntibus homínibus præ timóre, et exspectatióne, quæ supervénient univérso orbi:
  nam virtútes cælórum movebúntur.
  Et tunc vidébunt Fílium hóminis veniéntem in nube cum potestáte magna et maiestáte.
  His autem fíeri incipiéntibus, respícite, et leváte cápita vestra:
  quóniam appropínquat redémptio vestra.
  Et dixit illis similitúdinem: Vidéte ficúlneam, et omnes árbores:
  cum prodúcunt iam ex se fructum, scitis quóniam prope est aéstas.
  Ita et vos cum vidéritis hæc fíeri, scitóte quóniam prope est regnum Dei.
  Amen dico vobis, quia non præteríbit generátio hæc, donec ómnia fiant.
  Cælum et terra transíbunt: verba autem mea non transíbunt.
}{\trMatEvangelium}{10cm}

\vspace{2cm}

\cantusSineNeumas

\pars{Antiphona} \scriptura{\Abardot{} Is. 45, 8; \Vbardot{} Cf. ibid. 64, 9-11; 64, 5-7; 16, 1; 40, 1; 41, 4}

\vspace{-0.3cm}

{
\grechangedim{interwordspacetext}{0.10 cm plus 0.15 cm minus 0.05 cm}{scalable}%
\antiphona{I}{temporalia/ant-rorate.gtex}
\grechangedim{interwordspacetext}{0.32 cm plus 0.15 cm minus 0.05 cm}{scalable}%
}

\begin{translatioMulticol}{3}
Dej rosu, nebe nad námi,\\
ať z oblak skane spása.\\
\\
Odlož hněv svůj, ó Pane náš\\
a zapomeň už na naše nepravosti.\\
Hle, tvé svaté město je pouští,\\
opuštěný je Sión, Jeruzalém je liduprázdný,\\
to místo tobě zasvěcené, dům tvé slávy,\\
kde k chvále tvé zpívali otcové naši.\columnbreak

Pro hříchy své stali jsme se lidem nečistým\\
a odpadli jsme jako zvadlé listí.\\
Jako vichr nás uchvátily naše viny,\\
když jsi před námi ukryl svou tvář\\
a vydal nás napospas nepravosti naši.\\

Pohlédni, Pane, na ponížení lidu svého,\\
ať příjde ten, jenž přijít má.\\
Pošli Beránka, ať vládne zemi,\\
od Skály na poušti až k hoře siónské dcery.\\
Ať on sám sejme jho poroby naší.\columnbreak

Přijmi útěchu, přijmi útěchu, můj lide drahý,\\
neboť blízko je tvoje spása.\\
Proč se stále trápíš v úzkostech,\\
proč tě bolest svírá?\\
Zachráním tě, neboj se, doufej!\\
Vždyť já to jsem, já Hospodin, Pán a Bůh tvůj,\\
izraelův Svatý a Spása tvoje.
\end{translatioMulticol}


\vfill
\pagebreak

\pars{Oratio ante divinum Officium.}

\lettrine{{\color{red}A}}{peri,} Dómine, os meum ad benedicéndum nomen sanctum tuum:
munda quoque cor meum ab ómnibus vanis, pervérsis, et aliénis
cogitatiónibus:
intelléctum illúmina, afféctum inflámma,
ut digne, atténte ac devóte hoc Offícium recitáre váleam,
et exaudíri mérear ante conspéctum Divínæ Maiestátis tuæ.
Per Christum, Dóminum nostrum.
\Rbardot{} Amen.

Dómine, in unióne illíus divínæ intentiónis,
qua ipse in terris laudes Deo persolvísti,
has tibi Horas \rubricatum{(vel \textnormal{hanc tibi Horam})} persólvo.

\trOratioAnteOfficium

\vfill

\pars{Oratio post divinum Officium.}

\rubrica{
  Orationem sequentem devote post Officium recitantibus
  Leo Papa X. defectus, et culpas in eo persolvendo ex humana
  fragilitate contractas, indulsit, et dicitur flexis genibus.
}

\lettrine{{\color{red}S}}{acrosánctæ} et indivíduæ Trinitáti,
crucifíxi Dómini nostri Iesu Christi humanitáti,
beatíssimæ et gloriosíssimæ sempérque Vírginis Maríæ
fecúndæ integritáti, 
et ómnium Sanctórum universitáti
sit sempitérna laus, honor, virtus et glória
ab omni creatúra,
nobísque remíssio ómnium peccatórum,
per infiníta sǽcula sæculórum.
\Rbardot{} Amen.

\noindent \Vbardot{} Beáta víscera Maríæ Virginis, quæ portavérunt
ætérni Patris Fílium.\\
\Rbardot{} Et beáta úbera, quæ lactavérunt Christum Dominum.

\rubrica{Et dicitur secreto \textnormal{Pater noster.} et \textnormal{Ave María.}}

\trOratioPostOfficium

\vfill

\hora{Sabbato ad Vesperas.} %%%%%%%%%%%%%%%%%%%%%%%%%%%%%%%%%%%%%%%%%%%%%%%%%%%%%
\sideThumbs{I. Vesperæ}

\cantusSineNeumas

{
\grechangedim{interwordspacetext}{0.28 cm plus 0.15 cm minus 0.05 cm}{scalable}%
\cuminitiali{}{temporalia/deusinadiutorium-communis.gtex}
\grechangedim{interwordspacetext}{0.32 cm plus 0.15 cm minus 0.05 cm}{scalable}%
}

\vfill
\pagebreak

\cantusCumNeumis

\pars{Psalmus 1.} \scriptura{Ioel 3, 18; \textbf{H18}}

\vspace{-0.6cm}

\antiphona{VIII G}{temporalia/ant1.gtex}

\trAntI

\vspace{-0.5cm}

\cantusSineNeumas

\scriptura{Ps. 144, 10-21}

\initiumpsalmi{temporalia/ps144ii-initium-viii-G-auto.gtex}

\vspace{-0.5cm}

\psalmusEtTranslatioT{temporalia/ps144ii-comb.tex}{10cm}

%\antiphona{}{temporalia/ant1.gtex}

\vfill
\pagebreak

\pars{Psalmus 2.} \scriptura{Is. 26, 1.2; \textbf{H18}}

\vspace{-0.5cm}

\antiphona{VIII G\textsuperscript{2}}{temporalia/ant2.gtex}

\trAntII

\scriptura{Ps. 145}

\initiumpsalmi{temporalia/ps145-initium-viii-G2-auto.gtex}

\vspace{-0.2cm}

\psalmusEtTranslatioT{temporalia/ps145-comb.tex}{10cm}

\vfill
\pagebreak

\pars{Psalmus 3.} \scriptura{Cf. Zach. 14, 5-6; \textbf{H18}}

\vspace{-0.5cm}

\antiphona{V a}{temporalia/ant3.gtex}

\trAntIII

\scriptura{Ps. 146}

\initiumpsalmi{temporalia/ps146-initium-v-a-auto.gtex}

\psalmusEtTranslatioT{temporalia/ps146-comb.tex}{10cm}

\vfill
\pagebreak

\pars{Psalmus 4.} \scriptura{\textbf{H18}}

\vspace{-0.5cm}

\antiphona{IV A\textsuperscript{*}}{temporalia/ant5.gtex}

\trAntV

\scriptura{Ps. 147}

\initiumpsalmi{temporalia/ps147-initium-iv-A_-auto.gtex}

\psalmusEtTranslatioT{temporalia/ps147-comb-smaller.tex}{10cm}

\vfill
\pagebreak

\raggedcolumns

% Capitulum. %%%
\cantusSineNeumas

\pars{Capitulum.} \scriptura{Rom. 13, 11}

\cuminitiali{}{temporalia/capitulum-FratresHora.gtex}

% preklad Jeruz. bible
\trCapituli

\vfill
\pars{Responsorium breve.} \scriptura{Ps. 84, 8; \textbf{H20}}

{
\grechangedim{interwordspacetext}{0.28 cm plus 0.15 cm minus 0.05 cm}{scalable}%
\cuminitiali{IV}{temporalia/resp-vesp.gtex}
\grechangedim{interwordspacetext}{0.32 cm plus 0.15 cm minus 0.05 cm}{scalable}%
}

\trRespVesp

\vfill
\pagebreak

% Hymnus. %%%
\pars{Hymnus.}

{
\grechangedim{interwordspacetext}{0.17 cm plus 0.15 cm minus 0.05 cm}{scalable}%
\cuminitiali{IV}{temporalia/hym-ConditorAlme.gtex}
\grechangedim{interwordspacetext}{0.32 cm plus 0.15 cm minus 0.05 cm}{scalable}%
}
\input{../AdventusDominicaII/cantus/amon33/hym-ConditorAlme-bohtext.tex}

\vspace{-0.4cm}

\pars{Versus.} \scriptura{Is. 45, 8}

% Versus. %%%
\sineinitiali{temporalia/versus-rorate.gtex}
\noindent \trVersusVesp

\vfill
\pagebreak

\cantusCumNeumis

\pars{Canticum B. Mariæ V.} \scriptura{Cf. Is. 30, 27; \textbf{H18}}

\vspace{-0.5cm}

{
\grechangedim{interwordspacetext}{0.24 cm plus 0.15 cm minus 0.05 cm}{scalable}%
\antiphona{I f}{temporalia/ant-magn-vesp1.gtex}
\grechangedim{interwordspacetext}{0.32 cm plus 0.15 cm minus 0.05 cm}{scalable}%
}

\trAntMagnificatI

\vfill

\scriptura{Lc. 1, 46-55}

\cantusSineNeumas
\initiumpsalmi{temporalia/magnificat-initium-i-f.gtex}

\psalmusEtTranslatioT{temporalia/magnificat-comb.tex}{10.3cm}

\vfill
\pagebreak

\anteOrationem

\pagebreak

% Oratio. %%%
\pars{Oratio.}

\cuminitiali{}{temporalia/oratio.gtex}
\trOrationis

\vspace{1cm}
\rubrica{Hebdomadarius dicit iterum Dominus vobiscum. Postea cantatur a cantore:}
\vspace{2mm}

\cuminitiali{IV}{temporalia/benedicamus-dominica-advequad.gtex}

\vfill
\newpage
\RemoveSideThumbs
\pagestyle{empty}

%%% COLOPHON
\begin{center}
\includegraphics[width=6cm]{../AdventusDominicaII/imagines/angelus.jpg}
\end{center}

\vfill

Fontes.
Textus et cantus officii divini secundum
Antiphonale Sacrosanctæ Romanæ Eclesiæ Pro Diurnis Horis, Romæ 1912
et Nocturnale Romanum, 2002, præter: psalmi 149 et 150 post
psalmum 148 in Laudibus additi secundum Antiphonale Monasticum pro Diurnis Horis,
Solesmis 1934; lectio sancti Evangelii et hymnus Te Decet Laus post ultimum
responsorium additi secundum ritum monasticum vetum; responsorium breve
in Laudibus et Vesperis additum secundum Antiphonale Monasticum. /
Textus et cantus missæ secundum
Graduale triplex, Solesmis 1979. /
Translatio capituli et lectionis sumpta est ex:
Jeruzalémská bible, Praha-Kostelní Vydří 2009. /
Translationes psalmorum ex
Hejčl Jan: Žaltář čili Kniha žalmů, Praha 1922. /
Neumæ super canto missæ de codicibus Cantatorium, Stiftsbibl. 359 et
Einsiedeln,
Stiftsbibl. 121 et neumæ super canto officii divini de codice Hartker,
Stiftsbibl. 391.

Collaborantes.
Textus latinos cantusque transcripsit et omnem laborem typographicum peregit
Jakub Jelínek. /
Proprios cantus festi in linguam bohemicam Václav Ondráček transtulit. /
Psalmos in lingua bohemica de libro supra dicto transcripsit
Barbora Maturová et idem Jakub Jelínek. /
Filip Srovnal librum istum præparare mandavit et laborem exprobrationibus
utilissimis comitabatur. Iste etiam librum totum ante publicationem
diligenter legit et plurimos errores invenit. /
Imaginem, quæ paginam tituli ornat, Klára Jirsová pinxit.

Instrumenta adhibita.
LuaTeX, %http://www.luatex.org / 
Gregorio, %http://home.gna.org/gregorio /
typi Junicode. %http://junicode.sourceforge.net

\begin{center}
Liber hic imprimis ad usum chori 
\guillemotright Conventus Choralis\guillemotleft\ 
paratus est
et secundum eius consuetudines.
http://www.introitus.cz

\vspace{1cm}

{\large Editio Sancti Wolfgangi 2019.}

\vspace{2mm}

Series \guillemotright Conventus\guillemotleft, vol. II.

\vspace{1cm}

http://stiwolfgangi.xf.cz

\vfill

\today

\end{center}

\end{document}
