% LuaLaTeX

\documentclass[a4paper, twoside, 12pt]{article}
\usepackage[latin]{babel}
%\usepackage[landscape, left=3cm, right=1.5cm, top=2cm, bottom=1cm]{geometry} % okraje stranky
\usepackage[portrait, a4paper, mag=1414, truedimen, left=0.8cm, right=0.8cm, top=0.8cm, bottom=0.8cm]{geometry} % okraje stranky

\usepackage{fontspec}
\setmainfont[FeatureFile={junicode.fea}, Ligatures={Common, TeX}, RawFeature=+fixi]{Junicode}
%\setmainfont{Junicode}

% shortcut for Junicode without ligatures (for the Czech texts)
\newfontfamily\nlfont[FeatureFile={junicode.fea}, Ligatures={Common, TeX}, RawFeature=+fixi]{Junicode}

\usepackage{multicol}
\usepackage{color}
\usepackage{lettrine}
\usepackage{fancyhdr}

% usual packages loading:
\usepackage{luatextra}
\usepackage{graphicx} % support the \includegraphics command and options
\usepackage{gregoriotex} % for gregorio score inclusion
\usepackage{gregoriosyms}
\usepackage{wrapfig} % figures wrapped by the text
\usepackage{parcolumns}
\usepackage[contents={},opacity=1,scale=1,color=black]{background}
\usepackage{tikzpagenodes}
\usepackage{calc}
\usepackage{longtable}

\setlength{\headheight}{12pt}

\input{conventuscommune.tex} % Often used macros
%%%% Preklady jednotlivych zpevu (nektere se opakuji, a je dobre mit je
% vsechny na jedne hromade)

% HOURS ---

\newcommand{\trAntI}{\translatioCantus{Muž boží měl kožený toulec, pečlivě
zavázaný, jenž mu visel na šíji a~často se ho dotýkal.}}

\newcommand{\trAntII}{\translatioCantus{Klíč od~něho tak dobře střežil, že
dokud žil v~těle, nikdo z~jeho žáků nezvěděl, co je uvnitř.}}

\newcommand{\trAntIII}{\translatioCantus{Ale když se odebral z~tohoto
života, schránku otevřeli a~objevili v~ní žíněné roucho a~měděný řetěz
potřísněný krví.}}

\newcommand{\trAntIV}{\translatioCantus{A když prohlédli mistrovo tělo,
nalezli jeho tělo na čtyřech místech hluboce zbrázděno ranami od řetězu.}}

\newcommand{\trAntV}{\translatioCantus{Krev vytékající z~těch ran, místy
prostoupila i~žíněným rouchem.}}

\newcommand{\trCapituli}{\translatioCantus{
Miláčkovi Boha a~lidí,
Mojžíšovi požehnané paměti,~\gredagger{}
dopřál slávu rovnou slávě svatých~\grestar{}
učinil ho mocným na postrach nepřátelům
a~jeho slovy zastavil divy.}}

\newcommand{\trLectioBrevis}{\translatioCantus{
Pamatujte na své představené,
kteří vám hlásali Boží slovo.
Uvažte, jak oni skončili život, a~napodobujte jejich víru.
Ježíš Kristus je stejný včera i~dnes i~navěky.
Nenechte se svést věelijakými cizími naukami.}}

\newcommand{\trRespLaud}{\translatioCantus{Spravedlivého vodil Hospodin~\grestar{}
po přímých stezkách. \Vbardot{} A~ukázal mu Boží království.}}

\newcommand{\trRespLaudB}{\translatioCantus{Na tvých hradbách, Jeruzaléme,
ustanovil jsem strážné;~\grestar{}
budou bdít nad mým lidem. \Vbardot{} Ani ve dne, ani v~noci nesmějí nikdy
mlčet.}}

\newcommand{\trVersus}{\translatioCantus{\Vbardot{} Ústa spravedlivého šeptají moudrost, aleluja.
\Rbardot{} A~jeho jazyk ohlašuje právo, aleluja.}}

\newcommand{\trAntBenedictus}{\translatioCantus{Když na bujné oře vložili
nosítka a~sňali jim uzdu, vydali se přímo k~cele božího muže.}}

\newcommand{\trPreces}{\translatioCantus{
\noindent S vděčností chvalme Krista, dobrého Pastýře, \gredagger{} který dal život za své ovce, \grestar{} a~pokorně ho prosme: \Rbardot{} Pane, buď pastýřem svého lidu.

\noindent Kriste, ty dáváš církvi pastýře, a~jejich službou se ujímáš svého lidu, \grestar{} dej, ať v~lásce těch, kteří nás vedou, poznáváme, jak nás miluješ. \Rbardot{} Pane, buď pastýřem svého lidu.

\noindent Ty stále konáš skrze své zástupce službu pastýře a~učitele, \grestar{} nepřestávej nás nikdy vést prostřednictvím svých služebníků. \Rbardot{} Pane, buď pastýřem svého lidu.

\noindent Ty prokazuješ svému lidu skrze jeho pastýře službu lékaře duše i~těla, \grestar{} ochraňuj náš život a~veď nás ke svatosti. \Rbardot{} Pane, buď pastýřem svého lidu.

\noindent Ty posíláš své svaté, aby slovem i~příkladem vedli tvůj lid k~tobě, \grestar{} na jejich přímluvu nás posiluj, abychom vytrvali na cestě, která vede k~věčnému životu. \Rbardot{} Pane, buď pastýřem svého lidu.}}

\newcommand{\trOrationis}{\translatioCantus{Bože, jenž nám dopřáváš radovat
se z~výroční slavnosti svatého tvého vyznavače Havla, uděl dobrotivě,
abychom když slavíme jeho narození, též se řídili podobou jeho skutků.
Skrze…}}
 % Czech translations of the proper texts

%%%% Vicekrat opakovane kousky

\newcommand{\anteOrationem}{
  \pars{Oratio Dominica.}

  \cuminitiali{}{temporalia/oratiodominica.gtex}

  \trOratioDominica

  \rubrica{Deinde dicitur ab Hebdomadario:}

  \vspace{1mm}

  \cuminitiali{}{temporalia/dominusvobiscum-solemnis.gtex}

  \trDominusVobiscum

  \rubrica{In choro monialium loco Dominus vobiscum dicitur:}

  \vspace{1mm}

  \sineinitiali{temporalia/domineexaudi.gtex}

  \trDomineExaudi
}

\setlength{\columnsep}{15pt} % prostor mezi sloupci

%%%%%%%%%%%%%%%%%%%%%%%%%%%%%%%%%%%%%%%%%%%%%%%%%%%%%%%%%%%%%%%%%%%%%%%%%%%%%%%%%%%%%%%%%%%%%%%%%%%%%%%%%%%%%
\begin{document}

% Here we set the space around the initial.
% Please report to http://home.gna.org/gregorio/gregoriotex/details for more details and options
\grechangedim{afterinitialshift}{2.2mm}{scalable}
\grechangedim{beforeinitialshift}{2.2mm}{scalable}
\grechangedim{interwordspacetext}{0.28 cm plus 0.15 cm minus 0.05 cm}{scalable}%
\grechangedim{annotationraise}{-0.2cm}{scalable}

% Here we set the initial font. Change 38 if you want a bigger initial.
% Emit the initials in red.
\grechangestyle{initial}{\color{red}\fontsize{36}{36}\selectfont}

\renewcommand{\headrulewidth}{0pt} % no horiz. rule at the header
\pagestyle{empty}

\grechangedim{spaceabovelines}{0.2cm}{scalable}%

\vfill

\hbox{}

\vspace{2.5cm}

\begin{center}
{\large \textit{DIE XX IULII}}

\vspace{0.5cm}

{\huge S. MARGARITÆ}

{\large VIRGINIS ET MARTYRIS}

\vspace{0.5cm}

% graphic
\vspace{1.5cm}
\begin{center}
\includegraphics[width=8cm]{margarita.jpg}
\end{center}

\vspace{1.8cm}

%\textsc{Secundum Antiphonale Breunowiensis}
\textsc{Secundum Antiphonale Raygradensis}
\end{center}

\vfill
\pagebreak

\cantusSineNeumas

\pars{Introductio}

\gregorioscore{temporalia/deusinadiutorium-communis.gtex}

\trIntroductio

\vspace{0.5cm}

\pars{Antiphona I} \scriptura{\textbf{Sg. 389, pag. 410}}

\vspace{-0.2cm}

\antiphona{II D}{temporalia/ant1.gtex}

\trAntI

\scriptura{Ps. 109}

\gregorioscore{temporalia/ps109-initium-ii-D-auto.gtex}

\psalmusEtTranslatioT{temporalia/ps109-comb.tex}{6cm}

% Repeat the antiphon - new page
%\antiphona{}{temporalia/ant1.gtex}

\vfill
%\pagebreak

\pars{Antiphona II} \scriptura{\textbf{Sg. 389, pag. 410}}

\vspace{-0.2cm}

\antiphona{III g}{temporalia/ant2.gtex}

\trAntII

\scriptura{Ps. 110}

\gregorioscore{temporalia/ps110-initium-iii-g-auto.gtex}

\psalmusEtTranslatioT{temporalia/ps110-comb.tex}{6cm}

% Repeat the antiphon - new page
\antiphona{}{temporalia/ant2.gtex}

\vfill
\pagebreak

\pars{Antiphona III} \scriptura{\textbf{Sg. 389, pag. 410}}

\vspace{-0.2cm}

\antiphona{IV E}{temporalia/ant3.gtex}

\trAntIII

\scriptura{Ps. 111}

\gregorioscore{temporalia/ps111-initium-iv-E-auto.gtex}

\psalmusEtTranslatioT{temporalia/ps111-comb.tex}{6cm}

\vfill
\pagebreak

\pars{Antiphona IV} \scriptura{\textbf{Sg. 389, pag. 410}}

\vspace{-0.2cm}

\antiphona{V a}{temporalia/ant4.gtex}

\trAntIV

\scriptura{Ps. 112}

\gregorioscore{temporalia/ps112-initium-v-a-auto.gtex}

\psalmusEtTranslatioT{temporalia/ps112-comb.tex}{6cm}

\vfill
\pagebreak

\pars{Antiphona V} \scriptura{\textbf{Sg. 389, pag. 410}}

\vspace{-0.2cm}

\antiphona{VI C}{temporalia/ant5.gtex}

\trAntV

\scriptura{Ps. 113}

\gregorioscore{temporalia/ps113-initium-vi-C-auto.gtex}

\psalmusEtTranslatioT{temporalia/ps113-comb.tex}{6cm}

\vfill
\pagebreak

\pars{Capitulum} \scriptura{2 Cor. 10, 17-18}

\cuminitiali{}{temporalia/capitulum-FratresQui.gtex}

\trCapituli

\vspace{0.5cm}

\pars{Responsorium breve} \scriptura{Ps. 45, 6; \textbf{H190}}

\vspace{-0.2cm}

\antiphona{\oldrbar.~ br.}{temporalia/resp.gtex}

\trResp

\vfill
\pagebreak

\pars{Hymnus}

\vspace{-0.5cm}

\antiphona{I}{temporalia/hym-VirginisProles.gtex}

\begin{translatioMulticol}{2}
Potomku panny, stvořiteli matky,\\
jehož jsi panna počavši zrodila.\\
Zpíváme chválu panenskému zrodu\\
jenž smrt pokořil.\\
\\
Nezděsila ji smrt, ani přečetné,\\
vskutku tisíceré strašlivé muky,\\
prolitím krve zásluhu získala\\
vystoupit k nebi.\columnbreak

Na prosby její, Pane Bože dobrý,\\
odpusť nám tresty zasloužené hříchy,\\
ať ti pak sami již v čistotě srdce\\
píseň zpíváme.\\
\\
Sláva buď Otci, Synu z něj zrozenému,\\
stejná pak Tobě, jenž společnou silou\\
a Duchem obou Bohem jedním věčně\\
i v času světa.\\
Amen.
\end{translatioMulticol}


\vspace{0.5cm}

\pars{Versus} \scriptura{Ps. 44, 3}

\sineinitiali{temporalia/versus-diffusa.gtex}

\trVersiculi

\vfill
\pagebreak

\pars{Antiphona ad Magnificat} \scriptura{\textbf{Sg. 389, pag. 407}}

\vspace{-0.2cm}

\antiphona{I g}{temporalia/ant-magn-vesp.gtex}

\trAntMagnificat

\scriptura{Lc. 1, 46-55}

\gregorioscore{temporalia/magnificat-initium-i-g.gtex}

\psalmusEtTranslatioT{temporalia/magnificat-comb.tex}{6cm}

% Repeat the antiphon - new page
\antiphona{}{temporalia/ant-magn-vesp.gtex}

\vfill
\pagebreak

\scriptura{Deprecatio ex Missale Gothicum, A.D. circa 700}

\vspace{-7mm}

\grechangedim{interwordspacetext}{0.10 cm plus 0.15 cm minus 0.05 cm}{scalable}%
\antiphona{C\textsuperscript{3}}{temporalia/deprecatio.gtex}
\grechangedim{interwordspacetext}{0.28 cm plus 0.15 cm minus 0.05 cm}{scalable}%

\trDeprecatio

\vspace{8mm}

\anteOrationem

\vfill
\pagebreak

\vspace{0.5cm}

\pars{Oratio conclusiva}

\gregorioscore{temporalia/oratio.gtex}

\trOratio

\rubrica{Hebdomadarius dicit iterum Dominus vobiscum. Postea cantatur a cantore:}

\vspace{2mm}

\cuminitiali{V}{temporalia/benedicamus-festis-sanctorum.gtex}

\trBenedicamus

\vfill

\vspace{4mm}

{\tiny\noindent
Fontes.
Cantus officii divini secundum
Antiphonale Raygradensis, Codice R 17 B/K.II.aa.18.
Neumæ super canto secundum Antiphonarium, Codice 389 Sangallensi.
Responsorium breve secundum Antiphonale Monasticum pro Diurnis Horis, Solesmis,
1934, et Antiphonarium Hartkeri, Codice 390 Sangallensi.
Hymnus secundum Nocturnale Romanum, 2002.
Preces secundum Missale Gothicum, Bibliotheca Vaticana, Reg. lat. 317.
Translatio capituli et versiculi sumpta est ex: Jeruzalémská bible, Praha-Kostelní Vydří 2009. /
Translationes psalmorum ex Hejčl Jan: Žaltář čili Kniha žalmů, Praha 1922. /

\noindent Collaborantes.
Textus latinos cantusque transcripsit et omnem laborem typographicum peregit Jakub Jelínek. /
Václav Ondráček textus hymnorum et antiphonarum in linguam bohemicam transtulit. /
Štěpán Němec et Filip Srovnal librum istum diligentissime examinavit, errores multos
inveniens.

\noindent Instrumenta adhibita.
LuaTeX, %http://www.luatex.org /
Gregorio, %http://home.gna.org/gregorio /
typi Iunicode. %http://junicode.sourceforge.net
}

\begin{center}
{\scriptsize
Liber hic imprimis ad usum
\guillemotright Amici ecclesiæ sublipniciensis\guillemotleft\
paratus est\\
et secundum eius consuetudines.\\
http://www.podlipnickekostely.cz
}
\end{center}

\end{document}
